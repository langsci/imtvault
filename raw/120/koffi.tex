\documentclass[output=paper,
modfonts
]{langscibook} 
\ChapterDOI{10.5281/zenodo.1251730}
\title{The acoustic vowel space of Anyi in light of the cardinal vowel system and the Dispersion Focalization Theory}
\shorttitlerunninghead{The acoustic vowel space of Anyi}
\abstract{The Cardinal Vowel System (CVS) and the Dispersion Focalization Theory (DFT) make an important assumption about the inventory of vowels in world languages. The claim is that languages organize their vowels in a certain way in the auditory-perceptual space so as to maximize intelligibility. The vowel diagrams of African languages in influential publications such as \citet[20--45]{Welmers1973} explicitly or implicitly reflect this assumption. However, persistent confusions between [ɪ] and [e] among Anyi Morofu speakers have aroused my curiosity and led me to investigate the matter acoustically. The findings reported here show that the vowel space of Anyi Morofu is in a between and betwixt state. The data indicates that this dialect is moving from a nine-vowel system to an eight-vowel system through the merger of [ɪ] and [e]. There are also signs of the impending merger of [ʊ] and [o].}  

\author{Ettien Koffi \affiliation{St. Cloud State University}}

\begin{document}
\maketitle 

\section{Introduction} 
\largerpage
The Cardinal Vowel System (CVS) and the Dispersion Focalization Theory (DFT) \isi{agree} on the principle that languages organize their vowel inventories in order to maximize intelligibility. The principle underlying both approaches is known as the Principle of Perceptual Separation (PPS). \citet[238]{LadefogedJohnson2015} explain it as follows, “One of the forces acting on languages may be called the principle of sufficient perceptual separation, whereby the sounds of a language are kept acoustically distinct to make it easier for the listener to distinguish one from another.” This important principle collides with how \ili{Anyi} Morofu organizes its vowel inventory. \ili{Anyi} Morofu is the biggest dialect of the \ili{Anyi} language spoken in Côte d’Ivoire, West Africa. According to the 2000 census (outdated, but there is no other official census data to go by), this dialect is spoken by more than half of the 755,365 \ili{Anyi} speakers in Côte d’Ivoire. \ili{Anyi} belongs to the \ili{Akan} family of languages. Before presenting the evidence for how \ili{Anyi} Morofu runs counter to the core principle of CVS and DFT, let’s acquaint ourselves briefly with these two phonetic frameworks. The goal here is not to review these two theories extensively, but rather to use the data they provide to explain the perceptual confusion between [ɪ] and [e] and the signs announcing the upcoming merger of [ʊ] and [o]. 

\subsection{A short history of the cardinal vowel system}
\largerpage
On the occasion of Ladefoged’s sixtieth birthday, \citet{Fromkin1985} put together a collection of papers from influential phoneticians. In \citeauthor{Abercrombie1985}’s article (1985:17), he gives us a glimpse of how CVS came about. His account is authoritative because he had a front row seat when Jones was designing his method. He was Jones’ student and later became Ladefoged’s teacher and mentor. He notes that for Jones, CVS was not a theory, but a technique. He describes this technique as follows: 

\begin{quote}
{This way of teaching phonetics meant intensive training of the proprioceptive, i.e., the tactile and kinesthetic senses concerned with the organs of speech, something that is not valued very highly by many other schools of phonetics. The proprioceptive senses, in the view of phoneticians in the Jones tradition, play an important part in the analysis and description of unfamiliar sounds. The phonetician, having learnt to make a sound of the language he is working on to the complete satisfaction of his native informant, then examines what he himself is doing with his vocal organs, and infers the informant is doing the same.}
\end{quote}

{Jones learned to produce a wide variety of vowels this way. \citet[146]{Thomas2011} provides in \tabref{tab:koffi:1} the formant frequencies of 18 vowels that Jones learned to produce.}

\begin{table}
\caption{Jones’ Vowels}
% \todo[inline]{content of first column is unclear}
\label{tab:koffi:1}
\begin{tabularx}{.66\textwidth}{XXrrr}
\lsptoprule
N0 & Vowels & F1 & F2 & F3\\
\midrule
1. & [i] & 266 & 2581 & 3627\\
2. & [ɨ] & 312 & 2078 & 2544\\
3. & [ɯ] & 337 & 1275 & 2180\\
4. & [u] & 248 & 490 & 2512\\
5. & [y] & 289 & 2231 & 2747\\
6. & [ʉ] & 285 & 1487 & 2066\\
7. & [e] & 376 & 2213 & 2652\\
8. & [ø] & 353 & 1946 & 2375\\
9. & [ɤ] & 569 & 1153 & 2282\\
10. & [o] & 354 & 724 & 2348\\
11. & [ɛ] & 588 & 1910 & 2328\\
12. & [œ] & 554 & 1549 & 2158\\
13. & [ɶ] & 722 & 1227 & 2180\\
14. & [ɒ] & 582 & 769 & 2150\\
15. & [ʌ] & 542 & 1145 & 2273\\
16. & [ɔ] & 522 & 932 & 2180\\
17. & [ɑ] & 650 & 940 & 2472\\
18. & [a] & 929 & 1688 & 2354\\
\lspbottomrule
\end{tabularx}
\end{table}

{For the purposes of this paper, the focus will be on nine vowels, [i, ɨ, e, ɛ, a, ɔ, o,} {ʉ}{, u], because \ili{Anyi} also has nine vowels. According to \citegen{Maddieson1984} UCLA Phonetic Segment Inventory Database (UPSID), 17 out the 266 languages in the database have nine vowels. Languages such as \ili{Anyi} with a nine-vowel system represent only 6.39\% of the total number of languages in UPSID. Furthermore, only seven of the 17 nine-vowel languages have a \isi{perfect} symmetry of four front vowels and four back vowels and one low central vowel. The vowel system of these seven languages is similar to the one we find in \ili{Anyi}. Jones’ cardinal vowel system did not include [ɪ] and [ʊ] because [-ATR] vowels were not known at the time. Even so, the plotting of his vowels gives us a realistic picture of what a nine-vowel system looks like.}\footnote { {} The diagrams were produced using Norm, available at \url{ http://lingtools.uoregon.edu/norm/norm1.php}. } {}


\begin{figure}  
\includegraphics[height=.3\textheight]{figures/koffi1.png}
\caption[Jones’ Acoustic Vowel Space]{The Norm website does not recognize certain IPA symbols.  The following legend is used 
<ii> = [ɨ],  
<uu> = [ʉ],
<ee> = [ɛ], 
<oo> = [ɔ] instead.}
\label{fig:koffi:1}
\end{figure}

A few cursory remarks need to be made. First and foremost, in the nine-vowel system produced by Jones, we see that the PPS obtains. No two vowels overlap in acoustic space. CVS has had a far-reaching impact on how the vowel inventories of African languages are plotted in \citet[20--45]{Welmers1973}, in \textit{Atlas des Langues Kwa de Côte d’Ivoire, Tome 1}, and in countless other publications. \citet[145--147]{Thomas2011} opines that Jones’ original intention in proposing CVS was only to “standardize impressionistic transcription to make it more useful for interlanguage comparisons,” not to idealize it as the acoustic vowel spaces for all languages. \citet{Koffi2009} and all who have described the \ili{Anyi} vowel quadrant have used this idealized system. This is the reason why the confusion between [ɪ] and [e] came as a surprise because under the idealized \ili{Anyi} vowel quadrant, unintelligibility was not expected. 


\subsection {A quick overview of the Dispersion Focalization Theory}  
\largerpage
PPS is also at the core of the Dispersion Focalization Theory (DFT) that \citet{SchwartzEtAl1997dispersion} put forth. However, \citet[10]{Becker-Kristal2010} contends that “the idea that vowel inventories are structured in a manner that enhances contrast, by maximally dispersing vowels in the auditory-perceptual space, is as old as the intuition that vowel inventories follow universal structural patterns.” It is worth stating clearly and unambiguously that the goal pursued in this paper is not to review, critique, or summarize DFT or the Dispersion Theory (DT) from which it sprang. Such an exercise would require us to make a long detour in the histories and developments of these two theories. It is not the theoretical claims of DFT that interest us as much as the impressive amount of formant frequency data provided for 22 “prototypical” vowels, as seen in \tabref{tab:koffi:2}. Aspects of this data will be used in \figref{fig:koffi:2} to highlight the acoustic vowel space of a prototypical language with a nine-vowel system. 

\begin{table}
\caption{Prototypical Vowel Frequencies}
\label{tab:koffi:2}
\begin{tabularx}{.66\textwidth}{XXrrr}
\lsptoprule
N0 & Vowels & F1 & F2 & F3\\
\midrule
1. & [i] & 277 & 2208 & 3079\\
2. & [ʏ] & 277 & 1937 & 2232\\
3. & [ɨ] & 277 & 1520 & 2310\\
4. & [ɯ] & 277 & 1218 & 2500\\
5. & [u] & 277 & 553 & 2420\\
6. & [ɪ] & 344 & 2170 & 2660\\
7. & [y] & 344 & 1770 & 2230\\
8. & [ʊ] & 344 & 635 & 2413\\
9. & [e] & 414 & 2065 & 2570\\
10. & [ø] & 414 & 1608 & 2250\\
11. & [ə] & 414 & 1516 & 2500\\
12. & [ɤ] & 414 & 1248 & 2500\\
13. & [o] & 414 & 721 & 2406\\
14. & [ɛ] & 565 & 1819 & 2528\\
15. & [œ] & 565 & 1520 & 2500\\
16. & [ɜ] & 565 & 1462 & 2500\\
17. & [ʌ] & 565 & 1258 & 2500\\
18. & [ɔ] & 565 & 915 & 2373\\
19. & [æ] & 648 & 1712 & 2490\\
20. & [ɐ] & 648 & 1405 & 2500\\
21. & [ɑ] & 735 & 1278 & 2500\\
22. & [a] & 800 & 1228 & 2500\\
\lspbottomrule
\end{tabularx}
\end{table}

{Nine of the prototypical vowels, [i, ɪ, e, ɛ, a, ɔ, o,} {ʊ}{, u], are also found in \ili{Anyi}. They are plotted in \figref{fig:koffi:2} to show how these prototypical vowels are organized in an acoustic space.}


  
\begin{figure}
\includegraphics[height=.3\textheight]{figures/koffi2.png}
\caption[Vowel space of nine prototypical vowels]{Vowel space of nine prototypical vowels\\
The Norm website does not recognize certain IPA symbols.  
The following legend is used for [-ATR] vowels: 
<ii> = [ɪ],  
<uu> = [ʊ],
<ee> = [ɛ], 
<oo> = [ɔ].}  

\label{fig:koffi:2}
\end{figure}

{The plotting shows that the claims of PPS hold here as they did in \figref{fig:koffi:1}. No vowel encroaches on the space of another vowel. Consequently, intelligibility is maximized. Let’s now turn to the \ili{Anyi} Morofu data and examine its vowel space in light of CVS and DFT.} 

\section{Data collection and participants}
~

{\citet{Koffi2009}, \citet[27]{Quaireau1987}, \citet[96]{Retord1980}, to name only the three main researchers on \ili{Anyi}, have all diagrammed the oral vowels of \ili{Anyi} as shown in \figref{fig:koffi:3}:}


\begin{figure}
\begin{tikzpicture}
	\node at (1.5,0) (a) {a};
	\node at (3,2.5) (ʊ) {ʊ};
        \node at (0,2.5) (ɪ) {ɪ};
	\node at (-.25,3) (i) {i};
	\node at (3.25,3) (u) {u};
	\node at (0.25,2) (e) {e};
	\node at (2.75,2) (o) {o};
	\node at (0.75,1) (E) {ε};
	\node at (2.25,1) (O) {ɔ};	
	
	\node at (-.5,3.25) [anchor=east] {High Front};
	\node at (3.5,3.25) [anchor=west] {High Back};
	\node at (-.5,2) [anchor=east] {Mid};
	\node at (-.5,0) [anchor=east] {Low};
\end{tikzpicture}
% 
%    High Front                                                                    High Back
% 
%                      i                                                              u 
% 
%                             ɪ                                                 ʊ 
% 
%                 Mid                  e                              o                   
% 
%                                                 ɛ                 ɔ 
% 
%                            Low                         a

\caption{Anyi Morofu vowel diagram}
% % \todo[inline]{Draw with tikz}
\label{fig:koffi:3}
\end{figure}

\ili{Anyi} also has seven nasal vowels: [ɪ, \~\i, \~u, ʊ, ɛ, ɔ, ã]. The vowels [ɛ] and [ɔ] are deemed unnasalizable in some \ili{Akan} languages, but not in Morofu. The only vowels that are unnasalizable are [e] and [o] \citep{Koffi2004}.
\figref{fig:koffi:3} may be an accurate representation of a nine-vowel phonemic system in \ili{Anyi} Morofu, but it is no longer an accurate representation of its contemporary phonetic vowel system. This became abundantly clear during a \isi{literacy} seminar in the summer of 2011. During the dictation task, teachers in training confused [ɪ] with [e], and [e] with [ɪ] regardless of who was doing the dictation.\footnote{\citet[126,130--131]{Ladefoged2003} tells a similar story about Banawa, a language of the Amazonian rain forest in Brazil where there was confusion between [u] and [o] that led to strong disagreements in the \isi{orthography} of the language.} Test takers would frequently stop to ask the reader whether he meant [e] or [ɪ] in instances where the contextual cues were not enough to disambiguate the lexical items containing these vowels. For example in a sentence such as, \textit{ɔ’a hɪ nnaán} ‘he trapped an animal’, some test takers wrote \textit{ɔ’a he nnaán} ‘he shared/gave away some of his meat’. The high number of confusion incidences such as these caused me to wonder if a merger was happening between these two vowels in the Morofu dialect spoken in the Bongouanou area. \figref{fig:koffi:4} shows this dialect in relation to the other \ili{Anyi} dialects. As noted earlier, Morofu has more speakers than all the other dialects of \ili{Anyi} combined.

\begin{figure}
\includegraphics[height=.3\textheight]{figures/koffi4.png}
\caption{The Anyi dialect area}
\label{fig:koffi:4}
\end{figure}

The matter was investigated further through data collection in the summers of 2012 and 2013 after securing the approval of the Institutional Review Board (IRB) from my university. The same ten male adult \isi{literacy} teachers were invited again. Female speakers were not intentionally excluded. At that time, there were no female \isi{literacy} teachers. The situation has now changed and we have three female teachers. The lack of female data does not affect the present analysis negatively because most of the various predictions of DFT and DT are based on male speech (Becker-Kristal, 2010:31). The participants are all bilingual in \ili{Anyi} and \ili{French}. They range in age from the 30s to 50s. Each participant produced nine sentences, each containing one of the nine vowels under consideration:
 
\ea
  \ea <ɔ’a hi> (he/she has refused to eat it)
  \ex <ɔ’a hɪ> (he/she has caught it)
  \ex <ɔ’a he> (he/she has shared it)
  \ex <ɔ’a hɛ> (he/she is late)
  \ex <ɔ’a hu> (it has boiled)
  \ex <ɔ’a hʊ> (a nonsense word)
  \ex <ɔ’a ho> (he/she has dug a whole)
  \ex <ɔ’a hɔ> (he/she has left)
  \ex <ɔ’a ha> (he/she has bitten) 
  \z
\z

Each sentence was repeated three times, for a total of 30 repetitions. The data set consists of 270 items (9 x 3 x 10). The data was collected on an Olympus Digital Voice Recorder WS-710. The participants wore a Panasonic head-mounted, noise cancellation fixed microphone. The recording took place in a quiet room on the premises of the \ili{Anyi} Literacy and Translation Center (CATA). 

\subsection{Methodology}
The elicitation word in each sentence begins with /h/. These words were chosen intentionally in order to replicate Peterson and Barney’s methodology as much as possible. Countless studies of vowels have followed this methodology. \citet[112]{Ladefoged1996} explains the benefits of choosing /h/ in these kinds of acoustic phonetic studies as follows: 

\begin{quote}
As the positions of the articulators during the sound [h] are similar to those of the surrounding sounds, such as the adjacent vowels, the frequency components in [h] sounds have relative amplitudes similar to those in vowels; but the complex wave has a smaller amplitude and no fundamental frequency, as it is not generated by regular pulses from the vocal cords.
\end{quote}

Since [h] exists in \ili{Anyi} as an allophone of /k/, Peterson and Barney’s methodology can be replicated without any problem. The entire duration of the vowel, from the onset to the offset, was measured. It was not deemed necessary for this study to take measurements at various points in the vowel because the environment in which the vowel occurred did not foster co-articulation. Furthermore, the methodology used by Peterson and Barney that is being replicated in this study did not sample vowels at multiple intervals. The onset of each vowel was easily identified because of the frication noise contained in [h]. However, it was more challenging to determine the offset of vowels. In annotating the offset, \citet[142]{Thomas2011} proposes three options:

\begin{quote}
… The same problem crops up frequently with vowels before a pause. In these cases, you have another choice to make. One option is to look for a spot where the vocal fold vibrations become more or less unrecognizable or start looking more like staticky patterns of aspiration than the sharper pattern usually evident with vocal fold vibrations. Often, the best way to determine this spot is by moving the cursor to different spots and listening; after a certain point, all you hear is aspiration, and that point is where you mark the offset. The other option is to mark the offset at the end of the recognizable aspiration, though this point may be quite difficult to define.
\end{quote}

For this study, the offset of the vowel was determined by following the second option in Thomas’ recommendation, that is, demarcating the offset right before the point at which aspiration is heard. The measurements for one speaker were done manually to ensure that the offsets of vowels are identified accurately. Once the pattern was well established, \citegen{Ryan2005} Grid-maker script for Praat was used to annotate all the vowels produced by the rest of the speakers. Subsequently, \citegen{Yoon2008} Stress-analysis script for Praat was employed to collect all the relevant information displayed in Tables \ref{tab:koffi:3} and~\ref{tab:koffi:4}. 

\begin{table}[p]
\caption{F1 Formant for all participants}
\label{tab:koffi:3}
\begin{tabularx}{\textwidth}{Xrrrrrrrrr}
\lsptoprule
 {\textbf{F1}} &  {[i]} &  {[ɪ]} &  {[e]} &  {[ɛ]} &  {[u]} &  {[ʊ]} &  {[o]} &  {[ɔ]} &  {[a]}\\
\midrule
 {Speaker 1} &  {325} &  {368} &  {355} &  {556} &  {423} &  {539} &  {481} &  {624} &  {942} \\
 {Speaker 2} &  {280} &  {407} &  {408} &  {576} &  {329} &  {595} &  {429} &  {639} &  {885}\\
 {Speaker 3} &  {307} &  {435} &  {408} &  {623} &  {345} &  {431} &  {414} &  {653} &  {983}\\
 {Speaker 4} &  {291} &  {344} &  {368} &  {573} &  {384} &  {583} &  {429} &  {677} &  {882} \\
 {Speaker 5} &  {291} &  {415} &  {361} &  {601} &  {396} &  {510} &  {469} &  {639} &  {980}\\
 {Speaker 6} &  {304} &  {402} &  {493} &  {662} &  {342} &  {546} &  {556} &  {662} &  {823}\\
 {Speaker 7} &  {338} &  {378} &  {381} &  {536} &  {473} &  {506} &  {544} &  {634} &  {981}\\
 {Speaker 8} &  {444} &  {469} &  {449} &  {597} &  {420} &  {491} &  {445} &  {654} &  {940}\\
 {Speaker 9} &  {255} &  {360} &  {350} &  {584} &  {365} &  {545} &  {544} &  {628} &  {815}\\
 {Speaker 10} &  {654} &  {421} &  {356} &  {583} &  {405} &  {490} &  {468} &  {635} &  {1028}\\
\midrule 
 {\textbf{Mean}} &  {348} &  {399} &  {392} &  {589} &  {388} &  {523} &  {477} &  {635} &  {925}\\
\lspbottomrule
\end{tabularx} 
\end{table}

\begin{table}[p]
\caption{F2 Formant for all participants} 
\begin{tabularx}{\textwidth}{Xrrrrrrrrr}
\lsptoprule
 {\textbf{F2}} &  {[i]} &  {[ɪ]} &  {[e]} &  {[ɛ]} &  {[u]} &  {[ʊ]} &  {[o]} &  {[ɔ]} &  {[a]}\\
\midrule
 {Speaker 1} &  {2265} &  {2167} &  {2192} &  {2068} &  {1383} &  {950} &  {1694}  &  {896} &  {1519} \\
 {Speaker 2} &  {2366} &  {2085} &  {2082} &  {1960} &  {856} &  {859} &  {839} &  {1047} &  {1525}\\
 {Speaker 3} &  {2298} &  {2462} &  {2350} &  {2397} &  {781} &  {716} &  {773} &  {964} &  {1551}\\
 {Speaker 4} &  {2202} &  {2231} &  {2211} &  {2043} &  {1216} &  {1151} &  {891} &  {1171} &  {1468} \\
 {Speaker 5} &  {2402} &  {2534} &  {2455} &  {2192} &  {1105} &  {841} &  {1041} &  {946} &  {1542}\\
 {Speaker 6} &  {1902} &  {1746} &  {1937} &  {1745} &  {918} &  {1248} &  {1754} &  {1103} &  {1382}\\
 {Speaker 7} &  {1911} &  {1977} &  {1931} &  {1906} &  {2103?} &  {817} &  {1429} &  {829} &  {1429}\\
 {Speaker 8} &  {2060} &  {2067} &  {2037} &  {2069} &  {1655} &  {1206} &  {1637} &  {1053} &  {1456}\\
 {Speaker 9} &  {2351} &  {2208} &  {2234} &  {2026} &  {1371} &  {1804} &  {1985} &  {993} &  {1538}\\
 {Speaker 10} &  {2304} &  {2269} &  {1985} &  {1982} &  {1960} &  {2228} &  {1885} &  {1561} &  {1455}\\
\midrule 
 {\textbf{Mean}} &  {2206} &  {2174} &  {2141} &  {2038} &  {1249} &  {1182} &  {1392} &  {1056} &  {1486}\\
\lspbottomrule
\end{tabularx} 
\label{tab:koffi:4}
\end{table}

Various statistical analyses can be run from the measurements in Tables 3 and 4. However, in this study they are used exclusively for the purpose of generating the acoustic vowel space in \figref{fig:koffi:5} and for explaining why \ili{Anyi} Morofu hearers have a problem distinguishing [ɪ] and [e] aurally. 
  
 
\begin{figure}
\includegraphics[height=.3\textheight]{figures/koffi5.png}
\caption{Anyi acoustic vowel space}
\label{fig:koffi:5}
\end{figure}


\subsection{The reason for the confusion}
\figref{fig:koffi:5} shows us visually why \ili{Anyi} Morofu hearers confuse [ɪ] and [e] aurally. We see that they overlap in perceptual space. The measurements in \tabref{tab:koffi:3} explain why. These two vowels mask each other aurally because [ɪ] (399 Hz) and [e] (392 Hz) are separated by only 7 Hz in the F1 domain. It is a well-known fact that a minimum of 20 Hz is needed for humans to perceive a difference between two sound segments \citep[34]{Ferrand2007}. It is also well known that the lowest frequency at which a sound is intelligible on an eight-octave frequency band is 63 Hz. Auditory frequency measuring devices and many audio applications use this baseline as their reference level \citep[12--16]{EverestPohlmann2015}. For acoustic phonetic analyses, this threshold has been rounded down to 60 Hz to make calculations simpler \citep[68]{Fry1979}. \citet[204--221]{LabovEtAl2006} use it in  \textit{Atlas of North American English}  (ANAE) to assess dialectal variations. \citet[43]{LabovEtAl2013} use it to assess vowel change in Philadelphia. The same 60 Hz threshold is used here to explain the confusion between [ɪ] and [e], and vice versa. Though F2 and F3 formants contribute to the overall perception of vowel quality, the calculations of vowel intelligibility are based on F1 because it alone contains 80\% of the acoustic energy in the vowel \citep[207]{LadefogedJohnson2015}. 

{The practical steps used to assess vowel intelligibility are as follows. On the F1 frequency band, if the acoustic distance between two contiguous front vowels or two contiguous back vowels is ${\geq}$ 60 Hz, the two vowels are perceived as distinct. However, if their acoustic distance falls between 59 and 21 Hz, intelligibility is compromised. If the acoustic distance between two vowels is ${\leq}$ 20 Hz, it means that a merger has taken place or is taking place. The reason for this is because human beings cannot perceive frequencies lower than 20 Hz. This is exactly what is going on with \ili{Anyi} Morofu. Hearers in general have a hard time distinguishing [ɪ] from [e], and vice versa, because the mean acoustic distance between them is only 7 Hz. There are, however, small inter-speaker variations. The segments [ɪ] and [e] produced by Speakers 5, 6, and 10 are intelligible because the acoustic distances between them are respectively 54 Hz, 91 Hz, and 65 Hz. However, for seven of the speakers [ɪ] and [e] are aurally indistinguishable. For Speakers 1, 2, 7, 8, and 9 the two vowels mask each other because the acoustic distances between them are ${\leq}$ 20 Hz, as shown in \tabref{tab:koffi:5}:}

\begin{table}
\caption{Inter-speaker variation}
\begin{tabularx}{.66\textwidth}{Xrrr}
\lsptoprule
 {\textbf{F1}} &  {[ɪ]} &  {[e]} &  {Distance}\\
\midrule
 {Speaker 1} &  {368} &  {355} &  {13}\\
 {Speaker 2} &  {407} &  {408} &  {1}\\
 {Speaker 3} &  {435} &  {408} &  {27}\\
 {Speaker 4} &  {344} &  {368} &  {24}\\
 {Speaker 5} &  {415} &  {361} &  {54}\\
 {Speaker 6} &  {402} &  {493} &  {91}\\
 {Speaker 7} &  {378} &  {381} &  {3}\\
 {Speaker 8} &  {469} &  {449} &  {20}\\
 {Speaker 9} &  {360} &  {350} &  {10}\\
 {Speaker 10} &  {421} &  {356} &  {65}\\
\midrule 
 {\textbf{Mean}} &  {399} &  {392} &  {7}\\
 {\textbf{Standard Deviation}} &  {37} &  {47} &  {28}\\
\lspbottomrule
\end{tabularx}
\label{tab:koffi:5}
\end{table}

{The situation in \ili{Anyi} is similar in this respect to the merger between [ɑ] and [ɔ] that is going on in several dialects of American \ili{English}. For Central Minnesota \ili{English}, \citet[5]{Koffi2013} reports that the merger between [ɑ] (855 Hz) and [ɔ] (851 Hz) is complete in the speech of female speakers because the acoustic distance between the two vowels is only 4 Hz.}

\largerpage
{Cross-linguistically, something is going on between [ɪ] and [e] that deserves further investigation. \citet[41--42]{Ladefoged1999} displays the vowels of a southern California speaker whose [e] has risen above [ɪ]. \citeauthor{Koffi2014}’s (2014:16--17) acoustic phonetic measurements of Central Minnesota \ili{English} show that that [e] is higher than [ɪ] in male and female speech. In male speech, [e] (434 Hz) is higher than [ɪ] (542 Hz) by 108 Hz. In female speech, [e] (508 Hz) has risen above [ɪ] (573 Hz) by 65 Hz. In these examples, the raising of [e] above [ɪ] does not result in unintelligibility because the acoustic distance between them is still higher than the 60 Hz threshold. However, this is not so in the case of \ili{Anyi} Morofu where only a mere 7 Hz separate these vowels. In the terminology that \citet{SchwartzEtAl1997dispersion} use to describe vowel systems, \ili{Anyi} is an “atypical” nine-vowel system because it does not conform to the predicted patterns. \citet[169]{Becker-Kristal2010} explains why:} 

\begin{quote}
{Across all analyses, inventories with ATR harmony often violate the principles of dispersion, in formant spans, in even vowel spacing and in phonetic adjustments in response to structural change. These deviations are understandable if such inventories are not treated as one large system but as two parallel smaller systems.}
\end{quote}

{More acoustic phonetic data such as the one used to describe the confusion between [ɪ] and [e] is needed from other African languages with [±ATR] vowel systems to see if \ili{Anyi} Morofu is really atypical or if this phenomenon is widespread. In the case of \ili{Anyi}, it is the vowels [ɪ] and [e]. In other languages, it may be different pairs of vowels.} 

\subsection{The future of the Anyi acoustic vowel space}
What does the future hold for the phonemic inventory of \ili{Anyi} vowels? How long before the acoustic vowel space is completely reduced to an eight-vowel system? Will the acoustic vowel space be reduced further to a seven-vowel system? \citet[113]{Becker-Kristal2010} discusses a possible scenario that may be in store for \ili{Anyi}:

\begin{quote}
{They [vowels] might fall closer to other vowels, which are repelled further, albeit by a smaller magnitude, and this process propagates as a push chain shift with gradual decay through other vowels until the entire system finds a new balance.}
\end{quote}

{What will the new balance look like for \ili{Anyi}? It is hard to predict the future. However, we can anticipate what the \ili{Anyi} Morofu vowel space will look like in the near future by learning from the current state of vowels in some languages in the \ili{Akan} family. \citet[430]{Mensah1983} reports that \ili{Krobou}, another \ili{Akan} language, has reduced the number of its vowels from nine to eight. It no longer has the vowel [ɪ], which has been replaced by [e]. If \ili{Anyi} finds a new balance in an eight-vowel system, this balance will be temporary because another shift is afoot. The data in \tabref{tab:koffi:6} shows that the next vowel targeted for disappearance is [ʊ] (523Hz):}

\begin{table}
\caption{The Impending Merger of [ʊ] and [o]}
\label{tab:koffi:6}
\begin{tabularx}{.75\textwidth}{Xrrr}
\lsptoprule
 {\textbf{F1}} &  {[ʊ]} &  {[o]} &  {Distance}\\
\midrule 
 {Speaker 1} &  {539} &  {481} &  {58}\\
 {Speaker 2} &  {595} &  {429} &  {166}\\
 {Speaker 3} &  {431} &  {414} &  {17}\\
 {Speaker 4} &  {583} &  {429} &  {154}\\
 {Speaker 5} &  {510} &  {469} &  {41}\\
 {Speaker 6} &  {546} &  {556} &  {10}\\
 {Speaker 7} &  {506} &  {544} &  {38}\\
 {Speaker 8} &  {491} &  {445} &  {46}\\
 {Speaker 9} &  {545} &  {544} &  {1}\\
 {Speaker 10} &  {490} &  {468} &  {22}\\
\midrule 
 {\textbf{Mean}} &  {523} &  {477} &  {46}\\
 {\textbf{Standard Deviation}} &  {48} &  {52} &  {57}\\
\lspbottomrule
\end{tabularx}
\end{table}

{It will most likely be replaced by [o] (477 Hz). The acoustic distance between them is 46 Hz. This merger may take a little while, but it is inevitable. Only Speakers 2 and 4 mark a clear contrast between these two vowels. Intelligibility is compromised in the speech of Speakers 1, 5, and 8. A merger has already taken place in the pronunciation of Speakers 3, 6, and 9; and it is on the verge of happening for Speaker 10. The merger between [ʊ] and [o] has already taken place in \ili{Baule}, which is closely related to \ili{Anyi}. \citet[284]{Kouadio1983} reports that \ili{Baule} no longer has [ɪ] or [ʊ]. Other languages in the \ili{Akan} family spoken in Côte d’Ivoire have a seven-vowel system instead of the nine traditionally associated with this language family. \citet[262]{Hérault1983} reports that Avikam has lost both [ɪ] and [ʊ], and so has Ebrie \citep[324]{Bole-Richard1983}. The acoustic vowel space of \ili{Anyi} Morofu will achieve stability when the number of its vowels goes from nine to seven.} 

\section{Summary}
\largerpage
The vowel spaces of languages are always shifting. \ili{English} underwent a \isi{major} shift between 1400 and 1600 \citep[342]{FromkinEtAl2014}). This change has been nicknamed the Great Vowel Shift. \citegen{LabovEtAl2006} voluminous \textit{ANAE} shows that another shift known as the Northern Cities Shift is slowly but surely fanning across the Midwest. Since change is a language universal process, one would expect the vowels of \ili{Anyi} to also shift. The vowel [e] is masking [ɪ] for now. How long will it take for [ɪ] to be swallowed up by [e]? It is hard to tell. However, the process that is underway is almost irreversible given what has taken place in other \ili{Akan} languages that are closely related to \ili{Anyi} Morofu. For now \ili{Anyi} is following the same path as \ili{Krobou}. In a not so distant future, the shift from [ʊ] to [o] will run its course, and \ili{Anyi} will have seven vowels like \ili{Baule} and other \ili{Akan} languages spoken in Côte d’Ivoire. However, synchronically, \ili{Anyi} is in a between and betwixt state which causes it to be atypical, that is, it does not conform to PPS as predicted by CVS and DFT.  

 {\sloppy
\printbibliography[heading=subbibliography,notkeyword=this]
}

\end{document}