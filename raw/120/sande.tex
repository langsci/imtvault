\documentclass[output=paper,modfonts]{langscibook} 
\title{Classification of Guébie within Kru} 
\author{Hannah Leigh Sande}
\abstract{Guébie, a Kru language spoken in Côte d'Ivoire, is currently doubly classified within Eastern Kru according to Ethnologue \citep{LewisEtAl2013}. It is listed as a dialect of two distinct subgroups, Bété and Dida. This double classification is clearly problematic, and this paper provides the initial work towards addressing the correct classification of the language. Here I compare the phonological and syntactic properties of Guébie with surrounding Bété and Dida languages in order to determine its relatedness to each subgroup. I conclude that Guébie is more closely related to Vata, a Dida language, than to Bété.\\}
\ChapterDOI{10.5281/zenodo.1251712}
\begin{document}
\maketitle
 
 
   

% \keywords{Kru, classification, relatedness, Guébie}

\section{Introduction}\label{sec:sande:Intro} 
\ili{Kru} is a branch of Niger-\isi{Congo} languages spoken in Côte d'Ivoire and Liberia. There are two \isi{major} subdivisions of \ili{Kru} languages: Eastern and Western \ili{Kru} \citep{Marchese1979}; however, there has been very little work done on the internal classification of those two branches. Many `languages' classified as either Eastern or Western \ili{Kru} are in fact subgroups of related languages. For this reason, many \ili{Kru} languages are grouped together as a single language for classification purposes, sometimes for empirical reasons and sometimes for political or geographic ones, leading to linguistic misclassifications and inadequate descriptions of individual \ili{Kru} varieties. In this paper I address one such case, that of Guébie, an Eastern \ili{Kru} language spoken in southwest Côte d'Ivoire. I attempt to provide an initial classification of this particular \ili{Kru} language.

Guébie is a particularly interesting case of misclassification. It is currently classified twice in Ethnologue \citep{LewisEtAl2013},\footnote{The revised version of this chapter was completed in 2015. The version of Ethnologue cited throughout this paper is the \citeyear{LewisEtAl2013} edition, which was the most recent at the time of writing.} and I argue here that neither classification is accurate. \cite{LewisEtAl2013} calls Guébie both a dialect of \ili{Bété}-Gagnoa (btg), and an alternative name for \ili{Dida}-\ili{Lakota} (dic). \ili{Bété}-Gagnoa has 150,000 speakers according to Ethnologue \citep{LewisEtAl2013}, and \ili{Dida}-\ili{Lakota} (dic) has 94,000 speakers. Based on my own observations during fieldtrips along with estimates from consultants, I estimate that Guébie, the language in question, has only 7,000 remaining speakers. %elaborate: when and wehre and how many speakers
Thus, classifying Guébie as a dialect of \ili{Bété}-Gagnoa or an alternative name for \ili{Dida}-\ili{Lakota} is incredibly misleading in terms of the number of remaining speakers and vitality of the language.

The goal of this paper is to determine the appropriate classification of Guébie and advocate that it be classified separately from both \ili{Bété}-Gagnoa and \ili{Dida}-\ili{Lakota}. I will demonstrate based on original Guébie data that Guébie is more closely related to \ili{Vata}, a dialect of \ili{Dida}-\ili{Lakota}, than to \ili{Gbadi}, a dialect of \ili{Bété}-Gagnoa. I choose these two particular varieties as standards of comparison because there is more data available for \ili{Vata} and \ili{Gbadi} than for other \ili{Dida} and \ili{Bété} languages. I conclude based on this initial study that Guébie is more closely related to \ili{Dida} languages than \ili{Bété} ones, and further, that it is distinct from \ili{Vata} (\ili{Dida}-\ili{Lakota}), thus it should be classified as a distinct \ili{Dida} language.

I begin in \sectref{sec:sande:Background} with background information on Guébie and its current classification within \ili{Kru}. I turn in \sectref{sec:sande:methods} to the methodology used here to determine relatedness of languages. In \sectref{sec:sande:P-side} I discuss the somewhat limited phonetic and phonological evidence that Guébie is more closely related to \ili{Dida}-\ili{Lakota} than to \ili{Bété}-Gagnoa, and in \sectref{sec:sande:S-side} I discuss the more readily available syntactic data which supports the conclusion that Guébie is a \ili{Dida} language, but is distinct from \ili{Dida}-\ili{Lakota}. I conclude in \sectref{sec:sande:Implications} with the implications and conclusions of the present study, arguing that Guébie be classified as distinct from both \ili{Bété}-Gagnoa and \ili{Dida}-\ili{Lakota}, contra the current Ethnologue classification \citep{LewisEtAl2013}.

%%comparative method doesn't work here, since we're not trying to reconstruct a proto-language; simply to show close genetic relationship between two languages and not two others.
%%Citations for lexical items as the first things borrowed (historical handouts?)
%%Not conforming to any specific theory or method of classification here, since none seems quite right for my purposes; rather showing what must be due to relationship and regular change as opposed to borrowings and later change.

\section{Background: The current classification of Guébie}\label{sec:sande:Background}

The current literature agrees that \ili{Kru} is a branch of Niger-\isi{Congo} (cf. \citealt{Marchese2012}). However, there have been many other contradictory proposals in the past. There have been claims that \ili{Kru} is related to Gur \citep{Vogler1974}, Kwa \citep{Greenberg1963}, and Mande \citep{Bennett1977}. \cite{Welmers1977} put forth the proposal that \ili{Kru} is a distinct branch of Niger-\isi{Congo}, and that has been corroborated by \cite{Marchese1979} and later work. There are two \isi{major} branches of \ili{Kru} languages, Eastern and Western, which are geographically split near the country boarder of Côte d'Ivoire and Liberia. Guébie is an undescribed Eastern \ili{Kru} language spoken in seven small villages in southwest Côte d'Ivoire about 30 kilometers southwest of Gagnoa and 30 kilometers east of \ili{Lakota}. The largest of these villages, Gnagbodougnoa, has a population of 1000, about two thirds of whom are Guébie-speaking. The remaining third are \ili{Lobi} speakers who were displaced during the national crisis of Côte d'Ivoire in the 1990s. Since Guébie is spoken in just seven villages, the largest of which has a population of 1000, I estimate that Guébie has at most 7000 speakers.

The data presented in this paper comes from original work with native speakers of Guébie. For eight months I worked with a native speaker temporarily living in Berkeley, California. This was followed by fieldwork in Gnagbodougnoa, Côte d'Ivoire in the summer of 2014. My consultants are five in number, include both men and women, and range in age from 19--76. This paper presents the first published documentation and description of Guébie.

Guébie villages are situated amidst a dense rainforest where temperatures are high year-round and there are two rainy seasons, one in June and the other in December. The community of Guébie speakers are subsistance farmers who grow cassava, rice, and plantains. Rarely are these plants farmed for profit. A small percentage of Guébie people farm and sell cacao and coffee. Only one third of children attend school, while the others work on their family's farm. There is only one known monolingual speaker, my consultant Serikpa Emil, who is 76 years old. The rest of the Guébie-speaking population is bilingual in \ili{French}. Children are still learning Guébie, however within a single generation, I predict that children will no longer learn Guébie as a first language.

The Eastern \ili{Kru} family tree in \figref{familytree} is adapted from Ethnologue \citep{LewisEtAl2013}.
%%%map!

% \begin{samepage}
% \begin{exe}

% \ex 
\begin{figure}
\caption{Eastern Kru family tree}
\label{familytree}
% \begin{tikzpicture}[scale=.7]
% \fittable{
% \Tree[.{Eastern Kru} [.Bakwe-Wane ][.\ili{Bété} [.Eastern [.Gagnoa [.\textbf{Guébie} ][.\ili{Gbadi} ]][.Kouya ]][.Western [.Daloa ][.Guiberoua ][.Godié ]]][.\ili{Dida} [.\ili{Lakota} [.\ili{Vata}/\textbf{Guébie} ]][.\ili{Dida}-Yocoboué ][.Neyo ]][.Kwadia ]]
% }
\resizebox{\textwidth}{!}{%
\begin{forest} forked edges, for tree={fit=rectangle, s sep=2mm, inner xsep=0mm}
[Eastern \ili{Kru} [Bakwe-Wane] [\ili{Bété} [Eastern [Gagnoa [\textbf{Guébie}] [\ili{Gbadi}] ] [Kouya] ] [Western [Daloa] [Guiberoua] [Godié] ] ] [\ili{Dida} [\ili{Lakota} [\ili{Vata}/\textbf{Guébie}] [\ili{Dida}-Yocoboué] [Neyo]] ] [Kwadia] ]
\end{forest}}
% \end{tikzpicture}
% \end{exe}
% \end{samepage}
\end{figure}

Guébie, bolded in the tree above, is currently classified twice, as part of two distince Eastern \ili{Kru} varieties. Due to geographic, historical, and possibly ethnic reasons, Guébie has been called a dialect of \ili{Bété}-Gagnoa, spoken by about 150,000 people \citep{LewisEtAl2013}. Due to linguistic similarity, Guébie is also listed as a second name for \ili{Dida}-\ili{Lakota}, an Eastern \ili{Kru} language spoken by about 94,000 people in the city of \ili{Lakota}, east of the Guébie-speaking area.

%%%Mention \ili{Gbadi} and \ili{Dida}-\ili{Lakota}, comparing the three, all geographically close, very little contact today between speech communitites

\section{Methodology}\label{sec:sande:methods}

The most widely accepted means of establishing genetic relationship between languages is the Comparative Method (cf. \citealt{Bloomfield1933} chapter 18, among others), which determines whether sound correspondences across languages are regular, thus the result of regular sound change. For the Comparative Method to be of use requires dictionaries or lists of lexical items in the languages in question to be used for comparison. 

Ideally, we would compare sound correspondences across cognates in the lexicon of Guébie with the lexicon of \ili{Bété}-\ili{Gbadi} and \ili{Vata} (\ili{Dida}-\ili{Lakota}) in order to determine relatedness. However, there has not been enough thorough documentation across \ili{Bété} and \ili{Dida} languages for comparing sound correspondences to be informative. The available resources for Guébie include only my own data elicted from September 2013 to July 2014. %%Add number of lexical items, number of texts, lines, hours, something.
For \ili{Bété}-\ili{Gbadi}, there is a dictionary written by a native speaker linguist \citep{Zogbo2005}. For \ili{Vata} there is a small list of lexical items compiled from various works on the phonology and syntax of the language \citep{Marchese1979, Kaye1981, Koopman1984}. Amongst these resources, I have found fewer than 100 cognates to compare across the three languages. This number of cognates is not enough to determine relatedness based on sound change; however, I will present what minimal data exists in \sectref{sec:sande:P-side}.

For each of the languages in question, Guébie, \ili{Bété}-\ili{Gbadi}, and \ili{Vata}, there is some not insignificant amount of syntactic description in the literature. Basing genetic relationship on syntactic correspondences is less widely accepted than on sound correspondences and morphological paradigms (\citealt{Garrett2012, RingeEska2014}, and citations therein). However, since there are no available lexicon lists for most \ili{Kru} languages, I propose that we examine syntactic features of \ili{Bété} and \ili{Dida} languages, comparing them with Guébie to determine how these languages are related.

Lexical items are likely to be borrowed heavily from neighboring languages, and morphosyntactic correspondences are more likely to be conservative or undergo change less rapidly over time. Thus, if we can find convincing syntactic similarities between two languages, A and B (say, Guébie and \ili{Vata}), but not between two others, A and C (say, Guébie and \ili{Bété}-\ili{Gbadi}), we may be able to tentatively say that the first two languages, A and B, are more closely related than the latter two. Comparing morphosyntactic features of languages has been crucial in determining the relatedness of Proto-Anatolian languages \citep{Melchert2013}, and the place of \ili{Armenian} within Indo-European \citep{Hübschmann1875}. Thus, it is possible, however rare, for morphosyntactic similarity to influence decisions about language classification.

In the following two sections I compare the linguistic properties of Guébie with its two geographically closest documented neighbors, \ili{Bété}-\ili{Gbadi} and \ili{Vata}. \ili{Bété}-\ili{Gbadi} is spoken in just north of the Guébie-speaking villages. The \ili{Bété}-\ili{Gbadi} data here comes from \cite{Marchese1979, Zogbo2005, Koopman1984}. \ili{Vata} is a \ili{Dida} language spoken east of the Guébie-speaking villages. The data here comes from \cite{Marchese1979, Kaye1981, Kaye1982, Koopman1984}. \ili{Vata} is a dialect of \ili{Dida}-\ili{Lakota} spoken slightly northeast of \ili{Lakota}. Ideally we would compare Guébie with \ili{Dida}-\ili{Lakota} spoken in the city of \ili{Lakota}; however, there is very little work on the \ili{Dida} spoken in \ili{Lakota}. \ili{Vata} is slightly geographically further from Guébie; however, since it is a well documented and closely related to \ili{Dida}-\ili{Lakota}, I compare \ili{Vata}, not the \ili{Dida} of \ili{Lakota}, with Guébie.

I demonstrate throughout the remainder of this paper that the linguistic properties of Guébie and \ili{Vata} are too similar to be due to chance, and that Guébie is less closely related to \ili{Bété}-\ili{Gbadi} than to \ili{Vata}. The words and sounds that do correspond in Guébie and \ili{Bété} must be due to borrowing, or must have been present in Proto-Eastern-\ili{Kru}. There is very little evidence that \ili{Bété} and Guébie underwent any regular changes that other Eastern \ili{Kru} languages did not undergo. Any assumptions made here about the features of Proto-\ili{Kru} or Proto-Eastern-\ili{Kru} come from \cite{Marchese1979} and \cite{Marchese2012}.


%%%If a syntactic structure exists in guebie and dida that is not in \ili{Bété} AND it was not in the proto-language, it would be strange to posit that it (and many others like it) changed separately in both guebie and dida while gbadi and other \ili{Bété} languages did not undergo the change. How do we know it's not contact? an overwhelming amount of structural similarity and much lexical similarity. Not just lexical as we might expect if it were only due to contact. Without lexical lists from all of the languages in question, we cannot say for sure about the sound changes, but the syntactic evidence makes it quite clear that guebie is more closely related to dida languages than \ili{Bété}. %%references from Garret "Sound change", bloomfield, 

\section{Phonetic and phonological evidence}\label{sec:sande:P-side}

There is very little, if any, phonetic or phonological evidence that can concretely determine the classification of Guébie as \ili{Bété} or \ili{Dida}. I present here some basic similarities and differences between \ili{Bété}-\ili{Gbadi} and Guébie, and \ili{Vata} and Guébie.

\subsection{Tone}

The data in (\ref{Guebietone}) shows that there are four contrastive tonal heights in Guébie. The words in (\ref{Guebietone}a) are segmentally identical and are distinguished only by the height of their level tone. The data in (\ref{Guebiecontours}) shows that Guébie also makes use of contrastive contour tones. There are two distinct rising tones and three distinct falling tones in Guébie.

\begin{exe}
\ex \textbf{Four contrastive tone heights in Guébie}\label{Guebietone}\\
\begin{tabular}{lllllllll}
% \hline
a. & \textipa{ko}\textsuperscript{1} & `line/row' & \textipa{ko}\textsuperscript{2} & `pestle' & \textipa{ko}\textsuperscript{3} & `skin' & \textipa{ko}\textsuperscript{4} & `cadavre'\\
% \hline
b. & & & \textipa{no}\textsuperscript{2} & `beverage' & & & \textipa{no}\textsuperscript{4} & `woman'\\
% \hline
c. & \textipa{\textbardotlessj e}\textsuperscript{1} & `egg' & & & \textipa{\textbardotlessj e}\textsuperscript{3} & `star' & \textipa{\textbardotlessj e}\textsuperscript{4} & `number'\\
% \hline
\end{tabular}
\ex \textbf{Contour examples}\label{Guebiecontours}\\
\begin{tabular}{ll}
\textipa{\textbardotlessj a}\textsuperscript{31} & `coconuts'\\
\textipa{vO}\textsuperscript{13} & `horns'\\
\textipa{su}\textsuperscript{2} & `tree'\\
\textipa{su}\textsuperscript{13} & `to shove'
\end{tabular}
\end{exe}

There are four contrastive tonal heights in Guébie. There are also four contrastive tonal heights in \ili{Vata} \citep{Marchese1979,Kaye1981,Koopman1984}. It is controversial whether there are three or four contrastive tonal heights in \ili{Bété}-\ili{Gbadi}. \cite{Zogbo2005}'s dictionary lists four contrastive tones for \ili{Bété}-\ili{Gbadi}; however, \cite{Marchese1979, Marchese1989} says that the four tonal heights posited for Proto-\ili{Kru} have collapsed into three heights in \ili{Bété}. Tones throughout this paper are marked with numbers 1--4, where 4 is the highest tone and 1 is the lowest. A dot between tones separates syllables, and two numbers within a syllable signifies a contour tone.

\subsection{Vowels}

There are ten contrastive vowels in Guébie, distinguished by height, backness, rounding and ATR value. /\textipa{@}/ is the +ATR counterpart of /a/ in Guébie.

\pagebreak
\begin{exe}
\ex
 \textbf{Guébie vowel inventory}

\begin{tikzpicture}
\aeiouEO
 \node at (1.5,1.5) (ә) {ә};
 \node at (2,2.5) (ʊ) {ʊ};
 \node at (1,2.5) (ɪ) {ɪ};

\end{tikzpicture}
% % \todo[inline]{add other vowels}
% 		\begin{vowel}
% 			%    \putcvowel[l]{i}{1}
%     		\putvowel[l]{i}{0pt}{0pt}
%    			%\putcvowel[r]{y}{1}
%    			\putcvowel[l]{e}{2}
%    			%\putcvowel[r]{\o}{2}
%    			\putcvowel[l]{\textepsilon}{3}
%   			%\putcvowel[r]{\oe}{3}
%     		%\putcvowel[l]{a}{4}
%     		%\putcvowel[r]{\textscoelig}{4}
%    			%\putcvowel[l]{\textscripta}{5}
%     	    %\putcvowel[r]{\textturnscripta}{5}
%     		%\putcvowel[l]{\textturnv}{6}
%     		\putcvowel[r]{\textopeno}{6}
%     		%\putcvowel[l]{\textramshorns}{7}
%     		\putcvowel[r]{o}{7}
%     		%\putcvowel[l]{\textturnm}{8}
%     		\putcvowel[r]{u}{8}
%     		%\putcvowel[l]{\textbari}{9}
%     		%\putcvowel[r]{\textbaru}{9}
%     		%\putcvowel[l]{\textreve}{10}
%     		%\putcvowel[r]{\textbaro}{10}
%     		\putcvowel{\textschwa}{11}
%     		%\putcvowel[l]{\textrevepsilon}{12}
%     		%\putcvowel[r]{\textcloserevepsilon}{12}
%     		\putcvowel{\textsci\ %\textscy
%     		}{13}
%     		\putcvowel{\textupsilon}{14}
%     		\putcvowel{%\textturna
%     		a}{15}
%     		%\putcvowel{\ae}{16}
% 		\end{vowel} 
\end{exe}

There are also ten contrastive vowels in \ili{Vata}, identical to those in the vowel inventory in the chart above. There are only seven contrastive vowels in \ili{Bété}-\ili{Gbadi}. \ili{Bété}-\ili{Gbadi} lacks an ATR distinction between the mid and low vowels. \cite{Marchese2012} posits seven contrastive vowels for Proto-\ili{Kru}, which means that \ili{Vata} and Guébie each separately underwent vowel splits resulting in a larger vowel inventory. Alternatively, \ili{Vata} and Guébie share a common ancestor that \ili{Bété}-\ili{Gbadi} does not share, and that ancestor underwent regular phoneme splits, adding three vowels to the inventory. This ten-vowel inventory was then passed down to both \ili{Vata} and Guébie. Without more information we cannot say for sure which path of development of these vowel systems is the correct one.

There are marginal nasal vowels in Guébie. The three vowels [\textipa{\~E, \~O, ã}] are found in a just a handful of words. Proto-Eastern-\ili{Kru} did not have nasal vowels. However, nearby Kwa and Mande languages, frequently in contact with \ili{Kru} languages, have these same three nasal vowels. This contact could have resulted in the borrowing of Kwa and Mande words into Guébie, so that the borrowed words but no native words contain nasal vowels in Guébie. The three Guébie words that I have found containing nasal vowels are given in (\ref{nasal}). 


\begin{exe}
\ex \textbf{Guébie nasal vowels}\label{nasal}\\
\begin{samepage}
\begin{tabular}{lll}
a. & kã\textipa{\~O}\textsuperscript{4.2} & `spine'\\
b. & \textipa{\textbardotlessj i\~E}\textsuperscript{3.1} & `sea'\\
c. & \textipa{kpã\~E}\textsuperscript{4.4} & `very, a lot'\\
\end{tabular}
\end{samepage}
\end{exe}

In \ili{Bété}-\ili{Gbadi}, nasal vowels are also found in a few, likely non-native, words. One of these is the same word, \textipa{kpã\~E} `very', that contains a nasal vowel in Guébie. Because both Guébie and \ili{Bété}-\ili{Gbadi} have had contact with surrounding Kwa languages in the past, it seems likely that both languages borrowed words containing nasal vowels from those Kwa languages. Alternatively, one of the two, Guébie or \ili{Bété}-\ili{Gbadi}, could have borrowed the words in (\ref{nasal}) and these words could have in turn been borrowed into the other. Either way, it seems unlikely that nasal vowels were an innovation via regular sound change shared by Guébie and \ili{Bété}; borrowing seems like a more plausible option because there are so few words in each language that contain nasal vowels, and the words that do have nasal vowels overlap in Guébie and \ili{Bété}-\ili{Gbadi}.

Guébie has pervasive ATR harmony from roots to suffixes. \ili{Vata} shares the same ATR harmony process, though \ili{Bété}-Guébie does not \citep{Koopman1984}. Guébie ATR harmony can be seen in (\ref{harmony}) where the \isi{causative} morpheme is a low vowel suffix on verbs. Verb roots that contain +ATR vowels take the +ATR \isi{causative} suffix [-\textipa{@}] while those containing -ATR vowels in the \isi{root} take the -ATR \isi{causative} suffix [-a]. The same process can be seen with the \ili{Vata} definite marker, where +ATR \isi{root} vowels result in +ATR suffix vowels. There is rounding harmony in the \ili{Vata} data that does not occur in Guébie. The \ili{Gbadi} data shows that +ATR roots do not result in +ATR suffixes. This is likely a factor of the limited vowel inventory of \ili{Gbadi}, however, where only high vowels show an ATR distinction. There are no suffixes in \ili{Gbadi} that contain high vowels, so we can not say for certain whether or not there would be ATR harmony between roots and suffixes in high vowels in \ili{Gbadi}.

\begin{exe}
\ex\label{harmony} \textbf{Vowel harmony data}\\
\begin{tabular}{lllll}
% \hline
\textbf{Guébie} & \textipa{ci-@}$^{3.1}$& `to cause to learn' & \textipa{jE-a}$^{3.2}$ & `to cause to dance' \\
% \hline
\textbf{Vata} & \textipa{sle-e}$^{2.3}$ & `the house' & \textipa{gbU-O}$^{2.3}$ & `the cause' \\
% \hline
\textbf{Gbadi} & \textipa{li-a}$^{2.2}$& `to cause to eat' & \textipa{jue-a}$^{4.4.2}$ & `the children'\\
% \hline	
\end{tabular}
\end{exe}

Without a high-vowel suffix in \ili{Bété}-\ili{Gbadi}, we cannot say for certain whether all three languages have ATR harmony, or whether only \ili{Vata} and Guébie share this ATR harmony process. If the latter is true, either this harmony process arose separately in \ili{Vata} and Guébie, or it was lost relatively recently in \ili{Gbadi}. Alternatively, \ili{Vata} and Guébie share a common ancestor that \ili{Bété} does not share, and that ancestor acquired a harmony process that \ili{Bété} did not. More data is needed to know for certain.

\subsection{Consonants}
There is too little lexical data available from documented \ili{Bété} and \ili{Dida} languages to show regular sound changes in consonants that led from a Proto-language to the currently spoken languages.

\subsection{Summary}
Until further data is collected, the existing phonological data on these languages do not tell us much about their genetic relationship. What we can conclude from the above is that there are no known shared changes between only Guébie and \ili{Bété}-\ili{Gbadi}. All of the features that Guébie shares with \ili{Bété} are also present in \ili{Vata}, or there is evidence that those features a result of borrowing (nasal vowels). In the following section I turn to syntactic evidence of relatedness.

\section{Syntactic evidence}\label{sec:sande:S-side}
In this section I compare certain syntactic properties of Guébie with those of \ili{Bété}-\ili{Gbadi} and \ili{Vata} (\ili{Dida}). I am limited by the specific syntactic properties that have been described for all three languages in question, though there are numerous syntactic features of Guébie that more closely resemble \ili{Vata} than \ili{Bété}-\ili{Gbadi} and other \ili{Bété} dialects. 

Before describing the differences, it is important to note that all three languages share S AUX O V alternating with SVO \isi{word order}. I turn now to those properties that are not shared amongst all three languages.

\subsection{Aspect marking}
The two \isi{major} aspectual distinctions in Guébie, imperfective and \isi{perfect}, are distinguished by tone. All imperfective verbs have tone one step lower on the four-tone scale than the corresponding \isi{perfect} \isi{verb}, (\ref{aspecttone}).

\begin{exe}
\ex \textbf{Perfective vs Imperfective in Guébie}\label{aspecttone}
\begin{itemize}
\item[a.] \gll \textipa{O}\textsuperscript{3} \textipa{li}\textsuperscript{2} \textipa{\textbardotlessj a}$^{31}$\\
\textsc{3.sg} eat.\textsc{impf} coconuts\\
\trans `He eats coconuts'
\item[b.] \gll \textipa{O}\textsuperscript{3} \textipa{li}\textsuperscript{3} \textipa{\textbardotlessj a}$^{31}$\\
\textsc{3.sg} eat.\textsc{perf} coconuts\\
\trans `He ate coconuts (recently)'
\end{itemize}
\end{exe}

\ili{Vata} (and \ili{Dida}-\ili{Lakota}, \citealt{Kaye1982}) also distinguishes imperfective from \isi{perfective} aspect with tone.

\ili{Bété}-\ili{Gbadi}, on the other hand, distinguishes imperfective from \isi{perfective} aspect with auxiliary particles, and the \isi{verb} surfaces finally: S AUX O V.

\begin{exe}
\ex \textbf{Comparing aspect distinctions}\\
\begin{tabular}{lcc}
\textbf{Language} & \textbf{Perfective} & \textbf{Imperfective}\\
% \hline
\textbf{Guébie} & Tonal & Tonal \\
\textbf{Vata} & Tonal & Tonal\\
\textbf{Gbadi} & Particle & Particle\\
\end{tabular}
\end{exe}

This difference means that in Proto-Eastern-\ili{Kru} there was either an aspect-marking auxiliary that was reduced to a tonal morpheme in Guébie and \ili{Vata}, or there was historically a tonal morpheme that was replaced by an auxiliary in \ili{Bété}-\ili{Gbadi}, but not in all Eastern \ili{Kru} languages.

\subsection{Causation}
There are two methods of adding a \isi{causative} meaning to a \isi{verb} in Guébie. One of these is suffixal, shown in (\ref{guebiecause}).

\begin{exe}
\ex \textbf{Guébie suffixal causative}\label{guebiecause}\\
\begin{tabular}{lllll}
a. & \textipa{ci}$^{31}$ & `to learn' & \textipa{ci-@}$^{3.1}$& `to cause to learn, to teach'\\ 
b. & \textipa{jE}\textsuperscript{3} &   `to dance' & \textipa{jE-a}$^{3.2}$ & `to cause to dance' \\
\end{tabular}
\end{exe}

The second means of causativization in Guébie is with a clausal construction meaning literally ``X speaks and Y Zs,'' where X is the causer, and Y is the subject of the \isi{verb} Z.

\begin{exe}
\ex \textbf{Clausal \isi{causative} in Guébie}
\begin{itemize}
\item[ ] \gll \textipa{kOgUlI\textltailn O-wa}$^{4.2.2.2.3}$ gba\textipa{}\textsuperscript{2} n\textipa{e}\textsuperscript{4} j\textipa{u-wa}$^{4.4}$ \textipa{O}\textsuperscript{3} \textipa{li}\textsuperscript{2}\\
farmer-\textsc{def} speak and boy-\textsc{def} \textsc{3.sg} eat.\textsc{perf}\\
\trans `The farmer is making the boy eat'
\end{itemize}
\end{exe}

The suffixal \isi{causative} in (\ref{guebiecause}) is present in both \ili{Vata} and \ili{Bété}-\ili{Gbadi} as well. However, the clausal \isi{causative} is only found in \ili{Vata} and Guébie, not \ili{Bété}-\ili{Gbadi}.

\protectedex{
\begin{exe}
\ex \textbf{Clausal \isi{causative} in Vata}
\begin{itemize}
\item[ ] \gll \textipa{n}\textsuperscript{3} \textipa{gba}\textsuperscript{2} \textipa{le}\textsuperscript{3} \textipa{yO-O}$^{3.3}$ \textipa{li}\textsuperscript{2}\\
I speak and boy-\textipa{def} eat\\
\trans `I made the boy eat.'
\end{itemize}
\end{exe}
}

It is possible that the clausal \isi{causative} construction was borrowed into Guébie from \ili{Vata} or vice versa; however, it is also possible that the two share a common ancestor that \ili{Bété} does not share, and the clausal \isi{causative} was innovated in that ancestor language, inherited into both Guébie and \ili{Vata}. All we can say is that there is no commonality between \ili{Bété} and Guébie that \ili{Vata} does not also share.


\subsection{WH-questions}
Wh-questions in Guébie are formed with a clause-final question marker and a clause-initial question word meaning literally `person, place, thing' for `who, where, what,' respectively.

\begin{exe}
\ex \textbf{Wh-questions in Guébie}
\begin{itemize}
\item[a.] \gll \textipa{\textltailn Okpa}$^{3.3}$ \textipa{touri}$^{1.1.3}$ %\textipa{ka\textsuperscript{3}/
\textipa{ji}\textsuperscript{3} \textipa{lEtrI}$^{3.2}$ \textipa{kOpa}$^{3.23}$ \textipa{na}\textsuperscript{3}\\
who Touri.\textsc{name} %\textsc{irr}/
will letter send \textsc{q}\\
\trans `To whom will Touri send a letter?'
\item[b.] \gll \textipa{bEba}$^{2.2}$ \textipa{touri}$^{1.1.3}$ %\textipa{ka\textsuperscript{3}/
\textipa{ji}\textsuperscript{3} \textipa{\textbardotlessj aci}$^{2.2}$ \textipa{kOpa}$^{3.23}$ \textipa{na}\textsuperscript{3}\\
what Touri.\textsc{name} %\textsc{irr}/
will Djatchi.\textsc{name} send \textsc{q}\\
\trans `What will Touri send to Djatchi?'
\end{itemize}
\end{exe}

According to (\citealt[87]{Koopman1984}), Wh-questions are formed in \ili{Bété}-\ili{Gbadi} with an initial Wh-word and a clause-medial question marker, while in \ili{Vata} they are formed with an initial Wh-word and a final question marker. This means that once again the \ili{Vata} construction (\ref{vatawh}) is identical to the Guébie construction, while the \ili{Bété} construction is distinct. The chart in (\ref{whchart}) shows the similarity between the Guébie and \ili{Vata} but not the Guébie and \ili{Bété} Wh-construction.
\newpage 

\begin{exe}
\begin{samepage}
\ex \textbf{Wh-questions in Vata}\label{vatawh}
\begin{itemize}
\item[ ] \gll \textipa{alO}$^{1.4}$ \textipa{O}\textsuperscript{1} \textipa{le}\textsuperscript{2} \textipa{saka}$^{3.4}$ \textipa{la}\textsuperscript{1}\\
who he eat rice \textsc{wh}\\
\nopagebreak
\trans `Who eats rice?'
\end{itemize}


\ex \textbf{Comparing Wh-constructions}\label{whchart}\\
\begin{tabular}{lccc}
\textbf{Language} & \textbf{Initial} & \textbf{Medial} & \textbf{Final}\\
% \hline
\textbf{Guébie} & Wh & -- & +\textsc{q}\\
\textbf{Vata} & Wh & -- & +\textsc{q}\\
\textbf{Gbadi} & Wh & +\textsc{q} & --\\
\end{tabular}
\end{samepage}
\end{exe}

Much like the \isi{causative} construction in the previous section, we see a similarity between Guébie and \ili{Vata} that could be the result of borrowing or common inheritance. More data, prefereably historical data, is needed to know for certain.

\subsection{Gerunds}
There are two types of gerund formation in Guébie. One form of the gerund is formed by reduplicating the \isi{verb} and adding the suffix /-je/. The other is formed by adding the suffix /-li/ to the \isi{verb}\footnote{The two gerund formation strategies described here have distinct distributions that I cannot yet cleanly define.}. The former gerund construction is found in \ili{Bété}-\ili{Gbadi} and not \ili{Vata}, while the latter gerund construction is found in \ili{Vata} but not \ili{Bété}-\ili{Gbadi} \citep{Koopman1984}.

\begin{exe}
\ex \textbf{Gerunds in Guébie}\label{15}
\begin{itemize}
\item[a.] \gll \textipa{saka}$^{3.3}$ \textipa{la}\textsuperscript{2} \textbf{\textipa{li-li-je}$^{2.2.1}$}\\
rice \textsc{gen} eat-eat-\textsc{nom}\\
\trans `Rice-eating'
\item[b.] \gll \textipa{saka}$^{3.3}$ \textipa{la}\textsuperscript{2} \textbf{\textipa{pi-li}$^{3.1}$}\\
rice \textsc{gen} cook-\textsc{nom}\\
\trans `Rice-preparing'
\end{itemize}

\ex \textbf{Gerunds in Vata}\label{16}
\begin{itemize}
\item[ ] \gll \textipa{saka}$^{3.4}$ \textipa{la}\textsuperscript{2} \textbf{\textipa{pi-li}$^{2.1}$}\\
rice \textsc{gen} cook-\textsc{nom}\\
\trans `Rice-preparing'
\end{itemize}
\ex \textbf{Gerunds in \ili{Bété}-Gbadi}\label{17}
\begin{samepage}
\begin{itemize}
\item[ ] \gll \textbf{\textipa{li-li-je}$^{2.2.1}$}\\
eat-eat-\textsc{nom}\\
\trans `Eating'
\end{itemize}
\end{samepage}
\end{exe}

The Guébie construction in (\ref{15}b) is identical to the \ili{Vata} construction in (\ref{16}). Likewise, the Guébie construction in (\ref{15}a) is shared by \ili{Bété}-\ili{Gbadi}, as shown in (\ref{17}). Without knowing which gerund formation strategy was present in the proto language, we cannot make any conclusions about whether each of these gerund constructions was inherited or borrowed. Without further information, gerunds do not shed light on the classification of Guébie.

\subsection{Summary}
We have seen that Guébie syntax is similar to \ili{Vata} in aspect marking, causation, and wh-question formation. It shares ones gerund construction with \ili{Vata} and another with \ili{Gbadi}. However, there are very few other syntactic similarities between Guébie and \ili{Bété}-\ili{Gbadi}. \ili{Vata} and Guébie share further syntactic properties not presented here for purposes of space. For example, they have identical \isi{applicative} constructions, similar \isi{negation} marking, and they both can undergo long-distance wh-\isi{movement} while \ili{Bété}-\ili{Gbadi} cannot.%%%Are there \isi{noun class} differences? what about \isi{word order}? question formation? embedding? \isi{focus} differences? etc.

The table in (\ref{table}) below summarizes the phonological and syntactic features compared here in Guébie, \ili{Vata}, and \ili{Bété}-\ili{Gbadi}.

\begin{exe}
\ex \textbf{Summary table}\label{table}\\
\begin{tabular}{llll}
\textbf{Feature} & \textbf{Bété} & \textbf{Guébie} & \textbf{Vata} \\
% \hline
\textbf{\# of tones} & 3 & 4 & 4 \\
\textbf{\# of vowels} & 7 & 10 & 10 \\
\textbf{Aspect} & particle & tone & tone \\
\textbf{Clausal causation} & -- & yes & yes \\
\textbf{\textsc{Wh}-question particle} & medial & final & final \\
\textbf{Gerunds} & verb-verb-je & verb-verb-je, verb-li & verb-li \\
\end{tabular}
\end{exe}

The \ili{Vata} and Guébie columns above are nearly identical. The two share the same number of contrastive tones and the same vowel inventory. They both mark aspect distinctions with tone, they have clausal \isi{causative} constructions, they have final Wh-particles, and they share a gerund construction. Guébie shares a gerund construction with \ili{Bété}-\ili{Gbadi} but in all other respects there are key differences between the two.

Based on the limited phonological and syntactic data available for Guébie, \ili{Vata}, and \ili{Bété}-\ili{Gbadi}, it seems that Guébie shares far more features with \ili{Vata}, a \ili{Dida} language, than with \ili{Bété}-\ili{Gbadi}, a \ili{Bété} language. In the following section I argue that we should classify Guébie as a distint language in the \ili{Dida} subgroup of Eastern \ili{Kru}.

\section{Implications and conclusions}\label{sec:sande:Implications}

We have seen evidence from the phonological and prosodic systems, and the morphosyntax of Guébie, that it resembles \ili{Vata}, a \ili{Dida} language, more closely than \ili{Bété}-\ili{Gbadi}, a \ili{Bété} langauge. Further research on \ili{Bété} and \ili{Dida} languages will allow for lexical and sound-correspondence comparison as well, which will confirm or deny the claims made here.

I return here to the problem defined in \sectref{sec:sande:Intro} of this paper: where within in Eastern \ili{Kru} should Guébie be classified. We saw that in Ethnologue \citep{LewisEtAl2013} Guébie is currently classified twice, once as a dialect of \ili{Bété}-Gagnoa, and once as an alternative name for \ili{Dida}-\ili{Lakota}. \ili{Dida} and \ili{Bété} are distinct subgroups of Eastern \ili{Kru} with distinct histories and distinct lingusitic features. The goal of this paper is to make an initial step towards determining whether Guébie is a \ili{Bété} or \ili{Dida} language. No firm conclusions can be made without further research; however, Guébie is culturally distinct from both of its neighboring \ili{Bété} and \ili{Dida} languages, and I argue here that it is also linguistically distinct. Guébie should be classified as a distinct language, and based on the data presented throughout this paper, I tentatively conclude that Guébie is a \ili{Dida} language, closely related to \ili{Dida}-\ili{Lakota} and \ili{Vata}, though not synonymous with either of them. This conclusion is based on the fact that there are many phonological and syntactic features shared between Guébie and \ili{Vata} that are not present in \ili{Bété}-\ili{Gbadi}. It seems unlikely that all of the similarities between Guébie and \ili{Vata} are due to chance or borrowing; thus, inheritence is a more plausible history.

\begin{figure}
% \begin{exe}
% \ex 
\caption{My proposed classification of Guébie within Kru}
\label{myclassification}
% \fittable{
% % \begin{tikzpicture}[scale=.7]
% \Tree[.{Eastern Kru} [.Bakwe-Wane ][.\ili{Bété} [.Eastern [.Gagnoa [.\ili{Gbadi} ]][.Kouya ]][.Western [.Daloa ][.Guiberoua ][.Godié ]]][.\ili{Dida} [.\ili{Lakota} [.\ili{Vata} ][.\textbf{Guébie} ]][.\ili{Dida}-Yocoboué ][.Neyo ]][.Kwadia ]]
% % \end{tikzpicture}
% }
\resizebox{\textwidth}{!}{%
\begin{forest} forked edges, for tree={fit=rectangle, s sep=2mm, inner xsep=0mm}
[Eastern \ili{Kru} [Bakwe-Wane] [\ili{Bété} [Eastern [Gagnoa [\ili{Gbadi}]] [Kouya] ] [Western [Daloa] [Guiberoua] [Godié] ] ] [\ili{Dida} [\ili{Lakota} [\ili{Vata}] [\textbf{Guébie}] ] [\ili{Dida}-Yocoboué] [Neyo]]  [Kwadia] ]
\end{forest}}
% \end{exe}
\end{figure}

The tree in \figref{myclassification} shows my proposed classification of Guébie, as a \ili{Dida}-\ili{Lakota} language related to but distinct from \ili{Vata}. Crucially, I claim that Guébie is \textit{not} a dialect of \ili{Bété}-Gagnoa, and in fact it is not a \ili{Bété} language at all.

Far more data is needed to prove the above classification and the further internal classification of \ili{Kru}; however, this paper provides and initial step towards a more detailed understanding of the \ili{Kru} languages and how they are related to each other. Arguing for the classification of Guébie at least provides a strong argument that can be argued for or against in future work when further data becomes available. 

Finer grained classification than that shown in the tree above will require extensive further research. Comparative work such as \cite{Kaye1982}, ``Les dialects dida," is a start toward this kind of comparative research in \ili{Kru}.

%%changes between dida and guebie are regular, less so between guebie and gbadi; we can attribute the similarities that do exist to borrowings or a much older relationship between the languages. Further research should be done to conclude which aspects that are shared by \ili{Guebie} and \ili{Gbadi} can be attributed to Proto-Eastern \ili{Kru} as opposed to borrowings

%  
% \bibliographystyle{linquiry2}
% \bibliography{mybib}{}




\newpage 
\printbibliography[heading=subbibliography,notkeyword=this]

\end{document}