\documentclass[output=paper,
modfonts
]{langscibook} 
% \bibliography{localbibliography}
\ChapterDOI{10.5281/zenodo.1251752}

 

\title{Temporal remoteness and vagueness in past time reference in Luganda} 

\author{ M. Ryan Bochnak\affiliation{University Konstanz}\lastand 
 Peter Klecha\affiliation{The Ohio State University}}

\abstract{In this paper, we point out that past time operators (PTOs) in Luganda, a language that makes three past time remoteness distinctions, are vague and context-dependent, and provide an analysis whereby PTOs contain context-sensitive measure functions akin to gradable adjectives. We call the relevant PTOs RECENT, INTERMEDIATE, and DISTANT, respectively.  Luganda PTOs give rise to borderline cases, where it is difficult to decide whether a past reference time (RT) counts as ‘recent’, ‘intermediate’ or ‘distant’. What counts as ‘recent’, ‘intermediate’ or ‘distant’ is context dependent; e.g., there are contexts where REC is acceptable with an RT of a few weeks ago, and contexts where DIST is acceptable for an RT of a few minutes ago. We assume that like tenses in English, PTOs in matrix clauses in Luganda restrict the relation between utterance time (UT) and RT. However, while English past tense presupposes that RT precedes UT (e.g. \citealt{kratzer98more}), Luganda PTOs additionally encode as part of their meaning a vague, context-dependent measure function that compares the length of a time interval to a contextual standard.​
}

\begin{document}
\maketitle

\section{Introduction}

So-called `graded tense' systems are cross-linguistically quite common \citep{comrie85tense, dahl83temporal, mithun99languages, nurse08tense}. Such systems make use of multiple morphemes that make more fine-grained distinctions than simply `past' or `future'. Rather, they encode varying degrees of remoteness (e.g., recent vs.~distant) from a reference point. Languages with such systems vary with respect to what distinctions are made, and how many. 

However, grammaticalized \isi{temporal remoteness} morphemes have only recently begun to attract attention within formal semantics (see e.g.,~\citealt{cable12beyond} for G\~\i k\~uy\~u, \citealt{hayashi11structure} for South Baffin \ili{Inuktitut}; \citealt{mucha14past} for \ili{Medumba}). We aim to expand upon this discussion here by investigating the paradigm of graded past tenses in \ili{Luganda} (Bantu; JE15). \ili{Luganda} has three remoteness categories for past time reference. We refer to these as {\sc recent} (\ref{tense1}), {\sc intermediate} (\ref{tense2}), and {\sc distant} (\ref{tense3}). They are bound morphemes that are obligatory on verbs with past time reference, i.e., a speaker must choose one of these three forms for finite clauses with past time reference.




\ea
\label{tense1}
\gll \textbf{Nzinye} (ku matya). \\
{\sc1sg}.dance{\sc.\textbf{rec.pst}} ({\sc loc} morning) \\
\glt `I danced (this morning).' \hfill {\sc recent past} \\
\fbox{ Template ({\sc\textbf{rec.pst}}): {\sc agr}-$\emptyset$-{\sc root}-{\bf ie} }
\z

 
\ea\label{tense2}
\gll \textbf{Nazinye} (jjo).\\
{\sc 1sg}.dance{\sc.\textbf{int.pst}} (yesterday)\\
\glt `I danced (yesterday).' \hfill {\sc intermediate past}\\
\fbox{ Template ({\sc\textbf{int.pst}}): {\sc agr}-{\bf a}-{\sc root}-{\bf ie} }
\z

\ea
\label{tense3}
\gll \textbf{Nazina} (luli). \\
{\sc1sg}.dance{\sc.\textbf{dist.pst}} (another.time) \\
\glt `I danced (the other day).'  \hfill {\sc distant past} \\
\fbox{ Template ({\sc\textbf{dist.pst}}): {\sc agr}-{\bf a}-{\sc root}-{\bf a} } \hfill (modulo irregular verbs)
\z




The boundaries of the temporal delineations are often described in the literature as being quite precise, e.g., up to 4 hours ago; one day ago. In this connection, it is quite common for authors to use labels such as `hesternal' or `hodiernal'. Despite such characterizations, it is also often reported that the use of these morphemes can be rather flexible and context-sensitive  (e.g., \citealt{hyman80relative} on Bamileke; \citealt{sharman56tabulation} on ChiBemba). Our primary objective in this paper is to show that the \isi{temporal remoteness} morphemes\footnote{Following \citet{cable12beyond}, we use the term `\isi{temporal remoteness} morphemes' so as not to prejudge their analysis as true tenses or something else, e.g., modifiers of events as Cable argues for G\~\i k\~uy\~u. We do, however, analyze them as tenses in \sectref{sec:bochnakklecha:5}.} (TRMs) in \ili{Luganda} display the hallmarks of \isi{vagueness} found elsewhere in unrelated phenomena across languages (e.g., certain nouns and gradable adjectives); in other words, in certain contexts, even native speakers with full knowledge of the facts are unable to determine which TRM should be used.  We provide an analysis of TRMs in \ili{Luganda} that takes into account the properties of \isi{vagueness} we observe. In doing so, we also compare the behavior of temporal operators cross-linguistically with that of relative and absolute gradable adjectives.


The paper proceeds as follows. In \sectref{sec:bochnakklecha:3} we describe in more detail the use of TRMs in \ili{Luganda}. In \sectref{sec:bochnakklecha:2} we outline the behavior of vague expressions in general, and in \sectref{sec:bochnakklecha:4} we show that \ili{Luganda} TRMs indeed display the properties of vague expressions. Our analysis is presented in \sectref{sec:bochnakklecha:5}, and in \sectref{sec:bochnakklecha:6} we draw a comparison between types of temporal operators on one hand and gradable adjectives on the other. \sectref{sec:bochnakklecha:7} concludes and hints at areas for further research.





\section{Temporal remoteness morphemes in Luganda}\label{sec:bochnakklecha:3}


Recall that \ili{Luganda} has three remoteness categories in the past tense paradigm, which we label as {\sc recent}, {\sc intermediate}, and {\sc distant} past in (\ref{tense1})-(\ref{tense3}).\footnote{Our data come from elicitation sessions with two native speakers of \ili{Luganda}, bilingual in \ili{English}, living in the United States. All elicitation sessions were conducted in \ili{English}, and involved a mix of translation tasks and felicity judgements of grammatical sentences in particular contexts. 
%We have worked with our \ili{Luganda} consultant since 2008, but only on this topic since 2012.
} \citet[122--123]{ashton54luganda} characterize these morphemes in the following way: the recent past ``expresses the completion of an action and/or state entered upon within the immediate past"; the intermediate past ``expresses an action actually finished and accomplished as an action, but confined or limited $\dots$, roughly speaking, to the past twelve hours"; and the distant past ``denotes an action in the past, but is indefinite as to the exact time. $\dots$ It corresponds to a Past Aorist." We spend time in this section going into more detail about the semantics and use of these morphemes.\footnote{Note that we restrict ourselves here to the three ``simple'' past time markers in matrix clauses. We also do not talk about so-called ``special'' and ``\isi{compound}" tenses of \ili{Luganda} (see  \citealt{kisubika86tense}). Grammars such as \citet{ashton54luganda} have also cited future TRMs which similarly divide future time into nearer and more distant time zones. However, our consultants have indicated to us that these forms are not colloquial, and the commonly used future forms do not carry \isi{temporal remoteness} inferences, so we do not discuss these here.}



First, we argue that these morphemes are not aspectual, since they can freely combine with predicates of all Aktionsarten without coercing any aspectual interpretation. Witness (\ref{state})-(\ref{ach}).



\ea\label{state} Statives:
  \ea\label{state-rec}
  \gll Babilie abadde lubuto.\\
  B. {\sc3sg}.be.{\sc rec.pst} {\sc nc11}.pregnant.woman\\
  \glt `Babilie was just pregnant.'
  \ex\label{state-int}
  \gll Babilie yabadde lubuto.\\
  B. {\sc3sg}.be.{\sc int.pst} {\sc nc11}.pregnant.woman\\
  \glt `Babilie was pregnant (a little while ago).'
  \ex\label{state-dis}
  \gll Babilie yali lubuto.\\
  B. {\sc3sg}.be.{\sc dis.pst} {\sc nc11}.pregnant.woman\\
  \glt `Babilie was pregnant (a long time ago).'
  \z
\z

\ea\label{act} Activities:
  \ea\label{act-rec}
  \gll Kato azinye.\\
  K. {\sc3sg}.dance.{\sc rec.pst}\\
  \glt `Kato just danced.'
  \ex\label{act-int}
  \gll Kato yazinye.\\
  K. {\sc3sg}.dance.{\sc int.pst}\\
  \glt `Kato danced (a little while ago).'
  \ex\label{act-dis}
  \gll Kato yazina.\\
  K. {\sc3sg}.dance.{\sc dis.pst}\\
  \glt `Kato danced (a long time ago).'
  \z
\z

\ea\label{acc} Accomplishments:
  \ea\label{acc-rec}
  \gll Kato addusse mairo biri.\\
  K. {\sc3sg}.run.{\sc rec.pst} mile three\\
  \glt `Kato just ran three miles.'
  \ex\label{acc-int}
  \gll Kato yaddusse mairo biri.\\
  K. {\sc3sg}.run.{\sc int.pst} mile three\\
  \glt `Kato ran three miles (a little while ago).'
  \ex\label{acc-dis}
  \gll Kato yadduka mairo biri.\\
  K. {\sc3sg}.run.{\sc dis.pst} mile three\\
  \glt `Kato ran three miles (a long time ago).'
  \z
\z


\ea\label{ach} Achievements:
  \ea\label{ach-rec}
  \gll Ensuwa eyatisse.\\
  {\sc aug}.{\sc nc9}.pot {\sc nc9}.break.{\sc rec.pst}\\
  \glt `The pot just broke.'
  \ex\label{ach-int}
  \gll Ensuwa yeyatisse.\\
  {\sc aug}.{\sc nc9}.pot {\sc nc9}.break.{\sc int.pst}\\
  \glt `The pot broke (a little while ago).'
  \ex\label{ach-dis}
  \gll Ensuwa yeyatika.\\
  {\sc aug}.{\sc nc9}.pot {\sc nc9}.break.{\sc dist.pst}\\
  \glt `The pot broke (a long time ago).'
  \z
\z




Second, certain TRMs in \ili{Luganda} asymmetrically entail others. In particular, the intermediate past appears to be usable in any context where the recent past is. For instance in (\ref{ex10})-(\ref{ex11}), the intermediate past is possible both with {\it tano emabega} `five hours ago' and {\it satu emabega} `three hours ago', while the recent past is possible only with the latter.

\ea\label{ex10}
\gll Nzinye saawa \{satu/\#tano\} emabega. \\
{\sc1sg}.dance{\sc.\textbf{rec.pst}} hour \{three/five\} behind \\
\glt `I danced \{three/five\} hours ago.' \hfill (\textsc{recent past})
\z

\ea\label{ex11}
\gll Nazinye saawa  \{satu/tano\} emabega. \\
{\sc 1sg}.dance{\sc.\textbf{int.pst}} hour \{three/five\} behind \\
\glt `I danced \{three/five\} hours ago.' \hfill (\textsc{intermediate past})
\z


 The distant past cannot as easily be used in contexts where the intermediate or recent past can, but there is some evidence that it is the most general past, and that the inference of `distance' is due to scalar implicature.
 For instance, a question with a the recent or intermediate past gives rise to an inference that the speaker has at least some knowledge about the temporal location of the event at issue -- but a question with the distant past gives rise to no such inference.
 The contexts below are adapted from \citet{cable12beyond}, who finds the same thing for G\~\i k\~uy\~u.
 

\ea
Context: You have known Kato for a long time, but have never been to his house. You are finally invited over and you see that he often buys very old things for fun. He tells you he bought an old Apple computer from 1985 just a few hours before you arrived. You see his TV, which is also quite old, but you can't tell if he just recently bought it, or if he's actually had it since the 80's. 
  \ea\label{ignorance.q}
  \gll  Eno TV wagigguladi? \\
  this.\sc{nc9} TV {\sc 2sg.subj-nc9.obj}-buy{\sc-dis.pst}-when \\
  \glt `When did you buy this TV?' 
  \ex\label{ignorance.a}
  \gll  Naagigguzze jjo. \\
  {\sc 1sg.subj.nc9.obj}.buy{\sc.int.pst} yesterday \\
  \glt `I just bought it yesterday.'
  \z
\z


\ea
Same context as above.
  \ea\label{ignorance.q2}
  \gll  Eno TV wagigguzzedi? \\
  this.\sc{nc9} TV {\sc 2sg.subj.nc9.obj}.buy{\sc.int.pst}.when \\
  \glt `When did you buy this TV?' (Suggests it happened recently)
  \ex\label{ignorance.a2}
  \gll  Naagiggula luli. \\
  {\sc 1sg.subj.nc9.obj}.buy{\sc.dis.pst} another.time \\
  \glt `I bought it a while ago.' (Sounds contradictory)
  \ex Speaker notes that one might precede the answer in \ref{ignorance.a2} by saying ``What makes you think I bought it recently?''.
  \z
\z

Crucially, the speaker notes that (\ref{ignorance.q2}) can be challenged, by saying something like ``What makes you so sure I just bought it?'', whereas (\ref{ignorance.q}) \textit{cannot} be challenged similarly, i.e., with something like ``What makes you so sure I bought it a long time ago?''.
Thus we conclude that the inference of temporal distance that is produced by what we call the distant past is an implicature.
But note that this implicature is vague and context-sensitive in exactly the same way that the semantic inferences of the recent and intermediate past are vague and context-sensitive.
Thus for the following discussion we will not distinguish between implicatures and semantic inferences when discussing \isi{vagueness} and context-sensitivity. 






\section{Vagueness in natural language}\label{sec:bochnakklecha:2}

Certain expressions in natural language display {\it vagueness}, for instance certain `scalar' nouns like {\it heap}, and relative gradable adjectives like {\it tall}. Vague expressions are those where the criteria of application are not clear-cut and can shift in different contexts of use. Here we outline three properties of vague expressions and sentences that contain them  \citep{fine75vagueness, kamp75two, fara00shifting, kennedy07vagueness}.


First, {\sc context dependence}: what counts as a heap or tall varies from context to context. Consider the sentence in (\ref{context1}).


\ea\label{context1}
Tom is tall.
\z


\noindent In a context where Tom has a height of 165 cm, and is in the second grade, (\ref{context1}) is intuitively true. Meanwhile, in a context where Tom has a height of 165 cm and is a professional basketball player, (\ref{context1}) is intuitively false. Crucial here is the {\sc comparison class}, namely the set of objects against which Tom's height is compared in order to judge whether (\ref{context1}) counts as true or not.



Second is the existence of {\sc borderline cases}. That is, there are certain cases where it's hard to judge whether a vague \isi{predicate} holds or not. For instance, consider a context where Tom is a 12-year-old boy. Is (\ref{context1}) true or false? This case seems harder to judge one way or the other, compared with the previous contexts we considered where the judgements were more clear-cut. 



The third property of vague expressions is that they give rise to the \isi{sorites paradox}. Consider (\ref{sorites1}).



\ea\label{sorites1}
P1: 1,000,000 grains of sand is a heap of sand.\\
P2: A heap of sand less one grain is still a heap of sand.\\
C: \#One grain of sand counts as a heap of sand.
\z


\noindent If we accept P1 \& P2, then with enough iterations of P2 we should also accept C. Nevertheless, the intuition is that C does not hold, resulting in a paradox.



There are various proposals on the market for how \isi{vagueness} comes about, and we do not wish to take a stand on this issue here.\footnote{See \citealt{fara00shifting}, \citealt{grinsell12social}, \citealt{soames99understanding} and \citealt{williamson94vagueness} for various proposals.} However, we would like to highlight the following passage from \citet[p.~42]{kennedy07vagueness}: ``[V]agueness comes from epistemic uncertainty about where we actually draw the line and metalinguistic resistance to treating highly similar objects differently relative to the property expressed \dots Whether this analysis extends to an account of \isi{vagueness} in other categories is an issue that must be addressed in future work." It is this last point that we aim to address in this paper, namely whether these properties of \isi{vagueness} manifest themselves in temporal operators in natural language.







\section{Vagueness in the Luganda past paradigm}\label{sec:bochnakklecha:4}


In this section we show that \ili{Luganda} TRMs display the same properties of \isi{vagueness} that we find in vague nouns and adjectives. First, a methodological point: we presented discourse contexts to the consultant in \ili{English}, which does not have TRMs, because we wanted to avoid prejudicing the response given by the consultants by having a TRM in the prompt.\footnote{See \citet{anderbois15linguistically} for comments on choosing elicitation language in semantic fieldwork.}






{\bf Context-dependence:} for many events, the restrictions on when certain TRMs can be used more or less follows the pattern described in \citet{ashton54luganda}. However, what counts as recent or distant can vary across contexts. For instance, the recent past can be used to talk about a time months prior to the utterance time (UT), as in (\ref{planting}). Likewise, the distant past can be used to talk about a time only a few minutes prior to UT, as in (\ref{party}).\footnote{One of our consultants finds (\ref{party}) not completely natural, but does accept it.} Thus, the notions of what counts as recent or distant can vary with the context, much like what counts as tall varies across contexts.


\ea\label{planting}
\ea
 Context: You plant your crops every year in February. It is now April, and I ask you what you planted this year. You tell me that you planted maize.
\ex \gll Nsimbye kasooli. \\
{\sc1sg}.plant{\sc.\textbf{rec.pst}} maize \\
\glt `I planted maize.'
\z
\z

\ea\label{party}
\ea Context: We are at a party, and I ask you why you are not dancing to the song that's playing. You tell me that you danced a few songs ago.
\ex \gll Nazina luli. \\
{\sc 1sg}.dance{\sc.\textbf{dist.pst}} another.time \\
\glt `I danced a while ago (to another song).'
\z
\z

In other words, the use of a particular TRM seems to depend on a relevant comparison class, just like for predicates like {\it tall}. Here is a first stab at characterizing the relevant comparison class for TRMs: in (\ref{planting}) the comparison class is time intervals between UT and previous crop plantings, while in (\ref{party}) the comparison class is time intervals between UT and previous songs.




{\bf Borderline cases:} there are also cases where it is difficult to decide whether an event counts as `recent', `intermediate' or `distant'. There is an interesting contrast here between TRMs in \ili{Luganda} and gradable adjectives. Namely, for gradable adjectives, a speaker can refuse to make a judgment (e.g.~``I'm not sure whether Bill would count as tall or not"). However, verbs in \ili{Luganda} naming past events are {\it obligatorily} marked with TRMs - a speaker always has to choose one. Consider the following responses to the question ``What have you been doing?"



\ea\label{borderline}
\gll Nzinye saawa \{satu/?nnya/\#tano\} emabega. \\
{\sc1sg}.dance{\sc.\textbf{rec.pst}} hour \{three/four/five\} behind \\
\glt `I danced \{three/four/five\} hours ago.' \hfill (\textsc{recent past})
\z

\ea\label{borderline.int} \textit{Nazinye saawa  \{satu/nnya/tano\} emabega.} \\
`I danced \{three/four/five\} hours ago.' \hfill (\textsc{intermediate past})
\z

\ea\label{borderline.dist} \#\textit{Nazina saawa  \{satu/nnya/tano\} emabega.} \\
`I danced \{three/four/five\} hours ago.' \hfill (\textsc{distant past})
\z


According to our consultants, in the context in (\ref{borderline}), the use of the recent past is definitely fine for `3 hours ago', definitely excluded for `5 hours ago', but borderline for `4 hours ago'.\footnote{Given the variable nature of context dependency and borderline cases, we imagine that judgments could also vary somewhat between speakers.}





{\bf Sorites sequences:} \ili{Luganda} TRMs also give rise to \isi{sorites paradox} effects, just like scalar nouns and gradable adjectives. Consider the party scenario in (\ref{soritesparty}), and the subsequent sentences offered and rejected by our consultant to describe the scenario in (\ref{soritesparty1})-(\ref{soritespartyfinal}).\footnote{We have performed this activity with only one of our consultants so far (Kisuule Magala Katende).}



\ea\label{soritesparty}
Context: You are at a party, and they play the following songs (in order): \\
Twist and Shout \\
YMCA \\
Dancing Queen \\
Sweet Caroline \\
Gangnam Style \\
Blue Suede Shoes \\
Don't Stop Believing \\
Macarena \\
Rollin' in the Deep \\
Zamboni Driver
\z

\ea\label{soritesparty1}
\ea Context as in (\ref{soritesparty}). Peet comes to the party late, just as Zamboni Driver is finishing. He is always really interested in the music they play, so you want to tell him what songs were played.
\ex\label{party1}
 \gll Baakubye Twist and Shout. \\
{\sc 3pl}.play{\sc.int.pst} Twist and Shout \\
\glt `They played Twist and Shout.' 
\ex Observation: Speaker uses {\sc int.pst} discourse-initially in this scenario.
\z
\z

\ea
\ea Context: same as above
\ex[\#]{\gll Baakuba Twist and Shout. \\
{\sc 3pl}.play{\sc.dist.pst} Twist and Shout \\
\glt `They played Twist and Shout.'}
\ex Speaker's comment: ``Sounds like it's a different party, another day.''
\z
\z

\ea
\ea Context: same as above
\ex[\#]{\gll Bakubye Twist and Shout. \\
{\sc 3pl}.play{\sc.rec.pst} Twist and Shout \\
\glt `They played Twist and Shout.' }
\ex Speaker's comment: ``That's like they just played it. That's like a couple of minutes ago. $\dots$ Just missed it.''
\z
\z

 
\ea\label{soritespartyfinal}
\ea Context: Immediately after uttering (\ref{party1})
\ex\label{party2}
 \gll Baaziza-ko YMCA. \\
{\sc 3pl}.add{\sc.int.pst-part} YMCA \\
\glt `Then they played YMCA.' \\
(`They added on YMCA.') \hfill (offered by speaker) 
\ex Observation: Speaker uses {\sc int.pst} again for event immediately following the first
\z
\z


\ea
\ea
 Context: Immediately after uttering (\ref{party1}) and (\ref{party2})
\ex {\it Baazizako Dancing Queen.} \\
`Then they played Dancing Queen.' \hfill {\sc int.pst} \\
Judgment: Accepted by speaker
\ex {\it Baazizako Sweet Caroline.} \\
`Then they played Sweet Caroline. \hfill {\sc int.pst} \\
Judgment: Accepted by speaker \\
$\vdots$

\ex
 {\it Baazizako Zamboni Driver.} \\
`Then they played Zamboni Driver.' \hfill {\sc int.pst} \\
Judgment: Accepted by speaker
\ex Speaker's comment: ``If you just played it, you would say {\it bazaako} [{\sc rec.pst}]. $\dots$ When you are coming towards the end, it has to be {\it bazaako}. $\dots$ Because it's more recent than the other ones.'' 
\ex Peet: ``Can you be really sure about where you would switch?" \\
Speaker: ``No, the time difference is in your mind. $\dots$ There is no obvious timeline to stop it.''
\z
\z



A similar effect is observed in the following OCD friend scenario in (\ref{sorites2a})-(\ref{sorites3}).


\ea\label{sorites2a}
\ea\label{sorites2}
Context: Your (crazy) friend wants to know what songs were played on the radio in the last few days in order. Being an indulgent friend, you tell him what songs were played.
\ex\label{sorites2b}
 \gll Baakuba Twist and Shout. \\
{\sc 3.sg}.play{\sc.dist.pst} Twist and Shout \\
\glt `They played Twist and Shout.'
\ex Speaker's comment: ``That's another day. That's not today.''
\z
\z

\ea
\ea Context: Immediately after uttering (\ref{sorites2b})
\ex  \gll Baakuba YMCA. \\
{\sc 3.sg}.play{\sc.dist.pst} YMCA \\
\glt `They played YMCA.'
\z
\z

\ea\label{sorites3}
\ea Context: Immediately after uttering (\ref{sorites2b})
\ex[\#]{\gll Baakubye YMCA. \\
{\sc 3.sg}.play{\sc.int.pst} \\
\glt `They played YMCA.'}
\ex Speaker's comment: ``No. It would be very awkward. $\dots$ You never know when you switch to {\it baakubye} [{\sc int.pst}]. But that should be within a day or something like that. $\dots$ There's no clear place where you would stop to apply it [the distant past form].''
\z
\z

Thus, we seem to have evidence that sorites sequences behave like expected for sentences containing vague expressions.  Small differences in time don't prompt speakers to suddenly switch TRMs, and speakers are not sure when exactly it is appropriate to switch TRMs in these listing contexts.\footnote{It has been suggested to us that these effects might be obviated if the list is given out of order. We have not been able to test this idea yet.} 

In sum, sentences containing TRMs in \ili{Luganda} behave like sentences containing other vague expressions (in \ili{English}).



% ======== 2

%\section{Vagueness in the Luganda past paradigm}
%
%This section shows that \ili{Luganda} TRMs are vague.
%They present the behaviors discussed in Section 2, which are characteristic of vague predicates; context dependence, borderline cases, and a propensity to participate in Sorites sequences.
%We discuss each of these characteristic behaviors in turn.
%
%Before we present our data, first a methodological point:
%We elicited this data from a \ili{Luganda} consultant by presenting the discourse contexts in \ili{English}.
%This choice was made because \ili{English} does not have TRMs, and we wished to avoid prejudicing the response given by the consultant by having a TRM in the prompt.
%See \citet{anderbois12linguistically} for discussion of elicitation language choice in semantic fieldwork. 
%
%\subsection{Context-dependence}
%
%For many event-varieties, the restrictions on when certain TRMs can be used more or less follows the pattern described in the \citet{ashton54luganda}.
%However, what counts as `recent' or `distant' can vary across contexts, with event-variety a \isi{major} determining factor.
%The recent past can be used to talk about an event that happened months ago.
%Consider (\ref{planting}), in which case that event is a crop-planting event, in the context of a discussion of different crop-plantings.
%
%
%
%
%%\begin{exe}
%%\ex\label{planting}
%%\begin{xlist}
%%\ex Context: You plant your crops every year in February. It is now April, and I ask you what you planted this year. You tell me that you planted maize.
%%\ex
%\gll Nsimbye kasooli. \\
%{\sc1sg}-plant{\sc-\textbf{rec.pst}} maize \\
%\glt `I planted maize.'
%%\end{xlist}
%%\end{exe}
%
%Likewise, the distant past can be used to talk about an event that happened a few minutes ago.
%In (\ref{party}), the event is a dancing-event, in the context of a discussion of the happenings at a single party.
%
%%\begin{exe}
%%\ex\label{party}
%%\begin{xlist}
%%\ex Context: We are at a party, and I ask you why you are not dancing to the song that's playing. You tell me that you danced a few songs ago.
%%\ex
%\gll Nazina luli. \\
%{\sc 1sg}-dance{\sc-\textbf{dist.pst}} another.time \\
%\glt `I danced a while ago (to another song).'
%%\end{xlist}
%%\end{exe}
%
%Just like for predicates like {\it tall}, the use of a particular TRM depends on a relevant comparison class.
%In the case of (\ref{planting}), the comparison class is determined by the set of other planting events under discussion.
%Accordingly, in the case of o(\ref{party}), the comparison class is determined by other dancing events (and possibly other events subsumed by the party).
%
%Given, however, that the semantics of the TRMs characterize not the events themselves, but the temporal intervals between those events and the evaluation time, we propose that comparison class actually consists of the time intervals between evaluation time and like events.
%So, as provided by the context of (\ref{planting}), planting events occur only in February, so the set of events under discussion include the maize-planting from two months ago, the bean-planting from fourteen months ago, the maize-planting from twenty-six months ago, etc.
%The comparison class determined by that set, therefore, is the two-month interval between the last maize-planting and now; the fourteen-month interval between the last bean planting and now; etc.
%In this context, a two month interval is indeed a short one. 
%
%Likewise, as determined by the context of (\ref{party}), relevant events include the song currently playing as well as several other more recent ones.
%The comparison class determined by this set of events determines a set of temporal intervals, including the null interval, a two-minute interval, a five-minute interval, etc.
%In this context, a perhaps ten-minute interval, the temporal distance between the song the speaker last danced to and the evaluation time, counts as a long one.
%
%This behavior is inconsistent with the rigid diurnal system described by \citet{ashton54luganda}, but consistent with the behavior of vague predicates like {\it tall}.
%
%\subsection{Borderline cases}
%
%Recall that a borderline case is a case where it is difficult to decide if an object falls in the negative or positive extension of a \isi{predicate}.
%For gradable adjectives, signs of a borderline case include a refusal by the speaker to make a truth-value or acceptability judgment (e.g.~``I'm not sure whether Bill would count as tall or not.").
%
%As expected if TRMs are vague, there are also cases where it is difficult to decide whether an event counts as `recent', `intermediate' or `distant'.
%What intriguingly distinguishes \ili{Luganda} TRMs from, say, vague adjectives in \ili{English}, is that verbs in \ili{Luganda} are {\it obligatorily} marked with TRMs .
%Thus, despite the presence of \isi{vagueness}, a speaker, morphosyntactically speaking, always has to choose one.
%
% Consider the following responses to the question ``What have you been doing?"
%
%
%
%
%%\begin{exe}
%%\ex\label{borderline}
%\gll Nzinye saawa \{satu/?nnya/\#tano\} emabega. \\
%{\sc1sg}-dance{\sc-\textbf{rec.pst}} hour \{three/four/five\} behind \\
%\glt `I danced \{three/four/five\} hours ago.' \hfill (\textsc{recent past})
%%\ex\label{borderline.int} \textit{Nazinye saawa  \{satu/nnya/tano\} emabega.} \\
%`I danced \{three/four/five\} hours ago.' \hfill (\textsc{intermediate past})
%%\ex\label{borderline.dist} \#\textit{Nazina saawa  \{satu/nnya/tano\} emabega.} \\
%`I danced \{three/four/five\} hours ago.' \hfill (\textsc{distant past})
%%\end{exe}
%
%
%The judgment of interest is the one in (\ref{borderline}) marked with the $?$ stigmata.
%This indicates a refusal by the speaker to provide an acceptability judgment.
%While the speaker affirmed the use of the recent past with a dancing event of three hours ago\footnote{This was in a different context than the one provided for (\ref{party}); thus the discrepancy with that example.}, and clearly judged as infelicitous the use of the recent past with a dancing event of five hours ago, he explicitly refused to make such a judgment for a dancing event of four hours ago.
%
%\subsection{Sorites sequences}
%
%Recall that reconstructing the Sorites Paradox is possible only with vague predicates.
%In order to show that \ili{Luganda} TRMs participate in sequences that give rise to a Sorites Paradox, we constructed the following conext and presented it to our speaker.
%
%%\begin{exe}
%%\ex\label{soritesparty}
%Context: You are at a party, and they play the following songs (in order): \\
%Twist and Shout \\
%YMCA \\
%Dancing Queen \\
%Sweet Caroline \\
%Gangnam Style \\
%Blue Suede Shoes \\
%Don't Stop Believing \\
%Macarena \\
%Rollin' in the Deep \\
%Zamboni Driver
%%\end{exe}
%
%Having established this context, we presented the speaker with the following further elaboration, in order to elicit an exhaustive listing of the songs that were played.
%The speaker begins with the intermediate past\footnote{Suggested uses of the distant or recent past was ruled out by the speaker, as seen in (\ref{partya}) and (\ref{partyb}) respectively.
%
%
%%\begin{exe}
%\ex%\begin{xlist}
%%\ex Context: same as above
%\ex[\#] {\gll Baakuba Twist and Shout. \\
%{\sc 3pl}-play{\sc-dist.pst} Twist and Shout \\
%\glt `They played Twist and Shout.'  }\label{partya}
%%\ex Speaker's comment: ``Sounds like it's a different party, another day.''
%%\end{xlist}
%%\end{exe}
%
%%\begin{exe}
%\ex%\begin{xlist}
%%\ex Context: same as above
%\ex[\#]{\gll Bakubye Twist and Shout. \\
%{\sc 3pl}-play{\sc-rec.pst} Twist and Shout \\
%\glt `They played Twist and Shout.'  }\label{partyb} 
%%\ex Speaker's comment: ``That's like they just played it. That's like a couple of minutes ago. $\dots$ Just missed it.''
%%\end{xlist}
%%\end{exe}
%
%}and continues to use that TRM for every utterance after that, including the description of the most recent song, seen in (\ref{partylast}). 
%However, immediately upon uttering (\ref{partylast}), the speaker qualifies his usage and expresses an uncertain judgment about the acceptability of the utterance.
%
%
%
%%\begin{exe}
%%\ex %\begin{xlist}
%%\ex Context as in (\ref{soritesparty}). Peet comes to the party late, just as Zamboni Driver is finishing. He is always really interested in the music they play, so you want to tell him what songs were played.
%%\ex\label{party1}
% \gll Baakubye Twist and Shout. \\
%{\sc 3pl}-play{\sc-int.pst} Twist and Shout \\
%\glt `They played Twist and Shout.' 
%%\end{xlist}
%%\end{exe}
%
%
%%\begin{exe}
%\ex%\begin{xlist}
%%\ex Context: Immediately after uttering (\ref{party1})
%%\ex\label{party2}
% \gll Baaziza-ko YMCA. \\
%{\sc 3.pl}-add{\sc-int.pst-part} YMCA \\
%\glt `Then they played YMCA.' \\
%(`They added on YMCA.') \hfill (offered by speaker) 
%%\end{xlist}
%%\end{exe}
%
%%\begin{exe}
%\ex%\begin{xlist}
%%\ex Context: Immediately after uttering (\ref{party1}) and (\ref{party2})
%%\ex {\it Baazizako Dancing Queen.} \\
%`Then they played Dancing Queen.' \hfill {\sc int.pst} \\
%Judgment: Accepted by speaker
%%\ex {\it Baazizako Sweet Caroline.} \\
%`Then they played Sweet Caroline. \hfill {\sc int.pst} \\
%Judgment: Accepted by speaker \\
%$\vdots$
%
%%\ex \label{partylast}{\it Baazizako Zamboni Driver.} \\
%`Then they played Zamboni Driver.' \hfill {\sc int.pst} \\
%Judgment: Accepted by speaker
%%\ex Speaker's comment: ``If you just played it, you would say {\it bazaako} [{\sc rec.pst}]. $\dots$ When you are coming towards the end, it has to be {\it bazaako}. $\dots$ Because it's more recent than the other ones.'' 
%%\ex Investigators: ``Can you be really sure about where you would switch?" \\
%Speaker: ``No, the time difference is in your mind. $\dots$ There is no obvious timeline to stop it.''
%%\end{xlist}
%%\end{exe}
%
%The equivocation by the speaker as to the acceptability of the intermediate past in (\ref{partylast}) is consistent with the Sorites Paradox, as is the following expression of uncertainty; the speaker is never comfortable switching from one TRM to another on the basis of a small shift in time. 
%
%A similar scenario provided the same result.
%
%%\begin{exe}
%\ex%\begin{xlist}
%%\ex\label{sorites2}
%Context: Your (crazy) friend wants to know what songs were played on the radio in the last few days in order. Being an indulgent friend, you tell him what songs were played.
%%\ex\label{sorites2b}
% \gll Baakuba Twist and Shout. \\
%{\sc 3.sg}-play{\sc-dist.pst} Twist and Shout \\
%\glt `They played Twist and Shout.'
%%\ex Speaker's comment: ``That's another day. That's not today.''
%%\end{xlist}
%%\end{exe}
%
%%\begin{exe}
%%\ex %\begin{xlist}
%%\ex Context: Immediately after uttering (\ref{sorites2b})
%%\ex  \gll Baakuba YMCA. \\
%{\sc 3.sg}-play{\sc-dist.pst} YMCA \\
%\glt `They played YMCA.'
%%\end{xlist}
%%\end{exe}
%
%%\begin{exe}
%\ex%\begin{xlist}
%%\ex Context: Immediately after uttering (\ref{sorites2b})
%\ex[\#]{\gll Baakubye YMCA. \\
%{\sc 3.sg}-play{\sc-int.pst} \\
%\glt `They played YMCA.'}
%%\ex Speaker's comment: ``No. It would be very awkward. $\dots$ You never know when you switch to {\it baakubye} [{\sc int.pst}]. But that should be within a day or something like that. $\dots$ There's no clear place where you would stop to apply it [the distant past form].''
%%\end{xlist}
%%\end{exe}
%
%
%
%Based on this evidence we conclude that TRMs in \ili{Luganda} are vague.

\section{Analysis}\label{sec:bochnakklecha:5}

Following \citet{kennedy97projecting, kennedy07vagueness}, we take simple relative gradable adjectives like \textit{tall}, which in their positive (i.e., bare) forms are vague, to denote \textbf{measure functions}, i.e., functions from an object to a \textit{degree}, an abstract unit of measurement.
The denotation of \textit{tall} on this basis is given below.

\ea
$\llbracket$ {\it tall} $\rrbracket = \lambda x.${\bf height}$(x)$  \hfill type $\langle e, d\rangle$
\z

The above says that \textit{tall} maps an individual to her height.
Any simple adjective phrase, however, should ultimately denote something of type $\langle e, t\rangle$; i.e., should map an individual to true or false, depending on her height.
So some assumption must be made about how the type of a bare adjective shifts when in its positive form.
Kennedy's proposal is an abstract morpheme \textit{\textsc{pos}} which has the following interpretation.

\ea
$\llbracket$ {\it \textsc{pos}} $\rrbracket = \lambda G_{\langle e, d\rangle}.\lambda x. G(x) \succeq \textbf{s}(G)  \hfill $type $\langle\langle e, d\rangle,\langle e,t \rangle\rangle$
\z

Following \citet{kennedy07vagueness}, the function \textbf{s} is a contextually provided one which takes a measure function and returns the degree that that function would have to map an individual to in order for that individual to `stand out', in terms of that property, in the context.
If \textit{tall} combines with \textsc{pos} rather than a degree modifier, the result is a vague \isi{predicate}.

\ea
\label{tall} $\llbracket$ {\it \textsc{pos} tall} $\rrbracket = \lambda x. \textbf{height}(x) \succ s  \hfill $type $\langle e,t \rangle$
\z

Thus a sentence like \textit{John is tall}, which includes the positive form of an adjective, is true iff John's height stands out in the context, i.e., if it exceeds the determined standard.

On Kennedy's account, borderline cases arise because speakers are unsure about where the cut-off is for `standing out'.
Cases like the Sorites Paradox arise because given a context that contains any two objects that have gradable property $G$ to very slightly varying degrees, neither one can `stand out' with respect to the other. 

Note that the reason that \textit{tall} is not simply given the denotation in (\ref{tall}) above is that it may instead combine with degree modifiers or measure phrases, which do not
necessarily involve comparison of superiority.
However, \ili{Luganda} TRMs do not compose with anything like degree modifiers, and so we need not commit to any account of the positive form of gradable adjectives.\footnote{A reviewer suggests that temporal adverbs might be treated as degree modifiers under this view. We leave this intriguing suggestion for future work.}
We provide denotations for the \ili{Luganda} TRMs below which build in the \isi{vagueness} witnessed in the positive morpheme above.

Our denotations are built upon the following measure functions, which relate time intervals to degrees.

\ea
 {\bf close}$(t, t')$ assigns to a time $t'$ a degree on a scale of closeness to $t$
\z

\ea
 {\bf far}$(t, t')$ assigns to a time $t'$ a degree on a scale of distance from $t$
\z

We embed our analysis in a theory of tense which says that, like tenses in \ili{English}, TRMs in \ili{Luganda} in matrix clauses constrain the relation between an evaluation time and a reference time (RT), where that evaluation time is, at least in matrix contexts, utterance time \citep{reichenbach47tenses, klein94time}. However, cf.~\citet{cable12beyond,cable14nels45} for a different view of TRMs in G\~\i k\~uy\~u.
Following \citet{heim94comments} and \citet{kratzer98more} we assume that temporal operators constrain the relation between RT and UT by placing presuppositions on the reference of a temporal \isi{pronoun} which corresponds to RT.
Following \citet{cable12beyond} and others, we assume that TRMs do not denote this \isi{pronoun} itself, but adjoin to such pronouns.
Note that these are all assumptions, and moreover our analysis does not crucially hinge on any of them.


\ea\label{recent}
$\llbracket$ {\sc rec.pst} $\rrbracket^t = \lambda t': $ {\bf close}$(t, t') \succ \textbf{s}(\textbf{close})$ \& $t' < t$ . $t'$ 

Presupposition: the degree of closeness of $t'$ to $t$ exceeds a contextual standard 
\z

\ea\label{intermediate}
$\llbracket$ {\sc int.pst} $\rrbracket^t = \lambda t': $ {\bf far}$(t, t') \prec \textbf{s}(\textbf{far})$ \& $t' < t$ . $t'$

Presupposition: the degree of far-ness of $t'$ to $t$ is less than a contextual standard \\
Implicature: the degree of closeness of $t'$ to $t$ is less than a contextual standard 
\z

\ea\label{distant}
$\llbracket$ {\sc dist.pst} $\rrbracket^t = \lambda t': %$ {\bf far}$(t, t') \succ \textbf{s}(\textbf{far})$ \& 
t' < t$ . $t'$

Implicature: the degree of far-ness of $t'$ to $t$ exceeds a contextual standard.\footnote{Crucially the scalar implicature generated by comparison to the other tenses carries the same \isi{vagueness} and context-dependence that those tenses do.} 
\z

In the denotations above, $t'$ corresponds to RT, i.e., the temporal \isi{pronoun} that the TRMs adjoin to, while $t$ corresponds to the evaluation time, which in matrix contexts is UT.%
\footnote{The same may also be true for embedded contexts; see \citet{cable12beyond}}

Each of these morphemes is a partial identity function on RT, presupposing two things about it: 
First, that it is in the past relative to UT ($t' < t$), and second, its distance from $t$, denoted by the measure function and the comparative operator ($\succ$ or $\prec$).

Our analysis of these TRMs makes them comparable to the positive forms of relative gradable adjectives like \textit{tall}.
So our analysis captures the \isi{vagueness} associated with these expressions.
Note that this analysis does not commit us to one view of \isi{vagueness} over another; since our analysis equates these expressions with positive gradable adjectives, whatever account can be made of gradable adjectives can be extended to TRMs. 





\section{Accounting for crosslinguistic variation: scale type and vagueness}\label{sec:bochnakklecha:6}

Why are \ili{Luganda} TRMs vague while \ili{English} tenses are not?
We suggest it is for the same reason that \textit{relative} gradable adjectives are vague in their positive forms, while \textit{absolute} gradable adjectives are not.
Absolute adjectives, like \textbf{full}, are still perfectly gradable (\ref{full}), but fail the tests described above for \isi{vagueness}.

%{\bf Proposal:} difference between \ili{Luganda} and \ili{English} past tenses akin to the relative vs.~absolute distinction in gradable predicates
%
%\begin{itemize}
%\item \ili{Luganda} TRMs: similar to relative standard gradable predicates
%\begin{itemize}
%\item context-dependent and vague
%\end{itemize}
%
%\item \ili{English} past tense: similar to absolute minimum standard gradable predicates
%\begin{itemize}
%\item used for any non-zero temporal distance before UT; not vague
%\end{itemize}
%
%\end{itemize}
%

%
%Distinction in gradable predicates with respect to \isi{vagueness} (see \citealt{kennedy07vagueness})
%
%\begin{itemize}
%\item relative standards, e.g., {\it tall}, {\it expensive} $\rightarrow$ vague
%\item minimum/maximum standards, e.g, {\it bent}, {\it closed} $\rightarrow$ not vague
%\end{itemize}

\begin{exe}
\ex\label{full} This cup is more full than that one.
\end{exe}

For example, for (\ref{bent1}) to be true, the degree of fullness exhibited by the cup must simply be the maximum; there is no contextually determined standard.

\begin{exe}
\ex\label{bent1} This cup is full.
%\ex\label{closed1} This door is closed.
\end{exe}

Likewise, such adjectives do not have borderline cases or give rise to Sorites Paradoxes.\footnote{Note that, as \citet{kennedy07vagueness} discusses, such adjectives may give rise to \textit{imprecision}, e.g., in a case where a cup which is not strictly speaking full is called `full' in a particular context; this is different from \isi{vagueness}. See \citealt{kennedy07vagueness} for more discussion.}
\citet{kennedy07vagueness} attributes this to the differences between these adjectives in terms of their \textit{scale structure}.
In other words, while there is an inherent upper-bound to fullness, there is no inherent upper-bound for height.
So for an object to `stand-out' on the fullness scale it simply must occupy the highest point on said scale, while the same cannot be said for the height scale.
Instead, since such scales lack obvious `milestones' for determining what stands out, interlocutors must appeal to context.
This in turn gives rise to \isi{vagueness}.

\ili{English} past tense is like an absolute gradable adjective; it does not exhibit \isi{vagueness} because it does not depend upon context (not in the same way as a relative gradable adjective, anyway).
We therefore predict that no language with a single past tense like \ili{English} should exhibit \isi{vagueness} either.
We also predict that, if any languages with graded tense systems like \ili{Luganda} actually do differentiate them in a rigid, diurnal way, as \ili{Luganda} was described to do by \citet{ashton54luganda}, then we also predict that those graded tense systems should not exhibit \isi{vagueness} either. 
And finally, we predict that any graded tense system which is context dependent like \ili{Luganda}'s should also be vague.

\section{Conclusion and future research}\label{sec:bochnakklecha:7}

Contrary to descriptions like that of \citet{ashton54luganda}, the graded tense system of \ili{Luganda} exhibits context dependence and \isi{vagueness}.
This finding expands the empirical domain for work on \isi{vagueness}, which so far has been focused mostly on adjectives and other lexical categories. 

Many questions remain.
For example, \citet{cable12beyond} argues against treating TRMs in G\~\i k\~uy\~u as tenses, relying on data from adverbials and embedding. 
Tenses, according to \citet{klein94time}, relate reference time to utterance time, while the relation between reference time and the event described by the \isi{verb} is mediated by aspect.
According to Cable, TRMs in G\~\i k\~uy\~u behave like neither tense nor aspect according to these definitions, instead relating utterance time directly to the event described by the \isi{verb}.
It remains to be seen whether \ili{Luganda} TRMs are like those in G\~\i k\~uy\~u in this regard.
This question is left for future research.

%
%The truth conditions of (\ref{bent1})-(\ref{closed1}) are not context dependent.
%
%\begin{itemize}
%\item an object must have only a non-zero degree of bend to count as bent
%\item a door must be completely closed to count as closed (modulo imprecise uses)
%\end{itemize}

%
%{\bf Borderline cases:}
%
%There are (almost?) no contexts where it's hard to judge whether the \isi{predicate} holds or not
%
%\begin{itemize}
%\item so long as the minimal/maximal condition is met, the sentences in (\ref{bent1})-(\ref{closed1})
%\item see \citet{burnett14delineation} for an opposing view
%\end{itemize}
%
%
%{\bf Sorites sequences:}
%
%%\begin{exe}
%%\ex
%P1: A nail that is bent at an angle of 45 degrees counts as bent. \\
%P2: \#A nail that is bent at an angle of 1 degree less than a bent nail counts as a bent nail.
%%\end{exe}
%
%\begin{itemize}
%\item We don't accept the second premise
%\item Not true for a bent nail that has 1 degree of bend
%\end{itemize}



%\ili{English} past tense: any non-zero time before now counts as past
%\begin{itemize}
%\item not vague or context dependent
%\item behaves like minimum standard adjectives
%\end{itemize}
%
%
%Why compare \ili{English} past tense to non-vague gradable adjectives in the first place? What's the evidence that they are gradable at all?
%\begin{itemize}
%\item no context-dependent 
%\item can't appear in comparative constructions
%\end{itemize}
%
%Other temporal operators in \ili{English}-like languages are nevertheless gradable:
%
%%\begin{exe}
%%\ex
%%\begin{xlist}
%%\ex Alla came {\bf earlier}/{\bf later} than we thought.
%%\ex Alla kam {\bf fr\"uher}/{\bf sp\"ater} als wir dachten.  \hfill \citep{vonstechow09temporal}
%%\end{xlist}
%%\end{exe}
%
%Temporal morphology comes in vague and non-vague varieties across languages
%\begin{itemize}
%\item languages can choose to lexicalize vague or non-vague relations in tenses, temporal adverbs, etc.
%\end{itemize}
%
%

%
%\section{Outstanding issues}
%
%
%\begin{itemize}
%
%\item \citet{cable12beyond} argues that TRMs in G\~\i k\~uy\~u are not true ``tenses'', since they do not restrict utterance time and reference time
%
%\item \citet{reichenbach47tenses, klein94time}: three key times for establishing temporal and aspectual reference: \\
%\hspace*{10mm} $\rightarrow$ utterance time (UT); reference time (TT); eventuality time (ET)
%
%\begin{itemize}
%\item tense features are relations between UT and TT
%\item aspectual features are relations between TT and ET
%\end{itemize}
%
%\item TRMs in G\~\i k\~uy\~u restrict UT and ET, so they share semantic properties with time adverbials like {\it yesterday}
%\begin{itemize}
%\item preliminary evidence suggests \ili{Luganda} TRMs may behave the same
%\end{itemize}
%
%\item but still much more to say about temporal embedding, temporal adjuncts, etc.
%
%\end{itemize}
%
%
%
%\section{Conclusions}
%
%Can analysis be extended to certain temporal adverbs in \ili{English} and other languages, e.g. {\it just}, {\it recently}, etc.?
%
%%%\begin{exe}
%\ex%\begin{xlist}
%%\ex Context: I come to Malte's office for a meeting, and he tells me he wants to go to a cafe to have a coffee for our meeting. I say: 
%%\ex `I was just at Classics Cafe (a few minutes ago/right before I came here), I could have brought you a coffee.' 
%%\ex `I was just at Classics Cafe a couple of days ago, so why don't we go somewhere else.'
%%\end{xlist}
%%\end{exe}
%
%%\begin{exe}
%\ex%\begin{xlist}
%%\ex Context: You are telling me about your summer plans, and tell me you are going to visit Paris. I say: 
%%\ex `I was just in Paris (a few months ago), and I had a really good time.' 
%%\ex ?? `I was just in Paris a few minutes ago.' 
%%\end{xlist}
%%\end{exe}








\section*{Abbreviations}
\begin{tabularx}{.45\textwidth}{lQ}
1, 2, 3 & first, second, third person\\
 {\sc agr} & agreement\\
 {\sc aug} & augment\\
 {\sc dist} & distant\\
 {\sc int} & intermediate\\
 {\sc loc} & locative\\
 {\sc nc} &  {noun class}\\
 \end{tabularx}
\begin{tabularx}{.45\textwidth}{lQ}
 {\sc obj} & object\\
 {\sc part} & particle\\
 {\sc pst} & past\\
 {\sc rec} & recent\\
 {\sc sg} & singular\\
 {\sc subj} & subject\\
 \\
 \end{tabularx}




\section*{Acknowledgements}
We thank Kisuule Magala Katende and Waiswa Nkwanga for their patience and enthusiasm for working with us on their language. We also thank audiences at the University of Chicago, the University of Potsdam, SWAMP at University of Michigan, the 40th Berkeley Linguistics Society, ACAL 45 at the University of Kansas, and two anonymous reviewers for comments and criticisms. The usual disclaimers apply.

{\sloppy
\printbibliography[heading=subbibliography,notkeyword=this]
}

\end{document}
