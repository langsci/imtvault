\documentclass[output=paper,modfonts]{langscibook} 
\title{How multilingual policies can fail: Language politics among Ethiopian political parties} 
\shorttitlerunninghead{How multilingual policies can fail}
\author{%
Mehari Zemelak Worku\affiliation{Addis Ababa University}
}
\ChapterDOI{10.5281/zenodo.1251716}
\abstract{Because language has instrumental as well as symbolic values, the issue of language will always have a political aspect \citep{Smith2008}. Often, the choice of language and its use is construed as one of the central traits to people’s definition of themselves. Besides, any given state must decide or determine the language that it deems appropriate to carry out its development and to generate, disseminate and enrich the knowledge necessary for such development. However, the case grows problematic when it comes to Sub-Saharan Africa where “every language carries a distinct and weighty baggage” of identity (\citealt{ObengPurvis1999}). The decision was not easy for different regimes in Ethiopia, home of more than 80 ethnic groups (CSA 2008). The three consecutive regimes which have ruled the country for the last 75 years followed different paths in addressing this diversity management question. The reframing of the country under ethnic federalism, which legislates Amharic as the working language of the federal government (hereafter WL) and guarantees the right of each ethnic state to decide its own WL, is the recent attempt to respond to the same politics of recognition. However, dissatisfied voices regarding the current language policy (hereafter LP) can still be heard among political groups. Some see it as ‘not enough’ while others see it as Balkanization. Despite a few research efforts and publications on the LPs of the consecutive governments of Ethiopia, there has been no research done on the alternative policies and options available among the political parties or their relative value as LPs. Thus, the grand objective of this study is to survey, analyze and evaluate the linguistic proposals of Ethiopian political parties in government, education, and endangered languages. }
% \keywords{language policy; language politics; working language; language in education; Ethiopia} 
\begin{document}
\maketitle
 
 


~
\newpage 
\section{Language politics in Ethiopia: Historical overview}
 

In this section I will briefly discuss the issue of language throughout Ethiopian political history.\footnote{In this regard the works of \citet{Cooper1976,McNab1990,Mekuria1998,GetachewDerib2006,HeughEtAl2007,Smith2008,Mehari2011} and \citet{Zelealem2012} are worth reading for a more detailed discussion on the typology of the LPs of different regimes in \isi{Ethiopia}.  Even though it is not going to be confined to their outlooks, the timing frame of the discussion in this paper is basically sketched out by the marked works.}  A compact version of the history will be presented with a focus on three sections: pre-1974 (before the outbreak of the student-led Ethiopian Revolution), 1974--1991 (the time where the military junta called the ``därg'' stayed in power) and post-1991, the era of the incumbent EPRDF.
 
\subsection{Pre-1974: Language in imperial Ethiopia}
 

“In the historical, political and literary fields, two languages have dominated Ethiopian Studies: \ili{Amharic} and \ili{Geez}” \citep{BenderEtAl1976intro}. Along with their dominant status in the field of Ethiopian Studies, Bender’s words can be extended to show how overriding the two languages were in the overall social life of \isi{Ethiopia}. As the antiquities from the Axumite Dynasty indicate, \ili{Geez} was the official \isi{indigenous language} of the Axumite Dynasty \citep{Philipson2012}. This official domain had been kept to \ili{Geez} until \ili{Amharic} took over as 
% \textcyrillic{
lәsanä nәgus %\todo{check spelling of this word}
% }
`the language of the king' some time during the reign of the Zagwe Dynasty in the 11th–13th centuries \citep{Girma2009}. We do not have any document to answer why and how \ili{Geez} took over the higher domains as the official language in that time. Nevertheless, we can say it can be one of the two cogent reasons given by \citet{Cooper1976}: as a result of government-implemented official LP or as a consequence of societal dynamics among speakers of different languages, or maybe both.

Irrespective of efforts by evangelists to reduce other Ethiopian languages into writing like Onesimos Nesib, an \ili{Oromo} missionary who translated the Holy Bible into Affan \ili{Oromo}\footnote{\citet{Cobarrubias1983} wrongly writes “Gallah [sic] in \isi{Ethiopia}” (Galah is a derogatory reference to mean \ili{Oromo}) as an example for a language “without a writing system.” But, by the time Cobarrubias’ article was published, Afan \ili{Oromo} speakers had been using the Afan \ili{Oromo} Bible for about 90 years.} using the Ethiopic script in the early 20th century, \ili{Amharic} and \ili{Geez} continued to confluence dominantly in the educational, religious and political domains of the region until the 20th century, when western education stepped in to the country with the complete support of emperor Menilik II ({\amharicfont\footnotesize ኃይለየሱስ} 2008). The \isi{policy} of the Ethiopian kingdom regarding other languages was only to be found in the indigenous traditional collection of religious and secular laws known as the 
Fәtha Nägäst %\todo{check spelling of this word}
(law of kings). The code has been used in the country for several centuries and contains legal provisions for speakers of other indigenous languages to have judges speaking their native tongues (\citealt{Paulos1968} cited in \citealt{Cooper1976}).

The boost of modern education and its legacy during the reign of emperor Haile Selassie I created a new domain for \ili{Amharic} and \ili{Geez}. Meanwhile other indigenous languages, regardless of their powers in numerical terms, were kept only to non-official domains. The education arena, in its early stages, was generally a field for the learning of foreign languages. Remarkably, it was a stage of competition between \ili{French} and \ili{English}, as the envoys of both neighboring colonial powers vied for swift influence over the Ethiopian crown. Initially, the \ili{French} were more successful than their British contenders in retaining a positive attitude of the oligarchy towards \ili{French}. However, this lasted briefly, only up to 1935 when Fascist Italy declared war on the Ethiopian Empire and invaded to avenge the humiliating defeat it suffered at the battle of Adowa in 1896. \citet[610]{BowenHorn1976} note that during the \ili{Italian} occupation, which lasted until 1941, the burgeoning educational efforts of \isi{Ethiopia} were interrupted and most schools were closed. The mission of the few functioning schools was altered from training indigenous children to be citizens defending the country’s interest in all possible domains, to educating the colonial masters’ \ili{Italian} children to keep them in touch with their European “mother civilization.” This offered them a corner of the earth where they could enjoy ‘sun bathing’, to use Mussolini’s terms. Ethiopians were obliged to stay confined to primary level education.

With the implementation of the age-old \textit{divide et impera} \isi{policy} of oppressors, the Italians divided the country into six regions according to \isi{major} ethnic lines. They also ordered, without any deliberate effort to capacitate the languages in corpus planning, each of the \isi{major} indigenous languages to be used in native schools in the respective regions. In addition, the Governor- General of each province had the authority to establish any other language as additional instructional medium in the schools (\citealt[322]{Pankhurst1976}, cited in \citealt{BowenHorn1976,Bahru2002}).

\begin{table}
\begin{tabularx}{\textwidth}{llQ}
\lsptoprule
{\bfseries No.} & {\bfseries Regions named by the Italians} & {\bfseries Languages proclaimed to be media of instruction}\\
\midrule 
{1} & {\isi{Eritrea} and Tigray} & {Tigrinya and Arabic}\\
{2} & {Amhara} & {Amharic}\\
{3} & {Addis Ababa} & {{Amharic} and Oromifa}\\
{4} & {Harar} & {Harari and Oromifa}\\
{5} & {Galla\textsuperscript{*} and Sidama} & {Oromifa and Kafficho}\\
{6} & {Somalia} & {Somali}\\
\lspbottomrule
\end{tabularx} 
\caption{Administrative regions and languages of instruction during the Italian occupation (1935--1941).\\
* This term is used by the Italians to refer to the \ili{Oromo} people. However, the \ili{Oromo} people do not refer themselves with this word and is considered offensive.}
\label{tab:zemelak:1}
\end{table} 
 

This LP, nevertheless, was not put into full effect except in \isi{Eritrea}, which had been a colony of Italy for about fifty years after the battle of Adowa in 1896. First, the \isi{policy} was understood by the natives as a separatist \isi{movement} to dissuade the country’s noble men and gentry in to ethnic division to enervate the patriotic movements. Second, during the \ili{Italian} occupation the education system was barely functioning. Third, as a result of the multitude of patriotic movements against them, the \ili{Italian} Fascists did not have enough time to establish a lasting educational system. Nonetheless, this time is remarkable because of the deliberate abandonment of any effort in the education sector to promote a common \isi{indigenous language} of communication among the variegated linguistic groups of \isi{Ethiopia}. The LP of the occupiers came to an end when the Italians were ousted by the coalition of Ethiopian patriotic forces and the British troops from eastern Africa in 1941.

After the Italians, the British took their shot at attempting to influence the crown. Since then ‘the British way’ of doing things came to be accepted as the better way in alignment with His Imperial Majesty Haile Selassie’s I aspirations for a strong, modernized nation. This was particularly reflected when gradually \ili{English} took away the education domain that \ili{French} occupied.\footnote{According to \citet{BowenHorn1976} the erosion, nevertheless, began in the 1930s before the Italians stepped in to the Ethiopian soil. Aleme Eshete (cited in \citealt{BowenHorn1976}) points at one person’s influence on the crown, Dr. Workneh Eshete, who was taken to India, a British colony by then, and studied medicine. As Aleme remarks, because of this man’s beliefs in the British system and conviction to make the country as strong as the Great Britain, a group of 18 Indians were brought to \isi{Ethiopia} in the 1930 to serve as technicians, doctors, teachers and other badly needed specialists. Since Indians were not well versed in \ili{French}, if not oblivious at all, it was necessary for the students to become literate in \ili{English} to exploit the available expatriate resource. However, a few exceptions can be mentioned in relation to the use of local languages in the Ethiopian primary education. For example, in \ili{Swedish} and American missionary schools, mother tongues were used before European languages at the primary level (\citealt{McNab1988}, cited in \citealt[45]{HeughEtAl2007}).}

\ili{Amharic} had also reclaimed its former dominance over all indigenous languages as Haile Selassie I aspired to build a unified modern empire under one crown and one \isi{dominant language}. In the revised constitution of \isi{Ethiopia} it was stated explicitly that “the official language of the empire is \ili{Amharic}.” (Revised Constitution of \isi{Ethiopia}: 125).
\nocite{revisedconstituionethiopia}
It was a must that new laws be published in \ili{Amharic} and \ili{English} in the official gazette of \isi{Ethiopia}. The statutes of domestic companies were required to be filled in \ili{Amharic} or \ili{English}. Foreigners were also required to write and speak \ili{Amharic} ‘perfectly’, in order to attain Ethiopian citizenship. According to the regulation decreed in 1944 concerning missionary activities in \isi{Ethiopia}, missionaries also were obliged to learn \ili{Amharic} to pursue their mission for it was outlawed to use other indigenous languages except at the early stages of the missionary activity. The credo insists on both the missionary and the pupil to have a working knowledge of the “official language of the empire” (Cited in \citealt{Cooper1976}). Such proclamations, charters and codes vividly display the dominant status \ili{Amharic} had in the state.

In the 1960s, as a result of external and internal political pressure exacerbating through time, the state-owned broadcast media had shown some signs of a more relaxed stance towards embracing indigenous languages other than \ili{Amharic}. The languages that were introduced to the government-owned broadcasting radio stations were \ili{Afar}, \ili{Somali}, Tigrinya and Tigre \citep{Smith2008}. However, the time allotment for them was extremely scanty when compared with the time allotment to \ili{Amharic}. The imperial administration did not have the political will to abandon the counterproductive effort of unifying a multilingual country under one language and a unitary state.
 
\subsection{1974--1991: The Ethiopian Student Movement and the därg on the language issue}
 

\ili{Amharic} as an established state language was challenged with the increase in influence of modern education in the country’s social, economic, and political domains. University students who were becoming increasingly political espoused strong criticisms against the regime. The imperial regime did not have the administrative ability among its agents to undertake any successful amendment to appease the public demand \citep{Bahru2002}.

Left wing Marxism became the dominant ideology among the students. Immediate recognition of “the oppressed identities of nations and nationalities”{} \footnote{It is at this particular time where ethnic groups of the country started to be referred as “nations and nationalities” in the political discourse of the country.} was one of the main demands of the students. The issue reached a boiling point when the Lenin/Stalin-Fanon inspired Wallelign Mekonen published an article entitled “On the Question of Nationalities” on 17th November, 1969 in the Haile Selassie I University student publication, \textit{Struggle}. Walelign merged the Leninist/Stalinist idea of \textit{the nation} with Frantz Fanon’s justification of violence against colonial oppression. His argument was that \isi{Ethiopia} was not a nation; rather it was a collection of nations (Gurage, \ili{Somali}, \ili{Oromo}, etc.) that had been subdued by the “Amhara-Tigre (two of the dominant ethnic groups) hegemony.” These oppressed nations, he explained, can only reclaim their freedom through “revolutionary armed struggle” \citep{Walelign1969}.{} \footnote{For \citet{Stalin1913} one of the defining characteristics of a nation is speaking one common language. After defining the nation as “a historically constituted, stable community of people, formed on the basis of a common language, territory, economic life, and psychological make-up manifested in a common culture,” he asserts that “it is sufficient for a single one of these characteristics to be lacking and the nation ceases to be a nation.” Wallelign’s denial of the existence of the “Ethiopian nation” as one nation seems to be based on influence from his Marxist tendency.} Since then, the language issue has clung to Ethiopian politics as a symbol of identity recognition for ethno-linguistic groups in the country.

The imperial regime was toppled in 1974 by a revolution rooted in the student \isi{movement}, and abetted by the distraught military personnel of the regime. In the wake of the victorious revolution, a sense of euphoria spread with the prevailing idea of creating a modernized and prosperous \isi{Ethiopia}, where all people, irrespective of their linguistic identity, would be treated equally. The trajectory of ethnic equality spilled over to language, as every linguistic group in the country was to be recognized as equals (\citealt{Bahru2002,Merera2006,Balsvik2007}).

Consequently, the military council,{} \footnote{Often called “the därg,” a word derived from \ili{Geez} to mean ‘committee’ or ‘council’.} which hijacked the revolution,{} \footnote{The aim of the campaign had two aims: to show the military council’s intention to the linguistic equality and empowerment of all Ethiopians, to weaken the \isi{voice} of the students, who were calling for the establishment of civilian administration, by dispersing them in the rural parts of the country as teachers in the \isi{literacy} campaign \citep{Balsvik2007}.} waged a \isi{literacy} campaign across the country using 16 indigenous languages as media of instruction for basic \isi{literacy}. This was in direct contrast to the former overtly monolingual education \isi{policy}. The 1976 political program of the National Democratic Revolution of \isi{Ethiopia} overtly proclaimed the right of nationalities as the following:

\begin{quote}
The right of self-determination of all nationalities will be recognized and fully respected. No nationality will dominate another one since the history, culture, language, and religion of each nationality will have equal recognition in accordance with the spirit of socialism…. Given \isi{Ethiopia}’s existing situation, the problem of nationalities can be resolved if each nationality is accorded full rights of self-government. This means that each nationality will have regional autonomy to decide on matters concerning its internal affairs. Within its environs, it has the right to determine the contents of its political, economic, and social life use its own languages and elect its own leaders and administrators to head its own internal organs (Ethiopian Government Programme (1976) cited in \citealt{McNab1990}).
\end{quote}

Despite the overflowing rhetoric, the centralist ideology of the military regime and its ambition to create a unified \isi{Ethiopia} was uncompromising to the idea of transferring power to a civilian government. It did not take much longer for the politics of recognition and autonomy to ascend to the dome of political discourse as an unanswered question. Because of its extremely repressive politics the military junta forced most of the leftwing socialists and ethnic liberation groups, some of whom were already guerilla fighters, to armed struggle. Later, the military council unsuccessfully tried to transform itself into a civilian government. In 1987  it promulgated a new constitution, which renamed the country as the Peoples’ Democratic Republic of \isi{Ethiopia}. All indigenous languages were guaranteed equal recognition as languages of the country. \ili{Amharic} lost its status as the national language but maintained its degree of dominance as the constitution declared it as the official Working Language (hereafter WL) of the government. 

Under this constitution the country was reframed under 14 autonomous provinces as way to appease the guerilla fighting ethnic rebel groups. It was too late! The rebels disparaged the transformation attempt as a theatre to cover all the cantankerous and assimilationist natures of the dictator Mengistu Hailemariam to pen in the rebels, but they did not give in.
 
\subsection{Post-1991: Language in the country of “Nations, Nationalities, and Peoples”}
 
The military regime’s LP was essentially a \textit{de jure} claim of equal status of all languages in the country, while the \textit{de facto} LP was merely a preservation of the imperial LP which prescribed \ili{Amharic} for all educational and official domains at the expense of other indigenous languages. Research done by the Ministry of Education in 1986, proves the \isi{language policy} of the military regime was merely the extension of the imperial \isi{policy}, jeopardized the enrolment and efficacy of students whose mother tongue was not \ili{Amharic} (cited in \citealt{HeughEtAl2007}. 

On May 28th 1991 the coalition of ethnic cored armed groups under the umbrella of EPRDF took control of the capital Addis Abäba, ousting the military regime. The questions of autonomy and recognition of linguistic identity were reinvigorated. In July, 1991 a conference was held for the plethora of newly formed as well as old, but highly enervated, political parties. At the conference a charter, which served as a foundation for the forthcoming constitution, was provided. That was the stance where the political arena, which used to be entirely occupied by a military junta, officially proclaimed to embrace several parties to give sufficient significance for all the voices of cultural and political groups (\citealt{Bahru2008,Kassahun2003}). The question of language appeared to be a subject of notability to the political agendas held by most of the political groups. The historian Bahru Zewde recalls an event from the conference:

\begin{quote}
The first order of business in the course of the eventful deliberations of that conference was the determination of the working language. In view of the difficulty getting interpreters for the multiplicity of languages represented by the ethno-nationalist organizations attending the conference, it was decided to adopt \ili{Amharic} and \ili{English}. However, organizations that brought their own interpreters could use their language. The EPLF leader, Issayas Afeworki, took the latter option and addressed the conference in his native Tegreñña. But what is of particular interest to this investigation is the ire that he visibly demonstrated at what he felt was a less than adequate \ili{Amharic} rendering of his delivery, showing that his \ili{Amharic} was as good as, if no better than, his designated interpreter. The spokesman of OLF, also understandably, addressed the conference in his native \ili{Oromo} language. But it was translated not in to \ili{Amharic} but into \ili{English}, there by leaving the overwhelming majority of the national audience in the dark \citep[77]{Bahru2008}.
\end{quote}

This is a self-explanatory situation to portray the linguistic sensitivity of the post 1991 political system. In 1992  the new Transitional Government of \isi{Ethiopia} first proclaimed the rights of every ethnic group of \isi{Ethiopia} to use and develop its languages and culture. This was further strengthened and confirmed in the 1995 Ethiopian Constitution. 

In contrast to the 1987 constitution which starts with the phrase “We, the working people of \isi{Ethiopia}”, the preamble of the 1991 constitution begins with “We, the Nations, Nationalities and Peoples of \isi{Ethiopia}.”{} \footnote{However, the constitution doesn’t give a single hint, let alone definition, to state the difference between these three terms is. Rather, it defines all the three terms as follows:  {\textquotedbl}Nation, Nationality or People for the purpose of this Constitution, is a group of people who have or share large measure of a common culture or similar customs, mutual intelligibility of language, belief in a common or related identities, a common psychological make-up, and who inhabit an identifiable, predominantly contiguous territory” (Article 39: 4). Hence, in the current political discourse of \isi{Ethiopia} any linguistic group is considered as a “nation/nationality” \citep{Vaughan2006}.} The constitution also divides the country into nine autonomous ethnic states each with its own constitution, flag and the right to self-determination including and up to secession (Article 39). This took nobody by surprise since most of the political players who had a greater role in writing the constitution were members of the student \isi{movement}. In the retention of \ili{Amharic} as the WL of the government and the egalitarian approach to all languages in the country, the new constitution follows in the footsteps of the 1987 constitution (Article 5: 2). Audaciously supporting the ethnic cause, the new constitution gives member ethnic states of the federation unsurpassable right to determine their respective WL by law (Article5: 3). Subsequently, indigenous languages became widely used in primary education, media, administrative and judiciary systems. In  1994 the Ministry of Education of \isi{Ethiopia} proclaimed a new Education and Training Policy, which declares: “Cognizant of the pedagogical advantage of the child learning in mother tongue and the rights of nationalities to promote the use of their languages, primary education will be given in nationality languages” \citep{FDRE1994}. The existing language in education \isi{policy} frames the entire system as indicated by the following table.
 
\begin{table}
\begin{tabularx}{\textwidth}{lQQQ}
\lsptoprule
{\bfseries No.} & {\bfseries Languages} & {\bfseries Level of education} & {\bfseries Function}\\
\midrule 
{1} & {Mother tongues/ Nationality languages} & {Primary}  & {Medium of instruction}\\
\tablevspace 
{2} & {Amharic} & {Primary up to secondary level education} & {As a subject and language of country wide communication}\\
\tablevspace 
{3} & {English} & {From grade 1} & {As a subject}\\
\tablevspace 
{4} & {English} & {Secondary and higher education}  & {As a medium of instruction}\\
\tablevspace 
{5} & {One more nationality and foreign language} & {All levels} & {As subjects for intercultural and international communication} \\
\tablevspace 
{6} & {Mother tongues/ Nationality languages} & {Teachers training for primary level education} & {Medium of Instruction}\\
\lspbottomrule
\end{tabularx}
\caption{The Existing Language in Education Framework.}
\label{tab:zemelak:2}
\end{table} 

Currently, there are twenty five indigenous languages used as media of instruction. Most regions selected indigenous languages spoken by the ‘majority’ of the denizens of that particular area. It is about twenty years now since this LP favoring indigenous languages has been implemented, and there is a sizeable amount of research reckoning the effects of this \isi{policy}. However, it is still considered by the EPRDF government as one of its achievements from both the linguistic rights as well as pedagogical point of view. There are political groups who consider the current system as ‘The Answer’ for all questions on linguistic rights. On the other hand, there are others who criticize the system as Balkanization and enhancing deadly ethnic conflicts. These voices have representations in the political discourse of the country through political parties which suggest options through their political programs and manifestos. In the next part of this paper I will try to expound what \isi{policy} proposals are forwarded by these political groups of the country.
 
\section{Sampling methodology and conceptual tools} 

Scarcity of resources would not allow this study to include all the National Electoral Board of \isi{Ethiopia} recognized (hereafter NEBE) 79 political parties, since most of them are regional parties \citep{NEBE2009}. Particularly the number of regional parties, 63, and their scattered location all over the country is unmanageable to be taken as a whole. Hence, this study is done by taking representative samples. According to the NEBE, there are 79 National\footnote{According to the \citet{NEBE2009}, national political parties are those whose campaigns are not limited to a certain ethnic state. In contrast, parties which run only in a certain ethnic state are called regional parties.} and Regional Political Parties. Out of these 19 are categorized by the NEBE as National Political Parties while the remaining 60 are labeled as regional. Out of the 19 parties labeled as national 10 were selected randomly.\footnote{GSPP, EDP, EPRDF, Forum, EJDFF, AEUP, EPAP, ERaeiP, UEDF, AEDP.} The same was done to the parties categorized as regional. Fourteen regional parties were selected randomly.\footnote{ONC, OPDU, DDQPDU, DDDP, OLP, WPDF, HNL, GPRDF, KPC, ANDO, ARDUF, BMPDO, BGDUF, WSDP.} This enabled the sample to include about 30.4 \% of the entire party population. The size of the sample was intentionally expanded to include parties from different ethnic and ideological bases. Moreover, it is worth mentioning that the distribution of regional parties is extremely uneven throughout the regional states of the country. Among the regional parties there were those which reported to NEBE not to be functioning anymore and thus excluded from the sampling. 

The conceptual tools used in this descriptive study to analyze the types of the LPs as well as the motivations and ideologies beneath the LPs are garnered eclectically from \citet{Fishman1972,Cobarrubias1983,Schiffman1996} and \citet{Patten2001}.
 
\section{Language politics among the functioning political parties of Ethiopia} 

In this section an overall assessment of the LPs of the political parties is presented first. Then, I will deal with the proposals of the political parties about the language that should take the status of the WL of the federal government. The next focus of the analysis is the proposals about language in education. Through the analysis besides providing typology of the LPs motivations and ideologies of the proposals are inferred.
 
\subsection{Overall assessment of the political parties’ LPs} 
Although the need for a thoroughly thought-out and well planned LP for a multilingual country such as \isi{Ethiopia} is unquestionable, out of the 24 political parties included in this study only seven\footnote{EDP, AEDP, EPRDF, OLP, Forum, GPRDF and ONC.} have put relatively clear and implementation orientated LPs. It is only these seven parties which explicitly propose to legislate, if they are voted to power, what the federal WL should be or how it should be selected, and what the language use in education shall be. Moreover, they also suggest establishment and expansion of language research centers and the deployment of resources by the federal government to the corpus development of the languages. Out of the remaining 17 parties 16 of them do not give detailed and implementation oriented LPs. The remaining one party\footnote{DDPDO.} proposes nothing about the language issue. 

Depending on the broadness or narrowness of their LPs, the 16 parties without detailed LPs can also be divided into two sub-categories. First,\footnote{HNL, AEUP, GSPP, WPDF, DDQPDO, UEDF, EJDFF, OPDU, EPAP, ERaeiP.} there are ten parties which present a \textbf{too-broad} and highly generalized claim to ensure the recognition of the languages and cultures, history and identities ethnic groups. Even though these parties have not offered any detailed implementation frame work, they have displayed support to mother tongue education at primary level. The second group holds five\footnote{ARDUF, WSDP, ANDO, BMPDO, KPC.} parties, all of them regional parties, which provided \textbf{too-narrow} LPs in their political programs such as, language X shall be included in schools as a subject. The proposals in this group neither tell what the medium of instruction at the schools shall be nor what is to be done in the development of the languages they suggest to be included in the education system; nor do they say anything about the linguistic rights of other linguistic groups in the area where these parties aim to gain legitimacy.
 
\subsection{Issue of the federal WL} 
Out of the seven parties that give relatively detailed and implementation orientated LPs, two\footnote{OLP and ONC.} propose the promotion of Afan \ili{Oromo}, to have the WL status beside \ili{Amharic}. The reason for such legislation, the two parties forward, is the numerical muscle Afan \ili{Oromo} have.\footnote{\ili{Oromo} is the largest ethnic group in \isi{Ethiopia} (CSA 2008).} Three\footnote{EDP, AEDP, and FORUM.} parties from this group propose the adoption of an additional WL beside \ili{Amharic} without mentioning a particular language. One\footnote{FORUM.} of these three parties forwarded that the additional WL should be chosen from the indigenous languages based on its numerical muscle; while the other two\footnote{EDP and AEDP.} preferred the selection to be on the basis of “facilitating science, technology and market exchange.” On the other hand, two\footnote{EPRDF and GPRDF.} of these seven parties stand in peculiarity by not mentioning the promotion of any specific language overtly; rather they propose egalitarian preservation, usage and development to all languages in the country. All\footnote{DDQPDO, WSDP, ANDO, HNL, WPDF, BGDUF, BMPDO, OPDU, ARDUF, EPAP, ERaeiP, AEUP, UEDF, EJDFF, GSPP, KPC.} the proposals from the parties grouped under the non- detailed LPs have the same stand. All propose the preservation of ‘all languages’ in the country. 

As described above, the LPs from these two parties are overt and dedicated to the promotion of Afan \ili{Oromo} to be the federal state’s WL. As the language they propose is an indigenous one, they are endoglossic LPs. Besides, these overt and promotion LPs of the parties can be said to follow the Language Maintenance model to formulate their LPs for the following reasons. First, the aspiration is to preserve and promote the language and the identity represented by the language through making the language the WL of the FDRE. This provides a better opportunity to the language and speakers of the language to preserve their language and identity as something worth preserving as it opens access to higher domains of state business and economic benefit. In other words, the parties aspire to win symbolic affirmation and identity preservation simultaneously. However, because none of the policies proposed by the political parties have explicitly included anything about the economic, social, cultural and/or political benefits to be gained by making the language they propose the WL of FDRE, the motivation can be argued to be sentiment and authentication. The only points the LPs mention to justify the need for Afan \ili{Oromo} to be the national language is the numerical muscle the language has and as part of the “emancipation” of \ili{Oromo} identity from the “cultural oppression” perceived by the parties. 

The second group is constituted from three parties\footnote{EDP, AEDP and FORUM.} which follow linguistic rationalization model and overtly propose the adoption of an additional WL besides \ili{Amharic}. One of the parties in this group proposes the numerical muscle of the language to be the rationale for selection. The remaining two\footnote{AEDP and EDP.} parties, forward that the selection should be based on “the people’s will” and suitability for science, technology and market exchange.  As all the parties in this group favor the selection and promotion to be from the languages of the nationalities of the country their LPs can be labeled as endoglossic and promoting. On the other hand, from the point of view of the model provided by \citet{Patten2001}, the policies of these political parties always confirm that all the languages and identities of nationalities shall be respected and allowed to flourish. With this their proposals show a \isi{major} feature of the official \isi{multilingualism} model. 

The third group holds two\footnote{EPRDF and GPRDF.} of the detailed and all of the non- detailed parties. These parties propose nothing as a precise answer to the question of the WL of the federal government. Instead, they propose the equal preservation and cultivation of languages and linguistic identities. 

Because the parties in this group have no explicitly stated language proposal about the WL of the federal government, it will be ambiguous to point their sway exactly whether they have agreed with the status quo or not. This makes their LPs regarding the WL of the federal government to be a covert one. However, it is also worth mentioning that they have an overtly stated proposal for an egalitarian preservation of all languages in the country. Besides, since their policies affirm the preservation and cultivation of the languages of the “nations, nationalities and peoples” of \isi{Ethiopia}, it can be said that as a trajectory they lean towards the indigenous languages. Although all the parties in this group do not explicitly promote any particular language, their proposals for the dedication of the state towards full support to the development and preservation of the indigenous languages makes them holders of the promoting and official \isi{multilingualism} models. Yet, there are two\footnote{EPRDF and WPDF.} parties in this group that presented the value of recognition of the indigenous languages from an angle of political participation and democratic rights as follows:

\begin{quote}
Unless the people are using their own language they would not be able to sufficiently participate in the political system through interpreters. Unless citizens use their language to express about their culture and history it is impossible to them to exercise their right to freedom of expression (Press and Audio Visual Directorate/ EPRDF 2002).
\end{quote}

The motivations behind the LP proposals of this group can be said to be both sentimental and instrumental. Those who propose the preservation and recognition of the languages for the sake of identity preservation are the ones with a sentimental motivation and those that propose the recognition and preservation of the languages to be used by the respective people emphasize the political benefits to be reaped from using one’s own language as an instrument for communication. 

The fourth group is constituted from one regional party\footnote{DDPDO.} with no language proposal at all. This party has suggested no proposal regarding the WL of the federal government. Yet, it is still possible to argue that the party hasn’t mentioned the language issue in its political programs may be because, it has a covert LP that is comfortable with the current LP of the country.
 
\subsection{Language in education} 

From the parties that presented relatively expatiated language proposals,\footnote{FORUM, GPRDF, EPRDF, EDP, AEDP, ONC and OLP.} two parties\footnote{ONC and OLP.} explicitly propose one particular language to be used as medium of instruction. These parties overtly state Afan \ili{Oromo} to be used as medium of instruction in schools. However, no instrumental benefit that is to be gained from using the language as medium of instruction in this multilingual country is pointed out by the parties. It can be argued that such LPs covertly marginalize other languages from being used as instructional media within the Oromia ethnic state itself since other ethnic groups also live there. In the proposals from these parties there is no mentioned space in education to other languages of the country. Since the LP is aimed at promoting the language and the linguistic identity the ideology beneath can be supposed to be vernacularization. The motivation behind can be, thus, inferred as authentication of the \ili{Oromo} identity. However, one can contend that there is an assimilationist and hegemonic tendency in this \isi{policy} because it tries to impose a particular language over “all schools” in the vicinity to be ruled by the party. Such a \isi{policy} does not seem to look how imposing one’s language in a multilingual arena would certainly contradict with the economic development and political participation of minority groups. This in turn can result resentment from other ethno-linguistic groups. At the worst case scenario, it could invite a total discordance in a region.

From the non- detailed group, two parties\footnote{KPC and ARDUF.} in the \textbf{too-narrow} sub-group share common feature that they mention a language to be used in education. For instance, the social program of the ARDUF suggests that “\ili{Afar} language shall be included in the school curriculum as a subject.”  The social program from the KPC also suggests that “Kambata language shall be introduced and taught.” In the proposals from these parties, there is nothing stated about the other indigenous languages in the region. In addition, the proposals do not seem to push beyond making the languages they mention to be included in the school curriculum as a subject. They also do not justify their suggestions based on the benefits that the policies will bring to the people. Nor they give any additional \isi{policy} about developing the languages to sustain their use in education.   

The remaining five\footnote{EDP, EPRDF, AEDP, FORUM, and GPRDF.} parties in the detailed group, one\footnote{ANDO.} of the parties from the \textbf{too-narrow} and all the parties in the \textbf{too-broad} subgroups accede with one axiom: mother tongue education at the primary level. Particularly at the primary level education all of them consent that the medium of instruction shall be the mother tongues of the students.  None of the policies of these parties restrict any \isi{indigenous language} from being used as medium of instruction. However, it is worth mentioning that two\footnote{EDP and AEDP.} parties in this group propose to take special consideration towards metropolitan cities in the implementation of mother tongue education. With such policies the two parties have shown special consideration and a peculiar treatment of the metropolitan cities of the country that are practically variegated in ethnic composition. This gives space to the interest of people from different ethnic groups as their children probably pick the ethnic language of their parents as mother tongue. There is also the possibility for the children to assume the \isi{dominant language} of the city as their native language. Besides, in addition to promoting the indigenous languages to be used as media of instruction, such policies also show caution from imposition of one language over another linguistic community without the interest and consent of the latter. 

On the other hand, from the parties that offered relatively detailed LP, three\footnote{EDP, AEDP, and FORUM.} of them which are among parties that proposed the selection of additional language(s) to the status of the federal WL, affirm that, after selection of the additional language the selected language(s) of the federal government shall be taught as (a) subject(s) in the entire schools of the country; and shall also serve as media of instruction depending on the need. They also have conferred the responsibility onto the shoulders of the government to implement the teaching of the selected language(s) in the regular education system.  Unification is the \isi{major} motive at play in such LPs because teaching the federal WL in all schools of the regional states is basically to facilitate communication and social mobility among different language speakers of the federation as one political society. Since these LPs confer the authority to the regional states to select their respective WLs and use them as media of instruction in schools located in the region, there will be no space for the oppressive assimilationist ideology. 

Except for one\footnote{OLP.} party none of the parties included in this study have proposed an LP regarding the language to be used as a medium of instruction beyond the primary level education. On the other hand, except for the above pointed one party and the parties that propose the maintenance of the selected additional WL of the federal government, none of the political parties have articulated anything about the medium of instruction after the primary level education or on second or foreign language choice. It might be conjectured that they are either comfortable with the status quo, i.e. using \ili{English} as medium of instruction, or that they have no alternative \isi{policy} to offer the public in this regard.
 
\section{Conclusions and recommendations} 
Although there exists no political party that overtly proposes the adoption of a monolingual LP, none of the parties can take pride in having a thoughtful, all-encompassing and strategically implementable LP. These policies have failed in addressing burning linguistic issues in \isi{Ethiopia}.  

The failure of the LPs to address the issue of language beyond primary language education is one instance. Notwithstanding the necessity of language of wider communication (LWC), the best communicative medium in education is the best known language by both the learners and the teacher: in most of the cases the mothertongue. However, in \isi{Ethiopia}, \ili{English} -a language barely known by the overwhelming majority- takes the secondary and tertiary level education dominantly as the only official medium of instruction. The reality in Ethiopian schools as well as researches investigating the effectiveness of this \isi{policy}, nevertheless, indicate that \ili{English}, a foreign tongue for both the teachers and the students, has become the “medium of obstruction” in classes (\citealt{Stodart1986} cited in \citealt{BerhanuEtAl2009}). It is ironic to see none of the political parties addressing this very critical issue! The other point where all the political parties have failed to address is the issue of \isi{language endangerment} in the country. According to \citet{UNESCO2015}, 28 Ethiopian languages are endangered. The numerical muscle of most of Ethiopian languages is in hundreds and a few thousands. I argue that none of the current multilingual LPs of the political parties discussed above will benefit these languages. Putting an LP which treats all languages as equals can by no means be beneficial to these endangered languages, no matter how endoglossic it is. Equality should not mean treating the advantaged and the disadvantaged in the same way. Rather, it is supporting the disadvantaged without diminishing the advantaged unfairly.  LPs cannot benefit their respective societies just because they appear to be proponents of \isi{multilingualism}. Without well thought and strategically well designed LPs neither empowerment of indigenous languages nor preservation of endangered languages can happen.      

\newpage 
Since \isi{Ethiopia} is a multilingual state, with all the centuries old squabbling between elites of different ethno-linguistic groups, any societal development plan should always consider language as an issue to be thought thoroughly in planning economic development as well as regional tranquility. Hence, all the political parties should try to work in cooperation with the concerned academia for a well detailed discussion of the pros and cons of LPs.\footnote{At least, the three language model forwarded by \citet{Wodajo2014}, the replacement of \ili{Amharic} by \ili{English} as the sole WL suggested by \citet{Gebreselassie2015}, etc., should all be considered, discussed and evaluated by political groups to come up with better LPs with well-designed implementation strategies.} In addition, beyond the promotion of ethnic identities, parties should look for policies that will have a positive effect on durable regional stability, since imposing one’s language over the other will inevitably ostracize the latter from easily accessing economic development and political participation. 

Furthermore, as there is a sizeable number of parties that are calling for an additional WL of the federal government, all of the people in the political business and stake holders of the language issue, such as researchers and consultants of \isi{policy} makers, should begin to deeply look at the economics of such LP from the perspective of language economics (\citealt{Grin1994,Chiswick2008}). Any proposal regarding the WL of the federal government should not be a result of incessant lobbying from the political entrepreneurs. A thorough cost benefit analysis that carefully considers the ethnographic and economic realities of the country is the only best way to design an effective LP.

\section*{Acknowledgments}

I would like to say the age old adage “thank you” for the ACAL45 conference organizers at KU, particularly Dr. Jason Kandybowicz, for the travel grant I was awarded to participate in ACAL45. My heartfelt gratitude also goes to my colleagues Biniam Jembere and Haile Gezae, both from Addis Ababa University, for their valuable comments and suggestions on this paper.

\section*{Abbreviations}
 

\begin{tabularx}{\textwidth}{lQlQ}
AEDP &    All Ethiopian Democratic Party                                       & GSPP &    Geda System Proponents Party \\
AEUP &    All Ethiopian Unity Party                                            & HNL &    Harari National League \\
ANDO &  Argoba Nation’s Democratic Front                                       & KPC &    Kembata People’s Congress \\
ARDUF &  {Afar} Revolutionary Democratic Unity Front                             & MEISON &  All Ethiopian Socialist Movement \\
BGDUF &  Benishangul {Gumuz} Democratic Front                                    & OLF &    {Oromo} Liberation Front \\
BMPDO &  Bahrworq {Mesmes} People’s Democratic Organization                      & OLP &    {Oromo} Liberation Party \\
DDPDP &  Dubie and Degeni People Democratic Party                              & ONC &    {Oromo} National Congress \\
DDQPDO &  Denta, Dubamo Qinchilchila People’s Democratic Organization          & OPDU &  {Omo} People Democratic Unity \\
EDP &    Ethiopian Democratic Party                                            & PDRE &    People’s Democratic Republic of {Ethiopia} \\
EJDFF &  Ethiopian Justice and Democratic Forces Front                         & PMAC &  Provisionary Military Administrative Council \\
EPAP &    Ethiopian Pan-Africanist Party                                       & SNNPR &  Southern Nations Nationalities and People’s Region \\
EPLF &     Eritrean People Liberation Front                                    & TPLF &    Tigray People’s Liberation Front \\
EPRDF &  Ethiopian People’s Revolutionary Democratic Front                     & UEDF &    United Ethiopian Democratic Forces \\
EPRP &    Ethiopian People’s Revolutionary Party                               & UJDP &    Unity for Justice and Democracy \\
ERAEIP &  Ethiopian Vision Party                                               &   WPDF &  Wolaita People Democratic Front \\
FDRE &    Federal Democratic Republic of {Ethiopia}                              &    WPE &    Workers’ Party of {Ethiopia}  \\
 FORUM &  Forum for Federalist Democratic Unity of {Ethiopia}                    &   WSDP &  Western {Somalia} Democratic Party \\
GPRDF &  Gambella People’s Revolutionary Democratic Front                    & & \\                                                                        
\end{tabularx}                        
\printbibliography[heading=subbibliography,notkeyword=this]

\end{document}