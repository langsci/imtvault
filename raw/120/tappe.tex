\documentclass[output=paper,modfonts]{langscibook} 
\title{What about Southern African story grammar? Promoting language specific macrostructures in educational settings} 
\shorttitlerunninghead{What about Southern African story grammar?}
\author{Heike Tappe\affiliation{University of KwaZulu-Natal, Howard College Campus, Linguistics Programme} 
}
\abstract{The current research  is motivated by the assumption that the “canonical” story grammar \citep{Stein1979} which is widely held to reflect universal story organisation (e.g. \citealt{MandlerEtAl1980}) may not be fully suited to assess the narrative potential of children with primary languages of non-European origin (e.g. \citealt{Souto-Manning2013}). It thus juxtaposes the supposedly universal “canonical” story grammar with the idea of an as yet under-researched Southern African story grammar, which may represent common macro-structural features in Southern African storytelling. 

The analysis of Southern African folktales from Lesotho, South Africa, Malawi and Zambia as well as narrative retellings produced by multilingual children in Malawi, whose dominant languages are Chichewa and English (\citealt{Tappe2013,Hara2014}), and in South Africa, whose dominant languages are isiZulu and English, provides evidence that Southern African narratives systematically deviate from the allegedly universal “canonical” story grammar (\citealt{Anderson1996,Stein1979}). In both Southern African countries currently under investigation the “canonical” story grammar (CSG) is, however, promoted through the language of teaching and learning (English) and determines the assessment instruments used to measure the children’s narrative skills. 
 
Against this background the emancipation of Southern African ways of storytelling might be a crucial step in the fight against literacy in Southern African countries.  }
% \chapterDOI{} %will be filled in at production
\ChapterDOI{10.5281/zenodo.1251714}
\begin{document}
\maketitle
   

% Keywords: Southern African storytelling, story grammar, contrastive rhetoric, emancipatory pedagogy
 



\section{Introduction}\label{sec:tappe:1}


Globally, education is conceptualized as a tool that serves the improvement of social justice and economic advancement. Accordingly, equal access to the full spectrum of the educational system was one \isi{major} concern during protests against colonisation in all Southern African countries as the redistribution in social justice was widely equated with access to the kind of education that many term “quality education” (e.g. \citealt{Tikly2011}). Quality education can be conceptualised as education that enables learners to become “economically productive, healthy, secure and active citizens” (\citealt[9]{Tikly2011}). 

However, in post-colonial Southern Africa “economically productive, healthy, secure and active citizens” are still predominantly of European descent, or have had access to education that is based on European curricula; hence the term “quality education” has become largely synonymous with European curricula taught in a European \textit{language}. 

A considerable number of empirical studies (e.g. \citealt{Orman2008,Nomlomo2010,Nkosana2011,EvansCleghorn2014}) provide evidence for this assessment as they observe that speakers of African languages constantly face the conundrum of cherishing their native language(s) on the one hand but wanting their children to be speakers of a European language (even if it is at the expense of their home language) on the other hand. Consequently, it has become a generally acknowledged common practise for schools in Southern Africa to offer instruction in a locally spoken, \isi{indigenous language} for the first three to five years of schooling before an international language is introduced as a medium of instruction (e.g. \citealt[3]{Bloch2006}). 

Tragically, the negligence of children’s native language(s) leads to great number of almost insurmountable problems and has been identified as one of the key factors in the \isi{literacy} crisis in Southern Africa. \citet[16]{Wagner2000} and many others report that a significant proportion of students in primary schools are illiterate in their native language(s). Moreover they do not necessarily succeed in attaining a sufficiently high competence in the \isi{second language} either. Instead poor proficiency levels in the \isi{second language} are the norm rather than the exception. Given that the \isi{second language} (L2) is usually the language of instruction, low proficiency in L2 has been identified as a primary reason for low throughput and graduation rates at all levels of the educational system (primary, secondary and tertiary) with profound consequences for employment opportunities.

\begin{quote}
It is widely acknowledged that, despite decades of \isi{literacy} campaigns by UNESCO and others and efforts like Education for All (EFA) \citep[21]{Triebel2001}, attempts to ensure that what is often referred to as the ‘spread of \isi{literacy}’ in Africa have failed. \citep[3]{Bloch2006}
\end{quote}

The current research project attempts to contribute to alleviating the \isi{literacy} crisis in Southern Africa by suggesting that the acknowledgment of Southern African ways of storytelling might lead to the emancipation of African learners on a number of different levels and may contribute to an attitudinal change in speakers of African languages with respect to their primary language(s). In the outcome, it is anticipated that it might be possible to redefine the term “quality education” to mean education that integrates Southern African knowledge systems like Southern African storytelling. Such curricula reforms could in turn help to achieve quality education in Southern Africa as they might enable more learners to exit the educational system as economically productive, healthy, secure and active citizens.


\section{Possible differences between the “canonical” story grammar and a Southern African Story Grammar}\label{sec:tappe:2}


The “western” story grammar tradition suggests that a “canonical” story is comprised of a setting and an episode system and that stories basically share the same universal micro- and macro-structural features \citep{MandlerEtAl1980}. On a macro-structural level it is assumed that a “canonical” story encompasses all of the traditional story grammar elements as identified by \citet{Stein1979}.{}\footnote{These elements are summarized by \citet[541]{GriffithEtAl1986}.} \tabref{tab:tappe:1} presents these elements.\\


\begin{table}
\begin{tabularx}{\textwidth}{lX}
\lsptoprule
{\bfseries Story grammar element} & {\bfseries Definition}\\
\midrule
{{Setting}} & {The spatial and/or temporal location where story events take place; the introduction of the main characters, the protagonist, and the spatio-temporal context.}\\
{{Initiating event}} & {An event which typically introduces a state-of-affairs that is ‘out of the ordinary’ for the protagonist, i.e. the occurrence that influences the main character to action.}\\
{{Protagonist’s internal response}} & {An affective or emotive reaction to the initiating event. Indicates the thought(s), feeling(s) of the protagonist in response to the initiation event; may include an interpretation of the event, formulation of a goal and serves to motivate action.}\\
{{Plan}} & {A set of intentions formed in the mind of the person affected by the initiating event. Indicates the intended action of the protagonist (the announcement of the intended action).}\\
{{Attempt}} & {The protagonist’s effort to execute the plan. Indicates the overt actions of the protagonist in pursuit of the goal.}\\
{{Consequences or outcomes}} & {The attainment or non-attainment of the goal, or other events that are the result of the attempt.}\\
{{Resolution or outcome}} & {Any emotional or evaluative response by the protagonist to the preceding chain of events.}\\
\lspbottomrule
\end{tabularx} 
\caption{Story grammar elements according to \citet{Stein1979}}
\label{tab:tappe:1}
\end{table} 

However, an investigation of relevant literature (e.g. \citealt{Canonici1990,Makgamatha1991,Motshwari1998,Obiechina1992}) reveals that Southern African folktales from various language groups differ both in the function and – importantly – terms of their macro- and microstructural features from this “universal” schema. 

Against the background of these insights and a corpus analysis of popular folktales from \isi{Malawi} and South Africa \citep{TappeInPreparation}, we propose in \tabref{tab:tappe:2} a first approximation to a Southern African story grammar. Importantly the elements \textit{setting} and \textit{internal response} of the CSG are demoted in our proposal of a Southern African Story Grammar, while the elements \textit{traditional opening} and \textit{traditional closing} are added.

\begin{table}
\begin{tabularx}{\textwidth}{lX}
\lsptoprule
{\bfseries Story grammar element} & {\bfseries Definition}\\
\midrule
{{Traditional opening}} & {The traditional opening initiates the dialogue between the story teller and the audience and reaffirms the readiness on both sides for the commencement of the story telling event.}\\
{{Initiating event}} & {An event which typically introduces a state-of-affairs that is ‘out of the ordinary’ for the protagonist, i.e. the occurrence that influences the main character to action.}\\
{{Plan}} & {A set of intentions formed in the mind of the person affected by the initiating event. Indicates the intended action of the protagonist (the announcement of the intended action).}\\
{{Attempt}} & {The protagonist’s effort to execute the plan. Indicates the overt actions of the protagonist in pursuit of the goal. The description of the execution of the plan typically involves repetitions and dialogues which reflects the performative character of the storytelling.}\\
{{Consequences or outcomes}} & {The attainment or non-attainment of the goal, or other events that are the result of the attempt.}\\
{{Resolution or outcome}} & {Moral lesson(s) that emanate from the story and which are of general/communal significance.}\\
{{Traditional ending}} & {The story is concluded by a traditional formula which signals the return to the real world and binds any mystical creatures to the story realm.}\\
\lspbottomrule
\end{tabularx} 
\caption{Story grammar elements of a proposed Southern African Story Grammar}
\label{tab:tappe:2}
\end{table} 

The \textit{setting} (i.e. the spatial and/or temporal location where story events take place, the introduction of the main characters, the protagonist, and the spatio-temporal context) seems to be demoted or absent in Southern African storytelling which places the emphasis on communal knowledge and universal applicability of the story. The second demoted element is the internal response of the protagonist to the initiating event: Thought(s) and feeling(s) are not necessarily ascribed to the protagonist by the story teller. This may be a function of the performative aspect of the storytelling where the feelings of the protagonist are enacted rather than verbalised; it may also reflect a taboo.\footnote{In an informal survey, students and colleagues from \isi{Lesotho}, \isi{Malawi}, South Africa, Zambia and Zimbabwe suggested that it may be widely considered inappropriate to verbalise one’s feelings and to ascribe feelings to others. This observation needs further research.} Moreover, the main characters seem to be conceptualised representatives of a particular type of person (e.g. a young man, a girl), rather than individuals, to allow for maximum identification with the protagonist. In line with the demotion of the internal response element, the protagonist’s emotional or evaluative response to the chain of events as it unfolds in the story also seems to be demoted in Southern African storytelling, whereas the moral lesson(s) that may be gained from the story events seem(s) to be promoted. Importantly, the performative character of Southern African storytelling also implies the use of repetitions at various levels of granularity (word, phrase, and paragraph) and the inclusion of dialogues as a stylistic means in the main section of the story, i.e. \textit{the attempt} (\citealt{Canonici1990,Makgamatha1991,Motshwari1998,Obiechina1992,TappeInPreparation}).


\section{The relevance of narratives for literacy development and the consequences of a misalignment between the “canonical” story grammar and language-specific story grammars}\label{sec:tappe:3}


\citet[46]{Bloch1999} highlights that storytelling – just like reading – exposes children to a special form of language which is holistic, rich and complex. In a later publication, she elaborates on this point; “[t]his allows them [the children] to tune into the rhythms and structures of language and broadens their conceptual worlds and their vocabulary to express themselves” \citep[11]{Bloch2006}. Moreover, the repetitive and rhyming play with language that is characteristic for storytelling may aid in developing a child’s phonological awareness, which has been acknowledged as a critical \isi{literacy} skill by a number of researchers (e.g.  \citealt{Goswami1990,DuncanEtAl2013}). 

In addition, \citet[13]{Bloch2006} emphasises the role that narratives play in identity building and in emotional development, which adds further weight to their significance in child development. Referencing authors like \citet[44]{Meek1992} and \citet[10]{Egan2005}; \citeauthor{Bloch2006} points out that through narratives children learn to overcome “egocentric feelings” and develop a communal sense. This aspect of communal identity is a central aspect of traditional Southern African folktales, which are means to preserve and to express “the collective memories of a group, amassed over a long period of time” \citep[65]{Ntuli2011}. 

However, the questions that arise in the current context are: How can the benefits of the “holistic, rich and complex” nature of narratives be unlocked to further \isi{literacy} development if said narratives are in a foreign language and – importantly – follow a foreign macro-structure? How can identity building happen if the narratives that children are exposed to and are required to reproduce in an educational setting do not follow a child’s story telling experiences at home? How “naturalistic and ecologically valid” is the assessment of children’s narratives in Southern African educational systems, which seem to be deeply rooted in the “western” story grammar tradition and which do not seem to widely recognise the narrative traditions of the children’s primary languages?

While many researchers emphasise that it is necessary for teachers to recognise the cultural contexts that learners originate from in order to be able to aid them in appropriate ways (e.g. \citealt{Heath1983,Michaels1991,Bloch2006,Souto-Manning2013}) and to do them justice in assessments, this insight does not seem to have widely filtered through into assessment tool development and curriculum design. On the contrary, research which addresses the question of how a misalignment between the indigenous narratives and “mainstream” narratives may affect \isi{literacy} development in Southern African children is largely outstanding (see, however, e.g. \citealt{Obiechina1992} and \citealt{Bloch2006}).

Research findings, e.g. from the United States of America, clearly indicate a need to rectify this oversight. \citet{Westby2002} analyse the situation of academically underperforming students from an original people’s background in the USA, where more than 30\% of all students are not Anglo-American, whereas the majority of teachers are whites of European descent. The authors emphasise that the indigenous narratives that the children are exposed to at home differ drastically form the narratives they hear in the media and at school. The task of reconciling the often conflicting demands of the two narrative traditions is left to the pupils, much to their disadvantage. \citet[238]{Westby2002} point out that research on the misalignment between narrative practises at school and narrative practises in the pupils’ primary languages is predominantly available on US-American children from “African-American and Hispanic backgrounds (\citealt{Gee1989,Gutierrez-Clellen1995,Hester1996,HyterWestby1996,LabovWaletzky1967,Shuman1986storytelling}), but very little is available on children from other cultural/linguistic backgrounds.” Even though it seems as if an elaborate academic discourse about the cultural differences of students’ schooling experiences and the significance of such differences is readily available, the dominant paradigm appears to be one where an assimilation to the mainstream is globally more highly prized than acknowledging multi-cultural diversity (see, e.g. \citealt{Souto-Manning2013} for an identical assessment of the situation).


\section{The psycho-linguistic “reality” of a Southern African Story Grammar: Findings from case studies conducted in Malawi and South Africa}\label{sec:tappe:4}


In order to provide evidence for the psycho-linguistic “reality” of a Southern African story grammar the author embarked on a research project that currently operates in \isi{Malawi} and South Africa and is envisioned to bring together researchers and to include findings from as many sub-Saharan countries and languages as possible. To date our research has yielded a number of surprising results which suggest that despite the prevalence of modern media, urbanisation and the language hegemony of European languages, Southern African children are still aware of Southern African folktales and that these folktales and the underlying story grammar shape the children’s ways of (re-)telling stories.

The research methodology that we use entails that children in the age range between 10 and 12 answer a language questionnaire to assess their language background and their experiences with different types of narratives (oral, written, in the form of films). Thereafter, we present the children either with a wordless picture book or we show them wordless video clips, which the children narrate first in their one \isi{dominant language}, and after a period of about three weeks, in their other \isi{dominant language} (i.e. currently an African language and \ili{English}).{}\footnote{The children who participate in our project are often multilingual. Most of them do have two dominant languages, however, one of which is an African language and the other is the language of teaching and learning (LOTL). For the Malawian and South African children in our current sample the LOLT is \ili{English}.} The children narrate their stories to an interlocutor who is a speaker of the target language and who (from the children’s perspective) does not know the stimulus material.

In the first phase of the project we use stimulus material that is not culturally “adequate” for children from Southern Africa which reflects a common experience that they have in the educational setting; in the second phase we will use culturally adequate material and compare the findings from both phases.

Currently we have data from retellings of two non-verbal video clips\footnote{“The mole and the rocket” (\citealt{Miler1966}, accessible at: \url{http://www.veoh.com/watch/v74680117WxTgAWxp}) and “The little mole and the radio” (\citealt{Miler1968} accessible at: \url{http://www.veoh.com/watch/v74680119r3m6t4Rd}).} and four pre-recorded audio files which narrate the stories depicted in the video clips in either \ili{Chichewa} or \ili{English}. We also elicited narratives by employing the widely used wordless picture book “Frog where are you” \citep{Mayer1969}. Some of the characters in the selected stories – land animals (mole, deer, gopher) as well as sea creatures (starfish, crab, seahorse) – and some of the objects used (a rocket) are alien to the children who participate in our investigation.

\subsection{Exposure to indigenous story telling}\label{sec:tappe:4.1}



Our questionnaire studies indicate that children in both \isi{Malawi} and in South Africa may still be exposed to African folktales on a regular basis. \citet{Tappe2013} found that 108 of the 127 Malawian children who participated in their investigation and who live in the city of Lilongwe in central \isi{Malawi} heard Malawian folktales on a regular basis. Similarly, 35 out of 41 South African children from urban Durban who are currently participating in the project rated their exposure to isi\ili{Zulu} folktales on a \isi{Likert scale} from 1 (I do not \isi{agree}) to 5 (I totally \isi{agree}) by choosing 5; while another three children chose 4. One child whose primary language is SiSwati chose 3 and two children, whose primary language is isi\ili{Xhosa} chose 1. In addition many of the children volunteered to tell us folktales and were very excited when they heard that we are interested in traditional storytelling. This is in stark contrast to the often heard assumption by parents, teachers and headmasters that children who grow up in the urban centres do not know and are not interested in indigenous folktales.

\subsection{The presence of a South African story grammar in the children’s narratives}\label{sec:tappe:4.2}



Overall, our results indicate that rather than producing stories that adhere to the “western”/ “canonical” story grammar (see \tabref{tab:tappe:1}) (\citealt{Stein1979}), the children produce narratives that reflect features prevalent in Southern African folktales and which are captured in our first approximation to a Southern African story grammar (\tabref{tab:tappe:2}). 

The main differences between the two story grammars are reflected in the narratives that we elicited. As an illustration, \tabref{tab:tappe:3} presents a summary of the findings from  \citet{Tappe2013} and \citet{Hara2014} with respect to narratives of the Malawian children in our investigation. The most striking differences pertain to the “canonical” story grammar elements \textit{setting} and \textit{internal response}. We found that the children did not produce these in the frequency that one would have expected if the children followed the “canonical” story schema.


\begin{table}
\begin{tabularx}{\textwidth}{l@{}C@{}C@{}cc}
\lsptoprule
{ Element} & { Western story grammar} & { African story grammar} & \multicolumn{2}{c}{ { Realised} }\\\cmidrule(lr){4-5}
&  &  & \ili{Chichewa} & \ili{English} \\
\midrule
{ {Opening formula}} & -- & \chckmark & 37\% & { 39\% 

}\\
{ {Setting}} & \chckmark & -- & 23\% & { 17\% 

}\\
{ {Initiating event}} & \chckmark & \chckmark & \chckmark & { \chckmark 

}\\
{ {Protagonist’s internal response}} & \chckmark & -- & 2.49\% & { 5.25\% 

}\\
{ {Plan}} & \chckmark & \chckmark & \chckmark & { \chckmark 

}\\
{ {Attempt}} & \chckmark & \chckmark & \chckmark & { \chckmark 

}\\
{ {Consequences or outcomes}} & \chckmark & \chckmark & \chckmark & { \chckmark 

}\\
{ {Resolution or outcome}} & \chckmark & \chckmark & \chckmark & { \chckmark 

}\\
{ {Closing formula}} & -- & \chckmark & 33\% & { 32\% 

}\\
\lspbottomrule
\end{tabularx} 
\caption{The realization of story grammar elements in the Malawian narratives}
\label{tab:tappe:3}
\end{table} 

In stark contrast to these percentages, \citet{Acker2012}{}\footnote{There are no comparative data from monolingual Malawian children with a primary language of European origin available. We investigated in at a number of schools in Lilongwe whether they had any student enrolment that fits these criteria and received exclusively negative responses.} who investigated South African monolingual \ili{English} or \ili{Afrikaans} speaking children between the ages of five to nine-and-a-half years old, found her participants predominantly inserted settings in their narratives elicited by a wordless picture book \citep[80]{Acker2012}. In particular, 55\% of the five to six year olds, 88\% of the six to seven year olds and 100\% of the eight-and-a-half to nine-and-a-half year olds in her sample started their narratives with a setting. Moreover, \citet[57]{Acker2012} found that 50\% of her five to six year olds, 65\% of her six to seven year olds, and 68\% of her eight-and-a-half to nine-and-a-half year olds described internal responses of the protagonist. 

About 30\% of the Malawian children inserted an opening and/or a closing formula as would be expected in traditional Southern African storytelling. The formulae that the children use are clearly no imitation from the stimulus material. Examples of such formulae are provided in \tabref{tab:tappe:4}.{}\footnote{Compare \citealt[247 and 250]{Hara2014}.}


\begin{table}
\begin{tabularx}{\textwidth}{QQ}
\lsptoprule
{\bfseries \ili{Chichewa} opening formulae} & {\bfseries \ili{Chichewa} closing formulae}\\
\midrule
{\textit{Padangokhala} `once upon a time'} & {\textit{Nkhani yanga yathera pomwepo, nkhani yathera pomwepo} `this is the end of the story'} \\
{\textit{Kalekale} `long time ago'} & {\textit{Ndamaliza} `I have finished'}\\
{\textit{Tsiku lina} `one day'} & {\textit{Basi} `That’s all' or other similar versions.}\\
{\textit{Panali} or \textit{padali} or \textit{kunali} or \textit{kudali} `There was (once)'} & {\textit{Ndi zomwe ndinamvapo} `That’s how I heard it'}\\
\lspbottomrule
\end{tabularx} 
\caption{Opening/closing formulae in the Malawian children’s narratives}
\label{tab:tappe:4}
\end{table} 

In addition to the elements already discussed, the children’s narratives in our corpus appear to generally diverge from the stories presented to them as stimuli in that their narratives contain a greater number of repetitions and different repetitions than are present in the stimulus material. Such repetitions occur both at word, phrase and sentence level. Moreover, the children tend to insert lively dialogues into the stories that are also not part of the stimulus material. Both dialogues and repetitions are important stylistic elements of traditional Southern African story telling; see e.g. \citet[218]{Obiechina1992} for the role of repetitions in Southern African folktales.

Our general observation so far is that children both in \isi{Malawi} and in South Africa deviate from the stimulus material in that they have a tendency to both omit elements that were present in the stimulus material and to add elements that were not present in the stimulus material in a systematic way that reflects features of the proposed Southern African Story Grammar. We also find that the children who participate in our project sometimes try to “assimilate” the stimulus material to their background knowledge: In one narrative, e.g. the protagonist of the “frog where are you” book is placed on a farm and upon arriving in the forest he meets an “antelope” rather than a deer.


\section{The relevance of our findings in an educational setting}\label{sec:tappe:5}


Language related obstacles (extremely low \isi{literacy} rates; insufficiently developed textual production skills; low levels of discourse/text comprehension skills) have been identified to be a \isi{major} hindrance to educational success in the developing world where the “lack of \isi{literacy}” appears to be paralysing the developing economies. Indeed national and international efforts to measure \isi{literacy} in Southern African countries have consistently returned devastatingly bad figures over a number of years and across a number of different countries. \citet[23]{Smith2011} report on the Progress in International Reading Literacy Study (PIRLS) 2006 survey, which found that “less than 22\% of Grade 4 students in South Africa […] achieved the “lowest international benchmark” compared to more than 95\% in North American and Western countries (for more detail see \citealt[69]{MullisEtAl2007}).” 

Successful intervention is thus largely overdue. The integration of African storytelling into Southern African school curricula may be one very powerful intervention. Past and current attempts to promote Southern African languages at school level seem to have mainly focussed on terminology development to transform the Southern African languages into languages of academic learning and writing (which – due to budget constraints – has been a very slow process). However, I want to argue that the mere translation of existing teaching materials into the African languages does not solve the problem that narratives (and maybe other text types, too) may have different macro-structural features in the European and in the African languages. Hence a translated text may still not alleviate the misalignment between the “canonical” story grammar that is currently required in the educational setting and a Southern African story grammar that is acquired during pre-school socialisation.

Tragically, however, we do not seem to acknowledge that the Southern African languages have a particular value in the educational setting, which is in part the legacy of colonial education acts (like the “Bantu education act”{}\footnote{The \isi{major} provision of the Bantu Education Act (Act No. 47 of 1953; later renamed the Black Education Act, 1953) was enforcing racially separated educational facilities.} in South Africa). The outcome seems to be a “rhetoric of deficiency” with dire consequences for the majority of learners in Southern African countries. They seem to enter school feeling doubly undervalued. Sentiments like: “My primary language is neither a language of teaching and learning nor a language of prestige” and “My primary language does not have any value in academia; it still has to be developed” seem to be prevalent among speakers of African languages. However the perception that one’s abilities are valuable has been found to be tightly linked with achievement motivation (e.g. \citealt{Hardré2007,HardréEtAl2007}). Achievement motivation is one of the key ingredients of academic success (e.g. \citealt{Beck2004,Hidi2000,Schunk1996}). 

Hence the acknowledgment of existing indigenous narrative language skills could strengthen the self-worth and the achievement motivation of Southern African children. Narratives have been identified as means of “self-presentation” because “identity work” is “being accomplished by the narrating subject” \citep[100]{Baynham2000}. If we were to strengthen and nourish the narrative skills that Southern African children have in their native languages, we could amplify their academic skills development in two ways: First, indirectly, by strengthening their achievement motivation (as explained above) and second, directly, through promoting their narrative skills themselves. Narrative skills have consistently been found to be closely linked to academic success \citep{O’NeillEtAl2004} and to reading comprehension development (\citealt{Cain2003,Oakhill2012}). Fostering narrative growth in the early years is hence of utmost important to academic achievement and alleviating illiteracy. In conclusion the current paper proposes that bringing stories that the majority of our children may be able to relate to into the classrooms across Southern Africa may be a powerful tool to encourage children to engage with language and with each other across linguistic and cultural divides. Such a reform would help to alleviate learner frustration and might bolster learner motivation which are essential prerequisites for successful learning.

\section*{Acknowledgements}\label{sec:tappe:6}

The researcher is grateful for an ACAL45 travel grant which facilitated her conference attendance in Lawrence, Kansas. Suggestions and comments from the conference audience and two anonymous reviewers aided enhancing the current contribution. Remaining shortcomings are the author’s sole responsibility. I would also like to thank the children who participate in our project as well as my co-workers, Agness Hara, Mandisa Ndlovu, Thandeka Maphumulo, Mbali Mntungwa and Megan Govender. Funding from the South African Research Foundation (NRF) for the funding period 2015--2017 is gratefully acknowledged (CSUR93632: \textit{Southern African text and discourse structures and their relevance for education}).
 
 
\sloppy
\printbibliography[heading=subbibliography,notkeyword=this]

\end{document}