\documentclass[output=paper,modfonts]{langscibook} 
\title{Optional ergativity and information structure in Beria} 
\author{Andrew Wolfe \affiliation{SIL International}\lastand
Tajeldin Abdalla Adam
}
\ChapterDOI{10.5281/zenodo.1251748}

\abstract{Ergativity in Africa is rare; \citet[95--96]{König2008} lists only twelve African languages that have been described as exhibiting ergative phenomena. Even more rarely does \textsc{optional ergative marking} (OEM) appear, in which the use of an ergative marker may depend on information-structural or discourse-pragmatic considerations. \citet[1631]{McGregor2010} lists a sole instance in Africa.

Previous literature on Beria, a Saharan language, describes a system of focus marking that shows ergative alignment, wherein one marker, =\emph{gu}, focuses transitive actors (A) and another, =\emph{di}, focuses either transitive patients (P) or intransitive single arguments (S) (\citealt[151--154]{JakobiCrass2004}; \citealt{Jakobi2006}).

Based on new data from texts and judicious elicitation, we suggest that this =\emph{gu} functions as an optional ergative marker which speakers employ in diverse pragmatic and syntactic contexts, not simply for assigning argument focus to the A term \citep{Lambrecht1994} but also for identifying an A when the P term is in focus; for highlighting brand new A participants \citep{Prince1981} in sentence focus contexts; for disambiguating grammatical roles; and for marking the A of quotative constructions and embedded relative clauses.

We reanalyze =\emph{di} as a specificational copula \citep{Mikkelsen2005}, contrasting with the predicational copula =\emph{i}. Speakers may use either of these copulas in cleft constructions in order to focus constituents other than transitive actors.}
% \keywords{Optional ergative marking, information structure, copulas, Beria}

\begin{document}
\maketitle
 
  
 



\section{Introduction}
\subsection{Optional ergative marking in Africa}\label{sec:wolfe:1.1}


The term \textsc{ergative} describes any linguistic system in which the single or principal argument of an intransitive \isi{verb} (S) patterns with the patient (P) of a transitive \isi{verb} rather than with the actor (A) of the transitive \isi{verb}. See \figref{fig:wolfe:1}.
  
%%please move the includegraphics inside the {figure} environment
%%\includegraphics[width=\textwidth]{wolfeacal2014LSPtemplate-img1.png}


\begin{figure}
\begin{tabular}{ll@{\hspace{3em}}c@{\hspace{3em}}ll}
Nominative  &  Accusative  &  &  Ergative  &  Absolutive\\\\
\multicolumn{1}{c}{\tikz[remember picture,overlay,baseline] \node (WolfeLeftA) {A};} & \multicolumn{1}{c}{\tikz[baseline] \node[draw,circle,thick,inner sep=4pt] (WolfeLeftP) {P};} & transitive & \multicolumn{1}{c}{\tikz[remember picture,overlay,baseline] \node[draw, circle, thick, inner sep=4pt] (WolfeRightA) {A};} &  \multicolumn{1}{r}{\tikz[remember picture,overlay,baseline] \node (WolfeRightP) {P};}\\
\multicolumn{2}{c}{\tikz[remember picture,overlay,baseline] \node (WolfeLeftS) {S};\tikz[remember picture,overlay] \node [rotate=330,inner sep=0pt,draw,ellipse,fit=(WolfeLeftA) (WolfeLeftS)] {}; }  & intransitive & \multicolumn{2}{c}{\tikz[remember picture,overlay,baseline] \node (WolfeRightS) {S}; \tikz[remember picture,overlay] \node [rotate=30,inner sep=0pt,draw,ellipse,fit=(WolfeRightS) (WolfeRightP)] {};}\\\\
\end{tabular}
\caption{Ergative-absolutive vs. nominative-accusative (\citealt{Dixon1994}; \citealt[129--168]{Payne1997})}
\label{fig:wolfe:1}
\end{figure}

In \textsc{optional ergative marking} (or OEM; \citealt{McGregor2009,McGregor2010,McGregorVerstraete2010}), use or non-use of the ergative marker is conditioned by principles of information structure, discourse, or semantics. In the analysis below we follow the definition in \citet[493]{McGregor2009}:

\begin{quote}
Optional ergative case-marking refers to the situation in which the ergative marker may be present or absent from the Agent NP without affecting the grammaticality or interpretation of the \isi{clause} in terms of who is doing what to who. The term ‘optional case-marking’, like ‘free variation’, is potentially misleading, and […] does not mean that the marker is used randomly.
\end{quote}

Only rarely does ergativity appear in descriptions of African languages. \citet[95--96]{König2008} lists twelve languages with potential ergative phenomena, clustered in four families: West Nilotic, Kordofanian, Mande, and Chadic. \citet[1631]{McGregor2010} offers only one of the West Nilotic languages from König’s list, Shilluk, as the lone example of OEM from the over 2,000 languages in Africa.


 

\subsection{Beria}
Using new data from texts and elicitation, we propose adding \ili{Beria}, a Saharan language of eastern Chad and western \isi{Darfur}, to this short list of African languages known to exhibit OEM. 

In \citegen{JakobiCrass2004}  grammar of the \ili{Kube} dialect of \ili{Beria} (henceforth J\&C, 2004: 151--154; cf. \citealt{Jakobi2006}), they present two enclitic \isi{focus} markers, =\textit{gu} and =\textit{di}. They argue that these two \isi{focus} markers follow an ergative distribution pattern: speakers employ \textit{=gu} to mark focused A terms and \textit{=di} to mark focused P and S terms. A third focus-marking strategy they characterize as a cleft construction, as it uses the copula \textit{=i}. All three of these, =\textit{gu}, =\textit{di} and =\textit{i}, encliticize to lexical NPs or free pronominals, though most participant reference tracking in natural discourse with unmarked pragmatics appears only in \ili{Beria}’s verbal cross-reference system, wherein suffixes indicate the person and number of A (agent / actor) arguments and prefixes indicate the person and number of P (patient / object) arguments. 

The present study seeks to extend and refine Jakobi \& Crass’s analysis of =\textit{gu}, =\textit{di} and =\textit{i} by examining contextualized discourse data with the tools of information structure (especially \citealt{Prince1981} and \citealt{Lambrecht1994}). In a preliminary review of two texts from a local news broadcast genre, we describe a previously unrecognized usage of =\textit{gu} in contexts with \textit{sentence focus} \citep[233--235]{Lambrecht1994}, i.e., where the entire sentence consists of new information. Specifically, we propose that in our data, this optional use of =\textit{gu} only appears with \textit{brand new} transitive-agent referents \citep[233--237]{Prince1981}. We also confirm that =\textit{gu} may also occur in contexts with \textit{argument focus} \citep[228--233]{Lambrecht1994} on A.

In addition to cases where the A term features more prominently in its context, we delineate three specific constructions that feature a \textit{less} prominent A argument, also marked with =\textit{gu}. All of these occur in contexts where the speaker accords higher prominence to the P argument. This prominence on the P may be pragmatic, as with contexts where P is in argument \isi{focus} or serves as the subject of a pseudo-\isi{passive}, or it may be syntactic, as the head of a transitive \isi{relative clause}. We also identify a quotative use of =\textit{gu}. 

Turning from =\textit{gu}, we reanalyze =\textit{di} and =\textit{i} as two different copulas with distinct semantics, and recast each of their argument-\isi{focus} usages as variations on the same cleft construction. Copular =\textit{i}, which J\&C consider a “copula of identification,” we rebrand as a \textit{predicational} \textit{copula} \citep{Mikkelsen2005}, and we call =\textit{di} a \textit{specificational} \textit{copula}. In one documented difference between the two cleft constructions, OEM marker \textit{=gu} appears on the A term in =\textit{di} clefts but not in =\textit{i} clefts. We suggest that this difference owes to the higher transitivity of the =\textit{di} cleft, where the \isi{clause} satisfies \citegen[252]{HopperThompson1980} high-transitivity criterion of the \textit{individuation of O}.


\section{Overview of key concepts and text sources}\label{sec:wolfe:2}
\subsection{Information structure}\label{sec:wolfe:2.1}
Space considerations preclude an extensive review of the literature on information structure and related subdisciplines, where terminological choices and definitions vary widely. For present purposes it is helpful to highlight \citegen{Lambrecht1994} definitions of \isi{focus} as well as \citegen{Prince1981} taxonomy of new information, both of which figure into the analysis below. 

\citet[211]{Lambrecht1994} offers a shorthand definition of \textit{focus} as that which is “unpredictable” or “non-recoverable” in any utterance. As a corollary, every sentence has some element in \isi{focus}. He delineates three domains of \isi{focus}: \isi{predicate} \isi{focus}, argument \isi{focus}, and sentence \isi{focus}.

\begin{quote}
The unmarked subject-\isi{predicate} (topic-comment) sentence type […], in which the \isi{predicate} is in the \isi{focus} and in which the subject (plus any other topical elements) is in the presupposition, will be said to have \textsc{predicate-\isi{focus} structure}; the identificational type […] in which the \isi{focus} identifies the missing argument in a presupposed open proposition will be said to have \textsc{argument-\isi{focus} structure}; and the event-reporting or presentational sentence type, in which the \isi{focus} extends over both the subject and the \isi{predicate} (minus any topical non-subject elements), will be said to have \textsc{sentence-\isi{focus} structure}. \citep[222]{Lambrecht1994}.
\end{quote}

\citet{Prince1981} sketches out a taxonomy of discourse referents in terms of newness and givenness. New discourse referents may be either \textsc{brand new}, where the speaker asks the hearer to create a new entity in her mental representation; or \textsc{unused}, where the speaker invokes a referent new to the discourse but assumed to be familiar to the hearer. Given discourse referents may be \textsc{textually evoked}, referred to previously in the discourse itself; or \textsc{situationally evoked}, which Prince primarily uses for the discourse participants themselves (first and second person). Between these new and given categories lie \textsc{inferable} referents, whose existence the hearer is able to deduce by logical or cultural implication, as the presence of a “bus” in the discourse implies the existence of a “driver.”


\subsection{Text sources}\label{sec:wolfe:2.2}
New data analyzed below come from publicly available Radio Dabanga newscasts from March 18, 2013 and May 1, 2013, downloaded from the Radio Dabanga website (\url{http://www.radiodabanga.org}) the day following their broadcast. Based in the Netherlands, Radio Dabanga (henceforth RD) broadcasts daily Darfur-related news in \ili{Arabic} and various languages of \isi{Darfur}. The news anchor for the \ili{Beria} portion of the broadcast, Tayiba Abdelkarim Abdul, is a well-respected native speaker of the \ili{Kube} dialect. For the purposes of the broadcast, she translates from \ili{Arabic}. Having been given the text well in advance of her studio time, she considers beforehand how best to render the meaning of each news piece as a whole into \ili{Beria}. The final recording undergoes a light editing process before broadcast. 

In general, the paragraph-by-paragraph translation method and the linguistic reputation of the translator notwithstanding, some interference from the source language cannot be ruled out. For present purposes, these concerns are mitigated by two factors: \ili{Arabic} – even Darfuri \ili{Arabic} – has no ergative case marking; and the second author of this study, as a native \ili{Kube} \ili{Beria} speaker, can critically evaluate the naturalness of the data in what concerns the conclusions below. Nonetheless, future studies will need to confirm the present findings with a more extensive data set that has no foreign-language origin. Accordingly, we qualify our conclusions below as preliminary while we work to assemble a larger, more diverse corpus.

Elicited data from the present study’s second author appear marked with “TAA” below. In our discussions, elicitation never involved direct translation from \ili{English}. Rather, we would discuss a pragmatic communicative context and ask what a native \ili{Beria} speaker might say. Additionally, while recognizing the limitations of ungrammaticality judgments as linguistic evidence, in limited contexts below we deemed evaluations of infelicity important enough to the argument to include them.


\section{Uses of =\textit{gu}}\label{sec:wolfe:3}
\subsection{A-argument focus constructions with =\textit{gu}}\label{sec:wolfe:3.1}
Based on data like that given in \REF{ex:wolfe:1}\footnote{We have reproduced J\&C’s examples as-is, with translation when in \ili{French}. Note that they wrote before a working \isi{orthography} was in place, so their examples use IPA transcription. Abbreviations from their work that do not follow the Leipzig Glossing Rules are listed in the Abbreviations section.}, J\&C argue that =\textit{gu} and =\textit{di} are \isi{focus} markers with \isi{complementary distribution}, whose usage follows an ergative-absolutive alignment (2004: 151--154):

\ea\label{ex:wolfe:1}
\ea
\langinfo{}{}{Focus markers with ergative alignment \citealt{Jakobi2006}}\\
     \gll bágʊ=ɔgɔ=gʊ Ø:kú-gú-r̩-í. \textup{focused A:} =\textit{gu}\\
     wife-\textsc{poss}.\textsc{3sg}=\textsc{foc}\textsc{\textsubscript{erg}} \textsc{oj}:3:\textsc{pfv}:3-call-\textsc{sj}:3-\textsc{pfv}\\
\glt ‘It’s his wife who called him.’ \citep[136]{Jakobi2006}
\ex
     \gll náá=dɪ nɛ-gɛr-g-ɪ. \textup{focused P:} =\textit{di}\\
     \textsc{pp}:\textsc{2sg}=\textsc{foc}\textsc{\textsubscript{abs}} \textsc{oj}:2-look.for-\textsc{sj}:\textsc{1sg}-\textsc{ipv}\\
\glt ‘It’s you I’m looking for.’ \citep[137]{Jakobi2006}
\ex
     \gll sʊltǎn=dɪ Ø:nɪ-Ø-ɪ. \textup{focused S:} =\textit{di}\\
     sultan=\textsc{foc}\textsc{\textsubscript{abs}} \textsc{oj}:3:die-\textsc{sj}:3-\textsc{pfv}\\
\glt ‘It’s the sultan who has died.’ \citep[137]{Jakobi2006}
\z
\z

The enclitic =\textit{gu} only appears with transitive actors. When speakers want to \isi{focus} patients or intransitive arguments, they invariably cliticize =\textit{di} or =\textit{i} to the focused \isi{noun} phrase. In light of \ili{Beria}’s split-S verbal system (\citealt{Mithun1991}; \citealt[Chapter 4]{Dixon1994}; see \citealt{Jakobi2011} for detailed analysis of the \ili{Beria} facts), it bears noting that it is ungrammatical to use =\textit{gu} with the S of even active intransitive verbs such as ‘run’ or ‘leave,’ which \ili{Beria} codes with the A suffixes:\footnote{New \ili{Beria} examples are written in the working \isi{orthography} approved by local administration \citep{ADESK2006}. High, [+ATR] vowels are represented with a circumflex diacritic: \textit{û}, \textit{î}. Mid, [+ATR] vowels remain unmarked, predictable from the presence of [+ATR] high vowels within the harmonic domain. Lexical tone is not written. Pluralization of both nouns and verbs uses the simplified orthographic convention of doubling the final letter of the word, capturing what is in actuality a complex system of tonal patterns. Imperfective verbs end in –\textit{ì} and \isi{perfective} verbs in –\textit{í}, reducing grammatical tone phenomena to an iconic visual form. Both rhotics are written \textit{r}. Note that we have maintained a single representation for enclitics like =\textit{gu} even though they harmonize for ATR with their host.}

\ea\label{ex:wolfe:2}
\gll *Ai=gu suk=tu hiri-g-í.\\
      1\textsc{s}=\textsc{erg} market=\textsc{dat} run-\textsc{1sg.a-pfv}\\
\glt ‘It was I who ran to the market.’ (TAA)
\z

cf. \textit{Ai-di suk-tu hirigí.}

\ea\label{ex:wolfe:3}
\gll *Ai=gu sur-g-í.\\
      1\textsc{s}=\textsc{erg} exit-\textsc{1sg}.\textsc{a}-\textsc{pfv}\\
\glt ‘It was I who left.’ (TAA) 
\glt cf. \textit{Ai-di surgí.}
\z

Because the label of ergative properly applies at the level of the particular grammatical construction, not of whole languages \citep[132--171]{Croft2001}, it is unsurprising to find languages like \ili{Beria} with both split-S and ergative alignments in operation in different components of the grammar. Note also that since =\textit{gu} does not appear with S arguments, it indeed follows an ergative pattern and not a \textit{marked nominative} one \citep[138--203]{König2008}. 

In fact, not even all clauses with two core arguments allow =\textit{gu} on the more \isi{agentive} argument. Certain low-transitive (\citealt{HopperThompson1980}) bivalent verbs such as ‘have,’ ‘learn’ and ‘know’ do not accept =\textit{gu} but may use =\textit{di} instead to \isi{focus} the more \isi{agentive} argument:

\ea\label{ex:wolfe:4}
\gll *Ber=gu arabie kidí. A arab tigo.\\
     3\textsc{s}=\textsc{erg} car 3.\textsc{a}:have:\textsc{pfv}:\textsc{sg} 1\textsc{s} car have:1\textsc{s}.A:\textsc{neg}\\
\glt ‘It is he who has a car. I don’t have a car.’ (TAA) 


\glt cf. \textit{Ber-}\textbf{\textit{di}} \textit{arabie kidí. Ai arabie tigo.}
\z

\ea\label{ex:wolfe:5}
\gll ɟàmâl=\textbf{dɪ} àrmá Ø:áwáá-r̩-ɪ.\\
     Jamal=\textsc{foc}\textsc{\textsubscript{abs}} \ili{Arabic} \textsc{oj}:3:learn-\textsc{sj}:3-\textsc{pfv}\\
\glt ‘It’s Jamal who has learnt  Arabic.’ \citep[139]{Jakobi2006}
\z

\ea\label{ex:wolfe:6}
\gll *Hawa=gu tir=egî. ege=giní. Maha ege=gino.\\
     Hawa=\textsc{erg} name=\textsc{1sg}.\textsc{poss} know=3\textsc{a}.\textsc{pfv}:\textsc{aux} Maha know=3\textsc{a}.\textsc{pfv}:\textsc{aux}:\textsc{neg}\\
\glt ‘It was Hawa who learned (lit., ‘knew’) my name. Maha didn’t learn it.’ (TAA)\\
 cf. \textit{Hawa-\textbf{di}  tir-egî ege-giní. Maha ege-gino.}
\z

Turning to information structure, among Lambrecht’s three \isi{focus} categories (see \sectref{sec:wolfe:2.1}), J\&C have already established the use of =\textit{gu} for \textit{argument focus}, with \isi{focus} on the A term.\footnote{They do not use Lambrecht’s terms, but argument \isi{focus} is clearly indicated in their work by cleft translations (\citealt[151--153, examples 232--236 and 248]{JakobiCrass2004}; and in \citealt[136, example 19]{Jakobi2006}). In the other four examples given of =\textit{gu} (2004: 152, example 237; 2006: 137--138, examples 20, 21, and 26), it is unclear, based on the example and translation given, what the nature of the “\isi{focus}” is. Some of these are treated below.} Within this usage, argument \isi{focus} on the A can accomplish at least two distinct purposes: asserting the \textit{identity} of the A term, or asserting its \textit{role}. This becomes clear in negative assertions as in (\ref{ex:wolfe:7}-\ref{ex:wolfe:8}). In \REF{ex:wolfe:7}, the identity of the agent is corrected, whereas in \REF{ex:wolfe:8} it is the \isi{clause}’s role assignments that are contradicted. Both contexts use =\textit{gu.}

\ea\label{ex:wolfe:7}
\gll Bur=do\textbf{=gu} kana tene sai=gi-n-o. O gie=i.\\
     boy=that\textbf{=}\textsc{\textbf{erg}}\textbf{} \textsc{neg}.\textsc{pfv} girl hit=3.\textsc{a}.\textsc{pfv}-\textsc{aux}-\textsc{neg} person other=\textsc{cop}\textsc{\textsubscript{pr}}\\
\glt ‘It wasn’t \textit{that boy} that hit the girl. It was someone else.’ (TAA)
\z

\ea\label{ex:wolfe:8}
\gll Bur\textbf{=gu} kana tene sai=gi-n-o=ru, tene\textbf{=gu}.\\
     boy\textbf{=}\textsc{\textbf{erg}}\textbf{} \textsc{neg}.\textsc{pfv} girl hit=3.\textsc{a}.\textsc{pfv}-\textsc{aux}-\textsc{neg}=\textsc{conj} girl\textbf{=}\textsc{\textbf{erg}}\\
\glt ‘It wasn’t that the \textit{boy} that hit the \textit{girl}. The \textit{girl} hit the \textit{boy}.’ (TAA)
\z

\subsection{Sentence focus, brand new A construction}\label{sec:wolfe:3.2}
Not all uses of =\textit{gu} entail argument \isi{focus}, however. In both RD texts, the speaker uses =\textit{gu} in contexts such as \REF{ex:wolfe:9} where the remainder of the sentence is being asserted as new, not presupposed. 


\ea\label{ex:wolfe:9}
\gll barûgûî yom\_al\_ahad=tu, oo toûra kuni\textbf{=gu}, arabie tijari ni genî Gireda=re ji-e, je Jokhana=ru si-r-ì=gi, toûû ki-si-n-e=ge, bodo ajas=te oo wetti ku-nu-e, ere oo sogodî gie noko=ru hiara=gine=ii giníí.\\
A.few.days.ago Sunday=on person:\textsc{pl} armed \textsc{indf}.\textsc{pl}\textbf{=}\textsc{\textbf{erg}} vehicle commercial \textsc{indf} village Gireda=\textsc{abl} \textsc{cop}\textsc{\textsubscript{loc}}\textsc{.3-}\textsc{cvb} \textsc{prog} Jokhana=\textsc{dat} go.\textsc{ipfv}\textsc{.3-3}\textsc{a}-\textsc{ipfv}=\textsc{rel} gun:\textsc{pl} \textsc{3.}\textsc{a}.\textsc{pfv}-shoot-\textsc{3.}\textsc{a}-\textsc{cvb}=\textsc{rsn} that reason=\textsc{abl} person:\textsc{pl} three \textsc{3}\textsc{a}.\textsc{pfv}-kill:\textsc{p}.\textsc{pl}-\textsc{cvb} again person:\textsc{pl} ten and one=\textsc{dat} wounded\textsc{=3}\textsc{a}.\textsc{pfv}:\textsc{aux}:\textsc{cvb}=\textsc{cop}\textsc{\textsubscript{pr}}:\textsc{pl} \textsc{3}\textsc{a}.\textsc{pfv}:say: \textsc{pl}\\
\glt ‘Last Sunday some armed people(=\textsc{erg}) opened fire on a commercial vehicle that was from the town of Gireda, going to Jokhana. Because of that they killed three people, and 11 people have been injured, they said.’ [RD, 3/18/14, 5.1]
\z

The news anchor does not expect the hearer to know there were people who opened fire on a commercial vehicle on the road from Gireda to Jokhana, nor does she merely assert that it was certain unnamed armed men who committed this deed; this is a news broadcast, and the entire utterance is unpredictable. 

In the immediately following sentence, she restates and elaborates on the news just announced. 

\ea\label{ex:wolfe:10}
\gll oo Ø-nuíí=gi araa=gi ni Radio Dabanga kerigi er k-î-í=gi oo toûra malisha hakuma=ru tabi=gi dîî=ra hirdee=ra k-ori-e=gine=re arabie tijari yom\_al\_ahad tûî=gi gardi genî Jokhana=re ji-e=re je Gireda=r ke-si-r-ì, toûû ki-si-n-e=ge, bodo ajas=te oo wetti ku-nu-e, ere oo sogodî gie noko=ru hiara=giníí giníí.\\
     person:\textsc{pl} 3.\textsc{p}-die:\textsc{p}.\textsc{pl}:\textsc{pfv}=these person:\textsc{pl} \textsc{indf} Radio Dabanga within \textsc{rel}.\textsc{pr} \textsc{3.}\textsc{a}.\textsc{pfv}-say-\textsc{pfv}=\textsc{rel} person:\textsc{pl} armed militia government=\textsc{dat} affiliated=\textsc{rel} camel:\textsc{pl}=\textsc{conj} horse:\textsc{pl}=\textsc{conj} 3.\textsc{a}.\textsc{pfv}-mount-\textsc{cvb}=\textsc{3}\textsc{a}.\textsc{pfv}:\textsc{aux}:\textsc{cvb}=\textsc{nfin} vehicle commercial Sunday evening=this road village Jokhana=\textsc{abl} \textsc{cop}\textsc{\textsubscript{loc}}\textsc{.3-}\textsc{cvb}=\textsc{nfin} \textsc{prog} Gireda=\textsc{dat} \textsc{ven}-go.\textsc{ipfv}\textsc{.3-3}\textsc{a}-\textsc{ipfv} gun:\textsc{pl} \textsc{3.}\textsc{a}.\textsc{pfv}-shoot-\textsc{3.}\textsc{a}-\textsc{cvb}=\textsc{rsn} that reason=\textsc{abl} person:\textsc{pl} three \textsc{3}\textsc{a}.\textsc{pfv}-kill:\textsc{p}.\textsc{pl}-\textsc{cvb} again person:\textsc{pl} ten and one=\textsc{dat} wounded\textsc{=3}\textsc{a}.\textsc{pfv}:\textsc{aux}:\textsc{pl} \textsc{3}\textsc{a}.\textsc{pfv}:say: \textsc{pl}\\
\glt ‘One of the relatives of the deceased told Radio Dabanga that the armed, government-affiliated militia people mounted on camels and horses and opened fire on a commercial vehicle on Sunday evening on the road coming from the village of Jokhana to Gireda. Because of that they killed three people and 11 people were injured, they said.’ [3/18, 5.2]
\z

This example well illustrates the optionality of the ergative marking as defined by McGregor and cited in \sectref{sec:wolfe:1.1}: a “situation in which the ergative marker may be present or absent from the Agent NP without affecting the grammaticality or interpretation of the \isi{clause} in terms of who is doing what to who” \citep[493]{McGregor2009}. On the level of grammatical relations – and, in fact, even of sentence \isi{focus} vs. argument \isi{focus} – nothing changes between \REF{ex:wolfe:9} and \REF{ex:wolfe:10}, yet =\textit{gu} appears in \REF{ex:wolfe:9} and not in \REF{ex:wolfe:10}. In both, ‘armed people’ (\textit{oo toûra}) ‘shot’ (\textit{toûû kisine}) a ‘commercial vehicle’ (\textit{arabie tijari}) and ‘killed’ (\textit{kunue}) people. In neither sentence is any one argument in \isi{focus}. Everything that changes is irrelevant to the syntax of core arguments: the speaker adds the source of the information (‘one of the relatives of the deceased’), expands on her description of the transitive agent (characterizing them as ‘government-affiliated militia’), notes attendant action (transitive: they ‘mounted camels and horses’), and shifts some minor details (mentioning the ‘road’ and changing the aspect of ‘have been injured’ to ‘were injured’). 

No salient grammatical details change, but on a second repetition the once \textsc{brand new} identity and role of the transitive agent have become \textsc{textually } \citep[233--237]{Prince1981}. All four of the non-quotative tokens of sentence-\isi{focus} =\textit{gu} occur with \textit{brand new} transitive-agent participants.

\subsection{Constructions with backgrounded A=\textit{gu}, foregrounded P}\label{sec:wolfe:3.3}

Another distinct pragmatically marked construction with =\textit{gu} appears in one token in the RD texts, which fits the pattern for neither A-argument \isi{focus} nor sentence \isi{focus} constructions:

\ea\label{ex:wolfe:11}
\gll genîa ha Sharq al Jabal hie=ru Yara=ra Nimra=ra Dalma=ra genîa=kî=ra, oo toûra malishat hakuma=ru tabi begî\textbf{=gu} a-we gami=giníí.\\
     villages Mt. Sharq al Jebel direction=\textsc{loc} Yara=\textsc{conj} Nimra=\textsc{conj} Dalma=\textsc{conj} villages=these=\textsc{conj} person:\textsc{pl} armed militia government=\textsc{dat} affiliated this\textbf{=}\textsc{\textbf{erg}} go.\textsc{3}\textsc{a}.\textsc{pl}-\textsc{cvb} ambush=\textsc{3}\textsc{a}.\textsc{pfv}:\textsc{aux}:\textsc{pl}\\
\glt ‘The villages around Mt. Sharq al Jebel – Yara, Nimra, and Dalma – these villages too, these armed government-affiliated militiamen(=\textsc{erg}) went and ambushed.’ [3/18, 3.1]
\z

Here the A is not new material but \textit{textually evoked}, as shown by the deictic \textit{begî} ‘this, above.’ In context this sentence immediately follows the description of another ambush by these militia, in the ‘villages that lie to the northeast of Kutum’ (line 2.1). The speaker marks continuity with the previous episode here not only through anaphoric participant reference (\textit{begî}) but also with the lexical repetition of \textit{gami-giníí} ‘they ambushed’ and \textit{genîa} ‘villages.’ In other words, everything here is textually evoked except the names of the ambushed villages and the occurrence of a new incident in ‘these villages, too’ \textit{genia-kî-ra}. 

The new village names added to the list of ‘villages ambushed by the Janjaweed’ are the unrecoverable content in sentence \REF{ex:wolfe:11}. The speaker fronts the brand new P argument and demotes the textually evoked A argument in a sort of pseudo-\isi{passive} construction. As noted by Zakaria Fadoul \citet[80]{Khidir2005}, \ili{Beria} has no exact equivalent to the \ili{French} \isi{passive} construction because the presence of \isi{passive} verbal morphology in \ili{Beria} requires the total omission of an \isi{agentive} lexical \isi{noun} phrase. To express a demoted agent, the verbal morphology must remain the same but the constituent order changes to PAV and A gets marked with =\textit{gu}.

Sentence \REF{ex:wolfe:11} also exemplifies a phenomenon known to the literature as \textsc{ergative hopping }(\citealt[155]{Haviland1979}, cited in \citealt[1657]{Rumsey2010}) whereby an ergative-marked A term may simultaneously function as the S argument of an intransitive \isi{verb} (here ‘went and’) that intervenes between the ergative-marked \isi{noun} phrase and the bivalent \isi{verb} that licenses it. Three tokens of =\textit{gu} exhibit ergative hopping.

One of J\&C’s examples also falls into the pattern where P is promoted and constituent order is reversed:

\ea\label{ex:wolfe:12}
\gll bɪɛ kí=dî ábā égí\textbf{=gú} Ø:sí-é-r̩-î.\\
    house this=\textsc{foc}\textsc{\textsubscript{abs}} father my\textbf{=}\textsc{\textbf{foc}}\textsc{\textbf{\textsubscript{erg}}} \textsc{oj}:3:build-\textsc{pfv}:3-\textsc{sj}:3-\textsc{pfv}\\
\glt ‘It’s this house that my father has built.’ \citep[138]{Jakobi2006}
\z

Although the situational context for utterance \REF{ex:wolfe:12} is not given, it seems likely that the A term here is part of a presupposition ‘my father built a house’ or ‘my father built something,’ putting the A ‘my father’ in the pragmatic background (i.e., not in the \isi{focus} domain) and focusing on the clefted P term ‘this house.’

\subsection{Quotative construction}\label{sec:wolfe:3.4}
Two separate quotative constructions account for the remaining tokens of =\textit{gu} in the RD texts. In one of them, which is used in our data exclusively for \textit{direct} reported speech, no \isi{verb} of saying appears between the quoted agent and the quotation itself. The speaker uses =\textit{gu} to signal to the hearer that she is transitioning from the A argument to a direct quotation, as in \REF{ex:wolfe:13}:


\ea\label{ex:wolfe:13}
\gll …genî=gi kerigi oo jii=\textbf{gu,} “Ta-rdasin-e baa=ru ere je korekk=tu ba=gi je oû-d-í,” gi-n-e kîíí.\\
     village=this within person:\textsc{pl} \textsc{cop}\textsc{\textsubscript{loc}}\textsc{.3:}\textsc{pl}=\textsc{\textbf{erg}} \textsc{1}\textsc{pl.p}-come.together-\textsc{cvb} hand:\textsc{pl}=\textsc{ins} again \textsc{prog} spade:\textsc{pl}=\textsc{ins} mine=this \textsc{prog} dig-\textsc{1pl}.\textsc{a}-\textsc{pfv} 3.\textsc{a}.\textsc{pfv}-say-\textsc{cvb} 3.\textsc{a}.\textsc{pfv}-say-\textsc{pfv}:\textsc{pl}\\
\glt ‘… the people who are in this village said, “We are coming together and we are digging this mine out with our hands, and again with spades.”’ [5/1, 4.3]
\z

This “A – direct Quotation – quotative Verb” pattern satisfies \ili{Beria}’s default APV \isi{word order}, with the quotation serving as the P. Quotative OEM constructions are well documented for at least two other languages \citep{Rumsey2010}: Ku \ili{Waru} (Trans-New \isi{Guinea}) and \ili{Bunuba} (non-Pama-Nyungan). 

The existence of a quotative =\textit{gu} construction sheds light on one of J\&C’s examples as well:

\ea\label{ex:wolfe:14}
\gll bágâ tɔ-gʊ-ɛ ɟɪɪ gɪnɛ, bɛr\textbf{=gʊ} tɛbɪ gɪ-n-ɪ.\\
     millet:\textsc{pl} \textsc{rv}-scatter-3:\textsc{sj}:\textsc{cvb}\textsc{\textsubscript{1}} \textsc{lc}:3:\textsc{aff}:\textsc{p} \textsc{sub} \textsc{pp}:3\textbf{=}\textsc{\textbf{foc}}\textsc{\textbf{\textsubscript{1}}} take:\textsc{imp} \textsc{prf}:3-say-3:\textsc{sj}:\textsc{aff}:\textsc{prf}\\
\glt ‘Since the millet had been scattered, he said: “Take it!”’ (\citealt[152]{JakobiCrass2004})
\z

Without knowing the communicative context we cannot know for sure, but a direct quotative =\textit{gu} reading here seems to fit more naturally than an argument \isi{focus} reading.

The quotative construction appears in information-structurally diverse contexts: In \REF{ex:wolfe:13} and the other RD token, the quoted speaker is a brand new referent and the utterance displays sentence \isi{focus} whereas in \REF{ex:wolfe:14}, the quoted speaker is presumably textually evoked (by virtue of its pronominal reference) and the utterance thus has \isi{predicate} \isi{focus}.

\subsection{Subordinate constructions with =\textit{gu}}\label{sec:wolfe:3.5}
In the other quotative construction the quotation is introduced with a \isi{relative clause} ‘what A said.’ In this \isi{relative clause}, the A argument appears marked with =\textit{gu}, as in (\ref{ex:wolfe:15}; quotation omitted for space):

\ea\label{ex:wolfe:15}
\gll o      wakil\_amin\_al-am       iga 	   Umam\_al\_Muttahida  hifz\_al\_salaam kî tir-ogo                    Hervé Ladsous gine\_îrì, [o=kî=\textbf{gu} er\textbf{} k-î-í]=gi, [“…”] gine\_kîí.\\
     person Undersecretary.General mission United.Nations     peacekeeping   \textsc{gen} name-\textsc{3sg}.\textsc{poss} Hervé Ladsous is.called person=this=\textsc{\textbf{erg}} \textsc{rel}.\textsc{pr} \textsc{3}\textsc{a}.\textsc{pfv}-say-\textsc{pfv}.\textsc{sg}=\textsc{rel} [“…”] he.said\\
\glt ‘The UN Under-Secretary General for Peacekeeping, whose name is Hervé Ladsous, what this person said: [“”], he said.’ [5/1, 5.1]
\z
% \todo{please check example alignment}

In both such tokens in our data, a brand new participant enjoys an elaborated introduction from the speaker, which is topicalized at the beginning of the utterance. The speaker then proceeds to preface the participant’s quoted speech act with the short \isi{relative clause} ‘what he said.’ This \isi{relative clause} uses a reduced \isi{noun} phrase to refer to the quoted participant: ‘this person’ \textit{o-kî} in \REF{ex:wolfe:16}, and in the other example even more simply ‘he’ \textit{ber}. Such a reduced form of reference signals that the quoted participant has become an \textit{evoked}, less prominent entity; critically it allows the speaker to pivot from the communicative purpose of introducing the participant to that of telling the hearer what it was that the participant said.

The ergative marking of A constituents embedded within relative clauses extends beyond quotative uses. In \REF{ex:wolfe:16}, \textit{tene} ‘girl’ functions as both the patient of the \isi{verb} \textit{kidigarí} ‘loves’ in the embedded \isi{relative clause} and as the single argument of the \isi{verb} \textit{karí} ‘came’ in the matrix \isi{clause}. As the pivot, or shared argument between the two clauses, the P argument enjoys greater syntactic prominence than the A term.

\ea\label{ex:wolfe:16}
\gll Tene [bur=gu ki-dig-a-r-í]=do ka-r-í.\\
    girl boy=\textsc{erg} \textsc{pfv}.3-love-\textsc{pfv}.3-3.\textsc{sg}-\textsc{pfv}:\textsc{sg}=that come-3.\textsc{a}-\textsc{pfv}:\textsc{sg}\\
\glt ‘The girl [that the boy loves] came.’ (TAA)
\z

\subsection{Summary}\label{sec:wolfe:3.6}
\tabref{tab:wolfe:1} shows a taxonomy of =\textit{gu} constructions identified to this point. Two of these constructions, shown at bottom – the disambiguation of roles shown in ‘it wasn’t that the \textit{boy} hit the \textit{girl}; the \textit{girl} hit the \textit{boy}’ and the signaling of transition from quoted speaker to quoted speech – have no discernible function in assigning prominence, whether informa\-tion-structural prominence or syntactic prominence. Rather, their only function is to disambiguate roles and constituents. This in itself provides the strongest argument for characterizing =\textit{gu} as primarily a marker of ergativity and not as a marker of \isi{focus}.

Common to the remaining constructions is the assignment of \textit{marked prominence} to the A term of a transitive \isi{verb}, whether higher prominence, as in the cases of sentence \isi{focus} with a brand new A and of argument \isi{focus} on A, or lower prominence, as in the cases of A embedded in relative clauses or of PAV \isi{word order} (due to argument \isi{focus} on P or to pseudo-passivization).


\begin{table}
 \small
\begin{tabularx}{\textwidth}{p{2cm}>{\raggedright}p{2.4cm}lQ}
\lsptoprule
assigning prominence & higher prominence of A & main & argument  {focus} on A (\sectref{sec:wolfe:3.1}, \REF{ex:wolfe:7})\\
&  &  & sentence  {focus} on brand new A (\sectref{sec:wolfe:3.2})\\
\tablevspace
% \hhline{~---} 
& {lower prominence of A} & main (PAV) & clefted P with argument  {focus} (=\textit{di}), A in presupposed material (\sectref{sec:wolfe:3.3}, \REF{ex:wolfe:12})\\
\tablevspace
% \hhline{~---} 
&  &  & P as subject of pseudo-{passive}, 
evoked, demoted A (\sectref{sec:wolfe:3.3}, \REF{ex:wolfe:11})\\
\tablevspace
\tablevspace
% \hhline{~~--} 
&  & relative & P as pivot of \isi{clause} linkage,
A within \isi{relative clause} (\sectref{sec:wolfe:3.5})\\
disambiguating & \multicolumn{3}{l}{disambiguating roles (\sectref{sec:wolfe:3.1}, \REF{ex:wolfe:8})}\\
& \multicolumn{3}{l}{quotative, signaling transition to speech complement (\sectref{sec:wolfe:3.4})}\\
% \hhline{~---}
\lspbottomrule
\end{tabularx}
%%please move \begin{table} just above \begin{tabular
\caption{Summary explanatory matrix of constructions in which A is marked with =gu}
\label{tab:wolfe:1}
\end{table}

\newpage 
This schema provides a clear account of all data points in our RD texts as well as elicited data, and may also elucidate examples given in previous literature that did not seem to express argument \isi{focus} on the A term. One significant line of evidence remains unresolved, however: native speaker intuitions as to the function of =\textit{gu}. The next section briefly turns to this before moving on to =\textit{di} in §4.

\subsection{Native speaker intuition: =\textit{gu} as \textit{deixis of role}}\label{sec:wolfe:3.7}
When Zakaria Fadoul Khidir, a native speaker of \ili{Kube} \ili{Beria}, discusses =\textit{gu} under the category of “\isi{passive} \isi{voice}” (2005: 80; see \sectref{sec:wolfe:3.3} above), he glosses it as a “deictic” marker. While at first glance this does not seem to integrate easily with the analysis above, it well captures native speakers’ intuitions about =\textit{gu}, as the second author of this study attests and as is further confirmed by Amir Libiss (p.c.), a third \ili{Beria} speaker. 

\ili{Beria} already has two sets of deictic markers, proximal =\textit{kî} / =\textit{gi} and \isi{distal} =\textit{to} / =\textit{do} (\citealt{JakobiCrass2004}: 126), both with wide syntactic distribution. These markers can even co-occur with =\textit{gu}, as in the \isi{noun} phrase \textit{o-kî-gu} ‘this person=\textsc{erg}’ in \REF{ex:wolfe:15} above. Clearly =\textit{gu} does not encode mere referential deixis.

At the same time, a deictic reading of =\textit{gu} accords well with the broader social and cognitive purpose of deixis and demonstrative marking. If, as \citet[463]{Diessel2006} argues, “demonstratives function to coordinate the interlocutors’ joint \isi{focus} of attention,” =\textit{gu} could be understood not perhaps as pointing at a referential entity per se, but as pointing at that entity’s grammatical relation within the transitive \isi{clause}. By using =\textit{gu}, a speaker “points” and invites his hearer to \isi{focus} attention on the \isi{agentive} grammatical role played by the indicated referent. Deixis and argument \isi{focus} perform related social and cognitive functions in terms of coordinating interlocutors’ attention.

Of course, as argued at length above, =\textit{gu} appears in more diverse constructions than just argument \isi{focus} on the A term. Nevertheless this does not conflict with the singling out of A-argument \isi{focus} as the prototypical use of =\textit{gu}. By way of comparison, if asked to define the word ‘that,’ most \ili{English} speakers would likely point – yet it is no less true that \ili{English} ‘that’ has also been grammaticalized to serve in discourse-deictic functions and even as a \isi{relativizer}. In fact, one would \textit{expect} the evolution and grammaticalization of a deictic marker into such varied additional constructions (\citealt{Diessel2006}, §4.2).

With this foundation laid, we propose \textsc{deixis of role} as a descriptive term for the prototypical function of optional ergative marking in \ili{Beria}. This not only incorporates native speaker insights about how \ili{Beria} OEM works but it may provide fruitful directions for future OEM theorizing and research. 

To fully comprehend the function of \ili{Beria}’s =\textit{gu}, it is also necessary to understand other \isi{focus} markers in its environment, especially =\textit{di}, which was previously described as an absolutive \isi{focus} marker (\citealt[151--154]{JakobiCrass2004}; \citealt{Jakobi2006}). \sectref{sec:wolfe:4} turns to this.

\section{Copulas and clefts with =\textit{di} and =\textit{i}}\label{sec:wolfe:4}
J\&C characterize =\textit{di} as an absolutive \isi{focus} marker because of data such as in \REF{ex:wolfe:1} above. To assign argument \isi{focus} to the A term, \ili{Beria} speakers use =\textit{gu} but to \isi{focus} P or S, they select =\textit{di.} We reframe =\textit{di}, however, as a previously unrecognized copula, which can then be employed in a cleft construction.

The strongest rationale for this shift is that Jakobi recognizes the existence of a “nonverbal predication marker” (2006: 138) =\textit{di} sharing a presumed “common origin” with =\textit{di} the absolutive \isi{focus} marker. These two are in fact formally identical, as shown in (\ref{ex:wolfe:17}-\ref{ex:wolfe:18}, cf. (\ref{ex:wolfe:1}b) above):

\ea\label{ex:wolfe:17}
\gll O kese-r-ì=gi Tayiba Abdelkarim Abdul=di.\\
     person speak-\textsc{3}\textsc{a}-\textsc{ipfv}.\textsc{sg}=\textsc{rel} Tayiba Abdelkarim Abdul=\textsc{cop}\textsc{\textsubscript{sp}}\textsc{.3}\\
\glt ‘The person who is speaking is Tayiba Abdelkarim Abdul.’ [5/1, 2.2]
\z

\ea\label{ex:wolfe:18}
\gll áɪ=dɪ.\\
     \textsc{pp}:\textsc{1sg}=\textsc{pred}\\
\glt ‘It’s me.’ \citep[138]{Jakobi2006}
\z

Furthermore, in argument \isi{focus} constructions, =\textit{di} shares common distribution with =\textit{i}. Both =\textit{di} and =\textit{i} may \isi{focus} S and P arguments but not A arguments. On the other hand, in our data, neither of these markers shows the complex and varied distribution summarized for =\textit{gu} in \tabref{tab:wolfe:1} above. 

As copular verbs, both =\textit{di} and =\textit{i} share a negative form, =\textit{do}, while =\textit{gu} has no negative form (cf. \ref{ex:wolfe:7}-\ref{ex:wolfe:8}):

\ea\label{ex:wolfe:19}
\gll \v{g}ǐm=d-\={o}.\\
     owl=\textsc{ic}-\textsc{neg}\\
\glt ‘It’s not an owl.’ (J\&C 2004: 101) (compare 23)
\z

\ea\label{ex:wolfe:20}
\gll O arabie kidí=do ber=d-o kire=ego=di.\\
     person car 3.\textsc{a}:have:\textsc{pfv}:\textsc{sg}=\textsc{rel} 3\textsc{s}=\textsc{cop}\textsc{\textsubscript{sp}}-\textsc{neg} brother=\textsc{3sg}.\textsc{poss}=\textsc{cop}\textsc{\textsubscript{sp}}\\
\glt ‘The person who has a car is not him; it is his older brother.’ (TAA)
\z

Semantically, a clean division of labor exists between =\textit{i} and =\textit{di}. The first classifies or ascribes an attribute to the topic: X displays the characteristic Y (\ref{ex:wolfe:21}-\ref{ex:wolfe:22}), or X is a member of the set Y (\ref{ex:wolfe:23}-\ref{ex:wolfe:24}):

\ea\label{ex:wolfe:21}
\gll mɪsā hɛrr=ɪ.\\
     pot full=\textsc{ic:3:aff}\\
\glt ‘The pot is full.’ (J\&C 2004: 100)
\z

\ea\label{ex:wolfe:22}
\gll bɔɔ=égí tákkʊrɛ=ɪ.\\
     ram:\textsc{pl}=1\textsc{s}.\textsc{poss} very.fat=\textsc{ic}:3:\textsc{aff}\\
\glt ‘My rams are very fat.’ (J\&C 2004: 100)
\z

\ea\label{ex:wolfe:23}
\gll gīmm=ī.\\
     owl=\textsc{ic}:3:\textsc{aff}\\
\glt ‘It’s an owl.’ (J\&C 2004: 100)
\z

\ea\label{ex:wolfe:24}
\gll tàmár̩ā=ɪ.\\
     \ili{Tama}.person=\textsc{ic}:3:\textsc{aff}\\
\glt ‘It’s a \ili{Tama}.’ (J\&C 2004: 100)
\z

The second marker predicates co-extensive reference between the two terms: X is fully described by Y and there are no other members in the set; see (\ref{ex:wolfe:17}-\ref{ex:wolfe:18}) above. Accordingly, following \citet{Mikkelsen2005}, we label =\textit{i} a predicational copula, and =\textit{di} a specificational copula. As she explains (2005: 1), predicational copulas “tell us something about the referent of the subject” whereas specificational copulas “says who or what the referent is.”\footnote{\citet[102--106]{JakobiCrass2004} catalogue two other \ili{Beria} copulas: locative/existential \textit{ji} and comitative \textit{bei}.}

These semantics for =\textit{di} and =\textit{i} carry over into their corresponding versions of the cleft construction. In \REF{ex:wolfe:12} above, for instance, the speaker presupposes ‘a certain house exists which my father built’ and, through the specificational cleft, he predicates that ‘this is that house.’ The two referents of ‘this house’ and ‘the house that my father built’ are co-extensive. In \REF{ex:wolfe:25}, by contrast, the speaker assumes ‘a certain substance that the mother gave to the child’ and then through an predicational cleft, clarifies the nature of that substance. What sort of thing is that which the mother has given the child? It is of the class of ‘paste.’

\ea\label{ex:wolfe:25}
\gll gʊʊ\textbf{=ɪ} ɪà ɟàá=r Ø:kɛkk-ɪ.\\
     paste\textbf{=}\textsc{\textbf{cop}}\textbf{.}\textsc{\textbf{sg}}\textbf{.}\textsc{\textbf{abs}}\textbf{} mother child=\textsc{adv} \textsc{oj}:3:give:\textsc{sj}:3-\textsc{pfv}\\
\glt ‘It’s paste that the mother has given to the child.’ \citep[139]{Jakobi2006}
\z

One complicating factor in this rendering of the facts is the presence of =\textit{gu} in the =\textit{di} cleft in \REF{ex:wolfe:14} but not in the =\textit{i} cleft in \REF{ex:wolfe:25}. We propose to resolve this complication with reference to \citegen{HopperThompson1980} scalar transitivity criteria. In \sectref{sec:wolfe:3.1}, scalar transitivity already helped explain why low-transitive verbs ‘have,’ ‘learn’ and ‘know’ cannot occur with ergative-marked agents. If =\textit{gu} correlates with high transitivity, then \citeauthor{HopperThompson1980}’s criterion of the individuation of O (1980: 252--253) may motivate the use or non-use of =\textit{gu} in these two sub-constructions of the \ili{Beria} cleft. The focused element of a =\textit{di} cleft is invariably a referential, definite expression like ‘this house’ whereas in =\textit{i} clefts it is a non-referential, indefinite class or attribute like ‘paste.’ In other words, the fronted P or O in a =\textit{di} cleft is \textit{individuated} whereas in a =\textit{i} it is not.\footnote{In fact, the \textit{individuation of O} may also be relevant to the quotative =\textit{gu} construction (\sectref{sec:wolfe:3.4}). \citet{Rumsey2010} theorizes that the tendency of quotative OEM marking to apply more to direct reported speech than to indirect stems from this same \textit{individuation of O} criterion, since the shift of \isi{voice} and perspective involved in a direct speech act sets it off more sharply from the framing speech act; it is more clearly individuated than indirect reported speech.}  

Distributional, morphological and semantic lines of evidence converge to support the assertion that =\textit{di} is a copula. The only counterevidence we are aware of finds natural explanation in the dynamics of scalar transitivity.


\section{Directions for further research}\label{sec:wolfe:5}
Directions for future research on OEM phenomena in \ili{Beria} abound. As discussed in \sectref{sec:wolfe:2.2}, the present study, which represents the first results of an ongoing program of \ili{Beria}-language corpus collection and discourse analysis, necessarily offers only preliminary conclusions due to the small corpus size and its foreign-language source material. With a larger annotated corpus, a higher-resolution picture will emerge of what additional factors may influence speakers’ choices of when to use =\textit{gu}, potentially including animacy, discourse macrostructure, activation status, unexpectedness of agency, aktionsart, and zero anaphora of other core constituents, among others. More robust explorations of relative clauses, quotative constructions, topic chains, and ergative hopping will complement this fuller analysis.

Another open question is why Jakobi \& Crass’s two folk tale texts (\citeyear{JakobiCrass2004}: 185--192) include no instances of =\textit{gu}, even in contexts with transitive predicates and newly introduced participants, such as in \REF{ex:wolfe:26}:


\ea\label{ex:wolfe:26}
\gll sàgʊr tɛnɛ tɛbɪ-ɛ-r̩-ɛ\\
     jackal girl take-\textsc{prf}\textsc{:3-3:}\textsc{sj}-\textsc{cvb}\textsc{\textsubscript{1}}\\
\glt ‘a jackal took a girl, then…’ (\citealt[185]{JakobiCrass2004})
\z

We suggest that the lack of =\textit{gu} here owes in part to the hearers’ assumed familiarity with stock folk tale participants, which among the \ili{Beria} speech community would perhaps better be classified as \textit{unused} participants than \textit{brand new} ones. This is consistent with the fact that J\&C’s storyteller omits, throughout the story, lexical \isi{noun} phrases that would clarify the identity of referents for the benefit of uninitiated readers. J\&C supply these identities in square brackets in their \ili{French} translation – 21 times in the first four-page text alone. Considerations of genre and register may also be playing a role here: we expect to see =\textit{gu} especially in other formal settings calling for precise speech, such as testimony in traditional courtroom proceedings or recitation of cultural history unfamiliar to hearers. At least, if anything, the lack of =\textit{gu} in J\&C’s texts would seem to \textit{strengthen} the \textit{optional} ergativity analysis of =\textit{gu} given above.

The diachronic origins of =\textit{gu} also remain opaque. Descriptions of case marking for Western Saharan (WS) languages (surveyed in \citealt[38--57]{König2008}) demonstrate similar optional case marking to \ili{Beria}’s: Core arguments go unmarked in default APV constituent order, and the presence of case marking is influenced by “syntactic, semantic and pragmatic factors” that are “yet to be adequately described” \citep[193]{Hutchison1986}.  While WS’s “nominative” marker, \textit{ye} in \ili{Kanuri}-Kanembu and \textit{i} in \ili{Teda}-Daza, bears little formal resemblance to \ili{Beria}’s =\textit{gu}, its usage as an optional marker on A terms is not unlike =\textit{gu}. In fact, it is not certain that \textit{ye} / \textit{i} indeed functions as a nominative marker uniformly across WS. \citet[203--205]{Hutchison1986} does establish that in some dialects of \ili{Kanuri}, \textit{ye} may mark an S term – though rarely, and only in narrative discourse when the subject switches – but in Keshirda \ili{Dazaga}, for instance, Josiah Walters (p.c.) has yet found no tokens of an S marked with \textit{i}. Possibly OEM is not uncommon in the Saharan family.

\section{Conclusions}\label{sec:wolfe:6}
Close scrutiny of =\textit{gu} data in \ili{Beria}’s \ili{Kube} dialect yields much of interest to the typologist, discourse researcher, and Africanist alike. In the analysis of two radio news broadcasts and a handful of carefully elicited examples, we have teased apart as many as seven distinct constructions in which =\textit{gu} marks only the A term of transitive clauses, and never P or S terms. Speakers employ =\textit{gu} in these constructions to various communicative ends, including the disambiguation of roles and the assigning of either higher or lower prominence to a transitive agent. What brings unity to this diversity is the ergative nature of =\textit{gu}: It is indeed foremost a case marker and not a \isi{focus} marker. At the same time, it is essentially an \textit{optional} ergative marker, the use of which varies according to syntactic and discourse-pragmatic principles. This makes it only the second such system described as such on the African continent.

Furthermore, we offered multiple lines of evidence showing that =\textit{di}, which has been described as an absolutive \isi{focus} marker, is in fact a previously unrecognized copula. We contrasted this specificational copula =\textit{di} with the predicational copula \textit{=i}. Both of these may be used in a clefted construction to \isi{focus} an S or P argument. This reconfiguration may be summarized as in \tabref{tab:wolfe:2}.

\begin{table}
\begin{tabularx}{\textwidth}{llXX}
\lsptoprule
{} &  & {\bfseries   \citealt{JakobiCrass2004,Jakobi2006}} &  \\
\midrule 
{\bfseries =gu} & {\bfseries focus} & {{focus} marker, ergative} & {optional ergative marker}\\
{\bfseries =di} & {\bfseries focus} & {{focus} marker, absolutive} & {specificational cleft}\\
&  & {non-verbal predication marker} & {specificational copula}\\
{\bfseries =i} & {\bfseries focus} & {cleft construction} & {predicational cleft} \\
&  & {copula of identification} & {predicational copula}\\
\lspbottomrule
\end{tabularx}
\caption{Summary of reanalysis of \citealt{Jakobi2006}}
\label{tab:wolfe:2}
\end{table} 

In addition, this analysis lends further support to multiple components of \citegen{Rumsey2010} account of OEM phenomena in Ku \ili{Waru} (Trans-New \isi{Guinea}) and \ili{Bunuba} (Non-Pama-Nyungan), especially in its reliance on scalar transitivity, particularly the \textit{individuation of O}, in the description of a quotative OEM construction, and in the recognition of \textit{ergative hopping} phenomena, in a language far removed from Ku \ili{Waru} and \ili{Bunuba} both geographically and genetically. 

Finally, drawing from   \citet{Khidir2005} and \citet{Diessel2006} we explored reframing \ili{Beria}’s OEM as a variety of deixis we called \textit{deixis of role}, because in its prototypical usage =\textit{gu} “points” to the grammatical role of the ergative-marked \isi{noun} phrase, directing a listener’s \isi{focus} and attention to that agent role.

Much work remains to be done to understand \ili{Beria} information structure as a whole, beginning with the assembly of a significantly larger and more diverse corpus, but the present study represents some basic steps forward toward this end. \citet[1626]{McGregor2009} laments that the study of optional case marking suffers from “not enough data, not enough ideas.” It is our hope that the present study makes a modest contribution toward addressing these lacunae.

\section*{Acknowledgements}

We gratefully acknowledge insightful responses given after our initial presentation at ACAL, especially from Colleen Ahland, Michael Ahland and Malte Zimmermann, who pointed us in helpful directions for both clearer presentation and further literature review and engagement. Thanks also to Ryan Pennington, who tipped us off to a number of particularly valuable references. Special thanks to Josiah Walters who shared unpublished data from Keshirda \ili{Dazaga}. We are grateful to Amir Libiss, another \ili{Kube} speaker who reviewed our analysis and offered insights. Thanks finally to Deborah Morton for reviewing an early draft and offering a multitude of incisive comments. All mistakes are, of course, ours.

\section*{Abbreviations}
Abbreviations follow the Leipzig Glossing Rules, except for the following:

\begin{multicols}{2}
 \begin{tabbing}
\textsc{pred}\hspace{1em} \= Patient of transitive \isi{verb} or patientive single argument of intransitive \isi{verb}\kill
\textsc{a}\> agent of transitive \isi{verb} or\\ \> \isi{agentive} single argument of \\ \> intransitive \isi{verb}\\
\textsc{a}\textsc{\textsubscript{impers}} \> impersonal agent\\
\textsc{adv} \> adverbializer\\
\textsc{aff} \> affirmative\\
\textsc{conj}\> conjunctive\\
\textsc{cop}\textsc{\textsubscript{pr}}\> predicational copula\\
\textsc{cop}\textsc{\textsubscript{sp}}\> specificational copula\\
\textsc{cop}\textsc{\textsubscript{loc}}\> locative copula\\
\textsc{ic} \> identificational copula\\
\textsc{ipv} \> imperfective\\
\textsc{lc} \> locative copula\\
\textsc{lm} \> lexical morpheme\\
\textsc{man}\> manner\\
\textsc{med} \> marker of medium verbs\\
\textsc{mid}\> middle \isi{voice}\\
\textsc{nfin}\> non-final \isi{verb} form\\
\textsc{o} \> object\\ 
\textsc{oj} \> object marker\\
\textsc{p}\>Patient of transitive \isi{verb} or patientive \\ \> single argument of intransitive \isi{verb}\\
\textsc{pp} \> person \isi{pronoun}\\
\textsc{pred} \> verbless \isi{predicate}\\
\textsc{rel}\> \isi{relativizer} \\
\textsc{rel}.\textsc{pr}\> relative \isi{pronoun}\\
\textsc{rsn}\> reason \isi{clause}\\
\textsc{rv} \> reduced valence\\
\textsc{sa} \> agent-like subject\\
\textsc{sj} \> subject marker\\
\textsc{sp} \> patient-like subject\\
\textsc{sub} \> subordinator\\
\textsc{v} \> \isi{verb}\\
\textsc{ven}\> venitive
\end{tabbing}
\end{multicols}
 
{\sloppy
\printbibliography[heading=subbibliography,notkeyword=this]
}
\end{document}