\documentclass[output=paper,
modfonts
]{langscibook} 
% \bibliography{localbibliography}
% \bibliography{acalExceed} 

 

\ChapterDOI{10.5281/zenodo.1251750}
\title{Two-place {exceed} comparatives in Luganda} 

\author{M. Ryan Bochnak\affiliation{University of Konstanz} }

\abstract{Research on comparative constructions recognizes the need for both a 3-place (“phrasal”) comparative operator, alongside a 2-place (“clausal”) operator (e.g., \citealt{heim85notes,bhatt11reduced}). Recent cross-linguistic work on comparatives has argued that exceed comparative constructions are phrasal comparatives, making use of a 3-place operator (e.g., \citealt{beck09crosslinguistic,howell12abstracting} for Yorùbá). While certain exceed constructions in Luganda can indeed be analyzed in this way, I argue here for the idea that others involve a 2-place operator that compares two degrees directly. I treat nominalized adjectives as measure functions in the sense of \citealt{bartsch72grammar} and \citealt{kennedy97projecting}: they map an individual to its maximal degree on a scale. This allows us to model possessed adjective nominalizations similar to \citegen{barker95possessive} analysis of relational nouns, although whereas for Barker a possessive DP denotes a predicate of individuals, in this case the resulting DP denotes a degree.
}


\usepackage{ulem}
\usepackage{amssymb}
\usepackage{stmaryrd}
\usepackage{wasysym}
 
\usepackage{amssymb}  
% \usepackage{tipa}
\usepackage{qtree}
\usepackage{stmaryrd}
\usepackage{wasysym}
\usepackage{multicol}
\usepackage{verbatim}
% \usepackage{sectsty}

\begin{document}
\maketitle
\section{Introduction}\label{sec:bochnak:1}



 Formal research on comparatives distinguishes between {\sc phrasal comparatives} (\ref{comp1}) and {\sc clausal comparatives} (\ref{comp2}), depending on the syntactic category of the standard phrase (complement of {\it than}; angled brackets represent ellipsis).


% \largerpage
\begin{exe}

\ex\label{comp1} 
\ili{Kim} is taller than [\textsubscript{DP}Lee]. \hfill {\sc phrasal}

\vspace*{-1mm}
\ex\label{comp2} 
\ili{Kim} is taller than [\textsubscript{CP}Lee is $<$\sout{tall}$>$ ]. \hfill {\sc clausal}

\end{exe}


Along with their different syntax, phrasal and clausal comparatives are taken to have different semantic representations \citep[among others]{bhatt11reduced, heim85notes, kennedy97projecting}. While phrasal comparatives involve an operator with three argument positions, clausal comparatives involve an operator with two argument positions (see (\ref{comp3}); the standard semantic analyses of these will be unpacked in \S \ref{sec2}).



\begin{exe}
\ex\label{comp3}
\begin{xlist}
\ex phrasal comparison ↔ 3-place comparative operator
\ex clausal comparison ↔ 2-place comparative operator
\end{xlist}
\end{exe}


 In this paper, I challenge the assumption that the syntax-semantics mapping in comparatives is necessarily as in (\ref{comp3}), based on a study of {\it exceed}-comparison in \ili{Luganda} (Bantu; JE15). Recent work has argued that {\it exceed}-comparatives are strictly phrasal and therefore make use of a 3-place comparative operator (\citealt{beck09crosslinguistic, howell12abstracting} for Yor\`ub\'a). I too will argue that {\it exceed} comparatives in \ili{Luganda} are syntactically phrasal comparatives, and that most {\it exceed} constructions can be analyzed as involving a 3-place operator. {\it Exceed} constructions in \ili{Luganda} come in two varieties: the \isi{verb} {\it okusinga} `to exceed' can appear as the main \isi{verb} as in (\ref{exceed1a}), or in a subordinate (\isi{infinitive}) form as in (\ref{exceed2a}). In main \isi{verb} {\it exceed} constructions, the gradable \isi{predicate} appears in a nominalized form (with the \isi{noun class} 14 prefix {\it bu-}), while in subordinate {\it exceed} constructions, the gradable \isi{predicate} appears as the main \isi{predicate} of the sentence (showing \isi{noun class} agreement with the subject).



\begin{exe}
\ex\label{exceed1a}
\glll  Kizito \textbf{asinga} Kato obukulu. \rm \\
 Kizito a-singa Kato o-bu-kulu \\ 
 Kizito \scshape nc1\rm-exceed Kato \scshape aug-nc14\rm-old \\
 \glt  `Kizito is older than Kato.' \\
lit.: `Kizito exceeds Kato in oldness.' \hfill {\sc main \isi{verb} exceed}

\ex\label{exceed2a}
\glll Kizito mukulu \textbf{okusinga} Kato. \\
Kizito mu-kulu o-ku-singa Kato  \\ 
Kizito \scshape nc1\rm-old \scshape aug-nc15\rm-exceed Kato  \\
\glt `Kizito is older than Kato.' \\
lit.: `Kizito is old exceeding Kato.' \hfill {\sc subordinate exceed}
\end{exe}


However, I will argue that  ``subcomparatives" in \ili{Luganda} are syntactically phrasal but make use of a 2-place ``clausal"-like comparative operator. An example of this construction is given in (\ref{length1aa}), where the two arguments of {\it exceed} are delineated by square brackets, and are both DPs headed by nominalized gradable predicates. (A licit subcomparative in \ili{English} is given in the translation line of (\ref{length1aa}).)


\begin{exe}
\ex\label{length1aa}
 \glll $[$\textbf{Obuwanvu} \textbf{bw'} \textbf{emmeeza}$]$ businga  $[$\textbf{obugazi} \textbf{bwayo}$]$.\\
\textbf{o-bu-wanvu} \textbf{bu-a} \textbf{e-N-meeza} bu-singa \textbf{o-bu-gazi} \textbf{bu-ayo} \\
\scshape aug-nc14\rm-long \scshape nc14-gen \scshape aug-nc9\rm-table \scshape nc14\rm-exceed \scshape aug-nc14\rm-wide \scshape nc14-poss \\
\glt `The table's length exceeds its width.'   %\hfill \scshape main \isi{verb}
\end{exe}


\noindent In order for this idea to go through, I will also need to provide an analysis of nominalized (NC14 {\it bu-}marked)  adjectives in \ili{Luganda}. To do this, I will build on the intuitions of \citet{moltmann09degree} and \citet{nicolas04semantics} that nominalizations of gradable adjectives are relational, as well as on the standard degree-based analysis of gradable adjectives \citep[among others]{cresswell76semantics, kennedy05scale, vonstechow84comparing}.
 
 \largerpage
The consequences of this analysis are the following: (i) there is novel evidence for two-place comparatives for {\it exceed} languages, which has not previously been adduced; and (ii) at least some syntactically phrasal comparatives can receive a two-place comparative analysis, contra the mappings in (\ref{comp3}). This paper proceeds as follows: \S\ref{sec2} provides an overview of the phrasal vs.~clausal distinction; \S\ref{sec3} outlines a three-place operator analysis of phrasal comparatives in \ili{Luganda}; in \S\ref{sec4} I argue for the existence of a two-place operator in \ili{Luganda} based on evidence from subcomparatives, and I consider two types of analyses for nominalized gradable predicates; \S\ref{sec5} concludes. 







\section{The composition of comparatives}\label{sec2}
 \largerpage


\subsection{Phrasal comparatives}



A phrasal comparative like (\ref{comp1a}) in \ili{English} can be analyzed as involving the 3-place comparative operator in (\ref{er1}). The arguments of the operator are two individual arguments (the standard and target of comparison), and the gradable \isi{predicate} that provides the scale for comparison. A gradable \isi{predicate} is taken to denote a relation between an individual and a degree, as in (\ref{tall1}).



\begin{exe}
\ex\label{comp1a} 
\ili{Kim} is taller than [\textsubscript{DP}Lee]. %\hfill {\sc phrasal}

\ex\label{er1}

{\bf 3-place \textit{-er} for phrasal comparatives:} \\
$\llbracket${\it-er}$_3\rrbracket =  \lambda x \in D_{e}\lambda G \in D_{\langle d, \langle e,t \rangle\rangle}\lambda y \in D_{e}.$max$(\lambda d.G(d)(y))\succ$ max $(\lambda d'.G(d')(x))$


\ex\label{tall1}
$\llbracket${\it tall}$\rrbracket = \lambda d\lambda x.${\bf height}$(x) \succeq d$ \hfill \citep{cresswell76semantics, kennedy05scale}

\end{exe}


A sample derivation of (\ref{comp1a}) is given in (\ref{tree1})-(\ref{derivation2ch2}), assuming that DegP is the sister of A$'$ \citep{heim01degree}, {\it than} is semantically vacuous, and the {\it than} phrase extraposes at PF.


%\begin{comment}

\begin{exe}
\ex\label{tree1}
\resizebox{.66\textwidth}{!}{
\Tree [.{S \\ $\langle t\rangle$} [.{DP \\ $\langle e\rangle$} {\it Kim} ] 
[.{VP \\ $\langle e, t\rangle$} [.V {\it is} ] [.{AP \\ $\langle e, t\rangle$} 
[.{DegP \\ $\langle\langle d, \langle e, t\rangle\rangle, \langle e, t\rangle\rangle$} 
[.{{\bf Deg} \\ $\langle e,\langle\langle d, \langle e, t\rangle\rangle, \langle e, t\rangle\rangle\rangle$} {\textit{\bf-er$_3$}} ] 
[.{PP \\ $\langle e \rangle$ } [.P {\it than} ] [.{DP} {\it Lee} ] ] ] 
[.{A$'$ \\ $\langle d, \langle e, t\rangle\rangle$} [.A {\it tall} ] ] ] ] ]
}
\end{exe}

%\end{comment}


%\begin{exe}
%\ex\label{tree1}
%\end{exe}

%\vspace*{-2em}
% % \includegraphics[scale=0.8]{Tree1.pdf}


\begin{exe}
\ex\label{derivation2ch2}
\begin{xlist}
\ex
$\llbracket$\ili{Deg}$\rrbracket = \llbracket${\it-er}$_3\rrbracket =  \lambda x_{\langle e\rangle}\lambda G_{\langle d, \langle e,t \rangle\rangle}\lambda y_{\langle e\rangle}.$max$(\lambda d.G(d)(y))\succ$ max $(\lambda d'.G(d')(x))$
\ex
$\llbracket$PP$\rrbracket = l$
\ex
$\llbracket$A$'\rrbracket = \llbracket${\it tall}$\rrbracket = \lambda d\lambda x.${\bf height}$(x) \succeq d$
\ex 
$\llbracket$DegP$\rrbracket = \llbracket$\ili{Deg}$\rrbracket(\llbracket$PP$\rrbracket) = 
\lambda G\lambda y.$max$(\lambda d.G(d)(y))\succ$ max $(\lambda d'.G(d')(l))$ 
\ex 
$\llbracket$VP$\rrbracket = \llbracket$AP$\rrbracket = \llbracket$DegP$\rrbracket(\llbracket$AP$\rrbracket) = \lambda y.$max$(\lambda d.${\bf height}$(y) \succeq d)\succ$\\\hspace*{60mm} max$(\lambda d'.${\bf height}$(l) \succeq d')$
\ex\label{derivation2ch2c} 
$\llbracket$S$\rrbracket$$=$$\llbracket$VP$\rrbracket(\llbracket$DP$\rrbracket)$$=$$1$ iff  max$(\lambda d.${\bf height}$(k) \succeq d)\succ$ max $(\lambda d'.${\bf height}$(l) \succeq d')$
\end{xlist}

\end{exe}


The truth conditions for the sentence can be paraphrased as ``the maximal degree to which \ili{Kim} is tall is greater than the maximal degree to which Lee is tall."




\subsection{Clausal comparatives}


Meanwhile, clausal comparatives like (\ref{comp2b}) are analyzed in terms of the two-place comparative operator in (\ref{er2}). The two arguments of two-place {\it-er} are both sets of degrees (type $\langle d, t\rangle$ functions), which are derived in syntax by \isi{movement}.



\begin{exe}
\ex\label{comp2b} 
\ili{Kim} is taller than [\textsubscript{CP}Lee is $<$\sout{tall}$>$ ].

\ex\label{er2}
{\bf 2-place \textit{-er} for clausal comparatives:} \\
$\llbracket${\it-er$_2$}$\rrbracket = \lambda D1 \in D_{\langle d, t\rangle}\lambda D2\in D_{\langle d, t\rangle}.$max$(D2) \succ $ max$(D1)$
\end{exe}

A sample derivation of (\ref{comp2b}) is given in (\ref{tree2}), assuming null operator \isi{movement} within {\it than} phrase to derive a $\langle d, t\rangle$ function \citep[this step not shown here; see][]{chomsky77whmovement}, \isi{movement} of DegP to derive another $\langle d, t\rangle$ function, and ellipsis within the {\it than} phrase.



%\begin{comment}

\begin{exe}
\ex\label{tree2}
\Tree [.{S$_1$ \\ $\langle t\rangle$} [.{. \\ $\langle d, t\rangle$} 1 [.{S \\ $\langle t \rangle$} 
[.{DP  \\ $\langle e\rangle$} {\it Kim} ] [.{VP \\ $ \langle e, t\rangle$} [.{V} {\it is} ] 
[.{AP \\ $\langle e, t\rangle$}  
[.{{\bf DegP} \\ $\langle d\rangle$} {\bf t$_1$} ] [.{A$'$ \\ $ \langle d, \langle e, t\rangle\rangle$} [.A {\it tall} ] ] ] ] ] ] 
[.{{\bf DegP$_1$} \\ $\langle \langle d, t\rangle, t\rangle$} [.{{\bf Deg} \\  $\langle \langle d, t\rangle, \langle\langle d, t\rangle , t\rangle\rangle$} {\bf-er}$_2$ ] [.{PP \\ $\langle d, t\rangle$ } [.P {\it than} ] \qroof{{\it Lee is $<$$d-$tall$>$}}.{CP \\ $ \langle d, t\rangle$} ] ] ]
\end{exe}

%\end{comment}

%\begin{exe}
%\ex\label{tree2}
%\end{exe}

%\vspace*{-2em}
% % \includegraphics[scale=0.8]{Tree2.pdf}

\begin{exe}
\ex\label{derivation1ch2}
\begin{xlist}
\ex\label{derivation1ch2b}
 $\llbracket$PP$\rrbracket = \lambda d'.${\bf height}$(l) \succeq d'$
\ex
$\llbracket${\it-er$_2$}$\rrbracket = \lambda D1 \in D_{\langle d, t\rangle}\lambda D2\in D_{\langle d, t\rangle}.$max$(D2) \succ $ max$(D1)$
\ex $\llbracket$DegP$\rrbracket = \llbracket${\it-er}$_2$$\rrbracket(\llbracket$PP$\rrbracket) = \lambda D2.$max$(D2) \succ $max$(\lambda d'.${\bf height}$(l) \succeq d'$)
\ex
$\llbracket$A$'\rrbracket = \llbracket${\it tall}$\rrbracket = \lambda d\lambda x.${\bf height}$(x) \succeq d$
\ex
$\llbracket$VP$\rrbracket = \llbracket$AP$\rrbracket = \lambda x.${\bf height}$(x) \succeq d$
\ex
$\llbracket$S$\rrbracket = $ {\bf height}$(k) \succeq d$
\ex 
$\llbracket$.$\rrbracket = \lambda d.${\bf height}$(k) \succeq d$
\ex\label{derivation1ch2d} 
$\llbracket$S$_1\rrbracket = \llbracket$DegP$\rrbracket(\llbracket$.$\rrbracket) = $ max$(\lambda d.${\bf height}$(k) \succeq d) \succ $max$(\lambda d'.${\bf height}$(l) \succeq d'$)
\end{xlist}

\end{exe}



Note that the exact same truth conditions are derived for 3-place and 2-place comparatives (compare (\ref{derivation2ch2c}) and (\ref{derivation1ch2d})). With this background in place, we now turn to the analysis of {\it exceed} comparatives in \ili{Luganda}.







\section{A 3-place comparative operator in Luganda}\label{sec3}


Reviewing what we have already seen, comparatives in \ili{Luganda} are formed using the \isi{verb} {\it (oku)singa} `exceed', where the direct object of {\it(oku)singa} is a DP naming the standard of comparison. {\it Exceed} comparatives in \ili{Luganda} come in two varieties. In (\ref{exceed1}), the {\it exceed} \isi{verb} is the main \isi{verb}, with a gradable \isi{predicate} in nominalized form with the NC14 {\it bu-}prefix. In (\ref{exceed2}), the {\it exceed} \isi{verb} phrase is in a subordinate (\isi{infinitive}) form marked with the NC15 {\it ku-}prefix, while the gradable \isi{predicate} is the main \isi{predicate}.


\begin{exe}
\ex\label{exceed1}
\glll  Kizito \textbf{asinga} Kato obukulu. \rm \\
 Kizito a-singa Kato o-bu-kulu \\ 
 Kizito \scshape nc1\rm-exceed Kato \scshape aug-nc14\rm-old \\
 \glt  `Kizito is older than Kato.' \\
lit.: `Kizito exceeds Kato in oldness.' \hfill {\sc main \isi{verb} exceed}

\ex\label{exceed2}
\glll Kizito mukulu \textbf{okusinga} Kato. \\
Kizito mu-kulu o-ku-singa Kato  \\ 
Kizito \scshape nc1\rm-old \scshape aug-nc15\rm-exceed Kato  \\
\glt `Kizito is older than Kato.' \\
lit.: `Kizito is old exceeding Kato.' \hfill {\sc subordinate exceed}
\end{exe}


Comparatives with {\it (oku)singa} can be used in quality comparisons (above), as well as amount comparisons, as shown in (\ref{exceed3})-(\ref{exceed4}).\footnote{I assume a covert adjective {\it ma-ngi} `many'  is present in (\ref{exceed3})-(\ref{exceed4}) to derive the amount comparison reading. The adjective may also appear overtly, but not shown here due to space. See \citealt{bochnak13crosslinguistic}.}


\begin{exe}
\ex\label{exceed3}
\glll Charlotte \textbf{yasinga} Rita o\textbf{ku}wandiika amabaluwa. \\
Charlotte a-a-singa Rita o-ku-wandiika a-ma-baluwa  \\
Charlotte \scshape nc1-pst\rm-exceed Rita \scshape aug-nc15\rm-write \scshape aug-nc6\rm-letter  \\
\glt `Charlotte wrote more letters than Rita.' \\
lit.: `Charlotte exceeds Rita in writing letters.' \hfill {\sc main \isi{verb} exceed}

\ex\label{exceed4}
\glll Charlotte yawandiika amabaluwa  \textbf{okusinga} Rita. \\
Charlotte a-a-wandiika a-ma-baluwa  o-ku-singa Rita \\
Charlotte \scshape nc1-pst\rm-write \scshape aug-nc6\rm-letter  \scshape aug-nc15\rm-exceed Rita \\
\glt `Charlotte wrote more letters than Rita.'  \\
lit.: `Charlotte wrote letters, exceeding Rita.' \hfill {\sc subordinate exceed}
\end{exe}




Given that the object of {\it(oku)singa} is a DP, this looks like phrasal comparison. Let us propose that {\it (oku)singa} has the semantics of three-place comparative {\it-er$_3$} as in (\ref{singa1}), where the direct object of  {\it (oku)singa} (i.e., the standard of comparison) is an individual-denoting DP. Evidence for a degree-based analysis for \ili{Luganda} comes from the fact that both the main \isi{verb} and subordinate {\it exceed} constructions pass \citeauthor{kennedy07modes}'s (\citeyear{kennedy07modes}) tests for explicit comparison, including acceptability in crisp judgment contexts, and the ability to be formed with absolute-standard gradable predicates (see \citealt{bochnak13crosslinguistic} for these tests). This proposal thus makes \ili{Luganda} similar to other {\it exceed} languages, such as Yor\`ub\'a, which has also been argued to use a 3-place operator \citep{beck09crosslinguistic, howell12abstracting}. 



\begin{exe}
\ex\label{singa1}
$\llbracket${\it(oku)singa}$_3\rrbracket =\llbracket${\it-er}$_3\rrbracket = \\  \lambda x \in D_{e}\lambda G \in D_{\langle d, \langle e,t \rangle\rangle}\lambda y \in D_{e}.$max$(\lambda d.G(d)(y))\succ$ max $(\lambda d'.G(d')(x))$ 
\end{exe}



The proposed structure and derivation of the main \isi{verb} {\it exceed} comparative in (\ref{exceed1}) is given in (\ref{MainVerbTree1})-(\ref{Tree1Sem}).\footnote{See \citealt{bochnak13crosslinguistic} for arguments for the syntax proposed here.}


\begin{exe}
\ex\label{MainVerbTree1}
\textit{Kizito asinga Kato obukulu}.\\
`Kizito exceeds Kato in oldness.' %\hfill {\sc main verb}

%\begin{comment}
\Tree [.S [.DP {\it Kizito} ] [.VP$_2$ [.VP$_1$ [.V {\it asinga} ] [.DP {\it Kato} ] ] [.DP {\it obukulu} ] ] ] \\
%\end{comment}

% % \includegraphics[scale=0.8]{Tree3.pdf}


\ex\label{Tree1Sem}

\begin{xlist}
\ex
$\llbracket$\textit{asinga}$\rrbracket = \lambda x\lambda G\lambda y.$max$(\lambda d.G(d)(y)) \succ $ max$(\lambda d'.G(d')(x))$
\ex
$\llbracket${\it obukulu}$\rrbracket = \lambda d\lambda x.${\bf old}$(x) \succeq d$
\ex
$\llbracket$VP$_1$$\rrbracket = \lambda G\lambda y.$max$(\lambda d.G(d)(y)) \succ $ max$(\lambda d'.G(d')(Kato))$
\ex
$\llbracket$VP$_2$$\rrbracket = \lambda y.$max$(\lambda d.${\bf old}$(y) \succeq d) \succ $ max$(\lambda d'.${\bf old}$(Kato) \succeq d')$
\ex
\fbox{
$\llbracket$S$\rrbracket = 1$ iff max$(\lambda d.${\bf old}$(Kizito) \succeq d) \succ $ max$(\lambda d'.${\bf old}$(Kato)\succeq d')$ }
\end{xlist}

\end{exe}


This analysis derives the intuitively correct truth conditions for the comparative: (\ref{exceed1}) is true if and only if the maximal degree to which Kizito is old is greater than the maximal degree to which Kato is old. It thus provides a straightforward derivation of main \isi{verb} {\it exceed} constructions parallel to \ili{English} phrasal comparatives. However, note that I have assigned a semantics for nominalized {\it bu-kulu} `{\sc nc14}-old' identical to that of gradable adjectives in \ili{English} (cf.~\ref{tall1}). If the underlying semantics of the gradable adjective stem {\it-kulu} `old' is the same as gradable adjectives in \ili{English}, this means that the \isi{nominalization} morphology {\it bu-} is semantically vacuous. This assumption will be revised later on in \S\ref{sec4}, but for now it allows us to straightforwardly derive the truth conditions for the comparative with the semantic tools familiar from \ili{English}.



For subordinate {\it exceed} constructions, let us assume the same three-place comparative operator semantics for {\it (oku)singa} in (\ref{singa1}). Given the (simplified) structure for (\ref{exceed2}) as in (\ref{SubordTree1}), the semantic derivation proceeds as in (\ref{Tree2Sem}).%\footnote{NOTES ON DEFENDING THIS SYNTAX?}


\begin{exe}
\ex\label{SubordTree1} 
\textit{Kizito mukulu okusinga Kato}.\\
`Kizito is old, exceeding Kato.'


%\begin{comment}
\Tree [.S [.DP$_3$ {\it Kizito} ] [.AP$_2$ [.AP$_1$ {\it mukulu} ] [.DP$_2$ [.{\sc aug} {\it o-} ] [.NP {\it ku-} [.VP [.V {\it -singa} ] [.DP$_1$ {\it Kato} ] ] ] ] ] ] \\
%\end{comment}

% \includegraphics[scale=0.8]{Tree4.pdf}


\ex\label{Tree2Sem}

\begin{xlist}
\ex
$\llbracket$\textit{-singa}$\rrbracket = \lambda x\lambda G\lambda y.$max$(\lambda d.G(d)(y)) \succ $ max$(\lambda d'.G(d')(x))$
\ex
$\llbracket$DP$_2$$\rrbracket = \lambda G\lambda y.$max$(\lambda d.G(d)(y)) \succ $ max$(\lambda d'.G(d')(Kato))$
\ex
$\llbracket${\it mukulu}$\rrbracket = \lambda d\lambda x.${\bf old}$(x) \succeq d$
\ex
$\llbracket$AP$_2$$\rrbracket = \lambda y.$max$(\lambda d.${\bf old}$(y) \succeq d) \succ $ max$(\lambda d'.${\bf old}$(Kato) \succeq d')$
\ex
\fbox{
$\llbracket$S$\rrbracket = 1$ iff max$(\lambda d.${\bf old}$(Kizito) \succeq d)) \succ $ max$(\lambda d'.${\bf old}$(Kato) \succeq d')$  }
\end{xlist}

\end{exe}


Parallel truth conditions are thus derived for (\ref{exceed1}) and (\ref{exceed2}), and indeed these constructions appear to have the same truth conditions.\footnote{Again, they are both acceptable in crisp judgment contexts and are productive will gradable predicates of all scale structures \citep{bochnak13crosslinguistic}. However, my consultants seem to have a preference for subordinate {\it exceed} constructions, in that they are almost always offered first as translations of \ili{English}. The corresponding main \isi{verb} versions are nevertheless always accepted by speakers when offered by the researcher. I have no explanation for this apparent preference.} Note once again the assumption that \isi{nominalization} morphology (this time {\it ku-} on the \isi{verb} phrase) is semantically vacuous.


In sum, both main \isi{verb} and subordinate {\it exceed} comparatives in \ili{Luganda} receive a straightforward analysis as three-place phrasal comparatives by borrowing the familiar tools from \ili{English} and other better-studied languages. This seems like a nice result, given that other {\it exceed} languages have also been analyzed in this way.\footnote{In an LFG-based analysis, \citet{beerman05comparison} likewise propose a common semantics for comparison between \ili{English} and \ili{Luganda}, despite their different syntactic structures.} Furthermore, it has been noted for other languages that only three-place comparatives exist (e.g., \citealt{bhatt11reduced} on \ili{Hindi}/Urdu). That is, there is no reason to expect that a language necessarily uses both 3-place and 2-place comparison. However, in the next section I present evidence for the existence of two-place comparatives in \ili{Luganda}, and propose an analysis involving a two-place version of the {\it exceed} \isi{verb}.








\section{A 2-place comparative operator in Luganda}\label{sec4}



One of the tests for diagnosing clausal (2-place) comparatives is the availability of multiple standards of comparison \citep{lechner01reduced, merchant09phrasal, bhatt11reduced}. For instance, (\ref{ellipsis1}) contains three standards of comparison following {\it than}. The idea is this: the apparent phrasal surface form of comparatives like {\it \ili{Kim} is taller than Lee} is the result of ellipsis applying to an underlyingly clausal complement of {\it than}, where the standard has moved out of the ellipsis site. Multiple standards can appear after {\it than} so long as they have all moved out of the ellipsis site.\footnote{See \citet{lin09chinese} for arguments that comparatives with multiple standards in \ili{Chinese} are still phrasal in nature, and not the result of ellipsis from a clausal source. In other words, the ability to have multiple standards does not necessarily entail that there is an underlying clausal source, so this test does not provide strong evidence for diagnosing clausal comparatives.} 


\begin{exe}
\ex\label{ellipsis1} \ili{Kim} read more books on Tuesday than Lee magazines on Thursday.
\end{exe}

Multiple standards in \ili{Luganda}, however, are not licensed, either in main \isi{verb} (\ref{ex213}) or subordinate (\ref{ex213a}) {\it exceed} constructions.\footnote{(\ref{ex213})-(\ref{ex213a}) contain the adjective {\it ma-ngi} `many'; cf.~footnote 1 and (\ref{exceed3})-(\ref{exceed4}). Its presence/absence does not affect the grammaticality of these sentences.}

\begin{exe}
\ex[*]{ \glll \rm[Charlotte\rm] \rm[ku mande\rm] yasinga \rm[Rita\rm] \rm[ku lw'okubiri\rm] okuwandiika amabaluwa amangi. \\
Charlotte ku mande a-a-singa Rita ku lw'okubiri o-ku-wandiika a-ma-baluwa a-ma-ngi \\
Charlotte \scshape loc Monday \scshape nc1-pst\rm-exceed R \scshape loc Tuesday \scshape aug-nc15\rm-write \scshape aug-nc6\rm-letter \scshape aug-nc6\rm-many \\
\glt Intended: `Charlotte wrote more letters on Monday than Rita wrote \\on Tuesday.' \hfill {\sc main verb}
}\label{ex213} 
\end{exe}

\begin{exe}
\ex[*]{
 \glll Charlotte yawandiika amabaluwa mangi ku mande okusinga \rm[Rita\rm] \rm[ku lw'okubiri\rm]. \\
Charlotte a-a-wandiika a-ma-baluwa ma-ngi ku mande o-ku-singa Rita ku lw'okubiri \\
Charlotte \scshape nc1-pst\rm-write \scshape aug-nc6\rm-letter \scshape nc6\rm-many \scshape loc Monday \scshape aug-nc15\rm-exceed Rita  \scshape loc Tuesday \\
\glt Intended: `Charlotte wrote more letters on Monday than Rita wrote\\ on Tuesday.' \hfill \scshape subordinate
}\label{ex213a}
\end{exe}



A second test for the availability of clausal comparatives comes from {\sc subcomparatives}: comparisons based on two different dimensions. An \ili{English} example is given in (\ref{subcomp1}). In this case, the complement of {\it than} has overt clausal syntax.



\begin{exe}
\ex\label{subcomp1} The table is longer than it is wide.
\end{exe}


Versions of these types of comparisons are possible using the \ili{Luganda} main \isi{verb} {\it exceed} construction, as shown in (\ref{length1a}). However, the constituents forming the comparison are not full clauses. Rather, the subject and object of {\it okusinga} are headed by nominalized gradable adjectives (bolded in \ref{length1a}), which name the two dimensions of comparison. These nominalizations appear in a possessive construction, the general form of which is given in (\ref{p1}), where {\sc gen} is the genitive particle.\footnote{Note that {\it bu-} on {\sc gen},{\sc poss}, and {\it-singa} are inflectional prefixes for agreement with a {\sc nc14} \isi{noun}.}




\begin{exe}
\ex\label{p1}
 possessed adjective nominalizations = [{\it bu-}adj + {\sc gen} + possessor]
 
\ex\label{length1a}
 \glll \textbf{Obuwanvu} \textbf{bw'} \textbf{emmeeza} businga  \textbf{obugazi} \textbf{bwayo}.\\
\textbf{o-bu-wanvu} \textbf{bu-a} \textbf{e-N-meeza} bu-singa \textbf{o-bu-gazi} \textbf{bu-ayo} \\
\scshape aug-nc14\rm-long \scshape nc14-gen \scshape aug-nc9\rm-table \scshape nc14\rm-exceed \scshape aug-nc14\rm-wide \scshape nc14-poss \\
\glt `The table's length exceeds its width.'   %\hfill \scshape main \isi{verb}
\end{exe}




I suggest that this is a case of  2-place comparison in \ili{Luganda}, despite the fact that these are not syntactically clausal comparatives (i.e., no clausal syntax in the standard). To see how this would work, we need an analysis of possessed nominalized adjectives. I consider two styles of analysis here, which both deliver the desired result.


The first style of analysis I will call the {\sc relational analysis}, following the intuitions of \citet{moltmann09degree} and \citet{nicolas04semantics} that nominalized properties like {\it length} are inherently relational. That is, a nominalized adjective like {\it bu-wanvu} (`{\sc nc14}-long' $\approx$ `length'/`longness') relates an individual to its length. Given that gradable adjectives are also standardly taken to denote relations between individuals and degrees \citep{cresswell76semantics, heim01degree, kennedy05scale}, it seems at first blush that the assumption we made in \S\ref{sec3} that \isi{nominalization} is vacuous is a reasonable one. However, rather than assuming that nominalizing {\it bu-} is vacuous, I propose it has the function of reversing the argument relations of a gradable adjective; the difference between the gradable adjective and its nominalized form is thus given in (\ref{wanvu1})-(\ref{wanvu2}).


\begin{exe}
\ex\label{wanvu1} 
$\llbracket${\it-wanvu}$\rrbracket = \llbracket${\it long}$\rrbracket = \lambda d\lambda x.${\bf length}$(x)\succeq d$ \hfill $\langle d, \langle e, t\rangle\rangle$
\ex\label{wanvu2} 
$\llbracket${\it bu-wanvu}$\rrbracket = \lambda x\lambda d.${\bf length}$(x)\succeq d$ \hfill $\langle e, \langle d, t\rangle\rangle$

\end{exe}


This change of argument structure means that a nominalized adjective expects an individual as its first argument, which will be the possessor $x$ of the property. After saturating the individual argument position, we are left with a set of degrees that $x$'s length is greater than or equal to.  Thus, the DP {\it obuwanvu bw'emmeeza} ($\approx$ `the length of the table') denotes the set of degrees in (\ref{wanvu3}).


\begin{exe}
\ex\label{wanvu3}
$\llbracket${\it obuwanvu bw'emmeeza}$\rrbracket = \lambda d.${\bf length}$(t) \succeq d$
\end{exe}


For a compositional analysis, I follow the syntax of possession proposed by \citet{barker95possessive}. On Barker's analysis, the possessive DP is headed by a possessive D head {\sc poss}, which is null in \ili{English}. For non-relational nouns (e.g., {\it table}), {\sc poss} introduces the possessive relation $\pi$, as shown in (\ref{p2}). For inherently relational nouns, {\sc poss} is still present, but simply denotes the identity function ($\lambda R.R$). Barker's analysis of possessed relational nouns in \ili{English} in outlined in (\ref{barker1})-(\ref{barker3}).


\begin{exe}
\ex\label{p2}
 $\llbracket${\sc poss}$\rrbracket = \lambda P\lambda y\lambda z.\pi(y, z) \wedge P(z)$

\ex\label{barker1}
\end{exe}
\vspace*{-2em}

%\begin{comment}
\Tree [.DP$_2$ [.DP$_1$ {\it John's} ] [.D$'$ [.NP {\it child} ] [.D {\sc poss} ] ] ]
%\end{comment}

\hspace*{10mm}
% \includegraphics[scale=.8]{Tree5.pdf}


\begin{exe}
\ex\label{barker3}
\begin{xlist}
\ex
$\llbracket${\it child}$\rrbracket = \lambda x\lambda y.${\bf child}$(x,y)$
\ex
$\llbracket${\sc poss}$\rrbracket = \lambda R.R$
\ex
$\llbracket${\it John's child}$\rrbracket = \lambda y.${\bf child}$(j, y)$
\end{xlist}
\end{exe}


I propose for \ili{Luganda} that the genitive particle {\it-a} is the overt spell-out of Barker's {\sc poss}. The analysis for the possessive nominalized adjective {\it obuwanvu bw'emmeeza} `the length of the table' is thus given in (\ref{poss1})-(\ref{poss2}), whereby the nominalized adjective denotes a relational \isi{noun}, and the result of applying the possessor is a (characteristic function of a) set of degrees of length.



\begin{exe}
\ex\label{poss1}
\end{exe}

%\begin{comment}
\Tree [.DP$_2$ [.D$'$ [.NP {\it obuwanvu} ] [.D {\it bu-a} ] ] [.DP$_1$ {\it emmeeza} ] ]
%\end{comment}


%\vspace*{-2em}
% \includegraphics[scale=0.8]{Tree6.pdf}


\begin{exe}
\ex\label{poss2}
\begin{xlist}
\ex 
$\llbracket${\it obuwanvu}$\rrbracket = \lambda x\lambda d.${\bf length}$(x) \succeq d$
\ex
$\llbracket${\it -a}$\rrbracket = \llbracket${\sc poss}$\rrbracket = \lambda R.R$
\ex
$\llbracket ${\it obuwanvu bwa}$\rrbracket = \lambda x\lambda d.${\bf length}$(x) \succeq d$
\ex
$\llbracket${\it obuwanvu bw'emmeezza}$\rrbracket = \lambda d.${\bf length}$(t) \succeq d$
\end{xlist}
\end{exe}



Now let's return to the exceed comparative we want to analyze, namely (\ref{length1a}). Under the analysis of possessed adjective nominalizations proposed here, the subject and object of {\it okusinga} `exceed' both denote sets of degrees. But these are exactly the arguments that a 2-place comparative operator expects (cf.~\ref{er2}). If we submit that {\it okusinga} has a 2-place variant as in (\ref{singa2}), the analysis of (\ref{length1a}) can proceed straightforwardly as in (\ref{singaTree2})-(\ref{singaTree2a}).



\begin{exe}
\ex\label{singa2}
$\llbracket${\it(oku)singa}$_2$$\rrbracket = \llbracket${\it-er}$_2$$\rrbracket = \lambda D1 \in D_{\langle d, t\rangle}\lambda D2\in D_{\langle d, t\rangle}.$max$(D2) \succ $ max$(D1)$

\ex\label{singaTree2}
\end{exe}

%\begin{comment}
\Tree [.S \qroof{{\it Obuwanvu bw'emmeeza}}.DP$_2$ [.VP [.V {\it bu-singa} ] \qroof{{\it obugazi bwayo}}.DP$_1$ ] ]
%\end{comment}

%\vspace*{-2em}
% \includegraphics[scale=0.8]{Tree7.pdf}


\begin{exe}
\ex\label{singaTree2a}
\begin{xlist}
\ex
$\llbracket$V$\rrbracket = \llbracket${\it(oku)singa}$_2$$\rrbracket = \llbracket${\it-er}$_2$$\rrbracket = \lambda D1 \in D_{\langle d, t\rangle}\lambda D2\in D_{\langle d, t\rangle}.$max$(D2) \succ $ max$(D1)$
\ex
$\llbracket$DP$_1$$\rrbracket = \lambda d.${\bf width}$(t) \succeq d$
\ex
$\llbracket$VP$\rrbracket = \lambda D2.$max$(D2) \succ $ max$(\lambda d.${\bf width}$(t) \succeq d)$
\ex
$\llbracket$DP$_2$$\rrbracket = \lambda d.${\bf length}$(t) \succeq d$
\ex
\fbox{
$\llbracket$S$\rrbracket = 1 $ iff max$(\lambda d.${\bf length}$(t) \succeq d) \succ $ max$(\lambda d.${\bf width}$(t) \succeq d)$
} \\
 = ``the maximal degree to which the table is long is greater than the maximal degree to which it is wide"
\end{xlist}
\end{exe}



I refer to the second style of analysis for adjective nominalizations as the {\sc measure function} analysis. The idea is that instead of denoting relations, nominalized gradable adjectives instead denote measure functions directly, i.e., functions from individuals to the maximal degree to which they hold a property \citep{kennedy07vagueness}. Under this analysis, the denotation of {\it obuwanvu} `length' would be modeled as in (\ref{mf1}).



\begin{exe}
\ex\label{mf1}
$\llbracket${\it obuwanvu}$\rrbracket = \lambda x.${\bf length}$(x)$ \hfill type $\langle e, d\rangle$
\end{exe}


Such an analysis has the following consequences. First, in the genitive construction, a possessed nominalized adjective denotes a degree (instead of a set of degrees), as in (\ref{mf2}).



\begin{exe}
\ex\label{mf2}
$\llbracket${\it obuwanvu bw'emmeezza}$\rrbracket = $ {\bf length}$(t)$
\end{exe}


Second, we require a modified lexical entry for 2-place {\it exceed}, reflecting this type difference.  Under this analysis, {\it okusinga} compares two degrees directly, as in (\ref{singa2a}). The derivation in (\ref{singaTree2}) proceeds in the same was as before; it's only the type of the arguments of the comparative operator that are different. Also note that there are no maximality operators within the semantics of {\it oksuinga} under this analysis: maximality comes for free from the measure function itself.


\begin{exe}
\ex\label{singa2a}
 $\llbracket${\it -singa}$_2\rrbracket = \lambda d \in D_d\lambda d' \in D_d.d' \succ d$
\end{exe}


Third, if we continue to assume a relational (type $\langle d, \langle e, t\rangle\rangle$) analysis for gradable adjectives, then the nominalizing {\it bu-} morphology now has the function of turning the gradable adjective into the corresponding measure function, as in (\ref{mf3}).\footnote{Alternatively, if we take as a starting point a measure function analysis of gradable adjectives \citep{bartsch72grammar, kennedy07vagueness}, then we maintain that {\it bu-} is semantically vacuous:

\begin{exe}
\ex
$\llbracket${\it -wanvu}$_A\rrbracket = \llbracket${\it obuwanvu}$_N\rrbracket = \lambda x.${\bf length}$(x)$
\end{exe}
}



\begin{exe}
\ex\label{mf3}
$\llbracket${\it bu-}$\rrbracket = \lambda G \in D_{\langle d, \langle e, t\rangle\rangle}\lambda x.${\bf m}$_G(x)$ \\
where {\bf m}$_G$ is the measure function associated with a gradable \isi{predicate} $G$
\end{exe}




Finally, the measure function analysis allows for a more straightforward analysis of nominalized adjectives in argument position, an example of which is given in (\ref{mf4}).



\begin{exe}
\ex\label{mf4}
\glll Obuwanvu bwa Lydia bwe wunyisa. \\
o-bu-wanvu bu-a Lydia bu-e wunyisa \\
{\sc aug-nc14}-tall {\sc nc14-gen} Lydia {\sc nc14-cleft} surprise \\
\glt `Lydia's height is surprising.'
\end{exe}


Intuitively, it is Lydia's maximal degree of height that is surprising, not any smaller degree of height. Once again, the measure function analysis of nominalized adjectives gets maximality for free, whereas the relational analysis must posit an ad hoc maximality operator to turn the $\langle d, t\rangle$ expression into a degree.\footnote{Of course, this also involves making the non-trivial assumption that predicates like {\it wunyisa} `surprise' can be predicates of degrees. See \citet{castroviejo08amazing} and \citet{moltmann09degree} for discussion of related issues.} 


Summarizing, I take evidence from subcomparatives to indicate that \ili{Luganda} has a two-place version of the comparative operator, alongside a three-place variant.  There are two plausible analyses of nominalized gradable adjectives that allow for a straightforward analysis of two-place {\it exceed} comparatives like (\ref{length1a}). While I do not come down definitively in favor of either the relational or measure function analysis for nominalized gradable adjectives, examples like (\ref{mf4}) may point towards the measure function analysis.
 



\section{Conclusion}\label{sec5}


I have argued for the existence of (at least) two versions of {\it (oku)singa} qua comparative operator in \ili{Luganda}, namely those in (\ref{concl1})-(\ref{concl2}).\footnote{Which version of the 2-place operator in (\ref{concl2}) we choose depends on whether we adopt the relational or measure function analysis of nominalized gradable adjectives.}


\begin{exe}
\ex\label{concl1}
 3-place comparative: \\
 $\llbracket${\it(oku)singa)}$_3\rrbracket =  \lambda x\lambda G\lambda y.$max$(\lambda d.G(d)(y))\succ$ max $(\lambda d'.G(d')(x))$  

\ex\label{concl2}
 2-place comparative: \\
$\llbracket${\it(oku)singa}$_2$$\rrbracket = \lambda D1\lambda D2.$max$(D2) \succ $ max$(D1)$ \\ or \\
$\llbracket${\it(oku)singa}$_2$$\rrbracket = \lambda d\lambda d'.d' \succ d$
\end{exe}



Significantly, \ili{Luganda} has a 2-place {\it exceed} comparative despite having only syntactically ``phrasal" standards. This results in an important consequence for theories of comparatives cross-linguistically, namely that having only syntactically phrasal standards does not necessarily entail the absence of 2-place comparatives. Meanwhile, the right analysis for \isi{nominalization} depends on our starting assumptions about the underlying meaning of gradable adjectives and the semantic type of possessed adjective nominalizations.





\section*{Acknowledgments}

I thank Kisuule Magala Katende for his patience and enthusiasm for working with me on his language. I also thank Karlos Arregi, Chris Kennedy, Jason Merchant, two anonymous reviewers, as well as audiences at the University of Chicago, UC Berkeley, ACAL 45 at the University of Kansas, and the Semantics of African, Asian, and Austronesian Languages at the University of T\"ubingen for comments and criticisms. The usual disclaimers apply.



\section*{Abbreviations}

\begin{tabularx}{.45\textwidth}{lQ}
{\sc aug} & augment\\
{\sc cleft} & cleft\\
{\sc gen} & genitive marker\\
{\sc loc} & locative\\
\end{tabularx}
\begin{tabularx}{.45\textwidth}{lQ}
{\sc nc}\# &  {noun class}\\
{\sc poss} & possessive\\
{\sc pst} & past\\
\\
\end{tabularx}


 








{\sloppy
\printbibliography[heading=subbibliography,notkeyword=this]
}
\end{document}
