\chapter{Die Konstruktionalisierung von [Definitartikel\,+\,N]} \label{bicpic}

In Abschnitt \ref{sec:konstruktionalisierung} der Arbeit wurde dafür argumentiert, dass die Entwicklung des Definitartikels \is{Definitartikel} einen Konstruktionalisierungsprozess \is{Konstruktionalisierung} darstellt: Ein Form-Funk"-tions"-paar etabliert sich als neuer Knoten im \is{Konstruktikon} Konstruktionsnetzwerk. Mit der \is{Korpuslinguistik} Korpusuntersuchung \is{Korpus} wurden die drei zentralen Variablen dieser \isi{Konstruktionalisierung} beleuchtet -- das funktionale Spektrum von \object{dër}, die kategoriale Füllung des N-Slots sowie der Status der gesamten Struktur im sich wandelnden \is{Nominalsyntax} NP-System \is{Nominalphrase (NP)} des Althochdeutschen. Im vorliegenden Kapitel werden diese Ergebnisse genutzt, um den diachronen Werdegang von [\object{dër}\,+\,N] vor dem Hintergrund eines gebrauchsbasierten Modells zu rekonstruieren. 
Der Fokus liegt damit auf dem Prozess der \isi{Konstruktionalisierung}. Es bleibt zukünftigen Studien überlassen, die Analysen in (formale) Entwürfe eines ahd. Konstrukionsnetzwerkes \is{Konstruktikon} zu integrieren und  mit späteren Entwicklungsstufen der \isi{Konstruktion} in Verbindung zu bringen. 
In Abschnitt~\ref{diskussion:der} erfolgt eine Diskussion zum kategorialen Wandel von \object{dër}.  In Abschnitt \ref{sec:disk-expansion} wird ein mehrdimensionales Expansionsmodell \is{Expansion} basierend auf den Faktoren \isi{Belebtheit},  \isi{Individualität} und \isi{Agentivität} vorgeschlagen. Anschlies\-send wird in Abschnitt \ref{sec:disk-ana-entrench} die Netzwerk-Perspektive eingenommen, indem Ana"-lo"-gie- und \is{Analogie} Entrenchmentprozesse \is{Entrenchment} betrachtet werden, die begünstigend oder auch blo"ckierend auf den Wandel eingewirkt haben. 

\section{Der funktionale Wandel von \object{dër}} \label{diskussion:der}

In den meisten diachron angelegten Untersuchungen zur Entwicklung des Definitartikels \is{Definitartikel} geht es um die Frage, ob und zu welcher Zeit ein ursprüngliches \isi{Demonstrativum} als \isi{Definitartikel} klassifiziert werden kann. Mögliche Antworten für den deutschen \isi{Definitartikel} liefert Abschnitt \ref{sec:abwann}. Anschließend werden in Abschnitt \ref{sec:disk-bruecken} Sprachdaten diskutiert, die  als Brückenkontexte \is{Brückenkontext} für den Übergang von pragmatischen \is{Pragmatische Definita} zu semantischen \is{Semantische Definita} Definitheitskontexten \is{Definitheitskontext} vorgeschlagen werden. Abschnitt \ref{sec:disk-entwicklung} stellt eine neue, aus den Daten abgeleitete Version des Grammatikalisierungspfades \is{Grammatikalisierungspfad} vor. 

\subsection{Ab wann ist \object{dër} ein Definitartikel?}\label{sec:abwann}

Die Frage, ab wann man in der deutschen Sprachgeschichte davon ausgehen kann, dass sich das ursprüngliche \isi{Demonstrativum} \object{dër} zum \isi{Definitartikel} gewandelt hat, ist nicht eindeutig zu beantworten -- zu heterogen ist die Datenlage und zu vielseitig sind die Kriterien, die den Artikelstatus rechtfertigen. Dennoch kann man aus den Ergebnissen der vorliegenden Untersuchung eine Annäherung wagen. 

Die Korpusuntersuchung hat offengelegt, dass \object{dër} bereits in den frühesten Texten äußerst frequent auftritt und als typischer Einleiter für definite Phrasen fungiert. Die hohe Frequenz und breite Kombinierbarkeit mit unterschiedlichen Substantivtypem \is{Substantiv}\is{Type}ist ein wichtiges Indiz dafür, dass das ursprüngliche \isi{Demonstrativum} schon zu Beginn der althochdeutschen Überlieferung über eine breite funktionale Spannweite verfügt. Auch die Tatsache, dass sich mit \object{dëser} bereits ein neuer Demonstrativartikel herausgebildet und der Definitheitszyklus \parencite{Greenberg1978,vanGelderen2007} von Neuem eingesetzt hat, spricht dafür, dass \object{dër} funktional breiter geworden ist und seinen Platz als prototypisches \isi{Demonstrativum} allmählich räumt \parencite[ähnlich auch][]{Schlachter2012}: Schon im Isidor ist nicht \object{dër}, sondern  \object{dëser} \is{Demonstrativum}das bevorzugte Mittel, um unmittelbare anaphorische \is{anaphorisch} Referenzbezüge im Text herzustellen \parencite[139]{Oubouzar1989}. Und diese gehören bekanntermaßen zum Hauptarbeitsfeld von Demonstrativa \is{Demonstrativum}  \parencite{Diessel1999}. Die Analysen zu Otfrids Evangelienbuch weisen ebenfalls darauf hin, dass die beiden Artikelwörter sich die Definitheitskontexte \is{Definitheitskontext} aufteilen: Während \object{dëser} nur in pragmatisch-definiten \is{Pragmatische Definita} Kontexten auftritt, dienen \object{dër}-Phrasen fast ausschließlich dazu, semantische \isi{Definitheit} \is{Semantische Definita} auszudrücken. 

Für definite Gebrauchskontexte, in denen Sprecherinnen und Sprecher von Artikelsprachen einen Artikel setzen \herkur{müssen} \parencite[832]{Himmelmann2001}, finden sich bereits in den frühesten Schriftstücken  \object{dër}-Belege. Dies wurde insbesondere an Superlativen deutlich, die schon im Isidor, dem ältesten Text der Untersuchung, mit \object{dër} auftreten. Sie sprechen dafür, dass das genuine \isi{Demonstrativum} seine funktionale Reichweite bereits im frühen Althochdeutschen in Richtung semantische \isi{Definitheit} \is{Semantische Definita}ausgebaut hat. Mit der Stichprobenanalyse (zufällige Auswahl von 100 \is{Nominalphrase (NP)} NPs) wurden in den drei untersuchten Schriftstücken (Isidor, Tatian und Otfrid) Belege von [\object{dër}\,+\,N] in abstrakt-situativen \is{abstrakt-situativ} Definitheitskontexten \is{Definitheitskontext} nachgewiesen, also den Definitheitskontexten, die mit \isi{Definitartikel}, aber nicht mit Demonstrativa \is{Demonstrativum} markiert werden können. Dabei nimmt der Anteil von determinierten zu undeterminierten Phrasen diachron zu: Im Isidor sind es ungefähr ein Fünftel, im Tatian ein Drittel und bei Otfrid die Hälfte. Für Notker steht eine solche Kontextanalyse noch aus. Es ist zu erwarten, dass hier der Anteil an \object{dër}-Phrasen noch größer ist. Hierfür sprechen die Analysen von Oubouzar, die bei Notker keine semantischen Restriktionen mehr für den Gebrauch von \object{dër} sieht \parencite[573]{Oubouzar1989}.\footnote{Eine Ausnahme sind Prädikative \is{Prädikativ} wie \object{Er ist der Lehrer}.} Ihre Beobachtungen werden durch die vorliegende Korpusuntersuchung \is{Korpus} gestützt. Nicht nur Superlative, sondern auch Unika \is{Unikum} werden bei Notker regelmäßig mit \object{dër} determiniert. Spätestens zu dieser Zeit kann man daher -- gemessen an der funktionalen Spannbreite -- \object{dër} den Status eines Definitartikels \is{Definitartikel} zuschreiben. 

Die Stichprobenanalyse hat darüber hinaus auch Belege für generische \is{generisch} Ausdrücke mit \object{dër} zu Tage gefördert. Dies ist insbesondere für den Isidor bemerkenswert, da bislang davon ausgegangen wurde, dass in diesem frühen Text generische \is{generisch} Ausdrücke ausschließlich in Form von blanken Nomen erscheinen \parencites()()[80]{Oubouzar1992}[145]{Kraiss2012}. Auch im etwas jüngeren Monseer Matthäus finden sich \textcite{Hodler1954} zufolge bereits generische \is{generisch} Referenzen mit \object{dër} (vgl. Abschnitt \ref{sec:nicht-referentiell}). Somit gehört der generische \is{generisch} Gebrauch schon seit dem frühen Althochdeutschen zum Arbeitsgebiet von [\object{dër}\,+\,N].  Allerdings verläuft die Durchsetzung hier viel zögerlicher als in anderen semantischen \is{Definitheitskontext} \is{Semantische Definita}Definitheitskontexten. Bis heute können generische \is{generisch} Phrasen im  Plural auch ohne \isi{Definitartikel} auftreten, vgl. \object{(die) Pandabären sind vom Aussterben bedroht} \parencite[][145]{Barton2015}. Zudem wird neben dem \isi{Definitartikel} auch der \isi{Indefinitartikel} als Generizitätsmarker \is{generisch} genutzt \parencite{Petrova2020}.

Mit der Untersuchung konnte also nachgewiesen werden, dass \object{dër} schon im frühen Althochdeutschen viel mehr ist als lediglich ein funktionales Aquivalent zum heutigen Demonstrativartikel, wie es u.a. von \textcite{Philippi1997} und \textcite{Demske2001} postuliert wird. Der Ausschnitt des Althochdeutschen, der uns über die Überlieferungen gegeben ist, setzt demnach erst ein, nachdem \object{dër} schon in für Demonstrativa \is{Demonstrativum} untypische Kontexte eingedrungen ist. 


\subsection{Brückenkontexte}\label{sec:disk-bruecken}

In Abschnitt \ref{sec:bruecke} wurden der anaphorische \is{anaphorisch} und der anamnestische \is{anamnestisch} Gebrauch als Brückenkontexte \is{Brückenkontext} gehandelt. Beide Typen sind in den Daten belegt, wenn auch nicht besonders häufig (vgl. die Zahlenwerte zu den ambigen Fällen in Abschnitt~\ref{sec:ergeb-defkontexte}). Bezüglich der Frage, wie der funktionale Wandel von Demonstrativ- \is{Demonstrativartikel} zu \isi{Definitartikel} vonstatten geht, sind ambige Fälle, die sich zwischen prag\-ma\-tisch-de\-fi\-ni\-ter \is{Pragmatische Definita} und semantisch-definiter \is{Semantische Definita} Lesart bewegen, besonders interessant, da sich an ihnen die \isi{Reanalyse} zum \isi{Definitartikel} rekonstruieren lässt. Sie werden nachfolgend an Beispielen\footnote{In den untersuchten Daten sind diese Fälle als \hervor{ambig} \is{Annotation}annotiert, vgl. \textcite{HZKYL4_2020}.} illustriert. 

 
%\subsubsection{Zwischen anaphorischem und abstrakt-situativem Gebrauch}

Die Korpusuntersuchung \is{Korpus} hat \object{dër}-Belege zum Vorschein gebracht, die auf den ersten Blick wie einfache anaphorische \is{anaphorisch} Wiederaufnahmen aussehen. In diesen Fällen befindet sich ein vorausgehender koreferenter Ausdruck in der unmittelbaren Textumgebung. Allerdings leitete die Wiederaufnahme -- wie sonst bei Demonstrativa \is{Demonstrativum} häufig (s. Abschnitt \ref{sec:anaphorisch}) -- keinen Topikwechsel \is{Informationsstruktur}\is{Topik}ein. Entweder ist der Referent bereits eindeutig als \isi{Topik} etabliert worden. Dies ist besonders gut an \object{heilant} zu sehen, der im Tatian fast immer mit \object{dër} wiederaufgenommen wird. Oder es gibt kein explizites Abgrenzungsmoment zu anderen potentiellen Referenten.
In diesen Fällen braucht der Rezipient das Antezedens sozusagen nicht mental zu aktivieren, damit die eindeutige Referenz \is{Referentialität} glückt. Die Identifizierung kann auch über das Weltwissen erfolgen, s. Beispiel \REF{ex:I2245} aus dem Isidor. Die \isi{Phrase} \object{oba} \object{dhem} \object{uuazsserum} bezieht sich auf ein kurz zuvor genanntes Bibelzitat \object{endi gotes gheist suueiboda oba uuazsserum} (I. 4,4).

% basiert auf
%\input{generated/examples/I2245}

\begin{exe}
\ex \label{ex:I2245} \gll {In dhiu} auh dhanne, dhazs ir \textit{oba} \textit{dhem} \textit{uuazsserum} suueiboda, dhen heilegun gheist dhar bauhnida  \\
{dabei} {auch} {dann}, {dass} {er} {auf} {dem} {Wasser} {schwebt}, {den} {heiligen} {Geist} {da} {zeigte}\\
\glt \extrans{Darin, dass er über d(ies)em Wasser schwebte, zeigte sich der heilige Geist} (I 4,4)
\end{exe}

\noindent 
Zwar ist hier  eine demonstrativ-wiederaufnehmende Lesart zugunsten einer höheren Expressivität nicht auszuschließen. Es ist aber auch möglich, die \isi{Phrase} als \herkur{über dem Wasser} zu übersetzen und damit eine abstrakt-situative \is{abstrakt-situativ} Lesart zu erhalten. 


%\subsubsection{Zwischen anamnestischem und abstrakt-situativem Gebrauch}

Das nachfolgende Beispiel illustriert die konzeptuelle Überschneidung von anamnestischer \is{anamnestisch}  und abstrakt-situativer \is{abstrakt-situativ} Lesart: Mit \object{thie geba} wird auf eine Gabe referiert, die einer gläubigen Leserschaft bekannt sein müsste und die mit dem restriktiven Nebensatz in Erinnerung gerufen wird. Der Relativsatz fungiert dann als \object{aktivierende Modifizierung} \parencite[78--79]{Himmelmann1997}, vgl. hierzu die Ausführungen in Abschnitt \ref{sec:amnamnestisch}.\largerpage

%basiert auf
%\input{generated/examples/T9827}

\begin{exe}
\ex \label{ex:T9827} \gll {gisih} {thaz} {thu} {iz} {niomanne} {ni} {quedes} {,} {ouh} {fár} {inti} {giougi} {thih} {themo} {biscofe} {inti} {bring} {\textit{thie}} {\textit{geba}} {thie} {thar} {gibót} {Moyses} {ín} {zi} {giuúiznesse} \\
{siehe} {dass} {du} {es} {niemandem} {nicht} {sagst} {} {sondern} {fahre} {und} {zeige} {dich} {dem} {Bischof} {und} {bringe} {die} {Gabe} {die} {da} {gebot} {Moses} {ihn} {zu} {Zeugnis} {}\\
\glt \extrans{Siehe, dass du es niemandem sagst, sondern gehe hin und zeige dich dem Priester und opfere die Gabe, die Moses befohlen hat zu einem Zeugnis über sie.} (T 46,4)
\end{exe}


\noindent 
Der Nebensatz kann aber auch als bloße Identifikationshilfe dienen, um der Leserschaft zu verdeutlichen, um welche Gabe es sich genau handelt, also als \object{etablierende Modifizierung} \parencite[79]{Himmelmann1997}. Die Referenz \is{Referentialität} basiert dann nicht auf der Aktivierung einer spezifischen Erinnerung. Sie gelingt alleine durch den Bezug auf einen durch das Weltwissen bekannten Referenten (hier: \object{Moses}). Der eingeführte Referent (\object{die Gabe}) wird dadurch eindeutig definiert.  In diesem Fall würde der Beleg unter die \object{nicht-familiären} Gebrauchskontexte fallen, in denen nach \textcite{Hawkins1978} nur \isi{Definitartikel}, aber keine Demonstrativa \is{Demonstrativum} möglich sind (vgl. Abschnitt \ref{sec:nicht-fam}). 

%\subparagraph*{Die Brücke zum generischen Gebrauch}
%
%Oben wurde gezeigt, dass generische Ausdrücken schon in den frühsten Texten mit \object{dër} auftreten, also noch bevor der Gebrauch in den semantisch-definiten Kontexte zur Regel wurde. Der Brückenkontext von referentiell-definiter zu nicht-referentiell-generischem Gebrauch sollte daher bei den pragmatischen Definita gesucht werden. Der nachfolgende Beleg aus dem Isidor könnte sowohl generisch als auch anamnestisch interpretierbar sein und deswegen ein möglichen Brückenkontext abbilden. %
%\input{generated/examples/I4134}

\subsection{Modellierung des Entwicklungspfades}\label{sec:disk-entwicklung}\largerpage[.25]

Es ist deutlich geworden, dass sowohl der anamnestische \is{anamnestisch} als auch der anaphorische \is{anaphorisch} Kontext leicht von einer anderen Lesart überlagert werden kann, nämlich dem Rückbezug auf das Weltwissen. Der Sprecher oder die Sprecherin kann mit  \object{dër} auf den Referenten, um den es geht, expressiv \herkur{zeigen} -- der Rezipient könnte diese Zeigegeste allerdings leicht ignorieren, wenn der Referent ohnehin eindeutig identifizierbar ist. Umgekehrt kann ein Rezipient \object{dër} als verbale Zeigegeste verstehen, selbst wenn sie vom Sprecher oder der Sprecherin gar nicht intendiert ist.
Der Einsatz von \object{dër} sichert sowohl beim anaphorischen \is{anaphorisch} als auch beim anamnestischen \is{anamnestisch} Gebrauch das Verständnis und macht deutlich, um welchen Referenten es geht. Je häufiger \object{dër} auf einen Referenten verweist, der auch ohne den Kontext oder den Rückbezug auf eine gemeinsame Erinnerung, sondern über das generelle Wissen etabliert wird, umso mehr rückt die expressive Zeigefunktion in den Hintergrund. Für immer mehr Diskursteilnehmer wird der nicht expressive Gebrauch zum Normalfall, so dass diese semantisch-definite \is{Semantische Definita} Lesart nach und nach zum Bestandteil der \isi{Konstruktion}[\object{dër}\,+\,N] wird. 

Anstatt den anaphorischen \is{anaphorisch} und anamnestischen \is{anamnestisch} Gebrauch als zeitlich aufeinanderfolgende Gebrauchskontexte auf dem Grammatikalisierungspfad \is{Grammatikalisierungspfad} abzubilden, schlage ich vor, dem funktionalen Wandel vom Demonstrativ- \is{Demonstrativartikel} zum \isi{Definitartikel} eine Phase vorauszuschicken, die diese beiden Kontexttypen gleichermaßen umfasst.\footnote{Es sei angemerkt, dass es theoretisch durchaus möglich ist, dass \object{dër} ursprünglich primär anaphorisch \is{anaphorisch} und erst später auch anamnestisch \is{anamnestisch} gebraucht wurde. Diese Chronologie lässt sich an den Daten allerdings nicht ablesen.} Darüber hinaus darf auf Basis der Daten ein generischer \is{generisch} \herkur{Seitenpfad} angenommen werden, der schon im frühen Althochdeutschen angelegt wird und parallel zur \herkur{Hauptstraße} der \isi{Grammatikalisierung} innerhalb der referentiell-de\-fi\-ni\-ten Kontexte verläuft, s. Abbildung~\ref{abb:expansion-definitheit}. Der Pfad basiert auf dem im Theorieteil der Arbeit vorgestellten Modell aus \textcite{Schmuck2014} (s. Abschnitt \ref{sec:stufen}). 

 
\begin{figure}
  \includegraphics[width=.75\textwidth]{images/diskussion_generisch.png}
\caption {Entwicklungspfad von [\object{dër}\,+\,N]\label{abb:expansion-definitheit}} 
\end{figure} 
 
\textcite[105]{Schmuck2014} argumentieren dafür, dass mit der Ausbreitung auf Unika \is{Unikum} der Weg für den onymischen \is{Eigenname} Artikel \is{Expletiver Artikel} geebnet wird, da sich hier eine konzeptuelle Überschneidung finden lässt. Die Frage, wie man den Übergang in Richtung generischen \is{generisch} Gebrauch modellieren müsste, ist hingegen noch offen. Es wäre denkbar, dass es Kontexte gibt, in denen eine Aussage über einen zuvor im Text oder Diskurs erwähnten Referenten gemacht wird, die auch generisch \is{generisch} interpretierbar ist, z.B. die Referenz auf eine bestimmte Gruppe, über die eine allgemeine Aussage gemacht wird, etwa: \object{Diese/Die Heiden glauben nicht an Jesus}. 
Auf Basis der Korpusuntersuchung \is{Korpus} sowie unter Bezugnahme der bisherigen Forschung\footnote{Für den Tatian hat die Stichprobenuntersuchung bspw. keinen Beleg für eine generische \is{generisch} Referenz mit \object{dër} hervorgebracht. \textcite{Oubouzar1992} hat diesen Gebrauchstypus allerdings eindeutig nachgewiesen, vgl. ausführlich Abschnitt \ref{sec:nicht-referentiell} der vorliegenden Arbeit.} können die untersuchten Textdenkmäler in diesem Entwicklungspfad verortet werden. Wie man in Abbildung~\ref{abb:expansion-definitheit} sieht, übernimmt  \object{dër} in allen Texten bereits Aufgaben aus der Domäne des \is{Definitartikel} Definitartikels, d.h. der \is{Semantische Definita} semantischen Definita. Unika \is{Unikum} sind bereits im Tatian und noch häufiger auch bei Otfrid in \object{dër}-Phrasen zu finden. Bei Notker steht \object{dër} bereits an der Schwelle zu den \is{Eigenname} Eigennamen, da es hier vereinzelt schon als syntaktisch motivierter Artikel auftritt \parencite[\object{der einrihtigo cato} \extrans{der unbeugsame Cato}, s.][638]{Oubouzar1989}.
Die eigentliche Expansionsphase \is{Expansion} für den onymischen Artikel \is{Expletiver Artikel} setzt allerdings erst im Frühneuhochdeutschen ein \parencite{Schmuck2020}. 


\section{Expansionspfade von [\object{dër}\,+\,N]} \label{sec:disk-expansion}

Der funktionale Wandel von \object{dër}, der im vorhergehenden Abschnitt aufgezeigt wurde, spiegelt die von \textcite[32--33]{Himmelmann2004} postulierte \is{Expansion} \object{semantic-prag\-mat\-ic context expansion}. Neben diesem Expansionspfad \is{Expansion} beschreibt Himmelmann zwei weitere Richtungen, in die sich das \isi{Demonstrativum} auf seinem Weg zum \isi{Definitartikel} ausbreiten kann: Die \is{Expansion} \object{host-class expansion}, also die Ausbreitung von [\object{dër}\,+\,N] auf neue semantische Substantivklassen und die \is{Expansion} \object{syntactic context expansion}, d.h. die Ausbreitung von zentralen zu weniger zentralen Argumentpositionen \parencite[32--33]{Himmelmann2004}, vgl. Abschnitt \ref{sec:parameter}. Mit den Ergebnissen aus der Korpusuntersuchung \is{Korpus} können diese groben Stufen der Kontextexpansion \is{Expansion} weiter ausdifferenziert werden.  

\subsection{Host-class expansion}

Eine der zentralen Hypothesen, die im Theorieteil der Arbeit formuliert wurde, lautet, dass die kontinuierliche Ausbreitung von [\object{dër}\,+\,N] auf neue Substantivklassen (\object{host-class expansion}) \is{Expansion} belebtheitsgesteuert \is{Belebtheit} verläuft. Es wurde ein erweitertes Belebtheitskonzept \is{Belebtheit} zugrunde gelegt: Die Hauptstufen \textsc{menschlich > belebt > unbelebt} wurden auf Basis der aus der Forschung bekannten Belebtheitshierarchien \is{Belebtheitshierarchie} \parencite[u.a.][]{Comrie1989,Yamamoto1999,Croft2006,Enger2011} am unteren (\hervor{unbelebten}) Ende der Skala um Abstrakta \is{Abstraktum} und \isi{Massennomen} erweitert. Diese weisen nicht nur konzeptuell eine maximale Entfernung zum Menschen auf, sondern verfügen auch über einen geringen \is{Individualität} Individualitätsgrad. Da sich diese Eigenschaften nur schlecht mit dem emergierenden \isi{Definitartikel} in seiner Rolle als Individualisierer \is{Individualität} (s. Abschnitt \ref{sec:individualisierer}) und Marker für Diskursprominenz (s. Abschnitt \ref{sec:kata}) vertragen, gehört diese Substantivklasse zur letzten Bastion des appellativen \is{Gattungsname} Wortschatzes, die sich der obligatorischen Definitheitskennzeichnung widersetzt. Zudem korreliert der Faktor (kulturelle) \isi{Relevanz} (Abschnitt \ref{sec:relevanz}) mit einem hohen \is{Belebtheit} Belebtheitsgrad. 

Die Ergebnisse der Korpusuntersuchung \is{Korpus} haben gezeigt, dass die \object{host-class expansion} \is{Expansion}tatsächlich belebtheitsgesteuert \is{Belebtheit} verläuft. Eine schematische Zusammenfassung bietet Abbildung~\ref{abb:expansion-belebtheit}. Sie beruht auf den Ergebnissen aus Abschnitt \ref{sec:ergeb-faktoren}. Je intensiver die Farbe, umso stärker ist die Präferenz bzw. die Ablehnung der \isi{Konstruktion}gegenüber der jeweiligen Kategorie. 

\begin{figure}
  \includegraphics[width=\textwidth]{images/belebtheitsexpansion-neu.jpg}
\caption {Expansion von [\object{dër}\,+\,N] entlang der Belebtheitskala\label{abb:expansion-belebtheit}} 
\end{figure} 
  
In den frühesten Denkmälern -- dem Isidor und dem Monseer Matthäus -- ist noch kein signifikanter Einfluss der \isi{Belebtheit} zu beobachten. Allerdings lässt sich an den Daten zum Monseer Matthäus zumindest die Tendenz erkennen, dass belebte Referenten eher determiniert werden als abstrakte, während im älteren Isidor auch Abstrakta \is{Abstraktum} zu \object{dër} tendieren. Eine Erklärung könnte im Faktor \isi{Relevanz} liegen, der durch die Inventur der häufigsten Lemmata \is{Lemma} mit und ohne \object{dër} aufgedeckt werden sollte:  Im Isidor steht die Determinierung im Dienste einer eindeutigen Argumentationsführung. Dabei werden sowohl zentrale biblische Referenten regelmäßig mit \object{dër} ausgestattet (\object{forasago, magad}) als auch abstrakte (\object{drinissa}). Im Monseer Matthäus sind es vor allem Referenten, die in den Herrschaftsstrukturen zur Zeit Jesus an der Spitze stehen (\object{ewawart, herizoho, herizo}). Im Tatian begünstigt nicht nur die kulturelle Wichtigkeit die Determination, sondern vor allem das Merkmal [+menschlich]. So stehen bspw. \object{heilant, keisar} und \object{grafo} regelmäßig mit \object{dër}. Interessanterweise scheint im Tatian und bei Notker auch das Geschlecht eine Rolle zu spielen. Der Anteil an \object{dër}-Phrasen fällt bei weiblichen Referenten niedriger aus als bei männlichen; bei Otfrid ist die Verteilung allerdings genau umgekehrt.

Andere belebte \is{Belebtheit} und konkrete \is{Konkretum} Referenten zeigen im Tatian keine Präferenz hinsichtlich der Determination. Abstrakta \is{Abstraktum} bleiben hingegen signifikant häufiger undeterminiert. Bei Otfrid und Notker treten auch Tiere und Konkreta \is{Konkretum} präferiert mit \object{dër} auf. In diesen Texten sind außerdem die Schnitte zwischen den Belebtheitskategorien \is{Belebtheit} weniger scharf, so dass man schließen kann, dass die \object{dër}-Setzung innerhalb der obersten Belebtheitsstufen \is{Belebtheit} zu dieser Zeit schon generalisiert wurde. Für Massenomen hat die Untersuchung nur Einzelbelege mit \object{dër} hervorgebracht, so dass keine Expansion \is{Expansion} in diesen Bereich nachgewiesen werden konnte.  

Aus dem empirischen Befund für das Althochdeutsche lassen sich drei implikative Expan"-sions"-stadien \is{Expansion} modellieren, s. \is{Belebtheitshierarchie} 
 \REF{ex:belebtheitsskala}. Implikativ meint Folgendes:  
Wenn eine Sprache Appellativa \is{Gattungsname} der dritten Stufe regelmäßig mit \isi{Definitartikel} ausstattet, ist anzunehmen, dass auch Appellativa \is{Gattungsname} aus der zweiten und ersten Stufe obligatorisch determiniert werden; das zweite Stadium impliziert das erste. Die Skala müsste an weiteren Sprachen überprüft werden.  

\begin{exe}
	\ex \label{ex:belebtheitsskala} (Kulturell) relevante \is{Relevanz} und menschliche Referenten (1)\\>  belebte \is{Belebtheit} und unbelebte \is{Konkretum} Konkreta (2)\\> Abstrakta \is{Abstraktum} und \isi{Massennomen} (3)
	\end{exe}


\subsection{Syntactic-context expansion} \label{sec:syn-expansion}

Für die \object{syntactic-context expansion} \is{Expansion}nennt \textcite{Himmelmann2004} zwei Phasen: Zunächst erfasst der emergierende Artikel zentrale Satzglieder (\isi{Subjekt} oder \isi{Objekt}), später auch weniger zentrale Argumente, d.h. solche, die formseitig als \hervor{adpositional expressions} realisiert werden, darunter Präpositionalobjekte \is{Objekt} und Adverbiale \is{Adverbial} \parencite{Himmelmann1998}. In Abschnitt \ref{sec:partizipanten} wurde diese Art der \isi{Expansion} mit der semantischen Rolle \is{Semantische Rolle} in Verbindung gebracht; sie ist also  die Folge einer semantischen \isi{Expansion}. Während die \is{Subjekt} Subjekts- und Objektsposition \is{Objekt} meist zentralen Patizipanten einer Situation entsprechen, werden weniger zentrale (und meist fakultative) Partizipanten als Adverbiale \is{Adverbial} realisiert \parencite{Lehmann2004a}. 

In der Untersuchung wurden formale Kriterien genutzt, um sich dieser Partizipanten"=Opposition zu nähern und zwar, indem alle Präpositionalphrasen \is{Präpositionalphrase (PP)} in den Blick genommen wurden. Es hat sich gezeigt, dass die Struktur [Präp\,+\,N] zahlenmäßig in allen Denkmälern deutlich gegenüber [Präp\,+\,\object{dër}\,+\,N] dominiert. Eine diachrone Ausweitung lässt sich an den Daten nicht eindeutig nachweisen, da der Anteil an determinieren Phrasen vom ältesten bis zum jüngsten Text nicht linear zunimmt. Eine Ausweitung des Definitartikels \is{Definitartikel} auf Präpositionalphrasen \is{Präpositionalphrase (PP)} muss also zu einem späteren Zeitpunkt in der Sprachgeschichte einsetzen. 

Darüber hinaus wurde qualitativ vorgegangen und 100 NPs im Isidor, Tatian und Otfrid in Agens und Nicht-Agens \is{Agentivität} eingeteilt. Es zeigt sich, dass die Agens-Belege \is{Agentivität} stärker zur \object{dër}-Setzung tendieren als die Nicht-Agens-Fälle. In den zwei jüngeren Texten ist dieser Unterschied auch signifikant. Die Daten weisen darauf hin, dass die Merkmalskombination [+Agens] \is{Agentivität} und [+(über-)menschlich] am häufigsten eine Determinierung auslösen. In zukünftigen Studien könnte dieser Hypothese mit der Analyse größer Datenmengen auf den Grund gegangen werden. Interessant wäre in diesem Zusammenhang auch, wie die Variablen \isi{Belebtheit} und \object{dër}-Setzung bei anderen semantischen Rollen \is{Semantische Rolle} ausgeprägt sind, etwa beim Patiens, das sowohl mit belebten als auch mit unbelebten Referenten besetzt werden kann.   

\section{Analogien und Entrenchment im NP-Netzwerk} \label{sec:disk-ana-entrench}

Die Entwicklung des Definitartikels \is{Definitartikel} wird von strukturellen Umbauprozessen auf phrasaler Ebene begleitet. Mit der Korpusuntersuchung \is{Korpus} wurden die Anfänge dieser Prozesse offengelegt. Die Ergebnisse können jetzt genutzt werden, um Schematisierungsprozesse \is{Schematisierung} abzuleiten und analogische Relationen \is{Analogie} im NP-Kon"-struktions"-netzwerk \is{Konstruktikon} zu modellieren. In Abschnitt \ref{sec:disk-schema} wird die \isi{Konstruktionalisierung} von \is{Definitartikel} [Definitartikel\,+\,N] vor dem Hintergrund eines sich etablierenden Determiniererschemas \is{Determiniererschema} beleuchtet. Gegenstand von Abschnitt \ref{sec:disk-weg-block} sind mögliche \textit{Wegbereiter} und \textit{Blockaden}, die den Wandel beeinflusst haben.  

\subsection{Determiniererschema} \label{sec:disk-schema}

In den vorherigen Abschnitten wurde gezeigt, wie sich der \isi{Definitartikel} durch die einzelnen Definitheitskontexte \is{Definitheitskontext} arbeitet und immer mehr Substantivklassen erfasst. Diese \isi{Expansion} verdeutlicht, dass sich [\object{dër}\,+\,N] im Laufe des Althochdeutschen von einer Demonstrativ- \is{Demonstrativartikel} zu einer Definitartikelkonstruktion \is{Konstruktion}\is{Definitartikel}wandelt. 
Wie in Abschnitt \ref{sec:entrenchment} erläutert wurde, kann man davon ausgehen, dass mit diesem Wandel ein Entrenchmentprozess \is{Entrenchment} einhergegangen ist: Die Kombination mit immer neuen Substantiv-Types \is{Substantiv}\is{Type}sorgt für die kognitive Verfestigung des Schemas \is{Definitartikel} [Definitartikel\,+\,N], welches dazu dient, definite Referenten zu kennzeichnen, s. Abbildung~\ref{abb:entrenchment-ther}. 

\begin{figure}
\begin{center}
  \includegraphics[width=\textwidth]{images/expansionbw.pdf}
\caption {Herausbildung eines Definitartikelschemas} 
\label{abb:entrenchment-ther}
\end{center}
\end{figure} 

Die Korpusuntersuchung \is{Korpus} dokumentiert, wie \object{dër} nach und nach an Gebrauchsfrequenz gewinnt und damit die Determiniererposition besetzt (vgl. Abschnitt \ref{sec:ergeb-ther-freq}). Dies geschieht zu Lasten von anderen pränominalen Determinierern. Die \hervor{Leittragenden} sind vor allem die pränominalen \is{Genitivattribut} Genitivattribute, welche an den rechten Rand der \isi{Phrase} gedrängt werden \is{Nominalsyntax}\is{Wortstellung}\parencite[zur weiterführenden Diskussion s.][]{Demske2001}. Und andererseits die \is{Possessivum} Possessivartikel -- die  Referenten aus der Klasse der Körperteile an \object{dër} abtreten (s. Abschnitt \ref{sec:ergeb-belebtheit}). 

Die Proxy-Suche \is{Proxy} nach NP-Strukturtypen \is{Nominalphrase (NP)} (s. Abschnitt \ref{sec:ergeb-np-struktur}) hat offengelegt, dass der nominale Kopf bereits im frühen Althochdeutschen (im Isidor) in ca. der Hälfte der Belege von einem pränominalen Element begleitet wird. Bei der Mehrzahl handelt es sich um definite Einleiter -- neben dem ursprünglichen \isi{Demonstrativum} \object{dër}, sind es \is{Possessivum} Possessiva, demonstratives \object{sëlb} oder \object{dëser} \is{Demonstrativartikel}sowie \is{Genitivattribut} Genitivattribute. Man kann schlussfolgern, dass Sprecherinnen und Sprecher auf Basis dieser unterschiedlichen Determinierertypen ein übergeordnetes \isi{Schema} abstrahieren, bestehend aus einem definiten pänominalen Slot\,+\,N \parencite[ähnlich fürs Altenglische][]{Sommerer2011}. 
  
\begin{figure}
\begin{forest} for tree = {edge={thick,dashed},grow=west,child anchor=parent}
[DET\,+\,N\\Determiniererschema,align=center,name=det,l sep=1cm,calign=child,calign primary child=3,parent anchor=173
  [\textit{dër}\,+\,N [Token,tier=token] [Token,tier=token] [Token,tier=token]]
  [Possessivum\,+\,N [Token,tier=token] [Token,tier=token] [Token,tier=token]]
  [\textit{sëlb}\,+\,N [Token,tier=token] [Token,tier=token] [Token,tier=token]]
  [\textit{dëser}\,+\,N [Token,tier=token] [Token,tier=token] [Token,tier=token]]
  [Genitivattribut\,+\,N [Token,tier=token] [Token,tier=token] [Token,tier=token]]
]
\end{forest}
\caption {Herausbildung eines \is{Determiniererschema} Determiniererschemas\label{abb:schematisierung}}
\end{figure}
 
Der gemeinsame Nenner für die Besetzung des Slots ist die definite Bedeutungskomponente. Das ursprüngliche \isi{Demonstrativum} ist ein guter Kandidat, um diesen Slot regelmäßig zu besetzen. Im Vergleich zu \object{dëser} oder \object{sëlb} hat es eine viel generellere Bedeutung und anders als ein Possessivum \is{Possessivum} oder Genitivattribut \is{Genitivattribut} setzt sein Gebrauch nicht notwendigerweise eine Besitzrelation oder sonstige Zugehörigkeit voraus. Das Funktionsspektrum von \object{dër} hat also die größte Reichweite, was die  Kombinierbarkeit mit unterschiedlichen N-Types \is{Type} begünstigt.

Wichtig für die Herausbildung des Determiniererschemas \is{Determiniererschema} sowie die Obligatorisierung von \object{dër} in pränominaler Position ist auch die Frage der \is{Wortstellung} Stellungsfestigkeit. Die Korpusuntersuchung \is{Korpus} hat offengelegt, dass die Voranstellung \is{Wortstellung} der \isi{Determinierer} (neben \object{dër} wurden Possessiva \is{Possessivum} und Indefinita \is{Indefinitartikel} wie \object{ein} oder \object{sum} untersucht, s. Abschnitt \ref{sec:ergeb-np-struktur}) in allen Texten klar dominiert. Die meisten Nachstellungen \is{Wortstellung} finden sich bei Otfrid. Da dies der einzige poetische Text ist, lassen sich diese Fälle durch den Einfluss der \isi{Metrik} zu erklären. Neben der Stellung \is{Wortstellung} ist auch die Anzahl der Phraseneinleiter schon in den frühesten Texten stark reguliert. Fälle, in denen zwei flektierende \is{Flexion} \isi{Determinierer} gleichzeitig vor einem Bezugsnomen stehen, sind nur vereinzelt belegt. Auch die Struktur [Artikelwort\,+\,pränominaler Genitiv\,+\,N] ist sehr selten und kommt nur im ältesten Text mit knapp 2\% auf der Liste der zehn häufigsten Strukturtypen vor. Für den jüngsten Text, Notkers Boethius, hat die Untersuchung keine solche Phrasen dokumentiert. Die zunehmende Abwanderung attributiver Genitive an die postnominale Position sorgt zusätzlich dafür, dass der linke, phraseneinleitende Slot die \object{Default}-Position für flektierende \is{Flexion} Elemente wird. Aus der Tatsache, dass die pränominalen Elemente in Bezug auf \is{Kasus}\is{Genus}Kasus-, Genus- und Numerusinformationen \is{Numerus} mit ihrem Bezugsnomen übereinstimmen, kann man ableiten, dass Sprecherinnen und Sprecher die Phraseneinleiter als eigene, auf formaler \isi{Analogie} basierende Kategorie begreifen und sich eine schematische Konstruktion \is{Konstruktion}\is{Flexion}\is{Phrase}[flektierender Phraseneinleiter + N] kognitiv einschleift. Die \isi{Schematisierung} steht im Dienste des klammernden Verfahrens \is{Nominalklammer} \parencite{Ronneberger-Sibold1994,Ronneberger-Sibold2010a,Szczepaniak2010,Szczepaniak2011a,Flick2018}. Man kann davon ausgehen, dass die paradigmatische Einbindung in die Klasse der Phraseneinleiter sich positiv auf die Obligatorisierung von [\object{dër} + N] auswirkt.     


\subsection{\herkur{Wegbereiter} und \herkur{Blockaden}} \label{sec:disk-weg-block}

Die Korpusuntersuchung \is{Korpus} hat gezeigt, dass in allen Texten eine klare Korrelation zwischen schwach flektierten Adjektiven \is{Adjektiv} und \object{dër}-Setzung vorliegt. Die Korrelation ist semantisch begründet: Ein schwaches \is{Flexion} Adjektiv \is{Adjektiv} sorgt für eine individualisierende Lesart des Bezugsnomens (vgl. hierzu ausführlich Abschnitt \ref{sec:flexion}). Die Kombination mit \object{dër} unterstreicht diese Lesart zusätzlich.  Die einzelnen Elemente der \is{Nominalphrase (NP)} NP sind aber nicht nur semantisch miteinander verknüpft, sondern auch über ihre formalen Merkmale. Denn sowohl das Artikelwort als auch das Adjektiv \is{Adjektiv} und das Bezugsnomen stimmen in \isi{Kasus}, \isi{Genus} und \isi{Numerus} überein. Die daraus resultierende kooperative \is{Nominalklammer} \isi{Flexion} hilft, Mehrdeutigkeiten, die durch Synkretismen im Flexionssystem \is{Flexion} entstehen, aufzulösen \parencite[127]{Szczepaniak2010}. Die hohe Frequenz und die enge sowohl semantische als auch morphosyntaktische Verzahnung sind ein Indiz dafür, dass Sprecherinnen und Sprecherinnen die Struktur \is{Adjektiv} [\object{dër}\,+\,Adjektiv\textsubscript{schwach} + N] in ihrer Gesamtheit als \isi{Schema} abspeichern. In diesem syntaktischen Kontext wird der Gebrauch von \object{dër} also schon früh zum Normalfall. Dies \is{Schematisierung} kurbelt die \isi{Reanalyse} des ursprünglich demonstrativen und fakultativen \object{dër} zum obligatorischen Definitheitsmarker an. In dieser Funktion kann das Artikelwort analogisch \is{Analogie} auf andere Kontexte übertragen werden und so die Entwicklung auf Systemebene vorantreiben. Dass ein schwach flektiertes \is{Flexion} Adjektiv \is{Adjektiv} fast immer die Setzung von \object{dër} zur Folge hat, könnte auch für die Superlativ-Konstruktionen von Bedeutung sein: Das \isi{Schema} [\object{dër} + Adjektiv\textsubscript{schwach} + N], das sich vor allem auf Formen im Positiv und Komparativ bezieht, könnte analogisch \is{Analogie} auch auf Superlative ausgeweitet worden sein. Dass die \object{dër}-Setzung vor attributiv gebrauchten Adjektiven \is{Adjektiv} im Superlativ etwas häufiger ist als bei den substantivierten \is{Substantivierung} Superlativen,  stützt diese Vermutung. 

Aus den Daten lassen sich darüber hinaus auch ganz spezifische \isi{Token} herausgreifen, die auf ähnliche Weise wie die bisher genannten Schemata dem funktionalen Wandel von \object{dër} dienlich sind und die somit zu \herkur{Wegbereitern} für die Entwicklung werden. Gemeint sind Phrasen wie \object{dër heilant} oder \object{dër ewawart}, die eine hohe Tokenfrequenz aufweisen und daher als eigene \isi{Konstruktion} abgespeichert werden \is{Token-Entrenchment} \parencite[= Token-Entrenchment, s.][]{Ziem2013}. Das Besondere ist, dass diese Konstruktionen zum Ausdruck semantischer \is{Semantische Definita} \isi{Definitheit} genutzt werden. So ist mit  \object{dër heilant} im Tatian immer Jesus gemeint und damit ein eindeutig identifizierbarer Referent. Auch Belege von  \object{ewawart} (\extrans{Hohepriester}) oder  \object{herizoho} (\extrans{Statthalter}) referieren immer auf einen einzigen Referenten innerhalb der (Stadt-)Gemeinschaft. Sie werden im Monseer Matthäus in allen Fällen mit \object{dër} determiniert, obwohl die zusätzliche situative \is{situativ} Verortung eigentlich redundant ist. Solche Belege können daher als abstrakt-situativ \is{abstrakt-situativ} (s. Abschnitt \ref{sec:abst-sit}) eingeordnet werden und damit als erste Instanzen des Schemas \is{Definitartikel} [Definitartikel\,+\,Nomen] gelten (ähnlich: \object{der Kaiser}, \object{die Jünger}). Weil die einzelnen Bestandteile solcher Kollokate für Sprecherinnen und Sprecher transparent sind, können sie als Vorbild für die analogische \is{Analogie} Ausbreitung des Schemas [\object{dër} + N] dienen (vgl. Abschnitt \ref{sec:entrenchment}). 

Nicht nur \herkur{Wegbereiter}, sondern auch \herkur{Blockaden} wurden in den Daten sichtbar: So bleiben \is{Nominalphrase (NP)} NPs, die in PPs eingebettet sind, in der großen Mehrzahl undeterminiert. Die \object{dër}-Resistenz kann mit semantischen Restriktionen erklärt werden: PPs fungieren sehr häufig als Adverbiale \is{Adverbial} und enthalten damit nicht-referentielle Nomen, z.B. \object{in costunga} (\extrans{in Versuchung}) oder auch Unika \is{Unikum} (\object{fon mittilgarte} \extrans{von Erdkreis}, T 1,78). In beiden Fällen ist keine zusätzliche Hervorhebung oder Markierung der Identifizierbarkeit notwendig, so dass die Struktur [Präp\,+\,N] häufiger vorkommt als [Präp\,+\,\object{dër}\,+\,N] (vgl. auch Abschnitt \ref{sec:syn-expansion}). Sie kann sich dadurch als \isi{Schema} kognitiv einschleifen, was den Gebrauch von \object{dër} blockiert. Im Laufe der Sprachgeschichte wurde -- im Rahmen der zunehmenden Obligatorisierung von [\object{dër}\,+\,N] -- dieses \isi{Schema} entweder aufgebrochen: Statt \object{fon mittilgarte} ist heute bspw. die \isi{Phrase} \object{vom Erdkreis} (mit klitisiertem Artikel) konventionalisiert. Oder es hat sich die artikellose Variante durchgesetzt (z.B. \object{in Versuchung}).
