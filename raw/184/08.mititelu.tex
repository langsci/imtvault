\documentclass[output=paper]{langsci/langscibook}
\ChapterDOI{10.5281/zenodo.1182601}
\author{Verginica Barbu Mititelu\affiliation{Romanian Academy Research Institute for Artificial Intelligence}%
\lastand Svetlozara Leseva\affiliation{Institute for Bulgarian Language, Bulgarian Academy of Sciences}}
\title{Derivation in the domain of multiword expressions}


\abstract{Multiword expressions and derivation have rarely been discussed
together, even though analyzing the interaction between them is of great
importance for the study of each topic and, in general, for the
 study of the language and for Natural Language Processing. Derivation is a
means of enriching the lexicon with both words and multiword
expressions. Various types of derivation (suffixation, prefixation or
both, as well as other derivational devices) can act upon either words
or multiword expressions. The focus of our work here is the formation
of multiword expressions from other multiword expressions via
derivation. We analyze the morphological, syntactic and semantic
aspects of this process, providing examples from Romanian and
Bulgarian, languages,  which belong to different families but  have been in contact throughout
their history. The study can be further extended with data from other
languages. The perspective adopted here is paradigmatic, but the
syntagmatic approach, which can only be mentioned as further work, will add
to the quality of the analysis of facts: corpus data will contextualize
the phenomena discussed here and offer quantitative information about
them.}

\maketitle

\begin{document}

\section{Introduction}
\label{introduction}

\is{morphology!suffixation|(}
\is{morphology!derivation|(}

Widely accepted as a difficult task to deal with, the identification of
multiword expressions (MWEs) in processing natural languages becomes
even more difficult when the MWEs are new creations in the language or
even ad-hoc creations in the text as a result of the linguistic
creativity of speakers, usually carrying an emotional load  (\ref{ex:8:1}):\footnote{
As convention of writing: 
%\renewcommand{\theenumi}{(\roman{enumi})}%
\begin{itemize}
\item[(i)]  We adopt
the use of the international two letter code of the country in which the language is spoken in front
of each example to mark the language to which it belongs: RO for
Romanian, BG for Bulgarian. 
\item[(ii)] We show the base
MWEs on the left and the derived MWEs on the right.
\end{itemize}
}

 


\is{affix!suffix}
\begin{exe}
\ex \label{ex:8:1}
\settowidth\jamwidth{RO}
\textit{a băga de seamă – băgător de seamă}  \jambox{(RO)}
\glt ‘to pay attention to’ – ‘(the one) who only watches without playing any
role (in the action)’
\end{exe}


In example (\ref{ex:8:1}) the latter MWE is derived from the
former and carries a negative connotation.

While the interest in the origin of MWEs has been manifested in all
languages, specialists have normally investigated the social, economic,
ethnographic, and other aspects motivating the process of turning certain
word combinations into MWEs. When the origins cannot be found in
the national background, MWEs are attributed to other languages, so
they are borrowings or linguistic calques. Another (language internal)
source of MWEs can be found in the inventory of already existing
MWEs. In this paper we focus on one type of MWE formation: derivation
from other MWEs, as shown in \REF{ex:8:1}. We put together two topics that have
rarely been discussed together in the same study.



On the one hand, MWEs have been classified and characterized according
to syntactic \is{syntactic variability} and morphological variability \is{morphology!morphological variability} (\citealt{Nunberg1994}; \citealt{Sag:2002}; \citealt{baldwin2003}; \citealt{Baldwin2010}, among others)
and/or semantic decomposability (\citealt{Nunberg1994}; \citealt{baldwin2003}, among others), as well as according to types of idiomaticity
(\citealt{baldwin2004, baldwin2006}, among others). From a morphological perspective,
only inflection and the reflexive form of verbs were discussed for each
type of MWE (\citealt{Sag:2002}; \citealt{Savary:2008}).

\is{morphology!derivation|(}
On the other hand, derivation is a process defined as involving words \linebreak
\citep{marouzeau1933}: it is the process of creating new words out of
existing ones, by means of attaching or detaching affixes to or
from a stem respectively, the latter type being better known as
back-formation. An example of derivation is the word \textit{survival}, created
by attaching the suffix \textit{\textendash al} to the stem \textit{survive}. An example of
back-formation is the verb \textit{to back-form}, obtained from
\textit{back-formation} by removing the suffix \textit{\textendash ation}. However, derivation
can act both on words and MWEs. In the former case, it always results
in a new word; in the latter, it creates either a new word or a new
MWE, as we will show below.

 
 
In the literature dedicated to either of the two topics (derivation,
MWEs), one can identify two predominant trends: on the one hand, the
discussion about derivation has always implied that words are the output and only rarely
MWEs; on the other hand, the discussion about MWEs has implied, from
time to time, reference to derivation: this interest has also been
expressed, although sporadically, in studies on phraseology,
particularly in analyzing the behavior of idioms with respect to their
derivational morphology. 



In this chapter, we describe the way derivation affects MWEs, providing
examples from Romanian and Bulgarian, languages which belong to different language families (Romance and Slavic, respectively) but have had a long history of contact. We focus on MWEs derived from other MWEs,
highlighting morphological, syntactic and semantic modifications
triggered by these transformations.



In both Romanian and Bulgarian, derivation is much more productive than
compounding or other internal means of enriching the vocabulary.
Moreover, progressive derivation is more frequent than back–formation.
In both languages suffixation is the prevalent derivational means.
Prefixation in Romanian is much less productive. Bulgarian has a very
developed deverbal verb formation as verbal prefixes express
aktionsart and the language has a rich Aktionsart system. Cases of
prefixation were not found in our data involving cross\textendash part\textendash of\textendash speech
derivation. Derivation affects all content word classes, simple or
compound words.

\is{morphology!derivation|)}

\is{affix!prefix}
\section{Types of lexemes derived from MWEs}


When subject to derivation, MWEs can serve as bases for the creation of
either other MWEs or words. We discuss these types in the subsections
below. In our discussion, we will use the term \textsc{base MWE} to denote the
MWE that serves as the input to the derivation process. 

\subsection{MWEs derived from MWEs}
\label{mwesfrommwes}

We offer here some examples of MWEs derived from MWEs, in both \ili{Romanian}
and \ili{Bulgarian}:


\begin{exe} 
\ex \label{ex:8:2}
\settowidth\jamwidth{RO}
\gll a mustra cugetul (pe cineva) -- mustrare de cuget\\ 
to chide  the.conscience  (on someone) -- chiding by conscience \\ \jambox{(RO)}
\glt ‘to have remorse’ -- ‘having remorse’
\end{exe}

\begin{exe}
\ex \label{ex:8:3}
\settowidth\jamwidth{BG}
\glll съвестта гризе \textnormal{(}някого\textnormal{)} -- гризене на съвестта\\
săvestta grize (nyakogo) -- grizene na săvestta\\
the.conscience  gnaws (someone) --  gnawing of 
the.conscience\\ \jambox{(BG)}
\glt ‘to have remorse’ -- ‘having remorse’
\end{exe}


\begin{exe}
\ex \label{ex:8:4}
\settowidth\jamwidth{RO}
\textit{cronică literară -- cronicar literar} \jambox{(RO)}
‘literary review’ -- ‘literary reviewer’
\end{exe}
 
\begin{exe}
\ex \label{ex:8:5}
\settowidth\jamwidth{BG}
\gll моден дизайн -- моден дизайнер\\
moden dizayn -- moden dizayner\\ \jambox{(BG)}
\glt `fashion design' -- `fashion designer'
\end{exe}
 

These examples show that different types of MWEs can feed derivations:
idioms in (\ref{ex:8:2}) and (\ref{ex:8:3}), terms in (\ref{ex:8:4}) and compounds in (\ref{ex:8:5}).



One content word (usually the syntactic head) of the base MWE is subject
to affixation: e.g.\@ in (\ref{ex:8:2}) above \textit{mustrare} (noun) is derived from the
verb \textit{a mustra} with the suffix \textit{–re}; in (\ref{ex:8:4}) \textit{cronicar} is derived
from \textit{cronică} (the head of the base MWE noun phrase) with the agentive
suffix \textit{–ar}. Likewise, in (\ref{ex:8:3}) the process of derivation is carried out
by means of suffixation of the verb 
\lnexlat{гриза}{griza}{gnaw} with the suffix \lnex{–не},
thus obtaining the deverbal noun 
\lnexlat{гризене}{grizene}{gnawing}, and in (\ref{ex:8:5}) the head noun
\lnexlat{дизайнер}{dizayner}{designer} is obtained from the noun дизайн, dizayn,  `design' by means of the agentive
suffix  \lnex{–ер}.


\is{affix!suffix|(}
\subsection{Words derived from MWEs}
\label{wordsfrommwes}

When one content word of the MWE is subject to affixation and the
derived word has the semantic content of the base MWE, we regard this as words being  derived from MWEs; the other words of the base MWE simply do not
occur in the result of the derivation:


\begin{exe}
\ex \label{ex:8:6}
\settowidth\jamwidth{RO}
\gll a face un lucru mușama -- a mușamal\textbf{iza}  \\
 to make a~ thing oilcloth \\ \jambox{(RO)}
\glt {‘to cover something up’}
\end{exe}

\begin{exe}
\ex \label{ex:8:7}
\settowidth\jamwidth{BG}
\glll
извадя \textnormal{(}бизнес, \ldots\textnormal{)} на светло – изсветл\textbf{я} \textnormal{(}бизнес, \ldots\textnormal{)} \\
izvadya (biznes,  \ldots) na svetlo -- izsvetl\textbf{ya}
(biznes,  \ldots)\\
 bring.\textsc{v} (business, \ldots) to light -- make.brighter.\textsc{v} (business, 
\ldots) \\ \jambox{(BG)}
\glt {‘to legalize (business, \ldots)’}
\end{exe}
  
\begin{exe}
\ex \label{ex:8:8}
\settowidth\jamwidth{RO}
  \textit{a face la rotisor -- rotis\textbf{a}} \jambox{(RO)}
‘to cook in a rotisserie’
\end{exe}

\begin{exe}
\ex \label{ex:8:9}
\settowidth\jamwidth{RO}
  \textit{instala\textbf{tor }de gaze -- gaz\textbf{ist}} \jambox{(RO)}
‘gas installer’
\end{exe}
 
\begin{exe}
\ex \label{ex:8:10}
\settowidth\jamwidth{BG}
  \gll въздух под налягане -- въздух\textbf{ар}\\
  văzduh pod nalyagane -- văzduh\textbf{ar}\\ \jambox{(BG)}
\glt{`air under
pressure’ -- ‘an unreliable or incompetent person (especially one who pretends otherwise)’} 
\end{exe}
 
This type of derivation involves semantic condensation as one of the
content words of the MWE, the one that carries most of the semantic
load, takes up the meaning of the whole. The word may be adapted
morphologically to express the relevant part of speech, for example by
means of \is{affix!suffix}\is{morphology!suffixation} suffixation with a verbal suffix, e.g.\ RO \textit{–iza} (\ref{ex:8:6}), where
the noun \textit{mușama} ‘oilcloth’ yields the derivative verb \textit{mușamaliza},
by means of \is{morphology!back-formation} back–formation, e.g.\ (\ref{ex:8:8}) where the verb \textit{rotisa} is created 
from \textit{rotisor}, or parasynthetically, e.g.\@ BG \lnex{из–}, \lnex{–я} (\ref{ex:8:7}), where
the nominalized adjective светло, svetlo, `light' gives the verb 
\lnexlat{из–светл–я}{iz–svetl–ya}{make brighter}. In
addition, noun suffixes, such as the agentive suffixes RO \textit{–ist} (\ref{ex:8:9})
and BG \lnex{–ар} (\ref{ex:8:10}), express the semantic role of the derived noun.

These types of derivation seem to affect \is{multiword expression!collocation} collocations (\ref{ex:8:8}),  terms (\ref{ex:8:9}) and \is{multiword expression!idiom}
idioms (\ref{ex:8:6}), (\ref{ex:8:7}), (\ref{ex:8:10})  alike. In Romanian linguistics, the phenomenon has
been described as very frequent and systematic \citep{groza2011}. However,
no quantitative support has been offered for these claims and, as a
consequence, we will not adhere to this estimation. In the Bulgarian
literature, the specialists have remarked that while dephraseologization
(and semantic condensation) is a productive process in the contemporary
language, word–formation processes, including derivation, are
relatively rare \citep{blagoeva2011}. We will not investigate this
phenomenon here.


\is{affix!suffix|)}
\is{morphology!suffixation|)}
\section{Data selection and processing}
\label{dataselection}
In order to study the behavior of MWEs with respect to derivation, we
worked with an inventory of MWEs extracted from big Romanian and
Bulgarian dictionaries containing MWEs.



For Romanian, this inventory was created starting from DELS (Dictionary
of Expressions, Idioms and Collocations) \citep{muaruanduc2010}. The
dictionary was automatically parsed, MWEs were extracted and those
marked as archaic were eliminated, along with expressions, as they are
unproductive with respect to derivation; for the remaining 11,158 MWEs
(collocations and idioms) we looked for derivationally related MWEs by
searching the web and manually inspecting the results. Only for
about 500 MWEs could we find derivationally related MWEs. So, a first
remark is the relatively low impact that derivation has on MWEs, at
least judging from the Romanian data. This may be a reason why the two
phenomena have rarely been discussed together.



The Romanian MWEs were preprocessed and annotated morphosyntactically:
they were automatically tokenized, lemmatized, tagged for
part-of-speech (PoS) and chunked using the TTL web service \citep{ion2007}.
Each word form in the MWE was identified, lemmatized and a PoS tag
containing information about its part of speech and morphosyntactic
characteristics (number, gender, case, etc., depending on the PoS) was
attached to it. Syntactic groups were identified and marked as such, they
 are called chunks and are useful for the analysis in  \sectref{section6}.


The Bulgarian data were excerpted from a large electronic dictionary of
MWEs \citep{Stoyanova2014}. Named entities were removed since
they are unproductive with respect to the phenomena explored in this work.
The remaining MWEs were inspected and other unproductive types, such as
proverbs, sayings and other expressions, were also filtered out
automatically, using as a filter the code of the relevant type of MWE.
Finally, obsolete and dialect entries were manually removed. The
resulting dictionary of 4,039 entries consists predominantly of verb
idioms  and support verb constructions. The number of entries reflects
two facts: (i) many Bulgarian verbs form aspectual pairs, whose members
are distinct lexemes with their own inflectional and derivational
morphology; therefore, unless there are semantic restrictions to the
contrary, two MWE entries (one headed by a perfective aspect verb and
one by an imperfective aspect verb) were encoded in the dictionary;
\is{morphology!prefixation}
(ii) (to a lesser degree) prefixation is a regular process which creates new
verbs. Although the prefixed verbs meaning is modified to
a lesser or to a greater extent, a variant of the MWE headed by a
prefixed verb is often formed, thus the word family of a verb idiom may
include a number of derived verb idioms. In the dictionary we kept only
the more frequent MWEs, derived through prefixation, basically those
bearing resultative meaning, e.g.\@ (\ref{ex:8:15}). We found derivational MWEs for
2,612 entries in the dictionary, with a great prevalence of deverbal 
MWEs. The data were additionally supplemented with examples collected
by the authors, adding up to 2,725 pairs.



The MWEs were automatically tokenized, lemmatized and PoS-tagged using
the Bulgarian Language Processing Chain (LPC) \citep{koeva2011},
which is available as a web service, and subsequently chunked using a
stand-alone tool which uses the LPC output \citep{Stoyanova2015}. As a result, all the words in each MWE were marked with
the relevant grammatical information, and the basic syntactic structure
of the MWEs (head, dependent syntactic groups) was identified and
marked explicitly.




\is{morphology!derivation|)}
\section{Derivation types in the domain of MWEs }
\is{morphology!suffixation|(}

In this section we present the types of derivation detected in the
domain of MWEs: progressive derivation by means of suffixes, prefixes
or both, back-\linebreak formation and zero-derivation.


\subsection{Progressive derivation}

\is{morphology!progressive derivation}
The vast majority of derivation cases are \textbf{progressive} (i.e.,
MWEs are created by adding affixes to a word in a previously existent
MWE). In Bulgarian and Romanian, these affixes can be suffixes,
prefixes or both. Each subtype will be discussed in the subsections below.

\subsubsection{Suffixation}
\label{suffication}

\is{morphology!suffixation|(}
In Romanian all 339 cases of progressive derivation are represented by \is{affix!suffix}
\textbf{suffixation}. In Bulgarian almost all of the 2,704 instances of
progressive derivation are accounted for by suffixation, with the
exception of 10 cases of parasynthetic derivation. The productivity of
the suffixes in the two languages is represented in  \tabref{tab:8:1}.%
\footnote{Abbreviations used in  \tabref{tab:8:1}: Ag: Agent; Ev: Event;  Instn: Institution; Instr: Instrument; L: Language; Re: Result; SVs: Semantic Values; St: State.}





\begin{table}
\begin{tabular}{p{0.4cm} p{1.2cm} r p{1cm} p{6.9cm}}
\lsptoprule
{ \textbf{L}} &
{ \textbf{Suffix}} &
{ \textbf{P}} &
{ \textbf{SVs}} &
{ \textbf{Examples}}\\
\midrule
{ RO} &
{ \textit{–re}} &
{ 305} &
{ Ev} &
{ \textit{a–și băga mințile în cap},  to insert
one’s minds into head, ‘to come to reason’ -- \textit{băgare a minților în
cap}, inserting one’s minds into head, ‘coming to reason’}\\
{ RO} &
{ \textit{–(ă)tor}} &
{ 18} &
{ Ag} &
{ \textit{a face rele} ‘to do bad things’ --
\textit{făcător de rele} ‘wrongdoer’}\\
{ RO} &
{ \textit{–ie}} &
{ 7} &
{ St}

~

{ Ev}

~

{ Instn} &
{ \textit{sărac lipit} ‘dog poor’ -- \textit{sărăcie
lipită} ‘extreme poverty’}

{ \textit{călători de plăcere} ‘to travel for
pleasure’ -- \textit{călătorie de plăcere} ‘travelling for pleasure’}

{ \textit{judecător de pace} ‘justice of the
peace’ -- \textit{judecătorie de pace} ‘the court of a justice of the
peace’}\\

{ RO} &
{ \textit{–ătură}} &
{ 4} &
{ Ev} &
{ \textit{a–ți arunca ochii} ‘to cast a
glance’ –\textit{aruncătură de ochi} ‘glance’}\\
{ BG} &
{ \textit{–ne}} &
{ 2,604} &
{ Ev, Re} &
{ \textit{pisha istoriya}, to write history,
‘to make history’ – \textit{pisane na istoriya} ‘making of history’}\\
{ BG} &
{ \textit{–ba}} &
{ 5} &
{ Ev, Re} &
{ \textit{prodam na edro}, to sell in bulk,
‘to wholesale’ – \textit{prodazhba na edro} ‘a wholesale’}\\
{ BG} &
{ \textit{–ach}} &
{ 42} &
{ Ag}

~

~

{ Instr} &
{ \textit{svalyam zvezdi}, to take down stars,
‘to promise the moon’ -- \textit{svalyach na zvezdi} ‘one who promises the
moon’}

{ \textit{hvashtam brimki} ‘to mend
ladders/stitches (e.g.\@ in stockings)’ -- \textit{hvashtach na brimki}  ‘a tool
for mending ladders’}\\

{ BG} &
{ \textit{–or/–er/–ir}} &
{ 13} &
{ Ag} &
{ \textit{komandvam parada}, to command the
parade, ‘to call the shots’ -- \textit{komandir na parada} ‘one who calls
the shots’}\\

{ BG} &
{ \textit{–tel}} &
{ 10} &
{ Ag} &
{ \textit{stroya văzdushni kuli} ‘to build
castles in the air’ -- \textit{stroitel na văzdushni kuli} ‘one who builds
castles in the air’}\\

{ BG} &
{ \textit{–ets}} &
{ 4} &
{ Ag} &
{ \textit{tărguvam na edro} ‘to deal
wholesale’ -- \textit{tărgovets na edro} ‘wholesaler’}\\
\lspbottomrule
\end{tabular}
\caption{Suffixes. Their productivity and semantics.}
\label{tab:8:1}
\end{table}

%Abbreviations used in  \tabref{tab:8:1}: Ag: Agent; Ev: Event;  Instn: Institution; Instr: Instrument; L: Language; Re: Result; SVs: Semantic Values; St: State.

Other Romanian suffixes (\textit{–a}, \textit{–ime}, \textit{–iza}) are much less productive
in the set of pairs we dealt with, with only one or a maximum of two
occurrences. In Bulgarian, other suffixes denoting events or results of
events are instantiated with only a few examples in the data: –еж (three
cases), \lnex{–ов}  (one case), \lnex{–ица}  (one case). The noun suffix \lnex{–ост}, which
denotes properties, is found in three cases. Other agentive suffixes are
\lnex{–ант}/\lnex{–ент} (two cases), \lnex{–ия}  (two cases), \lnex{–ик}  (one case). The suffixes
\lnex{–ура}, \lnex{–ище}, \lnex{–ция} are found with institutions (one example per
suffix).


There are cases when the same MWE serves as a derivational base for two
different MWEs. There are two ways in which this can be achieved. The
first one is through separate derivational paths. The derivative MWEs
in (\ref{ex:8:11}) and (\ref{ex:8:12}) are formed through independent derivational processes:



\begin{exe}
\ex \label{ex:8:11}
\settowidth\jamwidth{RO}
\begin{xlist}
\ex \label{ex:8:11a}
\textit{a aduce laudă  -- aducere de laudă} \jambox{(RO)}
‘to give praise’ -- ‘giving praise’
\ex \label{ex:8:11b}
\textit{a aduce laudă  -- aducător de laudă} \jambox{(RO)} ‘to give praise’ -- ‘the one
who gives praise’
\end{xlist}
\end{exe}



\begin{exe}
\ex \label{ex:8:12}
\settowidth\jamwidth{BG}
\begin{xlist}
\ex \label{ex:8:12a} 
\gll разбивам сърца -- разбиване на сърца\\
razbivam sărtsa  -- razbivane na sărtsa\\ \jambox{(BG)}
‘to break hearts’ -- ‘breaking of hearts’ 
\ex \label{ex:8:12b} 
\gll разбивам сърца -- разбивач на сърца\\
razbivam sărtsa  -- razbivach na sărtsa\\ \jambox{(BG)}
‘to break hearts’ -- ‘heartbreaker’ 
\end{xlist}
\end{exe}




The verb MWEs in (\ref{ex:8:11}) and (\ref{ex:8:12}) undergo suffixation and yield either an
eventive noun (by means of the suffixes \textit{–re} and \lnex{–не}, respectively)
or an agentive one (by the suffixes \textit{–tor} and \lnex{–ач}, respectively)
in the derivationally related MWEs. 

\is{morphology!suffixation|)}



\begin{table}[b]
\begin{tabularx}{\textwidth}{l@{~~}p{19mm}@{}>{\raggedleft}p{1.5cm}Q}
\lsptoprule
{ \textbf{Language}} &
{ \textbf{Pattern}} &
{ \textbf{\parbox{15mm}{\hspace*{-5mm} \mbox{Productivity}}}} &
{ \textbf{Example}}\\ 
\midrule
{ RO} &
{ V–N–N} &
{ 3} &
\mbox{\textit{a ști carte}, to know book, ‘to be educated’}\newline 
\textit{știință de carte}  ‘education’ \newline 
\textit{neștiință de carte}
‘lack of education’\\\midrule
{ RO} &
{ V–A–A} &
{ 3} &
\mbox{\textit{a ști carte}, to know book, `to be educated'} \newline 
(\textit{știutor de carte}, 
`educated') \newline
\textit{neștiutor de carte}  ‘uneducated’\\\midrule

{ BG} &
{
V–N\textsubscript{AGENT}–N\textsubscript{LOCATION}} &
{ 3} &
{ \textit{pera pari}  ‘to launder money’\newline
\textit{perach na pari} ‘money launderer’\newline
\textit{perachnitsa na pari}  ‘a business involved in money
laundering’}\\
\lspbottomrule
\end{tabularx}
\caption{Multiple derivations.}
\label{tab:8:2}
\end{table}


\is{morphology!suffixation|)}
\is{morphology!derivation|(}

The second way to form two or more MWEs from the same source follows
several steps along a single derivational path. We spotted six such
instances in the Romanian data and three in Bulgarian (\tabref{tab:8:2}).
Typologically, the examples are different: in Romanian, the
noun\textendash to\textendash noun derivation yields antonyms. In Bulgarian, the derived
nouns lexicalize different semantic roles in the eventuality denoted by
the corresponding verb. Due to the small number of instances, no
conclusions can be reached for either of the languages. More examples
from these languages (as well as from others) would help to better
understand possible derivations. 


Besides, as the verbs belonging to a given aspectual pair in Bulgarian
are characterized by their own derivational morphology and derivational
patterns, MWEs (just like single words) headed by different members of
an aspectual pair may serve as a base for derived MWEs with similar
semantics, e.g.\@ the imperfective aspect verb gives rise to an eventive
nominalization \REF{ex:8:13a} , while the perfective aspect counterpart yields a
different deverbal MWE with an eventive (and possibly resultative)
interpretation \REF{ex:8:13}:



\begin{exe}
\ex \label{ex:8:13}
\settowidth\jamwidth{BG}
\begin{xlist}
\ex \label{ex:8:13a} 
\gll побеждавам по точки -- побеждаване по точки\\
pobezhdavam po tochki  -- pobezhdavane po tochki\\ \jambox{(BG)}
\glt{‘to outpoint, to outscore’ -- ‘outpointing’}
\ex \label{ex:8:13b} 
\gll победя по точки -- победа по точки\\
pobedya po tochki  -- pobeda po tochki\\ \jambox{(BG)}
\glt{‘to outpoint, to outscore’ -- ‘outpointing’}
\end{xlist}
\end{exe}



\is{affix!prefix|(}
\subsubsection{Parasynthetic derivation}
\label{parasynthetic}

\is{morphology!parasynthetic derivation}
Another derivational device detected only in the Bulgarian data is
parasynthetic derivation, when both a suffix and a prefix are attached
to an existing word. All ten cases we found in the data represent
derivations of verbs from adjectives:


\begin{exe}
\ex \label{ex:8:14}
\settowidth\jamwidth{BG}
\gll гладен като вълк -- огладнея като вълк\\
gladen kato vălk -- ogladneya kato vălk\\ \jambox{(BG)}
\glt{‘as hungry as a wolf’ -- ‘to become as hungry as a wolf’}
\end{exe}

\subsubsection{Prefixation}
\label{prefixation}

\is{morphology!prefixation}\is{affix!prefix}
Prefixation alone rarely serves as a means for deriving new MWEs in
Romanian (see the examples of consecutive derivation in \tabref{tab:8:2}). In
the Bulgarian data MWEs resulting from verb to verb derivation (\ref{ex:8:15}),
where prefixation is a productive process, were included as separate
entries in the dictionary and will not be discussed further below:

\begin{exe}
\ex \label{ex:8:15}
\settowidth\jamwidth{BG}
\gll пера пари -- из\textendash пирам пари\\
pera pari -- iz\textendash piram pari\\ \jambox{(BG)}
\glt{‘to launder money’ -- ‘to launder money up’ \textnormal{(resultative meaning)}}
\end{exe}

We made this decision because the derivationally related verb MWEs have
different (although related) meanings and can themselves be subject to
derivation, e.g.\@ 
\lnexlat{пера пари}{pera pari}{to launder money} -- 
\lnexlat{пра\textendash не на пари}{pra\textendash ne na pari}{money laundering},
\lnexlat{из\textendash пирам пари}{iz\textendash piram pari}{to launder money up} -- 
\lnexlat{из\textendash пира\textendash не на пари}{iz\textendash pira\textendash ne na  pari}{money laundering}, resultative
meaning.

\subsection{Back-formation}
\label{backformation}

\is{morphology!backformation}
We found only one case of back-formation in Romanian, in which the verb
\textit{lucra} is derived from the noun \textit{lucru} (\ref{ex:8:16}), and six cases in Bulgarian
(\ref{ex:8:17}), all of which are neologisms:


\begin{exe}
\ex \label{ex:8:16}
\settowidth\jamwidth{RO}
\textit{lucru de mână -- a lucra de mână} \jambox{(RO)} 
‘handiwork’ -- ‘to work by hand’
\end{exe}

\begin{exe}
\ex \label{ex:8:17}
\settowidth\jamwidth{BG}
\gll промиване на мозъци -- промивам мозъци\\
promivane na mozătsi -- promivam mozătsi\\ \jambox{(BG)} 
\glt {‘brainwash\textnormal{(}ing\textnormal{)})’ -- ‘to brainwash’} (example from \citealt{blagoeva2008})
\end{exe}

These data reflect a tendency noted in works on Bulgarian terminology
and neology (\citealt{baltova1986}; \citealt{kolkovska1993}; \citealt{kostova2013}, among others) concerning the creation of eventive nouns, in particular nouns
ending in \lnex{–не} or ending in a verbal suffix followed by \lnex{–не}  that do
not have a verb counterpart. The corresponding verbs are often formed
by back-formation (\ref{ex:8:17}) and the newly created verbs or verb MWEs can be
subject to further derivations:

\begin{exe}
\ex \label{ex:8:18}
\settowidth\jamwidth{BG}
\gll промивам мозъци -- промивач на мозъци\\
promivam mozătsi  --  promivach na mozătsi\\ \jambox{(BG)} 
\glt {‘to brainwash’ -- ‘brainwasher’}
\end{exe}


\is{affix!prefix|)}
\subsection{Zero-derivation (conversion)}
\label{zero}

\is{morphology!zero-derivation}
Fifteen cases in the Bulgarian data represent the process of conversion (also
called zero-derivation) in which the derived MWE is formed without the
attachment of a suffix and/or a prefix and usually involves detachment
of a grammatical affix such as the inflection:

\begin{exe}
\ex \label{ex:8:19}
\settowidth\jamwidth{BG}
\gll ударя под кръста -- удар под кръста\\
udarya pod krăsta  -- udar pod krăsta\\ \jambox{(BG)} 
\glt{‘to hit below the belt’ -- ‘a hit below the belt’}
\end{exe}


With Romanian MWEs, conversion manifests itself in two ways: (i) the
participle form functions as an adjective with more than 150 verb MWEs;
(ii) the supine form of several verb MWEs functions as a noun. The
participle and the supine are homonymous non-finite verb forms.
However, the discussion below will exclude such cases (and therefore
zero-derivation) and will focus only on affixal derivation.



\section{The morphological classes of the MWE heads involved in
MWE derivation}
\label{section5}


Тhe formal description and analysis of the basic syntactic structure of
MWEs and their representation in the lexicon are important for the
encoding and prediction of some of the major morphological and
syntactic properties of the MWEs, such as: the components that are likely to
inflect; the possibilities for modification by optional elements
(optional elements are placed in brackets), e.g.\@ BG \lnexlat{пера (мръсни) пари}{pera  (mrăsni) pari}{launder (dirty) money}; the possibility for eliding modifiers
with no change in meaning (placed in square brackets in this example),
e.g.\@ 
\lnexlat{вдигам летвата $[$високо$]$}{vdigam  letvata [visoko]}{raise the bar [high]}; paradigmatic
restrictions on agreement, on singular/plural forms, and so forth.
Among others, the syntactic analysis makes it possible to predict the
potential of MWEs for derivation and the structural changes that may take place
in this process (see \sectref{section6}).



The majority of the Romanian pairs extracted from the DELS involve verbs
as bases for derivation. The most frequent type is represented by pairs
of MWEs displaying verb nominalization, while derivative pairs
involving other parts of speech are much rarer (see \tabref{tab:8:3}). 



For Bulgarian 2,725 derivative pairs were found. The difference in the
number of pairs as compared with the initial set of 4,039 entries is
due largely to the fact that the perfective aspect verbs in the set are
very unproductive with respect to the derivational processes discussed.
Deverbal noun formation accounts for the majority of cases (2,663),
with much smaller numbers for the opposite noun to verb pattern, verb
to adjective, adjective to verb, noun to noun, adjective to noun (see
 \tabref{tab:8:3}). 

 \begin{table}
 \small
\begin{tabularx}{\textwidth}{lrrQQ}
\lsptoprule
\rotatebox{90}{\parbox{2.6cm}{\textbf{{Stem PoS}-\\{derived word PoS}}}} & 
\rotatebox{90}{\parbox{2.6cm}{{\textbf{\#RO\\[-.3em] examples}}}}  & 
\rotatebox{90}{\parbox{2.6cm}{{\textbf{\#BG\\[-.3em] examples}}}}   &
\textbf{Example from the RO data}   & 
\textbf{Example from the BG data}\\ 
\midrule
V–N & 349 & 2,663 & \textit{a depune jurământul} `to take the oath’      & \textit{potrivam rătse} `to rub (one's) hands’  \\
    &     &       &  \textit{depunerea jurământului} `taking the oath’   &   \textit{potrivane na rătse} `rubbing of (one's) hands’\\   
									 
\tablevspace									
V–A & 2 & 16 & \textit{a sări în ochi}, to jump into eyes, ‘to be straightforward’ & \textit{mălcha kato păn}, to keep silent like a log, ‘to be as mute as a maggot/fish’\\
    &   &    & \textit{săritor în ochi}, jumping into eyes, ‘straightforward’      & \textit{mălchaliv kato păn}, silent like a log, ‘(as) mute as a poker’\\         
										            
\tablevspace									
N–V & 4 & 8 & \textit{semnal luminos} ‘light signal’                & \textit{igra na nervi} ‘a battle of nerves’\\
    &   &   & \textit{a semnaliza luminos} ‘to signal with lights’ &   \textit{igraya na nervi} ‘to lead a battle of nerves’\\
								    
\tablevspace									
N–N & 5 & 18 & \textit{judecător de pace} ‘justice of the peace’                  &\textit{voenen prokuror} ‘military prosecutor’ \\
    &   &    & \textit{judecătorie de pace} ‘the court of a justice of the peace’ &  \textit{voenna prokuratura} ‘military prosecutor's office’\\
										  
\tablevspace									
A–N & 4 & 10 & \textit{sărac lipit} ‘dog–poor’             & \textit{nisht duhom} ‘poor in spirit’   \\
    &   &    & \textit{sărăcie lipită} ‘extreme poverty’  & \textit{duhovna nishteta} ‘spiritual poverty’\\
							          
\tablevspace									
A–V & – & 10 & – & \textit{byal kato platno} ‘as white as a sheet’\\
    &   &    & 	 &   \textit{pobeleya kato platno} ‘to become as white as a sheet’\\
\lspbottomrule
\end{tabularx}
\caption{Morphological alternations occurring in MWE derivations.}
\label{tab:8:3}
\end{table}

\clearpage 


\is{syntactic reorganization|(}
\section{Syntactic reorganizations resulting from derivations}
\label{section6}

\isi{Dependency Grammar} is used as a syntactic framework for our discussion.
In this framework, verbs admit subjects, complements and adjuncts,
nouns (even those derived from verbs) admit modifiers and adjectives
admit complements. Syntactic functions are understood as in \cite{quirk1985}.



Out of the total number of 414 Romanian pairs, fifty do not undergo any
internal reorganization in the process of derivation; in Bulgarian this
holds true for 54 out of the 2,725 pairs: 

\begin{exe}
\ex \label{ex:8:20}
\settowidth\jamwidth{RO}
\textit{agent de publicitate -- agenție de publicitate} \jambox{(RO)} 
‘advertising agent’ -- ‘advertising agency’
\end{exe}





\begin{exe}
\ex \label{ex:8:21}
\settowidth\jamwidth{BG}
\gll военен прокурор -- военна прокуратура\\
voenen prokuror – voenna prokuratura\\ \jambox{(BG)} 
\glt {‘military prosecutor’ -- ‘military prosecutor's office’}
\end{exe}

In (\ref{ex:8:20}), \textit{de publicitate} receives the same syntactic analysis in both
MWEs: it is a modifier of the nouns \textit{agent} and \textit{agenție},
respectively. In (\ref{ex:8:21}) the adjectives 
\lnexlat{военен}{voenen}{military} and 
\lnexlat{военна}{voenna}{military} which modify
the head noun 
\lnexlat{прокурор}{prokuror}{prosecutor}  and 
\lnexlat{прокуратура}{prokuratura}{prosecutor's office}, respectively, have the same
analysis. 



The cases without syntactic reorganization include the noun to noun,
verb to adjective and adjective to verb patterns. In the following
sections we will deal with the other two structural types of MWEs found
in the data, that is: verb to noun and noun to verb MWEs.



The syntactic structure of the base MWE determines whether the syntactic
expression of a dependent phrase is obligatory. For instance, a direct
object NP\textsubscript{DO} (\textsubscript{DO }stands for direct
object) that is not a fixed part of a base MWE but is licensed by a
transitive verb, as illustrated below, is not an obligatory
dependent of the derived MWE, while an internal argument that is a
fixed part of the base MWE is an obligatory component of the derived
MWE. For example, in BG,  
\lnexlat{пъхвам (някого) зад решетките}{păhvam  (nyakogo) zad reshetkite}{put (someone)
behind bars}, the internal argument position (NP\textsubscript{DO}) is
not a fixed part of the idiom; rather, it is an open position that is
filled by a suitable entity. In the nominalization 
\lnexlat{пъхване (на някого) зад решетките}{păhvane (na nyakogo) zad  reshetkite}{putting (of someone) behind bars} the position
corresponding to the direct object may be left empty. On the contrary, if the NP\textsubscript{DO} is a fixed part of the MWE, it cannot
be omitted, e.g.\@ 
\lnexlat{кърша ръце}{kărsha rătse}{wring hands} -- 
\lnexlat{кършене на ръце}{kărshene na rătse}{wringing of hands}. The syntactic structure of the base MWE also
determines the word order of obligatory and non\textendash obligatory components
in the derived MWE (e.g.\@ typically the object of the base MWE is closer
to the deverbal noun than other base MWE components).



Next, we present the syntactic reorganizations observed in derived MWEs,
as we found them in the available data for Romanian and Bulgarian.
Their documentation facilitates text processing. Given the limited MWE
dictionaries available, all knowledge facilitating the automatic
morphosyntactic analysis of text is considered valuable. Below we offer
rules that algorithms can use to process new MWEs which are derived from existing ones.

\is{syntactic reorganization|)}

\is{affix!suffix|(}
\subsection{Verb PP or AdvP complement or adjunct -- noun modifier}
\label{section61}

This pattern is observed when a verb MWE is related with a noun MWE via
\is{morphology!derivation} derivation (see (i1) below) or the other way round (i2). It accounts
for 260 Romanian and 792 Bulgarian pairs:\footnote{Patterns are enumerated in the text with Roman numbers.}


\renewcommand{\theenumi}{(\roman{enumi})}%
\begin{enumerate}
\item[(i)] (1) VP [V PP/AdvP] > NP [N\textsubscript{V–derived }PP/AdvP]\\ 
(2) NP [N PP/AdvP] > VP [V\textsubscript{N–derived} PP/AdvP]
\end{enumerate}


The verb admits a prepositional phrase (PP) or an adverbial phrase
(AdvP) functioning as a complement or an adjunct in the MWE, but it can
also admit other modifiers placed out of the MWE. Through derivation,
the constituents, except for the head word, preserve their syntactic
category and internal structure, as can be noticed in (i), where the
form of the modifying phrase is the same. Although semantically the
dependent functions similarly in the NP and the VP, its syntactic role
is different according to our analysis: when the head is a verb, we
analyse the particular dependent as a complement or an adjunct and,
when the head is a noun, we analyse it as a modifier.



Below we indicate the syntactic category (PP, AdvP) and the status
(complement, adjunct, modifier) of the dependent phrases. Complement or
adjunct status is determined with respect to the argument structure of
the verb that heads the respective MWE.



In the verb MWE in (\ref{ex:8:22}), \textit{de credință} is the prepositional object
(i.e., a complement) of the reflexive verb \textit{a se lepăda}, whereas in
the noun MWE the same PP is a modifier of the noun \textit{lepădare} 
derived from \textit{a se lepăda} with the suffix \textit{–re}. Likewise, in the BG
verb MWE in (\ref{ex:8:23}) the Goal PP 
\lnexlat{в джоба}{v dzhoba}{in the pocket} is the prepositional object of
the verb 
\lnexlat{бъркам}{bărkam}{thrust one's hand}, whereas, in the noun MWE, it functions as a modifier
of the noun 
\lnexlat{бъркане}{bărkane}{thrusting one's hand}, derived from the verb \lnexl{бъркам}{bărkam} by means of the
suffix \lnex{–не}. 



\begin{exe}
\ex \label{ex:8:22}
\settowidth\jamwidth{RO}
\textit{a se lepăda de credință -- lepădare de credință} \jambox{(RO)}
‘to depart from the faith’ -- ‘departing from the faith’
\end{exe}

\begin{exe}
\ex \label{ex:8:23}
\settowidth\jamwidth{BG}
\glll бъркам в джоба (на някого) -- бъркане в джоба\\
bărkam v dzhoba (na nyakogo) -- bărkane v dzhoba \\
{thrust.one’s.hand}.\textsc{v} in the.pocket (of someone) \\ \jambox{(BG)}
\glt {‘to incur expenses (on someone)’ -- `incurring of expenses’}
\end{exe}



The PP \textit{în peniță} (\ref{ex:8:24}) is a modifier in the first MWE, and an adjunct
of the verb in the second MWE and the verb is derived from the noun in
the former MWE. In (\ref{ex:8:25}), the PP 
\lnexlat{под кръста}{pod krăsta}{below the belt} is an adjunct of the verb
in the first MWE and a modifier of the noun 
\lnexlat{удар}{udar}{hit} in the second MWE
and the noun is derived from the verb 
\lnexlat{ударя}{udarya}{hit}.


\begin{exe}
\ex \label{ex:8:24}
\settowidth\jamwidth{RO}
\textit{desen în peniță  -- a desena în peniță} \jambox{(RO)}
‘pen drawing’ -- ‘to draw in pen’
\end{exe}

\begin{exe}
\ex \label{ex:8:25}
\settowidth\jamwidth{BG}
\gll ударя под кръста -- удар под кръста\\
udarya pod krăsta -- udar pod krăsta\\ \jambox{(BG)}
\glt{‘to hit below the belt’ -- ‘a hit below the belt’}
\end{exe}

In (\ref{ex:8:26}) the adverb \textit{aminte} is a complement of the verb in the former
MWE and a noun modifier in the latter. In (\ref{ex:8:27}) the adverb \lnexlat{отвисоко}{otvisoko}{from above} is
an adjunct of the verb \lnexlat{гледам}{gledam}{look} in the former MWE and a noun modifier
in the latter. 

\begin{exe}
\ex \label{ex:8:26}
\settowidth\jamwidth{RO}
\textit{a lua aminte  -- luare aminte} \jambox{(RO)}
‘to take into consideration’ -- ‘taking into consideration’
\end{exe}

\begin{exe}
\ex \label{ex:8:27}
\settowidth\jamwidth{BG}
\gll гледам отвисоко (някого / нещо) -- гледане отвисоко\\
gledam otvisoko (nyakogo / neshto) -- gledane otvisoko\\ \jambox{(BG)}
\glt{‘to look down on (someone / something)’ -- ‘looking down on (someone / something)’}
\end{exe}


\is{affix!suffix|)}

In both languages derivation from a noun MWE to a verb MWE is much rarer:

\begin{exe}
\ex \label{ex:8:28}
\settowidth\jamwidth{BG}
\glll игра на нерви -- играя на нерви\\
igra na nervi -- igraya na nervi\\ 
 {a.play} of nerves -- play.\textsc{v} {} nerves\\  \jambox{(BG)}
 \glt ‘a battle of nerves’ -- ‘to lead a battle of nerves’
\end{exe}



\subsection{Subject complement or object complement -- noun modifier}
\label{section62}


In the Bulgarian data we detected a small number of verb MWEs that have
a subject complement or an object complement (\citealt{quirk1985}; \citealt{downing2014}) as part of their structure.\footnote{A subject or an 
object complement is a constituent that does not represent a new
participant but completes the predicate by adding information about the
subject or the object referent, respectively \citep{downing2014}, e.g.\@ \textit{a
separate notion} in \textit{The country became a separate notion}, \textit{young} in
\textit{He died young} (subject complement); \textit{a genius} in \textit{People considered
Picasso a genius} (object complement).} Syntactically, these
complements are expressed as NPs, PPs or APs. We use the notation
  ($_{\textup{CS}}$) for subject complements and  ($_{\textup{CO}}$)
for object complements. 

With this type of derivation, the verb MWE subject
complement turns up as a modifier in the derived noun MWE (12 cases
altogether). The derivation may be represented as in (ii).


\renewcommand{\theenumi}{(\roman{enumi})}%
\begin{enumerate}
 \setcounter{enumi}{1}
\item[(ii)]  VP [V
NP\textsubscript{CS}/PP\textsubscript{CS}/AP\textsubscript{CS}] > NP
[N\textsubscript{V–derived}
NP\textsubscript{CS}/PP\textsubscript{CS}/AP\textsubscript{CS}]
\end{enumerate}

The derivation involves a copular verb, such as 
\lnexlat{съм}{săm}{be}, 
\lnexlat{ставам}{stavam}{become}, 
\lnexlat{оставам}{ostavam}{remain} or а verb that is not a typical
copula (e.g., 
\lnexlat{отивам}{otivam}{go}) and admits a subject complement in the
MWE. The deverbal noun (N\textsubscript{V–derived}) derived from this
verb heads the noun MWE, and the subject complement turns up as а
post–modifier that preserves both its syntactic category and the type
of syntactic linking to the head word. The examples below illustrate
subject complements – PP$_{\textup{CS}}$  (\ref{ex:8:29}), AP$_{\textup{CS}}$ (\ref{ex:8:30}), N$_{\textup{CS}}$  (\ref{ex:8:31}).


\begin{exe}
\ex \label{ex:8:29}
\settowidth\jamwidth{BG}
\glll ставам за смях -- ставане за смях\\
stavam za smyah -- stavane za smyah \\
{to.become} for ridicule -- becoming for ridicule\\ \jambox{(BG)}
\glt ‘to become a laughing stock’ -- ‘becoming a laughing stock’
\end{exe}

\begin{exe}
\ex \label{ex:8:30}
\settowidth\jamwidth{BG}
\glll ставам разноглед -- ставане разноглед\\
stavam raznogled -- stavane raznogled\\ 
{to.become} cross–eyed -- becoming cross–eyed \\ \jambox{(BG)}
\glt ‘to become confused or overwhelmed (by something)’
\end{exe}

\begin{exe}
\ex \label{ex:8:31}
\settowidth\jamwidth{BG}
\glll отивам войник -- отиване войник\\
otivam voynik -- otivane voynik\\
go.\textsc{v} {a.soldier} -- going {a.soldier}\\ \jambox{(BG)} 
\glt ‘to go into the army’ -- ‘going into the army’
\end{exe}


Derivations involving an object complement are exemplified with 44 cases
in the data. It typically applies on transitive verbs (but verbs
admitting PP–object do occur, see (iv), (\ref{ex:8:34}) below). The direct object
(NP\textsubscript{DO}) is licensed by the verb that heads the MWE but
it is not a fixed part of the MWE. In the formal representation this
NP\textsubscript{DO} is enclosed in curly brackets “\{\}”. As we are
particularly concerned with the way the structure of the MWE is
reorganized, we do not consider the expression of the MWE–external
NP\textsubscript{DO} if it occurs, although it obeys the rules
applying to any direct object:


\renewcommand{\theenumi}{(\roman{enumi})}%
\begin{enumerate}
 \setcounter{enumi}{2}
 \item[(iii)] VP [V \{NP\textsubscript{DO}\}
NP\textsubscript{CO}/PP\textsubscript{CO}/AP\textsubscript{CO}]
 > NP
[N\textsubscript{V–derived} \{PP [P NP\textsubscript{DO}]\}
NP\textsubscript{CO}/\linebreak PP\textsubscript{CO}/AP\textsubscript{CO}]
\end{enumerate}

Here are examples of an MWE headed by a transitive verb with different
realizations of the object complement: an AP\textsubscript{CO} (\ref{ex:8:32}) and
a PP\textsubscript{CO} (\ref{ex:8:33}):

 
\begin{exe}
\ex \label{ex:8:32}
\settowidth\jamwidth{BG}
\glll дера \textnormal{(}някого\textnormal{)} жив -- дране жив \textnormal{(}на някого\textnormal{)}\\
dera (nyakogo) zhiv -- drane zhiv (na nyakogo)\\ 
 skin.\textsc{v} (someone) alive -- skinning alive (of someone) \\ \jambox{(BG)}
 \glt {‘to cause great trouble (to someone)’}
\end{exe}

\noindent\begin{minipage}[t]{\linewidth-3em}%
\begin{exe}
\ex {\label{ex:8:33}
\glll  правя \textnormal{(}някого \textnormal{/} нещо\textnormal{)} на решето -- правене на решето  \textnormal{(}на някого \textnormal{/} нещо\textnormal{)}\\ 
pravya (nyakogo / neshto) na resheto -- pravene na resheto (na nyakogo / neshto)  \\ 
make.\textsc{v} (someone / something)  a riddle -- making  a riddle (of (someone / something) \\
\glt `to make a lot of holes in someone/something, to riddle someone/something’
}
\end{exe}
\end{minipage}%
\hfill
\noindent\begin{minipage}[t]{2em}%
\vspace*{0.15\baselineskip}
(BG)
\end{minipage}

The lack of preposition insertion is a structural difference between the
derivations involving subject/object complement NPs and direct object
NPs, since the latter normally turn up as prepositional modifiers of the
corresponding deverbal nouns. We leave aside the marginal cases of
direct objects (not introduced by a preposition) that occasionally
co\textendash occur with the canonical form in formal administrative language. 



The derivation involving an MWE headed by a verb that admits a
prepositional object (PP\textsubscript{O}) has the following
representation (iv):

\renewcommand{\theenumi}{(\roman{enumi})}%
\begin{enumerate}
 \setcounter{enumi}{3}
 \item[(iv)] VP [V PP\textsubscript{O}
AP\textsubscript{CO}/NP\textsubscript{CO}/PP\textsubscript{CO}] > NP
[N\textsubscript{V–derived} PP\textsubscript{O}
AP\textsubscript{CO}/NP\textsubscript{CO}/PP\textsubscript{CO}]
\end{enumerate}

(\ref{ex:8:34}) exemplifies an MWE with a verb admitting a PP\textendash object. The
prepositional object retains its syntactic expression when it turns up
as an NP modifier.


\begin{exe}
\ex \label{ex:8:34}
\settowidth\jamwidth{BG}
\glll казвам на черното бяло -- казване на черното бяло\\
kazvam na chernoto byalo -- kazvane na chernoto byalo\\ 
call.\textsc{v} to the.black white -- calling to the.black white \\ \jambox{(BG)}
\glt ‘to call black white’ -- ‘an instance of calling black white’ 
\end{exe}


\subsection{Subject or direct object -- noun modifier}
\label{section63}

This particular derivation pattern concerns fixed subject verb MWEs or
fixed direct object verb MWEs. Unlike the cases discussed in \sectref{section61} and
\sectref{section62}, in this category nominalization triggers either insertion of a
preposition (in both languages) that introduces the former subject or
direct object as a prepositional noun modifier (see \sectref{section631}), or mapping
of the former subject or direct object into a genitive modifier (only
for Romanian) (see \sectref{section632}). We will reserve the term ‘genitive modifier’
for modifiers whose head noun is marked with the genitive case.

\subsubsection{Subject or direct object -- prepositional modifier}
\label{section631}

A subject (v) or a direct object (vi) in a verb MWE turns up as a
prepositional modifier of the corresponding deverbal noun that heads
the corresponding noun MWE. In (v1) and (vi1) a noun MWE is derived
from a verb MWE, while in (v2) and (vi2) a verb MWE is derived from a
noun MWE.


\renewcommand{\theenumi}{(\roman{enumi})}%
\begin{enumerate}
 \setcounter{enumi}{4}
\item[(v)]  (1) NP\textsubscript{S} V > NP [N\textsubscript{V–derived }PP[P
NP\textsubscript{S}]] \\
(2) NP [N PP[P NP\textsubscript{S}]] > NP\textsubscript{S}
V\textsubscript{N–derived} 
\end{enumerate}

\renewcommand{\theenumi}{(\roman{enumi})}%
\begin{enumerate}
 \setcounter{enumi}{5} 
 \item[(vi)] (1) VP [V NP\textsubscript{DO}] > NP [N\textsubscript{V–derived} PP
[P NP\textsubscript{DO}]] \\ 
(2) NP [N PP [P NP\textsubscript{DO}]] > VP
[V\textsubscript{N–derived} NP\textsubscript{DO}] 
\end{enumerate}


There are eight pairs in Romanian and seventy-five in Bulgarian involving the subject,
and thirty-two pairs in Romanian and 1,732 in Bulgarian involving the direct
object. 



In Romanian the preposition \textit{de} is always used and Bulgarian usually adds
the preposition \lnex{на}. Both prepositions can be glossed in English with
\textit{of}. In Bulgarian, other prepositions may occur, generally when the
noun is derived with a suffix other than \lnex{–не}  or \lnex{–ние}  (the prevalent
suffixes for eventive and/or resultative deverbal nouns) or by other
derivational means, e.g.\@ zero-derivation, \lnexlat{обичам}{obicham}{to love} --
\lnexlat{обич}{obich}{love}. 



We repeat (\ref{ex:8:2}) and (\ref{ex:8:3}) as (\ref{ex:8:35}) and (\ref{ex:8:36}), respectively, in order to
exemplify a case where the subject of a verb MWE corresponds to a
prepositional modifier of the deverbal noun that heads the derived noun
MWE. The subjects of the verb MWEs (\textit{cugetul} and 
\lnexl{съвестта}{săvestta})
correspond to the prepositional modifiers of the deverbal nouns
\textit{mustrare} and \lnexl{гризене}{grizene} which are derived from the verbs \textit{mustra} and
\lnexl{гриза}{griza} respectively.


\is{morphology!derivation|)}

\noindent\begin{minipage}[t]{\linewidth-3em}%
\begin{exe}
\ex \label{ex:8:35}
\settowidth\jamwidth{RO}
\gll a mustra cugetul (pe cineva) -- mustrare de cuget \\ 
to chide  consciousness  (\textsc{ObjMarker} somebody) – 
chiding by consciousness \\ %\jambox{(RO)}
\glt ‘to have remorse’ -- ‘having remorse’
\end{exe}
\end{minipage}%
\hfill
\noindent\begin{minipage}[t]{2em}%
\vspace*{0.15\baselineskip}
(RO)
\end{minipage}

\begin{exe}
\ex \label{ex:8:36}
\settowidth\jamwidth{BG}
\glll съвестта гризе \textnormal{(}някого\textnormal{)} -- гризене на съвестта\\
săvestta grize (nyakogo) -- grizene na săvestta\\ 
the.consciousness gnaws (someone) --  gnawing of the.consciousness \\\jambox{(BG)}
\glt ‘to have remorse’ -- ‘having remorse’
\end{exe}


(\ref{ex:8:37}) and (\ref{ex:8:38}) exemplify the case where the direct object of the verb MWE
corresponds to a prepositional modifier of the deverbal noun that heads
the derived noun MWE (derived from the verb) of the noun MWE. The
direct objects \textit{carte} and локуми, lokumi  correspond to the prepositional
modifiers of the (negative) adjective \textit{(ne)știutor} and the noun
\lnexl{разтегач}{raztegach} respectively, which have been derived from the verbs \textit{ști}
and \lnexl{разтягам}{raztyagam} respectively.


\begin{exe}
\ex \label{ex:8:37}
\settowidth\jamwidth{RO}
\gll a ști carte -- (ne)știutor de carte \\
to know book  -- (not)knowing of book \\ \jambox{(RO)}
\glt ‘to be educated’ -- ‘(un)educated’
\end{exe}

\begin{exe}
\ex \label{ex:8:38}
\begin{xlist}
\ex \label{ex:8:38a}
\settowidth\jamwidth{BG}
\gll разтягам локуми\\
raztyagam lokumi\\  \jambox{(BG)}
‘to spin yarn, to tell tales’
\ex \label{ex:8:38b}
\settowidth\jamwidth{BG}
\glll разтегач на локуми\\
raztegach na lokumi\\
spinner of yarn \\\jambox{(BG)}
\glt ‘yarnspinner’
\end{xlist}
\end{exe}


As noted in \sectref{section62}, when the direct object is not a fixed part of the MWE,
it may be left unexpressed. This is not the case with direct objects
that are fixed parts of base MWEs: they are neither left out nor
replaced with a possessive pronoun.



Apart from a direct object, the MWEs in this and in other categories may
have other constituents, e.g.\@ (an)other complement(s), such as a
prepositional object or an adjunct. These constituents preserve their
syntactic category and the syntactic link to the head word, but assume
a different syntactic status (similarly to what was presented in \sectref{section61}.).


\begin{exe}
\ex \label{ex:8:39}
\settowidth\jamwidth{RO}
\gll a despica firul în patru -- despicarea firului în patru\\
to  split  {the.hair} in four --  splitting {of.a.hair} in four \\ \jambox{(RO)}
\glt ‘to make small and overly fine distinctions’ 
\end{exe}

\begin{exe}
\ex \label{ex:8:40}
\settowidth\jamwidth{BG}
\glll цепя стотинката на две -- цепене на стотинката на две\\
tsepya stotinkata na dve -- tsepene na stotinkata na dve\\
split.\textsc{v} {the.penny} in half -- splitting of {the.penny} in half\\ \jambox{(BG)}
\glt ‘to be very stingy’
\end{exe}


In (\ref{ex:8:39}) and (\ref{ex:8:40}), \textit{în patru} `in four' and \lnexlat{на две}{na dve}{in half}  are PPs functioning as
adjuncts in the VP and as modifiers in the derived NPs.



In the Bulgarian data we found MWEs headed by verbs that take a direct
object and an object complement that are both part of the MWE (three
pairs). The construction has the following form (vii):


\renewcommand{\theenumi}{(\roman{enumi})}%
\begin{enumerate}
 \setcounter{enumi}{6}
\item[(vii)] VP [V NP\textsubscript{DO
}AP\textsubscript{CO}/NP\textsubscript{CO}/PP\textsubscript{CO}] > NP
[N\textsubscript{V–derived} PP [P NP\textsubscript{DO}]
AP\textsubscript{CO}/\linebreak NP\textsubscript{CO}/PP\textsubscript{CO}]
\end{enumerate}


The syntactic status of the MWE constituents in the derived structure is
predictable: in the noun MWE, the NP\textsubscript{DO }constituent of
the base verb MWE corresponds to a modifier introduced by a preposition
(\lnex{на})  and the object complement phrase of the base MWE turns up as a
modifier of the derived MWE that is expressed in the same way: as an AP
in (\ref{ex:8:41}) – \lnexlat{развързани}{razvărzani}{untied}, as an NP in (\ref{ex:8:42}) 
\lnexlat{бяло}{byalo}{white}
and as a PP in (\ref{ex:8:43}) – 
\lnexlat{с истинските им имена}{s istinskite im imena}{by their proper
names}:

\noindent\begin{minipage}[t]{\linewidth-3em}%
\begin{exe}
\ex \label{ex:8:41}
\settowidth\jamwidth{BG}
\glll оставям ръцете \textnormal{(}на някого\textnormal{)} развързани – оставяне на ръцете \textnormal{(}на някого\textnormal{)} развързани\\
ostavyam rătsete (na nyakogo) razvărzani -- ostavyane na
rătsete (na nyakogo) razvărzani\\
leave.\textsc{v} the.hands (of someone) untied -- leaving of the.hands (of someone) untied\\ %\jambox{(BG)}
\glt ‘to untie someone’s hands’ -- ‘untying someone's hands’
\end{exe}
\end{minipage}%
\hfill
\noindent\begin{minipage}[t]{2em}%
\vspace*{0.15\baselineskip}
(BG)
\end{minipage}

\begin{exe}
\ex \label{ex:8:42}
\settowidth\jamwidth{BG}
\glll наричам черното бяло -- наричане на черното бяло\\
naricham chernoto byalo -- narichane na chernoto byalo\\
call.\textsc{v} {the.black} white -- calling of  {the.black} white \\ \jambox{(BG)} 
\glt ‘to call black white’ -- ‘an instance of calling black white’\footnote{ The
expression in (\ref{ex:8:42}) is synonymous to the one in (\ref{ex:8:34}). The structural
difference is due to the different syntactic properties of the
synonymous verbs казвам, kazvam, `call' and наричам, naricham, `call': казвам takes a PP-object in
the respective sense, while наричам takes an NP object.} 
\end{exe}

\noindent\begin{minipage}[t]{\linewidth-3em}%
\begin{exe}
\ex \label{ex:8:43}
\settowidth\jamwidth{BG}
\gll наричам нещата с истинските им имена --  наричане на нещата с истинските им имена\\
naricham neshtata s istinskite im imena --  narichane na
neshtata s istinskite im imena  \\ %\jambox{(BG)}
\glt {‘call things by their proper names’ -- ‘calling things by their proper}
names’
\end{exe}
\end{minipage}%
\hfill
\noindent\begin{minipage}[t]{2em}%
\vspace*{0.15\baselineskip}
(BG)
\end{minipage}

\subsubsection{Subject or direct object -- genitive modifier}
\label{section632}

In this case the subject or the direct object of the verb MWE
corresponds to a genitive modifier in the derived MWE. The
reorganization may be represented as (viii) and (ix) for the subject
and the object, respectively:



\renewcommand{\theenumi}{(\roman{enumi})}%
\begin{enumerate}
 \setcounter{enumi}{7}
\item[(viii)]  NP\textsubscript{S} V > NP [N\textsubscript{V–derived}
NP\textsubscript{S–Genitive}]
\end{enumerate}

\renewcommand{\theenumi}{(\roman{enumi})}%
\begin{enumerate}
 \setcounter{enumi}{8}
\item[(ix)] VP [V NP\textsubscript{DO}] > NP [N\textsubscript{V–derived}
NP\textsubscript{DO–Genitive}]
\end{enumerate}


We encountered 12 MWEs described by (viii) in our data. As shown in
(\ref{ex:8:44}), the subject \textit{întunericul} corresponds to the genitive modifier of
the noun \textit{lăsarea} derived from the verb \textit{lăsa}: 


\begin{exe}
\ex \label{ex:8:44}
\settowidth\jamwidth{RO}
\gll se lasă întunericul -- lăsarea întunericului\\ 
\textsc{refl} lower {the.darkness} -- {lowering} {of.the.darkness} \\\jambox{(RO)}
\glt ‘it is getting dark’ -- ‘the fact of getting dark’
\end{exe}


Forty\textendash six pairs display the type of derivation where the direct object of the
verb MWE corresponds to a genitive modifier in the noun MWE (\ref{ex:8:45}): 


\begin{exe}
\ex \label{ex:8:45}
\settowidth\jamwidth{RO}
\gll a băga zâzanie -- băgarea zâzaniei \\
to insert dissension -- insertion {of.dissension} \\\jambox{(RO)}
\glt ‘to sow dissent’ -- ‘the sowing of dissent’
\end{exe}


The direct object \textit{zâzanie} corresponds to a genitive modifier of the
noun \textit{băgarea}, derived from the verb \textit{a băga}.



\subsection{Adjunct -- adjectival modifier}
\label{section64}

\is{syntactic reorganization|(}

In this case, the adjunct (either a prepositional or an adverb phrase)
modifying the verbal head of an MWE corresponds to an adjectival phrase
in a noun MWE derived from a verb MWE (see (x1) below) or vice versa --
a verb MWE is derived from a noun MWE and the adjective modifier in the
noun MWE corresponds to an adjunct (either a prepositional or an adverb
phrase) in the verb MWE (see (x2) below). This structure, represented
in (x), was detected in six Romanian and sixteen Bulgarian pairs. 



\renewcommand{\theenumi}{(\roman{enumi})}%
\begin{enumerate}
 \setcounter{enumi}{9}
\item[(x)] (1) VP [V PP/Adv(P)] > NP [N\textsubscript{V–derived }, A(P)] \\
(2) NP [N\textsubscript{V–derived}, A(P)] > VP [V PP/Adv(P)]
\end{enumerate}


In Romanian the modifying adjective usually occurs after the modified
noun. The normal position of the modifier in Bulgarian is to the left
of the modified noun. The comma in (x) is used for signalling the
possibility of having the modifier and the modified noun in either
order with respect to each other.



Here are some examples of this type of syntactic reorganization: 



\begin{exe}
\ex \label{ex:8:46}
\settowidth\jamwidth{RO}
\textit{arest preventiv -- a aresta preventiv} \jambox{(RO)}
‘preventive detention’ -- ‘to subject to preventive detention’
\end{exe}

\begin{exe}
\ex \label{ex:8:47}
\settowidth\jamwidth{BG}
\gll честна игра -- играя честно\\
chestna igra  -- igraya chestno\\ \jambox{(BG)}
\glt{‘a fair play’ -- ‘to play fairly’}
\end{exe}


In (\ref{ex:8:46}) \textit{preventiv} is an adjective modifying the noun \textit{arest} in the
first MWE and an adverb modifying the verb \textit{aresta}, derived from
\textit{arest}. Likewise, in the BG example (\ref{ex:8:47}), 
\lnexlat{честна}{chestna}{fair} is the adjectival
modifier of the noun 
\lnexlat{игра}{igra}{play} and 
\lnexlat{честно}{chestno}{fairly} is the adverb modifier of the verb
\lnexlat{играя}{igraya}{play}. 



An example of a derivation of a noun MWE with an adjective modifier
(\lnexlat{трезва}{trezva}{straight}) from a verb MWE with an adverb (\lnexlat{трезво}{trezvo}{straight}) is shown in (\ref{ex:8:48}):


\begin{exe}
\ex \label{ex:8:48}
\settowidth\jamwidth{BG}
  \gll мисля трезво -- трезва мисъл\\
  mislya trezvo -- trezvа misăl\\ \jambox{(BG)}
\glt {‘to think straight’ -- ‘straight thinking’}
\end{exe}


Examples (\ref{ex:8:46}), (\ref{ex:8:47}) and (\ref{ex:8:48}) involve an adverb modifier in the verb MWE.
The other type of construction presented in (x) (involving a PP
adjunct) is exemplified by (\ref{ex:8:49}):


\begin{exe}
\ex \label{ex:8:49}
\settowidth\jamwidth{BG}
\gll търгувам на едро -- едра търговия\\
tărguvam na edro -- edra târgoviya\\ \jambox{(BG)}
\glt{‘to deal wholesale’ -- ‘wholesaling’}
\end{exe}

 
The PP modifier 
\lnexlat{на едро}{na edro}{big/in bulk} of the verb MWE 
\lnexlat{търгувам на едро}{tărguvam na edro}{to deal wholesale} corresponds to the adjective modifier  
\lnexlat{едра}{edra}{big} of the noun
MWE 
\lnexlat{едра търговия}{edra târgoviya}{wholesaling}. Note that variants are possible where the PP
adjunct of the verb MWE becomes a PP post\textendash modifier in the noun MWE;
these cases fall under \sectref{section61}.



 \tabref{tab:8:4} sums up the data presented in \sectref{section6}, along with their share
(in percentage) in the overall number of cases that undergo syntactic
reorganization (364 for Romanian and 2,671 for Bulgarian).


\begin{table}
\begin{tabularx}{\textwidth}{lSSSS}
\lsptoprule
\textbf{Type} &
\textbf{No. of RO examples} &
\textbf{No. of examples}&
\textbf{RO data \%} &
\textbf{BG data \%}\\ 
\midrule
\sectref{section61} PP/AdvP &
260 &
792 &
71.2\% &
29.65\%\\
%\hline
\parbox[t]{3cm}{\sectref{section62} Subject/object\\[-.3em] complement} &
0 &
56 &
0\% &
2.1\%\\
%\hline
\sectref{section631} Subject
&
8 &
75 &
2.5\% &
2.81\%\\
%\hline
\sectref{section631} Object
&
32 &
1,732 &
8.8.\% &
64.84\%\\
%\hline
\sectref{section632} Subject
&
12 &
0 &
3.3\% &
 0\%\\
%\hline
\sectref{section632} Object
&
46 &
0 &
12.63\% &
0\%\\
%\hline
\sectref{section64} Adjunct &
6 &
16 &
1.65\% &
0.6\%\\
\lspbottomrule
\end{tabularx}
\caption{Distribution of Romanian and Bulgarian MWEs across types.}
\label{tab:8:4}
\end{table}




Several conclusions can be drawn. The MWEs in which the fixed base MWE
subject corresponds to a fixed PP modifier in the derived MWE or vice
versa (\sectref{section631}) have a very similar share in the two languages
and, as the numbers show, the construction is relatively rare. The same
holds for verb MWE adjuncts that turn up as adjectival modifiers or
vice versa (\sectref{section64}). The cases involving subject complements or object
complements (\sectref{section62}) are found only in Bulgarian. Still, this pattern is
potentially productive as the head verbs involved in it are very common
(e.g.\@ \textit{make/do}). In Romanian, the correspondence between a subject or
an object and a genitive modifier (\sectref{section632}) is more common than the
correspondence to a PP modifier (\sectref{section631}).

\is{syntactic reorganization|)}


There is a striking difference with respect to the prevalent base MWE
structure in each of the languages. In Romanian the most frequent
construction is verb\textendash prepositional object/adjunct (\sectref{section61}), while the
verb\textendash direct object construction is quite uncommon (\sectref{section631}). In Bulgarian
the most frequent type is verb\textendash direct object (\sectref{section631});
verb\textendash prepositional object/adjunct (\sectref{section61}) is also typical, although there are twice as many verb\textendash direct object constructions. This points to a significant difference in the
syntactic expression of complements (as reflected in the structure of
MWEs); PP objects are by far the preferred choice in Romanian, while in
Bulgarian both direct objects and PP objects are common, with a marked
preference for the former.



 

\begin{table}
\small
\begin{tabularx}{\textwidth}{l@{~}p{.8cm}p{1cm}r Q}
\lsptoprule
\textbf{Language} &
\textbf{Base MWE} &
\textbf{Derived MWE} &
\textbf{Occurrences} &
\textbf{Examples}\\
\midrule
 RO &
 Event &
 Event &
 322 &
\textit{a cădea în păcat} ‘to fall into
sin’ \newline  \textit{cădere în păcat} ‘falling into sin’\\
 BG &
 Event &
 Event &
 2,590 &
 \textit{promivam mozătsi} ‘to brainwash’ \newline 
\textit{promivane na mozătsi} ‘brainwash’\\
 RO &
 Event &
 Agent &
 18 &
 \textit{a vâna zestre} ‘to hunt dowry’ \newline 
\textit{vânător de zestre}, hunter of dowry, ‘fortune
hunter’\\
 BG &
 Event &
 Agent &
 53 &
 \textit{promivam mozătsi} ‘to brainwash’ \newline 
\textit{promivach na mozătsi} ‘one who does brainwashing’\\
 RO &
 State &
 State &
 11 &
\textit{a mustra cugetul (pe cineva)} ‘to
have remorse’ \newline  \textit{mustrare de cuget} ‘having remorse’\\
 BG &
 State &
 State &
 12 &
\textit{zhiveya po tsarski} `to live regally’
\newline  \textit{zhivot po tsarski} ‘a regal life’\\

 RO &
 Event &
 Instrument &
 3 &
\textit{arunca flăcări} ‘to throw flames’ \newline 
\textit{aruncător de flăcări}, thrower of flames, 
‘flamethrower’\\
 BG &
 Event &
 Instrument &
 3 &
\textit{razbărkvam karti} ‘to shuffle cards’ \newline 
\textit{razbărkvach na karti} ‘card shuffler’\\
 RO &
 State &
 Experiencer &
 3 &
\textit{a voi binele} ‘to wish well’ \newline 
\textit{voitor de bine}, wisher of well, ‘well\textendash wisher’\\
 RO &
 Event &
 Distance &
 3 &
\textit{arunca cu bățul} ‘to throw with a
stick’ \newline  \textit{aruncătură de băț} ‘as far as the stick can be thrown’\\
 BG &
 Event &
 Institution &
 5 &
\textit{kova zakoni}, forge laws, 
‘to create and promulgate laws’ \newline  \textit{kovachnitsa na zakoni}, smithy of
laws, ‘the parliament’\\
\lspbottomrule
\end{tabularx}
\caption{Semantics of the base and the derived MWEs. Frequencies and examples.}
\label{tab:8:5}
\end{table}


\begin{table}
\small
\begin{tabularx}{\textwidth}{l@{~~}p{1.2cm}p{1cm}r Q}
\lsptoprule
\textbf{Language} &
\textbf{Base MWE} &
\textbf{Derived MWE} &
\textbf{Occurrences} &
\textbf{Examples}\\
\midrule
RO &
 Job &
 Institution &
 8 &
\textit{judecător de pace} ‘justice of the peace’ \newline 
\textit{judecătorie de pace} ‘the court of a justice of the peace’\\
BG &
 Job &
 Institution &
 3 &
\textit{voenen prokuror} ‘military
prosecutor’ \newline  \textit{voenna prokuratura} ‘military prosecutor's
office’\\
 RO &
 Event &
 Vehicle &
 1 &
\textit{a vâna submarine} ‘to hunt for
submarines’ \newline  \textit{vânător de submarine}, hunter of
submarines, `a vessel for locating and attacking submarines'\\
 RO &
 Result &
 Action &
 1 &
\textit{lucru de mână} ‘handiwork’ \newline  \textit{a
lucra de mână} ‘to work by hand’\\
 RO &
 Artefact &
 Event &
 2 &
 \textit{desen în peniță} ‘pen drawing’ \newline 
\textit{a desena în peniță} ‘to draw in pen’\\
 RO &
 Event &
 Characteristic &
 2 &
\textit{a sări în ochi}, to jump into eyes,
‘to be straightforward’ \newline  \textit{săritor în ochi}, jumping into eyes,
‘straightforward’\\

 BG &
 Event &
 Charac–teristic &
 8 &
\textit{rabotya kato vol}, work like an
ox/horse, `to work hard' \newline  \textit{rabotliv kato vol}, as hard\textendash working as an ox, ‘very
hard\textendash working’\\
 BG &
 State &
 Characteristic &
 8 &
\textit{mălcha kato păn}, keep silent like a log, ‘to be as
mute as a maggot/fish’ \newline  \textit{mălchaliv kato păn}, silent like a log, ‘(as)
mute as a maggot/fish’\\
 BG &
 Inchoative state &
 Characteristic &
 10 &
\textit{gladen kato vălk} ‘as hungry as a
wolf/bear’ \newline  \textit{ogladneya kato vălk} ‘to become as hungry as a
wolf/bear’\\
 BG &
 Job &
 Agent &
 18 &
 \textit{softuerno inzhenerstvo} ‘software
engineering’ \newline  \textit{softueren inzhener} ‘software engineer’\\
\lspbottomrule
\end{tabularx}
\caption{Semantics of the base and the derived MWEs. Frequencies and examples.}
\label{tab:8:6}
\end{table}

\section{Semantics of the derivational patterns}
\label{section7}

In this section we present the semantic aspects of the MWEs that involve derivation. Although we refer to the semantics of the base MWE, we
are more interested in the semantics of the derived MWEs. Tables \ref{tab:8:5} and  \ref{tab:8:6}
offer an overview of the derived MWE semantics.\footnote{
For verb–noun pairs we used the inventory of
morpho–semantic relations from PWN \citep{fellbaum2009}, but we added
to it some roles whenever they proved necessary.}

\is{event nominalisation|(}
In Romanian, the great majority of the base MWEs (349) designate
events. This remark correlates with the data in \tabref{tab:8:4}, where most
base MWEs are verbs. Furthermore, the derived nominalizations (mostly
with the suffix \textit{–re}) also denote events (322 cases): this correlates
with the number of V–N pairs in \nolinebreak \tabref{tab:8:3}.

The Bulgarian data also support the productivity and regularity of the
derivation of eventive nominalizations (2,590) predominantly with the
suffix \lnex{–не}. Another interesting tendency (though represented by few
examples), especially with respect to neologisms, is the back–formation
of verbs from nouns. The other semantic types encountered with derived
MWEs constitute a small number of the overall data. The derivatives
such as agents, experiencers, instruments and locations are derived
primarily from VP [V NP\textsubscript{DO}] MWEs, and less frequently
from VP [V PP/AdvP]. No examples were found for such nouns derived
from MWEs with the following syntactic structure: NP\textsubscript{S }V
or VP [V
NP\textsubscript{CS}/PP\textsubscript{CS}/AP\textsubscript{CS}].

 The productivity of event nominalization is not unexpected, because in the
process of MWE-to-MWE derivation the majority of cases account for
idiomatic (partial) predicate\textendash argument structures. As the structure of
eventive nominalizations may reflect the argument structure of the base
verb \citep{grimshaw1990}, it readily renders these idiomatic structures. The frequency of use of eventive nominalizations, whether single words
or MWEs, is substantiated by the fact that they make it possible to
refer to an action/event regardless of its doer and the time of
occurrence (as expressed by verbal categories) \citep{pometkova2006}, and
hence they may be used interchangeably with the verb-headed
construction, or even may be preferred contextually in certain cases or
in certain registers, such as scientific discourse.

In Romanian, the
number of nominalizations increases greatly (by almost 200 cases in our
data) when taking into account supine forms of verbs which, via
conversion, become nouns; and the rest of the MWE behave, in these
cases, similarly to the cases displaying affixal derivation, i.e.
almost the same types of syntactic changes occur. Moreover, supine
nouns are (for more than 150 cases in our data) alternatives to derived
nominalizations, with a semantic difference: \cite{cornilescu2001}
maintains that \textit{–re} nominalizations tend to express results, while supines
express events.
\is{event nominalisation|)}

As the data show, other types of verb–noun derivational patterns, such
as the ones resulting in agents, experiencers, instruments, locations
and so forth, are significantly fewer in number. In our opinion, the
semantic grounds for this phenomenon is that the situations described
by the respective verb MWEs frequently do not conceptualize a
particular type of agent, experiencer or instrument and so forth that
needs to be lexicalized. Moreover, in terms of their semantic and
syntactic properties, these types of nouns do not as readily inherit
and express the base verb arguments and/or adjuncts. This is supported
by the fact that when the need arises for expressing the relevant
agentive or instrumental, etc. meaning, participle\textendash headed constructions
are preferred, at least in Bulgarian (\ref{ex:8:50}).


\begin{exe}
\ex \label{ex:8:50}
\settowidth\jamwidth{BG}
\gll вземам решение -- вземащ решение\\
vzemam reshenie -- vzemasht reshenie\\ \jambox{(BG)}
\glt{‘to make a decision’ -- ‘(the one) making a decision’}
\end{exe}


These participial constructions may be either contextually used or may
undergo nominalization and lexicalization.



Besides, another word-formation device that is also frequent is
compounding, in which case the arguments/adjuncts are incorporated in
the word structure. Here are some examples of one-word compounds that
have MWE counterparts in the data e.g.\@ 
\lnexlat{сърцеразбивач}{sărtserazbivach}{heartbreaker}, 
\lnexlat{миторазбивач}{mitorazbivach}{mythbuster}, 
\lnexlat{кодоразбивач}{kodorazbivach}{codebreaker}, 
\lnexlat{мо-нетосекач}{monetosekach}{coiner/minter}, etc. 



\section{Conclusions}

Putting MWEs and derivation together, we notice that derivation affects
MWEs, creating either words or other MWEs. The productivity of this
phenomenon seems to depend on language characteristics: Bulgarian, a
language with aspect, allows for more cases of derivation than
Romanian, which lacks aspect. Another factor influencing productivity
is the data set: Romanian DELS lacks terms, which do occur in the
Bulgarian dictionary and are productive in terms of derivation, serving
the need for expressing different actors, instruments, objects, places,
etc. within a domain of activity.



We have presented data from Bulgarian and Romanian. However,  derivation has been reported to act upon MWEs in other 
languages: \cite{piela2007} discusses examples of words created from
idioms and argues that this process is productive in \ili{Polish}; in
\ili{Russian}, the process of creating MWEs from  MWEs seems to be
the most productive internal means of MWE formation \citep{ermakova2015}. We can conclude that MWEs are subject to derivation in more
languages and comparing and contrasting them from such a perspective
can be of linguistic \nolinebreak interest.



 

\section[Acknowledgements]{Acknowledgements}

Most part of the work reported here has been carried out within the project PARSing
and Multiword Expressions (PARSEME) IC1207 COST Action. Another part
has been carried out within the joint project “Enhanced Knowledge Bases
for Bulgarian and Romanian” of the Institute for Bulgarian Language,
Bulgarian Academy of Sciences, and the Research Institute for
Artificial Intelligence, Romanian  Academy.



We would like to thank Ivelina Stoyanova and Maria Todorova for
providing the Bulgarian MWE dictionary in electronic form, Ivelina
Stoyanova for the automatic processing of the Bulgarian data, Cătălina
Mărănduc for kindly providing us with the electronic version of DELS
and Cătălin Mititelu for the automatic processing of the Romanian data.



Last but not least, we are grateful to the anonymous reviewers of the
paper and to the editors for their comments on the previous versions of
the paper and for their suggestions that helped us to improve its
quality. We also thank Judith Elver for being kind enough to proofread
this paper on very short notice.


\section*{Abbreviations} 
\begin{tabularx}{.5\textwidth}{>{\scshape}lQ}
{a}g & Agent\\
{bg} & Bulgarian \\
{lpc} & Bulgarian Language Processing Chain  \\
{dels} & Dictionary of Expressions, Idioms and Collocations \\
{e}v & Event\\
{i}nstn & Institution\\
\end{tabularx}
\begin{tabularx}{.4\textwidth}{>{\scshape}lQ}
{i}nstr & Instrument \\
{l} & Language\\
{p}o{s} & part of speech \\
{r}e & Result\\
{ro} & Romanian  \\
{sv}s & Semantic Values \\
{s}t & State\\
{v} & Verb (in the glosses)  \\
\end{tabularx} 

\sloppy


\printbibliography[heading=subbibliography,notkeyword=this]
%\nocite{*}

 









\end{document}
