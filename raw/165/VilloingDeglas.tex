\documentclass[output=paper]{langsci/langscibook}
\ChapterDOI{10.5281/zenodo.1406997}


\title{Des lexèmes à forme unique~: comment le créole réanalyse les dérivations du français}
\shorttitlerunninghead{Comment le créole réanalyse les dérivations du français}

\author{Florence Villoing\affiliation{Modèles, Dynamiques, Corpus~ (MoDyCo) CNRS~: UMR7114,~Université Paris Nanterre}\lastand Maxime Deglas \affiliation{Structures Formelles du Langage~ (SFL), CNRS~:UMR7023,~Université Paris VIII - Vincennes Saint-Denis}
}

\abstract{Le présent article présente les conditions d'apparition de deux schémas morpho\-lo\-giques en créole guadeloupéen, la suffixation verbale dénominale\is{derivation!noun-based verb} en \emph{--\emph{é}} (N-\emph{é}\textsubscript{\textsc{v}}\is{suffixation!in --\emph{é}}) (ex : \emph{biké} `se réfugier' ← \emph{bik} `refuge' ; \emph{miganné} `mélanger ← \emph{migan} `purée') et la parasynthèse\is{parasynthesis} verbale dénominale\is{derivation!noun-based verb} (\emph{dé}-N-\emph{é}\textsubscript{\textsc{v}}\is{parasynthesis!in \emph{dé}--\emph{é}}\is{parasynthesis!in \emph{dé}--\emph{é}}) (ex : \emph{déchèpiyé} `mettre en charpie' ← \emph{chépi} `charpie', \emph{dépyété} `retirer les pattes (crabes)' ← \emph{pyèt} `pattes'). Nous montrons que ces shémas ont émergé via la réanalyse de paires morphologiques Verbe / Nom, massivement héritées\is{inheritance} du français, langue lexificatrice, issues soit de conversions\is{conversion} (\emph{bròs} `brosse' / \emph{brosé} `brosser') soit de préfixations\is{prefixation} (\emph{bwa} `bois' / \emph{débwazé} `déboiser'). L'article défend l'hypothèse que c'est notamment la spécificité des lexèmes créoles de n'apparaître que sous une forme unique qui a conduit à ces réanalyses : les verbes créoles ne variant pas flexionnellement, la finale flexionnelle française /e/ héritée est réanalysée comme suffixe dérivationnel, suivant ainsi un processus de déflexionnalisation propre au changement linguistique. }

\maketitle

\begin{document}
\selectlanguage{french}
\il{French-based creole|(}
\il{French-based creole!Guadeloupe Creole|(}
\is{lexeme|(}
\is{reanalysis|(}


\section{Introduction}\label{section:VilloingDeglas:introduction}

La réflexion menée ces cinquante dernières années sur l'identité
lexicale et la notion de lexème, notamment par les morphologues, a
permis d'éclairer l'analyse de dérivés français impliquant des verbes.
Ainsi, les verbes dénominaux\is{derivation!noun-based verb}, traditionnellement traités comme suffixés
au moyen de la marque de l'infinitif (\emph{boiser, plumer, neiger}) ou
comme parasynthétiques\is{parasynthesis} par adjonction simultanée d'un préfixe et d'un
suffixe d'infinitif (\emph{embar\-quer, désosser, décourager}) ont pu être
analysés comme des convers\is{conversion} (\emph{boiser, plumer, neiger}) ou des
préfixés\is{prefixation} (\emph{embarquer, désosser, décourager}) sur base nominale à
partir du moment où une réflexion théorique sur l'identité du lexème a
été menée (cf. §~\ref{section:VilloingDeglas:analyse-des-paires-nv-en-français}). Mais une telle analyse de ces dérivés français est
remise en cause une fois qu'ils intègrent les langues créoles à base
française, et on voit s'opérer comme un retournement de situation par
rapport aux analyses traditionnelles. En effet, bien que ces créoles
aient hérité\is{inheritance} d'une bonne partie des dérivés verbaux dénominaux\is{derivation!noun-based verb} convers\is{conversion}
et préfixés\is{prefixation} du français, l'analyse morphologique que l'on peut en faire
en créole est radicalement différente~: là où les paires nom/verbe
relèvent de conversions\is{conversion} en français, elles sont formées au moyen d'une
opération de suffixation en créole~; et là où les paires s'interprètent
comme des préfixations\is{prefixation} en français, on doit y voir des parasynthèses\is{parasynthesis} en
créole. Cette réanalyse des paires nom/verbe construites et héritées\is{inheritance} du
français a fait système en créole conduisant à la création de nouveaux
schémas morphologiques qui sont devenus parfaitement disponibles.

Le présent article présente les conditions d'apparition de ces deux
schémas morphologiques en créole, la suffixation verbale dénominale\is{derivation!noun-based verb} en
--\emph{é} (désormais N-\emph{é}\textsubscript{\textsc{v}}\is{suffixation!in --\emph{é}}\footnote{La
  représentation N-\emph{é}\textsubscript{\textsc{v}}\is{suffixation!in --\emph{é}} de la structure des verbes
  dénominaux\is{derivation!noun-based verb} suffixés en --\emph{é}\is{suffixation!in --\emph{é}} s'interprète comme suit~: N
  représente la base nominale, --\emph{é} le suffixe, et v la classe syntaxique
  (V pour verbe) du dérivé. }) et la parasynthèse\is{parasynthesis} verbale dénominale\is{derivation!noun-based verb}
(désormais \emph{dé}-N-\emph{é}\textsubscript{\textsc{v}}\is{parasynthesis!in \emph{dé}--\emph{é}}\footnote{La
  représentation \emph{dé}-N-\emph{é}\textsubscript{\textsc{v}}\is{parasynthesis!in \emph{dé}--\emph{é}} de la structure des verbes
  dénominaux\is{derivation!noun-based verb} affixés en \emph{dé}--...--\emph{é} s'interprète comme suit~: N
  représente la base nominale, \emph{dé}--...--\emph{é} l'affixe parasynthétique dont la
  forme phonologique comprend un préfixe \emph{dé}-- associé à un suffixe --\emph{é}, et
  v la classe syntaxique (verbe) du dérivé. }), en défendant
l'hypothèse que c'est notamment la spécificité des lexèmes créoles de
n'apparaître que sous une forme unique qui a conduit à ces réanalyses (§~\ref{section:VilloingDeglas:analyses-des-paires-nv-en-créole}).

L'analyse que nous présentons est pertinente pour plusieurs créoles à
base française (au moins le martiniquais\il{French-based creole!Martinique Creole}, le haïtien\il{French-based creole!Haitian Creole} et le
saint-lucien\il{French-based creole!Saint-Lucia Creole}), mais s'appuie uniquement sur des données du créole
guadeloupéen. Les ressources disponibles pour la constitution d'une base
de données de grande ampleur du lexique guadeloupéen font largement
défaut, tant du point de vue lexicographique que numérique %
(cf. 
%Villoing \& Deglas 2016
\citealt{Villoing16}%
%
, § 2.).
%\citep[cf. ][, § 2.]{Villoing16}%
%Villoing-Deglas
%
Devant l'absence de ressource fiable et
directement exploitable, nous avons basé notre étude sur un corpus
original établi par Maxime Deglas, locuteur natif, à partir de plusieurs
ressources~:

\begin{enumerate}\def\labelenumi{(\roman{enumi})}
\item
  les dictionnaires existants du Guadeloupéen %
%(Ludwig \& \emph{alii}
%  2012~; Poullet \& \emph{alii }1984~; Tourneux \& Barbotin 1990) 
\citep{Ludwig12,Poullet84,Tourneux90} %
%Tourneux-Barbotin
%
dont
  les entrées ont été filtrées grâce à des enquêtes de terrain vérifiant
  leur attestation auprès de locuteurs natifs~;
\item
  des enquêtes de terrain réalisées auprès d'une quarantaine de
  locuteurs natifs issus de toutes les îles de la Guadeloupe ;
\item
  d'un corpus issu d'une activité de veille terminologique réalisée au
  sein d'ouvrages littéraires en langue créole, d'émissions de
  télévision et de chansons traditionnelles (cf. Villoing \& Deglas
  2016, § 2. pour plus de détails).~
\end{enumerate}

Le corpus ainsi constitué est composé de 7680 unités lexicales du créole
guadeloupéen, soit une envergure équivalente à celle des dictionnaires
existants. Il comprend 1805 verbes et 4643 noms qui ont permis l'étude
spécifique des relations morphologiques Nom/Verbe dans le cadre de la
suffixation verbale dénominale\is{derivation!noun-based verb} en --\emph{é} et de la parasynthèse\is{parasynthesis}
verbale dénominale\is{derivation!noun-based verb}. Le corpus est enregistré sous format électronique
dans une base de données interrogeable selon plusieurs critères,
phonologiques, sémantiques, syntaxiques, qui permettent une étude fine.

Nous menons l'étude de ce corpus en suivant une approche théorique
relevant de la morphologie lexématique\is{morphology!lexeme-based} %
%(cf. par ex. Matthews 1991,
%Aronoff 1994, Anderson 1992, Fradin 2003, booij10)
\citep[cf. par ex.][]{Matthews91,Aronoff94,Anderson92,Fradin03,Booij10}%
%Matthews;Aronoff;Anderson;Fradin
%
, envisageant que
les unités de base de la morphologie sont les lexèmes (et non les
morphèmes). Nous nous inscrivons dans une perspective qui reconnaît aux
langues créoles une morphologie dynamique (tout au moins pour ce qui
concerne la morphologie lexicale), nous inscrivant en faux relativement
aux détracteurs du contraire %
(%
%Valdman 1978, Seuren \& Wekker 1986, McWorther 1998
\citealt{Valdman78,Seuren86,McWhorter1998}%
%
, par exemple).
%\citep[, Mc Worther 1998, par exemple]{Valdman78,Seuren86}%
%Valdman;Seuren-Wekker
%
La démonstration commencera par une
présentation des débats autour des analyses des paires Nom/Verbe convers\is{conversion}
et préfixés\is{prefixation} du français (§~\ref{section:VilloingDeglas:analyse-des-paires-nv-en-français}.) pour ensuite développer notre hypothèse
de leur réanalyse en créole qui a conduit à la création de nouveaux
schémas morphologiques, la suffixation N-\emph{é}\textsubscript{\textsc{v}}\is{suffixation!in --\emph{é}} et la
parasynthèse\is{parasynthesis} \emph{dé}-N-\emph{é}\textsubscript{\textsc{v}}\is{parasynthesis!in \emph{dé}--\emph{é}} (§~\ref{section:VilloingDeglas:analyses-des-paires-nv-en-créole}).

\section{Analyse des paires N/V en
français}\label{section:VilloingDeglas:analyse-des-paires-nv-en-français}

Les créoles à base française ont hérité\is{inheritance} une partie du lexique du
français, qui est encore aujourd'hui largement représenté dans la langue
créole (par exemple, pour le Guadeloupéen, 90\% de mots d'origine
française, issus principalement du français populaire du 17\textsuperscript{ème} siècle,
mais également d'emprunts contemporains, selon %
%Hazaël-Massieux 2002%
\citealt{HazaelMassieux2002}%
%
%
). Ce
lexique hérité\is{inheritance}, clairement reconnaissable malgré quelques divergences
phonologiques avec l'origine française, comprend des paires de lexèmes
morphologiquement construits en français tels que (\ref{ex:VilloingDeglas:1}) et (\ref{ex:VilloingDeglas:2}).

\ea\label{ex:VilloingDeglas:1}
  \ea bò / débòdé  (`bord' / `déborder')

  \ex bwa / débwazé (`bois' / `déboiser')

  \ex figi / défigiré (`figure' / `défigurer')

  \ex fòwm / défòwmé (`forme' / `déformer')

  \ex kras / dékrasé (`crasse' / `décrasser')

  \ex rasin / dérasiné (`racine' / `déraciner')
  \z
\ex\label{ex:VilloingDeglas:2}
  \ea adisyon / adisyonné (`addition' / `additionner')

  \ex bav / bavé (`bave' / `baver')

  \ex bròs / brosé (`brosse' / `brosser')

  \ex divòs / divòsé (`divorce' / `divorcer')

  \ex fèt / fété (`fête' / `fêter')

  \ex savon / savonné (`savon' / `savonner')
\z\z

Ces paires Nom/Verbe héritées\is{inheritance} sont prises dans une relation
morphologique en français que l'on ne peut plus leur reconnaître en
créole. Les paragraphes qui suivent donnent un rapide aperçu des
analyses morphologiques auxquelles elles répondent en français, pour
présenter, ensuite, l'analyse morphologique que nous en proposons en
créole guadeloupéen.

\subsection{Les paires du type \emph{bois} / \emph{déboiser}}\label{subsection:VilloingDeglas:les-paires-du-type-bois-déboiser}
\label{sec:vg:2.1.1.}

La formation en français des verbes de (\ref{ex:VilloingDeglas:1}) a été l'objet de grandes
discussions. Une tradition qui remonte au 19\textsuperscript{ème} siècle
les a analysés comme des construits morphologiques par parasynthèse\is{parasynthesis},
c'est-à-dire comme relevant d'une construction morphologique où une base
est simultanément préfixée\is{prefixation} et suffixée. Cette analyse remonte au moins à
Arsène Darmesteter.

\begin{quotation}
«~Cette sorte de composition\footnote{Darmesteter parle
  de composition pour caractériser la \isi{préfixation}, témoignant par-là du
  fait que certains préfixes sont issus de prépositions latines.} est
très riche~: les verbes qu'elle forme, et que l'on désigne sous le nom
de parasynthétiques\is{parasynthesis}, offrent ce remarquable caractère d'être le résultat
d'une composition et d'une dérivation agissant ensemble sur un même
radical\is{stem}, de telle sorte que l'une ou l'autre ne peut être supprimée sans
amener la perte du mot. C'est ainsi que de barque l'on fait em-barqu-er,
dé-barqu-er, deux compositions absolument uns et dans lesquelles on ne
retrouve ni des composés débarque, embarque, ni un dérivé barquer, mais
le radical\is{stem} barque.~» %
%Darmesteter (1894~: 24)
\citet[24]{darmesteter1894.traite-formation}%
%Darmesteter
%
\end{quotation}


L'analyse est largement reprise au 20\textsuperscript{ème} siècle par
%
%Nyrop (1936~: 215)
\citet[215]{Nyrop36}%
%Nyrop
%
, et a rencontré encore beaucoup de succès à partir des
années 70 dans d'autres théories, comme la Grammaire Générative
Transformationnelle %
%(Dubois 1962, Guilbert 1975, Zribi-Hertz 1972,
%Scalise 1994) 
\citep{dubois62,Guilbert1975,Zribi-Hertz72,Scalise1994} %
%Dubois;Guilbert;Zribi-Hertz;Scalise
%
ou encore dans le cadre lexicaliste %
%(Booij 1977)
\citep{booij1977.dutch-morphology}%
%Booij
%
. Elle
s'est également étendue aux grammaires traditionnelles %
%(Grevisse 1988~:
%253) 
\citep[253]{Grevisse1988} %
%Grevisse
%
et scolaires en France %
(cf. par exemple, %
%Chevalier et al. 1964
\citealt{Chevalier64}%
%
~: 54, %
%Béchade 1992
\citealt{Bechade92}%
%
~: 119), voire aux manuels de morphologie du français (%
%Gardes-Tamine 1988
\citealt{Gardes-Tamine1988}%
%
~: 65, %
%Apothéloz 2002
\citealt{Apotheloz2002}%
%
~: 91, %
%Huot 2006
\citealt{Huot06}
~:~121-122)
%\citep[cf. par exemple, ][~:54, Béchade 1992~: 119, voire aux manuels de morphologie du français (Gardes-Tamine 1988~: 65, Apothéloz 2002~: 91, Huot 2006~: 121-122]{Chevalier64}%
%Chevalier-al.
%
.
Malgré sa popularité, l'analyse parasynthétique est remise en cause pour
ces verbes par %
%Dell (1970~: 201-202) 
\citet[201--202]{Dell70} %
%Dell
%
puis plus largement par %
%Corbin (1987~: 121-139)
\citet[121--139]{Corbin87}%
, et à leur suite %
%Fradin (2003~: 288-307)
\citet[288-307]{Fradin03}%
%Fradin
%
. La critique
s'appuie unanimement sur l'erreur d'analyse récurrente qui est faite de
la forme du verbe prise métalinguistiquement~: l'affixe d'infinitif (qui
apparaît de façon conventionnelle dans la forme de citation du verbe)
est assimilé à un suffixe dérivationnel. Cette erreur provient en partie
d'une confusion entre la langue et la métalangue %
%(Corbin 1987~: 124) 
\citep[124]{Corbin87} %
%Corbin
%
et
en partie de ce que les cadres théoriques ne définissent pas
théoriquement l'individu lexical. Une double confusion est ainsi à
l'œuvre %
%(Kerleroux 2000)
\citep{Kerleroux2000}%
%Kerleroux
%
~: une première confusion entre la forme de
citation métalinguistique du verbe (qui est traditionnellement
l'infinitif en français) et sa forme phonologique, et une seconde
confusion entre la forme phonologique du verbe avec l'individu lexical.
Ainsi,

\begin{quotation}
«~le rapport catégoriel N\textgreater{}V va être vu comme une
suffixation, puisque la forme d'infinitif (dans son rôle citationnel)
est prise pour le verbe lui-même, et que l'infinitif français présente
un suffixe (à la différence de l'anglais). {[}\ldots{}{]} Tout le
problème est que cela implique de voir dans le suffixe flexionnel
d'infinitif un suffixe qui soit également dérivationnel\ldots{}~»
%
%(Kerleroux 2000~: 9)
\citep[9]{Kerleroux2000}%
%Kerleroux
%
\end{quotation}


Or il a été clairement démontré que l'affixe d'infinitif ne peut être
identifiable à un suffixe dérivationnel, comme le prouve le fait qu'il
n'apparaisse jamais en dérivation, où seul le radical\is{stem} sert toujours de
base %
(%
%Corbin 1987
\citealt{Corbin87}%
%
~: 129, %
%Lyons 1977
\citealt{Lyons77}%
%
~: 19, %
%Fradin 2003
\citealt{Fradin2003}%
%
~: 93, 
%Fradin, Kerleroux, Plénat 2009
\citealt{Fradin2009}%
%
~: 9, par exemple)%
%\citep[~: 129, Lyons 1977~:19, Fradin 2003~: 93, Fradin, Kerleroux, Plénat 2009~: 9, par exemple]{Corbin87}%
%Corbin
%
. Ainsi, il aura fallu plus d'un
siècle pour montrer que le suffixe d'infinitif de la forme citationnelle
n'appartient pas au lexème, en tant qu'unité lexicale.

Il résulte de cette remise en cause une nouvelle analyse selon laquelle
«~les pseudo-parasynthèses\is{parasynthesis} verbales ne sont en fait que des
préfixations\is{prefixation}~» %
%(Corbin 1987~: 129)
\citep[129]{Corbin87}%
%Corbin
%
~: la base est nominale et le dérivé
verbal. Ainsi, selon cette perspective, les données de (\ref{ex:VilloingDeglas:1}) sont-elles
analysées, en français, comme des verbes préfixés\is{prefixation} sur bases nominales
dont la structure correspond à (\ref{ex:VilloingDeglas:3})~:

\is{parasynthesis!in \emph{dé}--\emph{é}}
\ea\label{ex:VilloingDeglas:3}{[\emph{dé}-- {[}N{]} {]}}\textsubscript{\textsc{v}}
\z

Ces préfixes dénominaux\is{derivation!noun-based verb} verbalisateurs présentent, selon Corbin, une
propriété originale au regard de la majorité d'entre eux, ils entraînent
un changement de catégorie de la base, au même titre que la plupart des
suffixes. Cette propriété des préfixes n'ayant pas été reconnue par
toute une tradition, a également, selon Corbin, largement contribué à
l'analyse en terme de parasynthèse\is{parasynthesis}.

Les paires morphologiques Nom/Verbe en (\ref{ex:VilloingDeglas:2}) ci-dessus ont subi une erreur d'analyse du même type.

\subsection{\texorpdfstring{Les paires du type
\emph{brosse/brosser}}{Les paires du type brosse/brosser}}\label{les-paires-du-type-brossebrosser}

La formation des verbes du français en (\ref{ex:VilloingDeglas:2}) a également fait l'objet de
grandes discussions. L'analyse de ces paires s'est heurtée, dans la
littérature sur la morphologie du français, aux mêmes blocages que les
verbes dénominaux\is{derivation!noun-based verb} préfixés\is{prefixation}~: le suffixe d'infinitif de la forme
citationnelle du verbe a été interprété par toute une tradition comme un
suffixe dérivationnel.

\begin{quote} C'est cette même prétendue suffixation qui apparaît dans la formation
de verbes dénominaux\is{derivation!noun-based verb} non préfixés\is{prefixation} comme \emph{clouer}, ou dans les
déadjectivaux comme \emph{brunir}, \emph{rougir}. %
%(Dell 1970~:
%200-202)
\citep[200--202]{Dell70}%
%Dell
%
~
\end{quote}

Selon l'orientation de l'opération morphologique (de nom à verbe ou de
verbe à nom), la disparition (orientation V \textrightarrow{~} N) ou l'apparition du
suffixe (orientation N \textrightarrow{~} V) a été vue comme relevant de deux mécanismes
différents,

\begin{itemize}

\item[---]
  la «~dérivation régressive~» (terminologie que l'on retrouve chez
  %
%Nyrop (1936)
\citet{Nyrop36}%
%Nyrop
%
, dans les grammaires traditionnelles %
%(Grevisse 1988) 
\citep{Grevisse1988} %
%Grevisse
%
et
  certains manuels de morphologie %
%(Gardes-Tamines 1988)
\citep{Gardes-Tamine1988}%
%  
), rend compte
  d'une apocope du suffixe d'infinitif, permettant de former un nom à
  partir d'un verbe (par exemple \emph{voler} \textrightarrow{~} \emph{vol})~;
\item[---]
  un mécanisme de suffixation de l'infinitif permettant à un nom de
  devenir un verbe (\emph{plante} \textrightarrow{~} \emph{planter}). Cependant, ce
  rapport entre nom et verbe n'est pas clairement reconnu par les
  premiers grammairiens comme relevant de la morphologie comme l'atteste
  le flou dans lequel il est traité par exemple par %
%Nyrop (1936)
\citet{Nyrop36}, %
%Nyrop
%
%
%Meyer-Lübke (1890-99) 
\citet{Meyer-Lubke} % 1890-99 ???
%?Meyer-Lübke
%
et plus tard par les grammaires traditionnelles
%
%(cf. par exemple Grevisse 1988~: 238)
\citep[cf. par exemple][238]{Grevisse1988}%
%Grevisse
%
.
\end{itemize}

Là encore, le défaut de ces analyses est l'absence de questionnement
théorique quant à l'identité du lexème, confondant forme citationnelle
et unité lexicale. Les approches plus contemporaines répondent à ces
analyses erronées en voyant dans les paires en (\ref{ex:VilloingDeglas:2}) des construits
ressortissant à une opération de \isi{conversion} de nom à verbe ou de verbe à
nom %
%(cf. pour le français, Corbin 1987, 2004, Mel'čuk 1996, Kerleroux
%2000, Fradin 2003, Namer 2009, Tribout 2010)
\citep[cf. pour le français,][]{Corbin87,corbin04,Melcuk1996,Kerleroux2000,Fradin03,Namer09,Tribout2010a}%
%Corbin;Corbin;Mel'čuk;Kerleroux;Fradin;Namer;Tribout
%
. L'apparente différence
phonologique entre le nom et le verbe n'est liée qu'à la convention que
l'on adopte en français de citer les verbes au moyen de leur forme
d'infinitif et les noms à partir de leur forme de singulier. Mais les
formes phonologiques des lexèmes bases et dérivés (en d'autres termes,
leurs radicaux), sont bien en tous points identiques, ce qui autorise à
reconnaître entre eux une relation morphologique de \isi{conversion}.

Ainsi, les paires en (\ref{ex:VilloingDeglas:2}) sont-elles analysables soit selon la structure
(\ref{ex:VilloingDeglas:4a}), soit selon la structure (\ref{ex:VilloingDeglas:4b}), sans qu'aucune sorte d'affixe ne
soit en jeu~:


\ea\label{ex:VilloingDeglas:4}

  \ea\label{ex:VilloingDeglas:4a} {[}N{]}\textsubscript{\textsc{v}}
  \ex\label{ex:VilloingDeglas:4b} {[}V{]}\textsubscript{\textsc{n}}
  \z

\z


\section{Analyses des paires N/V en
créole}\label{section:VilloingDeglas:analyses-des-paires-nv-en-créole}

Les données en (\ref{ex:VilloingDeglas:1}) et (\ref{ex:VilloingDeglas:2}) formées par \isi{préfixation} ou \isi{conversion} verbale
dénominale\is{derivation!noun-based verb} en français et héritées\is{inheritance}, ne peuvent pourtant pas recevoir la
même analyse en créole. Dans les paragraphes qui suivent, nous
argumentons en faveur de la double hypothèse qu'en créole,

\begin{enumerate}\def\labelenumi{(\roman{enumi})}

\item
  la relation morphologique entre les noms et les verbes en --\emph{é}
  de (\ref{ex:VilloingDeglas:2}) correspond à une suffixation verbale sur base nominale
  (N-\emph{é}\textsubscript{\textsc{v}}\is{suffixation!in --\emph{é}}) et non à une \isi{conversion} comme en français~;
\item
  la relation morphologique entre les noms et les verbes en
  \emph{dé}-N-\emph{é}\textsubscript{\textsc{v}}\is{parasynthesis!in \emph{dé}--\emph{é}} de (\ref{ex:VilloingDeglas:1}) correspond à une parasynthèse\is{parasynthesis} plutôt
  qu'à une \isi{préfixation} comme c'est le cas en français.
\end{enumerate}

Ces résultats nous amènent à conclure que ces paires morphologiques
Nom/Verbe ont subi une réanalyse du français au créole\footnote{Nous
  entendons \enquote{réanalyse} au sens général de %
%Langacker (1977:58)
\citet[58]{Langacker77}%
%Langacker
%
, à
  savoir un changement dans la structure (morphologique) d'un lexème
  qui n'implique pas pour autant de modification dans sa forme
  phonologique de surface. Voir aussi le recours qu'en fait %
%DeGraff   (2001: 67-68)
\citet[67--68]{DeGraff2001}%
%
.}, réanalyse due en grande partie à la spécificité des
lexèmes créoles de n'apparaître que sous une unique forme. C'est sur
cette spécificité des verbes en créole guadeloupéen que s'ouvre le §~\ref{subsection:VilloingDeglas:les-verbes-du-creole-guadeloupeen}.

\subsection{Les verbes du créole
guadeloupéen}\label{subsection:VilloingDeglas:les-verbes-du-creole-guadeloupeen}

\subsubsection{Morphologie}\label{subsubsection:VilloingDeglas:morphologie}

Les verbes du créole guadeloupéen, comme toutes les autres unités
lexicales, ne présentent pas de morphologie flexionnelle, ce que la
littérature pointe en évoquant soit l'absence de flexion dans les
langues créoles, soit une morphologie pauvre, voire inexistante. Les
propriétés liées au Temps-Aspect-Mode sont prises en charge par des
particules qui précèdent le verbe, comme on l'observe en général dans
les créoles à base française %
(cf.  %
%Valdman 1978, Bernabé 1987, Mufwene \& Djikhoff 1989, Hazaël-Massieux 2002
\citealt{Valdman78,Bernabe87,Mufwene89,HazaelMassieux2002}%
%
~: 71 ; voir aussi %
%Germain 1976
\citealt{Germain76}%
~:~109--134, pour le guadeloupéen)
%\citep[cf. ][, Hazaël-Massieux 2002~:71 ; voir aussi Germain 1976~:~109-134, pour le guadeloupéen]{Valdman78,Bernabe87,Mufwene89}%
%Valdman;Bernabé;Mufwene-Djikhoff
%
.

Lorsque les verbes sont hérités\is{inheritance} du français, une seule forme du verbe
est conservée en créole. Il s'agit, \emph{a priori}, soit de la forme de
l'infinitif soit de la forme du participe passé, soit d'une de celles du
présent indicatif ou impératif %
%(Germain 1976~: 110)
\citep[110]{Germain76}%
%Germain
%
. Pour les verbes du
1\textsuperscript{er} groupe et 2\textsuperscript{ème} groupe, l'origine
de la forme héritée\is{inheritance} n'est pas décidable puisque les formes du participe
passé et de l'infinitif sont homonymes à l'oral avec une finale en

\begin{itemize}

\item[---]
  /e/ pour les verbes du 1\textsuperscript{er} groupe,
\item[---]
   /i/ pour les verbes du 2\textsuperscript{ème} groupe (sachant qu'au
  17\textsuperscript{ème} siècle, époque où la majorité du lexique
  français est hérité\is{inheritance}, le /r/ final des infinitifs en --\emph{ir} ne se
  prononçait plus avant d'être réhabilité ultérieurement sous
  l'influence des grammairiens et des poètes).
\end{itemize}

La table \ref{tab:VilloingDeglas:1} présente les différentes finales verbales des verbes
créoles hérités\is{inheritance} des verbes français et les formes fléchies supposées
originelles.

\begin{table}
  %\resizebox*{!}{\dimexpr\textheight-2\baselineskip\relax}{
  \resizebox{\textwidth}{!}{
  \begin{tabular}[c]{lllllll}
  \lsptoprule
  Groupe & Finale & Finale & Issus de & Issus de  & Issus de formes & Glose\\
en  & créole & en  & formes & participes  & d'indicatif/\\
français &  &  français & d'infinitifs & passés & impératif présent & \\
  \midrule
  1 & -é & /e/ & karésé  & & & `caresser'\\
  & & & brosé  & & & `brosser'\\
  & & & gomé & & &  `gommer'\\
  & & & blagé & & & `blaguer' \\
  \midrule
  2 & -i & /ir/\textasciitilde{}/i/ & asòrti  & & &`assortir'\\
  & & & chwazi  & & & `choisir'\\
  & & & fléri  & & & `fleurir'\\
  & & & nwasi  & & & `noircir'\\

  \midrule
  3 & -è & /ɛr/ & & dékouvè  & & `découvrir'\\
  & & & & ofè  & & `offrir'\\
  & & & & wouvè  & & `ouvrir'\\
  & & & & soufè & & `souffrir' \\
  & & /ɛ/\textasciitilde{}/ɛr/ & fè & fè & fè
  & `faire'\\
  & & & plè & & plè &`plaire'\\

  \cline{2-7}
  & -an & /ã/\textasciitilde{}/ãdr/ & aprann & & aprann
  & `apprendre'\\
  & & & défann  & & défann & `prendre la
  défense'\\
  & & & étann  & & étann&  `étendre'\\
  & & & fann  & & fann & `fendre'\\

  \cline{2-7}
  & -wè & /wa/\textasciitilde{}/war/ & bwè  & & bwè &
  `boire'\\
  & & & kwè & & kwè & `croire'\\
  & & & pèsivwè  & & pèsivwè  & `apercevoir'\\
  & & & wousouvwè  & & wousouvwè &`recevoir'\\

  \cline{2-7}
  & -i & /ir/\textasciitilde{}/i/ & bouyi& & &  `bouillir' \\
  & & & fui  & fui  &&`fuir'\\
  & & & manti  & &&`mentir'\\
  & & & rèdi  & rèdi &&`redire'\\
  & & & sòti  & &&`sortir'\\

  \cline{2-7}
  & -enn & /ɛ̃/\textasciitilde{}/ ɛ̃dr/ & détenn  & détenn
 & détenn &`déteindre'\\
  & & & étenn  & étenn & étenn
  & `éteindre'\\
  & & & & krenn & krenn &`craindre'\\
  & & & & & soutyenn &`soutenir'\\
  & & & tenn & tenn & tenn &`teindre'\\

  \cline{2-7}
  & -èt & /ɛtr/ & admèt  & & & `admettre'\\
  & & & disparèt  & & & `disparaître'\\
  & & & pwomèt  & & & `promettre'\\
  & & & rèkonnèt  & & & `reconnaître'\\
  & & /ɛt/ & & défèt  & & `défaire'\\
  \lspbottomrule
\end{tabular}}
\caption{Verbes hérités\is{inheritance} du français.}
% {Cette table ne respecte pas les guidelines: elle est trop longue et trop large.}
\label{tab:VilloingDeglas:1}
\end{table}


Chacune des finales n'a pas la même représentativité au sein du lexique
guadeloupéen, et on note une très large majorité de verbes à finale en
--\emph{é} (toute origine confondue, hérités\is{inheritance}, construits en créole ou
autre, cf. Table~\ref{tab:VilloingDeglas:2})\footnote{Dans la table~\ref{tab:VilloingDeglas:2}, la classe «~autres~» inclut principalement des verbes à
    finale consonantique dont une bonne part sont construits par
    composition d'un verbe et d'un nom (\emph{bat chat} `battre en
    retraite', \emph{pèd lakat} `perdre la tête).}. Nous supposons que cette très forte proportion
est liée à un héritage\is{inheritance} massif de verbes français à finale en --\emph{é},
héritage\is{inheritance} qui aurait eu un impact important dans la morphologie du créole
(cf. §~\ref{verbes-hérités-versus-verbes-créoles}. ci-dessous).

\begin{table}
\begin{tabular}{lll}
\lsptoprule
Verbes à finale en & \multicolumn{2}{c}{Total}\\
& nbr & \% des V totaux \\
\midrule
-é & 1451 & 80\%\\
-i & 147 & 8,1\%\\
-è & 25 & 1,4\%\\
-ann & 30 & 1,6\%\\
-wè & 10 & 0,5\%\\
-enn & 11 & 0,6\%\\
-èt & 15 & 0,8\%\\
-an & 10 & 0,5\%\\
-o & 8 & 0,4\%\\
autres & 98 &
5,4\%\\
Total & 1805 & 100\%\\

\lspbottomrule
\end{tabular}
% {Cette table ne respecte pas les guidelines: "You should not add footnotes to tables or figures."}
\caption{Proportion des verbes guadeloupéens selon leur finale.}
\label{tab:VilloingDeglas:2}
\end{table}



\subsubsection{\texorpdfstring{Verbes hérités \emph{versus} verbes
créoles}{Verbes hérités versus verbes créoles}}\label{verbes-hérités-versus-verbes-créoles}

\is{inheritance}
La discrimination, au sein du lexique créole, entre verbes hérités\is{inheritance} et
verbes créoles --ou «~indigènes~», pour reprendre la terminologie de
%
%Lefebvre (2003) 
\citet{Lefebvre2003} %
%Lefebvre
%
et %
%Brousseau (2011)
\citet{Brousseau11}%
%Brousseau
%
-- suscite discussion, dans la mesure
où rares sont les cas où l'héritage\is{inheritance} est total. En effet, les verbes, en
passant du français au créole, peuvent avoir subi des modifications
phonologiques, sémantiques ou syntaxiques. Une position consiste à
considérer comme non français tout lexème hérité\is{inheritance} ayant subi une
variation en créole~: par exemple, pour %
%Brousseau (2011~: 68)
\citet[68]{Brousseau11}%
%Brousseau
%
, les
lexèmes \emph{pitiab} `pitoyable' et \emph{lonvi} `longues-vues' en
Saint-Lucien, sont considérés comme des bases inexistantes en français à
cause de l'écart phonologique entre les deux langues, et \emph{kouvé}
`couvrir' à cause de la différence sémantique avec le verbe
\emph{couver}. Nous nous distinguerons de cette position en considérant
comme hérité\is{inheritance} du français tout verbe dont l'origine française est
reconnaissable, phonologiquement et sémantiquement, malgré les
modifications subies en créole. Ainsi, parmi les exemples de Brousseau,
seul \emph{kouvé} `couvrir' ne serait pas reconnu comme d'origine
française à cause du sens trop éloigné du verbe \emph{couver} du
français. Notre choix repose sur le fait (i) d'une part qu'il est
extrêmement difficile de connaître précisément la phonologie et la
sémantique des lexèmes hérités\is{inheritance} d'un état ancien ou régional du français,
et en conséquence, de déterminer, avec certitude, l'écart entre le
supposé verbe français et son correspondant hérité\is{inheritance} en créole~; (ii)
d'autre part que quasiment tout lexème hérité\is{inheritance} du français a subi une
modification phonologique voire sémantique, même mineure, et qu'il
serait difficile d'établir des critères départageant les lexèmes
suffisamment altérés pour être classés créoles et les autres.

Afin de déterminer l'origine française d'un lexème créole, nous nous
sommes appuyés sur leur attestation en entrée d'un dictionnaire de
français, tout dictionnaire, registre de langue et variétés dialectales
confondus %
(voir aussi %
%Brousseau 2011
\citealt{Brousseau11}%
~: 68 sur l'utilité des dictionnaires du 16\textsuperscript{ème} au 20\textsuperscript{ème} siècle).
%\citep[voir aussi ][68 sur l'utilité des dictionnaires du 16\textsuperscript{ème} au 20\textsuperscript{ème} siècle]{Brousseau11}%
%Brousseau
%
La recherche est largement facilitée par la Toile qui met à
notre disposition plusieurs types de dictionnaires du français,
permettant notamment de retrouver des verbes aujourd'hui perdus mais
relevant d'un état de langue ancien ou d'un dialecte du français, dont
on suppose qu'ils constituent le fond du lexique créole 
%(cf. par exemple Thibault (éd.) 2012~: 12)
\citep[cf. par exemple][12]{Thibault12}%
.

Ces critères nous permettent de distinguer les verbes hérités\is{inheritance} de deux
autres types de verbes~:

\begin{enumerate}\def\labelenumi{(\roman{enumi})}

\item
  les verbes morphologiquement construits en créole, tout procédé
  morphologique et toutes bases confondues (bases non héritées\is{inheritance} (\ref{ex:VilloingDeglas:5}),
  bases héritées\is{inheritance} (\ref{ex:VilloingDeglas:6}), bases héritées\is{inheritance} avec changement phonologique (\ref{ex:VilloingDeglas:7}) ou
  sémantique (\ref{ex:VilloingDeglas:8})).
\end{enumerate}



\ea\label{ex:VilloingDeglas:5}
  \ea \gll bik \textrightarrow{~} biké\\
  {`refuge'} {} {`se réfugier'}\\
  \ex \gll fifin \textrightarrow{~} fifiné\\
  {`bruine'} {} {`bruiner'}\\
  \ex \gll migan \textrightarrow{~} miganné\\
  {`purée' `mélange'} {} {`mélanger'}\\
  \ex \gll plich \textrightarrow{~} pliché\\
  {`correction'} {} {`donner une correction'}\\
  \ex \gll vonvon \textrightarrow{~} vonvonné\\
  {`bourdon'} {} {`bourdonner'}\\
  \z

\ex\label{ex:VilloingDeglas:6}

  \ea \gll balkon \textrightarrow{~} balkonné\\
  {`balcon'} {} {`être au balcon'}\\
  \ex \gll garé \textrightarrow{~} dégaré\\
  {`garer, stationner'} {} {`sortir de la place de garage, de stationnement'}\\
  \ex \gll lang \textrightarrow{~} langé\\
  {`langue'} {} {`embrasser'}\\
  \ex \gll pyé \textrightarrow{~} dépyété\\
  {`pattes'} {} {`retirer les pattes (crabe)'}\\
  \ex \gll tik \textrightarrow{~} détiké\\
  {`tique'} {} {`retirer les tiques'}\\
  \z

\ex\label{ex:VilloingDeglas:7}

  \ea \gll fouch \textrightarrow{~} fouchté\\
  {`fourche'} {} {`bêcher'}\\
  \ex \gll katyé \textrightarrow{~} dékatyé\\
  {`morceau'} {} {`couper en quartier'}\\
  \ex \gll nwèl \textrightarrow{~} nwélé\\
  {`noël'} {} {`fêter Noël'}\\
  \ex \gll pengné \textrightarrow{~} dépengné\\
  {`peigner'} {} {`défaire une coiffure'}\\
  \ex \gll vès \textrightarrow{~} vèsté\\
  {`veste'} {} {`mettre sa veste'}\\
  \z

\ex\label{ex:VilloingDeglas:8}

  \ea \gll kabann \textrightarrow{~} kabanné\\
  {`lit'} {} {`traîner au lit'}\\
  \ex \glll kaz \textrightarrow{~} dékazé\\
  {`maison'} {} {`déplacer une maison à l'aide}\\
  {} {} {d'un véhicule pour l'installer ailleurs'}\\
  \ex \gll loup \textrightarrow{~} loupé\\
  {`boursouflure'} {} {`enfler'}\\
  \ex \gll parad \textrightarrow{~} paradé\\
  {`étalage'} {} {`parader'}\\
  \z
\z

\begin{enumerate}

\item[(ii)]
  les verbes ne répondant à aucun de ces critères, ni hérités\is{inheritance} du
  français, ni construits en créole, et dont l'origine peut être 
  connue (cf. un emprunt à l'anglais, à l'espagnol, aux langues
  africaines ou autre), ou non.
\end{enumerate}

\ea\label{ex:VilloingDeglas:9}

  \ea \gll bénékaki \\
  `hésiter'\\
  \ex \gll griji \\
  `s'égratigner'\\
  \ex \gll kòviyé \\
  `tordre'\\
  \ex \gll lolé \\
  `remuer'\\
  \ex \gll tòtòy \\
  `agacer'\\
\z\z


Sur la base de cette répartition tripartite des verbes en créole (verbe
hérité\is{inheritance}, verbe construit en créole, verbe autre), nous obtenons les
proportions suivantes (cf. Table~\ref{tab:VilloingDeglas:3} qui ne représente que les trois
finales les plus représentées, les finales verbales en --\emph{é},
--\emph{i}, et --\emph{ann}).


  \begin{table}
      \resizebox{\textwidth}{!}{
  \begin{tabular}{p{0.15\textwidth}lp{0.1\textwidth}lp{0.1\textwidth}lp{0.1\textwidth}lp{0.1\textwidth}}
  \lsptoprule
  \multicolumn{9}{c}{Verbes en créole guadeloupéen} \\
Verbes à
  & \multicolumn{2}{c}{Total}
  & \multicolumn{2}{c}{Verbes hérités\is{inheritance}}
  & \multicolumn{2}{c}{Verbes construits }
  & \multicolumn{2}{c}{Autres}\\
finale en & & & & & \multicolumn{2}{c}{ en créole} & & \\
  & nbr & \% des V totaux
  & nbr & \% des V à finale en \ldots{}
  & nbr & \% des V à finale en \ldots{}
  & nbr & \% des V à finale en \ldots{} \\
  \midrule
  -é & 1451 & \textbf{80\%} & 1230 & 84 \% & 153 & 10,5\% & 66 &
  4,5\%\\
  -i & 147 & 8,1\% & 122 & 83 \% & 17 & 11,5 \% & 8 & 5,5\%\\
  -ann & 30 & 1,6\% & 27 & 90\% & 3 & 10\% & 0 & 0\%\\
  \midrule
  Total & 1805 & 100\% & 1468 & 81\% des V & 248 & 14\% des V & 86 & 5\%
  des V\\
  \lspbottomrule
  \end{tabular}}
  \caption{Proportion de verbes hérités\is{inheritance}, construits ou autres selon
  leur finale.}
  \label{tab:VilloingDeglas:3}
  \end{table}


Notre corpus comprend ainsi une part majeure de verbes hérités\is{inheritance} du
français~: sur les 1805 verbes listés, 1468 sont hérités\is{inheritance}, soit 81 \% des
verbes du créole. Parmi ces verbes hérités\is{inheritance}, la majorité sont des verbes
à finale en --\emph{é} (soit 84 \%). Loin derrière se trouvent les
verbes hérités\is{inheritance} à finale en --\emph{i} qui ne représentent que 8,3\% des
verbes hérités\is{inheritance} (122 verbes hérités\is{inheritance} à finale en \emph{--i} parmi 1468
verbes hérités\is{inheritance}). Les verbes présentant d'autres finales \emph{(--ann, --è,
--wè} etc.) sont encore moins nombreux et très peu représentés. Cet ordre
de préférence se reflète largement dans les verbes construits en
créole~: là encore, les verbes à finale en \emph{--é} sont les plus
représentés (61,5\% correspondant à 153 verbes construits en --\emph{é
}relativement à 248 verbes construits), suivis de loin par les verbes à
finale en --\emph{i} (moins de 7\%). Les autres verbes restent de
l'ordre de l'épiphénomène. Ce parallèle entre finale des verbes hérités\is{inheritance}
et verbes construits en créole conduit raisonnablement à faire
l'hypothèse que le lexique hérité\is{inheritance} a fortement pesé sur la formation
morphologique des verbes créoles. Ainsi, dans la mesure où la majorité
des verbes hérités\is{inheritance} sont ceux à finale en --\emph{é} et que les verbes
créoles dérivés sur base nominale présentent également majoritairement
cette finale, nous émettons l'hypothèse que la finale flexionnelle en
--\emph{é} des verbes hérités\is{inheritance} a été réanalysée, dans certaines
circonstances, comme un suffixe dérivationnel en créole. Le paragraphe
3.2. présente des hypothèses sur les conditions de cette réanalyse. Nous
n'examinerons pas plus avant ici la possible réanalyse des finales de
verbes hérités\is{inheritance} en --\emph{i}, mais remarquons néanmoins qu'en dépit de
la très faible proportion de ces verbes dans le lexique créole (8,1\%),
la part des verbes construits en --\emph{i} est proportionnellement
équivalente à celles des verbes construits en --\emph{é} (11,5 \% contre
10,5 \% pour les verbes en \mbox{--\emph{é}}), ce qui conduirait à rendre
crédible l'hypothèse de la création d'un suffixe verbalisateur
--\emph{i} en créole guadeloupéen.

\subsection{Réanalyses des paires N/V de convers\is{conversion} comme
suffixations}\label{subsection:VilloingDeglas:réanalyses-des-paires-nv-de-convers-comme-suffixations}

Selon notre hypothèse, la réanalyse des verbes à finale en \emph{--é} du
français en créole n'a été possible que dans le contexte lexical créole
où ces verbes français sont hérités\is{inheritance} avec les noms français en relation
de \isi{conversion} avec eux, soit une \isi{conversion} de nom à verbe (N\textrightarrow{~} V) soit
une \isi{conversion} de verbe à nom (V\textrightarrow{~} N) (cf. (\ref{ex:VilloingDeglas:10})). Ainsi, le lexique du
créole guadeloupéen comprend des paires de convers\is{conversion} Nom/Verbe héritées\is{inheritance} du
français, pour lesquels l'analyse en terme de \isi{conversion} n'est pas
valide en créole.

\subsubsection{De la conversion en français à la suffixation en
créole}\label{subsubsection:VilloingDeglas:de-la-conversion-en-français-uxe0-la-suffixation-en-créole}
\is{conversion}

La raison principale qu'une relation de suffixation soit perçue en
créole entre ces paires Nom/Verbe tient au fait que le --\emph{é}
final du verbe apparaît comme du matériel phonologique supplémentaire
par rapport à la forme phonologique du nom base (\ref{ex:VilloingDeglas:10}). Y voir une
\isi{conversion} de nom à verbe serait alors contraire à la notion de
\isi{conversion} puisque les radicaux ici se différencient phonologiquement.


\ea\label{ex:VilloingDeglas:10}
  \ea \gll adisyon / adisyonné\\
  {`addition'} {} {`additionner'}\\
  \ex \gll bav / bavé\\
  {`bave'} {} {`baver'}\\
  \ex \gll bròs / brosé\\
  {`brosse'} {} {`brosser'}\\
  \ex \gll divòs / divòsé\\
  {`divorce'} {} {`divorcer'}\\
  \ex \gll fèt / fété\\
  {`fête'} {} {`fêter'}\\
  \ex \gll mank / manké\\
  {`manque'} {} {`manquer'}\\
  \ex \gll savon / savonné\\
  {`savon'} {} {`savonner'}\\
\z\z

Comme les verbes créoles n'ont qu'une forme, les verbes en (\ref{ex:VilloingDeglas:10}) ne
présentent donc que la forme comprenant un --\emph{é} final. Ce
--\emph{é} final, de fait, appartient bien au verbe en tant qu'unité
lexicale et n'est pas le marqueur du mode infinitif apparaissant dans la
forme citationnelle du verbe français. Ainsi, les paires Nom/Verbe en
(\ref{ex:VilloingDeglas:10}) héritées\is{inheritance} du français ne peuvent subir la même analyse en français
et en créole. Elle se distinguent des paires de Nom/Verbe en (\ref{ex:VilloingDeglas:11}) qui,
au contraire, entretiennent bien une relation morphologique de
\isi{conversion} en créole (de type N\textrightarrow{~} V ou V\textrightarrow{~} N). En effet, en créole, comme
dans toutes les autres langues, les noms et les verbes en relation de
\isi{conversion} sont phonologiquement en tous points identiques (cf. en (\ref{ex:VilloingDeglas:11a})
des paires de convers\is{conversion} Nom/Verbe à finale en --\emph{é} et en (\ref{ex:VilloingDeglas:11b}) des
paires de convers\is{conversion} Nom/Verbe présentant une autre finale vocalique).


\ea\label{ex:VilloingDeglas:11}

  \ea\label{ex:VilloingDeglas:11a}

  \ea \gll balyé\textsubscript{\textsc{n}} / balyé\textsubscript{\textsc{v}}\\
  {`balai'} {} {`balayer'}\\
  \ex \gll chanté\textsubscript{\textsc{n}} / chanté\textsubscript{\textsc{v}}\\
  {`chanson'} {} {`chanter'}\\
  \ex \gll goumé\textsubscript{\textsc{n}} / goumé\textsubscript{\textsc{v}}\\
  {`combat'} {} {`se battre'}\\
  \ex \gll lélé\textsubscript{\textsc{n}} / lélé\textsubscript{\textsc{v}}\\
  {`touillette'} {} {`touiller'}\\
  \ex \gll manjé\textsubscript{\textsc{n}} / manjé\textsubscript{\textsc{v}}\\
  {`repas, mets'} {} {`manger'}\\
  \ex \gll tété\textsubscript{\textsc{n}} / tété\textsubscript{\textsc{v}}\\
  {`sein'} {} {`téter'}\\
  \z

  \ex\label{ex:VilloingDeglas:11b}

  \ea \gll anvi\textsubscript{\textsc{n}} / anvi\textsubscript{\textsc{v}}\\
  {`envie'} {} {`avoir envie'}\\
  \ex \gll bobi\textsubscript{\textsc{n}} / bobi\textsubscript{\textsc{v}}\\
  {`assoupissement'} {} {`somnoler'}\\
  \ex \gll kaka\textsubscript{\textsc{n}} / kaka\textsubscript{\textsc{v}}\\
  {`excrément'} {} {`déféquer'}\\
  \ex \gll mò\textsubscript{\textsc{n}} / mò\textsubscript{\textsc{v}}\\
  {`mort'} {} {`mourir'}\\
  \ex \gll travay\textsubscript{\textsc{n}} / travay\textsubscript{\textsc{v}}\\
  {`travail'} {} {`travailler'}\\
  \z
  \z
\z

Par ailleurs, on ne peut, en aucun cas, tenir l'hypothèse de la
\isi{conversion} en traitant le --\emph{é} final des verbes en (\ref{ex:VilloingDeglas:10}) comme une
marque spécifiquement verbale~:

\begin{enumerate}[label=(\alph*)]
\item soit une marque de classe flexionnelle (une voyelle thématique).

\item soit une marque identifiant la catégorie verbe.
\end{enumerate}

En effet, aucune des deux hypothèses ne tient~: l'hypothèse (a) d'une
voyelle thématique tombe car le créole n'a pas de système flexionnel
pour les verbes, et il n'y aurait aucune pertinence à exploiter une
voyelle thématique~; et l'hypothèse (b) tombe aussi parce que les
finales vocaliques des verbes sont variées (finale en /i/, /e/, /wɛ/
présentées ci-dessus Table~\ref{tab:VilloingDeglas:1}), auxquelles on peut ajouter celles en
/o/, /j/, /õ/ en (\ref{ex:VilloingDeglas:12}), et on peut difficilement imaginer que la langue
dispose d'autant de marqueurs verbaux, en particulier parce que les
noms, aussi, présentent des finales vocaliques en /e/, qu'ils soient ou
non hérités\is{inheritance} (cf.~(\ref{ex:VilloingDeglas:13a}) pour les noms hérités\is{inheritance}, et~(\ref{ex:VilloingDeglas:13b}) pour les noms
créoles)~:

\ea\label{ex:VilloingDeglas:12}
  \ea\label{ex:VilloingDeglas:12a} \gll bo, mò, cho, {fè kò}\\
  `embrasser', {`être mort'}, {`avoir chaud'}, {`unifier'}\\

  \ex\label{ex:VilloingDeglas:12b} \gll tòtòy, kay\\
  `agacer', `aller'\\
  \ex\label{ex:VilloingDeglas:12c} \gll mawon, fè-fon\\
  `s'échapper', {`compter sur'}\\
  \z

\ex\label{ex:VilloingDeglas:13}
  \ea\label{ex:VilloingDeglas:13a} \gll chaplé, bondyé, fiyansé, pyé, zyé, sòsyé\\
  `chapelet', `dieu', `fiancé', `pied', `yeux', `sorcier'\\

 \ex\label{ex:VilloingDeglas:13b} \gll bankoulélé, kyolé, matété, wélélé\\
  `vacarme', `ribambelle', {`plat à base de riz et crabe'}, `brouhaha'\\
 \z
\z

L'hypothèse d'une \isi{conversion} ne tient donc dans aucun cas. Comme le
--\emph{é} qui apparaît sur le verbe correspond à du matériel
phonologique supplémentaire par rapport au nom, et que la relation
catégorielle et sémantique change, tout porte à croire que le verbe est
morphologiquement plus complexe que le nom. Il faut donc faire
l'hypothèse d'une formation impliquant une suffixation verbale en
--\emph{é} sur bases nominales.

\subsubsection{L'impossible règle de formation des noms par suppression
du
--é}\label{subsubsection:VilloingDeglas:limpossible-ruxe8gle-de-formation-des-noms-par-suppression-du-é}

Une autre hypothèse aurait également pu être envisagée, celle d'une
règle de construction de noms sur base verbale, par suppression du
--\emph{é} final du verbe (ou une «~rétroformation~»). Mais cette
hypothèse rencontre plusieurs difficultés~:

\begin{enumerate}[label=(\alph*)]
\item la première tient à ce que ce mode de formation est jugé
traditionnellement rare dans les langues %
(sur la «~subtractive morphology~» ou «~deletion~» et sa rareté, voir ce qu'en disent les
manuels, comme %
%Anderson 1992
\citealt{Anderson92}%
%
~: 64-66~; %
%Haspelmath 2002
\citealt{Haspelmath2002}%
%
~: 24~; 
%Fradin 2003
\citealt{Fradin2003}%
%
~: 47)
%\citep[sur la «~subtractive morphology~» ou «~deletion~» et sa rareté, voir ce qu'en disent les manuels, comme ][~: 64-66~; Haspelmath 2002~: 24~; Fradin 2003~: 47]{Anderson92}%
%Anderson
%


\item la deuxième s'appuie sur des paires Nom/Verbe dont le nom est
hérité\is{inheritance} du français mais pas le verbe qui est nécessairement construit en
créole (\ref{ex:VilloingDeglas:14})~; or ce verbe laisse apparaître un --\emph{é} final
supplémentaire.
\end{enumerate}


\ea\label{ex:VilloingDeglas:14}
\ea \gll alyans \textrightarrow{~} alyansé \\
  `alliance' {} {`se lier'}\\
\ex \gll bwa \textrightarrow{~} bwaré \\
  `bras' {} {`enlacer'}\\
\ex \gll chikann \textrightarrow{~} chikanné \\
  `contestation' {} {`contester'}\\
\ex \gll fèr \textrightarrow{~} féré \\
  {`fer à cheveux'} {} {`défriser les cheveux'}\\
\ex \gll janm \textrightarrow{~} janbé \\
  `jambe' {} {`enjamber'}\\
\ex \gll tij \textrightarrow{~} tijé \\
  `bourgeon' {} {`bourgeonner'}\\
\z\z

\begin{enumerate}
\item[] Comme le nom est hérité\is{inheritance} du français, et le verbe construit en créole, le
nom ne peut pas être dérivé du verbe par une règle de suppression du
--\emph{é} final du verbe~; c'est bien le verbe qui est formé par
suffixation sur la base du nom.

\item[(c)] le troisième argument s'appuie sur l'absence de noms déverbaux\is{derivation!verb-based noun}
créoles construits par suppression du --\emph{é} d'un verbe hérité\is{inheritance}. En
effet, notre corpus ne fournit aucun nom dérivé à partir de verbes
hérités\is{inheritance} par simple suppression de la finale en --\emph{é.} La
disparition de la finale en --\emph{é} des verbes hérités\is{inheritance} peut avoir
lieu à l'occasion d'une dérivation, mais uniquement lorsque la
dérivation se fait par suffixation (voir par exemple, (\ref{ex:VilloingDeglas:15}) pour la
suffixation V\textrightarrow{~} N en --\emph{è}/--\emph{ez}, (\ref{ex:VilloingDeglas:16}) pour la suffixation V\textrightarrow{~} N
en --\emph{aj}, et (\ref{ex:VilloingDeglas:17}) pour la suffixation V\textrightarrow{~} N en --\emph{asyon}).
\end{enumerate}

\ea\label{ex:VilloingDeglas:15}
  \ea \gll fiyansèz $\leftarrow$ fiyansé \\
  {`fiancée'} {} {`se fiancer'}\\
  \ex \gll kouyonnèz $\leftarrow$ kouyonné \\
  {`celle qui couillonne'} {} {`couillonner'}\\
  \ex \gll soutirèz $\leftarrow$ soutiré \\
  {`celui qui couvre les bêtises de qqun'} {} {`couvrir les bêtises de qqun'}\\
  \z
\ex\label{ex:VilloingDeglas:16}
  \ea \gll bokantaj $\leftarrow$ bokanté \\
  {`échange'} {} {`échanger'}\\
  \ex \gll diraj $\leftarrow$ diré \\
  {`qui dure'} {} {`durer'}\\
  \ex \gll konblaj $\leftarrow$ konblé \\
  {`comblement'} {} {`combler'}\\
  \z
\ex\label{ex:VilloingDeglas:17}
  \ea \gll pwofitasyon $\leftarrow$ pwofité \\
  {`profit'} {} {`profiter'}\\
  \ex \gll anmerdasyon $\leftarrow$ anmerdé \\
  {`emmerdement'} {} {`emmerder'}\\
  \ex \gll poursuivasyon $\leftarrow$ poursuiv \\
  {`poursuite par le diable'} {} {`poursuivre'}\\
  \z
\z

Une dérivation par \isi{conversion} (\ref{ex:VilloingDeglas:18}) n'imposera pas, quant à elle, la
disparition de la finale vocalique du verbe.

\ea\label{ex:VilloingDeglas:18}
  \ea \gll déboulé / déboulé \\
  {`défilé'} {} {`défiler rapidement'}\\
  \ex \gll lélé / lélé \\
  {`touillette'} {} {`touiller'}\\
  \ex \gll mayé / mayé \\
  {`mariage'} {} {`se marier'}\\
  \ex \gll pété / pété \\
  {`pet'} {} {`faire un pet'}\\
  \z
\z


La voyelle finale du verbe disparaissant uniquement dans le contexte
d'une dérivation dont le suffixe est à initiale vocalique, tout porte à
croire qu'une contrainte morpho-phonologique est en jeu (contrainte
d'évitement du hiatus, contrainte de taille\ldots{}) et invalide
l'hypothèse de l'existence d'une règle dérivationnelle de suppression.

\subsubsection{Conditions d'apparition}\label{subsubsection:VilloingDeglas:conditions-dapparition}

Ces arguments conduisent à envisager que les paires de convers\is{conversion} Nom/Verbe
du français ont subi une réanalyse de telle sorte qu'en créole, la
relation morphologique entre les noms et les verbes en --\emph{é} de
(\ref{ex:VilloingDeglas:13}) ne relève pas d'une \isi{conversion}, comme en français, mais d'une
suffixation verbale sur base nominale (N\textrightarrow{~}V). Ces paires ont été héritées\is{inheritance}
en nombre suffisant pour avoir fait système et permis de former
productivement, par analogie, d'autres verbes dénominaux\is{derivation!noun-based verb} suffixés par
-\emph{é}\is{suffixation!in --\emph{é}} sur des bases françaises ou non françaises comme en (\ref{ex:VilloingDeglas:19}).

\ea \label{ex:VilloingDeglas:19}
  \ea \gll bòk / boké\\
  {`affront'} {} {`faire un affront'}\\
  \ex \gll chiktay / chiktayé\\
  {`émiettage'} {} {`émietter'}\\
  \ex \gll fèr / féré\\
  {`fer à cheveux'} {} {`défriser'}\\
  \ex \gll lyann / lyanné\\
  {`liane'} {} {`se servir d'un tuteur pour grimper'}\\
  \ex \gll dousin / dousiné\\
  {`câlin'} {} {`câliner'}\\
  \ex \gll djòb / djobé\\
  {`petit boulot'} {} {`faire un petit boulot'}\\
  \ex \gll plòk / ploké\\
  {`cloque'} {} {`avoir des cloques'}\\
\z\z


Ainsi, la réanalyse de ces paires Nom/Verbe héritées\is{inheritance} a abouti à la
création d'un suffixe verbal --\emph{é} en créole, inexistant dans la
langue lexificatrice. Ce schéma morphologique est représenté sous (\ref{ex:VilloingDeglas:20})
où X est mis pour le lexème base (et non le radical\is{stem} qui peut subir des
modifications phonologiques lors de la suffixation comme nous le
présentons en §~\ref{subsection:VilloingDeglas:réanalyse-des-paires-nv-de-préfixé-en-parasynthétiques})~:

\ea \label{ex:VilloingDeglas:20} X\textsubscript{\textsc{n}} \textrightarrow{~} Xé\textsubscript{\textsc{v}}
\z

La création de ce schéma morphologique n'a rien d'inédit à travers les
langues~; il peut s'apparenter à ce que la littérature dédiée aux
mécanismes et aux motivations du changement dans la formation des mots
appelle «~secretion~» %
%(Rainer 2015~: 1771)
\citep[1771]{Rainer15}%
%Rainer
%
. Ce concept repris à %
%Jesperson (1922~: 384)
\citet[384]{Jespersen22}
, réfère à un processus par lequel une séquence
purement phonologique acquiert le statut de «~morphème~» (phénomène déjà
signalé, selon %
%Rainer 2015
\citealt{Rainer15}%
, par %
%Bloomfield 1891
\citealt{Bloomfield1891}%
, ou 
%Lass 1990
\citealt{Lass90} %
%
 qui parle
de «~exaptation~»\footnote{Ce cas est à distinguer de
  ce que %
%Haspelmath (1995: 8-10) 
\citet[8--10]{Haspelmath1995} %
%Haspelmath
%
appelle «~secretion~» qui fait
  référence à une extension d'un affixe par l'incorportation d'une
  partie non affixale de la racine (schématisé sous (a))

\begin{exe}
\exi{(a)} Affix secretion

  Xyz \textrightarrow{~} xyz-a

  R $\Rightarrow$ -za

  $\Rightarrow$ new suffix --za, e.g. klm \textrightarrow{~} klm-za
\end{exe}}%
).


  \begin{quote}
By secretion I understand the phenomenon that one portion of an
indivisible word comes to acquire a grammatical signification which it
had not at first, and is then felt as something added to the word
itself. %
%(Rainer 2015~: 1771)
\citep[1771]{Rainer15}%
%Rainer
%

\end{quote}

Il peut également s'apparenter à un cas de «~degrammaticalization~» ou
de «~deinflectionalization~» %
%(Rainer 2015~: 1768-69) 
\citep[1768--69]{Rainer15} %
%Rainer
%
dans la mesure où
la finale flexionnelle du verbe français héritée\is{inheritance} (/e/) devient un
suffixe dérivationnel.

Quoi qu'il en soit, les conditions requises pour aboutir à la naissance
du suffixe verbal dénominal\is{derivation!noun-based verb} --\emph{é} en créole lui sont spécifiques.
Nous stipulons qu'elles sont les suivantes~:

\begin{enumerate}
\def\labelenumi{\arabic{enumi})}

\item
  d'une part, la très forte représentativité, dans le lexique créole, de
  couples morphologiques Nom/Verbe hérités\is{inheritance} du français où ils
  entretiennent une relation de \isi{conversion} ;
\item
  d'autre part, au sein de ces couples, une très forte majorité de
  verbes à finale en --\emph{é}~;
\item
  et enfin, la propriété des lexèmes verbaux créoles de n'apparaître que
  sous une forme unique~: ainsi la marque flexionnelle des verbes
  hérités\is{inheritance} n'a pu être interprétée comme flexionnelle en créole.
\end{enumerate}

C'est la conjonction de ces trois conditions qui a rendu possible la
création de ce suffixe en créole guadeloupéen. Si l'une de ces
conditions n'avait pas été remplie, il y a fort à parier qu'aucun
nouveau schéma morphologique n'aurait pu voir le jour. Par exemple, tous
les verbes créoles hérités\is{inheritance} du français remplissent la condition 3), mais
seules les finales en --\emph{é} des verbes hérités\is{inheritance} du français ont été
réanalysées comme une règle de suffixation de verbes dénominaux\is{derivation!noun-based verb}. Cela
tient aux conditions 1) et 2) réunies~: seules les paires héritées\is{inheritance} du
français Nom / Verbe à finale en --\emph{é} ont été héritées\is{inheritance} en grand nombre, à
l'exception d'autres finales verbales. Toutes les autres paires
Nom/Verbe apparaissent en nombre infime et la deuxième condition
présentée ci-dessus n'est pas remplie. En effet, même si le guadeloupéen
compte un certain nombre de verbes hérités\is{inheritance} présentant une autre finale
que --\emph{é} (cf. la table ci-dessus), ces verbes soit ne sont
reliés à aucun nom (comme~(\ref{ex:VilloingDeglas:21}) pour les verbes en \emph{--i}), soit ils le
sont, mais uniquement dans une relation de \isi{conversion} (\ref{ex:VilloingDeglas:22} pour les
verbes en (\emph{--i})), soit le nom relié est difficile à mettre en
relation morphologique avec le verbe à cause d'une variation
phonologique entre les deux trop importante (cf.~(\ref{ex:VilloingDeglas:23}) pour les verbes en
--\emph{i}).

\ea \label{ex:VilloingDeglas:21}
  \ea \gll abouti\\
  {`aboutir'}\\
  \ex \gll aji\\
  {`agir'}\\
  \ex \gll dégarni\\
  {`dégarnir'}\\
  \ex \gll flétri\\
  {`flétrir'}\\
  \ex \gll konstwi\\
  {`construire'}\\
  \z
\ex \label{ex:VilloingDeglas:22}
  \ea \gll amòrti / amòrti\\
  {`amortir'} {} {`amorti'}\\
  \ex \gll anvi / anvi\\
  {`avoir envie'} {} {`envie'}\\
  \ex \gll griji / griji\\
  {`s'égratigner'} {} {`égratignure'}\\
  \ex \gll jwi / jwi\\
  {`jouir'} {} {`sperme'}\\
  \ex \gll vèrni / vèrni\\
  {`vernir'} {} {`verni'}\\
  \z
\ex \label{ex:VilloingDeglas:23}
  \ea \gll chwa / chwazi\\
  {`choix'} {} {`choisir'}\\
  \ex \gll fen / fini\\
  {`fin'} {} {`finir'}\\
  \ex \gll flè / fléri\\
  {`fleur'} {} {`fleurir'}\\
  \ex \gll kous / kouri\\
  {`course'} {} {`courir'}\\
  \ex \gll trèt / trayi\\
  {`traître'} {} {`trahir'}\\
\z\z

Finalement, les verbes hérités\is{inheritance} qui ne remplissent pas les conditions 1)
et 2) ne donnent lieu à aucune création créole. Pour reprendre l'exemple
des verbes en --\emph{i}, les seuls de notre corpus qui ne soient pas
hérités\is{inheritance} ne sont pas dérivés par un suffixe verbalisateur --\emph{i}
(\ref{ex:VilloingDeglas:24})~:

\ea \label{ex:VilloingDeglas:24}
  \ea \gll bigidi\\
  {`faiblir'}\\
  \ex \gll bénékaki\\
  {`hésiter'}\\
  \ex \gll siri\\
  {`devenir aigre'}\\
  \ex \gll tini\\
  {`avoir'}\\
  \z
\z

Les trois conditions nécessaires à la création du suffixe --\emph{é} ne
sont pas propres au guadeloupéen et se sont retrouvées dans d'autres
créoles à base française. En effet, plusieurs créoles ont suivi le même
processus et la suffixation en --\emph{é} compte parmi les schémas
morphologiques disponibles du Haïtien %
%(DeGraff 2001, Lefebvre 1998,
%2003) 
\citep{DeGraff2001,Lefebvre98,Lefebvre2003} %
%DeGraff;Lefebvre;Lefebvre
%
et du Saint-Lucien %
%(Bath \& Nickiema 2000, Brousseau 2011)
\citep{Bhatt00,Brousseau11}%
%Brousseau
%
. Elle
n'a néanmoins jamais fait l'objet d'études de détails dans les travaux
portant sur ces créoles.

\subsubsection{\texorpdfstring{Propriétés du suffixe verbal dénominal
--\emph{é }en
créole}{Propriétés du suffixe verbal dénominal --é en créole}}\label{propriétés-du-suffixe-verbal-dénominal-é-en-créole}

\subsubsubsection{Forme phonologique du
suffixe}\label{forme-phonologique-du-suffixe}

Nous postulons que la forme phonologique du suffixe verbal dénominal\is{derivation!noun-based verb}
ainsi créé est /e/ (orthographié --\emph{é}). Cet affixe vocalique
apparaît dans certains contextes précédé d'une consonne, /t/ par défaut
(cf.~(\ref{ex:VilloingDeglas:25})) et il y a lieu de se demander si cette consonne à la frontière
entre le radical\is{stem} et le suffixe n'appartient pas au suffixe. Tout porte à
croire néanmoins que la consonne intercalaire est de nature
épenthétique, permettant, dans un contexte lexical, d'éviter la
succession de deux voyelles à la frontière entre la base et l'affixe.

\ea \label{ex:VilloingDeglas:25}
  \ea \gll konplo 	\textrightarrow{~} konploté\\
  {`complot'} {} {`comploter '}\\
  \ex \gll niméwo 	\textrightarrow{~} niméroté\\
  {`numéro'} {} {`numéroter'}\\
  \ex \gll soulyé 	\textrightarrow{~} soulyété\\
  {`chaussures'} {} {`mettre des chaussures'}\\
\z\z

Un premier argument en ce sens est le fait que l'évitement du hiatus en
créole guadeloupéen s'observe régulièrement à la frontière morphologique
dans les cas de dérivation~: citons, à titre d'exemple, la formation de
dérivés suffixés dont le suffixe à initiale vocalique entraîne la
suppression de la finale vocalique du verbe en \emph{--é}. Un deuxième
argument est le développement d'autres stratégies d'évitement du hiatus
en contexte morphologique, comme le recours à des règles de dérivation
permettant de contourner le problème, en l'occurrence la \isi{conversion} ou
la \isi{préfixation}. On peut ainsi affirmer que la suffixation en --\emph{é}
entraîne des changements phonologiques sur les bases nominales, dont les
épenthèses ne sont qu'un exemple %
(voir %
%Villoing \& Deglas 2016 
\citealt{Villoing16} %
%
pour plus de détails)
%\citep[voir ][ pour plus de détails]{Villoing16}%
%Villoing-Deglas
%
.

La présence de toute autre consonne entre le radical\is{stem} et le suffixe
relève de cas différents de l'épenthèse consonnantique ou de
l'allomorphie suffixale. Ainsi,

\begin{enumerate}[label=(\roman*)]
\item une réalisation spécifique des voyelles nasales en contexte de
dérivation en guadeloupéen, comme dans d'autres créoles à base française
%
%(cf. Bhatt \& Nikiema 2000)
\citep[cf.][]{Bhatt00}%
%Bhatt-Nikiema
%
, laisse apparaître une consonne nasale à la
suite de la voyelle nasale du radical\is{stem} lors de la suffixation en
--\emph{é} (cf.~(\ref{ex:VilloingDeglas:25b})) ;

\item la réalisation de consonnes lexicales héritées\is{inheritance} des lexèmes français
qui se révèlent uniquement dans ce contexte dérivationnel (le suffixe
protégeant la consonne), puisqu'elles ont par ailleurs disparu en finale
(cf.~(\ref{ex:VilloingDeglas:25c}))~:

\end{enumerate}


\ea \label{ex:VilloingDeglas:25b}
  \ea \gll boukan \textrightarrow{~} boukanné\\
  {`feu de brindille'} {} {`griller au feu de bois'}\\
  \ex \gll dirèksyon \textrightarrow{~} dirèksyonné\\
  {`direction'} {} {`montrer la direction'}\\
  \ex \gll gidon \textrightarrow{~} gidonné\\
  {`guidon'} {} {`mener'}\\
  \ex \gll losyon \textrightarrow{~} losyonné\\
  `lotion' {} {`se parfumer'}\\
  \ex \gll migan \textrightarrow{~} miganné\\
  {`purée'} {} {`mélanger'}\\
  \z
\ex \label{ex:VilloingDeglas:25c}
  \ea \gll arbit \textrightarrow{~} arbitré\\
  `arbitre' {} {`arbitrer, trancher'}\\
  \ex \gll chalè \textrightarrow{~} chaléré\\
  {`chaleur'} {} {`s'inquiéter'}\\
  \ex \gll janm \textrightarrow{~} janbé\\
  {`jambe'} {} {`enjamber'}\\
  \ex \gll tè \textrightarrow{~} téré\\
  {`terre'} {} {`enterrer'}\\
  \ex \gll penti \textrightarrow{~} pentiré\\
  {`peinture'} {} {`peindre'}\\
\z
\z

\subsubsubsection{Propriétés sémantiques de la
règle}\label{propriétés-sémantiques-de-la-ruxe8gle}

La relation sémantique entre le nom de base (désormais Nbase) et le
verbe dénominal\is{derivation!noun-based verb} suffixé en --\emph{é}\is{suffixation!in --\emph{é}}  apparaît, pour une part, typique
de ce type de construction morphologique en français et pour une autre
part originale.

Elle est typique dans les cas où le Nbase renvoie aux actants du verbe
comme l'instrument en (\ref{ex:VilloingDeglas:26}) (qui comprend aussi bien les artefacts (\ref{ex:VilloingDeglas:26a})
que les parties du corps (\ref{ex:VilloingDeglas:26b})), à un agent en (\ref{ex:VilloingDeglas:27}), à une entité
déplacée (\emph{locatum verbs, figure verbs}) en (\ref{ex:VilloingDeglas:28a}), au lieu du
procès\is{process} (\emph{location verbs, grounds-verbs}) en (\ref{ex:VilloingDeglas:28b}), et à l'objet
résultant du procès\is{process} en (\ref{ex:VilloingDeglas:29}).

\ea \label{ex:VilloingDeglas:26} N~: instrument

  \ea\label{ex:VilloingDeglas:26a}
  \ea \gll fak \textrightarrow{~} faké\\
  {`bêche'} {} {`bêcher'}\\
  \ex \gll kòn \textrightarrow{~} koné\\
  {`klaxon'} {} {`klaxonner'}\\
  \ex \gll graj \textrightarrow{~} grajé\\
  {`rape'} {} {`raper'}\\
  \ex \gll pikwa \textrightarrow{~} pikwaté\\
  {`pioche'} {} {`piocher'}\\
  \z

  \ex\label{ex:VilloingDeglas:26b}
  \ea \gll lang \textrightarrow{~} langé\\
  {`langue'} {} {`embrasser avec la langue'}\\
  \ex \gll bwa \textrightarrow{~} bwaré\\
  {`bras'} {} {`enlacer'}\\
  \ex \glll zig \textrightarrow{~} zigé\\
  {`position des doigts} {} {`faire une
  pichenette'}\\
  {pour faire une pichenette'} {} {}\\
  \ex \gll zyé \textrightarrow{~} zyété\\
  {`yeux'} {} {`surveiller'}\\
  \z\z

\ex \label{ex:VilloingDeglas:27} N~: agent

  \ea \gll mako \textrightarrow{~} makoté\\
  {`mouchard'} {} {`moucharder'}\\
  \ex \gll makrèl \textrightarrow{~} makrélé\\
  {`celle qui se mêle de tout'} {} {`surveiller'}\\
  \ex \gll mandyan \textrightarrow{~} mandyanné\\
  {`mendiant'} {} {`mendier'}\\
  \z

\newpage 
\ex \label{ex:VilloingDeglas:28}

  \ea\label{ex:VilloingDeglas:28a} N~: entité déplacée
  \ea \gll bonda \textrightarrow{~} bondaté\\
  {`fesses'} {} {`poser ses fesses'}\\
  \ex \gll janb \textrightarrow{~} janbé\\
  {`jambe'} {} {`enjamber'}\\
  \ex \gll pyé \textrightarrow{~} pyété\\
  {`pied'} {} {`poser le pied'}\\
  \ex \gll soulyé \textrightarrow{~} soulyété\\
  {`chaussures'} {} {`mettre des chaussures'}\\
  \z\z

  \ea\label{ex:VilloingDeglas:28b} N~: lieu final du procès
  \ea \gll balkon \textrightarrow{~} balkonné \\
  `balcon' {} {`être au balcon'}\\
  \ex \gll kabann \textrightarrow{~} kabanné\\
  {`lit'} {} {`traîner au lit'}\\
  \ex \gll kan \textrightarrow{~} kanté\\
  {`côté'} {} {`se mettre sur le côté, sur le flanc'}\\
  \z\z

\ex \label{ex:VilloingDeglas:29} N~: objet résultant
  \ea \gll flang \textrightarrow{~} flangé\\
  {`entaille'} {} {`entailler'}\\
  \ex \gll migan \textrightarrow{~} miganné\\
  {`purée'} {} {`mélanger'}\\
  \ex \gll fifin \textrightarrow{~} fifiné\\
  {`bruine'} {} {`bruiner'}\\
  \ex \gll tij \textrightarrow{~} tijé\\
  {`bourgeon'} {} {`bourgeonner'}\\
\z\z

La relation sémantique entre le Nbase et le verbe dérivé suffixé en
--\emph{é}\is{suffixation!in --\emph{é}} est néanmoins atypique dans les exemples (\ref{ex:VilloingDeglas:30}) où le Nbase
dénote une situation dynamique %
(voir %
%Villoing \& Deglas 2016 
\citealt{Villoing16} %
%
pour une présentation des tests d'événementialité)
%\citep[voir ][ pour une présentation des tests d'événementialité]{Villoing16}%
%Villoing-Deglas
%
~:

\ea \label{ex:VilloingDeglas:30}
  \ea \gll bonbans \textrightarrow{~} bonbansé\\
  {`fête'} {} {`faire la fête'}\\
  \ex \gll chikann \textrightarrow{~} chikanné\\
  {`contestation'} {} {`contester'}\\
  \ex \gll chiktay \textrightarrow{~} chiktayé\\
  {`émiettage'} {} {`émietter'}\\
  \ex \gll dousin \textrightarrow{~} dousiné\\
  {`caresse'} {} {`caresser'}\\ 
  \ex \gll driv \textrightarrow{~} drivé\\
  {`promenade'} {} {`promener'}\\
  \ex \gll kalbann \textrightarrow{~} kalbanné\\
  {`culbute'} {} {`culbuter'}\\
\z\z

En effet, en français, les «~noms d'événément~» sont prototypiquement
déverbaux\is{derivation!verb-based noun} et les cas de noms d'événement\is{noun!event noun} servant de base à la formation
d'un verbe dérivé restent minoritaires. Par exemple, %
%Corbin (2004) 
\citet{corbin04} %
%Corbin
%
note,
en français, quelques verbes suffixés construits sur des noms simples
dénotant des procès\is{process} (\emph{guerroyer} et \emph{satiriser} construits sur
les noms processifs\is{process}, \emph{guerre} et \emph{satire)}. Mais ces exemples
sont forcément très peu nombreux,

\begin{itemize}
\item[---] d'une part parce que les noms simples dénotant un événement\is{noun!event noun} restent
rares dans le lexique du français (\emph{le concert, l'orage}~; ils
représentent 8,1\% des noms simples\is{noun!action noun} selon %
%Tribout et \emph{alii}. 2014
\citealt{Tribout2014}%
%
)
et sont, en général, issus de noms déverbaux\is{derivation!verb-based noun} en latin~;

\item[---] d'autre part, parce que si les bases des verbes en --\emph{iser}
peuvent être processives, ce cas se présente rarement, aussi bien en
anglais %
%(Plag 1999) 
\citep{Plag1999} %
%Plag
%
qu'en français %
%(Namer 2013)
\citep{namer2013}%
%Namer
%
~;

\item[---] et enfin, la suffixation en --\emph{oyer} apparaît peu productive.
\end{itemize}

Cette rareté vient confirmer l'hypothèse de %
%Croft (1991) 
\citet{Croft91} %
%Croft
%
selon laquelle
les noms dénotent prototypiquement des objets.

La situation semble \rephrase{}{être} différente lorsque les bases nominales processives
sont elles-mêmes complexes morphologiquement. En effet, quelques travaux
récents sur le français ont mentionné la relative disponibilité de
certains noms construits dénotant des événements\is{noun!event noun} à servir de base à la
formation d'un verbe. %
%Tribout (2010)
\citet{Tribout2010a}%
%Tribout
%
, par exemple, montre qu'un nombre
non négligeable de verbes dénominaux\is{derivation!noun-based verb} convers\is{conversion} sont formés sur des noms
événementiels\is{noun!event noun} déverbaux\is{derivation!verb-based noun} (\ref{ex:VilloingDeglas:31})~:

\ea \label{ex:VilloingDeglas:31}
  \ea\label{ex:VilloingDeglas:31a} louer \textrightarrow{~} louange \textrightarrow{~} louanger
  \ex\label{ex:VilloingDeglas:31b} vider \textrightarrow{~} vidange \textrightarrow{~} vidanger
  \ex\label{ex:VilloingDeglas:31c} recevoir \textrightarrow{~} réception \textrightarrow{~} réceptionner
  \ex\label{ex:VilloingDeglas:31d} frotter \textrightarrow{~} friction \textrightarrow{~} frictionner
  \ex\label{ex:VilloingDeglas:31e} partir \textrightarrow{~} partage \textrightarrow{~} partager
\z\z

%
%Tribout (2010) 
\citet{Tribout2010a} %
%Tribout
%
l'explique par le fait que le nom base a perdu sa
motivation morphologique et que la perception de sa construction sur
base verbale n'existe plus (par exemple, (\ref{ex:VilloingDeglas:31c}), (\ref{ex:VilloingDeglas:31d}), (\ref{ex:VilloingDeglas:31e})). Mais pour
d'autres paires, la relation entre le nom abstrait et son verbe base
reste tout à fait transparente (par exemple, (\ref{ex:VilloingDeglas:31a}), (\ref{ex:VilloingDeglas:31b})).

C'est un résultat que partagent %
%Lignon et Namer (2014) 
\citet{Lignon14} %
%Lignon-Namer
%
sur d'autres cas
de \isi{conversion} du français, les noms abstraits suffixés en --\emph{ion}
servant de bases à la formation de verbes convers\is{conversion}, alors que ces noms
sont construits sur des bases verbales facilement reconstructibles
(\ref{ex:VilloingDeglas:32})~:

\newpage
\ea \label{ex:VilloingDeglas:32}
\ea attirer \textrightarrow{~} attraction \textrightarrow{~} attractionner
\ex  intercéder \textrightarrow{~} intercession \textrightarrow{~} intercesser
\ex  soumettre \textrightarrow{~} soumission \textrightarrow{~} soumissionner
\ex  voir \textrightarrow{~} vision \textrightarrow{~} visionner
\z\z

Parallèlement, une autre formation permet de construire des verbes sur
des bases nominales événementielles\is{noun!event noun}, la rétroformation à partir de
composés néoclassiques %
%(Namer 2012) 
\citep{Namer12} %
%Namer
%
(cf. (\ref{ex:VilloingDeglas:33})).

\ea \label{ex:VilloingDeglas:33}
\ea photoémission \textrightarrow{~} photoémettre
\ex  hydromassage \textrightarrow{~} hydromasser
\ex  hydroextraction \textrightarrow{~} hydroextraire
\z\z

Ainsi, la formation d'un verbe ayant pour base un nom d'événement\is{noun!event noun} en
français (i) n'est disponible que pour des bases nominales
morphologiquement construites et (ii) la règle impliquée est
préférentiellement la \isi{conversion}. Cette configuration spécifique ne se
retrouve pas dans les données du créole guadeloupéen étudiées ci-dessus
qui font état d'une règle de suffixation sur base nominale
événementielle\is{noun!event noun} morphologiquement simple. Le créole présente donc une
originalité sémantique par rapport au français tout à fait intéressante.
Nous l'attribuons à la formation très spécifique de la règle de
suffixation en --\emph{é} qui est issue de la réanalyse de paires
Nom/Verbe du français relevant de deux règles de \isi{conversion}~: la
\isi{conversion} V\textrightarrow{~} N et N\textrightarrow{~} V.

\subsection{Réanalyse des paires N/V-préfixé en
parasynthétiques}\label{subsection:VilloingDeglas:réanalyse-des-paires-nv-de-préfixé-en-parasynthétiques}
\is{parasynthesis}
L'absence de flexion verbale en créole guadeloupéen et l'héritage\is{inheritance} d'une
forme unique du verbe français (en l'occurrence, pour les verbes qui
nous intéressent, la forme de l'infinitif ou du participe passé en /e/)
entraînent d'autres réanalyses morphologiques. Ainsi, les paires héritées\is{inheritance}
en (\ref{ex:VilloingDeglas:34}), dont le verbe est formé en français par \isi{préfixation}, ne peuvent
s'analyser en créole qu'en terme de parasynthèse\is{parasynthesis}.

\ea \label{ex:VilloingDeglas:34}
      \ea \gll bò / débòrdé\\
        {`bord'} {} {`déborder'}\\
      \ex \gll  frich / défriché\\
        {`friche'} {} {`défricher'}\\
      \ex \gll  kras / dékrasé\\
        {`crasse'} {} {`décrasser'}\\
      \ex \gll  mayo / démayoté\\
        {`étoffe d'emmaillotage de bébé'} {} {`démailloter'}\\
      \ex \gll  rasin / dérasiné\\
        {`racine'} {} {`déraciner'}\\
      \z
\z

Les paragraphes qui suivent argumentent en faveur de cette hypothèse et
présentent les propriétés phonologiques et sémantiques associées à ce
schéma morphologique \rephrase{}{qui est} propre au créole.

\subsubsection{\texorpdfstring{\emph{dé}-N-\emph{é}\textsubscript{\textsc{v}} parasynthétiques
}{Dé-N-éV parasynthétiques }}\label{dé-n-év-parasynthétiques}
\is{parasynthesis}
\is{parasynthesis!in \emph{dé}--\emph{é}}

Les exemples de paires morphologiques Nom/Verbe en (\ref{ex:VilloingDeglas:34}) héritées\is{inheritance} du
français ne supportent pas la même analyse morphologique en créole
guadeloupéen et conduisent à envisager un nouveau cas de réanalyse
morphologique. Là où, en français, l'analyse reconnaît un dérivé verbal
au moyen d'une \isi{préfixation} en \emph{dé}- sur base nominale, le créole,
quant à lui, forme un verbe par parasynthèse\is{parasynthesis} sur base nominale.

Le raisonnement qui conduit à ce résultat est proche de celui qui a mené
à identifier la création du suffixe dénominal\is{derivation!noun-based verb} verbalisateur --\emph{é}\is{suffixation!in --\emph{é}}~:
les verbes créoles ne se réalisant que sous une forme unique, la finale
en --\emph{é} appartient bien à la forme lexicale du verbe et ne
correspond pas à l'affixe d'infinitif apparaissant dans la forme
citationnelle du verbe. Ainsi, entre la base nominale et le verbe
dérivé, du matériel phonologique supplémentaire apparaît aux deux
extrémités~: à gauche de la base, un préfixe \emph{dé}--, et à droite de
la base, le suffixe verbalisateur --\emph{é}\is{suffixation!in --\emph{é}}. Or ces affixes ne relèvent
pas de l'application successive de deux règles morphologiques. En effet,
ni le verbe en --\emph{é}\is{suffixation!in --\emph{é}} (\ref{ex:VilloingDeglas:35b}) ni le nom en \emph{dé}-- (\ref{ex:VilloingDeglas:35c}) n'existent
indépendamment l'un de l'autre.

\is{parasynthesis!in \emph{dé}--\emph{é}}

\ea \label{ex:VilloingDeglas:35b}
      \ea[*]{\gll bòdé\textsubscript{\textsc{v}} \textrightarrow{~} débòdé\textsubscript{\textsc{v}}\\
        {} {} {`déborder'}\\}
      \ex[*]{\gll friché\textsubscript{\textsc{v}} \textrightarrow{~} défriché\textsubscript{\textsc{v}}\\
        {} {} {`défricher'}\\}
      \ex[*]{\gll krasé\textsubscript{\textsc{v}} \textrightarrow{~} dékrasé\textsubscript{\textsc{v}}\\
        {} {} {`décrasser'}\\}
      \ex[*]{\gll mayoté\textsubscript{\textsc{v}} \textrightarrow{~} démayoté\textsubscript{\textsc{v}}\\
        {} {} {`démailloter'}\\}
      \ex[*]{\gll rasiné\textsubscript{\textsc{v}} \textrightarrow{~} dérasiné\textsubscript{\textsc{v}}\\
        {} {}  {`déraciner'}\\}
      \z
\ex \label{ex:VilloingDeglas:35c}
      \ea[*]{\gll débòd\textsubscript{\textsc{n}} \textrightarrow{~} débòdé\textsubscript{\textsc{v}}\\
        {} {} {`déborder'}\\}
      \ex[*]{\gll défrich\textsubscript{\textsc{n}} \textrightarrow{~} défriché\textsubscript{\textsc{v}}\\
        {} {} {`défricher'}\\}
      \ex[*]{\gll dékras\textsubscript{\textsc{n}} \textrightarrow{~} dékrasé\textsubscript{\textsc{v}}\\
        {} {} {`décrasser'}\\}
      \ex[*]{\gll démayo\textsubscript{\textsc{n}} \textrightarrow{~} démayoté\textsubscript{\textsc{v}}\\
        {} {} {`démailloter'}\\}
      \ex[*]{\gll dérasin\textsubscript{\textsc{n}} \textrightarrow{~} dérasiné\textsubscript{\textsc{v}}\\
      {} {}   {`déraciner'}\\}
\z\z

Ainsi, les exemples en (\ref{ex:VilloingDeglas:34}) ne peuvent ni être analysés comme des
préfixés\is{prefixation} en \emph{dé}-- sur base verbale (le verbe n'existe pas), ni
comme des verbes suffixés en --\emph{é}\is{suffixation!in --\emph{é}} sur base nominale (ces bases
n'existant pas non plus). Ces propriétés rappellent les critères
traditionnellement avancés pour reconnaître une parasynthèse\is{parasynthesis} %
(cf. %
%Darmesteter 1894
\citealt{darmesteter1894.traite-formation}%
%
~: 24 présentés ci-dessus au §~\ref{sec:vg:2.1.1.}, %
%Corbin 1987
\citealt{Corbin87}%
%
~: 121-125, %
%Fradin 2003
\citealt{Fradin2003}%
%
~: 288-306).
%\citep[cf. ][~: 24 présentés ci-dessus au § 2.1.1., Corbin 1987~: 121-125, Fradin 2003~: 288-306]{darmesteter1894.traite-formation}%
%Darmesteter
%
Comme la seule relation morphologique
possible est celle existant entre le Nom base et le Verbe dérivé, et
qu'elle se manifeste par une \isi{préfixation} et suffixation simultanée
(\isi{préfixation} en \emph{dé}-- et suffixation en --\emph{é}\is{suffixation!in --\emph{é}}), alors on est
en droit de faire l'hypothèse d'une réanalyse en guadeloupéen des paires
Nom/Verbe-préfixé\is{prefixation} du français en parasynthétiques\is{parasynthesis} créoles.

De même que les paires Nom/Verbe à finale en --\emph{é} présentées en
\rephrase{}{section}
\ref{subsection:VilloingDeglas:réanalyses-des-paires-nv-de-convers-comme-suffixations},
les paires Nom/Verbe à initiale en \emph{dé}-- héritées\is{inheritance} l'ont été
en grand nombre et le schéma morphologique créé à l'issue de cette
réanalyse est devenu productif en créole, comme l'attestent les
créations de (\ref{ex:VilloingDeglas:36})~:

\ea \label{ex:VilloingDeglas:36}
      \ea \gll chèpi \textrightarrow{~} déchèpiyé\\
        {`charpie'} {} {`mettre en charpie'}\\
      \ex \gll  chouk \textrightarrow{~} déchouké\\
        {`souche'} {} {`déraciner'}\\
      \ex \gll  pat \textrightarrow{~} dépaté\\
        {`main de banane'} {} {`retirer les mains du régime de banane'}\\
      \ex \gll  tik \textrightarrow{~} détiké\\
        {`tique'} {} {`retirer les tiques'}\\
      \ex \gll  zo \textrightarrow{~} dézosé\\
        {`os'} {} {`désosser'}\\
\z\z

A l'image des paires héritées\is{inheritance} réanalysées de (\ref{ex:VilloingDeglas:34}), les créations créoles
de (\ref{ex:VilloingDeglas:36}) s'analysent comme des formations verbales parasynthétiques\is{parasynthesis} dans
la mesure où ni le verbe en --\emph{é}\is{suffixation!in --\emph{é}} (\ref{ex:VilloingDeglas:37}) ni le nom en \emph{dé}-- (\ref{ex:VilloingDeglas:38})
n'existent indépendemment l'un de l'autre~:

\ea \label{ex:VilloingDeglas:37}
      \ea[*]{\gll chèpiyé \textrightarrow{~} déchèpiyé\\
        {} {}   {`mettre en charpie'}\\}
      \ex[*]{\gll chouké \textrightarrow{~} déchouké\\
        {} {}   {`déraciner'}\\}
      \ex[*]{\gll paté \textrightarrow{~} dépaté\\
        {} {}   {`retirer la main de bananes du régime'}\\}
      \ex[*]{\gll tiké \textrightarrow{~} {détiké}\\
        {} {}   {`retirer les tiques' }\\}
      \ex[*]{\gll zosé \textrightarrow{~} dézosé\\
        {} {}   {`désosser'}\\}
      \z
\ex \label{ex:VilloingDeglas:38}
      \ea[*]{\gll déchèpi \textrightarrow{~} déchèpiyé\\
       {} {}    {`mettre en charpie'}\\}
      \ex[*]{\gll déchouk \textrightarrow{~} déchouké\\
      {} {}    {`déraciner'}\\}
      \ex[*]{\gll dépat \textrightarrow{~} dépaté\\
      {} {}    {`retirer la main de bananes du régime'}\\}
      \ex[*]{\gll détik \textrightarrow{~} {détiké}\\
      {} {}    {`retirer les tiques'}\\}
      \ex[*]{\gll dézo \textrightarrow{~} dézosé\\
      {} {}    {`désosser'}\\}
      \z
\z

Ainsi, les conditions requises pour aboutir à la naissance du schéma
morphologique (\ref{ex:VilloingDeglas:39}) en créole guadeloupéen, que nous avons posées au 
% §3.2.3,
§~\ref{subsubsection:VilloingDeglas:conditions-dapparition}
sont ici aussi respectées~:

\begin{enumerate}
\def\labelenumi{\arabic{enumi})}

\item
  la très forte représentativité, dans le lexique créole, de couples
  morphologiques Nom/Verbe à initiale en \emph{dé}-- hérités\is{inheritance} de
  préfixations\is{prefixation} verbales dénominales\is{derivation!noun-based verb} du français ;
\item
  une quasi-totalité de verbes à finale en --\emph{é}\is{suffixation!in --\emph{é}}, au sein de ces
  couples~;
\item
  et enfin, la propriété des lexèmes verbaux créoles de ne présenter
  qu'une forme, la marque flexionnelle des verbes hérités\is{inheritance} n'ayant pas
  été interprétée comme telle en créole.
\end{enumerate}

Nous pouvons ainsi poser que le créole guadeloupéen dispose d'un schéma
morphologique de parasynthèse\is{parasynthesis} du type (\ref{ex:VilloingDeglas:39}), où X représente le lexème de
base, de type nominal, et \emph{dé--\ldots{}--\emph{é}} l'affixe parasynthétique
(circumfixe) formant des verbes. Ce schéma rend aussi bien compte des
paires Nom/Verbe héritées\is{inheritance} du français de (\ref{ex:VilloingDeglas:34}) que de celles construites
en créole en (\ref{ex:VilloingDeglas:36})~:

\ea \label{ex:VilloingDeglas:39} Dé-X\textsubscript{\textsc{n}}-\emph{é}\textsubscript{\textsc{v}} 
\z

\subsubsubsection{Forme phonologique de l'affixe
}\label{forme-phonologique-de-laffixe}

La forme phonologique de l'affixe parasynthétique est /de-X-e/ (que nous
orthographions \emph{dé}-X-\emph{é}), où X représente la base nominale et \emph{dé}-- ... --\emph{é}
l'affixe. Les possibles con\-sonnes qui s'intercalent à droite, entre le
radical\is{stem} de base et le suffixe --\emph{é}\is{suffixation!in --\emph{é}} sont à analyser comme des
consonnes épenthétiques dans un contexte lexical gauche vocalique, à
l'image de ce que nous avons observé pour la suffixation en --\emph{é}\is{suffixation!in --\emph{é}}
(cf. §~\ref{subsubsection:VilloingDeglas:conditions-dapparition}), que ce soient pour les paires héritées\is{inheritance} (cf.~(\ref{ex:VilloingDeglas:40a})) ou
pour les paires créoles pour lesquelles nous n'observons qu'un exemple
(\ref{ex:VilloingDeglas:40b})~:

\ea \label{ex:VilloingDeglas:40}
      \ea\label{ex:VilloingDeglas:40a}
            \ea \gll  bò \textrightarrow{~} débòdé\\
              {`bord'} {} {`déborder'}\\
            \ex \gll  figi \textrightarrow{~} défigiré\\
              {`visage'} {} {`défigurer'}\\
            \ex \gll  ma \textrightarrow{~} dématé\\
              {`mât de bateau'} {} {`démâter (bateau), renverser, retourner'}\\
            \ex \glll  mayo  \textrightarrow{~} démayoté\\
             {`maillot, étoffe pour} {} `démailloter'\\
             {emmailloter un nouveau-né'} {} {} \\
            \ex \gll  zo \textrightarrow{~} dézosé\\
              {`os'} {} {`désosser'}\\
            \z
      \ex\label{ex:VilloingDeglas:40b} \gll  chèpi \textrightarrow{~} déchèpiyé\\
        {`charpie'} {} {`mettre en charpie'}\\
      \z
\z


L'allomorphie typique que présente le préfixe \emph{dé}-- en français et
dont a hérité\is{inheritance} le préfixe \emph{dé}-- créole (\emph{dé}-- devant verbe à
initiale consonantique et \emph{déz--} devant verbe à initiale
vocalique~; cf. (\ref{ex:VilloingDeglas:41a}) pour les paires héritées\is{inheritance} du français et (\ref{ex:VilloingDeglas:41b}) pour
les exemples de création créole) ne se retrouve pas dans notre corpus de
parasynthétiques\is{parasynthesis} \emph{dé}-X-\emph{é}.

  \ea \label{ex:VilloingDeglas:41}
        \ea\label{ex:VilloingDeglas:41a}
          \ea \gll dézabityé $\leftarrow$ abityé\\
            {`déshabituer'} {} {`s'habituer'}\\
          \ex \gll  dézakordé $\leftarrow$ akordé\\
            {`désaccorder'} {} {`accorder'}\\
          \ex \gll  dézanbalé $\leftarrow$ anbalé\\
            {`déballer'} {} {`emballer'}\\
          \ex \gll  dézanbwaté $\leftarrow$ anbwaté\\
            {`désemboîter'} {} {`emboîter'}\\
            \ex \gll  dézankonbré $\leftarrow$ ankonbré \\
              {`défaire ce qui était encombré'} {} {`occuper à l'excès un lieu'}\\
          \z
        \ex\label{ex:VilloingDeglas:41b}
            \ea \gll dézantòtiyé $\leftarrow$ antòtiyé\\
              {`détortiller'} {} {`entortiller'}\\
            \ex \gll  dézanbaglé $\leftarrow$ anbaglé\\
              {`débarasser'} {} {`encombrer une table, un meuble'}\\
            \ex \gll  dézanrajé $\leftarrow$ anrajé\\
              {`ne plus être fâché, enragé'} {} {`avoir la rage'}\\
            \ex \gll  dézapiyé $\leftarrow$ apiyé\\
              {`interrompre l'action de s'appuyer'} {} {`appuyer'}\\
            \z
  \z\z


En effet, nous ne relevons aucun verbe parasynthétique\is{parasynthesis} construit sur
base à initiale vocalique. Les seules données qui auraient pu paraître
pertinentes sont les hérités\is{inheritance} \emph{dézosé} `désosser' et \emph{dézèrbé}
`désherber', mais ils sont analysables en créole sur les bases nominales
\emph{zo} `os' et \emph{zèb} `herbe' à initiale consonantique.

\subsubsubsection{Propriétés sémantiques~: sens
privatif}\label{propriétés-sémantiques-sens-privatif}

\largerpage[-1]
Le sens le plus saillant associé à cette formation parasynthétique\is{parasynthesis} est
ce que la littérature sur les créoles appelle couramment le «~sens
privatif~» régulièrement reconnu pour les formations identiques dans
d'autres créoles %
(cf. %
%Chaudenson 1996
\citealt{Chaudenson96}
~: 27~; %
%Filipovitch 1987
\citealt{Filipovich87}
~: 44 ; %
%DeGraff 2001
\citealt{DeGraff2001}
~: 78-80, % 
%Lefebvre 2003
\citealt{Lefebvre2003}
~: 6-8~; %
%Brousseau 2011
\citealt{Brousseau11}
~: 70-71)%
%\citep[cf. ][~: 27~; Filipovitch 1987~: 44 ; DeGraff 2001~: 78-80, Lefebvre 2003~: 6-8~; Brousseau 2011~: 70-71]{Chaudenson96}%
%Chaudenson
%
.
Cette valeur sémantique peut être considérée comme héritée\is{inheritance} du français
où elle est déjà identifiée comme propre au préfixe verbalisateur
\emph{dé}-- sur base nominale %
%(cf. Corbin 1987~: 62-63 et 252, par
%exemple)
\citep[cf. ][62--63 et 252, par exemple]{Corbin87}%
%Corbin
%
. Plus précisément, ce sens privatif s'inscrit dans une relation
spatiale entre le nom de base et le verbe dérivé, relation que les
auteurs francophones représentent au moyen de la terminologie cible/site
de %
%Vandeloise 1986
\citealt{Vandeloise1986} %
%
(qui correspondent aux oppositions figure/ground ou trajector/landmark de la sémantique cognitive, cf.  %
%Fradin 2003
\citealt{Fradin2003}%
~: 298, %
%Amiot 2008
\citealt{Amiot08}%
~: 10, %
%Jalenques 2014
\citealt{Jalenques2014}%
~: 1783)%
%\citep[qui correspondent aux oppositions figure/ground ou trajector/landmark de la sémantique cognitive, cf. ][~: 298, Amiot 2008~: 10, Jalenques 2014~: 1783]{Fradin03}%
%Fradin
%
. La base nominale de la
\isi{préfixation} en \emph{dé}-- du français peut aussi bien dénoter la cible
que le site de la relation.

\begin{enumerate}\def\labelenumi{(\roman{enumi})}

\item
  Lorsque la base dénote le site de la relation, le verbe désigne
  l'action de «~sortir de ce que désigne la base~» %
%(Jalenques 2014~: 1782) %
\citep[1782]{Jalenques2014} %
%
(ce que %
%Corbin 1987
\citealt{Corbin87} %
%
 paraphrase par «~enlever de X~»)
%\citep[ce que ][ paraphrase par «~enlever de X~»]{Corbin87}%
%Corbin
%
~:
  \emph{déterrer, dérailler, débarquer }etc.
\item
  Lorsque la base dénote la cible de la relation, le verbe désigne
  l'action «~d'enlever ce que désigne la base~» (%
%Jalenques 2014
\citealt{Jalenques2014}%
~: 1782)
  %
(ce que %
%Corbin 1987 
\citealt{Corbin87}  %
%
paraphrase par `enlever X')
%\citep[ce que ][ paraphrase par `enlever X']{Corbin87}%
%Corbin
%
~: \emph{désosser,
  déneiger, dépoussiérer, déminer} etc.
\end{enumerate}

Le créole guadeloupéen, en réanalysant les paires
Nom/Verbe-préfixé-en-\emph{dé} héritées\is{inheritance} du français, construit de
façon privilégiée des parasynthétiques \emph{dé}-N-\emph{é}\textsubscript{\textsc{v}}\is{parasynthesis!in \emph{dé}--\emph{é}} dans
lesquels le nom de base (désormais Nbase) dénote la cible de la relation
(\ref{ex:VilloingDeglas:43})~:

\ea \label{ex:VilloingDeglas:43}
      \ea \gll chouk \textrightarrow{~} déchouké\\
        {`souche'} {} {`déraciner'}\\
      \ex \gll  jouk \textrightarrow{~} déjouké\\
        {`joug'} {} {`enlever le joug'}\\
      \ex \gll  pat \textrightarrow{~} dépaté\\
        {`main de banane'} {} {`retirer les mains du régime de banane'}\\
      \ex \gll  pyèt \textrightarrow{~} dépyété\\
        {`pattes'} {} {`retirer les pattes (crabe)'}\\
      \ex \gll  tik \textrightarrow{~} détiké\\
        {`tique'} {} {`retirer les tiques'}\\
\z\z


Comparativement, les parasynthétiques\is{parasynthesis} créoles \emph{dé}-N-\emph{é}\textsubscript{\textsc{v}}\is{parasynthesis!in \emph{dé}--\emph{é}}
dont le N dénote le site de la relation sont très faiblement représentés
dans notre corpus qui ne comprend que les exemples (\ref{ex:VilloingDeglas:44})~:

\ea \label{ex:VilloingDeglas:44}
      \ea \gll bous \textrightarrow{~} débousé\\
        {`bourse'} {} {`dépenser'}\\
      \ex \gll  tab \textrightarrow{~} détablé\\
        {`table'} {} {`enlever les couverts d'une table'}\\
\z\z

Cette tendance est largement confirmée par les triplets N /
N-\emph{é}\textsubscript{\textsc{v}}\is{suffixation!in --\emph{é}}/\emph{dé}-N-\emph{é}\textsubscript{\textsc{v}}\is{parasynthesis!in \emph{dé}--\emph{é}} (hérités\is{inheritance} ou créoles) dont
le schéma de construction n'est pas immédiatement transparent (V \textrightarrow{~}
déV\textsubscript{\textsc{v}} ou N \textrightarrow{~}\emph{dé}-N-\emph{é}\textsubscript{\textsc{v}}\is{parasynthesis!in \emph{dé}--\emph{é}}~?)\footnote{En effet, dans le
  cas des triplets, la difficulté tient à ce que l'on ne parvient pas
  toujours à identifier si le dérivé s'est construit sur le verbe par
  \isi{préfixation} ou sur le N par parasynthèse\is{parasynthesis}~; comme l'a noté %
%Corbin  (1987~:63)
\citet[63]{Corbin87} %
%
 et %
%Amiot (2008~:12)
\citet[12]{Amiot08}%
%Amiot
%
, il existe des «~cas d'ambiguïté
  catégorielle~» dont l'interprétation sémantique est compatible avec
  les deux constructions (par exemple~: \emph{débwasé} `inverse de
  boiser' ou `enlever le bois').} mais dont les \emph{dé}-N-\emph{é}\textsubscript{\textsc{v}}\is{parasynthesis!in \emph{dé}--\emph{é}}
sont compatibles avec une interprétation privative où le nom (N) serait
la cible de la relation (\ref{ex:VilloingDeglas:45})~: là encore, ils sont bien plus
nombreux que ceux dont le nom serait le site de la relation (cf. les
exemples uniques de (\ref{ex:VilloingDeglas:46}))~:

\ea \label{ex:VilloingDeglas:45}
      \ea \gll bach / baché / débaché \\
        {`bache'} {} {`bâcher'} {} {`débâcher'}\\
      \ex \gll  grès / gresé / dégrésé \\
        {`graisse'} {} {`graisser'} {} {`dégraisser, enlever la graisse'}\\
      \ex \gll  kabòs / kabosé / dékabosé \\
        {`bosse'} {} {`déformer'} {} {`débosseler'}\\
      \ex \gll  nat / naté / dénaté \\
        {`natte'} {} {`natter des cheveux'} {} {`enlever les nattes'}\\
      \ex \gll  sèl / salé / désalé \\
        {`sel'} {} {`saler'} {} {`dessaler'}\\
      \z
\ex \label{ex:VilloingDeglas:46}
      \ea \gll kof / kofré / dékofré \\
        {`coffre'} {} {`coffrer'} {} {`décoffrer'}\\
      \ex \gll  kouch / kouché / dékouché \\
        {`lit'} {} {`se coucher'} {} {`découcher'}\\
      \ex \gll  plas / plasé / déplasé \\
        {`place'} {} {`placer'} {} {`déplacer'}\\
      \ex \gll  tè / téré / détéré \\
        {`terre'} {} {`enterrer'} {} {`déterrer'}\\
      \ex \gll  kwen / kwensé / dékwensé \\
        {`coin'} {} {`coincer'} {} {`décoincer'}\\
\z\z


La raison de cette nette préférence tient certainement au fait que les
paires héritées\is{inheritance} du français présentent aussi majoritairement cette
relation sémantique entre le nom et le verbe (\ref{ex:VilloingDeglas:47}) comme l'atteste la
très faible représentation (3 paires uniquement), au sein de notre
corpus, de paires de parasynthétiques\is{parasynthesis} \emph{dé}-N-\emph{é}\textsubscript{\textsc{v}}\is{parasynthesis!in \emph{dé}--\emph{é}} dont le N
désigne le site de la relation (\ref{ex:VilloingDeglas:48})

\ea \label{ex:VilloingDeglas:47}
      \ea \gll fèy \textrightarrow{~} déféyé\\
        {`feuilles'} {} {`ôter les feuilles'}\\
      \ex \gll  fòwm \textrightarrow{~} défòwmé\\
        {`forme'} {} {`déformer'}\\
      \ex \gll  kouraj \textrightarrow{~} dékourajé\\
        {`courage'} {} {`décourager'}\\
      \ex \gll  kras \textrightarrow{~} dékrasé\\
        {`crasse'} {} {`décrasser'}\\
      \ex \gll  mayo \textrightarrow{~} démayoté\\
        {`maillot, étoffe pour emmailloter le nouveau-né'} {} {`démailloter'}\\
      \z
\ex \label{ex:VilloingDeglas:48}
      \ea \gll bò \textrightarrow{~} débòdé\\
        {`bord'} {} {`déborder'}\\
      \ex \gll  bous \textrightarrow{~} débousé\\
        {`bourse'} {} {`dépenser'}\\
      \ex \gll  moul \textrightarrow{~} démoulé\\
        {`moule du gâteau'} {} {`sortir du moule un gâteau'}\\
\z\z

\subsubsubsection{Propriétés sémantiques~: autres sens
minoritaires}\label{propriétés-sémantiques-autres-sens-minoritaires}

Parallèlement, d'autres sens émergent en créole mais en très faible
proportion, reflétant là encore leur faible représentativité dans les
paires et les triplets hérités\is{inheritance} du français~:

\begin{enumerate}\def\labelenumi{(\roman{enumi})}
\item
  le Nbase représente l'objet résultant du procès\is{process}
\end{enumerate}


\ea \label{ex:VilloingDeglas:49}
      \ea \gll chèpi \textrightarrow{~} déchèpiyé\\
        {`charpie'} {} {`mettre en charpie'}\\
      \ex \gll  gout \textrightarrow{~} dégouté\\
        {`goutte'} {} {`couler goutte à goutte'}\\
      \ex \gll  kal \textrightarrow{~} dékalé\\
        {`raclée'} {} {`tabasser'}\\
      \ex \gll  katyé \textrightarrow{~} dékatyé\\
        {`morceau'} {} {`couper en quartier'}\\
\z\z


\begin{enumerate}[resume]\def\labelenumi{(\roman{enumi})}

\item
  le Nbase représente l'objet déplacé lorsque le verbe réfère à une
  localisation ((\ref{ex:VilloingDeglas:50a}) pour les paires créoles, (\ref{ex:VilloingDeglas:50b}) pour les paires
  héritées\is{inheritance} du français)
\end{enumerate}

\ea \label{ex:VilloingDeglas:50}
      \ea\label{ex:VilloingDeglas:50a} 
      \gll kaz \textrightarrow{~} dékazé\\
      `maison' {} {\begin{minipage}[t]{.5\textwidth}\raggedright `déplacer une case à l'aide d'un véhicule
        pour l'installer ailleurs'\end{minipage}}\\
      \ex\label{ex:VilloingDeglas:50b} \gll  ménaj  \textrightarrow{~} déménajé\\
      {\begin{minipage}[t]{.4\textwidth}\raggedright `ensemble des meubles, des objets nécessaires à la vie
        domestique'\end{minipage}} {} `déménager'\\
\z\z

\subsubsection{Dé-V préfixés}\label{subsubsection:VilloingDeglas:dé-v-préfixés}

Ces formations par parasynthèse\is{parasynthesis} doivent être distinguées des
préfixations\is{prefixation} en \emph{dé}- sur base verbale qui (i) soit réfèrent au
procès\is{process} inverse de celui que désigne la base (\ref{ex:VilloingDeglas:51}), (ii) soit ne
déclenchent aucun changement sémantique relativement à la base verbale
(\ref{ex:VilloingDeglas:52}).
\ea \label{ex:VilloingDeglas:51}
      \ea \gll ankayé  \textrightarrow{~} dézankayé\\
  {\begin{minipage}[t]{.4\textwidth}\raggedright `se prendre dans les récifs\\ (pour un hameçon)'\end{minipage}} {} {\begin{minipage}[t]{.3\textwidth}\raggedright `enlever des récifs coraliens'\end{minipage}}\\
      \ex \gll  baké \textrightarrow{~} débaké\\
        {`embarquer'} {} {`débarquer'}\\
      \ex \gll  faché \textrightarrow{~} défaché\\
        {`être faché'} {} {`ne plus être fâché'}\\
      \ex \gll  manché \textrightarrow{~} démanché\\
        {`mettre un manche'} {} {`ôter le manche'}\\
      \ex \gll  rèspèkté \textrightarrow{~} dérèspèkté\\
        {`respecter'} {} {`manquer de respect'}\\
      \z

\ex \label{ex:VilloingDeglas:52}
      \ea chalviré \textrightarrow{~} déchalviré\\
        \glt {`chavirer'}\\
      \ex  chiktayé \textrightarrow{~} déchiktayé\\
        \glt {`émietter, mettre en charpie'}\\
      \ex  libéré \textrightarrow{~} délibéré\\
        \glt {`libérer (qqun de prison)'}\\
        \ex  rifizé \textrightarrow{~} dérifizé\\
         \glt `refuser'
      \ex  viré \textrightarrow{~} déviré\\
        \glt {`tourner en sens inverse'}\\
\z\z

Bien qu'elles présentent \emph{a priori} des segments phonologiques
initiaux et finaux identiques (le préfixe \emph{dé}- et la finale
verbale en --\emph{é}) les préfixations\is{prefixation} sur base verbale se distinguent
des parasynthétiques\is{parasynthesis} par le fait de ne dériver d'aucun nom. Concomitante
à cette différence de construction, se retrouve la relation sémantique
entre la base et le dérivé.

\subsubsubsection{Préfixation \emph{dé}-V à sens
inversif}\label{préfixation-dé-v-uxe0-sens-inversif}

Dans la majorité des cas, la \isi{préfixation} en \emph{dé}-V construit un
sens non pas privatif mais inversif, comme le reconnaissent les travaux
sur les créoles haïtien\il{French-based creole!Haitian Creole} et saint-lucien\il{French-based creole!Saint-Lucia Creole}. Le sens inversif est
différemment appréhendé par les auteurs ayant travaillé sur le français.
Si l'on s'en tient aux travaux les plus récents, par exemple de %
%Jalenques (2014: 1778)
\citet[1778]{Jalenques2014} %
%
qui suit la description proposée par %
%Gerhard-Krait~(2000)
\citet{Gerhard-Krait00}%
%
, les verbes préfixés\is{prefixation} par \emph{dé}-- et construits
sur base verbale présentent trois acceptions~:

\begin{enumerate}
\item[a)] inversion du résultat du procès\is{process} exprimé par la base verbale (en lien
à ses compléments éventuels)~: \emph{dénouer sa cravate }= agir de telle
sorte qu'on annule le résultat de « nouer la cravate »~;

\item[b)] l'inverse du procès\is{process} (non résultatif) exprimé par la base~:
\emph{décroître }= l'inverse de croître~;

\item[c)] la négation du procès\is{process} (non résultatif) exprimé par la base~:
\emph{déplaire }= ne pas plaire.
\end{enumerate}

Les paires Verbe / \emph{dé}-V\textsubscript{\textsc{v}} héritées\is{inheritance} du français par le
créole sont très largement majoritairement du type a) ou b) (\ref{ex:VilloingDeglas:53}).


  \ea \label{ex:VilloingDeglas:53}
        \ea \gll débatizé / batizé\\
          {`débaptiser'} {} {`baptiser'}\\
        \ex \gll  déchosé / chosé\\
          {`déchausser'} {} {`chausser'}\\
        \ex \gll  dégonflé / gonflé\\
          {`dégonfler'} {} {`gonfler'}\\
        \ex \gll  démayé / mayé\\
          {`démarier'} {} {`marier'}\\
        \ex \gll  dézabiyé  / abiyé\\
          déshabiller {} {`s'habiller'}\\
        \ex \gll  dézankastré / ankastré\\
          {`défaire ce qui était encastré'} {} {`encastrer'}\\
        \ex \gll  dézantòtiyé / antòtiyé\\
          {`détortiller'} {} {`entortiller'}\\
  \z\z


Les paires créoles sont aussi largement de type a)~:

\ea \label{ex:VilloingDeglas:54}
      \ea \gll dégaré  / garé\\
  {`sortir de la place de garage, de stationnement'} {} {`garer, stationner'}\\
      \ex \gll  dékouvè / kouvè\\
        {`découvrir'} {} {`couvrir'}\\
      \ex \gll  dékòviyé / kòviyé\\
        {`détordre, remettre en position initiale'} {} {`tordre'}\\
      \ex \gll  dépayé / payé\\
        {`annuler un pari'} {} {`parier'}\\
      \ex \gll  dézanbaglé / anbaglé\\
        {`débarrasser'} {} {`encombrer une table, un meuble'}\\
\z\z


On ne recense dans le corpus qu'un exemple de type c) cf. (\ref{ex:VilloingDeglas:55})

\ea \label{ex:VilloingDeglas:55}
 \gll dérèspèkté / rèspekté\\
 {`manquer de respect, y compris sexuellement'} {}   `respecter'\\
\z

Les données nous conduisent donc à envisager que le créole, ayant hérité\is{inheritance}
des paires V/ \emph{dé}-V\textsubscript{\textsc{v}}\is{parasynthesis!in \emph{dé}--\emph{é}} les plus disponibles du français
--celles à valeur inversive--, a formé sur ces paires, par analogie, les
dérivés créoles. Le sens inversif est donc probablement hérité\is{inheritance} de la
\isi{préfixation} en \emph{dé}-- du français. Néanmoins, cette valeur inversive
reste cantonnée aux préfixés\is{prefixation} sur base verbale et n'est représentée dans
aucun exemple de parasynthétiques\is{parasynthesis} en \emph{dé}-N-\emph{é}\textsubscript{\textsc{v}}. Ainsi, les
deux schémas morphologiques semblent s'être spécialisés sémantiquement
en créole~:

\begin{itemize}

\item[---]
  le sens privatif est réservé à la parasynthèse\is{parasynthesis} \emph{dé}-N-\emph{é}\textsubscript{\textsc{v}}\is{parasynthesis!in \emph{dé}--\emph{é}}
  (même si d'autres valeurs sémantiques sont possibles)~;
\item[---]
  le sens inversif est spécifique à la \isi{préfixation} \emph{dé}-V.
\end{itemize}

Cette spécialisation sémantique pourrait permettre de trancher l'analyse
des triplets N / V / \emph{dé}-N-\emph{é}\textsubscript{\textsc{v}}\is{parasynthesis!in \emph{dé}--\emph{é}} qui apparaissent en bien
plus grand nombre dans notre corpus que les parasynthétiques\is{parasynthesis}
\emph{dé}-N-\emph{é}\textsubscript{\textsc{v}}\is{parasynthesis!in \emph{dé}--\emph{é}} et les préfixés\is{prefixation} \emph{dé}-V\textsubscript{\textsc{v}}, tant
pour ceux hérités\is{inheritance} du français (\ref{ex:VilloingDeglas:56}) que ceux construits en créole (\ref{ex:VilloingDeglas:57}).

\ea \label{ex:VilloingDeglas:56}
        \ea \gll apui / apiyé  / dézapiyé\\
          {`appui'} {} {`s'appuyer} {}  {`ne pas s'appuyer'}\\
        \ex \gll  bwa / bwazé / débwazé \\
          {`bois'} {} {`boiser'} {} {`déboiser'}\\
        \ex \gll  klou / klouwé / déklouwé \\
          {`clou'} {} {`clouer'} {} {`enlever les clous'}\\
        \ex \gll  pengn / pengné / dépengné \\
          {`peigne'} {} {`peigner'} {} {`dépeigner'}\\
        \ex \gll  tach / taché / détaché \\
          {`tache'} {} {`tacher'} {} {`détacher'}\\
        \z
  \ex \label{ex:VilloingDeglas:57}
        \ea \gll bonda / bondaté / débondaté \\
          {`fesses'} {} {`s'asseoir'} {} {`se lever'}\\
        \ex \gll  bwa / bwaré / débwaré \\
          {`bras'} {} {`enlacer'} {} {`désenlacer'}\\
        \ex \gll  grij / griji / dégriji \\
          {`fronce'} {} {`faire des fronces'} {} {`retirer les fronces'}\\
        \ex \gll  lyann / lyanné / délyanné \\
          {`union'} {} {`s'unir'} {} {`se désunir'}\\
        \ex \gll  janm / janbé / déjanbé \\
          {`jambe'} {} {`enjamber'} {} {`procès inverse d'enjamber'}\\
        \z
  \z

\subsubsubsection{Préfixation \emph{dé}-V sans changement
sémantique}\label{préfixation-dé-v-sans-changement-sémantique}

Les formations par parasynthèse\is{parasynthesis} \emph{dé}-N-\emph{é}\textsubscript{\textsc{v}}\is{parasynthesis!in \emph{dé}--\emph{é}} doivent,
également, être distinguées des préfixations\is{prefixation} en \emph{dé}- sur base
verbale (\emph{dé}-V\textsubscript{\textsc{v}}) qui, à la différence des précédentes ne s'accompagnent
d'aucun changement sémantique (cf. en (\ref{ex:VilloingDeglas:58a}) les paires
V/\emph{dé}-V\textsubscript{\textsc{v}} héritées\is{inheritance} du français et en (\ref{ex:VilloingDeglas:58b}) celles
construites en créoles)~:

\ea \label{ex:VilloingDeglas:58a}
      \ea \gll partajé \textrightarrow{~} départajé\\
        {`partager'}\\
      \ex \gll  plimé \textrightarrow{~} déplimé\\
        {`plumer'}\\
      \ex \gll  tranpé \textrightarrow{~} détranpé\\
        {`tremper'}\\
      \ex \gll  vidé \textrightarrow{~} dévidé\\
        {`vider'}\\
      \ex   pozé \textrightarrow{~} dépozé\\
        {`déposer, remettre à sa place'}\\
      \z
\ex\label{ex:VilloingDeglas:58b}
      \ea   bwété \textrightarrow{~} débwété\\
        {`boîter, marcher en boitant'}\\
      \ex   chiktayé \textrightarrow{~} déchiktayé\\
        {`émietter, mettre en charpie'}\\
      \ex \gll  rifizé \textrightarrow{~} dérifizé\\
        {`refuser'}\\
      \ex \gll  sòti \textrightarrow{~} désòti\\
        {`sortir'}\\
      \ex   viré \textrightarrow{~} déviré\\
        {`tourner en sens inverse'}\\
\z\z


Cette absence de variation sémantique associée à la \isi{préfixation} n'a rien
de particulier au créole puisqu'elle est observée en français %
%(Muller
%1990, Gerhard 2000, Apothéloz 2007, Jalenques 2014) 
\citep{Muller90,Gerhard-Krait00,Apotheloz2007,Jalenques2014} %
%Muller
%
(\ref{ex:VilloingDeglas:59}) et dans
d'autres créoles à base française comme le haïtien\il{French-based creole!Haitian Creole} %
%(Filipovich, 1987,
%Lefebvre 2003, Valdman et al. 1981) 
\citep{Filipovich87,Lefebvre2003,Valdman1981} %
%Lefebvre;Valdman-al.
%
(\ref{ex:VilloingDeglas:60}) ou le saint-lucien %
%(Broussseau 2011~: 74)
\citep[74]{Brousseau11}
.

\ea \label{ex:VilloingDeglas:59}
      \ea couper \textrightarrow{~} découper
      \ex  doubler \textrightarrow{~} dédoubler
      \ex  marquer \textrightarrow{~} démarquer
      \ex passer \textrightarrow{~} dépasser
      \ex verser \textrightarrow{~} déverser
      \z
\ex \label{ex:VilloingDeglas:60}
      \ea chiré \textrightarrow{~} déchiré \\
      \glt `déchirer' \\
      \ex chifonnen \textrightarrow{~} déchifonnen \\
      \glt `froisser' \\
      \ex gengole \textrightarrow{~} dégengole \\
      \glt `se précipiter' \\
      \ex grennen \textrightarrow{~} dégrennen \\
      \glt `égrener' \\
      \z
\z

Une analyse souvent évoquée, tant pour le français que pour le créole,
est l'éventualité d'une valeur intensive du préfixé\is{prefixation} en \emph{dé--}
relativement au verbe de base. Bien que cette valeur soit justifiée
ponctuellement, elle ne peut tenir pour l'ensemble des cas (voir
critique de %
%Jalenques (2014~: 1779)
\citet[1779]{Jalenques2014} %
%
 pour le français et de %
%DeGraff (2001~: 80)
\citet{DeGraff2001} %
%
 pour le créole). Quoiqu'il en soit, cette propriété ne touche
pas les parasynthétiques \emph{dé}-N-\emph{é}\textsubscript{\textsc{v}}\is{parasynthesis!in \emph{dé}--\emph{é}}.

\section{Conclusion }\label{section:VilloingDeglas:conclusion}
\largerpage
Le développement, en créole guadeloupéen, de deux schémas morphologiques
de formation de verbes par affixation (la suffixation verbale dénominale\is{derivation!noun-based verb}
en --\emph{é}\is{suffixation!in --\emph{é}} (N-\emph{é}\textsubscript{\textsc{v}}\is{suffixation!in --\emph{é}}) et la parasynthèse\is{parasynthesis} verbale
dénominale \emph{dé}-N-\emph{é}\textsubscript{\textsc{v}}\is{parasynthesis!in \emph{dé}--\emph{é}}) ), est issu de la réanalyse de
paires Nom / Verbe héritées\is{inheritance} du français. Les conditions nécessaires à
ces réanalyses s'ancrent crucialement dans la propriété des lexèmes
guadeloupéens de ne se réaliser que sous une forme unique. En effet, la
majorité des verbes hérités\is{inheritance} du français présentent un --\emph{é }final
probablement issu des formes fléchies de l'infinitif ou du participe
passé du verbe français d'origine. Or, c'est ce --\emph{é}\is{suffixation!in --\emph{é}} final, qui,
dans le contexte des paires Nom/Verbe où il apparaît, est réanalysé
comme un suffixe dérivationnel, faisant ainsi émerger deux nouveaux
schémas morphologiques en créole, inexistants en français. En somme,
l'application de la notion de lexème à l'analyse des données créoles
permet de reconnaîre la validité de ces schémas morphologiques en
guadeloupéen alors qu'elle avait conduit à remettre en cause la
pertinence de ces mêmes schémas pour les données correspondantes en
français.

Ces deux exemples de réanalyse nous conduisent à réfuter la position qui
soutient que la dérivation n'émerge que \emph{via} une
grammaticalisation graduelle (cf. par exemple %
%Mc Whorter 1998
\citealt{McWhorter1998}%
). Les
données du créole guadeloupéen que nous avons examinées nous incitent
plutôt à suivre la proposition de %
%Rainer (2015) 
\citet{Rainer15} %
%Rainer
%
selon lequel la
grammaticalisation n'est qu'un des mécanismes du changement
morphologique parmi d'autres, la réanalyse en étant un autre.

Le mécanisme de la réanalyse, qui n'est pourtant pas propre aux langues
créoles, y prend néanmoins une place importante du fait de la part
massive qu'y occupe le lexique hérité\is{inheritance} du français. En témoignent
d'autres schémas morphologiques tels que la suffixation en
--\emph{asyon} en guadeloupéen (\emph{anmerdasyon} `tracas' ' ← \emph{anmerdé} `emmerder' ; \emph{pwofitasyon} `action d'abuser de la
faiblesse de qqun' ← \emph{pwofité} `profiter de la faiblesse de
l'autre'), dont la forme phonologique du suffixe est le résultat de
l'amalgame de la finale du radical\is{stem} du verbe de base et du suffixe
--\emph{ion} des verbes hérités\is{inheritance} du français (\emph{admirasyon}
`admiration' / \emph{admiré} `admirer'; \emph{ògmantasyon}
`augmentation' / \emph{ògmanté} `organiser') %
%(cf. Villoing \& Deglas (à paraître))
\citep{Villoing17}%
%
.

\nocite{Amiot08}
\nocite{Anderson92}
\nocite{Apotheloz2002}
\nocite{Apotheloz2007}
\nocite{Aronoff94}
\nocite{Bechade92}
\nocite{Bhatt00}
\nocite{Bernabe87}
\nocite{booij1977.dutch-morphology}
\nocite{Booij10}
\nocite{Brousseau11}
\nocite{Chaudenson96}
\nocite{Chevalier64}
\nocite{Corbin87}
\nocite{corbin04}
\nocite{Croft91}
\nocite{Daladier99}
\nocite{darmesteter1894.traite-formation}
\nocite{DeGraff2001}
\nocite{Dell70}
\nocite{dubois62}
\nocite{Filipovich87}
\nocite{Fradin03}
\nocite{Fradin2009}
\nocite{Gardes-Tamine1988}
\nocite{Gerhard-Krait00}
\nocite{Gerhard-Krait12}
\nocite{Germain76}
\nocite{Grevisse1988}
\nocite{Guilbert1975}
\nocite{Haspelmath2002}
\nocite{Haspelmath1995}
\nocite{Huot06}
\nocite{kerleroux96}
\nocite{Kerleroux2000}
\nocite{Jespersen22}
\nocite{Langacker77}
\nocite{Lass90}
\nocite{Lefebvre98}
\nocite{Lefebvre2003}
\nocite{Lignon14}
\nocite{Ludwig12}
\nocite{Lyons77}
\nocite{McWhorter1998}
\nocite{Matthews91}
\nocite{Meyer-Lubke}
\nocite{Melcuk1996}
\nocite{Mufwene89}
\nocite{Muller90}
\nocite{Namer09}
\nocite{Namer12}
\nocite{namer2013}
\nocite{Nyrop36}
\nocite{Plag1999}
\nocite{Plag03}
\nocite{Poullet84}
\nocite{Rainer15}
\nocite{Scalise1994}
\nocite{Seuren86}
\nocite{Thibault12}
\nocite{Tourneux90}
\nocite{Tribout2010a}
\nocite{Tribout2014}
\nocite{Valdman78}
\nocite{Valdman1981}
\nocite{Vandeloise1986}
\nocite{Villoing16}
\nocite{Villoing17}
\nocite{Zribi-Hertz72}

\il{French-based creole|)}
\il{French-based creole!Guadeloupe Creole|)}
\is{lexeme|)}
\is{reanalysis|)}


{\sloppy
\printbibliography[heading=subbibliography,notkeyword=this]
}


\end{document}
