\documentclass[output=paper]{langsci/langscibook}
\ChapterDOI{10.5281/zenodo.1407013}
 
\title{Reduplication across boundaries: The case of Mandarin}

\author{Chiara Melloni\affiliation{University of Verona} \lastand Bianca Basciano\affiliation{Ca' Foscari University of Venice}}

\abstract{In this chapter, we shed new light on the reduplicative processes of Mandarin Chinese and assess the structural and interpretive properties of the input/base and output of these word formation phenomena. In particular, we focus on the categorial status of the base and address the issue of whether reduplication applies to category-free roots\is{root} or full-fledged lexemes\is{lexeme}. Empirically, the privileged domain of research is \emph{increasing}\is{reduplication!increasing (iconic)} reduplication of disyllabic bases, or, as we dub it in the chapter, the AABB pattern, which is compared with \emph{diminishing}\is{reduplication!diminishing (countericonic)} reduplication, expressed by the template ABAB. The comparison between the two phenomena allows us to show that increasing\is{reduplication!increasing (iconic)} and diminishing\is{reduplication!diminishing (countericonic)} reduplication differ in the nature of the input units involved. On the grounds of a wide-ranging class of data, we argue that Mandarin reduplication takes base units of different `size': word/\isi{lexeme}-like units provided with category, namely verbs\is{verb} in the case of diminishing\is{reduplication!diminishing (countericonic)} reduplication, and categoryless roots\is{root} in the case of increasing\is{reduplication!increasing (iconic)} reduplication. Throughout the chapter, we explore some category neutral properties of increasing\is{reduplication!increasing (iconic)} reduplication and propose a unitary semantic operation capable to derive the various interpretive nuances of this phenomenon across lexical categories.}

\maketitle

\begin{document}
\selectlanguage{english}
\is{reduplication|(}
\il{Chinese!Mandarin Chinese|(}

\section{Introduction}

\subsection{Lexemes\is{lexeme} vs. words and reduplication phenomena}

Lexemes\is{lexeme} are usually understood as sound/meaning pairs, i.e. linguistic
signs provided with lexical category specification yet lacking inherent
inflectional specification. Lexemes\is{lexeme} and words are thus considered as
distinct entities in lexicalist approaches to word formation. As a
matter of fact, while a word proper is a fully inflected entity
functioning as a syntactic atom, a \isi{lexeme} is the abstract version of the
word-form lacking inflectional marking %
%(Fradin \& Kerleroux 2003)
\citep{Fradin03b}%
%Fradin-Kerleroux
%
. As
put forward by %
%Fradin \& Kerleroux (2003)
\citet{Fradin03b}%
%Fradin-Kerleroux
%
, the form of the \isi{lexeme} can
either be segmentally simple (\emph{viz}. a root\is{root}) or complex
(\emph{viz}. a \isi{stem}), with affixal derivation, \isi{compounding} and
reduplication as phenomena possibly involved in \isi{lexeme} formation.

Reduplication phenomena, however, are particularly challenging under
this approach, since cross-linguistically the functions of reduplication
are very varied and difficult to place categorically within the
\emph{derivational} domain of lexemes\is{lexeme}. In fact, whereas derivation
typically forms new lexemes\is{lexeme} and can be category changing, reduplication
often conveys values typically found in the \emph{inflectional} domain.
Although reduplication is attested with a variety of meanings (and
forms) across languages, this phenomenon is consistently associated with
its prototypical (iconic) function of intensification\is{aspect!intensification}. In its increasing\is{reduplication!increasing (iconic)}
value, reduplication in the nominal domain gives as a result \isi{plural}
nouns\is{noun}, and in the domain of verbs\is{verb} it usually conveys aspectual\is{aspect} meanings,
i.e. pluractionality\is{aspect!pluractionality}, iterative\is{aspect!iterative} or progressive\is{aspect!progressive} \isi{aspect}, which are
features prototypically expressed by inflection markings in most
Indo-European languages. With adjectives\is{adjective}, the prototypical value is
intensification\is{aspect!intensification} of the property/quality expressed by the base adjective\is{adjective}.
Nevertheless, independently of its semantic values, reduplication
manifests several properties of word/\isi{lexeme} formation and, formally,
approaches derivational phenomena. First of all, (full) reduplication
consists in the iteration of simple or complex roots\is{root} (\emph{viz}.
stems\is{stem}), since it may also involve complex objects, such as compounds\is{compounding}.
Crucially, however, it typically applies to uninflected bases, with
inflectional marking, if any, applying outside of/after reduplication.
Moreover, reduplication shows many properties of \isi{compounding}, since it
often induces a reanalysis of the stress or tonal pattern of its base,
or the insertion of epenthetic material between the two iterating units
and/or some other phonological readjustment. Further, semantic drift and
idiosyncrasy can characterize the outputs of reduplicative processes,
while inflection phenomena are very transparent at the interpretive
level %
(see %
%Forza 2011
\citealt{Forza2011}%
%
, for an enlightening typological perspective)
%\citep[see][, for an enlightening typological perspective]{Forza2011}%
%Forza
%
.

Therefore, under the \isi{lexeme}/word distinction approach, we could argue
that reduplication applies to roots\is{root} or stems\is{stem} (traditionally understood
as the phonological form of lexemes\is{lexeme}) and its domain of application is
below the level of the word, or below X° in the standard X-bar approach.

\subsection{Words, lexemes\is{lexeme}, and roots\is{root}/stems\is{stem} in Mandarin Chinese}

If the concept of \isi{lexeme} appears empirically motivated in fusional or
agglutinating languages whereby inflection markers modify the word form
conveying relevant features in the syntactic contexts, its motivation is
less grounded in isolating languages, where (concrete) words occur with
none or a very low number of inflection markers, typically show
invariable form and are virtually indistinguishable from the
corresponding (abstract) lexemes\is{lexeme}. Mandarin Chinese is one of those
languages where words have little or no inflection and where lexemes\is{lexeme},
expressing the abstract representation of a word, cannot be
distinguished from word forms on a formal basis.

In Mandarin, the crucial distinction at the morphological level lies in
the bound or free status of the root\is{root} (a lexical morpheme), i.e. whether
the root\is{root} can `stand alone' and occupy a syntactic slot (\ref{ex:BascianoMelloni:1}), equating
thus free standing words in fusional languages, or whether it must be
formally conjoined with another bound or free root\is{root}, or with a
derivational affix, to form an autonomous \isi{lexeme}/word (\ref{ex:BascianoMelloni:2}).

\ea\label{ex:BascianoMelloni:1} free roots\is{root}: 貓 \emph{māo} `cat', 走 \emph{zǒu} `walk, run away'

  % \ea \gll 貓 \\ \emph{māo} \\ \glt `cat'
  % \ex \gll 走 \\ \emph{zǒu} \\ \glt `walk, run away'
  % \z

\ex\label{ex:BascianoMelloni:2} bound roots\is{root}: 衣  \emph{yī}  `clothing, clothes' , 毆 \emph{ōu}  `beat'

  % \ea \gll 衣 \\  \emph{yī}\\ \glt  `clothing, clothes' ,
  % \ex \gll 毆 \\ \emph{ōu} \\ \glt `beat'
  % \z

\z

While the roots\is{root} in (\ref{ex:BascianoMelloni:1}) can be used by themselves in a sentence, those in
(\ref{ex:BascianoMelloni:2}) cannot stand alone but occur in complex words like e.g. 大衣
\emph{dà-yī} `big-clothes, overcoat, topcoat', 雨衣\emph{yǔ-yī}
`rain-clothes, raincoat', 衣櫃 \emph{yī-guì} `clothes-cupboard,
wardrobe', 衣鉤\emph{yī-gōu} `clothes-hook, clothes hook' %
%(Arcodia \&
%Basciano 2017: 105-106)
\citep[105-106]{ArcodiaBasciano2017}%
%Arcodia-Basciano
%
. Due to a strong tendency towards
disyllabification attested in the evolution of the Chinese language over
the centuries %
%(see Shi 2002: 70-72)
\citep[see][70-72]{Shi2002}%
%Shi
%
, most roots\is{root} are nowadays bound in
Standard Mandarin (about 70\% according to %
%Packard 2000
\citealt{Packard2000a}%
%
). Therefore, the
majority of words or lexemes\is{lexeme} are compounds\is{compounding} or other types of
morphologically complex forms, typically ranging over all major lexical
categories.

Another crucial aspect of Chinese morphology lies in the absence of
strictly morphological criteria for the identification of the lexical
category of roots\is{root} (or stems\is{stem}, if morphologically complex), with some
exceptions.\footnote{Examples are words containing suffixes such as 子
  \emph{-zi}, e.g. 刷子 \emph{shuāzi} `brush' (cf. 刷 \emph{shuā} `to
  brush'), and 頭 \emph{-tou}, e.g. 想頭 \emph{xiǎngtou} `idea' (cf. 想
  \emph{xiǎng} `to think'), which are always nouns\is{noun} %
%(see Basciano 2017)
\citep[see][]{Basciano2017}%
%Basciano
%
.}
As a matter of fact, no category-specific morphology (such as
declension/conjugation class markers in fusional languages) can be
deployed to partition roots\is{root} into lexical classes, with a \isi{verb} like 走
\emph{zǒu} `walk, run away' being virtually indistinguishable at the
morphological level from a noun\is{noun} like 書 \emph{shū} `book' (see %
%Basciano 2017
\citealt{Basciano2017}%
%
). Since there are no reliable morphological criteria to identify
\emph{lexeme}s\is{lexeme} as roots\is{root} (or stems\is{stem}) endowed with lexical category
features, the only reliable criterion is the distributional one. For
instance, syntactic distribution only can discriminate among the
adjectival, verbal or nominal use of a \isi{stem} (namely, a combination of
two roots\is{root}) like 麻煩 \emph{máfan} `annoying, bother, trouble' %
%(examples
%below from Basciano 2017: 561-562)
\citep[examples below from][561--562]{Basciano2017}%
%Basciano
%
:

\ea\label{ex:BascianoMelloni:3}

  \ea\label{ex:BascianoMelloni:3a}  這件事很麻煩。\\
  \gll \emph{zhè} \emph{jiàn} \emph{shì} \emph{hěn} \emph{máfan}\\
  this \textsc{clf} fact very troublesome\\
  \glt `This fact is troublesome.'

  \ex\label{ex:BascianoMelloni:3b} 他不願麻煩别人 。\\
  \gll \emph{tā} \emph{bù-yuàn} \emph{máfan} \emph{biérén}\\
  \textsc{3sg.m} not-willing trouble others\\
  \glt `He is unwilling to trouble other people.'

  \ex\label{ex:BascianoMelloni:3c} 你們在路上會遇到一些麻煩。\\
  \gll \emph{nǐ-men} \emph{zài} \emph{lù-shang} \emph{huì} \emph{yùdào} \emph{yīxiē} \emph{máfan}\\
  \textsc{2sg-pl} in street-on may/will meet some trouble\\
  \glt `You may/will run into some troubles on the road.'
  \z
\z

Thus, under the standard approach to lexemes\is{lexeme} proposed in 1.1, a relevant
issue concerns the very existence of these units in the Chinese language
where, at the lexical level, the very flexible distribution of lexical
items seems to point in the direction of a lexicon whose base units
(roots\is{root}/stems\is{stem}) lack inherent category features. Moreover, the examples in
(\ref{ex:BascianoMelloni:3}) shed light on the need for a very loose semantics of roots\is{root}/stems\is{stem},
arguably incompatible with the specific semantic meaning of lexemes\is{lexeme}, as
proposed in %
%Fradin \& Kerleroux (2003)
\citet{Fradin03b}%
%Fradin-Kerleroux
%
. Under the hypothesis that roots\is{root}
bear no category specification, their meaning should be `vague' enough
to make it compatible with the adjectival, verbal or nominal meanings
that might be instantiated in the syntax.\footnote{In syntactic
  approaches to word formation such as Distributed Morphology, the
  meaning of a word emerges constructionally once the \isi{root} has been
  categorized by a selecting head (\emph{n, v} or \emph{a}) in the
  course of syntactic derivation, and cannot be determined lexically.}
We may remark, however, that the great flexibility observed in previous
stages of the language has been largely reduced over the centuries,
first with a functional specialization of lexemes\is{lexeme} during the Han period
(206 BCE-220 CE), and then with the proliferation of \is{compounding}compound words,
whose functional preference has been always much more rigid and stable
%
%(see Zádrapa 2017)
\citep[see][]{Zadrapa2017}%
%Zádrapa
%
. Even though cases of `regular ambiguity' like the
one in (\ref{ex:BascianoMelloni:3}) are found, in Modern Chinese\il{Chinese!Modern Chinese} lexemes\is{lexeme} tend to be more fixed as
far as lexical category and distribution are concerned; many roots\is{root} have
a `prototypical' distribution and cannot be easily coerced into other
lexical categories. However, even very stable words may be occasionally
placed in syntactic slots usually occupied by other word classes,
creating ``innovative ambiguities'' %
(%
%Kwong \& Tsou 2003
\citealt{KwongTsou2003}%
%
: 116; see also
%Basciano 2017
\citealt{Basciano2017}%
%
)
%\citep[116; see also Basciano 2017]{KwongTsou2003}%
%Kwong-Tsou
%
. As observed by %
%Zádrapa (2017)
\citet{Zadrapa2017}%
%Zádrapa
%
, although it is not
possible to distinguish on a formal basis the prototypical from the
non-prototypical use, it is still possible to perceive a functional
``strain'' (or ``pragmatic markedness'' in %
%Bisang's 2008
\citeauthor{Bisang2008a}'s \citeyear{Bisang2008a} %
 terms), which
always results in a semantic shift of varying dimension %
%(see Croft 2001:
%73)
\citep[see][73]{Croft2001}%
%Croft
%
.

\subsection{Reduplication phenomena in Mandarin Chinese}

Among word formation phenomena in Mandarin, reduplication is one of the
most productive and, as we will see throughout this chapter, it is found
across all major lexical categories with both increasing (iconic\is{reduplication!increasing (iconic)}) and
diminishing (countericonic\is{reduplication!diminishing (countericonic)}) values. Whereas there is no perfect
correspondence between lexical categories and reduplication functions
(verbs\is{verb}, for instance, can be reduplicated along one or the other
function), we will see there is instead a tight correspondence between
the structural pattern of reduplication and its 
diminishing\is{reduplication!diminishing (countericonic)} or increasing\is{reduplication!increasing (iconic)} value, so that the two patterns are rigidly
differentiated at the segmental and suprasegmental level.

In recent years there has been a growing attention to reduplication in
Sinitic. In this chapter, we will try to shed new light on the
reduplicative processes of Mandarin, and try to assess the structural
and interpretive properties of the input (the bases of reduplication)
and the output of reduplicative processes. In particular, we will focus
on the question of the categorial status of the base of the
reduplicative processes in Mandarin, i.e. what the base units are and,
specifically, whether reduplication applies to category-less roots\is{root} or to
full-fledged lexemes\is{lexeme}/words. Empirically, the privileged domain of
research will be the increasing\is{reduplication!increasing (iconic)} reduplication of disyllabic bases, or,
as we dub it here, the AABB pattern, which will be compared with the
diminishing\is{reduplication!diminishing (countericonic)} pattern, characterized by the disyllabic template ABAB.

The comparison between the two patterns will allow us to show that they
differ in the type of units that constitute the basis of the
reduplicative process. Mandarin reduplication, indeed, involves base
units of different `size', ranging from word/\isi{lexeme}-like units provided
with category and, namely, involving the verbal domain in the case of
diminishing\is{reduplication!diminishing (countericonic)} reduplication, to category-less roots\is{root} in the case of
increasing\is{reduplication!increasing (iconic)} reduplication. Throughout the chapter, we will provide
evidence for the latter claim, i.e. that reduplication phenomena involve
roots\is{root}, and we will explore some category neutral properties of
increasing\is{reduplication!increasing (iconic)} reduplication. We will conclude with some remarks on the
semantic effects of this phenomenon, which we interpret as an
\emph{increased measure} function modifying the sortal type conveyed by
the (combination of) roots\is{root}.

\subsection{Outline of the chapter}

The chapter is organized as follows. Section 2 is dedicated to the
presentation of the main patterns of full reduplication in Mandarin
Chinese. Section 3 explores the characterizing features of increasing\is{reduplication!increasing (iconic)}
reduplication (AABB pattern) in some detail and discusses its formal and
interpretive properties across lexical categories. Section 4 contains
the structural analysis and some hypotheses about the semantics of AABB
increasing\is{reduplication!increasing (iconic)} reduplication, and section 5 draws the conclusions.

\section{Data description}

\subsection{Reduplication in Mandarin: An overview}

Reduplication in Mandarin Chinese is a widespread and productive
phenomenon, virtually affecting all major lexical categories (V, Adj, N)
and showing a tight relation between structural patterns (form) and
semantic meanings (function). Semantically, Mandarin reduplications have
augmentative/increasing\is{reduplication!increasing (iconic)} and diminishing\is{reduplication!diminishing (countericonic)} functions that are rigidly
associated with different structural and/or suprasegmental patterns.

The diminishing\is{reduplication!diminishing (countericonic)} function is only found in the verbal domain.
Reduplicated verbs\is{verb} typically convey `delimitative' or `tentative' \isi{aspect}
%
%(Chao 1968, Li \& Thompson 1981, Tsao 2004)
\citep{Chao1968,Li1981,Tsao2001}% Yuen => Chao, Tsao 2004 => 2001
%Li-Thompson;?Tsao
%
, meaning to do something ``a
little bit/for a while'' %
%(Li \& Thompson 1981: 29) 
\citep[29]{Li1981} %
%Li-Thompson
%
or, by extension, to
do something quickly, lightly, casually or just for a try.\footnote{Further,
  it has the pragmatic function of marking a relaxed tone, casualness
  %
%(Ding 2010)
\citep{Ding2010}%
%Ding
%
, and thus reduplicated verbs\is{verb} are also used as mild
  imperatives %
%(see Xiao \& McEnery 2004)
\citep[see][]{XiaoMcEnery2004}%
%Xiao-McEnery
%
.} Both monosyllabic (A → AA)
and disyllabic (AB → ABAB) bases can reduplicate, but only in the case
of monosyllabic reduplication the morpheme 一\emph{yi}
(\textless{}\emph{yī}) `one' can occur between the base and the
reduplicant:


\vspace{\topsep}\ea\label{ex:BascianoMelloni:4}
  \ea\label{ex:BascianoMelloni:4a}
   \begin{tabular}[t]{ll@{\hspace*{.1\linewidth}}l}
  {教 (A)} & $\rightarrow$ & {教(一)教 (AA)}\\
  \begin{minipage}[t]{.3\linewidth}
  \gll \emph{jiāo}\\ teach\\ \glt `teach'
  \end{minipage}
  & &
    \begin{minipage}[t]{.4\linewidth}
  \gll \emph{jiāo}  \emph{(yi)} \emph{jiao}\\ 
       teach one teach\\ 
  \glt `teach a little'
  \end{minipage}\\
  \end{tabular}
  \ex\label{ex:BascianoMelloni:4b}
  \begin{tabular}[t]{ll@{\hspace*{.1\linewidth}}l}  
  {休息 (AB)} & $\rightarrow$ & {休息休息 (ABAB)}\\
  \begin{minipage}[t]{.3\linewidth}
  \gll \emph{xiūxi}\\ rest\\ \glt `rest'
  \end{minipage}
  & &
    \begin{minipage}[t]{.4\linewidth}
  \gll \emph{xiūxi}  \emph{xiūxi}\\ 
       rest rest\\ 
  \glt `rest a little/for a while'
  \end{minipage}\\
  \end{tabular}
  \z
\z

It has been argued that this reduplicative process is a syntactic
phenomenon involving units in the \emph{v}P domain %
%(see Arcodia,
%Basciano \& Melloni 2014; Basciano and Melloni 2017)
\citep[see][]{ArcodiaBascianoEtAl2014,BascianoMelloni2017}%
%Arcodia-Basciano-Melloni;?Basciano-Melloni
%
. First of all, the
reduplicated complex is not a syntactic atom, since it is possible to
have intervening morphemes between the base and the reduplicant: beyond
the numeral 一\emph{yi} (\textless{}\emph{yī}) `one' mentioned above,
the perfective \isi{aspect} marker 了\emph{le}\footnote{Note that the
  perfective marker了\emph{le} is generally placed after the second \isi{verb}
  in resultatives and other kinds of \is{compounding}compound verbs\is{compounding!compound verb}: 喝醉了
  \emph{hē-zuì-le} `drink-drunk-PFV' \emph{vs.} *喝了醉 \emph{hē-le zuì}
  `drink-PFV drunk'.} can intervene between the base and the
reduplicant, as in (\ref{ex:BascianoMelloni:5}):

\ea\label{ex:BascianoMelloni:5}

  \ea 走了走\\
  \gll \emph{zǒu-le} \emph{zou}\\
  walk-\textsc{pfv} walk\\
  \glt `walked a bit'

  \ex 走了一走\\
  \gll \emph{zǒu-le} \emph{yi} \emph{zǒu}\\
  walk-\textsc{pfv} one walk\\
  \glt `had a walk'

  \z
\z

Moreover, diminishing\is{reduplication!diminishing (countericonic)} reduplication is subject to event structure
constraints %
(see %
%Fradin \& Kerleroux 2003
\citealt{Fradin03b}%
, for similar constraints in French word formation): the base \isi{verb} must be a process \isi{verb}, typically
controlled by an agent and crucially lacking a result, which captures
the fact that achievements, accomplishments and resultative compounds\is{compounding}
are systematically excluded from reduplication. Aspectually, the
reduplicated \isi{verb} is incompatible with the progressive\is{aspect!progressive} and durative
aspectual\is{aspect} markers while, as we have seen, it is perfectly compatible
with the perfective \isi{aspect} marker. Therefore, reduplication seems to
modify the event structure of the base \isi{verb}, providing a temporal
boundary to the unbounded process expressed by the base %
%(see Xiao \&
%McEnery 2004)
\citep[see][]{XiaoMcEnery2004}%
%Xiao-McEnery
%
. Other constraints, e.g. purely morphological constraints,
are not observed.

In view of these facts and under the assumption that aspectual\is{aspect}
properties are \emph{syntactically} encoded %
%(see e.g. Travis 2000, 2010;
%Borer 1994, 2005; McClure 1995; Ramchand 2008)
\citep[see e.g.][]{Travis2000,Travis2010,Borer1994,Borer2005,McClure1995,Ramchand2008}%
%Travis;Travis;Borer;Borer;McClure;Ramchand
%
, %
%Arcodia, Basciano \&
%
%Melloni (2014) 
\citet{ArcodiaBascianoEtAl2014} %
%Melloni
%
propose that diminishing\is{reduplication!diminishing (countericonic)} reduplication is a syntactic
phenomenon affecting the \emph{v}P domain, and develop a syntactic
analysis to account for it; the reader is referred to %
%Arcodia, Basciano
%\& %
%Melloni (2014) 
\citet{ArcodiaBascianoEtAl2014} %
%Melloni
%
and %
%Basciano \& Melloni (2017) 
\citet{BascianoMelloni2017} %
%?Basciano-Melloni
%
for further details of
the analysis.

Increasing\is{reduplication!increasing (iconic)} reduplication exhibits several properties that make it a very
different phenomenon from diminishing\is{reduplication!diminishing (countericonic)} reduplication. First, increasing\is{reduplication!increasing (iconic)}
reduplication is found mainly among adjectives\is{adjective}, but it can be found with
verbs\is{verb} and nouns\is{noun}/classifiers\is{classifier} too. Consider the following examples of
adjectival reduplication:\footnote{According
  to %
%Li \& Thompson (1981: 33)
\citet[33]{Li1981}%
%Li-Thompson
%
, in AABB reduplication of adjectives\is{adjective} the
  second syllable is unstressed, and thus has a neutral tone. However,
  there is no clear consensus on tonal patterns in this kind of
  reduplication. For example, according to %
%Tang (1988: 282)
\citet[282]{Tang88}%
%Tang
%
, the second
  syllable is in the neutral tone, while the third and fourth syllables,
  or just the fourth syllable, are in the first tone. Further, Tang
  observes that in Taiwan most people use the original tones,
  i.e. there is no tonal modification in this reduplication pattern %
%(see
%  also the examples in Paul 2010)
\citep[see   also the examples in][]{Paul2010a}%
%Paul
%
.}

\ea\label{ex:BascianoMelloni:6}

  \ea\label{ex:BascianoMelloni:6a}\sbsglll{小 (A)}{小小 (AA)}%
          {\emph{xiǎo}}{\emph{xiǎo\tld{}xiāo}}%
          {small}{small\tld{}small}%
          {`small'}{`very/really small'}%

  \ex\label{ex:BascianoMelloni:6b}\sbsgll{高興 (AB)}{高高興興 (AABB)}%
        {\emph{gāoxìng}}{\emph{gāo\tld{}gāo-xìng\tld{}xìng{}}}%
        {`happy'}{`very happy'}%

  \z

\z

In the adjectival domain, the increasing\is{reduplication!increasing (iconic)} function expressed by this kind
of reduplication is not necessarily `very Adj', but it rather makes the
adjectives\is{adjective} more descriptive, indicating a higher degree of liveliness
and vividness.\footnote{%
%Xu (2012a: 6) 
\citet[6]{Xu2012a} %
%Xu
%
states that, when adjectives\is{adjective} are
  reduplicated, the degree of the adjective\is{adjective}'s quality is generally
  intensified. However, this does not seem to be always the case in the
  modern language: for example, she observes that colour perception can
  be subjective and variable, and thus adjectives\is{adjective} indicating colours are
  prone to subjective interpretation.} As we will see in the next
section, differently from diminishing\is{reduplication!diminishing (countericonic)} reduplication, increasing\is{reduplication!increasing (iconic)}
reduplication requires that its base adjectives\is{adjective} and verbs\is{verb} have specific
structural properties.

Increasing\is{reduplication!increasing (iconic)} reduplication applies to verbs\is{verb} too, but only if the base is
bimorphemic and its constituents are in a relation of
coordination.\footnote{Reduplication of monosyllabic verbs\is{verb} (AA) in Modern
  Chinese\il{Chinese!Modern Chinese}  does exist but has a diminishing\is{reduplication!diminishing (countericonic)} meaning (see ex. (\ref{ex:BascianoMelloni:4a})).
  However, in previous stages of the language, before the appearance of
  the VV pattern with diminishing\is{reduplication!diminishing (countericonic)} meaning, reduplication of monosyllabic
  verbs\is{verb} had an increasing\is{reduplication!increasing (iconic)} function (repetition or action in progress);
  see e.g. \citet[7]{Xu2012a}.} 
  In (\ref{ex:BascianoMelloni:7}), for instance, the reduplicated \isi{verb}
portrays two interrelated actions which are performed alternately,
repeatedly, or an action performed by a great number of people.

\ea\label{ex:BascianoMelloni:7}%
\sbsglll{來往}{來來往往}% % (Side By Side glose macro, with central arrow)
        {\emph{lái-wǎng}}{\emph{lái\tld{}lái-wǎng\tld{}wǎng}}%
        {come-go}{come\tld{}come-go\tld{}go}%
        {`come and go'}{`come and go repeatedly, come and go in great numbers'}%
\z

AABB verbs\is{verb}, beside expressing pluractionality\is{aspect!pluractionality} or action in progress %
%(see
%Hu 2006, Ding 2010)
\citep[see][]{Hu06,Ding2010}%
%Hu;Ding
%
, can also express vividness (\ref{ex:BascianoMelloni:8}), or acquire an
extended meaning, losing their verbal meaning and becoming more similar
to adjectives\is{adjective} in meaning and distribution (\ref{ex:BascianoMelloni:9}),%
% start of long footnote with examples
%
\footnote{See the following
  examples, where
  偷偷摸摸\emph{tōu}\tld{}\emph{tōu-mō}\tld{}\emph{mō}
  is used as a nominal modifier \REF{fnex:basciano:a} and as an adverbial, both with
  \REF{fnex:basciano:bi} and without \REF{fnex:basciano:bii} the
  adverbial marker 地 -\emph{de} (examples from the Academia Sinica
  Balanced Corpus of Modern Chinese\il{Chinese!Modern Chinese}:
  \url{http://lingcorpus.iis.sinica.edu.tw/cgi-bin/kiwi/mkiwi/kiwi.sh?ukey=-78102521&qtype=1&ssl=7}
  [2017-08-25]).

% \begin{exe}
% \footnotesize
% 
% \exi{a.} 

\ea\label{fnex:basciano:a}
{[}\ldots{}{]}最機警與最偷偷摸摸的一種動物\\
\gll \emph{zuì} \emph{jījǐng} \emph{yǔ} \emph{zuì} \emph{tōu}\tld{}\emph{tōu-mō}\tld{}\emph{mō} \emph{de} \emph{yī} \emph{zhǒng} \emph{dòngwù}\\
most astute and most furtive \textsc{det} one {\textsc{clf} (type)} animal\\
\glt `{[}\ldots{}{]} the most astute and furtive animal.'

\ex
% \exi{b\textsuperscript{i}.}
\ea\label{fnex:basciano:bi}
不要偷偷摸摸寫\\
\gll \emph{bù} \emph{yào} \emph{tōu}\tld{}\emph{tōu-mō}\tld{}\emph{mō} \emph{xiě}\\
not have furtive write\\
\glt `You must not write furtively'

% \exi{b\textsuperscript{ii}.}
\ex\label{fnex:basciano:bii}
也儘量不要躲在角落裡偷偷摸摸地拍攝\\
\gll\emph{yě} \emph{jǐnliàng} \emph{bù} \emph{yào} \emph{duǒ-zài} \emph{jiǎoluò} \emph{lǐ} \emph{tōu}\tld{}\emph{tōu-mō}\tld{}\emph{mō-de} \emph{pāishè}\\
also as.mush.as.possible not have hide-at corner in
  furtive-\textsc{adv} take.picture\\
\glt `Also, as much as possible, you must not hide in a corner taking
  pictures furtively'
\z
\z

  Generally speaking, adjectives\is{adjective} may function as adverbs, modifying
  verbs\is{verb}. Adverbs are generally formed from adjectives\is{adjective} (though sometimes
  they can be formed from abstract nouns\is{noun}) but not from verbs\is{verb}. Basically,
  an adjective\is{adjective} may modify both a noun\is{noun}/NP or a \isi{verb}/VP, while a \isi{verb} may
  only modify a noun\is{noun}/NP %
%(see Arcodia 2014)
\citep[see][]{Arcodia2014}%
%Arcodia
%
.

  It must be noted, though, that basically all reduplicated AABB verbs\is{verb}
  can have an adverbial use, and thus they all share an important
  property of adjectives\is{adjective}:

% \begin{exe}
%   \exi{c.} 
\ea
  妻子和女兒說說笑笑地準備著晚飯。\\
  \gll \emph{qīzi} \emph{hé} \emph{nǚ'ér} \emph{shuō}\tld{}\emph{shuō-xiào}\tld{}\emph{xiào-de} \emph{zhǔnbèi-zhe} \emph{wǎnfàn}\\
  wife and daughter talk\tld{}talk-laugh\tld{}laugh-\textsc{adv}     prepare-\textsc{dur} dinner\\
  \glt `His wife and daughter were preparing dinner talking and laughing.'
\z
% \end{exe}

  (Center for Chinese Linguistics PKU corpus of Modern Chinese\il{Chinese!Modern Chinese}:
  \url{http://ccl.pku.edu.cn:8080/ccl_corpus/index.jsp?dir=xiandai}
  [2017-07-24])
} %
% end of long footnote with examples
%
 depending on the linguistic context %
%(on the meaning of
%AABB verbal reduplication, see Hu 2006)
\citep[on the meaning of AABB verbal reduplication, see][]{Hu06}%
%Hu
%
.

\ea\label{ex:BascianoMelloni:8}%
\sbsglll{跑跳}{跑跑跳跳}% % (Side By Side glose macro, with central arrow)
        {\emph{pǎo\textbf{-}tiào}}{\emph{pǎo}\tld{}\emph{pǎo}-\emph{tiào}\tld{}\emph{tiào}}%
        {run-jump}{run\tld{}run-jump\tld{}jump}%
        {`run and jump'}{`skip, run about, run and jump in a vivacious way'}

\ex\label{ex:BascianoMelloni:9}%
\sbsglll{偷摸}{偷偷摸摸}% % (Side By Side glose macro, with central arrow)
        {\emph{tōu-mō} }{\emph{tōu}\tld{}\emph{tōu-mō}\tld{}\emph{mō}}%
        {steal-touch}{steal\tld{}steal-touch\tld{}touch}%
        {`pilfer'}{`furtive, surreptitious, sneaky'}
\z

Finally, nouns\is{noun} can reduplicate too, conveying an overall increasing\is{reduplication!increasing (iconic)}
function, though AA reduplication no longer seems to be productive:

\vspace{\topsep}\ea\label{ex:BascianoMelloni:10}

  \ea\label{ex:BascianoMelloni:10a}%
  \sbsglll{天 (A)}{天天 (AA)}%
          {\emph{tiān}}{\emph{tiān}\tld{}\emph{tiān}}%
          {day}{day\tld{}day}%
          {`day'}{`every day'}%

  \ex\label{ex:BascianoMelloni:10b}%
  \sbsglll{花草 (AB)}{花花草草 (AABB)}%
          {\emph{huā-cǎo}}{\emph{huā}\tld{}\emph{huā-cǎo}\tld{}\emph{cǎo}}%
          {flower-plant/grass}{flower\tld{}flower-plant\tld{}plant}%
          {`flowers and plants'}{`(many) flowers and plants'}%
\z\z

Reduplicated monosyllabic nouns\is{noun} are said to have a distributive %
%(see
%e.g. Li \& Thompson 1981; Hu 1995; Li 2009; Xu 2012) 
\citep[see e.g.][]{Li1981,Hu95,Li09,Xu2012} %
%Li-Thompson;Hu;Li;Xu
%
or
\isi{plural}-collective %
%(Paris 2007) 
\citep{Paris2007} %
%?Paris
%
meaning. Given the specific meaning of
monosyllabic reduplications, their lack of productivity and the fact
that many of the nouns\is{noun} that can reduplicate display classifier\is{classifier}-like
properties, it is disputable whether AA reduplication applies
to actual nouns\is{noun} or nominal classifiers\is{classifier} (functional elements in the
\emph{extended} NP domain); we will go back to this in section 3.3. As
for disyllabic reduplicated nouns\is{noun}, the disyllabicity of the base
(classifiers\is{classifier} never are disyllabic) point to uncontroversially nominal
bases. Semantically, %
%Zhang (2015) 
\citet{Zhang2015} %
%Zhang
%
argues that AABB reduplication is a
\isi{plural} marker, expressing `greater plurality' %
%(see Corbett 2000)
\citep[see][]{Corbett2000}%
%Corbett
%
, but
according to %
%Xu (2012) 
\citet{Xu2012} %
%Xu
%
it indicates distributivity, as we will see in
section 3.3.

\subsection{Diminishing\is{reduplication!diminishing (countericonic)} vs. increasing\is{reduplication!increasing (iconic)} reduplication}

From the brief overview provided above, a first interesting
generalization arises. There is a correspondence between reduplicative
pattern (with consistent structure and meaning) and lexical category,
but limited to diminishing\is{reduplication!diminishing (countericonic)} reduplication: AA or ABAB diminishing\is{reduplication!diminishing (countericonic)}
reduplication applies only to verbs\is{verb}, as input and output categories.
Increasing\is{reduplication!increasing (iconic)} reduplication is very different in this respect because it
cross-cuts lexical categories rather than being firmly associated with a
word class (although AA/monosyllabic reduplication is unproductive
nowadays with nouns\is{noun} and classifiers\is{classifier}).

Let us now focus on other differences between the two types of
reduplication: it appears that the two functions of reduplication are
associated with a set of different formal and selectional properties. A
striking fact, especially in consideration of the great deal of unstable
meaning-structure correspondences in reduplication cross-linguistically,
is the tight correspondence between form and function observed in the
reduplication of disyllabic bases.\footnote{Many (if not most) languages
  do not exhibit such a clear correspondence between patterns and
  functions in reduplication %
%(Mattes 2014)
\citep{Mattes2014}%
%Mattes
%
.} While for monosyllabic
bases the difference between increasing\is{reduplication!increasing (iconic)} and diminishing\is{reduplication!diminishing (countericonic)} reduplication is
visible only at the suprasegmental level,\footnote{According to some,
  diminishing\is{reduplication!diminishing (countericonic)} reduplicated verbs\is{verb} are toneless, whereas the reduplicated
  adjective\is{adjective} always bears the first tone %
(%
%Tang 1988
\citealt{Tang88}%
%
: 282, 
%Paul 2010
\citealt{Paul2010a}%
%
: 120)%
%\citep[282, Paul 2010:   120]{Tang88}%
%Tang
%
. However, according to %
%Li \& Thompson (1981: 33)
\citet[33]{Li1981}%
%Li-Thompson
%
, the second
  syllable of reduplicated adjectives\is{adjective} too is unstressed. As for the few
  monosyllabic nouns\is{noun} that reduplicate in Modern Chinese\il{Chinese!Modern Chinese}, it seems that
  the reduplicant keeps the same tone as the base noun\is{noun}.} for disyllabic
bases (AB), the difference arises at the segmental level.

In the diminishing\is{reduplication!diminishing (countericonic)} function, the base is reduplicated as a whole (ABAB),
as in the ex. (\ref{ex:BascianoMelloni:4b}), while in the increasing\is{reduplication!increasing (iconic)} function, each morpheme is
reduplicated by itself (AABB), as seen in the examples (\ref{ex:BascianoMelloni:6b}), (\ref{ex:BascianoMelloni:7})-(\ref{ex:BascianoMelloni:9}) and
(\ref{ex:BascianoMelloni:10b}). Thus, it appears that there is a strong correlation between the
function and the form of reduplication: as hinted at in section 2.1, the
ABAB pattern always conveys diminishing\is{reduplication!diminishing (countericonic)} meaning, whereas the AABB
pattern is associated with increasing\is{reduplication!increasing (iconic)} semantics, regardless of the word
class of the input. Interestingly enough, the AABB pattern seems to be
associated with increasing\is{reduplication!increasing (iconic)} semantics also in other Sinitic languages
%
%(see Arcodia, Basciano \& Melloni 2015)
\citep[see][]{ArcodiaBascianoEtAl2015}%
%Arcodia-Basciano-Melloni
%
.

It is worth noting that some disyllabic words predominantly showing an
adjectival distribution can not only occur in the (standard) increasing\is{reduplication!increasing (iconic)}
template AABB, but they may also appear in the diminishing\is{reduplication!diminishing (countericonic)} ABAB
template, so that the same base eventually enters two reduplication
templates formally and functionally distinct:

\ea\label{ex:BascianoMelloni:11}

    \ea\label{ex:BascianoMelloni:11a}%
    \sbsgll{高興}{高高興興  (AABB) (cf. \ref{ex:BascianoMelloni:6b})}%
            {\emph{gāoxìng}}{\emph{gāo}\tld{}\emph{gāo-xìng}\tld{}\emph{xìng}}%
            {`happy'}{`very happy'}%

    \ex\label{ex:BascianoMelloni:11b}%
    \sbsgll{高興}{高興高興 (ABAB)}%
            {\emph{gāoxìng}}{\emph{gāoxìng} \emph{gāoxìng}}%
            {`happy'}{`have some fun'}%
\z\z

Crucially, these minimal pairs are restricted to disyllabic bases
amenable to a verbal/dynamic beyond an adjectival/stative
interpretation, as we can see in the ABAB pattern in (\ref{ex:BascianoMelloni:11b}). Therefore
(\ref{ex:BascianoMelloni:11b}) is not a counterexample to the generalization that only verbs\is{verb} can
be reduplicated along the ABAB pattern.

Moreover, the difference between diminishing\is{reduplication!diminishing (countericonic)} and increasing\is{reduplication!increasing (iconic)}
reduplication is not only semantic, but also concerns the restrictions
on the input and on the output. As for diminishing\is{reduplication!diminishing (countericonic)} reduplication, the
selection restrictions, as we have seen, seem to be aspectual\is{aspect} and
allegedly dependent on event structure constraints, while for increasing\is{reduplication!increasing (iconic)}
reduplication these restrictions are (mostly) morphological, as we will
see in the next section.

\section{Increasing\is{reduplication!increasing (iconic)} reduplication: input and output}

Different from diminishing\is{reduplication!diminishing (countericonic)} reduplication, increasing\is{reduplication!increasing (iconic)} reduplication
requires that its bases have specific morphotactic and semantic
properties. In what follows we focus on the category-specific and
category-neutral restrictions of increasing\is{reduplication!increasing (iconic)} reduplication and describe
the properties of the outputs of these reduplications across the major
lexical categories.

\subsection{Adjectives\is{adjective}}

In the adjectival domain increasing\is{reduplication!increasing (iconic)} reduplication applies indifferently
to monosyllabic and to disyllabic bases. In both cases, the base
adjective\is{adjective} must be gradable, thus absolute adjectives\is{adjective} cannot reduplicate:
e.g. 方 \emph{fāng} `square' cannot give rise to *\hspace*{-.1pt}方方
\emph{fāng\tld{}fāng} %
(see %
%Paris 1979
\citealt{Paris1979}%
, cit. in %
%Paul 2010
\citealt{Paul2010a}%
: 139, fn. 18).
%\citep[see][, cit. in Paul 2010: p. 139, fn. 18]{Paris1979}%
%Paris
%
%
% start of long footnote with examples
%
\footnote{However, %
%Tang (1988: 279-283) 
\citet[279-283]{Tang88} %
%Tang
%
lists方方
  \emph{fāng\tld{}fāng}
  `square\emph{\tld{}}square' among possible reduplicated
  adjectives\is{adjective}. This could be possibly the result of a coerced
  interpretation (see e.g. English  \emph{very square face}).
  Indeed, Tang highlights that adjectives\is{adjective} that express distinctive
  properties (e.g. appearance, size and colour) generally can
  reduplicate even when, as in the case of 方\emph{fāng} `square', they
  are not used predicatively and cannot be modified by degree adverbs
  %
%(examples from Tang 1988: 283)
\citep[examples from][283]{Tang88}%
%Tang
%
:

\ea
[\textsuperscript{??}]{他的臉很方\\
  \gll \emph{tā} \emph{de} \emph{liǎn} \emph{hěn} \emph{fāng}\\
  \textsc{3sg.m} \textsc{det} face very square\\
  \glt `His face is very square.'}
\z


\ea[\textsuperscript{??}]{( 很)方的臉\\
  \gll \emph{(hěn)} \emph{fāng} \emph{de} \emph{liǎn}\\
  (very) square \textsc{det} face\\
  \glt `A (very) square face'}
\z

\ea
[]{ 方方的臉\\
  \gll \emph{fāng\tld{}fāng} \emph{de} \emph{liǎn}\\
  square\emph{\tld{}}square \textsc{det} face\\
  \glt `A (very/really) square face'}
\zlast
  }%
% end of long footnote with examples
%
\ Therefore, adjectival reduplication only applies to
bases that encode a degree/scalar value %
%(see also Zhu 2003)
\citep[see also][]{Zhu03}. %
%Zhu
%
At the
morphotactic level, we find restrictions as far as disyllabic bases are
concerned: as a matter of fact, the AABB pattern requires a disyllabic
\emph{and} bimorphemic base, whereas disyllabic monomorphemic words
cannot be reduplicated %
%(Paul 2010: 137)
\citep[137]{Paul2010a}%
%Paul
%
:%
\footnote{窈窕 \emph{yǎotiǎo} is an example of partial
  reduplication in Old Chinese\il{Chinese!Old Chinese}, involving rhymes only, traditionally
  called 叠韵 \emph{diéyùn} `reduplicated rhymes': 窈窕 *\textsuperscript{a}ʔiwʔ-liwʔ
  \textgreater{} ewX-dewX \textgreater{} \emph{yǎotiǎo} %
%(Sagart 1999:
%  137)
\citep[  137]{Sagart1999}%
%Sagart
%
.}

\ea\label{ex:BascianoMelloni:12}%
\sbsgll{窈窕{}}{*\hspace*{-.0pt}窈窈窕窕}% % (Side By Side glose macro, with central arrow)
        {\emph{yǎotiǎo}}{*\emph{yǎo\tld{}yǎo-tiǎo\tld{}tiǎo}}%
        {`graceful, gentle'}{}%

\z

Also, the two morphemes must be lexical. For instance, adjectives\is{adjective} formed
with a prefix-like element cannot reduplicate %
%(see Zhu 2003)
\citep[see][]{Zhu03}%
%Zhu
%
:


\ea\label{ex:BascianoMelloni:13}%
\sbsglll{不安}{*\hspace*{-.0pt}不不安安}% % (Side By Side glose macro, with central arrow)
        {\emph{bù-ān}}{*\emph{bù\tld{}bù-ān\tld{}ān}}%
        {not-peaceful}{}%
        {`troubled/restless'}{}%
\z

It thus appears that units are here handled strictly on a morphemic
basis, rather than on a prosodic basis. Moreover, the possible bases for
AABB reduplication are either lexicalized, non-transparent bases (\ref{ex:BascianoMelloni:14a}),
or adjectives\is{adjective} formed by two morphemes with a similar meaning (\ref{ex:BascianoMelloni:14b}) or in
a logical coordination (\ref{ex:BascianoMelloni:14c}):

\ea\label{ex:BascianoMelloni:14}
    \ea\label{ex:BascianoMelloni:14a}%
    \sbsglll{馬虎}{馬馬虎虎}%
            {\emph{mǎ-hu}}{\emph{mǎ\tld{}ma-hū\tld{}hū}}%
            {horse-tiger}{horse\tld{}horse-tiger\tld{}tiger}%
            {`careless, casual'}{`careless, casual (stronger)'}%

    \ex\label{ex:BascianoMelloni:14b}%
    \sbsglll{快樂}{快快樂樂}%
            {\emph{kuài-lè}}{\emph{kuài\tld{}kuài-lè\tld{}lè}}%
            {pleased-happy}{pleased\tld{}pleased-happy\tld{}happy}%
            {`happy'}{`very/really happy'}%

    \ex\label{ex:BascianoMelloni:14c}%
    \sbsglll{高大}{高高大大}%
           {\emph{gāo-dà}}{\emph{gāo\tld{}gāo-dà\tld{}dà}}%
           {tall-big}{tall\tld{}tall-big\tld{}big}%
           {`tall and big'}{`very/really tall and big'}

\z
\z

These data show that the disyllabic AABB template applies to complex
bases that are structurally and semantically symmetrical, i.e.
exocentric or coordinative structures lacking a clearly identifiable
head. Adjectival reduplication, thus, seems to be conditioned by
morphosyntactic (word-internal) factors.

As for the output, the reduplicated adjective\is{adjective} loses its gradability:
while the base must be gradable, the reduplicated adjective\is{adjective} is no longer
gradable. As a matter of fact, whereas the (scalar) base adjective\is{adjective} is
compatible with degree modifiers such as `very' and `fairly', which
indicate a high level on the scale of the (gradable) property expressed
by the adjective\is{adjective} they modify, the reduplicated adjective\is{adjective} is not:


%\ea\label{ex:BascianoMelloni:15}
%
%    \ea\label{ex:BascianoMelloni:15a}%
%    長 \emph{cháng} `long' ~~~$\rightarrow$~~ 非常長 \emph{fēicháng cháng} `very long'
% 
%    \ex\label{ex:BascianoMelloni:15b}%
%    長長 \emph{cháng}\tld{}\emph{cháng} `long\tld{}long' \\ \hfill~~~$\rightarrow$~~ {*非常長長} \emph{fēicháng cháng}\tld{}\emph{cháng} `very long\tld{}long'%
%\z\z
\ea\label{ex:BascianoMelloni:15}

    \ea\label{ex:BascianoMelloni:15a}%
    \sbsgll{長}{非常長}%
            {\emph{cháng}}{\emph{fēicháng cháng}}%
%            {long}{very long}%
            {`long'}{`very long'}%
 
    \ex\label{ex:BascianoMelloni:15b}%
%    長長 \emph{cháng}\tld{}\emph{cháng} `long\tld{}long' \\ \hfill~~~$\rightarrow$~~ {*非常長長} \emph{fēicháng cháng}\tld{}\emph{cháng} `very long\tld{}long'%
    \sbsgll{長長}{*非常長長}%
            {\emph{cháng\tld{}cháng}}{\emph{fēicháng cháng\tld{}cháng}}%
            {`long\tld{}long'}{*`very long\tld{}long'}%
%            {`long'}{*`very long'}%
\z\z

Moreover, whereas the base adjective\is{adjective} can appear in the comparative
construction, the reduplicated adjective\is{adjective} cannot:

\ea\label{ex:BascianoMelloni:16}

    \ea\label{ex:BascianoMelloni:16a}  我的頭髮比他的長。\\
    \gll  \emph{wǒ} \emph{de} \emph{tóufa} \emph{bǐ} \emph{tā} \emph{de} \emph{cháng}\\
    1\textsc{sg} \textsc{det} hair \textsc{comp} 3\textsc{sg.m} \textsc{det}
long\\
  \glt `My hair is longer than his.'\\

    \ex\label{ex:BascianoMelloni:16b}  *我的頭髮比他的長長。\\
    \gll  \emph{wǒ} \emph{de} \emph{tóufa} \emph{bǐ} \emph{tā} \emph{de} \emph{cháng}\tld{}\emph{cháng}\\
    1\textsc{sg} \textsc{det} hair \textsc{comp} 3\textsc{sg.m} \textsc{det}
long\tld{}long\\

\z\z
However, there is a group of adjectives\is{adjective} for which reduplication works
differently. These are adjectives\is{adjective} that typically involve a modifier-head
structure, such as 雪白 \emph{xuě-bái} `snow-white', which reduplicates
as ABAB (雪白雪白 \emph{xuě-bái\tld{}xuě-bái}). The function
is reportedly increasing\is{reduplication!increasing (iconic)}, as in the case of AABB reduplicated
adjectives\is{adjective}. This might appear as an exception to the form-function
identity between ABAB reduplication and diminishing\is{reduplication!diminishing (countericonic)} meaning in
Mandarin.\footnote{According to %
%Paul (2010: 137, fn. 15)
\citet[137, fn. 15]{Paul2010a}%
%Paul
%
, ``{[}the{]}
  reduplication pattern for `modifier-adjectival head' compounds\is{compounding}
  deriving an adjective\is{adjective} of the form {[}\textsubscript{A°} ABAB{]} is not
  to be confounded with the repetition of a disyllabic \isi{verb} as a whole
  in syntax: {[}\textsubscript{V°} AB{]} {[}\textsubscript{V°} AB{]}''.}
It must be noted, though, that modifier-head adjectives\is{adjective} like雪白
\emph{xuě-bái} `snow-white' are not gradable and, indeed, they are not
compatible with degree adverbs and cannot be used in the comparative
construction. Therefore, reduplication does not result in a change in
gradability of the base adjective\is{adjective}, as it is the case with AA and AABB
adjectival reduplication. Adjectival ABAB reduplication, thus, seems to
be a phenomenon distinct from the other patterns of reduplications
described in this section. We will go back to this issue in section
3.5., when discussing the word/\isi{lexeme} status of the bases of increasing\is{reduplication!increasing (iconic)}
reduplication.

\subsection{Verbs\is{verb}}
\is{verb|(}
As for verbs, increasing\is{reduplication!increasing (iconic)} reduplication poses no aspectual\is{aspect} requirements
on the base unit since all kinds of verbs, including inherently telic
verbs like 來 \emph{lái} `come', 進 \emph{jìn} `enter' or 出 \emph{chū}
`exit', are allowed (see ex. (\ref{ex:BascianoMelloni:7}), repeated here as (\ref{ex:BascianoMelloni:17c})). Nonetheless,
increasing\is{reduplication!increasing (iconic)} reduplication requires base units that possess specific
structural properties. As a matter of fact, AABB increasing\is{reduplication!increasing (iconic)}
reduplication is generally possible only for coordinated complex verbs,
the constituents of which may be either in a relation of logical
coordination (\ref{ex:BascianoMelloni:17a}), synonymy (\ref{ex:BascianoMelloni:17b}) or antonymy (\ref{ex:BascianoMelloni:17c}):

\ea\label{ex:BascianoMelloni:17}

\ea\label{ex:BascianoMelloni:17a}%
\sbsglll{說笑}{說說笑笑}% % (Side By Side glose macro, with central arrow)
        {\emph{shuō-xiào}}{\emph{shuō\tld{}shuō-xiào\tld{}xiào}}%
        {talk-laugh}{talk\tld{}talk-laugh\tld{}laugh}%
        {`talk and laugh'}{`talk and laugh continuously'}%

\ex\label{ex:BascianoMelloni:17b}%
\sbsglll{叫嚷}{叫叫嚷嚷}% % (Side By Side glose macro, with central arrow)
        {\emph{jiào-rǎng}}{\emph{jiào\tld{}jiào-rǎng\tld{}rǎng}}%
        {call-shout}{call\tld{}call-shout\tld{}shout}%
        {`shout, howl'}{`shout repeatedly'}%

\ex\label{ex:BascianoMelloni:17c}%
\sbsglll{來往}{來來往往}% % (Side By Side glose macro, with central arrow)
        {\emph{lái-wǎng}}{\emph{lái\tld{}lái-wǎng\tld{}wǎng}}%
        {come-go}{come\tld{}come-go\tld{}go}%
        {`come and go'}{`come and go repeatedly, come and go in great numbers'}%
\z\z

Note that in (\ref{ex:BascianoMelloni:17}) the bases of reduplication are existing verbs, but
this is not necessarily always the case, as e.g.  走走停停
\emph{zǒu\tld{}zǒu-tíng\tld{}tíng} `walk and stop'
(there is no corresponding base verb 走停 \emph{zǒu-tíng}).\footnote{An
  alternative analysis might pose that verbal AABB reduplication is the
  result of the coordination of two reduplicated verbs,
  {[}A\emph{\tld{}}A{]} {[}B\emph{\tld{}}B{]}.
  However, note that since the reduplication of monosyllabic verbs
  expresses a delimitative meaning, the coordination of two monosyllabic
  reduplicated verbs should result in a delimitative semantics. Further,
  this analysis is not tenable because telic verbs like 來 \emph{lái}
  `come', as said above, cannot reduplicate by themselves, \mbox{*來來}
  \emph{lái\tld{}lái}.}

Also, it is worth remarking that the verbal reduplication pattern AABB
may also be found with disyllabic monomorphemic verbs, such as (\ref{ex:BascianoMelloni:18a}) or
other kind of \is{compounding}compound verbs\is{compounding!compound verb} (\ref{ex:BascianoMelloni:18b} and \ref{ex:BascianoMelloni:18c}):

\ea\label{ex:BascianoMelloni:18}
\ea\label{ex:BascianoMelloni:18a} \gll 哆嗦 \\ \emph{duōsuo} \\ \glt `tremble'
\ex\label{ex:BascianoMelloni:18b} \gll 飄悠 \\ \emph{piāo-you} \\ \glt `float-long/leisurely, wobble, stagger'
\ex\label{ex:BascianoMelloni:18c} \gll 鬧騰 \\ \emph{nào-teng} \\ \glt `noisy-jump, disturb/create confusion'
\z\z

As for the prosodic properties of the pattern, the second
morpheme/syllable of non-coordinate \is{compounding}compound verbs\is{compounding!compound verb} that can undergo AABB
reduplication generally has the neutral tone, suggesting that these are
lexicalized forms.\footnote{Toneless items in \ili{Chinese} are typically
  grammatical morphemes, such as e.g. aspectual\is{aspect} markers, (some) no
  longer productive derivational suffixes, and the second syllables of
  some reduplicated or \is{compounding}compound words, as e.g. 爸爸 \emph{bàba}
  `father', 學生\emph{xuésheng} `student'. Thus, lack of tone is a clue
  of either grammaticalization or lexicalization.} Thus, similarly to
adjectives\is{adjective}, the AABB template in the verbal domain basically applies to
structurally and semantically symmetrical bases, but it can also apply
to unanalyzable morphemes or to lexicalized forms.\footnote{The only
  constraint which does not seem to be morphological but rather
  aspectual\is{aspect} concerns coordination of telic verbs: as we have seen, telic
  verbs may appear in the AABB pattern of reduplication, but if they do
  they must be antonyms (as in ex. \ref{ex:BascianoMelloni:7}/\ref{ex:BascianoMelloni:17c}), i.e. reduplication of
  synonymic telic verbs does not seem to be possible %
%(see Zhang 2016)
\citep[see][]{Zhang2016}%
%Zhang
%
.
  This might be due to the fact that the coordination of two antonymic
  telic verbs (like \emph{enter}-\emph{exit}) results in the annulment
  of the \emph{télos}, which seems to suggest that, actually, the bases
  of this kind of reduplication too must express an overall
  \emph{atelic} event. This issue deserves further research.} For some
of these lexicalized forms, it is possible that they originate from
coordinating structures whose relationship became opaque with time, but
an in depth diachronic analysis is needed to substantiate this
hypothesis.

As for the output, AABB reduplication of verbs seems to operate at the
aspectual\is{aspect} level, expressing repetition or action in progress. However,
as we have seen, it can also express vividness (\ref{ex:BascianoMelloni:8}), or other kinds of
more abstract meanings (\ref{ex:BascianoMelloni:9}), closely approaching \emph{adjectival}
reduplicative processes.
\is{verb|)}
\subsection{Nouns\is{noun}}

As we have seen, reduplicated monosyllabic nouns\is{noun} are said to have a
`distributive' or `\isi{plural} collective' meaning:

\ea\label{ex:BascianoMelloni:19} 人人都喜歡受人稱贊。\\
\gll \emph{rén\tld{}rén} \emph{dōu} \emph{xǐhuan} \emph{shòu} \emph{rén} \emph{chēngzàn}\\
person\emph{\tld{}}person all like receive person praise\\
\glt `Everybody likes to be praised by people.'
\z

Several authors %
%(e.g. Hu 1995; Cai 2007; Li 2009) 
\citep[e.g.][]{Hu95,Cai07,Li09} %
%Hu;Cai;Li
%
stress the fact that
reduplication of monosyllabic nouns\is{noun} may be assimilated to classifier\is{classifier}
reduplication and that many of the nouns\is{noun} that can reduplicate show
classifier\is{classifier}-like properties. For example, %
%Hu (1995: 103) 
\citet[103]{Hu95} %
%Hu
%
observes that at
least part of these (alleged) nominal bases can directly follow a
numeral without an intervening classifier\is{classifier}, as e.g. 一年 \emph{yī nián}
`one year', 三戶 \emph{sān hù} `three households', and they can
themselves work as classifiers\is{classifier}, as e.g. 三戶人家 \emph{sān hù rénjiā}
`three household (\textsc{clf}) family, three families', thus exhibiting
properties of (nominal) classifiers\is{classifier}.

Reduplication of classifiers\is{classifier} -- how it is generally reported in
reference grammars -- seems to convey a distributive meaning:

\ea\label{ex:BascianoMelloni:20}看書的時候,書上的字不可能個個都認識。\\
\gll \emph{kàn} \emph{shū} \emph{de} \emph{shíhou}, \emph{shū} \emph{shàng} \emph{de} \emph{zì} \emph{bù} \emph{kěnéng} \emph{gè\tld{}gè} \emph{dōu} \emph{rènshi}\\
read book \textsc{det} time book on \textsc{det} character not can \textsc{clf}\emph{\tld{}}\textsc{clf} all know\\
\glt `You cannot know all the characters/each character of the books you
read.'\\
\z

According to %
%Paris (2007: 68)
\citet[68]{Paris2007}%
%?Paris
%
, however, reduplicated classifiers\is{classifier} get a
(\isi{plural}) distributive meaning when they appear in pre-verbal position
(\ref{ex:BascianoMelloni:21a}), while they get a \isi{plural} collective interpretation when they
occupy the post-verbal position (\ref{ex:BascianoMelloni:21b}):%
\footnote{Paris notes that it
  is not possible to have the noun\is{noun} preceded by the reduplicated
  classifier\is{classifier} in post-verbal position with the same meaning as (\ref{ex:BascianoMelloni:21a}), so
  that the following sentence is ungrammatical:
\begin{exe}
  \exi{(i)}[*]{他認得個個學生 。\\
  \gll \emph{tā} \emph{rènde} \emph{gè\tld{}gè} \emph{xuésheng}\\
  3\textsc{sg.m} be.acquainted.with
  \textsc{clf}\emph{\tld{}}\textsc{clf} student\\}
\end{exe}}


\ea\label{ex:BascianoMelloni:21}
\ea\label{ex:BascianoMelloni:21a} 他個個學生都認得。\\
\gll \emph{tā} \emph{gè\tld{}gè} \emph{xuésheng} \emph{dōu} \emph{rènde}\\
3\textsc{sg.m} \textsc{clf}\emph{\tld{}}\textsc{clf} student all be.acquainted.with\\
\glt `He knows all the students (individually).'
\ex\label{ex:BascianoMelloni:21b} 在分析上遇見種種困難 \\
\gll \emph{zài} \emph{fēnxī} \emph{shàng} \emph{yùjiàn} \emph{zhǒng\tld{}zhǒng} \emph{kùnnan}\\
at analysis on meet \textsc{clf}\emph{\tld{}}\textsc{clf} difficulty\\
\glt `Come across all kinds of difficulties during the analysis.'
\z\z
According to %
%Zhang (2014)
\citet{Zhang2014}%
%Zhang
%
, reduplication of classifiers\is{classifier} in Mandarin is a
type of \isi{plural} marking; it denotes plurality of \emph{units}
(groups/collectives) rather than of individuals. Units and individuals
can overlap, like in (\ref{ex:BascianoMelloni:22a}), but it is not always the case, like in
(\ref{ex:BascianoMelloni:22b}), where `lotus' is the individual, while `lotus pile' is the unit
that reduplicates (examples from %
%Zhang, 2014
\citealt{Zhang2014}: 6):

\largerpage 
\ea\label{ex:BascianoMelloni:22}

	\ea{\label{ex:BascianoMelloni:22a}河裏漂著(一)多多蓮花。\\
	\gll \emph{hé} \emph{lǐ} \emph{piāo-zhe} \emph{( yī )} \emph{duō\tld{}duō} \emph{liánhuā}\\
	river in float-\textsc{dur} (one) \textsc{clf}\emph{\tld{}}\textsc{clf} lotus\\
	\glt `There are many lotuses floating on the river.'}

	\ex{\label{ex:BascianoMelloni:22b} 地上有一堆堆蓮花。\\
	\gll \emph{dì} \emph{shàng} \emph{yǒu} \emph{yī} \emph{duī\tld{}duī} \emph{liánhuā}\\
	earth on have one \textsc{clf}(pile)\emph{\tld{}}\textsc{clf} lotus\\
	\glt `There are piles of lotuses on the ground.'}\\
	\z
\z

%
%Zhang (2014: 12) 
\citet[12]{Zhang2014} %
%Zhang
%
argues that the distributive meaning emerges when
reduplicated classifiers\is{classifier} occur with the adverb 都 \emph{dōu} `all' %
%(even
%when it is allowed but does not show up; see e.g. Guo 1999) 
\citep[even when it is allowed but does not show up; see e.g.][]{Guo99} %
%Guo
%
or other
kinds of adverbials:

\ea\label{ex:BascianoMelloni:23}個個學生都有自己的網頁。\\
\gll \emph{gè\tld{}gè} \emph{xuésheng} \emph{dōu} \emph{yǒu} \emph{zìjǐ} \emph{de} \emph{wǎngyè}\\
\textsc{clf}\emph{\tld{}}\textsc{clf} student all have own \textsc{det} webpage\\
\glt `All of the students have their own webpage.'\\
\z

In contrast, according to Zhang, in (\ref{ex:BascianoMelloni:24}), where no 都 \emph{dōu} `all'
is allowed, the distributive meaning is not possible (example from %
%Zhang 2014
\citealt{Zhang2014}%
%
: 12):

\ea\label{ex:BascianoMelloni:24}雙雙情人步入會場。\\
\gll \emph{shuāng\tld{}shuāng} \emph{qíngrén} \emph{bù-rù} \emph{huì-chǎng} \\
 {\textsc{clf} (pair)\emph{\tld{}}\textsc{clf}} lover step-enter meet-place\\
\glt `Many pairs of lovers stepped into the meeting place.'
\z


According to %
%Zhang (2014: 12)
\citet[12]{Zhang2014}%
%Zhang
%
, the fact that reduplicated classifiers\is{classifier} do
not have an intrinsic distributive reading is proven by the
compatibility with collective verbs\is{verb}.

Going back to reduplication of monosyllabic nouns\is{noun} proper, 
\citet{Paris2007}
argues that it expresses a `\isi{plural} collective' meaning, more
specifically it denotes a collectivity of elements sharing the same
properties, which can function either as an argument or as an adverbial.
According to %
%Paris (2007: 69-70)
\citet[69-70]{Paris2007}%
%?Paris
%
, reduplication of monosyllabic units
does not have a distributive meaning, as shown by the contrast between
(\ref{ex:BascianoMelloni:25a}) and (\ref{ex:BascianoMelloni:25b}), where the first one contains a reduplicated noun\is{noun}
(天天\emph{tiān\tld{}tiān} `day\emph{\tld{}}day,
every day'), while the second contains the quantifier 每\emph{měi}
`each'. In (\ref{ex:BascianoMelloni:25b}) the object is necessarily distributed, i.e. it must be
a different poem every day, while this is not necessarily the case in
(\ref{ex:BascianoMelloni:25a}).\footnote{Note that in (\ref{ex:BascianoMelloni:25a}) 都
  \emph{dōu} `all' is used but, according to Paris, we do not get the
  distributive reading. This contrasts with what Zhang argues about
  classifiers\is{classifier}, where the presence of this adverb would lead to a
  distributive reading (see above).}
  
  
\ea\label{ex:BascianoMelloni:25}
\ea\label{ex:BascianoMelloni:25a} 他天天都讀一首詩。\\
\gll \emph{tā} \emph{tiān\tld{}tiān} \emph{dōu} \emph{dú} \emph{yī} \emph{shǒu} \emph{shī}\\
 3\textsc{sg.m} day\tld{}day all read one \textsc{clf} poem\\
\glt `He reads a poem every day.'

\ex\label{ex:BascianoMelloni:25b} 他每一天都讀一首詩。\\
\gll \emph{tā} \emph{měi} \emph{yī} \emph{tiān} \emph{dōu} \emph{dú} \emph{yī} \emph{shǒu} \emph{shī}\\
 3\textsc{sg.m} each one day all read one \textsc{clf} poem\\
 \glt `Every day he reads a (different) poem.'\\

 \z\z

Providing a detailed picture of the kind of \isi{plural} readings expressed by
reduplicated classifiers\is{classifier} is beyond the scope of this chapter; however,
what we want to stress here is that it is not easy to trace a clear
boundary between different kinds of \isi{plural} readings and that arguably
different readings can be related to distributional/syntactic rather
than solely lexical factors.

As for reduplication of disyllabic nouns\is{noun}, a first element is the
undisputable categorial nature of the input, since classifiers\is{classifier} are all
monosyllabic. Structurally, nominal bases seem to be subject to the same
morphological constraints observed for AABB adjectives\is{adjective} and verbs\is{verb}. The AB
base nouns\is{noun} usually entail a relation of coordination between their
constituents: either logical coordination (see \ref{ex:BascianoMelloni:26a}), or synonyms or
antonyms (\ref{ex:BascianoMelloni:26b}) %
(see %
%Tang 1979
\citealt{Tang79}%
%
: 114; 
%Zhang 2015
\citealt{Zhang2015}
):\footnote{Note that some
  AABB lexicalized nouns\is{noun} do not have a AB \is{compounding}compound counterpart %
%(see Wu
%  \& Shao 2001: 12)
\citep[see][12]{Wu01}%
%Wu-Shao
%
: e.g.
  生生世世\emph{shēng\tld{}shēng-shì\tld{}shì}
  `life\tld{}life-generation\tld{}generation,
  generation after generation'(*\hspace*{-.1pt}生世\emph{shēng-shì}). Generally
  speaking, it is possible to form AABB nouns\is{noun} from the coordination of
  two items that do not form an AB \is{compounding}compound (see (\ref{ex:BascianoMelloni:28b}) and the related
  discussion).}

\ea\label{ex:BascianoMelloni:26}
\ea\label{ex:BascianoMelloni:26a}%
\sbsglll{家戶}{家家戶戶}% % (Side By Side glose macro, with central arrow)
        {\emph{jiā-hù}}{\emph{jiā\tld{}jiā-hù\tld{}hù}}%
        {family-household}{family\emph{\tld{}}family-household\emph{\tld{}}household}%
        {`household/family'}{`every family/each household/many families'}%

\ex\label{ex:BascianoMelloni:26b}%
\sbsglll{老少}{老老少少}% % (Side By Side glose macro, with central arrow)
        {\emph{lǎo-shào}}{\emph{lǎo\tld{}lǎo-shào\tld{}shào}}%
        {old-young}{old\emph{\tld{}}old-young\emph{\tld{}}young}%
        {`the old and\\ \textcolor{white}{`}the young'}{`old people and young people'}%
\z\z
As we have seen with adjectives\is{adjective} (\ref{ex:BascianoMelloni:14}), we can also find more lexicalized
forms like:

\ea\label{ex:BascianoMelloni:27}
\ea\label{ex:BascianoMelloni:27a}%
\sbsglll{風雨}{風風雨雨}% % (Side By Side glose macro, with central arrow)
        {\emph{fēng-yǔ}}{\emph{fēng\tld{}fēng-yǔ\tld{}yǔ}}%
        {wind-rain}{wind\emph{\tld{}}wind-rain\emph{\tld{}}rain}%
        {`wind and rain/\\ \textcolor{white}{`}trials and\\ \textcolor{white}{`}hardships'}{`trials and hardships/storms'}%

\ex\label{ex:BascianoMelloni:27b}%
\sbsglll{點滴}{點點滴滴}% % (Side By Side glose macro, with central arrow)
        {\emph{diǎn-dī}}{\emph{diǎn\tld{}diǎn-dī\tld{}dī}}%
        {dot-drip/drop}{dot\emph{\tld{}}dot-drop\emph{\tld{}}drop}%
        {`droplet'}{`dribs and drabs/bit by bit'}%
\z\z
The nominal AABB pattern of reduplication seems to be well-established
in the \ili{Chinese} lexicon %
%(see e.g. Hu 1995; Wu \& Shao 2001)
\citep[see e.g.][]{Hu95,Wu01}%
%Hu;Wu-Shao
%
, and can be
extended to disyllabic nouns\is{noun} that usually do not reduplicate (\ref{ex:BascianoMelloni:28a}, %
%Hu 1995
\citealt{Hu95}%
%
: 106). Also, two monosyllabic nouns\is{noun} A and B that do not form a AB
\is{compounding}compound word, but satisfy the coordination requirements seen above, can
reduplicate along the AABB pattern forming novel combinations %
%(\ref{ex:BascianoMelloni:28b}, see
%Wu \& Shao 2001: 12)
\citep[\ref{ex:BascianoMelloni:28b}, see][12]{Wu01}%
%Wu-Shao
%
:

%  \begin{tabular}[t]{ll@{\hspace*{.1\linewidth}}l}
%  {} & $\rightarrow$ & {}\\
%  \begin{minipage}[t]{.3\linewidth}
%  \gll \\ 
%        \\ 
%  \glt 
%  \end{minipage}
%  & &
%    \begin{minipage}[t]{.5\linewidth}
%  \gll \\ 
%        \\ 
%  \glt 
%  \end{minipage}
%  \end{tabular}


\ea\label{ex:BascianoMelloni:28}
\ea\label{ex:BascianoMelloni:28a}%
  \begin{tabular}[t]{ll@{\hspace*{.1\linewidth}}l}
  {情景} & $\rightarrow$ & {情情景景}\\
  \begin{minipage}[t]{.3\linewidth}
  \gll \emph{qíng-jǐng}\\ 
        feeling-scene\\ 
  \glt `scene, sight,\\ \textcolor{white}{`}circumstances'
  \end{minipage}
  & &
    \begin{minipage}[t]{.4\linewidth}
  \gll \emph{qíng\tld{}qíng-jǐng\tld{}jǐng}\\ 
        feeling\emph{\tld{}}feeling-scene\emph{\tld{}}scene\\ 
  \glt `every scene, all scenes'
  \end{minipage}
  \end{tabular}
\ex\label{ex:BascianoMelloni:28b}%
  \begin{tabular}[t]{ll@{\hspace*{.1\linewidth}}l}
  {盆罐} & $\rightarrow$ & {盆盆罐罐}\\
  \begin{minipage}[t]{.3\linewidth}
  \gll \emph{pén} \emph{guàn}\\ 
        `basin/pot' `jar'\\ 
  \end{minipage}
  & &
    \begin{minipage}[t]{.4\linewidth}
  \gll \emph{pén\tld{}pén-guàn\tld{}guàn}\\ 
        {{`pots and jars'}}\\ 
  \end{minipage}
  \end{tabular}
\z\z
According to %
%Zhang (2015: 7)
\citet[7]{Zhang2015}%
%Zhang
%
, though, the AABB nominal pattern is not
productive, since many acceptable \is{compounding}compound nouns\is{noun}\is{compounding!compound noun} formed by parallel
constituents do not reduplicate (she argues the same for verbs\is{verb} too).
This is however questionable since e.g. one of the example she mentions,
i.e. 桌椅\emph{zhuō-yī} `table-chair, tables and chairs' → 桌桌椅椅
\emph{zhuō\tld{}zhuō-yī\tld{}yī}
`table\emph{\tld{}}table-chair\emph{\tld{}}chair',
is listed as an example of reduplicated AABB noun\is{noun} by %
%Wu \& Shao (2001:
%12-13)
\citet[12-13]{Wu01}%
%Wu-Shao
%
, who put it among AABB `temporary' combinations with low
frequency. Even though it is not easy to establish the productivity of a
pattern, we believe that `occasional' usages and the possibility to coin
new AABB nouns\is{noun} are hints of its productivity.

As for its function, as we have mentioned, %
%Zhang (2015) 
\citet{Zhang2015} %
%Zhang
%
argues that AABB
expresses `greater plurality' %
%(see also Wu \& Shao 2001)
\citep[see also][]{Wu01}%
%Wu-Shao
%
, though it
sometimes seems to have a distributive meaning, like in the case of
reduplicated monosyllabic nouns\is{noun}; and, indeed, as we have seen, according
to %
%Xu (2012a)
\citet[]{Xu2012a}%
%Xu
%
, reduplicated AABB nouns\is{noun} indicate distributivity. See the
examples below:\footnote{Examples from Academia Sinica Balanced Corpus of
  Modern Chinese\il{Chinese!Modern Chinese}:
  \url{http://app.sinica.edu.tw/cgi-bin/kiwi/mkiwi/kiwi.sh}
  [2016-11-24].}

\ea\label{ex:BascianoMelloni:29}
\ea\label{ex:BascianoMelloni:29a} 家家戶戶的門前都掛著青天白日滿地紅的國旗 {[}\ldots{}{]}\\
\gll \emph{jiā\tld{}jiā-hù\tld{}hù} \emph{de} \emph{mén-qián} \emph{dōu}  \emph{guà-zhe} \emph{qíng-tiān-bái-rì} \emph{mǎn-dì} \emph{hóng} \emph{de} \emph{guó-qí} \\
family\emph{\tld{}}family-household\emph{\tld{}}household \textsc{det} door-front all hang-\textsc{dur} blue-sky-white-sun full-ground red \textsc{det} country-flag\\
\glt `In front of the door of each household hung the red national flag with
 the white sun in the blue sky {[}\ldots{}{]}'

\ex\label{ex:BascianoMelloni:29b} 海水浴場裡,男男女女、老老少少,都穿著各種不同款式的泳裝 {[}\ldots{}{]}\\
\gll \emph{hǎi-shuǐ} \emph{yù-chǎng} \emph{lǐ}, \emph{nán\tld{}nán-nǚ\tld{}nǚ}, \emph{lǎo\tld{}lǎo-shào\tld{}shào}, \emph{dōu} \emph{chuān-zhe} \emph{gè} \emph{zhǒng} \emph{bùtóng} \emph{kuǎnshì} \emph{de} \emph{yǒng-zhuāng}\\
sea-water bath-site in man\emph{\tld{}}man-woman\emph{\tld{}}woman old\emph{\tld{}}old-young\emph{\tld{}}young all wear-\textsc{dur} each \textsc{clf}(kind) different style \textsc{det} swim-suit\\
\glt `Every man, woman, old and young bathing in the sea was wearing all different styles of swimming suits'
\z\z

In any case, it is possible to argue that this reduplication pattern
expresses a kind of \isi{plural} and, indeed, %
%Xu (2012a) 
\citet{Xu2012a} %
%Xu
%
argues that
reduplication, like \isi{plural} marking, is one of the major devices for
indicating plurality in human languages.\footnote{
%Xu (2012b: 48)
\citet[48]{Xu2012} %
%Xu
%
  highlights some general tendencies in the languages of the world: 1)
  languages with obligatory \isi{plural} marking tend not to have classifiers\is{classifier}
  %
%(see Greenderg 1972, Sanches 1973; but see e.g. Bisang 2012)
(see \citealt{Greenberg1972}, \citealt{SanchesSlobin1973}; but see e.g. \citealt{Bisang2012})% typo sur Greenberg, Sanches & Slobin
%Sanches
%
; 2)
  languages without obligatory \isi{plural} marking tend to use reduplication
  to express plurality. In general, languages which do not have \isi{plural}
  marking seem to appeal to both reduplication and classifiers\is{classifier}.} This
\isi{plural} displays interesting properties: it is compatible with
`numeral+classifier\is{classifier}' constructions (\ref{ex:BascianoMelloni:30a}) and, most importantly, it seems
to be compatible with the \isi{plural} marker 們 -\emph{men}\footnote{The
  \isi{plural} marker們 -\emph{men} can be added only to human nouns\is{noun}; it is
  entirely optional and is generally used ``only when there is some
  reason to emphasize the plurality of the noun\is{noun}'' (%
%Li \& Thompson, 1981
\citealt{Li1981}%
%
: 40). It is obligatorily used only with personal pronouns. Moreover, if
  the noun\is{noun} is preceded by a `numeral+\textsc{clf}', the marker 們
  \emph{--men} cannot be used: *三個老師們\emph{sān ge lǎoshī-men}
  `three \textsc{clf} teacher-\textsc{pl}, three teachers' (cf. \ref{ex:BascianoMelloni:30a}).
  This can be taken as an indication of the fact that 們 --\emph{men} is
  a marker of pluralization connected to the determiner/classifier\is{classifier}
  domain, rather than being involved at the NP level.} (\ref{ex:BascianoMelloni:30b}):

\ea\label{ex:BascianoMelloni:30}
\ea\label{ex:BascianoMelloni:30a} 200多個子子孫孫前來祝壽\\
\gll \emph{èrbǎi} \emph{duō} \emph{ge} \emph{zǐ\tld{}zǐ-sūn\tld{}sūn} \emph{qiánlái} \emph{zhù-shòu}\\
200 more \textsc{clf} son\emph{\tld{}}son\emph{-}grandson\emph{\tld{}}grandson come congratulate-longevity\\
\glt  `More than 200 children and grandchildren came to congratulate {[}the
old woman{]} on her birthday.'\footnote{\url{http://news.xinhuanet.com/society/2007-10/06/content\_6833517.htm}  [2016-11-24].}

\ex\label{ex:BascianoMelloni:30b} {[}\ldots{}{]}讓我們的子子孫孫們還能依靠這個地球生活。\\
\gll \emph{ràng} \emph{wǒ-men} \emph{de} \emph{zǐ\tld{}zǐ-sūn\tld{}sūn-men} \emph{hái} \emph{néng} \emph{yīkào} \emph{zhè} \emph{ge} \emph{dìqiú} \emph{shēnghuó}\\
let 1\textsc{sg-pl} \textsc{det} son\emph{\tld{}}son\emph{-}grandson\emph{\tld{}}grandson-\textsc{pl} still can rely this \textsc{clf} earth live\\
\glt `{[}\ldots{}{]} to let the future generations still be able to rely on
this earth to live.' \footnote{\url{http://www.china-coop.org/index.php?ac=article\&at=read\&did=854}   [2016-11-24].}

\z\z

\largerpage
From a typological perspective, it is interesting to observe that in
languages where reduplication and classifiers\is{classifier} are found extensively,
\isi{plural} marking is not well developed and is sensitive to the semantic
feature {[}+human{]} %
%(Xu 2012a: 12)
\citep[12]{Xu2012a}%
%Xu
%
, just like in Mandarin (see %
%Corbett 2000
\citealt{Corbett2000} %
%
 for a more comprehensive overview of number marking across
languages). %
%Xu (2012a) 
\citet{Xu2012a} %
%Xu
%
further remarks that the more \isi{plural} marking is
developed, the less this semantic feature ({[}+human{]}) is required;
also, the more a language possesses developed \isi{plural} markers, the less
it needs reduplication and classifiers\is{classifier}.

At the distributional level, the possible co-occurrence of AABB
reduplication and of the \isi{plural} marker 們 --\emph{men} suggests that
these two forms of pluralization cannot be equated, and, in a
syntactically oriented approach to word formation and inflection, it
indicates that these two plurals\is{plural} occupy different syntactic positions in
the (extended) nominal projection. In particular, following %
%Wiltschko's (2008)
\citepos{Wiltschko2008} %
%
 analysis of \isi{plural} markers in Halkomelem Salish, we will argue
that the reduplicative process is a derivational process that operates
at the \isi{root} level, even before \isi{root} categorization is determined. This
analysis allows us to explain the otherwise unexpected occurrence of 們
--\emph{men} \isi{plural} marking on AABB (animate) nouns\is{noun}, which could be
analysed as a modifier in the DP domain. We will go back to this issue
in section 4.

\subsection{Further remarks on the AABB pattern}
\is{compounding!co-compound|(}
To sum up, the data above show that increasing\is{reduplication!increasing (iconic)} AABB reduplication is
sensitive to the morphological makeup of its input, and insensitive to
the categorial feature of the base (Adj, V, N) or, semantically, to its
ontological/sortal type (whether the base denotes a quality, an event,
or an entity/individual). As for the morphological restrictions on the
base units, it is worthwhile noting that the requirement of a \is{compounding}compound
base of a specific type is also category-neutral, since it is found with
AABB adjectives\is{adjective}, verbs\is{verb} and nouns\is{noun}. In particular, the kind of \isi{root}
combinations we find seem to have much in common with `co-compounds', in
particular, with the following categories singled out by %
%Wälchli (2005:
%138)
\citet[138]{Waelchli2005}%
%Wälchli
%
: `additive co-compounds', as e.g. Georgian \emph{xel-p'exi}
`hand-foot'; `generalizing co-compounds', as e.g. Mordvin
\emph{t'ese-toso} `here-there, everywhere'; collective co-compounds, as
e.g. Chuvash \emph{sĕt-śu} `milk-butter, dairy products'; synonymic
co-compounds, as e.g. Uzbek \emph{qadr-qimmat} `value-dignity, dignity'.

According to Wälchli, additive co-compounds denote pairs consisting of
the parts A and B; in a broader sense, they denote sets exhaustively
listed by A and B. Generalizing co-compounds denote general notions (as
e.g. `all', `always'); their parts express the extreme opposite poles of
which the whole consists. As for collective co-compounds, they are not
always easy to define since they obey to different criteria, which do
not always agree: the parts do not exhaustively list the whole; the
whole comprises all meanings having the properties shared by A and B;
collective co-compounds are co-compounds which denote
collectives.\footnote{The example from Chuvash reported above meets all
  the three criteria, but it is not always the case. It is difficult to
  distinguish between additive and collective co-compounds if the first
  two criteria do not apply at the same time.} Finally, in synonymic
co-compounds, the constituents (A and B) and the whole \is{compounding}compound have
(almost) the same meaning. Wälchli observes that synonymic co-compounds
``express homogeneous collection complexes in which (ideally) every
element contained in them can be referred to by both parts of the
co-compound'' (p. 140). This, according to Wälchli, explains the
affinity between synonymic co-compounds and plurality, though there is
no language in which synonymic compounds\is{compounding} work as fully grammaticalized
plurals\is{plural}. Synonymic co-compounds may have affinities either to
collective, to additive or to generalizing co-compounds. In any case,
each type of co-compound described above may be considered as complexes
where the referents are joint together to indicate a `set'.

\largerpage
Interestingly enough, the AABB pattern can apply to AB bases that are
not attested as coordinated bases (see sections 3.2, 3.3), and crucially
it can be `category-changing' %
%(see Paul 2010: 145-146; cf. also ex. (\ref{ex:BascianoMelloni:9})
\citep[see][145-146; cf. also ex. (\ref{ex:BascianoMelloni:9}]{Paul2010a}%
%Paul
%
):

\ea\label{ex:BascianoMelloni:31}
  \begin{tabular}[t]{ll@{\hspace*{.1\linewidth}}l}
  {婆婆媽媽} & $\rightarrow$ & {[}AABB{]} = Adj\\
  \begin{minipage}[t]{.5\linewidth}
  \gll \emph{pó\tld{}po-mā\tld{}mā }\\ 
        old.lady\emph{\tld{}}old.lady-mother\tld{}mother\\ 
  \glt `kindhearted/sentimental/effeminate'
  \end{minipage}
  & &
  \end{tabular}

\z

In (\ref{ex:BascianoMelloni:31}), the AB base is not an existing word, but AABB reduplication
applies to two free/non-conjoined lexical roots\is{root}. Reduplication of two
elements independently compatible with a nominal meaning\footnote{It is
  worth noticing that when the base is formed by a bound \isi{root}
  constituent, like 婆\emph{pó} `old.lady' in (\ref{ex:BascianoMelloni:31}), we cannot determine
  its lexical category since bound roots\is{root} do not occupy syntactic slots
  (see section 1.2); rather, it can be said that these roots\is{root} are
  `noun\is{noun}-like' \emph{semantically,} i.e. they denote entities/individuals
  (see section 3.5).} results in an \emph{adjectival} AABB \isi{lexeme}.

Furthermore, the AABB pattern extends to others categories too, like
numerals, place words, coordinated classifiers\is{classifier}, onomatopoeias, etc. %
%(see
%Hu 1995)
\citep[see][]{Hu95}%
%Hu
%
:
%
\ea\label{ex:BascianoMelloni:32}
\ea\label{ex:BascianoMelloni:32a} 千千萬萬\\
\gll \emph{qiān\tld{}qiān-wàn\tld{}wàn }\\
thousand\emph{\tld{}}thousand-ten.thousand\tld{}ten.thousand\\
\glt `thousands and thousands'\\
\ex\label{ex:BascianoMelloni:32b} 前前後後\\
\gll \emph{qián\tld{}qián-hòu\tld{}hòu }\\
front\tld{}front-back\tld{}back\\
\glt `whole story/ins and outs'
\ex\label{ex:BascianoMelloni:32c} 嘻嘻哈哈\\
\glll \emph{xī}\emph{\tld{}xī-hā}\emph{\tld{}hā} \\
giggling.onomatopoeia\tld{}giggling.onomatopeia-\\
laughter.onomatopoeia\tld{}laughter.onomatopeia\\
\glt `laughing and joking'\\
\z\z\marginpar{Sacha: gloses too long to keep it all on one line.}

\largerpage
All these facts seem to support the hypothesis that the AABB
reduplication pattern applies even before the conjoined bases get their
categories (and indeed the constituents can be bound roots\is{root}
too).\footnote{A reviewer observed that it is difficult to make such a
  claim if the cases mentioned in this section are well-established
  lexicalized formations. Actually, these cases seem to be quite
  marginal, and for category changing items it is quite expected, since
  intuitively we expect that reduplication of two roots\is{root} compatible with
  the nominal meaning leads to a nominal output. However, these examples
  further highlight the cross-categoriality of the pattern and further
  support the hypothesis of the acategoriality of the base roots\is{root}. In any
  case, it is undoubtable that bound roots\is{root} can enter this pattern of
  reduplication (see e.g. the reduplicated word in the examples (\ref{ex:BascianoMelloni:30})
  above, where both roots\is{root} are bound), which as mentioned above (footnote~\ref{ex:BascianoMelloni:26};
  see also section~3.5) do not have a lexical category, and this points toward
  the acategorical nature of the conjoined roots\is{root}.} This is consistent with an
analysis according to which word formation can apply to roots\is{root}, or in
this specific case, to combination/coordination of category-less roots\is{root},
which would explain why, different from ABAB diminishing\is{reduplication!diminishing (countericonic)} reduplication,
it is a phenomenon found across almost all word classes.\footnote{Reduplication
  of non-existent AB bases is not possible with diminishing\is{reduplication!diminishing (countericonic)} verbal
  reduplication; in ABAB verbal reduplication, the AB base must be an
  existing disyllabic \isi{verb}.} We will go back to this in section 4,
where we will put forth an analysis for this reduplication pattern.
\is{compounding!co-compound|)}

\subsection{On the base units of AABB reduplications}

As we have seen in 2.2, diminishing\is{reduplication!diminishing (countericonic)} reduplication does not form
syntactic atoms and can be analyzed as a syntactic operation whose
application is conditioned by structural restrictions in the \emph{v}P
domain %
%(see Arcodia, Basciano \& Melloni 2014; Basciano \& Melloni
%2017)
\citep[see][]{ArcodiaBascianoEtAl2014,BascianoMelloni2017}%
%Arcodia-Basciano-Melloni;?Basciano-Melloni
%
. In contrast, we have shown that increasing\is{reduplication!increasing (iconic)} reduplication is
subject to `morphological' restrictions. Keeping in line with previous
research on reduplication and \isi{plural} marking, we argue that AABB
increasing\is{reduplication!increasing (iconic)} reduplication is the result of the modification of roots\is{root} (see
section 4), understood here, as in most exoskeletal approaches %
%(see
%Borer 2003)
\citep[see][]{Borer2003}%
%Borer
%
, like elements crucially lacking category features.
Moreover, as we will show in details in the next section, AABB
reduplications are syntactic atoms which cannot allow for the insertion
of other material between the iterated units %
%(see e.g. Lapointe 1980)
\citep[see e.g.][]{Lapointe1980}%
%Lapointe
%
.

Different pieces of evidence speak in favour of the hypothesis that AABB
reduplication applies to elements smaller than a word, i.e. a \isi{root}/\isi{stem},
and possibly lack \emph{per se} a definite category specification. In
what follows, we will concentrate on the differences between
AABB/increasing\is{reduplication!increasing (iconic)} reduplication and other reduplicative processes to
illustrate our point.

First of all, let us consider the verbal domain, where we find both
diminishing\is{reduplication!diminishing (countericonic)} reduplication and increasing\is{reduplication!increasing (iconic)} reduplication. A first crucial
difference between the two patterns, namely ABAB and AABB verbs\is{verb},
concerns the distribution of aspectual\is{aspect} markers. With AABB reduplicated
verbs\is{verb}, if an aspectual\is{aspect} marker is present, it follows the whole
reduplicated \isi{verb} (\ref{ex:BascianoMelloni:33a}), as in the case of resultatives and other kinds
of \is{compounding}compound verbs\is{compounding!compound verb} (cf. fn. 4). In diminishing\is{reduplication!diminishing (countericonic)} reduplication, as we have
seen, the aspectual\is{aspect} marker 了 \emph{le} is unexpectedly placed between
the base and the reduplicant (\ref{ex:BascianoMelloni:33b}):

\ea\label{ex:BascianoMelloni:33}
\ea\label{ex:BascianoMelloni:33a} 連老郭都進進出出了好幾次。\\
\gll  \emph{lián} \emph{lǎo-Guō} \emph{dōu} \emph{jìn}\tld{}\emph{jìn}-\emph{chū}\tld{}\emph{chū-le} \emph{hǎojǐ} \emph{cì}\\
even old-Guo all enter\tld{}enter-exit\tld{}exit-\textsc{pfv} many time\\
\glt `Even old Guo entered and exited from there many times.'\footnote{\url{http://www.cctv.com/program/zoujinkexue/topic/science/C15580/20060413/100489.shtml} [2016-11-24].}\\
\ex\label{ex:BascianoMelloni:33b} 她試了試那件衣服。\\
\gll \emph{tā} \emph{shì-le} \emph{shì} \emph{nà} \emph{jiàn} \emph{yīfu }\\
3\textsc{sg.f} try-\textsc{pfv} try that \textsc{clf} dress\\
\glt `She tried on that dress.'\\
\z\z

A second piece of evidence comes from `rhotacization' or \emph{erhua}
(兒化 \emph{érhuà}), a mor\-pho-phonological phenomenon that is very
common in the speech varieties of Northern China, consisting in the
addition of a retroflex approximant (兒 \emph{-r}) at the end of a word.
More precisely, phonologically, this suffix incorporates into the final
syllable of a host \isi{stem} replacing an existing coda, as e.g. 公園
\emph{gōngyuán} → 公園兒\emph{gōngyuár} `park', 鳥 \emph{niǎo} → 鳥兒
\emph{niǎor} `bird'. The suffix兒 \emph{-r} can appear in reduplicated
adjectives\is{adjective}, and in the AABB pattern it occurs after the whole
reduplicated adjective\is{adjective}:

\vspace{\topsep}\ea\label{ex:BascianoMelloni:34}高高興興兒\\
\emph{gāo\tld{}gāo-xìng\tld{}xìng-r}\\
\glt `really happy'\\
\z


%
%Lee-Kim (2016) 
\citet{Lee-Kim2016} %
%Lee-Kim
%
observes that, even if to a lesser extent, this suffix can
be also found in the reduplication of modifier-head adjectives\is{adjective} (see
3.1). However, in this case the suffix attaches after each AB, i.e.
AB-\emph{r} AB-\emph{r}:

\ea\label{ex:BascianoMelloni:35}雪白兒雪白兒\\
\emph{xuě-bái-r\tld{}xuě-bái-r}\\
\glt `(very) snow-white'
\z

According to %
%Lee-Kim (2016)
\citet{Lee-Kim2016}%
%Lee-Kim
%
, this difference between the AABB pattern
and the ABAB pattern, as far as the suffix 兒 \emph{--r} is concerned,
suggests that these two types of reduplication have a distinct internal
structure. Assuming that 兒 \emph{--r} adjoins to a phrasal node that
introduces categorial information (\emph{n, v, a} in DM), since it
consistently occurs at the end of a full-fledged category, Lee-Kim
argues that the contrast between (\ref{ex:BascianoMelloni:34}) and (\ref{ex:BascianoMelloni:35}) indicates that each AB
forms an adjective\is{adjective} phrase in the adjectival ABAB pattern of
reduplication, while AABB as a whole forms a single adjectival phrase.
She further argues that modifier-head compounds\is{compounding} would undergo
\emph{erhua} before reduplication ({[}AB-\emph{r}{]}-RED), while
coordinate compounds\is{compounding} reduplicate before 兒 \emph{--r} adjoins (
{[}AB-RED{]}-\emph{r}). Since in the ABAB pattern 兒 \emph{--r} adjoins
\emph{before} reduplication, the double occurrence of this suffix
(AB-\emph{r} AB-\emph{r}) elegantly follows: reduplication applies to
the whole suffixed \is{compounding}compound AB-\emph{r}, copying it as a whole.
According to Lee-Kim, this also suggests that reduplication of
modifier-head compounds\is{compounding} is phrasal, while reduplication of coordinate
compounds\is{compounding} targets units smaller than a phrase. A corollary of this
analysis might be that reduplication applies both to units below and
above X°, but under this view it would be difficult to explain that
there are no constraints on the gradability of the base, in the case of
ABAB adjectival reduplication.

An alternative and more feasible hypothesis is that the ABAB pattern
instantiates another kind of phenomenon, which is well attested across
languages (even those ones that lack productive reduplication),
\emph{viz}. \emph{contrastive focus} reduplication/repetition. Different
from `morphological' reduplication, contrastive repetition phenomena
involve the copying of full fledge words and sometimes phrases, as in
the following examples from %
%Gomeshi \emph{et al.} (2004: 308)
\citet[308]{GomeshiEtAl2004}%
%
, and
typically have no phonological/tone reanalysis or other types of
morpho-phonological readjustment phenomena that characterize
reduplication in a cross-linguistic perspective:

\ea\label{ex:BascianoMelloni:36}
\ea\label{ex:BascianoMelloni:36a} I'll make the tuna salad, and you make the SALAD--salad.
\ex\label{ex:BascianoMelloni:36b} My car isn't MINE--mine; it's my parents'.
\ex\label{ex:BascianoMelloni:36c} Oh, we're not LIVING-TOGETHER--living-together.
\z\z


The semantic effect of this construction is, according to %
%Gomeshi \emph{et al.}
\citeauthor{GomeshiEtAl2004}%
%
, ``to focus the denotation of the reduplicated element on
a more sharply delimited, more specialized, range'' (p. 308). For
example, in (\ref{ex:BascianoMelloni:36a}) \emph{SALAD-salad} denotes green salads as opposed to
salads in general.

Although the interpretive difference between increasing\is{reduplication!increasing (iconic)} reduplication
and contrastive repetition is difficult to get from our
Mandarin-speaking informants, we suggest that reduplicated adjectives\is{adjective}
such as 雪白雪白 \emph{xuě-bái\tld{}xuě-bái}
`snow-white\emph{\tld{}}snow-white' might have a similar
semantic effect, which is to express a prototypical, standard property
denotation in the adjectival domain. As such, ABAB would be a different
phenomenon applying at the phrasal level and crucially lacking the
morphological constraints found with increasing\is{reduplication!increasing (iconic)} reduplication. In
contrast, the AABB pattern operates below the X° level and affects the
gradable property of the base, i.e. it turns a gradable base into a no
longer gradable one (see section 3.1).

A further element which seems to support the status of the AABB
reduplicated forms as syntactically atomic units\footnote{Whether they
  are category-less roots\is{root}/stems\is{stem} or standard lexemes\is{lexeme} endowed with
  category features will be discussed throughout section 4.} is that
they are often formed by at least one bound \isi{root} (either A or B, or both
of them) which cannot stand as a syntactic word by itself (see section
3.4, ex. (\ref{ex:BascianoMelloni:31}) and fn. 26 and 27). For instance, in the example (\ref{ex:BascianoMelloni:37}) the AB
base is formed by two bound roots\is{root} (cf. the free forms兒子 \emph{érzi}
`son' and 孫子\emph{sūnzi} `grandson'):

\ea\label{ex:BascianoMelloni:37}
\sbsglll{子孫}{子子孫孫}%
        {\emph{zǐ-sūn}}{\emph{zǐ\tld{}zǐ-sūn\tld{}sūn}}%
        {son-grandson}{son\emph{\tld{}}son\emph{-}grandson\emph{\tld{}}grandson}%
        {`\begin{tabular}[t]{@{}l@{}}children and\\ grandchildren'/\\ `descendants'\end{tabular}}{`\begin{tabular}[t]{@{}l@{}}heirs'/`generation after generation\\ of descendants'\end{tabular}}%
\z


This further corroborates the hypothesis that this process applies to
roots\is{root}, thus to acategorial elements; bound roots\is{root}, indeed, have `nouny',
`verby', `adjective\is{adjective}-like', etc. features, but, since they are not able
to occupy a syntactic slot by themselves, they do not have a syntactic
category proper.

\section{Analysis}

Given the properties illustrated thus far, in this section we will
propose that AABB reduplication is a phenomenon applying at the \isi{root}
level, as we briefly mentioned in section 3.5. In particular, in the
previous sections we have shown that the AABB pattern applies across
categories and even to non-attested AB units, can be `category changing'
(e.g. a coordination of two noun\is{noun}-like roots\is{root} may result in an adjective\is{adjective}),
can be formed by bound roots\is{root}, and displays syntactic atomicity/lexical
integrity.

We thus propose, along the line of %
%Wiltschko (2008) 
\citet{Wiltschko2008} %
%Wiltschko
%
and %
%Zhang (2015)
\citet{Zhang2015}%
%Zhang
%
,
that AABB reduplication constitutes a modification/adjunction process
which targets category-less roots\is{root}.

\subsection{Reduplication of (\is{compounding}compound) roots\is{root}}

Over the last two decades, frameworks of word formation, especially
Distributed Morphology or %
%Borer's 
\citeauthor{Borer2003}'s %
exoskeletal framework (%
%2003
\citeyear{Borer2003}%
%
), have
taken very seriously the hypothesis that roots\is{root}, as the invariant core of
full-fledged words (stripped away of all morphological formatives) are
category-less elements, and that they must be combined in the syntax
with category assigning heads %
%(see among others Marantz 2001; Embick and
%Marantz 2006; Embick and Noyer 2007)
\citep[see among others][]{Marantz2001,Embick2007,Embick2008}% EM 06 ms => EM 08 => article
%Marantz;?Embick-Marantz;Embick-Noyer
%
. Under this view, lexemes\is{lexeme}/words
never are atomic entities, but are the spell-out forms of roots\is{root} selected
by a functional head, i.e. \emph{a, n, v}, determining the corresponding
phrasal domain, so that: N = {[}\emph{n} + √{]}, V = {[} \emph{v} +
√{]}, A = {[}\emph{a} + √{]}.

Adopting this approach to word formation and its compositional analysis
of lexemes\is{lexeme}, a possibility allowed by the system is that morphological
phenomena traditionally described as `derivational' do not actually
target lexemes\is{lexeme} proper but category-less items, i.e. category-less roots\is{root}.
Increasing\is{reduplication!increasing (iconic)} reduplication in Mandarin would then fall within the realm of
those phenomena that apply at a very `low' level in the morphosyntactic
derivation, namely before categorization takes place. Leaving aside for
the moment the complicating factor that the base of increasing\is{reduplication!increasing (iconic)}
reduplication is not a single \isi{root} but a \is{compounding}compound form made up of two
roots\is{root} (see section 4.4 for further discussion on this), under this
analysis, it naturally follows that the whole reduplicated AABB form can
be assigned to different lexical categories, in accordance with the
ontological (/sortal) specification of the \isi{root}, i.e. whether it denotes
objects, events, or (gradable) qualities/attributes.

 \ea\label{ex:BascianoMelloni:38}
 \ea\label{ex:BascianoMelloni:38a} \Tree[.\emph{n}P [.\emph{n} ] [.√root RED √root ] ]
 \ex\label{ex:BascianoMelloni:38b} \Tree[.\emph{v}P [.\emph{v} ] [.√root RED √root ] ]
 \ex\label{ex:BascianoMelloni:38c} \Tree[.\emph{a}P [.\emph{a} ] [.√root RED √root ] ]
\z\z


In (\ref{ex:BascianoMelloni:38}) we limited our representation to nouns\is{noun}, verbs\is{verb} and adjectives\is{adjective},
but the analysis can be in principle extended to other categories too,
like adverbs. The assumption that roots\is{root} are atomic, non-decomposable
elements virtually independent of the traditional lexical categories
%
%(i.e. roots are not associated with categorial information, as e.g.
%nouns, verb, adjectives; see Marantz 1997) 
\citep[i.e. roots\is{root} are not associated with categorial information, as e.g. nouns\is{noun}, verbs\is{verb}, adjectives\is{adjective}; see][]{Marantz1997} %
%Marantz
%
allows for a unified analysis
of AABB reduplication across categories. Under this approach,
reduplication involves acategorial items, and categorization is
determined afterwards, in accordance with the type of
category-determining heads, i.e. \emph{n}, \emph{v}, \emph{a}, and under
the assumption that ``whatever category can select for roots\is{root} can also
select for pluralized roots\is{root}, because pluralized roots\is{root} are still roots\is{root}''
%
%(see Wiltschko 2008: 60)
\citep[see][60]{Wiltschko2008}%
%Wiltschko
%
.

While we argue, along the line of %
%Wiltschko (2008) 
\citet{Wiltschko2008} %
%Wiltschko
%
and %
%Zhang (2015)
\citet{Zhang2015}%
%Zhang
%
,
that a single structural analysis is capable to explain for all the
category patterns of increasing\is{reduplication!increasing (iconic)} reduplication, the interpretive outcomes
of reduplication are still in need of a satisfactory analysis in the
literature.

As can be observed in other languages too, reduplication of nouns\is{noun} and
verbs\is{verb} results is a (lexical) means of pluralization. The existence of
lexical plurals\is{plural}, in particular, in the nominal domain is well attested
across languages, with Italian, for instance, having a class of
(feminine) nouns\is{noun} that are lexically specified as being \isi{plural} %
%(e.g.
%\emph{braccia} `arms', see Acquaviva 2008)
\citep[e.g. \emph{braccia} `arms', see][]{Acquaviva2008}%
%Acquaviva
%
. As for the \ili{Chinese} cases
under consideration, according to %
%Zhang (2015)
\citet{Zhang2015}%
%Zhang
%
, AABB reduplication
expresses overall a `greater \isi{plural}' meaning, which can apply both to
individual-denoting and to action-denoting elements. In particular, this
\isi{plural} marker, according to Zhang, is integrated in the word-formation
domain, where instead of categorial features, semantic features %
%(see
%Cinque 1990: 1; Lieber 2004; 2006) 
(see \citealt{Cinque1990}, \citealt{Lieber2004}, \citealt{Lieber2006}) %
%Cinque
%
and probably phonological features,
take part in the selection.

Zhang's analysis relies much on %
%Wiltschko's (2008) 
\citepos{Wiltschko2008} %
%Wiltschko
%
analysis of
pluralization in Halko\-melem Salish. Wiltschko proposes, based on
different distributional properties, that in a language like English,
with obligatory \isi{plural} marking, and in a language like Halkome\-lem, with optional
\isi{plural} marking, \isi{plural} markers differ in their `way'
and place of merging. While in English, as it is generally assumed, the
\isi{plural} marker spells out the \isi{plural} value of a functional head selective
for a phrasal node such as little \emph{n}, in Halkomelem \isi{plural} marking
functions as a modifier of the category-less \isi{root}:

\ea\label{ex:BascianoMelloni:39}
\ea\label{ex:BascianoMelloni:39a} 
\begin{tabular}[t]{l}
English\\
\Tree[.D D [.{\#:PL} {\#:PL} [.\emph{n} \emph{n} √root ] ] ]
\end{tabular}
\ex\label{ex:BascianoMelloni:39b}
\begin{tabular}[t]{l}
 Halkomelem\\
\Tree[.D D [.\emph{n} \emph{n} [.√root PLURALIZER √root ] ] ]
\end{tabular}
\z\z


According to %
%Wiltschko (2008: 688)
\citet[688]{Wiltschko2008}%
%Wiltschko
%
, modifying \isi{plural} markers (\ref{ex:BascianoMelloni:39b}) have
the syntax of adjuncts, rather than of selecting heads, because of a set
of properties setting them aside from functional plurals\is{plural}: they are not
obligatory; they do not trigger agreement; their absence is not
associated with a specific meaning, but instead is truly unmarked; they
cannot be selected for; they do not allow for form-meaning mismatches.

We argue that the \isi{root}-adjoined analysis in (\ref{ex:BascianoMelloni:39b}) can be the correct
analysis for the Mandarin AABB reduplication under examination, where
the `pluralizer' is expressed by means of the reduplicative pattern
itself, i.e. by means of independent phonological copying of both base
units.\footnote{The intriguing issue of the peculiar phonological
  exponence of disyllabic increasing\is{reduplication!increasing (iconic)} reduplication is left for future
  investigation, but we refer to %
%Feng (2003) 
\citet{Feng2003} %
%Feng
%
for an interesting analysis
  within Optimality Theory framework. See section 4.4. for further
  remarks on this.} This explains for several peculiar features of AABB
reduplication, such as its non-obligatoriness and cross-categoriality,
as well as its compatibility with the \isi{plural} marker 們 \emph{--men},
possibly used to emphasize plurality (see fn. 22), and with nominal
classifiers\is{classifier}. In particular, as we have noticed in section 3.3~(\ref{ex:BascianoMelloni:30b}),
reduplication and pluralization are not incompatible:

\ea\label{ex:BascianoMelloni:40}子子孫孫們 (extracted from ex. (\ref{ex:BascianoMelloni:30b}))\\
\gll  \emph{zǐ\tld{}zǐ-sūn\tld{}sūn-men }\\
son\emph{\tld{}}son\emph{-}grandson\emph{\tld{}}grandson-\textsc{pl}\\
\glt `heirs/generation after generation of descendants'
\z


Furthermore, the \isi{plural} meaning of increasing\is{reduplication!increasing (iconic)} reduplication is not
merely `\isi{plural}': since it applies to a coordination of
entities/individuals which are \emph{per se} inherently \isi{plural} (AB means
the sum of the entities/individuals denoted by A and those denoted by B,
see section 3.4), its meaning is that of `excessive/greater \isi{plural}'.

Another striking feature shared by Halkomelem Salish and Mandarin lies
in the fact that their `lexical' \isi{plural} marking is not restricted to
nouns\is{noun}, different from inflectional \isi{plural} marking which is typically
bound to nominal lexemes\is{lexeme} (not counting agreement \isi{plural} marking, which
can occur wherever it is required). This leads us to discuss the other
lexical categories of the outputs of these reduplicative processes.

As for the verbal domain, pluractional meaning of reduplicated verbs\is{verb} is
certainly not exceptional in a cross-linguistic perspective. A great
deal of reduplicative processes across languages show a pattern close to
Mandarin, where (increasing\is{reduplication!increasing (iconic)}) reduplication in the verbal domain implies
repetition/iteration of the event expressed by the base, hence
operating over the \isi{verb} aspectual\is{aspect} structure. This means that increasing\is{reduplication!increasing (iconic)}
reduplication has an inherent quantificational meaning, resulting in a
plurality of individuals or in a pluractionality\is{aspect!pluractionality} of events, in
compliance with the (vague) \isi{root} meaning, ultimately determined by the
type of selecting head, \emph{n} vs. \emph{v}, taking the reduplication
as its complement (see (\ref{ex:BascianoMelloni:38})). Another property in common with nouns\is{noun} and,
to the best of our knowledge, specific of Mandarin Chinese, is the need
for a base composed of coordinated roots\is{root} (especially in the case of
verbs\is{verb}), standing in a symmetrical relation. We will come back to this
intriguing issue in section 4.3.

\subsection{Zooming in on adjectives\is{adjective}}

Whereas the \isi{plural} analysis seems to nicely fit the nominal and verbal
domains of AABB reduplication, it remains to be understood what the
interpretive analysis of adjective\is{adjective} reduplication is. Interestingly,
%
%Wiltschko (2008) 
\citet{Wiltschko2008} %
%Wiltschko
%
observes that in Halkomelem Salish the pluralizer (be
it an affix, ablaut or a reduplicated form) occurs productively not only
with nouns\is{noun} (\ref{ex:BascianoMelloni:41a}, \ref{ex:BascianoMelloni:41b}), but with verbs\is{verb} (\ref{ex:BascianoMelloni:41c}) and adjectives\is{adjective} (\ref{ex:BascianoMelloni:41d}) too
%
%(Wiltschko 2008: 641, 679)
\citep[641, 679-680]{Wiltschko2008}%
%Wiltschko
%
, conveying a meaning close to the one we
find in Mandarin AABB reduplication:\footnote{The reader should note that the unmarked form, here
      glossed as a singular form, is in fact compatible with both singular
      and \isi{plural} interpretation; as we have mentioned, the \isi{plural} marker is
      not obligatory in Halkomelem.}

\ea\label{ex:BascianoMelloni:41}
\ea\label{ex:BascianoMelloni:41a}
  \glll\emph{méle}~~~~~  \emph{mámele}\\
   child  child.\textsc{pl}\\
   `child' `children'\\
\ex\label{ex:BascianoMelloni:41b}
\glll  \emph{q'ámi}~~~~~  \emph{q'álemi}\\
   girl  girl.\textsc{pl}\\
   `girl' `girls'\\
\ex\label{ex:BascianoMelloni:41c}
  \glll \emph{qw'óqw-et}~~~~~  \emph{qw'óleqw-et}\\
   whip-\textsc{trans}  whip.\textsc{pl-trans}\\
    {`whip something/someone'} {`whip something/someone several times'}\\
\ex\label{ex:BascianoMelloni:41d}
\glll  \emph{kw'ós}~~~~~  \emph{kw'ó-kw'es}\\
   hot  hot.\textsc{pl}\\
   `hot' {`real hot/very hot'}\\
\z\z


%
%Wiltschko (2008) 
\citet{Wiltschko2008} %
%Wiltschko
%
argues that, no matter whether it occurs in the context
of nouns\is{noun}, verbs\is{verb} or adjectives\is{adjective}, the \isi{plural} marker is exactly the same.
She further observes that, if the \isi{plural} marker is exactly the same, we
expect it having exactly the same meaning in each of these contexts.
However, to determine what a \isi{root} pluralizer denotes, we need to know
what a \isi{root} denotes, i.e. what its sortal type is. Wiltschko thus
speculates that roots\is{root} do not have a specific denotation (vs. nouns\is{noun},
which denote individualities, verbs\is{verb}, which denote eventualities, or
adjectives\is{adjective}, which denote attributes/qualities); they are able to name
``Events, Things, States and Qualities %
%(see Harley 2005)
\citep[see][]{Harley2005}%
%Harley
%
, and the
pluralizer appears to simply assert that there are a lot of Events,
Things, States, Qualities, depending on the nature of the √root'' (p.
686).

While this intuitive explanation in principle could work for nouns\is{noun} and
verbs\is{verb}, it is nonetheless far less accurate for depicting the increased
semantics of reduplicated adjectives\is{adjective}. Looking at the semantic effects
that reduplication has on Mandarin adjectives\is{adjective}, it does not seem the case
that it denotes `lots of Qualities'. Rather, it seems that AABB
adjectives\is{adjective} express `increased intensity', thus affecting the gradable
property of the base, and this seems to be true also for many other
languages that exhibit reduplication with increasing\is{reduplication!increasing (iconic)} semantics (with
Halkomelem \emph{pluralized} adjectives\is{adjective} not counting as an exception in
this domain, see~(\ref{ex:BascianoMelloni:41d})).\footnote{According to %
%Xu (2012a)
\citet[]{Xu2012a}%
%Xu
%
, reduplication is
  iconically\is{reduplication!increasing (iconic)} motivated, and `positive degree' constitutes its core
  meaning.} Since reduplication affects gradability, providing a
greater/increased degree value expressed by the base \isi{root}, we might ask
what the interpretive relation is between increasing\is{reduplication!increasing (iconic)} reduplication in
the adjectival domain, on the one hand, and increasing\is{reduplication!increasing (iconic)} reduplication in
the verbal and nominal domain on the other, where reduplication is a
means of quantification over entities/individuals and events.

\subsection{%
%Wellwood's (2014, 2015) 
\citeauthor{Wellwood2014}'s (\citeyear{Wellwood2014,Wellwood2015}) %
%\citet['s][]{Wellwood2014,Wellwood2015} %
%Wellwood;Wellwood
%
analysis of measurement functions
across categories }

The analysis of adjectives\is{adjective}, especially the fact that only gradable
adjectives\is{adjective} can be reduplicated, sheds light on the core issue of
gradability/scalarity in increasing\is{reduplication!increasing (iconic)} reduplication. However, as we
mentioned in the previous section, the relation between increasing\is{reduplication!increasing (iconic)}
reduplication in the adjectival domain and increasing\is{reduplication!increasing (iconic)} reduplication in
the verbal and nominal domain still remains to be explained. In this
section, based on the existing literature, we show that concepts of
gradability and measurement, rather than being limited to the adjectival
domain, may be applied uniformly across categories. This will help to
support our hypothesis on the function of Mandarin increasing\is{reduplication!increasing (iconic)}
reduplication, namely that it expresses a unique function, i.e.
`increased measure', as will be discussed in the next section.

While according to some authors gradability is a distinctive property of
adjectives\is{adjective} %
%(see e.g. Jackendoff 1977)
\citep[see e.g.][]{Jackendoff1977}%
%Jackendoff
%
, a great deal of research over the
last decades found evidence of gradable properties across lexical
categories %
%(see e.g. Bolinger 1972; Bresnan 1973; Dotjes 1997; Neeleman \emph{et al.} 2004; Caudal \& Nicolas 2005; Bochnak 2010)
\citep[see e.g.][]{Bolinger1972,Bresnan1973,Doetjes1997a,NeelemanVandeKootEtAl2004,CaudalNicolas2005,Bochnak2010}%
%Bolinger;Bresnan
%
. As observed
by %
%Nicolas (2010)
\citet{Nicolas2010}%
%Nicolas
%
, gradable expressions are found among: \isi{plural} count
nouns\is{noun} (\emph{more dogs}), but not singular count nouns\is{noun} (*\emph{more dog,
*less cup}); mass nouns\is{noun}, concrete (\emph{more water, less wine}) or
abstract (\emph{more sadness}, \emph{less playfulness}); adjectives\is{adjective}
(\emph{smaller}, \emph{less sad}); verbs\is{verb} (\emph{to work more/less}).

%
%Wellwood (2015) 
\citet{Wellwood2015} %
%Wellwood
%
puts forward a unified account of comparison across
categories, challenging those theories that consider gradable adjectives\is{adjective}
as elements specifying measure functions (see above) vs. nouns\is{noun} and
verbs\is{verb}, which allegedly do not express such measure functions. According
to this scholar, ``which dimensions are possible across domains is a
consequence of what is measured, rather than which expressions measure''
(p.~69). %
%Wellwood (2015: 69) 
\citet[69]{Wellwood2015} %
%Wellwood
%
also observes that a noun\is{noun} like \emph{coffee}
introduces individuals that can be measured, while a \isi{verb} like
\emph{run} introduces events and an adjective\is{adjective} like \emph{tall}
introduces states; in any case, they all can be measured along certain
types of dimensions, specifically those which respect `part-whole'
relation (e.g. volume and weight for \emph{soup}, but not temperature;
time and distance for \emph{run}, but not speed\footnote{For example,
  she observes that larger portions of soup have greater measures by
  volume or weight than smaller portions, but generally this is not the
  case with measures by temperature.}). She posits a variable in nominal
and verbal domains ``that ranges over measure functions, restricted to
just those that are homomorphic to the measured domain'' (p. 68).
%
%Wellwood (2014, 2015) 
\citet{Wellwood2014,Wellwood2015} %
%Wellwood;Wellwood
%
argues that comparative sentences in the
adjectival, nominal and verbal domain all contain instances of a single
(phonologically overt or covert) morpheme that compositionally
introduces degrees; ``this morpheme, sometimes pronounced \emph{much},
contributes a structure-preserving map from entities, events, or states,
to their measures along some dimension.'' %
%(Wellwood 2015: 67)
\citep[67]{Wellwood2015}%
%Wellwood
%
.

This approach characterizes the notion of  ``measurement'' uniformly in terms
of struc\-ture-preservation across comparative constructions and unifies
the contrasts existing (within each category) between gradable and
non-gradable adjectives\is{adjective}, between mass and count nouns\is{noun}, and between
atelic and telic \isi{verb} phrases.\footnote{Gradability presupposes the
  existence of a scale, and can be seen as related to ±boundedness %
%(see
%  Paradis 2001, Alexiadou 2010)
\citep[see][]{Paradis2001,Alexiadou2010}%
%Paradis;?Alexiadou
%
.} Wellwood observes that mass nouns\is{noun}
tend to show cumulative reference: ``if \emph{coffee} applies to two
portions of matter, then it also applies to the mereological sum of
those portions'' (p. 71). In contrast, count nouns\is{noun}, when interpreted
singularly, tend to show non-cumulative reference: ``if a \emph{cup}
applies to a given object, it fails to apply to any of its (relevant)
proper parts'' (p. 71). Therefore, the semantics of mass nouns\is{noun} is
modelled in terms of a domain structured by the part-of relation, while
that of a noun\is{noun} like \emph{cup} lacks such structure. Similarly, atelic
predicates (like mass nouns\is{noun}) tend to show cumulative reference, while
telic predicates tend to show quantized, non-cumulative reference. If
\emph{run in the park} applies to two stretches of activity, it also
applies to their sum; thus atelic events have domains structured by the
part-of relation on events. In contrast, if \emph{run to the park}
applies to an event, it fails to apply to any of its relevant subparts;
thus telic events lack the part-of relation %
%(Wellwood 2015: 73)
\citep[73]{Wellwood2015}%
%Wellwood
%
.

As for adjectives\is{adjective}, Wellwood proposes that non-gradable adjectives\is{adjective}, which
express quantities that either exist or not (a table is either square or
not, it cannot be more or less square) are formally parallel to
(singular) count nouns\is{noun} and telic predicates, while gradable adjectives\is{adjective},
which express quantities that there may be more or less of (a thing can
be more or less hot), are parallel to mass nouns\is{noun} and atelic predicates.
They both express predicates of states, the difference being that
gradable adjectives\is{adjective}, unlike non-gradable ones, predicate of ordered
states: they associate directly with sets of ordered degrees, or scales.
Besides, Wellwood assumes that the measure functions introduced with
gradable adjectives\is{adjective} are not only homomorphic to the ordering relations
on the measured domain, but to non-trivial part-whole relations.

Therefore, instead of adopting a notion of `measurement' based on a
variety of measure functions acting on the same objects in unpredictable
ways, Wellwood proposes that language encodes measurement of different
sorts of things in limited ways. Accordingly, she elaborates a uniform
account of measurement as a monotonic mapping from ordered sets of
entities, events, or states to degrees.

\subsection{Reduplication as increased measure}

Let us now try to combine the structural analysis of increasing\is{reduplication!increasing (iconic)}
reduplication proposed in section 4.2 with the cross-categorial
(strictly compositional) analysis of measurement functions proposed by
%
%Wellwood (2014, 2015)
\citet{Wellwood2014,Wellwood2015}%
%Wellwood;Wellwood
%
. Keeping with Wellwood's proposal that there are
no differences in the type of measurement functions among the lexical
categories at a higher level of syntactic/semantic composition, we
speculate that reduplication conveys a similarly stable/unique function
but it targets elements lacking any specification in terms of formal
features.\footnote{It is worth reminding that roots\is{root} have a strongly
  underspecified semantics which allows them to be compatible with the
  semantics of adjectives\is{adjective} (as properties of attributes), verbs\is{verb} (as
  properties of events), nouns\is{noun} (as properties of individuals).} In
particular, we wish to argue that reduplication expresses a unique
function, i.e. `increased measure', that constantly applies to roots\is{root},
only differing in their ontological denotation. Therefore, increasing\is{reduplication!increasing (iconic)}
reduplication is a very low-level (`morphological') adjunction operation
which conveys the function `increased measure' to the roots\is{root} it applies
to: the semantic effects obtained (pluralization, pluractionality\is{aspect!pluractionality},
intensification\is{aspect!intensification} of the base gradable property) ultimately depend on the
different sort of things reduplication modifies, and arguably emerge
constructionally, that is, after \isi{root} categorization applies. It should
be noticed that, semantically, similar results might be obtained at
higher level of syntactic composition via different means, depending on
the categorial domain of application, i.e. through fully-fledged degree
phrases in the adjectival domain (see En. `very Adj', e.g. \emph{very
good}; Ch. `很 \emph{hěn} Adj', e.g. 很高興 \emph{hěn gāoxìng} `very
happy'), and through the use of \isi{plural} affixes and aspectual\is{aspect} markers in
the nominal and verbal domain respectively.

This analysis, however, does not account for some relevant asymmetries
across lexical categories previously noted in the literature (see %
%Zhang 2015
\citealt{Zhang2015}%
%
). As it has been argued in section 3, the main difference at the
structural level between adjectives\is{adjective}, on the one hand, and nouns\is{noun} and
verbs\is{verb}, on the other, concerns the obligatoriness of disyllabic bases for
the latter. That is, whereas increasing\is{reduplication!increasing (iconic)} reduplication applies to
quality-denoting roots\is{root} that may be either mono- or disyllabic, resulting
in AA and AABB patterns interpretively equivalent, with entity and event
denoting roots\is{root} it targets disyllabic units, resulting exclusively in the
AABB pattern.\footnote{The generalization holds under the assumption that
  AA monosyllabic reduplication in the nominal domain should be rather
  understood as reduplication of classifiers\is{classifier} (see section 3.3). We do
  not have an analysis of this type of reduplication yet, and we leave
  the issue for future research.}

As we have seen in 3.3, the AABB reduplication pattern requires a
coordinate base, i.e. two elements related in a symmetrical fashion,
either in a logical coordination, or synonyms or antonyms; thus, instead
of having a single \isi{root} we have a combination of roots\is{root}. These roots\is{root} are
joined together to form a set, whereby the two constituents equally
contribute to the semantics of the whole complex \isi{stem}, i.e. they are in
a symmetrical relation. Structurally, it is worth emphasizing that these
operations all apply at the \isi{root} level, resulting in a recursive
application of `morphological' phenomena, with (symmetrical) \isi{compounding}
and reduplication rigidly ordered in the derivation, yet both applying
before categorization %
%(see Zhang 2015)
\citep[see][]{Zhang2015}%
%Zhang
%
:

\ea\label{ex:BascianoMelloni:42}%
\begin{tikzpicture}[baseline]
\Tree[.{\emph{n/v/a}P}
          \emph{n/v/a}
          [.\node(categ){√root};
            RED
            [.\node(redup){√root};
                √rootA
                \node(compound){√rootB};
            ]
          ]
      ]
\draw[<-] (categ.east) ++(2.3,0) -- ++(0.8,0)  node[right]{\smaller A{[}AABB{]} \emph{n/v/a} \textsc{categorization}};
\draw[<-] (redup.east) ++(1.3,0) -- ++(0.8,0)  node[right]{\smaller A{[}AABB{]} \textsc{reduplication}};
\draw[<-] (compound.east) ++(0.5,0) -- ++(0.8,0)  node[right]{\smaller {[}AB{]} \textsc{compounding}};
\end{tikzpicture}
\z

This analysis seems to produce the surface pattern ABAB, since
reduplication applies to a \is{compounding}compound base AB. However, prosodic patterns
within AABB structures actually seem to support the structural analysis
in (\ref{ex:BascianoMelloni:42}). In particular, %
%Feng (2003) 
\citet{Feng2003} %
%Feng
%
examines tone \emph{sandhi} rules
within disyllabic reduplication and, for AABB, he argues that these
rules apply first between the second A and first B and then between the
first B and second B. On this basis, Feng argues that AB is the actual
morphological unit, whereas AA and BB are not, resulting in the
structural analysis {[}A{[}AB{]}B{]} %
%(Feng 2003: 7-8)
\citep[7-8]{Feng2003}%
%Feng
%
. The issue
deserves further investigation especially aimed at explaining the reason
for the mismatch between underlying structure, supra-segmental patterns
and surface order of morphemes, for which at the moment we cannot offer
an explanation. Suffice it to say that the prosodic pattern of AABB provides
evidence in favour of the analysis in (\ref{ex:BascianoMelloni:42}).

At the interpretive level, we put forward that the combination of two
roots\is{root} which act as the base for the AABB reduplication process forms
itself a sort of `\isi{plural}/collective' expression and reduplication
provides an \emph{increased} \emph{measure} for this kind of
expressions. It has been noted that AABB nouns\is{noun} express greater \isi{plural}
(possibly differing in the semantics from AA reduplication of
nouns\is{noun}/classifiers\is{classifier}, most typically expressing a distributive meaning),
and a similar effect is obtained with AABB verbs\is{verb} (ex. in (\ref{ex:BascianoMelloni:43a}) and (\ref{ex:BascianoMelloni:43b})
are adapted from examples (22, 24) in \citealt{Zhang2015}):

 \ea\label{ex:BascianoMelloni:43}
 \ea\label{ex:BascianoMelloni:43a} 枝枝葉葉 \\
 \gll \emph{zhī\tld{}zhī-yè\tld{}yè}\\
twig\tld{}twig-leaf\tld{}leaf\\
\glt `twigs and leaves'
\ex\label{ex:BascianoMelloni:43b}
 縫縫補補\\
 \gll \emph{féng\tld{}féng-bǔ\tld{}bǔ}\\
sew\tld{}sew-repair\tld{}repair\\
\glt `sew and repair repeatedly'
\z\z

A possible explanation for this structural requirement might lie in the
different ontological type of roots\is{root}: in particular, individual and event
denoting roots\is{root}, different from quality denoting roots, seem to require
an inherently \isi{plural} interpretation in order to be measured. As a matter
of fact, typically comparative expressions with \emph{more} in English
require either mass nouns\is{noun} or \isi{plural} nouns\is{noun}, but exclude singular nouns\is{noun}
(\emph{more dogs} vs. *\emph{more dog}). Similar effects obtain in the
domain of verbs\is{verb} with the contrasts between telic and atelic verbs\is{verb}
discussed by %
%Wellwood (2015)
\citet{Wellwood2015}%
%Wellwood
%
.

Although at this point the present analysis becomes very speculative, we
put forward here that a principled reason for the necessary
disyllabicity of nominal and verbal bases might have the same source of
the asymmetry observed in the domain of comparative expressions.
Specifically, if the semantics of roots\is{root} is very vague and compatible
with any interpretation which eventually emerges at higher levels of
syntactic composition, a way to introduce gradability at the level of
roots\is{root} is to merge them directly, so to create a collection of
individuals, like e.g. 男女\emph{nán-nǚ} `man and woman' (which is
reduplicated as
男男女女\emph{nán\tld{}nán-nǚ\tld{}nǚ} `men and
women'), or of events, e.g. 起伏\emph{qǐ-fú} `rise and fall' (which is
reduplicated as 起起伏伏
\emph{qǐ\tld{}qǐ-fú\tld{}fú} `rise and fall
repeatedly'). In this view, the first merger provides reduplication with
the `gradable base' over which it can apply its increased measure
function. On the contrary, roots\is{root} that are selected by an adjectival head
(i.e. \emph{a}) would inherently express a gradable property and,
accordingly, reduplication would not pose specific disyllabic
requirements on these base units. Furthermore, if this is the case, we
expect no difference in meaning between the reduplication of AA and AABB
adjectival forms, as confirmed by the data (see examples (\ref{ex:BascianoMelloni:6a}) and (\ref{ex:BascianoMelloni:6b}) in
section 2.1, repeated below for the reader's convenience):

\ea\label{ex:BascianoMelloni:44}
\ea\label{ex:BascianoMelloni:44a}\sbsglll{小 (A)}{小小 (AA)}%
        {\emph{xiǎo}}{\emph{xiǎo\tld{}xiāo}}%
        {small}{small\tld{}small}%
        {`small'}{`very/really small'}%
\ex\label{ex:BascianoMelloni:44b}\sbsglll{高興 (AB)}{高高興興 (AABB)}%
        {\emph{gāoxìng}}{\emph{gāo}\tld{}\emph{gāo-xìng}\tld{}\emph{xìng }}%
        {`happy'}{`very/really happy'}%
        {}{}%
\z\z
\section{Conclusion}

Reduplication is a challenging phenomenon in many respects: it is hardly
amen\-able to a uniform characterization in a cross-linguistic
perspective, given the extreme variety of forms and functions it is
associated with; further, it can surface with different forms and
meanings within a single language too, as we have shown with the
reduplicative processes of Mandarin under consideration; it can manifest
semantic functions closely related to the inflectional/functional
domain, but it approaches more closely the domain of derivation/word
formation; finally, it can take as its base units elements of different
size, ranging from \isi{lexeme}/word-like units in one domain (diminishing\is{reduplication!diminishing (countericonic)}
reduplication, which implies \emph{verbal} reduplication in Mandarin) to
category-less units in the other (increasing\is{reduplication!increasing (iconic)} reduplication).

The case of diminishing\is{reduplication!diminishing (countericonic)} reduplication seems to involve units as `big' as
lexemes\is{lexeme}, i.e. stems\is{stem} endowed with category features and with specific
(aspectual\is{aspect}) semantics, as we have shown in section 2.1. The case of
increasing\is{reduplication!increasing (iconic)} reduplication, however, points to the existence of word
formation phenomena that applies below the \isi{lexeme} level. In particular,
increasing\is{reduplication!increasing (iconic)} reduplication seems to suggest that it is a phenomenon that
can apply at a very `low level', namely, that it can merge with
roots\is{root}/stems\is{stem} lacking category specification. Further, it is \emph{per se}
unable to express a definite category, given its presence across all
major lexical categories at both input and output levels. Therefore, the
present case study sheds some light on the existence of word formation
that does not take lexemic inputs and does not give lexemic outputs
either.

On the one hand, this study brings further evidence in favor of a
neo-construc\-tionist\slash DM-like view of the lexemes\is{lexeme} or word units as
syntactically complex elements, and ultimately for the very existence of
category-less roots\is{root}. On the other hand, the curious asymmetries observed
in the domain of increasing\is{reduplication!increasing (iconic)} AA and AABB reduplication, whereby
adjectives\is{adjective} seem to part company from verbs\is{verb} and nouns\is{noun}, call into 
question the semantic (ontological?) character of roots\is{root} and their
alleged requirements for insertion in the syntactic structure
responsible for category assignment and, overall, for their
morphosyntactic properties and distribution. This is a very complex
issue on which we hope to have contributed some further empirical and
theoretical basis but that, it goes without saying, needs further
research and ampler empirical coverage to be satisfactorily addressed.

To conclude, our research has explored the structural and interpretive
effects of reduplication, so productive in Mandarin %
%(see Basciano and
%Melloni 2017) 
\citep[see][]{BascianoMelloni2017} %
%?Basciano-Melloni
%
and broadly attested across Sinitic %
%(see Arcodia, Basciano
%\& Melloni 2015) 
\citep[see][]{ArcodiaBascianoEtAl2015} %
%Arcodia-Basciano-Melloni
%
yet still lacking a satisfying analysis, despite of a
growing interest in the last years. So doing, we hope to have paved the
way for a better understanding of Mandarin reduplication specifically,
and more in general for an approach to word formation which seeks to
reinterpret morphology-specific properties and restrictions within a
more integrated model of grammar, where syntax is also responsible for
word formation.

\section*{Acknowledgments}

\paragraph*{Dedication
  and acknowledgments} To Bernard, who has never ceased to amaze us
  with his extraordinary intellectual vitality and authentic passion, a
  source of inspiration for us. The editors of this volume are
  gratefully acknowledged for inviting us to take part in this venture.
  We are especially grateful to Gilles Boyé for his insightful comments
  on a first version of this chapter. We also wish to thank Giorgio F.
  Arcodia for his careful reading of a first draft. All errors on the
  final version are our responsibility.

\paragraph*{Author
  contributions} The paper is the result of close collaboration between
  the two authors, who are listed in random order. For academic purposes
  only, Chiara Melloni takes responsibility for Sections 1, 3, 3.1, 3.2,
  3.5, 4.3, 4.4, 5, and Bianca Basciano takes responsibility for Sections
  2, 3.3, 3.4, 4, 4.1, 4.2.


%\nocite{Acquaviva2008}
%\nocite{Alexiadou2017}
%\nocite{Arcodia2014}
%\nocite{ArcodiaBascianoEtAl2015}
%\nocite{ArcodiaBascianoEtAl2014}
%\nocite{ArcodiaBasciano2017}
%\nocite{Basciano2017}
%\nocite{Bisang2008a}
%\nocite{Bisang2012}
%\nocite{Bochnak2010}
%\nocite{Bolinger1972}
%\nocite{Borer1994}
%\nocite{Borer2003}
%\nocite{Borer2005}
%\nocite{Bresnan1973}
%\nocite{CaudalNicolas2005}
%\nocite{Yuen1968}
%\nocite{Cinque1990}
%\nocite{Corbett2000}
%\nocite{Croft2001}
%\nocite{Ding2010}
%\nocite{Doetjes1997a}
%\nocite{Embick2008}
%\nocite{Embick2007}
%\nocite{Feng2003}
%\nocite{Forza2011}
%\nocite{Fradin03b}
%\nocite{GhomeshiJackendoffEtAl2004}
%\nocite{Greenberg1972}
%\nocite{Harley2005}
%\nocite{Jackendoff1977}
%\nocite{KwongTsou2003}
%\nocite{Lapointe1980}
%\nocite{Lee-Kim2016}
%\nocite{Li1981}
%\nocite{Lieber04}
%\nocite{Lieber2006}
%\nocite{Marantz1997}
%\nocite{Marantz2001}
%\nocite{Mattes2014}
%\nocite{McClure1995}
%\nocite{NeelemanVandeKootEtAl2004}
%\nocite{Nicolas2010}
%\nocite{Packard2000a}
%\nocite{Paradis2001}
%\nocite{Paris1979}
%\nocite{Paris2007}
%\nocite{Paul2010a}
%\nocite{Ramchand2008}
%\nocite{Sagart1999}
%\nocite{SanchesSlobin1973}
%\nocite{Shi2002}
%\nocite{Travis2000}
%\nocite{Travis2010}
%\nocite{Tsao2001}
%\nocite{Waelchli2005}
%\nocite{Wellwood2014}
%\nocite{Wellwood2015}
%\nocite{Wiltschko2008}
%\nocite{XiaoMcEnery2004}
%\nocite{Xu2012}
%\nocite{Xu2012a}
%\nocite{Zadrapa2017}
%\nocite{Zhang2014}
%\nocite{Zhang2015}
%\nocite{Zhang2016}
%\nocite{Cai07}
%\nocite{Guo99}
%\nocite{Hu06}
%\nocite{Hu95}
%\nocite{Li09}
%\nocite{Tang79}
%\nocite{Tang88}
%\nocite{Wu01}
%\nocite{Zhu03}

\is{reduplication|)}
\il{Chinese!Mandarin Chinese|)}

{\sloppy
    \printbibliography[heading=subbibliography,notkeyword=this]
}

\end{document}
