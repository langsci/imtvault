\documentclass[output=paper]{langsci/langscibook}
\ChapterDOI{10.5281/zenodo.1407017}
\title{Much ado about morphemes}

\author{Hélène Giraudo\affiliation{CLLE, Université de Toulouse, CNRS, Toulouse, France}}

\abstract{Most of the psycholinguists working on morphological processing\is{morphological processing} nowadays admit that morphemes are represented in long-term memory and the predominant hypothesis of lexical access is morpheme-based as it supposes a systematic morphological decomposition mechanism\is{morphology!Decomposition Hypothesis} taking place during the very early stages of word recognition. Consequently, morphemes would stand as access units for any item (i.e., word or nonword) that can be split into two morphemes. One major criticism of this prelexical hypothesis is that the mechanism can only be applied to regular and perfectly segmentable words and, more problematic, it reduces the role of morphology to surface/formal effects. Recently, %
%Giraudo and Dal Maso (2016) 
\citet{GiraudoDalMaso2016} %
%Giraudo-Dal Maso
%
discussed the issue of morphological processing\is{morphological processing} through the notion of morphological salience\is{morphological salience} -- as defined as the relative role of the word and its parts -- and its implications for theories and models of morphological processing\is{morphological processing}. The issue of the relative prominence of the whole word and its morphological components has  indeed been overshadowed by the fact that psycholinguistic research has progressively focused on purely formal and superficial features of words, drawing researchers' attention away from what morphology really is: systematic mappings between form and meaning. While I do not deny that formal features can play a role in word processing, an account of the general mechanisms of lexical access also needs to consider the perceptual and functional salience\is{salience} of lexical and morphological items. Consequently, if the sensitivity to the morphological structure is recognized, I claim that it corresponds to secondary and derivative units of description/analysis. Focusing on salience\is{salience} from a mere formal point of view, I consider in the present contribution how a decompositional hypothesis could deal with some phonological endings whose graphemic transcriptions are various. To this end, a distributional study of the final sound [o] in French is presented. The richness and the diversity of the distributions of this ending (in terms of type of forms, size and frequency) reveal that paradigmatic relationships are more suitable to guide morphological processing\is{morphological processing} than morphological parsing as suggested by the lexeme-based approach\is{morphology!lexeme-based}\is{morphology!lexeme-based} of morphology %
%(see Fradin 2003)
\citep[see][]{Fradin2003}%
%Fradin
%
. }

\maketitle

\begin{document}
\selectlanguage{english}
\il{French|(}
\is{morpheme|(}

\section{Introduction}\label{section:Giraudo:introduction}

In the domain of linguistics, morphological analysis is conceived
according to two antagonistic approaches. On the one hand, the
morpheme-based approach %
%(exemplified by the theoretical framework of Distributed Morphology, see Halle \& Marantz, 1993, 1994)
\citep[exemplified by the theoretical framework of Distributed Morphology, see][]{Halle1993,HalleMarantz1994}
%
%
 integrates
morphology with syntax and considers morphemes as basic minimal forms.
On the other hand, the lexeme-based approach\is{morphology!lexeme-based} postulates that words are
the first units of analysis %
%(e.g., Corbin, 1987; Aronoff, 1994; Fradin, 1996)
\citep[e.g.][]{Corbin87,Aronoff94,Fradin1996}
. Psycholinguistic research aiming to understand the cognitive
processes underlying word processing has broadly explored the effects of
morphological processing\is{morphological processing} on the underlying processes of lexical access.
Whereas it was widely admitted that morphological information plays a
crucial role during word processing, its representation is still
controversial. Nowadays most psycholinguists support a
decompositional view of morphological processing\is{morphological processing} %
(see 
%Rastle \& Davis, 2008 
\citealt{RastleDavis2008} %
for a review)
%\citep[see][for a review]{RastleDavis2008} %
%
%
 that can be linked to the morpheme-based hypothesis,
while a few of them defend an opposing view according to which words are
recognized holistically. This last procedural hypothesis, which is
clearly in line with the lexeme-based approach\is{morphology!lexeme-based}, is tested in the present
chapter through a qualitative and quantitative study of words ending
in {[}o{]}. The distribution of this ending is so diverse
that it would cause a huge number of procedural errors of morpheme\is{morpheme!decomposition}
decomposition. Conversely, the lexeme-based/holistic approach\is{morphology!holistic approach}\is{morphology!lexeme-based} to morphology seems to
be much more appropriate to encompass the diversity.

\section{Studying morphological processing
}\label{studying-morphological-processing}

\is{morphological processing}

In a seminal experimental study carried out by %
%Taft and Forster (1975) 
\citet{TaftForster1975} %
%Taft-Forster
%
on
the recognition of nonwords, the idea of morphological decomposition was
first introduced. They showed that 1) nonwords (e.g., \emph{juvenate})
corresponding to an English stem induced longer rejection latencies than
nonwords that were not stems (e.g., \emph{pertoire}) and 2) prefixed
nonwords constructed with an English prefix and stem (e.g.,
\emph{dejuvenate}) took longer to be classified compared to
morphologically simple control items (e.g., \emph{depertoire}). Longer
decision latencies were interpreted as reflecting a pre-lexical
mechanism of morphological decomposition by which all the words (real or
possible) would be accessed via the first activation of their stem.
Forty years of experimental research have been focused on testing this
decomposition hypothesis\is{morphology!Decomposition Hypothesis} by manipulating the characteristics of
morphologically complex words and nonwords (i.e., their form in terms of
decomposability and interpretability, their lexical frequency and more
rarely their lexical environment) in various perceptual tasks (with
 a large dominance of the lexical decision task\is{lexical decision task} which
consists in a word/nonword discrimination) and numerous languages
(most  studies focusing on English, however). Most of the
results have been interpreted as supporting the decompositional view %
%(see the reviews of Amenta \& Crepaldi, 2012 and Diependaele, Grainger \& Sandra, 2012)
\citep[see the reviews of][]{AmentaCrepaldi2012,DiependaeleGraingerSandra2012} %
%
%
 without really questioning the linguistic processes
underlying the construction of complex words. An overview of the tested
hypotheses and the materials used to explore complex word recognition
indeed reveals  a lack of consideration of the paradigmatic
characteristic of words for understanding the cognitive mechanisms\is{cognitive mechanism}
underlying lexical access. Numerous studies mainly focused on the formal
properties of the word and extended the morphological sensitivity
effects observed with complex nonwords to complex words %
%(e.g., Taft \& Forster, 1976; Caramazza, Laudanna \& Romani, 1988; Laudanna, Cermele \& Caramazza, 1997; Crepaldi, Rastle \& Davis, 2010)
\citep[e.g.][]{TaftForster1976,CaramazzaLaudannaRomani1988,LaudannaCermeleEtAl1997,CrepaldiRastleDavis2010} %
 failing to consider
semantic aspects of morphological complexity. Many experimental reports
examined morphological processing\is{morphological processing} using the masked priming\is{priming!masked} paradigm %
%(Forster \& Davis, 1984) 
\citep{ForsterDavis1984}
 that is supposed to reflect the automatic and
nonconscious processes engaged in the very early stages of word
recognition. In this paradigm, two visually related items are presented
successively and participants are asked to perform a lexical decision
indicating whether the second item is a word or not. However, because
the prime word is presented masked and very briefly, the reader is not even
aware of its presence before seeing the target item.\footnote{The
  \emph{Stimulus Onset Asynchrony} is usually less than 50 milliseconds,
  it corresponds to a subliminal processing.} Hence, the paradigm
allows examination of  the effects of the unconscious processes of the prime
processing on the target recognition %
(see 
%Kinoshita \& Lupker, 2003 
\citealt{KinoshitaLupker2004} %
 for a review on masked priming\is{priming!masked})
%
%
. Many masked priming\is{priming!masked} studies demonstrated
that when two words are morphologically related (e.g.,
\emph{singer--sing}), the prior presentation of the prime shortens the
recognition latency of the target relative to both a baseline condition
in which the prime is completely unrelated to the target (e.g.,
\emph{baker--sing}) and an orthographic condition that uses a prime that
is only formally related to the target (e.g., \emph{single--sing}).
Accordingly, morphological priming effects\is{priming!morphological priming effect} do not result from the mere
formal overlap shared by prime--target. Other studies showed that
semantic priming effects (e.g., \emph{cello--violin}) only arise when the
prime duration is sufficiently high %
(i.e., \textgreater{} 72 ms, see %
%Rastle, Davis, Marslen-Wilson \& Tyler, 2000 
\citealt{RastleDavisEtAl2000} %
 for a comparison between morphological, orthographic and semantic priming effects using different Stimulus--Onset Asynchronies). This general result suggests that priming effects result from
morphological relationships shared by prime--target pairs and that
morphologically related words are connected by some kind of excitatory
links. Most of the models of lexical access have tried to account for these
morphological effects\is{morphological effect}.

\section{ Psycholinguistic models of morphological
processing}\label{psycholinguistic-models-of-morphological-processing}

The architecture of psycholinguistic models of word recognition is
mostly based on symbolic interactive activation models %
%(e.g., McClelland and Rumelhart, 1981)
\citep[e.g.][]{McClellandRumelhart1981}%
%
%
. This type of model is organized in hierarchical
levels of processing containing symbolic units. Each level corresponds
to a linguistic characteristic of words, from letter features to
semantics. During word recognition, activation spreads from the lowest
to the highest levels. Within-level units are connected by inhibitory
links whereas inter-level units are by excitatory links. Consequently,
the model functions according to a principle of competition between
within-level units that is compensated by both bottom-up and top-down
excitations. The independence of the morphological effects\is{morphological effect} relative to
mere formal and semantic effects being established, morphological
information was usually represented as a separate level of processing.
However, its locus relative to the formal level (phonological and
orthographic descriptions of the words) and the semantic level is still
controversial. Morphological units have been situated variously: before the
formal level and stand as access units to the \isi{mental lexicon} (see Figure
\ref{fig:Giraudo:1A} depicting the sublexical model, %
%Taft, 1994
\citealt{Taft1994}%
),  at the interface of
the formal and the semantic level, organizing the word representations
in morphological families (see Figure \ref{fig:Giraudo:1B}, the supralexical model\is{supralexical model}, %
%Giraudo \& Grainger, 2001
\citealt{GiraudoGrainger2001}%
) or at either places, before and after the formal
level (see Figure \ref{fig:Giraudo:1C}, the hybrid/dual route model, %
%Diependaele, Sandra \& Grainger, 2009
\citealt{Diependaele2009}
; see also %
%Diependaele, Morris, Serota, Bertrand \& Grainger, 2013
\citealt{DiependaeleMorrisEtAl2013}%
).

\begin{figure}
  \smaller
\begin{subfigure}[t]{0.3\textwidth}
\input{figures/GiraudoFig1A.tex}
\caption{Sublexical model}\label{fig:Giraudo:1A}
\end{subfigure}\hspace*{0.5cm}%
\begin{subfigure}[t]{0.3\textwidth}
\input{figures/GiraudoFig1B.tex}
\caption{Supralexical model}\label{fig:Giraudo:1B}
\end{subfigure}\hspace*{0.5cm}%
\begin{subfigure}[t]{0.3\textwidth}
\input{figures/GiraudoFig1C.tex}
\caption{Hybrid model}\label{fig:Giraudo:1C}
\end{subfigure}
\caption{Alternative hierarchical models of morphological processing\is{morphological processing}.} % Sacha TODO: ask author for caption.
\label{fig:Giraudo:1}
\end{figure}

These three options nevertheless assume  morpheme\is{morpheme!representation} representations and by
extension, propose a decompositional view of morphology. The sublexical and
the hybrid models of morphological processing\is{morphological processing} actually state very
clearly that complex words are systematically decomposed into morphemes
during lexical access. This decomposition mechanism\is{morphology!Decomposition Hypothesis} is reflected by the
obligatory activation of morphemes to gain the word representations
coded within the \isi{mental lexicon}. Each time a complex or a pseudo complex
word (i.e., a word with a surface morphological structure like for
example the word \emph{corner} which comprises a surface stem
\emph{corn-} and a surface suffix \emph{-er}) is processed by our
cognitive system, it triggers the activation of its constituent
morphemes that successively activate the wordforms containing it.
Moreover, the hybrid model supposes that ``In a priming context, opaque
morphological relatives will only be able to prime each other through
shared representations at the morpho-orthographic level, whereas
transparent items will also be able to do this via shared
representations at the morpho-semantic level'' %
%(p. 896, Diependaele et al., 2009)
\citep[896]{Diependaele2009}%
%
%
. Even if the authors claim that morphological representations\is{morphological representation}
\emph{per se} are not simply represented at both levels -- the first
being orthographically constrained and the second semantically
constrained -- these two levels actually correspond to surface morphemes
at least as far as the contained units are concerned. In these two
frameworks (sublexical and hybrid models), morphological priming effects\is{priming!morphological priming effect}
result from the pre-activation of the morpheme shared by the prime and
the target before accessing the word representations. These morphemic
units pre-select in a way the wordforms that can potentially match with
the target to be recognized. Lexical access takes place via the
obligatory activation of surface morphemes.

One major criticism of the prelexical hypothesis is that this mechanism
can only be applied to regular and perfectly segmentable words. Even
more problematic is the fact that it reduces the role of morphology to
surface/formal effects. This is certainly why Diependaele and
colleagues proposed a second level of representation for morphology, as
numerous experimental studies showed that two morphologically related
but orthographically unrelated words (e.g., \emph{bought--buy}) prime
each other. However,  this solution only considers
morphology from its syntagmatic dimension: that is according to the word
internal structure. Therefore, nothing is said about the influences of
family and series\footnote{The term `series' was, to our knowledge,
  first introduced by %
%Hathout (2005; 2008) 
\citet{Hathout2005,hathout2008.textgraphs3} %
%Hathout
%
and refers to groups of words
  sharing the same affix.} on word representations.

The original version of the supralexical model\is{supralexical model} %
%(Giraudo \& Grainger, 2001) 
\citep{GiraudoGrainger2001} %
%
%
 also integrated morphemes even though it did not suppose a
decomposition mechanism\is{morphology!Decomposition Hypothesis} by which word representations are decomposed
properly in order to activate their semantic representations. On the
contrary, the morphological level contained ``emerging'' base morphemes,
that is, morpheme\is{morpheme!representation} representations resulting from the acquisition of
complex words that are derived from the same base or the same series.
Accordingly the morphological node organizes the word level in paradigms
(i.e., morphological families and series), morphologically related words
being connected together thanks to a supralexical node. Concretely, when
the system processes a complex word, it first activates all the word
representations that match formally with it while at the same time the
complex forms activate their common nodes that feed back positively
these forms. As all units belonging to the same level compete with each
other, the activated formally related words inhibit each other, but
those which are also morphologically related receive facilitation from
their shared node. Words from the same family are then less inhibited
than the other representations at the word level. In masked priming\is{priming!masked}, the
morphological facilitation between two morphologically related words
observed relatively to two unrelated words is explained in terms of a
reduced inhibition effect compared to a regular inhibition effect for
unrelated items.

\section{The benchmark effects: lexicality, frequency,
regularity}\label{the-benchmark-effects-lexicality-frequency-regularity}

Among the factors that have been manipulated in order to better
understand the nature of morphological relationships and the locus of
morphological priming effects\is{priming!morphological priming effect} within the \isi{mental lexicon}, one can cite
\emph{lexicality}, \emph{frequency} and \emph{regularity}. Starting from
the dominant hypothesis according to which words are first decomposed
before accessing the \isi{mental lexicon}, some authors used the masked
priming paradigm to study the influence of lexicality (i.e., comparing
the processing of existing words coded in the \isi{mental lexicon} relative to
non-existing but morphologically structured items) in word recognition.
A series of masked priming\is{priming!masked} studies examined the effect of complex
nonword primes during the early processes of lexical access. For
example, %
%Longtin and Meunier (2005) 
\citet{LongtinMeunier2005} %
%?Meunier
%
have tested the effects of nonwords
constructed using legal and illegal combinations of existing stems and suffixes
in French (e.g., legal: \emph{infirmiser--infirme
`disabled+er'--`disabled'}; illegal: \emph{garagité--garage}
\emph{`garage+ité'--`garage'}) and found that both types of nonwords
produced facilitation relative to orthographic control primes (e.g.,
\emph{rapiduit--rapide}, \emph{`fast+uit'--`fast'}), that did not induce
any significant effect on word recognition %
(see also, %
%McCormick, Rastle \& Davis, 2009; Morris, Grainger \& Holcomb, 2013
\citealt{McCormickRastleEtAl2009,MorrisGraingerHolcomb2013} %
 for English materials)%
%
%
. %
%Giraudo and Voga (2013) 
\citet{GiraudoVoga2013} %
%?Giraudo-Voga
%
replicated these results using
French prefixed nonwords (e.g., \emph{infaire--faire, `un-do'--`do'})
suggesting that these effects apply to all affixed items. %
%Duñabeitia,
%
%Perea and Carreiras (2008)
\citet{AndoniDunabeitiaPereaEtAl2008}%
%Perea-Carreiras
%
, focused on affix priming in Spanish and
showed that isolated suffixes (e.g., \emph{dad--igualdad,
`ity'--`eguality'}) and suffixes in neutral context (e.g\emph{.,
\#\#\#\#\#dad--igualdad}) were also able to induce positive priming
effects (see also %
%Crepaldi, Hemsworth, Davis \& Rastle, 2015
\citealt{CrepaldiHemsworthAl2016} %
 using
English suffixed related nonword pairs like \emph{sheeter--teacher}).
Finally, %
%Crepaldi, Rastle, Davis and Lupker (2012) 
\citet{CrepaldiRastleDavisLupker2013} %
%?Crepaldi-Rastle-Davis-Lupker
%
examined reversed
compounds like \emph{fishgold--goldfish} and observed facilitation within
related prime--targets pairs.

Taken together these studies suggest that in the early stages of word
recognition -- in masked priming\is{priming!masked} conditions in which primes are
presented less than 50-60 ms -- lexicality does not impact lexical
access as far as complex nonwords are considered. Moreover, none of
these studies found priming effects using orthographic nonword primes
(e.g., \emph{blunana--blunt} tested by %
%McCormick, Rastle \& Davis, 2009%
\citealt{McCormickRastleEtAl2009}%
%
) suggesting a pre-lexical morphological analysis of the primes, blind to
lexicality. However, even if these data seem to strengthen the
pre-lexical decomposition hypothesis\is{morphology!Decomposition Hypothesis}, results obtained using nonword
primes created by letter transpositions have to be considered.
Following, the discovery in Cambridge University according to which
``\emph{it deosn't mttaer in waht oredr the ltteers in a wrod are, the
olny iprmoetnt tihng is taht the frist and lsat ltteer be at the rghit
pclae... it doesn't matter in what order the letters in a word are, the
only important thing is that the first and last letter be at the right
place}'' (see
\href{http://www.mrc-cbu.cam.ac.uk/personal/matt.davis/Cmabrigde/}{\emph{http://www.mrc-cbu.cam.ac.uk/personal/matt.davis/Cmabrigde/}}),
a series of masked priming\is{priming!masked} experiments aimed to explore this effect.
Some studies showed that reading comprehension of jumbled words are more
or less costly (as demonstrated for example by %
%Rayner, White, Johnson \& Liversedge, 2006
\citealt{RaynerWhiteEtAl2006}%
), this effect still constitutes a challenge for the
decompositionalists. It indeed contradicts the hypothesis according to
which lexical access takes place via the obligatory decomposition of
complex words into morphemes. Masked priming experiments explored
repetition priming effects (i.e., the same stimulus is presented as
prime and target, like in \emph{table--table}) and morphological priming
effects using jumbled primes and %
%Beyersmann, Castles and Coltheart (2011) 
\citet{BeyersmannColtheartCastles2012}
%
%
 first found that relative to unrelated primes, both repeated
simple primes (e.g., \emph{wran--warn}) and morphological primes (e.g.,
\emph{wranish--warn}) reduced the latencies of target word recognition %
(see also 
%Christianson, Johnson \& Rayner, 2005 and Duñabeitia, Perea \& Carreiras, 2007 
\citealt{ChristiansonJohnsonRayner2005,DunabeitiaPereaEtAl2007} %
for Spanish and Basque)%
%
%
. However, when orthographic
primes (e.g., \emph{wranel--warn}) were manipulated, no facilitation
priming was observed highlighting the need for priming effects to keep
the morpheme boundary intact. Then, a series of experiments compared of
primes with Transposed Letters (TL) at the morpheme boundary (e.g.,
\emph{speaekr--speak}) vs. outside the morpheme boundary (e.g.,
\emph{spekaer--speak}). Only one experiment  in the literature
reported a benefit for TL primes when the transposition fell within the
morpheme; no benefit was observed when the transposition fell across the
morpheme boundary %
(%
%Duñabeitia et al. 
\citealt{DunabeitiaPereaEtAl2007} %
, using Spanish materials)%
%Duñabeitia-al.
%
.
Subsequent investigations in both English and Spanish failed to
replicate these findings %
%(Beyersmann, Coltheart \& Castles, 2012; Beyersmann, McCormick, \& Rastle, 2013; Rueckl \& Rimzhim, 2011;
%Sanchez-Gutierrez \& Rastle, 2013) 
\citep{BeyersmannColtheartCastles2012,BeyersmannMcCormickRastle2013,RuecklRimzhim2011,Sanchez-GutierrezRastle2013} %
%
%
 and obtained equivalent facilitation
when the transposed letters appeared within a stem or across a morpheme
boundary.

Because TL benefit is not affected by the position of the TL relative to
the morpheme boundary, I consider this result as a strong challenge for
any decompositionalist model. If morphologically complex stimuli are
indeed systematically decomposed into morphemes before gaining the
\isi{mental lexicon}, the main predictions of such models is that when the
morphemes boundary is disrupted, no priming effect is expected since the
cognitive system cannot parse the item into potential morphemes.

%
%Diependaele et al. (2013) 
\citet{DiependaeleMorrisEtAl2013} %
%Diependaele-al.
%
furthermore investigated the TL effect by
comparing semantically transparent vs. opaque complex primes. Their first experiment showed that relative to formal primes, both transparent and opaque
primes induced positive priming (e.g., \emph{banker--bank} =
\emph{corner--corn} \textgreater{} \emph{scandal--scan}). However, when
morphological primes with TL were used, the transparent ones produced
priming while the opaque ones did not (e.g., \emph{baneker--bank} \textgreater{} \emph{corenr--corn} =
\emph{scandal--scan}). A second experiment manipulated derived nonword
primes in order to examine the effect of lexicality on the TL effect.
Materials were selected from %
%Longtin and Meunier's study (2005)
\citeauthor{LongtinMeunier2005}'s \citeyear{LongtinMeunier2005} study %
%
%
 and the
authors found, on the one hand, that relative to unrelated primes, both
intact derived word primes and intact derived nonword primes 
facilitated target recognition equally (e.g., \emph{garagiste--garage} =
\emph{garagité--garage} \textgreater{} \emph{diversion--garage}). On the
other hand, when comparable morphological primes with TL were
manipulated, a different pattern of priming emerged: only derived primes
induced priming (e.g., \emph{garaigste--garage} \textgreater{}
\emph{garaigté--garage} = \emph{diverison--garage}). According to the
authors, these data are line with the predictions of their hybrid/dual
route model of morphological processing\is{morphological processing} (presented above in Figure 3) in
which complex items are automatically parsed within two morphological
levels: morpho-orthographically and morpho-semantically, reflecting two
sources of morphemic activation in word recognition. Morphological
complex words (e.g., \emph{banker}) are actually supposed to be
processed twice at both morphemic levels, and pseudo-complex words (e.g.,
\emph{corner}) once at the morpho-orthographic level, letter
transposition across the morpheme boundary should interfere more with
morpho-orthographic  than morpho-semantic processing.
Accordingly, transparent words and nonwords with TL are supposed to
resist letter transpositions thanks to the morpho-semantic activation
while opaque words and nonwords with TL did not because the morphemic
activation at the morpho-orthographic level would be skipped.

According to me, the dual route model and the way masked priming\is{priming!masked} effects
are interpreted in this study are far from being convincing. ``The key
prediction of this account is that fast-acting effects of morphology are
not only morpho-orthographic in nature, but also morpho-semantic, and
most importantly, that these effects reflect two separate sources of
morphemic activation in word recognition'' (p. 989).

If genuine complex words benefit from two sources of activation (morpho-ortho\-graph\-ic and morpho-semantic) and pseudocomplex words from
one only (morpho-or\-tho\-graph\-ic), words like \emph{banker} should be more
efficient primes than \emph{corner}. Nevertheless, their results (experiment~1) and the ones obtained so far in the literature demonstrate on the
contrary that prime-target pairs like \emph{banker-bank} and
\emph{corner-corn} produce equivalent priming effects (cf. surface
morphology effects, see %
%Rastle \& Davis, 2008
\citealt{RastleDavis2008} %
 for a review). When TL
effects are considered, it has been shown that primes with TL at the
morpheme boundary (e.g., \emph{banekr--bank}) and within the stem (e.g.,
\emph{bakner--bank}) both induce equivalent facilitation effects. If the
morpho-orthographic level is much more sensitive to letter order than
the morpho-semantic level is, then one should have observed greater
priming effects when the morpheme boundary of the prime is intact (e.g.,
\emph{bakner--bank}) because two sources of activation could operate
while for jumbled morpheme boundary (e.g., \emph{banekr--bank}) only one
source is active. The results obtained so far did not show any
difference between these two types of primes, neither in the present
paper, nor in the literature. Moreover, %
%Diependaele et al. (2013) 
\citet{DiependaeleMorrisEtAl2013} %
%Diependaele-al.
%
found
in their experiment~2 that TL letter derived primes (e.g., \emph{banekr--bank})
produced faster reaction times than intact primes (e.g.,
\emph{banker--bank}). This surprising result is also very problematic for
a decompositional account since the letter recoding for the TL primes
that is necessary to activate morphemic representations should have
delayed lexical access, therefore reducing priming.

Word processing is also closely linked to input frequency. This
factor that has been broadly studied in the psycholinguistic literature
on word recognition showing a strong and very robust correlation between
lexical frequency and recognition latencies: the higher the frequency, the shorter the reaction time  (see %
%Ellis, 2002
\citealt{Ellis2002}
%
%
 for a review).
Generally, these experimental studies oppose derived or inflected words
of comparable surface frequency\is{frequency!surface frequency}, but crucially differing in their stem
frequency (high vs. low). In this kind of study, when reaction times
(RTs) were found to be a function of the stem frequency\is{frequency!stem frequency}, this is
considered as evidence of the fact that word recognition implies the
activation of the stem. For example in Italian, %
%Burani and Caramazza (1987)
\citet{BuraniCaramazza1987} %
%
%
 investigated derived suffixed forms (verbal roots combined with
highly productive suffixes such as \emph{-mento, -tore, -zione}) by
opposing stimuli matched for whole word frequency\is{frequency!whole word frequency}, but differing in root
frequency (experiment~1), to stimuli matched for root frequency but differing
in whole word frequency\is{frequency!whole word frequency} (experiment~2). Their results indicated that reaction
times were influenced by both root and whole word frequencies\is{frequency!whole word frequency} (faster
RTs were obtained for items containing a high frequency root in experiment~1
and for higher whole word frequency\is{frequency!whole word frequency} items in experiment~2), the authors
suggested that the access procedure crucially operates with both
whole word and morpheme access units. Frequency effects have been
observed also in French by %
%Colé, Beauvillain and Segui (1989)
\citet{ColeBeauvillainSegui1989}%
%Colé-Beauvillain-Segui
%
, who
similarly considered derived words matched for surface frequency\is{frequency!surface frequency} but
differing in their cumulative root frequency\is{frequency!root frequency} (e.g., \emph{jardinier
`gardener'}, containing a high frequency root, vs. \emph{policier
`policeman'}, containing a low frequency root). Since a clear cumulative
root effect was observed only for suffixed words but not for prefixed
ones, Colé and colleagues suggest that only the former are accessed
through decomposition via the root.

More recently, %
%Burani and Thornton (2003) 
\citet{BuraniThornton2003} %
%Burani-Thornton
%
conducted a study on the
interplay between the frequency of the root, the frequency of the suffix
and the whole word frequency\is{frequency!whole word frequency} in processing Italian derived words. More
precisely, in experiment~3, they considered low frequency suffixed words that
differed with respect to the frequency of their morphemic constituents.
As expected, the results showed that lexical decisions were faster and
more accurate when the derived words included two high-frequency
constituents (e.g., \emph{pensatore} `thinker') and slowest and least
accurate when both constituents had low frequency (e.g., \emph{luridume}
`filth' ). Interestingly, when the derived words included only one
high-frequency constituent (either the root or the suffix), the lexical
decision rate was found to be a function of the frequency of the root
only, irrespective of suffix frequency. The authors conclude that access
through activation of morphemes is beneficial only for derived words
with high frequency roots, while lexical decision latencies to suffixed
derived words are a function of their surface frequency\is{frequency!surface frequency} when they
contain a low frequency root.

To sum up, frequency effects\is{frequency!frequency effect} have been considered as a diagnostic for
determining whether an inflected or derived form is recognized through a
decompositional process that segments a word into its morphological
constituents or through a direct look-up of a whole word representation
stored in lexical memory. Frequency has therefore played a crucial role
in the debate which opposed full parsing models, which assume a
prelexical treatment of the morphological constituents with a consequent
systematic and compulsory segmentation of all complex words %
%(Taft and
%Forster 1975, Taft 1979)
\citep{TaftForster1975,Taft1979}%
%Taft-Forster;Taft
%
, and full listing models, which defend a
non-prelexical processing of the morphological structure and a complete
representation of all morphologically complex words %
%(see McClelland \&
%Rumelhart 1981)
\citep[see][]{McClellandRumelhart1981}%
%McClelland-Rumelhart
%
.

Despite the importance of the frequency for lexical access (the more 
 frequent a word, the faster its recognition, see 
%Solomon \& Postman, 1952
\citealt{SolomonPostman1952}%
%
%
) and the number of priming studies focused on its impact for word
recognition (see %
%Kinoshita, 2006
\citealt{Kinoshita2006} %
%
%
 for a review), very few studies 
manipulated frequencies using masked morphological priming\is{priming!masked}. In a
paradigm such as masked priming\is{priming!masked} in which the prime is presented for a
very brief duration, frequency is nevertheless a crucial factor since it
determines the access speed to lexical representations. Moreover, clear
opposite predictions can be derived for the two main approaches of
morphological processing\is{morphological processing}. According to the decompositional approach,
only the root/stem frequency\is{frequency!stem frequency}\is{frequency!root frequency} should interact with morphological priming
effects\is{priming!morphological priming effect} since complex words are supposed to be accessed via the
activation of their stem. The holistic hypothesis predicts no stem
frequency effect\is{frequency!frequency effect} but that surface frequency\is{frequency!surface frequency} strongly determines masked
morphological priming effects\is{priming!morphological priming effect} because lexical access takes place on the
whole word. %
%Giraudo and Grainger (2000) 
\citet{GiraudoGrainger2000} %
%Giraudo-Grainger
%
investigated the interaction of
both frequencies with morphological processing\is{morphological processing} through a series of
masked priming\is{priming!masked} experiments conducted in French. They manipulated the
surface frequencies\is{frequency!surface frequency} of derivatives used as primes for the same target
(high frequency primes like \emph{amitié--ami} `friendship'--`friend';
low frequency primes like\emph{amiable--ami} `friendly'--`friend'). They found an interaction between priming effects and
the prime surface frequency\is{frequency!surface frequency} (experiment~1), but no effect for the base
frequency. Experiments~1 and~3 demonstrated that the surface frequency\is{frequency!surface frequency}
of morphological primes affects the size of morphological priming\is{priming!masked}: high
surface frequency\is{frequency!surface frequency} derived primes showed significant facilitation
relative to orthographic control primes (e.g., \emph{amidon--ami}
`starch'--`friend'), whereas low frequency primes did not. The results
of experiment~4 revealed, conversely, that cumulative root
frequency does not influence the size of morphological priming\is{priming!masked} on free
root targets. Suffixed word primes facilitated the processing of free
root targets with low and high cumulative frequencies\is{frequency!cumulative frequency}. These data
suggest that during the early processes of visual word recognition,
words are accessed via their whole form (as reflected by surface
frequency effects\is{frequency!frequency effect}) and not via decomposition (since the base frequency\is{frequency!base frequency}
did not interact with priming).

Another piece of evidence against the decompositional hypothesis comes
from the study conducted by %
%Giraudo and Orihuela (2015)
\citet{GiraudoOrihuela2015}%
%Giraudo-Orihuela
%
, which
considered the effects of the relative frequencies\is{frequency!relative frequency} of complex primes and
their base target opposing the configuration with high frequency primes/low frequency targets to the configuration with low frequency primes/high frequency targets in French. Their results revealed that, relative
to both the orthographic and unrelated conditions, morphological priming
effects\is{priming!morphological priming effect} emerged only when the surface frequency\is{frequency!surface frequency} of the primes is higher
than the surface frequency\is{frequency!surface frequency} of the targets (see also 
%Voga \& Giraudo, 2009 
\citealt{VogaGiraudo2009} %
%
%
 for a similar conclusion). Again, these data contradict the
prediction of the classical decomposition hypothesis\is{morphology!Decomposition Hypothesis}, according to which
the reverse effects would be expected.

The interpretation of frequency effects\is{frequency!frequency effect} with respect to psycholinguistic
models, however, remains very controversial. %
%McCormick, Brysbaert and
%
%Rastle (2009) 
\citet{McCormickBrysbaertEtAl2009} %
%Rastle
%
defend a completely opposite position, in favour of an
obligatory decomposition of all kinds of stimuli (even for the
non-morphologically structured ones). They carried out a masked priming\is{priming!masked}
experiment manipulating the frequency of the primes, thus comparing high
frequency, low frequency and nonword primes. Their hypothesis was that
if morphological decomposition was limited to unfamiliar words, as
predicted by the horse-race style of dual-route models, then priming
should be limited to the last two conditions. On the contrary, if
morphological decomposition was routine, an obligatory process
applying to all morphologically structured stimuli should occur in all
three conditions. The results showed that the priming effect observed
with high frequency primes was equivalent to the one observed with low
frequency primes and with nonword primes. Such findings seem to confirm
the claim that a segmentation process is not restricted to low frequency
words or nonwords, as assumed by horse-race models.

Very recently, the masked priming\is{priming!masked} study carried by %
%Giraudo, Dal Maso and
%
%Piccinin (2016) 
\citet{GiraudoDalMasoEtAl2016} %
%Piccinin
%
on Italian materials explored the role
stem frequency\is{frequency!stem frequency} in morphological processing\is{morphological processing} even more deeply. They focused on the surface
frequencies of base targets (comparing high vs. low surface frequency\is{frequency!surface frequency}
targets, e.g., \emph{trasfire} `to transfer' vs. \emph{motivare} `to motivate')
primed by either the same base (e.g., \emph{trasfire--trasfire}), a
derivation of the base (e.g., \emph{trasferimento--trasfire}
`transfer'--`to transfer'), an orthographic control (e.g.,
\emph{trasparenza--trasfire} `transparence' --`to transfer') and an
unrelated control (e.g., \emph{sacrificio--trasfire} `sacrifice'--`to
transfer). The data showed that full morphological priming effects\is{priming!morphological priming effect} were
obtained whatever the frequency of the targets (high or low).
Accordingly, the frequency of the base contained in the derived primes
(e.g., \emph{trasferire} in \emph{trasferimento}) did not interfere with
morphological facilitation: primes whose base had a high frequency did
not induce stronger facilitation than primes with a low frequency base.
As a consequence, contrary to the predictions of a decompositional
approach of lexical access to complex words, the prior presentation of a
complex prime whose stem had a high surface frequency\is{frequency!surface frequency} did not accelerate
the access to its lexical representation relative to primes whose stem
frequency was low.

Taken together, the frequency effects\is{frequency!frequency effect} obtained using  masked priming\is{priming!masked}
 suggest that lexical access depends much more on the lexical
frequency of the prime (that determines its activation threshold) than
on its the stem frequency\is{frequency!stem frequency}. Stem frequency does not seem to interfere
with the access to the \isi{mental lexicon} and morphological priming effects\is{priming!morphological priming effect}
reveal instead that, as soon as a lexical representation is
activated within the \isi{mental lexicon}, such a representation automatically
triggers the activation of all its family members. The result of the
overall activation of the morphological family is revealed in those LDT
experiments in which it has been observed that both the lexical and the
base frequencies\is{frequency!base frequency} determine the recognition latencies of suffixed words.
Only models that consider the word as the main unit of analysis, be it
morphological %
%(e.g., Giraudo \& Voga 2014) 
\citep[e.g.,][]{GiraudoVoga2014} %
%Giraudo-Voga
%
or not %
%(e.g., Baayen, Milin,
%Filipovic-Durdevic, Hendrix \& Marelli 2011)
\citep[e.g.,][]{Baayen11}%
%Marelli
%
, are able to account for
these findings.

Finally regularity is another factor from which opposite predictions
can be drawn by the two views of morphological processing\is{morphological processing}. In the
psycholinguistic literature, this issue is intimately linked with the
ease with which a complex word can be segmented into morphemes. Most of
these studies consider morphology under the single angle of the word
internal structure and the reported experiments carried out with
irregular words aimed to test the predictions of decomposition
hypothesis according to which parsability should interact with the
magnitude of morphological priming effects\is{priming!morphological priming effect}. Regularity has been mainly
tested with irregular materials like the irregular verbs in English
(e.g., \emph{bought--buy}) and with complex words containing
various orthographic alterations (e.g., \emph{bigger--big}).
%
%Pastizzo and Feldman (2002) 
\citet{PastizzoFeldman2002} %
%Pastizzo-Feldman
%
carried a series of masked priming\is{priming!masked}
experiments on English irregular verbs (viz. allomorphs). They found
that allomorphs (e.g., \emph{fell}) whose construction enables
decomposition, primed their verbal base (e.g., \emph{fall}) more than
orthographically matched (e.g., \emph{fill}) and unrelated control words
(e.g., \emph{hope}) did. Contrary to the predictions of the
decompositional hypothesis, non-segmentable complex words then induce
priming effects that cannot be attributed to the formal overlap between
prime--target pairs but depend on the morphological relationships they
share. These results have been replicated later by %
%Crepaldi, Rastle, Coltheart, \& Nickels (2010; see also the MEG study carried by Fruchter, Stockall \& Marantz, 2013 leading to the same pattern of data)
\citeauthor{CrepaldiRastleDavis2010} (\citeyear{CrepaldiRastleDavis2010}; see also the MEG study carried by \citealt{FruchterStockallEtAl2013} leading to the same pattern of data) %
%\citet[; see also the MEG study carried by \citealt{FruchterStockallEtAl2013} leading to the same pattern of data]{CrepaldiRastleDavis2010} % complicated struct ???
%
who were
forced to admit the ``existence of a second higher-level source of
masked morphological priming\is{priming!masked}'' and proposed a lemma-level composed of
inflected words acting ``at an interface between the orthographic
lexicon and the semantic system'' (p. 949).

%
%McCormick, Rastle and Davis (2008) 
\citet{McCormickRastleEtAl2008} %
%McCormick-Rastle-Davis
%
manipulated another category of
derived stimuli that cannot be segmented perfectly into their morphemic
components (for example, missing `e' (e.g., \emph{adorable--adore}), 
shared `e' (e.g., \emph{lover--love}), and  duplicated consonant
(e.g., \emph{dropper--drop}) in order to test the flexibility of the
morpho-orthographic segmentation process described by morpheme-based
models. Once again, their results demonstrate the robustness of this
segmentation process in the case of various orthographic alterations in
semantically related (e.g., \emph{adorable--adore}) as well as in
unrelated prime--target pairs (e.g., \emph{fetish--fete}). The same
authors then addressed the same question using morphologically
structured nonword primes %
%(McCormick et al., 2009)
\citep{McCormickRastleEtAl2009}%
%
%
. To this end, they
created nonword primes with a missing <e> at the morpheme boundary
(e.g., \emph{adorage-adore}) and compared it to orthographically related
prime-target pairs (e.g., \emph{blunana-blunt}). The observed data
showed that morphologically structured nonword primes facilitated the
recognition of their stem targets, and that the magnitude of these
priming effects was significantly larger than for orthographic control
pairs. They interpreted this result as supporting their previous
conclusions on word primes (2008) according to which stems that
regularly lose their final <e> may be represented in an
underspecified manner (i.e., absent or marked as optional). But far to
call the decomposition mechanism\is{morphology!Decomposition Hypothesis} into question, they claimed that the
process of morphological decomposition was robust to regular
orthographic alterations that occur in morphologically complex words.

According to me, these results could be interpreted on the contrary as
being totally incompatible with the hypothesis of a mandatory
decomposition process based on the surface morphology because this
mechanism is only based on a minimalist condition of having two surface
morphemes. If not, the decompositionalist approach needs to explain to
how/on which criteria these words are actually decomposed. So far, the
decompositionalists only proposed the idea of fast acting morphological
effects (see %
%Diependaele et al., 2013
\citealt{DiependaeleMorrisEtAl2013}%
%
%
) without specifying on what
visual/perceptual base these effects could actually operate. Recently,
%
%Giraudo and Dal Maso (2016) 
\citet{GiraudoDalMaso2016} %
%Giraudo-Dal Maso
%
discussed this issue through the notion of
morphological salience\is{morphological salience} and its implications for theories and models of
morphological processing\is{morphological processing}. More precisely, the impact of the salience\is{salience} of
complex words and their constituent parts on lexical access was
questioned in light of the benchmark effects reported in the literature
and the way they have been unilaterally interpreted. The issue of the
relative prominence of the whole word and its morphological components
has been indeed overshadowed by the fact that psycholinguistic research
has progressively focused on purely formal and superficial features of
words, drawing researchers' attention away from what morphology really
is: systematic mappings between form and meaning. While I do not deny
that formal features can play a role in word processing, an account of
the general mechanisms of lexical access also needs to consider the
perceptual and functional salience\is{salience} of lexical and morphological items.
Consequently, the existence of morphemes is then recognized, but we
claimed that it corresponds to secondary and derivative units of
description. I hold that results obtained on the basis of masked
priming are in line with holistic models of lexical architecture in
which morphology emerges from the systematic overlap between forms and
meanings (%
%Baayen et al., 2011
\citealt{Baayen11}
)\footnote{And also to abstractive
  approaches assuming that ``the lexicon consists in the main of full
  forms, from which recurrent parts are abstracted'' (%
%Blevins, 2006
\citealt{Blevins06}%
%
%
: 537).} and for which the lexeme is the first unit analysis for the
cognitive system. In such models, salience\is{salience} is not only a matter of
internal structure, but also results from the organization of words in
morphological families and series. As a consequence, not only
syntagmatic, but also paradigmatic relationships contribute to
morphological salience\is{morphological salience}. Certainly, the notion of salience\is{salience} refers
primarily to formal aspects, because the perceptual body of the morpheme
is necessarily the starting point of the processing mechanism. However,
the notion of salience\is{salience} makes sense for complex word processing only if
the form it refers to is associated with a meaning or function.
Salience, in other words, is a property of the morpheme (i.e., a stable
association of form and meaning), not simply of a phonetic or graphemic
chain.

\section{The final sound {[}o{]} in
French}\label{the-final-sound-o-in-french}

Focusing on salience\is{salience} from a mere formal point of view leads to consider
how a decompositional hypothesis could deal with some phonological
endings whose graphemic transcriptions are various.

I present a distributional study of the final sound {[}o{]} in French
suggesting that paradigmatic relationships are more suitable to guide
morphological processing\is{morphological processing} than morphological parsing. The data have
selected from Lexique~3 database %
%(New, 2006)
\citep{new2006lexique}.

In French, the final sound {[}o{]} can be written in 9 different ways:

\ea   \emph{-au} as in:\\
  \gll noyau, préau, tuyau, bestiau\\
     `core', `courtyard', `pipe', `cattle'\\
\ex   \emph{-aud} as in:\\
  \gll noiraud, rougeaud, crapaud, nigaud\\
     `black+aud', `red+aud', `toad', `idiot'\\
\ex   \emph{-aut}, as in:\\
  \gll quartaut\\
     `quarter+aut\\
\ex   \emph{-eau} as in:\\
  \gll poireau, grumeau, tableau, drapeau\\
     `leek', `lump', `board', `flag'\\
\ex   \emph{-od} as in:\\
  \gll pernod\\
     `pernod'\\
\ex   \emph{-op} as in:\\
  \gll galop, sirop, trop\\
     `gallop', `syrup', `too much'\\
\ex   \emph{-os} as in:\\
  \gll gros, dos, enclos, chaos\\
     `big', `back', `pen', `chaos'\\
\newpage      
\ex   \emph{-ot} as in:\\
  \gll bistrot, cachot, chiot, jeunot\\
     `pub',  `dungeon', `puppy', `youngster'\\
\ex   \emph{-o} as in:\\
  \gll auto, ado, mécano, fluo\\
     `car', `teenager', `mechanic', `fluo'\\
\z

Among these words, one can distinguish semantically transparent complex
words (e.g., \emph{drap-eau}) M+, semantically opaque complex words
(e.g., \emph{crap-aud}) M$-$, simple words (e.g., \emph{trop}) O and
clippings (e.g., \emph{ado} from \emph{adolescent}) C, whose
distributions in terms of size, i.e., number of different words sharing
the same ending (N) and cumulative frequencies\is{frequency!cumulative frequency} of these words (F) are
sometimes very heterogeneous. Tables \ref{tab:Giraudo:1} and \ref{tab:Giraudo:2} present these different
distributions.

  \begin{table} 
    \begin{tabularx}{\textwidth}{lSSSSS}
    \lsptoprule
    Ending & \multicolumn{5}{c}{Distribution of the Size} \\
    \midrule
    &\centering Transparent complex words  (M+)
    &\centering Opaque complex words  (M$-$)
    &  ~\newline Simple\newline  words  (O)
    &  ~\newline Clippings  (C)
    &~\newline {Total N} (M$+$, M$-$, O)  \\
    \midrule
    \emph{-au} & 2 & 3 & 13 & 5 & 18\\
    \emph{-aud} & 20 & 15 & 11 & 35 & 46\\
    \emph{-aut} & 0 & 1 & 22 & 1 & 23\\
    \emph{-eau} & 74 & 47 & 74 & 121 & 195\\
    \emph{-od} & 0 & 0 & 1 & 0 & 1\\
    \emph{-op} & 0 & 0 & 4 & 0 & 4\\
    \emph{-os} & 0 & 0 & 179 & 0 & 0\\
    \emph{-ot} & 43 & 46 & 130 & 89 & 221\\
    \emph{-o} & 18 & 8 & 430 & 26 & 581\\
    \midrule
    {Total} & {157} & {120} & {864} &
    {277} & {1089}\\
    \lspbottomrule
    \end{tabularx} 
  \caption{Number of different words having the same ending.} % TODO: Ask author for caption
  \label{tab:Giraudo:1}
  \end{table}


    \begin{table} 
    \begin{tabularx}{\textwidth}{lSSSSS}
      \lsptoprule
      Ending & \multicolumn{5}{c}{Distribution of the cumulative frequency\is{frequency!cumulative frequency}} \\
      \midrule
    &\centering Transparent complex words  (M+)
    &\centering Opaque\newline complex\newline words  (M$-$)
    &  ~\newline Simple\newline  words  (O)
    &  ~\newline Clippings  (C)
    & ~\newline {Total N} (M$+$, M$-$, O)    \\
      \midrule
      \emph{-au} & 6.55 & 34.86 & 5350.20 & 41.41 & 5391.61\\
      \emph{-aud} & 2050.73 & 67.31 & 184.53 & 108.04 & 292.57\\
      \emph{-aut} & 0 & 0 & 2009.41 & 0.20 & 2009.61\\
      \emph{-eau} & 169.35 & 300.23 & 1559.39 & 427.10 & 1986.49\\
      \emph{-od} & 0 & 0 & 4.73 & 0 & 4.73\\
      \emph{-op} & 0 & 0 & 868.94 & 0 & 868.94\\
      \emph{-os} & 0 & 0 & 1596.89 & 0 & 1596.89\\
      \emph{-ot} & 1.80 & 4.05 & 1002.49 & 263.69 & 1493.18\\
      \emph{-o} & 1.17 & 1.05 & 703.47 & 29.48 & 1037.39\\
      \midrule
{Total} & {2229.60} & {407.50} & {13280.05} & {869.92} & {14681.41}\\
      \lspbottomrule
      \end{tabularx} 
  \caption{Cumulative frequencies of words having the same ending.} % TODO: Ask author for caption
    \label{tab:Giraudo:2}
    \end{table}

As one can see above, among the 9 possible transcriptions of the sound
{[}o{]}, 6 can correspond to suffixes (i.e., \emph{-au, -aud, -aut,
-eau, -ot, -o}). It means that 66\% of these endings can correspond to a
suffix. Moreover endings in {[}o{]} are globally carried by a larger
number of simple words (864 for O vs. 277 for M), and these simple words
are much more frequent than complex words (13280 occ./million for O vs.
870 occ./million for M).

If we examine the size distributions of the different transcriptions, it
appears that \emph{-o} represents more than a half of the overall
endings (581 words in \emph{-o} for a total of 1089 words ending in
{[}o{]}). The ending \emph{-eau} dominates among the other endings
(121/277 = .44) and only \emph{-eau} (121 complex words for 74 simple
words) and \emph{-aud} (35 complex words for 11 simple words) show a
morphological probability higher than an orthographic probability
(\emph{p}(M\emph{-eau}) = 121/195 = .62; \emph{p}(M\emph{-aud}) = 35/46
= .76). All the other endings are dominated by simple words. This means
that even if 66\% of {[}o{]} endings can function as suffixes, their
morphological probability is very low (\emph{p}(M) = 227/1084 = .21).
Therefore, morphological decomposition would conduct to a procedural
deadlock in 81\% of the cases. Finally, when the N distributions of M+
words are compared to M$-$ words, we can see that M+ globally dominates M$-$
(157 vs. 120) but when each ending is examined it appears that except
for \emph{-eau} (74 vs. 47) and \emph{-o} (18 vs. 8) it is more a
50/50 ratio than a clear dominance. It suggests than even when the
cognitive system encounters a complex word, morphological decomposition
is semantically useless in 50\% of the cases.

If one turns now to the details of frequency distributions, the
cumulative frequencies\is{frequency!cumulative frequency} of simple words are systematically higher than
those of complex words, the highest value being associated with simple
words ending in \emph{-au} (5350 occurrences per million). As for the N
distributions, the cumulated frequencies of the suffixed words ending in
\emph{-eau} dominates the other suffixed words (427 occ./million for a
total of 870 occ./million). M+ words are much more frequent than M$-$
words (2230 occ./million vs. 407 occ./million) but this dominance is
explained by the cumulated frequencies of M+ suffixed words in
\emph{-aud} (2051 occ./million). When the data of \emph{-aud} are
removed, the cumulated frequency of M$-$ words (340 occ./million ) becomes
almost twice as high as the one of M+ words (179 occ./million).
Altogether, this suggests that simple words and semantically opaque
complex words ending in {[}o{]} should be accessed more rapidly than the
semantically transparent complex ones.

To sum up, the reported study of the 9 possible transcriptions of
{[}o{]} according to the size and the cumulative frequency\is{frequency!cumulative frequency} reveals that
the probability for this phonological ending to correspond to a suffix
is low. More importantly, the cumulative frequency\is{frequency!cumulative frequency} of suffixed words
bearing a semantically transparent construction is weak relative to the
non-suffixed words. Consequently, a decomposition hypothesis\is{morphology!Decomposition Hypothesis} according
to which any item bearing a structured morphological surface is first
decomposed into morphemic constituents would lead to numerous useless
prelexical mechanisms.

\section{Something is rotten in the state of the decomposition
hypothesis}\label{something-is-rotten-in-the-state-of-the-decomposition-hypothesis}

In the present paper, I reviewed results from masked morphological
priming reported in the literature and I highlight the shortcuts made
by the decompositionalist to interpret some data, in particular those
related to formal effects, forgetting the semantic and the paradigmatic
aspects of morphology. Although I do not deny that morphology plays a
role during lexical access, I doubt that fast morphological effect\is{morphological effect} can
operate under masked priming\is{priming!masked} conditions (i.e., within a window of a
50--60 ms). In addition, I propose an alternative interpretation of its
role within the mental lexical

Recently, %
%Giraudo and Voga (2014) 
\citet{GiraudoVoga2014} %
%Giraudo-Voga
%
proposed a revised version of the
supralexical model\is{supralexical model}. This new model is  sensitive to both lexical (e.g.
frequency) and exo-lexical characteristics of the stimuli (e.g., family
size) and capable to cope with various effects induced by true
morphological relatives (e.g., \emph{singer--sing}) and pseudo relatives
(e.g., \emph{corner--corn}). According to the model, morphological
relationships are coded according to two different dimensions:
syntagmatic and paradigmatic. The first level captures the perceptive
regularity and the salience\is{salience} of morphemes within the language. It
contains stems and affixes that have been extracted during word
acquisition. Accordingly, during language acquisition, the most salient
perceptive units (i.e., recurrent and regular) will be caught and coded
by the cognitive system as lexical entries. At this very early level of
processing, morphologically complex words, pseudo-derived words and
nonwords whose surface structure can be divided into (at least two)
distinct morphemes are equally processed. As a consequence, this level
cannot properly be considered to be a morphological level, but rather as
a level containing morcemes\is{morceme} (from French \emph{morceau} `piece').
Morcemes\is{morceme} correspond to word pieces standing as access units that speed
up word identification each time an input stimulus activates one of
them. Therefore, there is no need to assume, at this stage, a process of
morphological decomposition; this would be unnecessary.

Contrary to the first level, the second level deals with the internal
structure of words, their formation according to morphological rules.
This level contains base lexemes, units abstract enough to tolerate
orthographic and phonological variations produced by the processes of
derivation and inflection. Base lexeme representations are connected to
morphologically related word representations and these connections are
determined by the degree of semantic transparency between wordforms and
base lexemes. Semantically transparent morphologically complex words are
connected both with their morphemes and their base lexeme. Words with a
semantically opaque structure, as for example, \emph{fauvette} `warbler'
(not related anymore to its free-standing stem \emph{fauve} `tawny') or
with an illusory structure, as for example \emph{baguette} `stick' in
which \emph{bagu-} is not a stem and has nothing to do with \emph{bague}
`ring', are not connected with a base lexeme. These two types of items
are only connected with their surface morphemes situated at the morceme\is{morceme}
level. Indeed, the model makes the fundamental assumption that
base lexeme representations are created in long-term memory according to
a rule that poses family clustering as an organisational principle of
the \isi{mental lexicon}. This rule stipulates that as soon as two words share
form and meaning, a common abstract representation emerges; all the
incoming forms respecting this principle then feed this representation.
In the course of language acquisition and learning, family size grows
and links are continually being strengthened.

Finally, if we turn back to priming effects, the model postulates that it
depends on the kind of relationships the prime entertains with the
target (formal and/or semantic) and consequently, on the number of
excitation sources that target recognition triggers: a) when the prime
is semantically transparent and complex M+O+S+ (like in the pairs
\emph{banker--bank} or \emph{hatched--hat}), its perception gives birth to
three sources of excitation, from morcemes\is{morceme}, wordforms and base lexemes;
b) when the prime is semantically transparent, complex but not
decomposable M+O$-$S+ (like in the prime target pair \emph{fell--fall}), it
activates two sources of excitation, from wordforms and base lexemes;
c) when the prime is semantically opaque M+O+S$-$ (it concerns complex or
pseudo-complex words like \emph{apartment--apart} or \emph{corner--corn}),
its recognition triggers two sources of excitation, from morcemes\is{morceme} and
wordforms; d) when the prime is not complex and not decomposable MO$-$S$-$
(like \emph{freeze--free}), it gives raise to only one source of
excitation, from wordforms.

In our view, much work still needs to be done on morphological
processing, but within the framework of a lexical network that codes word
representations as the result of both syntagmatic and the paradigmatic
influences. Separating form from meaning, words from their family and
series within experimental paradigms like the masked priming\is{priming!masked} paradigm
that exclusively focuses the attention of the readers on visual formal
aspects, leads to a confirmation bias and reduces the notion of
morphology to form only. It is indeed very important to consider that
masked priming\is{priming!masked} effects do not only correspond to the early processes of
lexical access as suggested by numerous authors, but to a picture of
lexical access that takes place at a given time within an ocean of
complex relationships.





\il{French|)}
\is{morpheme|)}
{\sloppy
\printbibliography[heading=subbibliography,notkeyword=this]
}


\end{document}
