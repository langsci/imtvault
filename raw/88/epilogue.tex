\addchap{Epilogue to the third edition}\ohead{Epilogue to the third edition}

More than thirty years after the original non-publication of \textit{Thoughts on grammaticalization}, a revised edition might seem overdue. Such a revision would, however, require adequate consideration of all the work published in the field during this period, which in turn would imply rewriting the entire text. (Some parts of it may deserve it that somebody take on this task.) Neither would it be easy to prolong the account of the research history offered in \chapref{chap:1} by these past decades, which have seen several times the amount of research on grammaticalization published in the preceding centuries. Instead of all this, I will confine myself to some topics treated in the present book. Some of these remarks are apologetic in nature, for which I ask the reader’s indulgence.

Research in grammaticalization over these decades has brought forward a large set of empirical insights into both the synchronic functioning and the diachronic dynamism of essential parts of the grammar of numerous languages over the world. Our understanding of the origins and workings of such fields of grammar as tense, aspect and mood, case relators, personal pronouns and indices, and of grammatical operations such as determination, auxiliation and negation continues to profit from the perspective afforded by grammaticalization theory. And sure enough, it is now permitted to speak of a theory of this strand of research, since the fundamentals of grammaticalization have become sufficiently clear over the years. The following points may deserve some attention:

A scientific definition of a concept mediates between theory and method. In order to assign the concept its place in a theory, the definition must presuppose a set of more elementary concepts and be compatible with another set of concepts akin to it. The concept of grammaticalization receives its place in the framework of a theory of language as a human activity bound up with the history and culture of a social community.  The main theoretical problem to be faced here is the dichotomy between a static and a dynamic view of language. Apart from certain restrictions which I will return to, grammaticalization is, in a dynamic view of language, what grammar is in a static view. A definition of grammaticalization whose definiens includes the concept of grammar bases a dynamic concept on the static view and is bound to fail to the extent that grammaticalization is precisely the kind of process which traverses the boundary between grammar and such aspects of the use of linguistic signs as remain outside grammar, like discourse and the lexicon. It therefore seems more profitable to base this definition on more general and possibly dynamic concepts of linguistic activity. It is for this reason that grammaticalization of a linguistic sign has been conceived as reduction of the speaker’s freedom in using it, which implies its increasing subjection to constraints specific of the particular language.

The methodological role of a definition is fulfilled by its operationalization. Since grammaticalization is a gradual process, operationalization of this concept does not answer the question whether something is or is not grammaticalized, but to what degree it is grammaticalized. Methods must be deduced from the theory, and \sectref{sec:4.1} goes to some lengths to provide such a foundation. The freedom of the speaker is converted into the autonomy of the sign, and this is measured by a set of parameters. These must be as formal as possible, not only in the interest of methodological reliability and validity, but also in the sense of pertaining to linguistic structure rather than substance. This, again, follows from the nature of the concept: the increasing subjection of the grammaticalized unit to rules of the particular language implies its increasing contribution to linguistic structure rather than to the sense of the message. Attempts to capture grammaticalization by semantic criteria have therefore failed. Instead, the six parameters spelling out the autonomy of the linguistic sign have proved their applicability in a large number of empirical cases. 

Grammaticalization starts from different sources, runs through different phases and takes different forms constrained by the type of language in which it occurs. It has been clear from the beginning that the six parameters do not always run in parallel; nor do they need to do so, for the reasons just mentioned. The concept must therefore be a prototypical one (cf. \citealt{Wiemer2014}). Examples of the prototype are provided by the ‘going-to’ future in English and other languages or by the development of a dative construction out of a benefactive construction, e.g. in current Portuguese. The transition of a derivational morpheme to an inflectional one, for instance, is more peripheral, since syntagmatic cohesion does not grow.

In order to model the prototypicality of the concept, one might consider as prototypical cases of grammaticalization such cases in which all six parameters correlate and accept as marginal such cases in which at least half of them correlate. This, however, seems a bit mechanistic. One may reach greater validity here by realizing that the three janus-headed principal parameters relate to the idea of the structural autonomy of the linguistic sign to different degrees. The parameter most directly reflecting the speaker’s freedom in using a sign is its variability. Enhanced cohesion may be considered as a symptom of reduced variability. At any rate, these two parameters are purely structural ones. The parameter of weight is more problematic.\footnote{This subdivision of the three principal parameters is first suggested in \citet{Lehmann1995}.} On the one hand, it is internally not quite consistent, as integrity is not exactly the paradigmatic counterpart of structural scope. On the other, integrity is not a purely structural parameter, since it tries to capture the substance of the sign in formal terms. Consequently, until the integrity parameter is repaired, the three principal parameters may be ranked as follows:

%\setcounter{itemize}{0}
\begin{enumerate}
\item Variability
\item Cohesion
\item Weight
\end{enumerate}

\noindent The operationalization of a prototypical concept of grammaticalization may then be based on this ranking of the criteria.

The most encompassing concept which one might term ``grammaticalization'' is probably the process by which anything becomes part of grammar. This, however, is not a unitary process, since there are many different ways in which this may happen. Such a wide concept would include, among many other things, the morphologization of a phonological contrast. There are quite a number of processes of grammatical change which bear paradigmatic and syntagmatic relations to grammaticalization and which have to be excluded from the latter concept simply because there are empirical cases in which any one of these processes occurs in isolation. Worth of particular mention here are analogical change and reanalysis. These are important in grammatical variation. However, they differ from grammaticalization in that they do not (necessarily) reduce the autonomy of the sign. And they may or may not cooccur, in a given historical case, with each other or with grammaticalization. Consequently, grammaticalization cannot be reduced to any other process of grammatical variation.

The two main aspects of a complete process of grammaticalization are the recruitment of new material for grammatical function and the subjection of this material to rules of grammar leading to its final reduction to zero. For convenience, these may be described as two subsequent phases in a diachronic perspective. As a matter of fact, however, even the first deployment of a lexical formative or a free construction to fulfill a grammatical function already involves an incipient subjection to rules of grammar. Anyway, the two aspects reflect the two main forces driving grammaticalization: Extravagance (\citealt{Haspelmath1999} after \citealt{Lehmann1985a}) is responsible for the incessant re-entry into the spiral; automatization (\citealt{Lehmann2004} after \citealt{Givón1989}) is responsible for its incessant passage down to the pole of pure structural function. Both of these forces are directed, which means that grammaticalization takes similar forms in different languages and different areas of grammar. Each of them has a logical opposite, viz. understatement (or parsimony) and reflection; but these are far from being antagonists on a par with the former two.

The question of the unidirectionality of grammaticalization has caused much stronger a stir in the linguistic community than any other aspect of this phenomenon. One of the most hotly debated issues in the theory of grammaticalization concerns the existence or otherwise of degrammaticalization, i.e. the inverse process to grammaticalization. While most specialists agree that most of the evidence for degrammaticalization adduced so far is defective in one or another way and that very few if any examples of degrammaticalization stand up to closer scrutiny, let alone emerge as prototypical instances of this concept, this state of affairs seems theoretically interesting only if it can be turned into a theorem. The theorem says: Grammaticalization is a unidirectional process; a process running in the inverse direction of grammaticalization, thus, does not exist.

This thesis should be understood as an empirical claim (cf. \citealt[§4.2]{Lehmann2004}). As such, it may be true or false. The idea that it may be true is apparently so provocative that it has triggered a wealth of literature aimed to falsify it. By common standards of scientific dispute, the falsification of a thesis is all the more impressive if it has been seriously upheld in the specialized literature (preferably, by somebody else). Thus, a \textit{locus classicus} for the thesis is sought. This has been detected, more than once, in \textit{Thoughts on grammaticalization}. Here are four representative references to previous editions:

\begin{quote}
a search for counterevidence [to unidirectionality], a task that Lehmann regards as largely futile given that grammaticalization is “an irreversible process” (1995:16).

\citep[569]{Howe2010}
\end{quote}

\begin{quote}\enlargethispage{2\baselineskip}
the term [degrammaticalization] was introduced by Lehmann for a phenomenon which he believed to be non-existent:

``Various authors (Givón 1975:96, Langacker 1977:103f, Vincent 1980[I]:56--60) have claimed that grammaticalization is unidirectional; that is, an irreversible process [...] there is no \textit{degrammaticalization}.” (Lehmann 1995 [1982]:16, emphasis original)

\citep[123]{Norde2010}
\end{quote}

\begin{quote}
However, unidirectionality has also been defined as a constraint on grammatical change in general (Heine, Claudi, and Hünnemeyer 1991; Heine 1994 and 1997; Lehmann 1995 [1982]; ...)\\
\citep[50]{Norde2009}
\end{quote}

\begin{quote}
Lehmann (1995a[1982]: 16) takes change of this type to be unidirectional and is thus led to claim that degrammaticalization does not exist.\\
\citep[163]{BörjarsEtAl2011}
\end{quote}

\noindent And here is the actual wording of the paragraph in question:

\begin{quote}
Various authors (Givón 1975:96, Langacker 1977:103f, Vincent 1980[I]:56--60) have claimed that grammaticalization is unidirectional; that is, it is an irreversible process, the scale in F1 cannot be run through from right to left, there is no \textit{degrammaticalization}. %\todo{changed bold face to italics}
Others have adduced examples in favor of degrammaticalization. The few that have come to my knowledge will be briefly discussed.\\
(\citealt[14]{Lehmann2002b}; unchanged from the earlier editions)
\end{quote}

\noindent The passage, thus, does not propose the thesis in question and instead ascribes it to other researchers. The quotations adduced before are, therefore, perfect examples of inadmissible quotations taken out of context and of misrepresentation of published views, respectively.

What the text, admittedly, does say towards the end of the discussion introduced by the previous quotation, is the following:

\begin{quote}
We may therefore conclude this discussion with the observation that no cogent examples of degrammaticalization have been found.\\
\citep[17]{Lehmann2002b}
\end{quote}

\noindent This, again, may or may not be a misrepresentation of the research situation obtaining in 2002; but it is obviously not a theoretical claim on the non-existence of degrammaticalization.

%\setcounter{page}{1}
Just to forestall any misunderstandings: The current state of research is essentially the same. I.e., the thesis of the non-existence of degrammaticalization is an empirical hypothesis which has not yet been thoroughly falsified. Some examples have been adduced in the literature (in particular, in \citealt{Norde2009}) that come rather close to being empirical evidence of degrammaticalization. Should a completely convincing case be found – something that no current theory is in a position to exclude –, then it would merit considerable interest. The theory of grammaticalization, however, would be only marginally affected. Empirical linguistic theorems are generally subject to a couple of exceptions – language is an activity of human beings, who (fortunately) sometimes oppose the general trend.

At the time of this writing, the most recent tendency in the field is to marry grammaticalization with construction grammar. Here again, some authors have misrepresented the approach taken in this book. It has been characterized as focused on the isolated morpheme or word and been opposed to “constructionalization”, assumed to be taking the appropriate broader perspective. These critics apparently have failed to map the traditional structuralist concept of the set of syntagmatic relations of a linguistic unit relied on in \chapref{chap:4} onto the contemporary notion of the construction. Neither have they, apparently, appreciated \sectref{sec:4.4}, which proposes a passage in step of the grammaticalization of a linguistic sign and of the construction containing it. Construction grammar is a conception of grammar which pays equal attention to the structure and meaning of complex linguistic signs and consequently does recommend itself as a model for the description of the steps involved in grammaticalization in a language.

On the other hand, all the cases of grammaticalization discussed in this book have it in common that they do contain some item which becomes a (more) grammatical formative by grammaticalization. In later publications \citep[§2.3]{Lehmann2002b}, I have entertained the possibility of the grammaticalization of a construction irrespectively of the presence of a particular item which may be said to be grammaticalized. This may or may not be a fruitful extension of the concept. However, two things should be noted: First, it is not clear how the parameters of grammaticalization would apply to such cases. The parameters themselves would have to be suitably extended or be replaced by better ones. Second, every extension of the concept of grammaticalization runs the risk of depriving it of its force to identify a genuine phenomenon in linguistic life which cannot be reduced to anything else.

Some construction grammarians have offered a new edition of the behavior pattern known from the model ousted by construction grammar, viz. generative grammar: previous approaches are either misguided or superfluous. Thus, we don’t need grammaticalization, since all it can tell us is taken care of by our constructionalization. This replicates the earlier generativist rhetorical figure: Grammaticalization is not needed, since our device of reanalysis is more than sufficient to account for the phenomena in question. These critics have failed to understand the nature of a monotonous directed variation. Consider the following numerical sequence: 0 1 3 6 10. Mathematician A has mastered elementary arithmetic. He will say: You have to analyze this as a set of pairs \textit{x} and \textit{y}. The relationship between the members of the first pair is: \textit{x} + 1 = \textit{y}. The relationship in the second pair is: \textit{x} + 2 = \textit{y}. In the third pair, it is \textit{x} + 3 = \textit{y}. And in the final pair, the relationship is \textit{x} + 4 = \textit{y}. Mathematician B has mastered analysis. He accepts such a binary subdivision as a first step in understanding the principle of the numerical sequence, but then goes on: This is a strictly monotonically increasing sequence, generated by the function \textit{x\textsubscript{i}} + \textit{i} = \textit{y}. Faced with a phenomenon like the evolution of the English immediate future out of a motion-cum-purpose construction, the grammarian – no matter whether the generative or the construction grammarian – acts like mathematician A, applying their analytic devices of reanalysis or constructionalization to every single step in the diachronic sequence. The grammaticalizationist acts like mathematician B, applying the concept of grammaticalization to the entire sequence and thus capturing its principle. Grammaticalization theory will only be truly supplanted by a theory embodying this kind of higher-level principle.
