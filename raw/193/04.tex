\chapter {Attributive Comparative Deletion} \label{ch:4}
\section {Introduction} \label{sec:4introduction}
This chapter aims at providing an adequate explanation for the phenomenon of \isi{Attributive Comparative Deletion}, as attested in \ili{English}, by way of relating it to the regular mechanism of \isi{Comparative Deletion} described in \chapref{ch:3}. I will show that \isi{Attributive Comparative Deletion} can only be understood as a descriptive term referring to a phenomenon that is a result of the interaction of more general syntactic processes; in other words, there is no reason to postulate any special mechanism underlying \isi{Attributive Comparative Deletion} in the grammar. Eliminating such a mechanism will allow one to achieve a unified analysis of all types of comparatives. On the other hand, \isi{Attributive Comparative Deletion} is not a universal phenomenon: I will show that its appearance in \ili{English} can be conditioned by independent, more general rules and that the absence of such restrictions may lead to the absence of \isi{Attributive Comparative Deletion} in other languages. Again, I will first review some of the existing analyses, partly because in certain respects I will strongly rely on them and partly because the advantages of my proposal can best be understood when measured against these ones.

\section{Earlier accounts} \label{sec:4earlier}
\subsection{The problems to be discussed} \label{sec:4problems}
\isi{Attributive Comparative Deletion} is a peculiar phenomenon that involves the obligatory \isi{deletion} of the quantified \isi{AP} and the \isi{lexical verb} from the comparative \isi{subclause} if the quantified \isi{AP} functions as an attribute within a \isi{nominal expression}. Consider the examples in (\ref{acdparadigm4}):

\ea \label{acdparadigm4}
\ea [] {Ralph bought a bigger cat than George did \sout{buy} a \sout{big} cat flap.} \label{acd}
\ex [] {Ralph bought a bigger cat than George \sout{bought} a \sout{big} cat flap.} \label{acdlex}
\ex [*] {Ralph bought a bigger cat than George bought a \sout{big} cat flap.} \label{adjdel}
\ex [*] {Ralph bought a bigger cat than George bought a big cat flap.}
\ex [*] {Ralph bought a bigger cat than George \sout{bought} a big cat flap.}
\ex [*] {Ralph bought a bigger cat than George did \sout{buy} a big cat flap.} \label{verbdel}
\z
\z

As can be seen, both the \isi{adjective} (\textit{big}) and the \isi{lexical verb} (\textit{buy}) have to be eliminated from the comparative \isi{subclause}; this is possible either by eliminating the tensed \isi{lexical verb}, as in (\ref{acdlex}), or by deleting the \isi{lexical verb} and leaving the auxiliary \textit{do} bearing the \isi{tense} morpheme intact, as in (\ref{acd}). Note that both the \isi{verb} and the \isi{adjective} have to be deleted, as indicated by the ungrammaticality of (\ref{adjdel})--(\ref{verbdel}).

Furthermore, the obligatory elimination of the \isi{adjective} is not merely due to the fact that it is \textsc{given}; the overt presence of the \isi{attributive adjective} is ungrammatical even if it is different from its matrix \isi{clausal} counterpart, as shown by (\ref{ralphcat4}):

\ea \label{ralphcat4}
\ea [*]{Ralph bought a bigger cat than George \sout{bought} a wide cat flap.}
\ex [*]{Ralph bought a bigger cat than George did \sout{buy} a wide cat flap.}
\z
\z

It seems that the elimination of the \isi{adjective} from that particular position is obligatory.

On the other hand, note that the \isi{deletion} of the \isi{lexical verb} is required only if part of the \isi{DP} is overt; in case the entire \isi{DP} is eliminated, as in (\ref{ralphbought4}), the \isi{lexical verb} can stay:

\ea Ralph bought a bigger cat than George bought \sout{a big cat}. \label{ralphbought4}
\z

A number of questions arise in connection with these phenomena. First, it has to be explained why the \isi{adjective} has to be deleted and cannot appear overtly even if it is contrastive. Second, one has to account for the fact that the \isi{deletion} of the \isi{adjective} happens alongside the \isi{deletion} of the \isi{lexical verb}: apart from answering the question why this should be so, the issue of how this can be carried out also has to be addressed since in structures like (\ref{acd}) and (\ref{acdlex}) the \isi{adjective} and the \isi{lexical verb} do not seem to be adjacent. In other words, though the strong interrelatedness of the elimination of both these elements suggests that they are deleted by one and the same process, their apparently distinct positions also raise the possibility of there being two separate processes at hand -- if so, one has to explain why and how these are interrelated.

In addition, the relation of \isi{Attributive Comparative Deletion} to ordinary \isi{Comparative Deletion} also has to be addressed; the fact that in structures such as (\ref{acd}) and (\ref{acdlex}) it is the lower \isi{copy} that may remain overt suggests that CD takes place regularly in these structures too -- if so, however, one has to account for the differences attested in the extent to which lower copies may remain overt.

Furthermore, the analysis of \isi{Attributive Comparative Deletion} also has to take cross-linguistic differences into consideration. For instance, in languages like \ili{Hungarian} the full structure may be visible in the \isi{subclause}, as shown by (\ref{ch4acdhung}):

\ea \label{ch4acdhung}
\gll Rudolf nagyobb macsk\'at vett, mint amilyen sz\'eles macskaajt\'ot Mikl\'os vett.\\
Rudolph bigger cat.\textsc{acc}	bought.\textsc{3sg} than how wide cat.flap.\textsc{acc}	Mike	bought.\textsc{3sg}\\
\glt `Rudolph bought a bigger cat then Mike did a cat flap.'
\z

On the other hand, languages such as \ili{German} do not permit \isi{Attributive Comparative Deletion}, as shown in (\ref{ralfwohnung4}):

\ea [*]{\gll  Ralf hat eine gr\"o{\ss}ere Wohnung als Michael ein Haus. \label{ralfwohnung4}\\
Ralph has a.\textsc{acc.f} bigger.\textsc{acc.f} flat than Michael a.\textsc{acc.n} house\\
\glt `Ralph has a bigger flat than Michael a house.'}
\z

Therefore, a sound analysis for \isi{Attributive Comparative Deletion} should account for cross-linguistic variation, besides providing an adequate explanation for the \ili{English} data.

In what follows I will briefly review two analyses concerning \isi{Attributive Comparative Deletion}. The first one is that of \citet{kennedymerchant2000}, who provided the most detailed description of the phenomenon in \ili{English}, also successfully explaining a number of related issues and making the occurrence of \isi{Attributive Comparative Deletion} partially predictable in cross-linguistic terms. Second, I will also review the article by \citet{reglero2006}, which makes use of the analysis by \citet{kennedymerchant2000} by extending it to \ili{Spanish}, thus providing important insights into cross-linguistic variation in this respect.

\subsection{Attributive modification -- \citet{kennedymerchant2000}} \label{sec:4attributivemodification}
Starting from the observation made by \citet{pinkham1982diss, pinkham1985}, \citet[91--92]{kennedymerchant2000} point out that \isi{Attributive Comparative Deletion} proves to be a challenge to \isi{deletion} analyses for \isi{Comparative Deletion} since ``in comparatives involving attributive adjectives, CD cannot target just the corresponding \isi{AP} in the \isi{comparative clause}.'' Consider the following example (\citealt[92, ex. 7a]{kennedymerchant2000}):

\ea [*] {Pico wrote a more interesting novel than Brio wrote a \underline{\hspace{0.5cm}} play.} \label{pico}
\z

As \citet[92]{kennedymerchant2000} argue, any analysis treating \isi{Comparative Deletion} as an unbounded \isi{deletion} process targeting left-branch constituents (cf. \citealt{bresnan1975}) would face a serious problem here, in that attributive APs are canonical left-branch constituents and yet they cannot be deleted in constructions such as (\ref{pico}). In other words, such an analysis would predict (\ref{pico}) to be grammatical, which is clearly not the case.

One of the fundamental claims made by \citet[103]{kennedymerchant2000} is that the derivation of ill-formed attributive CD constructions contains left-branch \isi{extraction} in the same way it happens in main clause \textit{wh}-questions. The examples in (\ref{polly4}) are essentially ruled out for the same reason (\citealt[103, exx. 25 and 26]{kennedymerchant2000}):

\ea \label{polly4}
\ea [*] {Erik drives a more expensive car than Polly drives \textbf{a motorcycle}.} \label{pollyacd}
\ex [*] {\textbf{How expensive} does Polly drive \textbf{a motorcycle}?} \label{pollyint}
\z
\z

In both cases, a \isi{DegP} is claimed to move out to a [Spec,\isi{CP}] position from within the \isi{DP}. This \isi{DegP} is phonologically null in comparative subclauses such as in (\ref{pollyacd}), see \citet[102--103]{kennedymerchant2000}.

The prediction is that languages that allow left-branch \isi{extraction} in questions like (\ref{pollyint}) should also allow constructions such as (\ref{pollyacd}), whereas languages that do not should have them. This prediction is borne out: \ili{Polish} and \ili{Czech} allow constructions like (\ref{pollyacd}) and (\ref{pollyint}) alike, while \ili{Bulgarian} and \ili{Greek} do not (\citealt[104--109]{kennedymerchant2000}). Note that the unavailability of left-branch \isi{extraction} in \ili{Greek} (and \ili{Polish}) is true for the constructions discussed here and does not necessarily have to hold for other structures. As far as \ili{Greek} is concerned, it is known that \ili{Greek} does allow certain left-branch extractions, cf. \citet[281]{uriagereka2006}, based on \citet{corver1992} and \citet{horrocksstavrou1987}. (On the relation between the article and the availability of \isi{extraction} in \ili{Greek}, see also \citealt{boskovic2005, boskovic2012}.)

Consider the examples in (\ref{polishacdint4}) from \ili{Polish} (\citealt[104, exx. 29 and 31a]{kennedymerchant2000}):

\ea \label{polishacdint4}
\ea \gll \textbf{Jak} \textbf{d\l{}ug\k{a}} \textbf{sztuk\k{e}} napisa\l{} Pawe\l{}? \label{polishint}\\ 
how long play wrote Pawel\\
\glt `How long a play did Pawel write?'
\ex \gll \textbf{Jak} \textbf{d\l{}ug\k{a}} napisa\l{} Pawe\l{} \textbf{sztuk\k{e}}? \label{polishintstrand}\\
how long wrote Pawel play\\
\glt `How long a play did Pawel write?'
\ex \gll Jan napisa\l{} d\l{}u\.{z}szy list, ni\.{z} Pawe\l{} napisa\l{} \textbf{sztuk\k{e}}. \label{polishacd}\\
Jan wrote longer letter than Pawel wrote play\\
\glt `Jan wrote a longer letter than Pawel wrote a play.'
\z
\z

As can be seen, \ili{Polish} allows both the \isi{extraction} of the entire \isi{nominal expression}, as in (\ref{polishint}), or the \isi{extraction} of the \isi{DegP} attribute from within that \isi{nominal expression}, as in (\ref{polishintstrand}); the availability of (\ref{polishintstrand}) predicts that (\ref{polishacd}) should be grammatical, which is indeed the case (\citealt[104]{kennedymerchant2000}).

By \isi{contrast}, consider the data in (\ref{petrosauto4}) from \ili{Greek} (\citealt[106--107, exx. 35 and 37a]{kennedymerchant2000}):

\ea \label{petrosauto4}
\ea [] {\gll \textbf{Poso} \textbf{megalo} \textbf{aftokinito} agorase o Petros? \label{greekint}\\
how big car bought the Petros\\
\glt `How big a car did Petros buy?'}
\ex [*] {\gll \textbf{Poso} \textbf{megalo} agorase o Petros ena \textbf{aftokinito}? \label{greekintstrand}\\
how big bought the Petros a car\\
\glt `How big a car did Petros buy?'}
\ex [*] {\gll O Petros agorase ena megalitero aftokinito apoti o Giannis agorase \textbf{ena} \textbf{dzip}. \label{greekacd}\\
the Petros bought a bigger car than.what the Giannis bought a jeep\\
\glt `Petros bought a bigger car than Giannis did a jeep.'}
\z
\z

Unlike in \ili{Polish}, the \isi{extraction} of the \isi{DegP} out of a \isi{nominal expression} is not allowed, as demonstrated by the ungrammaticality of (\ref{greekintstrand}): only \isi{movement} together with the rest of the \isi{DP} is allowed, as in (\ref{greekint}). The fact that (\ref{greekacd}) should be ungrammatical is predictable from the ungrammaticality of (\ref{greekintstrand}).

Another prediction is that the elimination of the \isi{lexical verb} or of the \isi{noun} in constructions like (\ref{greekacd}) should result in grammatical configurations, just as in the case of \ili{English}. This is again fulfilled, as shown by the \ili{Greek} data given in (\ref{greekcomparative4}) below (\citealt[108, ex. 39]{kennedymerchant2000}):

\ea \label{greekcomparative4}
\ea \gll O Petros agorase ena megalitero aftokinito apoti agorase o Giannis. \label{greekellipsis}\\
the Petros bought a bigger car than.what bought the Giannis\\
\glt `Petros bought a bigger car than Giannis bought.'
\ex \gll O Petros agorase ena megalitero aftokinito apoti o Giannis. \label{greeksluicing}\\
the Petros bought a bigger car than.what the Giannis\\
\glt `Petros bought a bigger car than Giannis did.'
\ex \gll O Petros agorase ena megalitero aftokinito apoti o Giannis \underline{\hspace{0.5cm}} ena dzip. \label{greekgap}\\
the Petros bought a bigger car than.what the Giannis {} a jeep\\
\glt `Petros bought a bigger car than Giannis did a jeep.'
\z
\z

In (\ref{greekellipsis}) the entire \isi{nominal expression} containing the \isi{DegP} is removed from the \isi{subclause}, while in (\ref{greeksluicing}) the \isi{finite verb} is also eliminated; in turn, in (\ref{greekgap}) only the \isi{finite verb} is absent. Most importantly, all of these constructions are grammatical and the same observation holds for \ili{Bulgarian} (\citealt[108--109]{kennedymerchant2000}).

What follows from all this is that constructions like (\ref{pico}) are ruled out because they violate the Left Branch Condition (\citealt[110]{kennedymerchant2000}). As pointed out by \citet[109--116]{kennedymerchant2000}, the Left Branch Condition is essentially a \isi{PF} constraint: the acceptability of the elliptical counterparts of constructions like (\ref{pico}) show that (\ref{pico}) cannot be ruled out by \isi{LF}.

Essentially, \citet{kennedymerchant2000} claim that the operation responsible for \isi{ellipsis} is VP-\isi{deletion}. First of all, they adopt the view that pseudogapping is in fact an instance of VP-\isi{deletion}, such that there is some additional mechanism that saves the remnant (\citealt[121]{kennedymerchant2000}, based on \citealt{kuno1981}). Consider the example  in (\ref{pizza4}) for pseudogapping (\citealt[121, ex. 60a]{kennedymerchant2000}):

\ea I eat pizza, but I don't seafood. \label{pizza4}
\z

In this case, there is a \isi{DP} remnant (\textit{seafood}) in the second conjunct; \citet[121--122]{kennedymerchant2000} adopt the view formulated by \citet{jayaseelan1990} and \citet{johnson1997} that the \isi{DP} moves out of its \isi{base position} within the \isi{VP} and is right-adjoined to the VP-node. Hence the structure of the string \textit{but I don't seafood} should be as given in (\ref{treepizza4}), based on \citet[122, ex. 61]{kennedymerchant2000}:\footnote{Note that \citet{kennedymerchant2000} treat \textit{don't} as a single inflection head and do not postulate a separate NegP; this may be a problem in itself, but since it has no bearing on the analysis, I will not attempt to provide an alternative to this later on either.}

\ea \upshape \label{treepizza4}
\begin{forest} baseline, qtree, for tree={align=center} 
[IP
	[DP\textsubscript{\textit{subj}}
		[I]
	]
	[I$'$
		[I
			[don't]
		]
		[VP
			[\sout{VP} [\textit{t\textsubscript{subj}}] [\sout{V$'$} [\sout{V} [eat]] [\textit{t\textsubscript{remnant}}]]]
			[DP\textsubscript{\textit{remnant}} [seafood,roof]]
		]
	]
]
\end{forest}
\z

The same is claimed to take place in attributive comparatives; however, if the \isi{DP} is moved to the right, the \isi{degree expression} moves alongside with it, and therefore it could not be deleted (cf. \citealt[122--124]{kennedymerchant2000}).

In order to overcome this problem, \citet[124--130]{kennedymerchant2000} propose a revised analysis for the syntax of \isi{attributive modification}. As argued by \citet[124]{kennedymerchant2000}, certain DegPs\footnote{\citet{kennedymerchant2000} treat the bracketed constituents in (\ref{fpmove}) as DegPs throughout their paper. However, based on the analysis in \chapref{ch:2}, they should rather be treated as QPs.} modifying nominal expressions end up in an inverted position. Consider the following examples (\citealt[124, exx. 65a, 66a, and 66c]{kennedymerchant2000}):

\ea \label{fpmove}
\ea \textbf{{[}How interesting a play]} did Brio write?
\ex I ate \textbf{[too big a piece]}.
\ex Bob didn't write \textbf{[as detailed a proposal]} as Sheila did.
\z
\z

As \citet[129--130]{kennedymerchant2000} note, based on \citet{bresnan1973}, there is considerable variation as to which \isi{degree} expressions must, can and cannot undergo inversion. The point is that if the \isi{DegP} does move up to a position within the \isi{nominal expression}, the uninterpretable [+wh] \isi{feature} of the \isi{DegP} -- which is involved in Left Branch Condition effects -- is transferred to some \isi{functional head} in the nominal projection (\citealt[124]{kennedymerchant2000}).

The \isi{functional projection} of this head is right above the \isi{DP}, and is referred to as \isi{FP} by \citet[124--125]{kennedymerchant2000}. The structure of the string \textit{how interesting a play} is as follows (\citealt[125, ex. 67]{kennedymerchant2000}):

\ea \label{fptree} \upshape
\begin{forest} baseline, qtree, for tree={align=center}
[FP
	[DegP\textsubscript{\textit{i}}
		[how interesting,roof ]
	]
	[F$'$
		[F]
		[DP
			[D [a]]
			[NP [\textit{t\textsubscript{i}} play,roof ]]
		]
	]
]
\end{forest}
\z

As can be seen, the \isi{DegP} moves up to the [Spec,\isi{FP}] position from within the \isi{NP}, thus producing an inverted \isi{word order}.

There are arguments in favour of such an analysis. First, in certain dialects the F position may be overtly filled by \textit{of} (\citealt[125--126]{kennedymerchant2000}; cf. also \citealt{bolinger1972, abney1987diss, bowers1987}). Consider the examples in (\ref{brio4}), taken from \citet[125--126, exx. 68a, 69a and 69c]{kennedymerchant2000}:

\ea \label{brio4}
\ea \textbf{{[}How long of a novel]} did Brio write?
\ex I ate \textbf{[too big of a piece]}.
\ex Bob didn't write \textbf{[as detailed of a proposal]} as Sheila did.
\z
\z

Second, there are certain ambiguities that can be explained only by accepting that the \isi{DegP} may move to a [Spec,\isi{FP}] position. This is demonstrated by the following set of examples (\citealt[127, ex. 70]{kennedymerchant2000}):

\ea \label{coordkennedymerchant}
\ea I have written a successful play, but you have \underline{\hspace{0.5cm}} a novel. \label{coordambig}
\ex I have written a successful play, but you have written a novel. \label{coordverb}
\ex I have written a successful play, but you have written a successful novel. \label{coordadj}
\z
\z

The sentence in (\ref{coordambig}) is ambiguous between the two readings paraphrased in (\ref{coordverb}) and (\ref{coordadj}); cf. \citet[127]{kennedymerchant2000}. As far as the one in (\ref{coordverb}) is concerned, it is completely unsurprising: under the analysis proposed by \citet{kennedymerchant2000}, what happens here is that ``the remnant \isi{DP} is removed from \isi{VP}, and the \isi{VP} is deleted'' (\citealt[128]{kennedymerchant2000}). By \isi{contrast}, the reading given in (\ref{coordadj}) is unexpected inasmuch as \isi{deletion} ``appears to be `reaching inside' the remnant \isi{DP} to delete the attributive \isi{modifier} along with \isi{VP}'' (\citealt[128]{kennedymerchant2000}). The way to overcome this apparent problem is to adopt the representation in (\ref{fptree}) for structures like (\ref{coordambig}): in that case, the \isi{VP} and the attributive \isi{modifier} are adjacent at \isi{PF} -- according to \citet[129--130]{kennedymerchant2000}, the \isi{DP} moves out of the \isi{FP}, leaving the \isi{DegP} in [Spec,\isi{FP}] behind within the \isi{VP} that counts as the \isi{extraction} site.

Essentially, the same is claimed to happen in the case of attributive comparative structures (\citealt[130--134]{kennedymerchant2000}). The F head has to be eliminated because \ili{English} lacks a [+wh] F head in the lexicon, as opposed to a [+wh] D head, which does exist (\citealt[130]{kennedymerchant2000}). The [+wh] \isi{feature} is uninterpretable on the F head at \isi{PF}; however, if \isi{deletion} takes place, then it also eliminates this \isi{feature} (\citealt[131]{kennedymerchant2000}). On the other hand, since the \isi{DP} may scramble out of the \isi{FP}, the \isi{DP} itself is not affected by \isi{deletion} (\citealt[131]{kennedymerchant2000}).

Consider the following example (\citealt[131, ex. 77]{kennedymerchant2000}):

\ea Pico wrote a more interesting novel than he did a play. \label{piconovel}
\z

The processes taking place in (\ref{piconovel}) are summarised in (\ref{remnantmovetree}); cf. \citet[132, ex. 78]{kennedymerchant2000}:

\ea \label{remnantmovetree} \upshape
\begin{forest} baseline, qtree, for tree={align=center}
[CP
	[Op.\textsubscript{\textit{i}}]
	[C$'$
		[C]
		[IP
			[DP [he]]
			[I$'$
				[I [did]]
				[VP
					[\sout{VP}
						[\textit{\sout{t}\textsubscript{subj}}]
						[\sout{V}$'$ 
							[\sout{V} [\sout{write}]]
							[\sout{FP} [\textit{\sout{t}$'$\textsubscript{i}}] [\sout{F}$'$ [\sout{F}\textsubscript{\sout{{[}+wh{]}}}] [\textit{\sout{t}\textsubscript{remnant}}]]]
						]
					]
					[DP\textsubscript{\textit{remnant}} [D [a]] [NP [\textit{t\textsubscript{i}} play,roof ]]]
				]
			]
		]
	]
]
\end{forest}
\z

As can be seen, the \isi{DP} moves rightwards and is adjoined to the \isi{VP} node; in turn, the lower \isi{VP} node is deleted, alongside with the \isi{FP} within it.

The analysis has its advantages, especially as far as the syntax of \isi{attributive modification} is concerned, and also because \isi{verb} \isi{gapping} is treated as an instance of VP-\isi{deletion} and not as a special process. In this respect, \citet{kennedymerchant2000} strongly rely on the results of \citet{kuno1981}, \citet{sag1976diss}, \citet{levin1986}, \citet{miller1992}, \citet{jayaseelan1990}, \citet{lasnik1995} and \citet{johnson1997}; but cf. also \citet{coppock2001} and \citet{johnson2004} for more recent analyses.

However, there are two main problems that arise in connection with the general mechanism of VP-\isi{ellipsis}. First, the \isi{rightward movement} of the \isi{DP} is unmotivated; moreover, \isi{rightward movement} -- within a minimalist framework -- is questionable in itself. Second, if VP-\isi{ellipsis} targets a VP-constituent, it remains also unexplained what mechanism may select only the lower \isi{VP} node.

In addition, there are two further problems concerning the application of this framework to attributive comparative structures. On the one hand, the \isi{DP} moves from within the \isi{FP}; however, there is no example in any analogous structure for the \isi{DP} to move out -- to the right -- from its own functional extension generated this way: a sequence such as *\textit{how big did you see a cat} is not grammatical either. On the other hand, the \isi{movement} of the \isi{operator} as indicated in (\ref{remnantmovetree}) is not valid, chiefly because \ili{English} has no other structures where the \isi{QP} containing the \isi{operator} moves out from within the \isi{FP} -- thus the sequence *\textit{how big did you see a cat} is obviously not grammatical if we do not suppose the \isi{DP} to be moving to the right either. At the same time, it would be rather ad hoc to assume that the \isi{QP} containing the \isi{operator} is phonologically empty in attributive structures: as was shown in \chapref{ch:3}, in predicative structures the \isi{QP} contains a phonologically visible \isi{AP}, and there is no reason for supposing that there would be a difference in the internal structure of the \isi{QP} between predicative and attributive structures.

In sum, though the proposal of \citet{kennedymerchant2000} accounts for both why the \isi{AP} must be deleted and how it can be adjacent to the \isi{lexical verb}, the mechanism of VP-\isi{deletion} has to be revised. Further, the analysis does not link \isi{Attributive Comparative Deletion} to a more general theory on \isi{Comparative Deletion}. This would be important especially because the higher \isi{copy} seems to be deleted in attributive structures as well, thus the \isi{deletion} taking place at the base-generation site has to be linked to the \isi{deletion} of lower copies – in turn, the overt presence of a remnant \isi{DP} also has to be linked to a more general theory on why and how lower copies may be phonologically realised.

Last but not least, though \citet{kennedymerchant2000} provide a cross-lin\-guis\-tic investigation as far as the extractability of the \isi{degree} \isi{modifier} from the \isi{DP} is concerned, they still do not address the issue of further cross-linguistic variation; that is, cases when the absence of \isi{Attributive Comparative Deletion} effects cannot be directly linked to the possibility of extracting attributive modifiers.

\subsection{Gapping in Spanish -- \citet{reglero2006}} \label{sec:4gapping}
Building on the findings of \citet{kennedymerchant2000}, \citet{reglero2006} investigates the formation of \ili{Spanish} \isi{subcomparative} constructions, showing that \ili{Spanish} does not allow nominal subcomparatives in the way \ili{English} does. The importance of this study lies chiefly in that it provides further cross-linguistic insights into the possible mechanisms behind \isi{Attributive Comparative Deletion} and in that it examines cases of nominal comparatives: this issue was neglected by \citet{kennedymerchant2000}, who considered only attributive structures.

As \citet[67]{reglero2006} points out, the term \isi{Comparative Subdeletion} was used by \citet{bresnan1972diss} to refer to constructions like (\ref{nomcomp}), cf. \citet[67, ex. 1]{reglero2006}:

\ea Mary read more books than John read magazines. \label{nomcomp}
\z

For the derivation of (\ref{nomcomp}), \citet[68, ex. 4]{reglero2006} adopts (\ref{nomcompderivation4}):

\ea Mary read more books than \textit{Op\textsubscript{i}} John read [\emph{t}\textsubscript{i} many] magazines. \label{nomcompderivation4}
\z

As opposed to \ili{English}, \ili{Spanish} does not allow constructions like (\ref{nomcomp}); consider the following example (\citealt[68, ex. 7]{reglero2006}):

\ea [*] {
\gll Mar\'ia ley\'o m\'as libros que Juan ley\'o revistas. \label{spanish}\\
Mary read more books than John read magazines\\
\glt `Mary read more books than John read magazines.'}
\z

Relying on the observation of \citet{price1990}, however, \citet[68]{reglero2006} notes that constructions like (\ref{spanish}) become fully grammatical if the \isi{verb} is deleted from the \isi{subclause}. Consider the example in (\ref{spanishellipsis}), taken from \citet[68, ex. 8]{reglero2006}:

\ea \gll Mar\'ia ley\'o m\'as libros que Juan revistas. \label{spanishellipsis}\\
Mary read more books than John magazines\\
\glt `Mary read more books than John read magazines.'
\z

\citet[68]{reglero2006} refers to this as the ``Obligatory Gapping'' strategy; this applies even if the \isi{verb} in the \isi{subclause} is different from the one in the \isi{matrix clause}, as shown by the ungrammaticality of (\ref{spanishdiffverb}), cf. \citet[69, ex. 11]{reglero2006}:

\ea [*] {
\gll Mar\'ia ley\'o m\'as libros que Juan compr\'o revistas. \label{spanishdiffverb}\\
Mary read more books than John bought magazines\\
\glt `Mary read more books than John bought magazines.'}
\z

This shows that the \isi{verb} must be deleted regardless of whether it is redundant or not, hence there is some other requirement at work here (\citealt[69]{reglero2006}).

As \citet[69]{reglero2006} notes, the chief difference from ordinary \isi{gapping} is that \isi{verb} \isi{deletion} in constructions like (\ref{spanishellipsis}) is obligatory. Based on the analyses of \citet{lasnik1995} and \citet{kennedymerchant2000} for \ili{English}, \citet[69--70]{reglero2006} proposes that the object \textit{revistas} contains a strong \isi{feature} ``that needs to be checked either by \isi{movement}, or by \isi{PF} \isi{deletion} of the strong \isi{feature} in \isi{PF}.'' The derivation of the \isi{subclause} in (\ref{spanishellipsis}) is given in (\ref{reglerotree}), see \citet[70, ex. 14]{reglero2006}:

\ea \label{reglerotree} \upshape
\begin{forest} baseline, qtree, for tree={align=center}
[\ldots{} \isi{TopP}
	[NP\textsubscript{l}
		[Juan]
	]
	[Top$'$
		[Top]
		[AgrSP
			[AgrSP, name=node 1
				[NP [t\textsubscript{l}]]
				[AgrS$'$ 
					[AgrS [ley\'o]]
					[TP
						[NP [t\textsubscript{l}]]
						[T$'$
							[T [t\textsubscript{k}]]
							[VP [NP [t\textsubscript{l}]] [V$'$ [V [t\textsubscript{k}]] [XP [t\textsubscript{i}] [X$'$ [X{[}+strong{]}] [t\textsubscript{j}]]]]]
						]
					]
				]
			]
			[NP\textsubscript{j} [t\textsubscript{i} revistas,roof]
			]
		]
	]
]
\path [draw, thick] (node 1.north west) -- (node 1.south east);
\path [draw, thick] (node 1.north east) -- (node 1.south west);
\end{forest}
\z

There is reason to believe that subjects move up as high as the \isi{topic} projection in \ili{Spanish} comparative subclauses (see \citealt[70--72]{reglero2006}). If so,  it is possible to delete the \isi{AgrSP} without affecting the subject.

Essentially, the proposal made by \citet{reglero2006} is similar to what \citet{kennedymerchant2000} claimed in connection with \ili{English}; however, there is also an important difference in that in \ili{Spanish} an \isi{AgrSP} is deleted, whereas in \ili{English} there is VP-\isi{ellipsis}. As for \ili{English}, it was shown that though the \isi{lexical verb} must be eliminated, an auxiliary or modal that is located higher (e.g. \textit{do}) may remain overt. This is not the case in \ili{Spanish} (\citealt[73, ex. 25]{reglero2006}):

\ea [*] {
\gll Mar\'ia puede leer m\'as libros que Juan puede revistas. \label{libros4}\\
Mary can read more books than John can magazines\\
\glt `Mary can read more books than John can magazines.'}
\z

The ungrammaticality of (\ref{libros4}) shows that the projection affected by \isi{deletion} is indeed larger than the \isi{VP}. Just as was proposed by \citet{kennedymerchant2000}, \isi{deletion} takes place because an \isi{uninterpretable feature} must be eliminated, which is present on the XP in (\ref{reglerotree}) because the \isi{degree expression} moves up to its specifier in the same way it does to the \isi{FP} in \citet{kennedymerchant2000}; see \citet[73--74]{reglero2006}.

As for the \isi{rightward movement} of the object \isi{NP} \textit{revistas} in (\ref{reglerotree}), \citet[75--77]{reglero2006} claims that it undergoes Heavy \isi{NP} Shift (HNPS) and is adjoined relatively high in the structure because in \ili{Spanish} objects moving out of quantified phrases (e.g. in the case of floating quantifiers) land high. This ensures that when the lower \isi{AgrSP} node is deleted, not only the subject in [Spec,\isi{TP}] but also the object adjoined to the \isi{AgrSP} escapes \isi{deletion}.

Again, there are a number of problems that arise with this proposal, most of which have already been mentioned in connection with \citet{kennedymerchant2000}. In particular, the \isi{rightward movement} of the \isi{nominal expression} is not motivated enough and it is not plausible that this movement would be an instance of HNPS since the \isi{nominal expression} in question is not (necessarily) heavy at all. Besides this, the \isi{deletion} mechanism is again problematic since it is not clear how the lower but not the higher \isi{AgrSP} node is selected.

Moreover, it is not quite clear why this particular \isi{deletion} has to take place in \ili{Spanish} but not in \ili{English}; in addition, the difference between attributive and nominal structures in \ili{English} in this respect is not addressed either, though this would be fairly important in understanding the reasons behind \isi{Attributive Comparative Deletion}.

\section{Verb deletion -- an alternative approach} \label{sec:4verbdeletion}
In the following, based on \citet{bacskaiatkari2012even}, I will present an analysis for \isi{Attributive Comparative Deletion}, as found in \ili{English}, by adopting the structure given by \citet{kennedymerchant2000} for the syntax of \isi{attributive modification} and by proposing a different approach to VP-\isi{ellipsis} from the one found in \citet{kennedymerchant2000} and in \citet{reglero2006}.

The starting point of the argumentation is the assumption, presented in detail in \chapref{ch:3}, that if \isi{deletion} takes place at \isi{PF}, it cannot affect F-marked material. This is highlighted by \citet[472--473]{reich2007} as a rule constraining \isi{verb} \isi{deletion} and, with respect to VP-\isi{ellipsis}, he basically implies that if the object is F-marked, then the F-\isi{markedness} of this object in itself may withstand \isi{deletion}. Consider the examples in (\ref{catsdogs4}):

\ea \label{catsdogs4}
\ea	[]{Ralph likes cats and Mike [\textsubscript{VP} likes [\textsubscript{DP} dogs]\textsubscript{F}].} \label{vpfull}
\ex	[]{Ralph likes cats and Mike [\textsubscript{VP} \sout{likes} [\textsubscript{DP} dogs]\textsubscript{F}].} \label{gapping}
\ex	[*]{Ralph likes cats and Mike [\textsubscript{VP} \sout{likes} [\textsubscript{DP} \sout{dogs}]\textsubscript{F}].} \label{vpellipsis}
\z
\z

The full structure is shown in (\ref{vpfull}). In case \isi{deletion} takes place, as in (\ref{gapping}), the following happens: the V head (\textit{likes}) is deleted but the F-marked \isi{DP} (\textit{dogs}) remains overt. Should the \isi{DP} be eliminated too, which would no longer be \isi{gapping} but stripping, then the sentence would not be grammatical since the F-marked \isi{DP} could not be recovered from the context, as shown in (\ref{vpellipsis}).

Following this, it can still be maintained that \isi{verb} \isi{gapping} is an instance of VP-\isi{ellipsis}: \isi{deletion} targets the \textsc{given} \isi{VP}, within which there is an F-marked \isi{DP}. Since \isi{deletion} operations proceed in a left-to-right fashion at \isi{PF} (which is why it is the copies on the \isi{left edge} that remain from a \isi{movement chain}, cf. \citealt{boskovicnunes2007}), when the \isi{PF} mechanism working this way arrives at the \isi{DP}, it stops.

This is further reinforced by the fact that when there is no F-marked \isi{DP}, as in (\ref{catscats4}), then there is nothing to prevent the elimination of the \isi{DP}:

\ea \label{catscats4}
\ea	[]{Ralph likes cats and Mike [\textsubscript{VP} likes [\textsubscript{DP} cats]].} \label{vpfullgiven}
\ex	[*]{Ralph likes cats and Mike [\textsubscript{VP} \sout{likes} [\textsubscript{DP} cats]].} \label{gappinggiven}
\ex	[]{Ralph likes cats and Mike [\textsubscript{VP} \sout{likes} [\textsubscript{DP} \sout{cats}]] too.} \label{vpellipsisgiven}
\z
\z

Taking the sentence in (\ref{vpfullgiven}), where the \isi{DP} (\textit{cats}) is not F-marked, it can be seen that in case VP-\isi{ellipsis} happens, only the entire \isi{VP} can be deleted, as in (\ref{vpellipsisgiven}); the elimination of the single V head, as in (\ref{gappinggiven}), is not sufficient. If \isi{verb} \isi{gapping} existed as a separate mechanism targeting the V head as such, then (\ref{gappinggiven}) should be grammatical. On the other hand, the phenomenon can be explained well with the mechanism of VP-\isi{ellipsis} described above: as there is no F-marked \isi{DP} within the \isi{VP}, \isi{deletion} will naturally affect the \isi{DP}, too. Note that the reason why (\ref{vpellipsisgiven}) contains \textit{too} is precisely because it is a stripping construction: without the presence of \textit{too}, coordination would be interpreted as holding between the two DPs \textit{cats} and \textit{Mike} and thus not containing \isi{ellipsis}. I will not try to explain here why this should be so as it would go far beyond the scope of the present investigation; for a more elaborate discussion, see \citet{vicente2010}.

Similarly, it is also VP-\isi{deletion} that takes place in attributive comparatives: here the F-marked constituent is the \isi{DP}, not the \isi{FP}. Consider the examples given in (\ref{catcatflap4}):

\ea \label{catcatflap4}
\ea [*]{Ralph bought a bigger cat than Mike [\textsubscript{VP} bought [\textsubscript{FP} x-big [\textsubscript{DP} a cat flap]\textsubscript{F}]].} \label{acdfull}
\ex	[*]{Ralph bought a bigger cat than Mike [\textsubscript{VP} \sout{bought} [\textsubscript{FP} x-big [\textsubscript{DP} a cat flap]\textsubscript{F}]].} \label{acdverb}
\ex	[]{Ralph bought a bigger cat than Mike [\textsubscript{VP} \sout{bought} [\textsubscript{FP} \sout{x-big} [\textsubscript{DP} a cat flap]\textsubscript{F}]].} \label{acdcorrect}
\ex	[*]{Ralph bought a bigger cat than Mike [\textsubscript{VP} \sout{bought} [\textsubscript{FP} \sout{x-big} [\textsubscript{DP} \sout{a cat flap}]\textsubscript{F}]].} \label{acdvp}
\z
\z

The sentence containing the full structure overtly in (\ref{acdfull}) is ungrammatical because the \isi{QP} (\textit{x-big}) in the \isi{subclause} should be deleted. The reason why (\ref{acdverb}) is not grammatical either is that VP-\isi{ellipsis} affects only the V head though the \isi{FP}, which is \textsc{given}, cannot stop \isi{deletion} at this point. The only grammatical sentence is (\ref{acdcorrect}), in which VP-\isi{ellipsis} is stopped by the first F-marked projection, that is, the \isi{DP} (\textit{a cat flap}). The sentence in (\ref{acdvp}) is again ungrammatical since the F-marked \isi{DP} is also deleted.

VP-\isi{ellipsis} is thus an optional process that may save the construction from ungrammaticality; in this respect it is similar to \isi{sluicing} (see the relevant discussion presented in \chapref{ch:3}), hence the phenomenon is not unique.

\section{The lack of Attributive Comparative Deletion} \label{sec:4lack}
One of the most important questions concerning the analysis above is whether it can be maintained when tested against cross-linguistic data. The chief claim is that \isi{Attributive Comparative Deletion} is not a separate mechanism in itself but the surface realisation of two more general processes: \isi{Comparative Deletion} and VP-\isi{ellipsis}. Therefore, the prediction is that in languages where either of the two processes is missing, \isi{Attributive Comparative Deletion} will not be attested. In what follows, I will briefly examine two languages in this respect, \ili{Hungarian} and \ili{German}, which were both claimed in the introduction of the present chapter to lack \isi{Attributive Comparative Deletion} constructions.

The fundamental difference between \ili{English} and \ili{Hungarian} lies in the fact that the former but not the latter exhibits \isi{Comparative Deletion}. Recall that in \ili{English} both copies of the \isi{degree expression} are eliminated from the \isi{subclause} by default, as shown in (\ref{marygeorge4}):

\ea \label{marygeorge4}
\ea Mary is taller than \sout{[x-tall]} George was \sout{[x-tall]}.
\ex	Mary bought bigger cats than \sout{[x-big cats]} George saw \sout{[x-big cats]}.
\z
\z

As was argued for in \chapref{ch:3}, the reason behind this is that there is an \isi{Overtness Requirement} on the \isi{operator} in the [Spec,\isi{CP}] position, such that an overt \isi{AP} (or \isi{NP}) is not licensed if the \isi{operator} is phonologically zero. Since, however, the syntactic features are checked off for the \isi{movement chain}, the lower \isi{copy} of the \isi{QP} (or the \isi{nominal expression} containing that \isi{QP}) can be regularly deleted and may remain overt only in case it is contrastive.

However, this is clearly not the case in \ili{Hungarian}, which has overt operators and the higher \isi{copy} can remain overt, as demonstrated by (\ref{marihungparadigm4}):

\ea \label{marihungparadigm4}
\ea \gll Mari	magasabb,	mint	amilyen	magas	Gyuri volt.\\
Mary	taller	than	how	tall	George was.\textsc{3sg}\\
\glt `Mary is taller than George was.'
\ex \gll	Mari	nagyobb	macsk\'akat	vett,	mint amilyen nagy	macsk\'akat	Gyuri	l\'atott.\\
Mary bigger cats.\textsc{acc} bought.\textsc{3sg} than	how big	cats.\textsc{acc} George saw.\textsc{3sg}\\
\glt `Mary bought bigger cats than George saw.'
\z
\z

As was discussed in \chapref{ch:3}, the \isi{QP} (\textit{amilyen magas} `how tall') and the quantified \isi{DP} (\textit{amilyen nagy macsk\'akat} `how big cats') may remain overt in the \isi{subclause} even if they are logically identical with their counterparts in the \isi{matrix clause}.  Moreover, these elements are overt in the [Spec,\isi{CP}] position and not in their \isi{base position}, as in \ili{Hungarian} there is clearly no \isi{Comparative Deletion} eliminating these constituents.

The higher \isi{copy} of the \isi{QP} or the quantified \isi{DP} in the \isi{subclause} is not obligatorily deleted and so the lower \isi{copy} can be regularly deleted. Thus the expectation is that \isi{Attributive Comparative Deletion} will not be attested in \ili{Hungarian}. This prediction is borne out by the data, exemplified in (\ref{ch4acdhung}), repeated here as (\ref{acdhung2}):

\ea \label{acdhung2}
\gll Rudolf nagyobb macsk\'at vett, mint amilyen sz\'eles macskaajt\'ot Mikl\'os vett.\\
Rudolph bigger cat.\textsc{acc}	bought.\textsc{3sg} than how wide cat.flap.\textsc{acc}	Mike	bought.\textsc{3sg}\\
\glt `Rudolph bought a bigger cat then Mike did a cat flap.'
\z

The full \isi{DP} \textit{amilyen széles macskaajtót} `how big a cat flap' is overtly located in [Spec,\isi{CP}], as CD does not eliminate it; therefore, the lower \isi{copy} can regularly be deleted without any part of it remaining. It can thus be concluded that \ili{Hungarian} does not have \isi{Attributive Comparative Deletion} because it does not have \isi{Comparative Deletion} at all.

As was seen in \chapref{ch:3}, \isi{Comparative Deletion} is not attested in \ili{German} in the way it is in \ili{English}. Consider:

\ea \label{germancdsub}
\ea	[?]{\gll Maria ist größer	als	Johann \textbf{groß} ist. \label{germannocd}\\
Mary is	taller than	John tall	is\\
\glt `Mary is taller than John.'}
\ex []{\gll	Der	Tisch	ist	länger als das	Büro \textbf{breit} ist. \label{germansubcomp}\\
the.\textsc{m} table	is longer	than the.\textsc{n}	office wide is\\
\glt `The table is longer than the office is wide.'}
\z
\z

As shown by the acceptability of (\ref{germansubcomp}), \ili{German} is similar to \ili{English} in that it allows subcomparatives in predicative structures. However, \ili{German} is also different from \ili{Hungarian} in that the \isi{movement} of the entire \isi{degree expression} to the [Spec,\isi{CP}] position would result in \isi{Comparative Deletion} in the same way as it applies in \ili{English}, since the \isi{comparative operator} is zero in both cases, resulting in a violation of the \isi{Overtness Requirement}. On the other hand, the possibility of moving the \isi{operator} on its own in predicative structures is an option not available in \ili{English}, hence the acceptability of structures like (\ref{germannocd}). In other words, if there is a \isi{copy} to be realised overtly in \ili{German}, then it is the lower one, just as in \ili{English}.

However, the \isi{operator} cannot be extracted on its own if the \isi{QP} is a \isi{modifier} within a \isi{DP}, and since the quantified \isi{DP} cannot occur overtly in the [Spec,\isi{CP}] position, this may suggest that \ili{German} actually has \isi{Attributive Comparative Deletion} in the same way as it is attested in \ili{English}. This is not the case, as shown by (\ref{ch4germanacd}):

\ea \label{ch4germanacd}
\ea	[*]{\gll Ralf hat eine	größere	Wohnung	als Michael	ein	Haus.\\
Ralph	has	a.\textsc{acc.f} bigger.\textsc{acc.f} flat	than Michael a.\textsc{acc.n} house\\
\glt `Ralph has a bigger flat than Michael a house.'}
\ex	[*]{\gll Ralf kauft	schnellere	Hunde	als	Michael Katzen. \label{hundekatzen}\\
Ralph	buys faster.\textsc{pl} dogs than	Michael cats\\
\glt `The dogs Ralph buys are faster than the cats that Michael buys.'}
\z
\z

The sentences in (\ref{ch4germanacd}) are not grammatical though the \isi{QP} is eliminated from the \isi{subclause}. Since this \isi{deletion} is VP-\isi{ellipsis} in \ili{English}, the root of the problem with (\ref{ch4germanacd}) may be related to VP-\isi{ellipsis} in \ili{German}: \ili{German} is known to lack VP-\isi{deletion} in the way \ili{English} has it (\citealt[120--124]{winkler2005}; \citealt[671]{merchant2004}). Moreover, the \ili{German} comparative \isi{subclause} is verb-final, just as any other \isi{subclause} in \ili{German}: this is also attested by (\ref{germancdsub}); therefore, a prenominal \isi{modifier} and a \isi{verb} could not be deleted together, even though the \isi{QP} is located at the \isi{left edge} of the plural \isi{nominal expression} in (\ref{hundekatzen}).

However, the chief problem is that \ili{German} does not require the \isi{deletion} of the \isi{QP} in the lower \isi{copy} in the way \ili{English} does. The following construction is fully grammatical for several speakers:

\ea \gll	Ralf	hat	eine	größere	Katze	als Michael	eine breite Katzenklappe hat. \label{germanacdkatze}\\
Ralph	has	a.\textsc{acc.f} bigger.\textsc{acc.f} cat than Michael a.\textsc{acc.f} wide.\textsc{acc.f} cat.flap has.\\
\glt `Ralph's cat is bigger than Michael's cat flap is wide.'
\z

While the acceptability of (\ref{germanacdkatze}) shows inter-speaker variation (with no identifiable regional differences), the point here is that speakers who accept (\ref{germanacdkatze}) still do not accept (\ref{ch4germanacd}). In (\ref{germanacdkatze}), the \isi{QP} \textit{breit(e)} `wide' can remain overt as part of the \isi{DP} in the lower \isi{copy} of that \isi{DP}; note also that this \isi{QP} is not inverted, that is, it is not moved to a [Spec,\isi{FP}] position: this should be clear from the fact that it appears between the \isi{indefinite article} \textit{eine} `a' and the \isi{noun} head \textit{Katzenklappe} `cat flap'. I will return to the question of inverted and non-inverted \isi{QP} modifiers in \sectref{sec:4moreonattr}; for the time being, suffice it to say that the lower \isi{copy} of the entire \isi{DP} can remain in a \ili{German} comparative \isi{subclause} just as the entire \isi{copy} of a \isi{QP} can in predicative structures. That is, in structures like (\ref{germanacdkatze}) the entire \isi{DP} moves up to the [Spec,\isi{CP}] position, since the \isi{QP} cannot be extracted on its own. The higher \isi{copy} is eliminated by \isi{Comparative Deletion} just as in \ili{English} because the \isi{Overtness Requirement} on the \isi{operator} is not met. However, the entire lower \isi{copy} may remain overt in \ili{German}, unlike in \ili{English}, and this is presumably related to the fact that \ili{German} does not display the kind of inversion \ili{English} does.

\section{The Overtness Requirement revisited} \label{sec:4overtness}
The analysis so far captures important cross-linguistic differences and is fully able to relate the phenomenon of \isi{Attributive Comparative Deletion} to whether and how \isi{Comparative Deletion} is attested in the language. In other words, \isi{Attributive Comparative Deletion} is a phenomenon that results from \isi{Comparative Deletion} and VP-\isi{ellipsis}. The way VP-\isi{ellipsis} is available in a given language is naturally subject to more general rules and, as was shown in \chapref{ch:3}, so is \isi{Comparative Deletion}, in that it is reducible to an \isi{Overtness Requirement} that holds on elements moving to a [Spec,\isi{CP}] position.

Therefore, \isi{Attributive Comparative Deletion} is not attested in cases when the higher \isi{copy} of the quantified expression can be overtly realised in the [Spec,\isi{CP}] position, that is, when there is a phonologically visible \isi{operator}. Problems seem to arise when it is the lower \isi{copy} that should be pronounced. This is true for languages such as \ili{English}, where the \isi{operator} is a Deg head and cannot be extracted in predicative structures either, and it also holds for languages like \ili{German}, where the \isi{QP} \isi{modifier} \isi{operator} could be extracted from a single \isi{QP} but not from within a \isi{DP}, as that would be a case of violating the Left Branch Condition. However, as was pointed out by \citet{kennedymerchant2000}, in languages where the \isi{QP} can be extracted from the \isi{nominal expression}, such as \ili{Polish} or \ili{Czech}, \isi{Attributive Comparative Deletion} does not arise.

In other words, \isi{Attributive Comparative Deletion} arises when there is an inverted \isi{QP} that moves to the [Spec,\isi{FP}] position in the extended \isi{nominal expression}. As was shown by \citet{kennedymerchant2000}, precisely this \isi{QP} is ungrammatical; however, they do not address the question why this should be so. In what follows I will argue that this is due to an \isi{Overtness Requirement} on the \isi{operator} element and that this \isi{Overtness Requirement} is essentially the same as the one that underlies \isi{Comparative Deletion}, thus extending the \isi{Overtness Requirement} in the CP-domain to the nominal domain.

As was shown by \citet[124--130]{kennedymerchant2000}, certain quantified expressions undergo upward \isi{movement} within the \isi{nominal expression}, landing in the specifier position of a \isi{functional projection} (FP) above the \isi{DP} layer. For the time being, I adopt the analysis given by \citet{kennedymerchant2000} in that the \isi{nominal expression} \textit{a novel} in examples like (\ref{novel4}) is indeed a \isi{DP}; I will return to this issue in \sectref{sec:4moreonattr}, showing that the different layers in the \isi{nominal expression} show different behaviour with respect to projecting an \isi{FP} layer (and hence \isi{attributive modification}), and I will treat \textit{a novel} rather as a \isi{NumP}. 

Recall the following \isi{contrast}:

\ea \label{novel4}
\ea	[]{[\textsubscript{FP} [\textsubscript{QP} How interesting]\textsubscript{i} [\textsubscript{DP} a [\textsubscript{NP} \emph{t}\textsubscript{i} novel]]] did Ralph read?} \label{fpqp}
\ex	[*]{[\textsubscript{DP} A [\textsubscript{NP} [\textsubscript{QP} how interesting] novel]] did Ralph read?} \label{fpqpfail}
\z
\z

As can be seen, the construction is grammatical only if the \isi{QP} moves up to the \isi{FP} level, as in (\ref{fpqp}): if it stays in its \isi{base position}, as in (\ref{fpqpfail}), the result is ungrammatical. The quantified expression in this case contains a \textit{wh}-\isi{operator} (\textit{how}), which has to move upwards because of its [\textsc{edge}] \isi{feature}; in addition, in the analysis given by \citet{kennedymerchant2000}, this is how the entire \isi{nominal expression} acquires a [+wh] \isi{feature}, which can be checked off in the [Spec,\isi{CP}] position. Otherwise, the [+wh] \isi{feature} is claimed to be uninterpretable on the F head for \isi{PF}. Essentially, the same kind of \isi{movement} is assumed to take place in \isi{Attributive Comparative Deletion} structures as well: however, since in these cases the higher \isi{copy} is not pronounced either (due to \isi{Comparative Deletion}, see \chapref{ch:3}), the lower \isi{copy} cannot be automatically eliminated. This is why, as has been seen, VP-\isi{ellipsis} applies, which can delete the \isi{lexical verb} and the \isi{AP} together since these are indeed adjacent at \isi{PF}, as illustrated in (\ref{catflapdid4}):

\ea	Ralph bought a bigger cat than George did \sout{[\textsubscript{VP} buy [\textsubscript{FP} [\textsubscript{QP} x-big]\textsubscript{i} }[\textsubscript{DP} a [\textsubscript{NP} \emph{t}\textsubscript{i} cat flap]]]]. \label{catflapdid4}
\z

The issue here is why the particular position of the \isi{QP} is ungrammatical. According to \citet{kennedymerchant2000}, the reason should be the presence of an unchecked [+wh] -- or, in comparative subclauses, rather a [+rel] -- \isi{feature} on an F head. This is problematic for a number of reasons: first, the \isi{feature} under discussion is checked off in the higher \isi{copy} and therefore should no longer cause a problem for any \isi{copy} in the \isi{movement chain}. Second, the F head is not visible in these cases and it is thus not straightforward why a given \isi{feature} on an invisible head should in itself be a PF-violation.

More importantly, as was also discussed by \citet{kennedymerchant2000}, there are constructions that clearly do not involve the \isi{movement} of the entire \isi{nominal expression} to an \isi{operator position} and yet inversion is attested. For instance, the \isi{degree element} \textit{too} also requires inversion, as illustrated in (\ref{too4}):

\largerpage[1]
\ea \label{too4}
\ea	[]{Ralph bought [\textsubscript{FP} [\textsubscript{QP} too big]\textsubscript{i} [DP a [NP \emph{t}\textsubscript{i} cat]]].} \label{toobigacat}
\ex	[*]{Ralph bought [\textsubscript{DP} a [\textsubscript{NP} [\textsubscript{QP} too big] cat]].}
\z
\z

In the case of (\ref{toobigacat}), it does not seem valid that the F head is equipped with a [+wh] \isi{feature} that happens to be uninterpretable at \isi{PF}: the \isi{QP} itself is not [+wh] in nature and the whole \isi{FP} does not move up to a [+wh] position. Moreover, the construction is grammatical so there seems to be no PF-violation at hand.

As was mentioned, while \isi{movement} to [Spec,\isi{FP}] is obligatory for \textit{too}, as is for \textit{so}, \isi{QP} \isi{degree} modifiers (e.g. \textit{more}, \textit{enough}, \textit{quite}) generally involve this \isi{movement} optionally, see \citet[129--130]{kennedymerchant2000}, based on \citet[287--288]{bresnan1973}. Yet, as noted by the same authors, there is one construction which does not allow this \isi{movement} and this is the case of bare adjectives (which are nevertheless analysed as QPs containing a null \isi{degree element} marking the positive \isi{degree}). As was shown in \chapref{ch:2} in detail, gradable adjectives are in the specifier of a \isi{DegP} irrespectively of whether the \isi{degree} is absolute, comparative or superlative, since the \isi{degree} itself is expressed by the Deg head and not the \isi{AP} itself. Moreover, modifiers are located in the [Spec,\isi{QP}] position and these show \isi{agreement} with the Q head with respect to its \isi{degree}, whereby the \isi{absolute degree} also has its modifiers as well, e.g. \textit{very}. 

Consider now the examples in (\ref{nofpnp4}):

\ea \label{nofpnp4}
\ea [*]{Ralph bought [\textsubscript{FP} [\textsubscript{QP} big]\textsubscript{i} [\textsubscript{DP} a [\textsubscript{NP} \emph{t}\textsubscript{i} cat]]].} \label{bareapfp}
\ex	[]{Ralph bought [\textsubscript{DP} a [\textsubscript{NP} [\textsubscript{QP} big] cat]].}
\z
\z

One may think that this is so because bare adjectives cannot move to the [Spec,\isi{FP}] position at all; indeed, if they lacked a \isi{degree element}, this would be a plausible consequence. However, it appears that even positive adjectives can undergo this \isi{movement}, as shown in (\ref{coordkennedymerchant}). Observe the following sentence of the same type:

\ea	Ralph saw a lilac cat and Mike did a tiger. \label{tigercat}
\z

Recall that sentences like (\ref{tigercat}) are ambiguous between two readings (see \citealt[127--131]{kennedymerchant2000}): under one reading Mike saw a tiger, which was not necessarily lilac, while under the other reading Mike saw a lilac tiger. Therefore, in the first case the \isi{adjective} \textit{lilac} is not even underlyingly present in the second clause, whereas in the second case it has to be deleted, given that the information it carries is also present. The two structures are shown in (\ref{tiger}) and (\ref{lilactiger}), respectively:

\ea \label{tigerfull}
\ea	Ralph saw a lilac cat and Mike did \sout{[\textsubscript{VP} see} [\textsubscript{DP} a [\textsubscript{NP} tiger]]]. \label{tiger}
\ex	Ralph saw a lilac cat and Mike did \sout{[\textsubscript{VP} see [\textsubscript{FP} [\textsubscript{QP} lilac]\textsubscript{i}} [\textsubscript{DP} a [\textsubscript{NP} \emph{t}\textsubscript{i} tiger]]]]. \label{lilactiger}
\z
\z

The \isi{deletion} of the \isi{adjective} together with the \isi{verb} in (\ref{lilactiger}) is possible only if the \isi{adjective} moves up to the specifier of the \isi{FP}. Note that in this case \isi{deletion} saves the construction from ungrammaticality as the overt presence of \textit{lilac} in (\ref{lilactiger}) would not be grammatical, just as in (\ref{bareapfp}) above.

It has to be mentioned that the acceptability of pseudogapping constructions seems to show interesting dialectal and/or idiolectal differences. Some speakers do not find structures like (\ref{tigerfull}) natural and prefer a construction like (\ref{tigerasdid}) below:

\ea	Ralph saw a lilac cat as Mike did a tiger. \label{tigerasdid}
\z

On the other hand, there is a difference in the availability of the two readings: a reading like (\ref{tiger}) is generally more available than one like (\ref{lilactiger}), and speakers who get an interpretation like (\ref{lilactiger}) also get (\ref{tiger}) but not vice versa. This should not be surprising, as the derivation in (\ref{tiger}) is more economical than the one in (\ref{lilactiger}): apart from the fact that there is more material elided in (\ref{lilactiger}), there is also an extra \isi{movement} operation. The same applies to structures like (\ref{tigerasdid}) above and also to cases like (\ref{mostinteresting4}) where the \isi{degree expression} is more complex:

\ea	Ralph saw a most interesting play as did Peter a movie. \label{mostinteresting4}
\z

In this case, the ambiguity of the sentence depends on the presence/absence of the \isi{QP} \textit{most interesting} in the \isi{subordinate clause} in the underlying structure.

At any rate, it seems that inverted \isi{degree} expressions are ungrammatical precisely when there is no overt \isi{degree element}. These QPs move to a left-peripheral position within the extended \isi{nominal expression} and just as the [Spec,\isi{CP}] position is reserved for elements with an \isi{overt operator} (see \chapref{ch:3}), the [Spec,\isi{FP}] position must have an overt \isi{degree element} to avoid PF-\isi{uninterpretability}.

This implies that the \isi{Overtness Requirement} is not specific to comparative structures. This is further reinforced by the fact that it can be observed in the [Spec,\isi{CP}] position in structures other than comparatives. That is, in relative clauses that may contain the sequence of a \isi{relative operator} and some lexical projection. Though this construction is generally not widespread, there are still some examples such as the one from \ili{Hungarian} given in (\ref{hungrelfull}):

\ea \label{hungrelfull}
\ea \gll Mari	Judith Hermann	könyvét olvasta, \textbf{amely} \textbf{könyvet} egyébként még én	küldtem	neki	Berlinből. \label{hungrelwhichbook}\\
Mary Judith	Hermann	book.\textsc{poss.acc} read.\textsc{pst.3sg} which	book.\textsc{acc} incidentally still I sent.\textsc{1sg} she.\textsc{dat} Berlin.\textsc{ela}\\
\glt `Mary was reading Judith Hermann's book, which actually I had sent her from Berlin.'
\ex \gll Leégett	a	gyár,	\textbf{amely} \textbf{esemény} megmozgatta	a	várost. \label{hungrelwhichevent}\\
down.burned.\textsc{3sg} the factory which event \textsc{prt}.moved.\textsc{3sg} the city.\textsc{acc}\\
\glt `The factory burned down, which moved the city.'
\z
\z

Such constructions are relatively rare, presumably because they either involve the repetition of the matrix \isi{clausal} \isi{nominal expression}, as in (\ref{hungrelwhichbook}), or the \isi{noun} in the \isi{subclause} must be general enough to be an anaphor for the entire \isi{matrix clause}, as in (\ref{hungrelwhichevent}). The configuration is also grammatical in the absence of an overt \isi{NP}, as in (\ref{reloplexical4}):

\ea \label{reloplexical4}
\ea \gll Mari	Judith Hermann	könyvét olvasta, \textbf{amelyet} egyébként még én	küldtem	neki	Berlinből. \label{hungrelwhich}\\
Mary Judith	Hermann	book.\textsc{poss.acc} read.\textsc{pst.3sg} which.\textsc{acc} incidentally still I sent.\textsc{1sg} she.\textsc{dat} Berlin.\textsc{ela}\\
\glt `Mary was reading Judith Hermann's book, which actually I had sent her from Berlin.'
\ex \gll Leégett	a	gyár,	\textbf{ami} megmozgatta	a	várost. \label{hungrelwhat}\\
down.burned.\textsc{3sg} the factory what \textsc{prt}.moved.\textsc{3sg} the city.\textsc{acc}\\
\glt `The factory burned down, which moved the city.'
\z
\z

\largerpage[1]
In (\ref{hungrelwhich}) the \isi{operator} takes the \isi{accusative case} suffix and, just like in (\ref{hungrelwhichbook}), marks the relative nature of the clause; in (\ref{hungrelwhat}) the \isi{relative pronoun} \textit{ami} `what' refers back to the entire \isi{matrix clause} just as the \isi{nominal expression} in (\ref{hungrelwhichevent}). However, an overt \isi{NP} is not grammatical in the [Spec,\isi{CP}] position without an \isi{overt operator}:

\ea \label{hungrelungrammatical}
\ea [*]{\gll Mari	Judith Hermann	könyvét olvasta, \textbf{könyvet} egyébként még én	küldtem	neki	Berlinből.\\
Mary Judith	Hermann	book.\textsc{poss.acc} read.\textsc{pst.3sg} book.\textsc{acc} incidentally still I sent.\textsc{1sg} she.\textsc{dat} Berlin.\textsc{ela}\\
\glt `Mary was reading Judith Hermann's book, which actually I had sent her from Berlin.'}
\ex [*]{\gll Leégett	a	gyár,	\textbf{esemény} megmozgatta	a	várost.\\
down.burned.\textsc{3sg} the factory event \textsc{prt}.moved.\textsc{3sg} the city.\textsc{acc}\\
\glt `The factory burned down, which moved the city.'}
\z
\z

The reason behind the ungrammaticality of (\ref{hungrelungrammatical}) is that \ili{Hungarian} lacks zero relative operators. Moreover, even if there were a zero operator, it would not be interpretable for \isi{PF} to have overt material in a [+rel] position without an overt element representing [+rel].\footnote{It has to be mentioned that some strings that look like the ones in (43) may in fact be grammatical. Consider the examples in (\ref{zeroreloplexical4}):

\ea \label{zeroreloplexical4}
\ea \gll Mari	Judith Hermann	könyvét olvasta, \textbf{a} \textbf{könyvet} egyébként még én	küldtem	neki	Berlinből.\\
Mary Judith	Hermann	book.\textsc{poss.acc} read.\textsc{pst.3sg} the	book.\textsc{acc} incidentally still I sent.\textsc{1sg} she.\textsc{dat} Berlin.\textsc{ela}\\
\glt `Mary was reading Judith Hermann's book; actually, I had sent it to her from Berlin.'
\ex \gll Leégett	a	gyár,	\textbf{ez} \textbf{az} \textbf{esemény} megmozgatta	a	várost.\\
down.burned.\textsc{3sg} the factory this the event \textsc{prt}.moved.\textsc{3sg} the city.\textsc{acc}\\
\glt `The factory burned down; this event moved the city.'
\z
\z

However, these are instances of coordination, and therefore the DPs \textit{a könyvet} and \textit{ez az esemény}, respectively, are not in a [Spec,\isi{CP}] position.} Note that this does not exclude the possibility of having null operators in [Spec,\isi{CP}] on their own if they are available, such as the zero \isi{relative operator} in \ili{English}, because in that case there would be no visible \isi{lexical material} to cause \isi{uninterpretability} either.

It seems justifiable that the \isi{Overtness Requirement} holds in a similar way in [Spec,\isi{CP}] positions as in [Spec,\isi{FP}] positions at left edges of nominal expressions.\footnote{The scope of the present investigation does not enable a broader investigation of the issue in the sense that there might be other \isi{overtness} issues related to left-peripheral positions. For instance, topicalised subordinate clauses in \ili{English} seem to constitute such a case:

\ea	I know [\textsubscript{CP} \textbf{(that)} he arrived late]. \label{thatcomplement}
\z
\ea	{}[\textsubscript{CP} *\textbf{(That)} he arrived late] is surprising. \label{thatfronted}
\z 

As indicated, the \isi{complementiser} \textit{that} can be omitted in (\ref{thatcomplement}), where it appears at the \isi{right edge}, but not when it does so at the \isi{left edge}, that is, when it is topicalised, as in (\ref{thatfronted}). The phenomenon is not restricted to \ili{English}; for instance, \citet{poletto1995} observes a similar issue in \ili{Italian}. The investigation of this problem would go far beyond the scope of the present book, and I will therefore leave this question open here.} Considering this, the following generalisation arises: certain phrase-sized constituents moving leftwards to an \isi{operator} position must have an overt marker on their \isi{left edge} so that the configuration converges. The overt marker may be the head but may also be a specifier element. In either case, the topmost projection of the given phrase is equipped with certain features either because the head itself inherently has that \isi{feature} or because it acquires that \isi{feature} via specifier–head \isi{agreement}. These features are interpretable at \isi{LF} but the same is not necessarily true for \isi{PF}: a \isi{feature} that is interpretable at \isi{LF} is not necessarily so at \isi{PF}, and vice versa (cf. \citealt[223]{tsimplidimitrakopoulou2007}).

In the case of \isi{Comparative Deletion} and the obligatory \isi{overtness} of relative operators, there is a zero element bearing the [+rel] \isi{feature} followed by overt material. Consider the grammatical,\footnote{As pointed out in \chapref{ch:3}, there is variation in the acceptability of \textit{hoe} `how' in these cases; the present discussion applies to dialects where \textit{hoe} is available as a \isi{comparative operator}.} non-deleting examples in (\ref{dutch4}) from \ili{Dutch}:

\ea \label{dutch4}
\ea	\gll Maria	is groter	dan	\textbf{hoe} \textbf{groot}	Jan	is.\\
Mary is	taller than	how	tall John	is\\
\glt `Mary is taller than John is.'
\ex	\gll De tafel is langer dan	\textbf{hoe} \textbf{breed}	het kantoor is.\\
the	table	is longer	than	how	wide the.\textsc{n} office is\\
\glt `The table is longer than the office is wide.'
\z
\z

The \isi{PF} string for \textit{hoe groot} `how tall' is as follows:

\ea	hoe\textsubscript{[+rel]} groot \label{pfhoegroot}
\z

The [+rel] \isi{feature} on \textit{hoe} instructs \isi{PF} to align the \isi{left edge} of the phrase with the \isi{left edge} of a phonological unit. However, in cases where the \isi{operator} is phonologically zero, the \isi{PF} string is the following:

\ea	\textsubscript{[+rel]} tall \label{pfzerotall}
\z

This causes a problem for \isi{PF} because the [+rel] \isi{feature} on its own, that is, without any visible element carrying it, is not alignable.

The problem is fundamentally similar in the case of \isi{Attributive Comparative Deletion} and the \isi{movement} of quantified expressions to the \isi{left edge} of a functional \isi{FP}. In a string such as \textit{how big a cat}, \isi{PF} sees the following string:

\ea	how\textsubscript{[+wh]} big \label{pfhowbig}
\z

By \isi{contrast}, the zero \isi{comparative operator} in \ili{English} attributive comparative structures produces a string similar to the one in (\ref{pfzerotall}):

\ea	\textsubscript{[+rel]} big \label{pfzerobig}
\z

The [+rel] \isi{feature} is not interpretable for \isi{PF} without a visible element: the string should be aligned to the \isi{left edge} of the extended \isi{nominal expression} (FP).

Given the similarity between (\ref{pfhoegroot}) and (\ref{pfhowbig}) on the one hand and between (\ref{pfzerotall}) and (\ref{pfzerobig}) on the other hand, it seems reasonable to assume the existence of some generalised pattern. Instead of the separate \isi{operator} features [+wh] and [+rel], there is a general \isi{operator} \isi{feature}: an \isi{operator} \isi{feature} is essentially responsible for elements moving to the \isi{left edge} (cf. \citealt{mueller2003}) and thus the generalised \isi{feature} may be called simply [\textsc{edge}]. This predicts that a zero \textit{wh}-element or a \isi{relative pronoun} is not ungrammatical in itself but they become PF-uninterpretable if they move to the \isi{edge}, that is, if they are equipped with an [\textsc{edge}] \isi{feature}.

The generalised PF-interpretable configuration of strings containing [\textsc{edge}] features is given in (\ref{xedgey}):

\ea	X\textsubscript{[\textsc{edge}]} Y \label{xedgey}
\z

The syntactic status of X and Y, as well as their exact structural relation, is not of importance in terms of PF-interpretability: X itself is naturally a head, such that it may be a head taking Y as its complement, or it may be the head of a phrase that is located in the specifier of the phrase headed by Y. In either case, the [\textsc{edge}] \isi{feature} itself is located on a phonologically visible head and the structure converges.

By \isi{contrast}, the PF-uninterpretable configuration should be assigned the representation in (\ref{edgey}):

\ea	{}[\textsc{edge}] Y \label{edgey}
\z

The syntactic status of Y is not important here as PF-\isi{uninterpretability} arises because the [\textsc{edge}] \isi{feature} is not attached to any phonologically visible material.

\section{More on attributive modification} \label{sec:4moreonattr}
Since the reason behind \isi{Attributive Comparative Deletion} in \ili{English} is that an inverted \isi{AP} is not allowed in an \isi{edge} position without an \isi{overt operator} element there, it is worth examining how languages and structures differ in this respect. The expectation is that if the \isi{QP} does not invert, then PF-\isi{uninterpretability} does not arise since the \isi{QP} is not in an \isi{edge} position.

It has been seen that in \ili{English} certain QPs require \isi{movement} to the [Spec,\isi{FP}] whereas others do not. Based on the analysis given by \citet{kennedymerchant2000}, the examples in (\ref{inversion4}) all involve this kind of \isi{movement}:

\ea \label{inversion4}
\ea	{}[How big a cat] did Ralph see? \label{howbigacat}
\ex	Ralph bought [too big a cat]. \label{inversiontoobigacat}
\ex	Ralph bought a bigger house than Michael did [a flat].
\z
\z

Underlyingly, in accordance with what has been claimed in the previous sections, the structures of the bracketed constituents are shown in (\ref{structuresinversion}), respectively:

\ea \label{structuresinversion}
\ea	{}[\textsubscript{FP} [\textsubscript{QP} how big]\textsubscript{i} [\textsubscript{DP} a [\textsubscript{NP} \emph{t}\textsubscript{i} cat]]]
\ex	{}[\textsubscript{FP} [\textsubscript{QP} too big]\textsubscript{i} [\textsubscript{DP} a [\textsubscript{NP} \emph{t}\textsubscript{i} cat]]]
\ex	{}[\textsubscript{FP} [\textsubscript{QP} x-big]\textsubscript{i} [\textsubscript{DP} a [\textsubscript{NP} \textit{}\textsubscript{i }flat]]]
\z
\z

Inversion is dependent on two factors: whether the \isi{nominal expression} enables the projection of an \isi{FP}, and whether the \isi{QP} can undergo such \isi{movement}. Regarding latter condition, it has already been shown that not all QPs require this kind of inversion: still, it is expected that the properties of the \isi{QP} can be projected to the entire \isi{nominal expression} via \isi{feature percolation}. Consider the following examples involving the optional \isi{movement} of the \isi{QP} \textit{more intelligent}:

\ea \label{moreintelligent}
\ea []{I have never seen [a more intelligent dog].} \label{amoreintelligentdog}
\ex	[?]{I have never seen [more intelligent a dog].} \label{moreintelligentadog}
\z
\z

In line with \citet[130]{kennedymerchant2000} and \citet[287--288]{bresnan1973}, the structure involving inversion, as shown in (\ref{moreintelligentadog}), is slightly less acceptable than the non-inverted one in (\ref{amoreintelligentdog}). The possibility of (\ref{moreintelligentadog}) shows that \textit{more}, composed of the Deg head -\textit{er} and the Q head \textit{much}, can move to [Spec,\isi{FP}] if that position is generated, but it is also grammatical in its \isi{base position}. By \isi{contrast}, the Q heads in (\ref{structuresinversion}) require \isi{movement} obligatorily. Since the structures in (\ref{moreintelligent}) are otherwise equivalent, it is reasonable to claim that \isi{feature percolation} is possible without \isi{movement} and \isi{movement} is triggered rather by the properties of the individual quantifiers. As the \isi{QP} appears in a [Spec,\isi{NP}] position it can enter into an \isi{agreement} relationship with the N head, which in turn can project its features upwards in the structure. Thus, inversion does not stem from the inability of a \isi{nominal expression} to be marked for quantification otherwise; rather, an [\textsc{edge}] \isi{feature} of a \isi{quantifier} needs to be satisfied by \isi{movement} to an \isi{edge} position.

On the other hand, the availability of a [Spec,\isi{FP}] position seems to be dependent on the internal structure of the \isi{nominal expression} as well. Consider:

\ea \label{howbignum}
\ea	[]{[How big a dog] did Peter see?} \label{howbigadog}
\ex	[*]{[How big dogs] did Peter see?} \label{howbigdogs}
\z
\z
	
As has been established, structures like (\ref{howbigadog}) involve the obligatory \isi{movement} of the \isi{QP} to the [Spec,\isi{FP}] position, which suggests that \textit{how} requires inversion. It follows that in (\ref{howbigdogs}) inversion should happen in order to derive a grammatical configuration; however, (\ref{howbigdogs}) is not grammatical. The difference between (\ref{howbigadog}) and (\ref{howbigdogs}) is that the latter involves a plural, while the former involved the \isi{indefinite article} \textit{a}. If one were to claim that in both cases there is a single \isi{DP} above the \isi{NP} (disregarding now the \isi{FP}), then the difference between (\ref{howbigadog}) and (\ref{howbigdogs}) would be unaccounted for.

Instead, I propose that the reason why (\ref{howbigadog}) allows inversion is that the \isi{DP} layer is not present in the structure, and hence \isi{extraction} is possible; furthermore, the \isi{indefinite article} is the head of a \isi{NumP}. As \citet[11]{zamparelli2008} describes, the \isi{NumP} must always be present in \ili{English}, partly because it is responsible for \isi{agreement} as well, contrary to \ili{Romance} languages that have a separate AgrP for this. The \isi{NumP} may be headed by numerals (e.g. \textit{one}, \textit{two}) or by indefinite articles. The \isi{DP} layer appears above the \isi{NumP}, which is also shown by the fact that Num heads may co-occur with the \isi{definite article}:

\ea \label{nump}
\ea	{}[\textbf{The two} dogs] are sleeping. \label{thetwodogs}
\ex	{}[\textbf{Two} dogs] are sleeping. \label{twodogs}
\z
\z

In (\ref{thetwodogs}), the \isi{nominal expression} is definite as it contains the D head \textit{the}; by \isi{contrast}, the \isi{nominal expression} in (\ref{twodogs}) is indefinite. As should be obvious, \textit{two} in itself does not determine [\textpm{}def] as it may occur in both constructions; in this way, the [--def] nature of the \isi{nominal expression} does not come from the Num head itself but rather from an indefinite zero D head, in the same way as [+def] is marked by \textit{the} in (\ref{nump}). The structure of the DPs in (\ref{nump}) is shown in (\ref{treenump}):

\ea \label{treenump} \upshape
\begin{forest} baseline, qtree, for tree={align=center}
[DP
	[D$'$
		[D
			[the\textsubscript{{[}+def{]}}\\$\emptyset$\textsubscript{{[}--def{]}}]
		]
		[NumP
			[Num$'$ [Num [two]] [NP [dogs,roof]]]
		]
	]
]
\end{forest}
\z

As can be seen, definiteness is encoded in the \isi{DP} layer and not in the \isi{NumP} for Num heads like \textit{two}. However, if the Num head is an \isi{indefinite article}, the situation is quite different because the \isi{indefinite article} is unambiguously associated with [--def], and thus there is no reason for introducing a \isi{DP} layer for marking definiteness separately. Hence, a \isi{nominal expression} such as \textit{a dog} should be assigned the following structure:

\ea \label{treenumpindef} \upshape
\begin{forest} baseline, qtree, for tree={align=center}
[NumP
	[Num$'$
		[Num
			[a\textsubscript{{[}--def{]}}]
		]
		[NP
			[dog,roof]
		]
	]
]
\end{forest}
\z

The structural difference between (\ref{treenump}) and (\ref{treenumpindef}) has a bearing on the availability of inversion, as demonstrated by the \isi{contrast} in (\ref{howbignum}). In cases like (\ref{howbigadog}), that is, with the string \textit{how big a dog}, an \isi{FP} layer is projected on top of the \isi{NumP}, as shown in (\ref{treefp}):

\ea \label{treefp} \upshape
\begin{forest} baseline, qtree, for tree={align=center}
[FP
	[QP\textsubscript{i}
		[how big,roof, name=qp]
	]
	[F$'$
		[F
			[(of)]
		]
		[NumP
			[Num$'$ [Num [a]] [NP [t\textsubscript{i}, name=trace] [N$'$ [N [dog]]]]]
		]
	]
]
\draw[->] (trace) to[out=south west,in=south,looseness=1.5] (qp);
\end{forest}
\z

As can be seen, the \isi{QP} can be extracted from within the \isi{NP} and move to the [Spec,\isi{FP}] position. Depending on the dialect, the F head can be filled by \textit{of}, and precisely due to this option, I do not want to claim that the \isi{FP} is in fact a \isi{DP}, since \textit{of} is clearly not a D head and, unlike D heads in structures like (\ref{treenump}), \textit{of} clearly plays no role in marking definiteness.

By \isi{contrast}, a structure like (\ref{howbigdogs}) involves the plural, that is, *\textit{how big dogs}, and the maximal projection, disregarding the \isi{FP}, is a \isi{DP}, not a \isi{NumP}, as given in (\ref{treenump}), since definiteness is not inherently determined by the Num head. Note that by assuming that in structures like (\ref{treefp}) there is a \isi{NumP} layer generated instead of a \isi{DP}, I propose an analysis that is fundamentally different from the one presented by \citet{kennedymerchant2000}, who do not distinguish between these functional layers in the \isi{nominal expression} and do not discuss the \isi{NumP}. 

\largerpage[1]
It would be highly problematic to claim that DPs project an \isi{FP} layer just as NumPs do: the \isi{DP} is a phase boundary in itself, and therefore the \isi{left edge} of the \isi{nominal expression} is already created. Thus there is no [Spec,\isi{FP}] position for the \isi{QP} to move to. The key difference between the \isi{FP} and the \isi{DP} is precisely this: once the \isi{FP} is projected, it requires material to move to its specifier and, as far as \isi{PF} is concerned, this material has to be associated with a phonologically visible marker equipped with designated properties. The \isi{DP} specifier is not an \isi{edge} position in this sense, since there is no requirement that would rule out [Spec,\isi{DP}] positions that remain unfilled. Consider the following representation:

\ea \label{treenomovedp} \upshape
\begin{forest} baseline, qtree, for tree={align=center}
[DP
	[{\phantom{how big}}, name=dp]
	[D$'$
		[D
			[$\emptyset$\textsubscript{{[}--def{]}}]
		]
		[NumP
			[Num$'$
				[Num [$\emptyset$\textsubscript{{[}plural{]}}]]
				[NP [QP\textsubscript{{[}\textsc{edge}{]}} [how big,roof, name=qp]] [N$'$ [N [dogs]]]]
			]
		]
	]
]
\draw[->] (qp) to[out=south west,in=south,looseness=1.5] (dp);
\path [draw, thick] (dp.north west) -- (dp.south east);
\path [draw, thick] (dp.north east) -- (dp.south west);
\end{forest}
\z

Movement is not available in the way it is in (\ref{treefp}). Therefore, the problem with strings like *\textit{how big dogs} is that though the [\textsc{edge}] \isi{feature} of the \isi{QP} should be checked, there is no element that would attract it to a relevant specifier position, and thus the structure is ungrammatical.

This accounts well for differences between (\ref{howbigadog}) and (\ref{howbigdogs}). There is one more difference to be explained, shown in (\ref{attrnomdiff}):

\ea \label{attrnomdiff}
\ea	[*]{[How big cats] did Ralph see?} \label{howbigcats}
\ex	[]{[How many cats] did Ralph see?} \label{howmanycats}
\z
\z

Interestingly, it seems that while a \isi{QP} like \textit{how big} cannot modify a plural \isi{nominal expression}, a \isi{QP} like \textit{how many} can.\footnote{As will be shown later in connection with \ili{German}, there are interesting cross-linguistic differences in this respect; at this point, what is important for us is that \ili{English} does not allow constructions like (\ref{biggerhouse}).} The same difference holds in comparative subclauses:

\ea \label{biggermore}
\ea	[*]{Ralph bought a bigger house than Michael bought [a flat].} \label{biggerhouse}
\ex	[]{Ralph bought more houses than Michael bought [flats].} \label{morehouses}
\z
\z
	
The difference lies in the fact that \ili{English} requires VP-\isi{ellipsis} in attributive comparatives like (\ref{biggerhouse}), as shown by \citet{kennedymerchant2000}, but does not do so in nominal comparatives like (\ref{morehouses}), as also pointed out by \citet{reglero2006}.

Observing the difference between (\ref{howbigcats}) and (\ref{howmanycats}), it should be clear that the nominal expressions themselves have the same layers, that is, a \isi{DP} and a \isi{NumP} above the \isi{NP} (the \isi{DP} being responsible for definiteness and the \isi{NumP} for marking the plural), and therefore the difference in the acceptability of the two structures stems from differences that hold within the QPs. In other words, QPs such as \textit{how many}, \textit{more} and \textit{x-many} are different from ones like \textit{how big}, and this difference is encoded in the quantifiers.

Recall from \chapref{ch:2} that the Deg head and the Q head are distinct projections but the upward \isi{movement} of the Deg head to the Q head may result in composite forms. For instance, the \isi{movement} of -\textit{er} to \textit{much} results in \textit{more} in strings like \textit{more intelligent}, and the \isi{movement} of -\textit{er} to \textit{many} results in \textit{more} in strings like \textit{more cats}. Note that while both cases result in the surface form \textit{more}, the Q heads themselves (\textit{much} or \textit{many}) are different.

In (\ref{attrnomdiff}), the Deg head itself is \textit{how}, and the Q head is a zero in (\ref{howbigcats}) and \textit{many} in (\ref{howmanycats}); as was argued for in \chapref{ch:2}, the upward \isi{movement} of the Deg head to the Q head results in a reverse order, that is, the original Deg head is adjoined from the right, in line with the Linear Correspondence Axiom (see \citealt{kayne1994}) and the Mirror Principle (see \citealt{baker1985, baker1988}). The zero Q head requires \isi{movement} to a [Spec,\isi{FP}] position, as in constructions such as \textit{how big a cat}; however, the \isi{DP} in (\ref{howbigcats}) has the structure given in (\ref{treenomovedp}), and there is no [Spec,\isi{FP}] position available; consequently, (\ref{howbigcats}) is ungrammatical. By \isi{contrast}, \textit{many} in (\ref{howmanycats}) does not require \isi{movement}, and thus the \isi{QP} may remain in situ, that is, within the \isi{NP}. It should be obvious that only the Q head can be held responsible for obligatory \isi{movement} as the Deg head is \textit{how} in both cases.

Turning now to the structures given in (\ref{biggermore}), it seems plausible that the Q heads (though zero in both cases) differ in a similar way, that is, the one in (\ref{biggerhouse}) requires \isi{movement} to the [Spec,\isi{FP}] position, which results in obligatory \isi{Attributive Comparative Deletion}. This condition of which is not met in (\ref{biggerhouse}), hence its ungrammaticality. By \isi{contrast}, the Q head in the \isi{subclause} in (\ref{morehouses}) does not require \isi{movement} to a [Spec,\isi{FP}] position and as the (phonologically not visible) \isi{QP} is not inverted, the \isi{Overtness Requirement} on left-peripheral elements is not violated, since the \isi{FP} layer is not generated at all.

It is the idiosyncratic property of a given Q head whether it is equipped with the [\textsc{edge}] \isi{feature} triggering \isi{movement} or not, and it is not directly linked to other features. Therefore, while certain quantifiers in \ili{English} require inversion, this may not be true for their counterparts in other languages; this was seen in connection with \citet{reglero2006} in terms of differences between \ili{English} and \ili{Spanish} nominal comparatives. As was discussed in \sectref{sec:4verbdeletion}, \ili{German} does not show \isi{Attributive Comparative Deletion} because no inversion is required. In (\ref{howbigacat}) and (\ref{inversiontoobigacat}), it was shown that QPs such as \textit{how big} and \textit{too big} are inverted in \ili{English}. Consider now the examples in (\ref{nofpgerman4}) from \ili{German}:

\ea \label{nofpgerman4}
\ea	[]{\gll [Eine	wie	große	Katze] hat Ralf gekauft? \label{einewiegrossekatze}\\
\phantom{[}a.\textsc{acc.f} how big.\textsc{acc.f} cat	has	Ralph bought.\textsc{ptcp}\\
\glt `How big a cat did Ralph buy?'}
\ex	[*]{\gll [Wie große eine	Katze] hat Ralf gekauft? \label{wiegrosseeinekatze}\\
\phantom{[}how	big.\textsc{acc.f} a.\textsc{acc.f}	cat	has	Ralph bought.\textsc{ptcp}\\
\glt `How big a cat did Ralph buy?'}
\ex []{\gll	Ralf hat [eine zu	große	Katze] gekauft. \label{einezugrossekatze}\\
Ralph	has \phantom{[}a.\textsc{acc.f}	too	big.\textsc{acc.f} cat bought.\textsc{ptcp}\\
\glt `Ralph bought too big a cat.'}
\ex	[*]{\gll Ralf	hat	[zu	große	eine Katze] gekauft. \label{zugrosseeinekatze}\\
Ralph	has \phantom{[}too big.\textsc{acc.f} a.\textsc{acc.f} cat bought.\textsc{ptcp}\\
\glt `Ralph bought too big a cat.'}
\z
\z

As can be seen, \ili{German} not only allows the non-inverted orders given in (\ref{einewiegrossekatze}) and (\ref{einezugrossekatze}), but actually requires them: (\ref{wiegrosseeinekatze}) and (\ref{zugrosseeinekatze}) -- which are structurally parallel with the \ili{English} examples in (\ref{howbigacat}) and (\ref{inversiontoobigacat}) -- are ungrammatical. This shows that even in cases involving an indefinite \isi{nominal expression}, the \isi{FP} layer is not generated. I do not wish to investigate the internal structure of \ili{German} nominal expressions here; the point is rather that since \ili{German} obviously lacks the \isi{FP} in structures involving overt operators, there is nothing unexpected in the claim that the \isi{FP} is not generated in comparative subclauses such as the one in (\ref{germanacdkatze}), repeated here as (\ref{germanacdkatzerepreated}), which is acceptable (depending on the speaker):

\ea \gll	Ralf	hat	eine	größere	Katze	als Michael	eine breite Katzenklappe hat. \label{germanacdkatzerepreated}\\
Ralph	has	a.\textsc{acc.f} bigger.\textsc{acc.f} cat than Michael a.\textsc{acc.f} wide.\textsc{acc.f} cat.flap has.\\
\glt `Ralph's cat is bigger than Michael's cat flap is wide.'
\z

Since the \isi{QP} in the \isi{subclause} in (\ref{germanacdkatzerepreated}) is obviously within the \isi{NP} and the result is grammatical, it is expected that the \ili{German} QP, contrary to the \ili{English} one, does not have to be deleted.

On the other hand, it is likewise expected that, unlike \ili{English} in (\ref{howbigcats}), \ili{German} allows plurals to appear together with QPs such as \textit{how big}. Consider:

\ea \gll	{}[Wie	große	Katzen]	hat	Ralf gesehen? \label{wiegrossekatzen}\\
\phantom{[}how	big.\textsc{acc.pl}	cats has Ralph seen\\
\glt `How big were the cats that Ralph saw?'
\z

Since there is no requirement on the \isi{QP} in \ili{German} to move to [Spec,\isi{FP}], structures like (\ref{wiegrossekatzen}) are grammatical, whereas they are not derivable in \ili{English}.

All this shows that \isi{Attributive Comparative Deletion} is not a separate mechanism as such but is the surface result of various other factors that interact with each other, namely: whether there is \isi{Comparative Deletion} in the [Spec,\isi{CP}] position (due to the \isi{Overtness Requirement}), whether the \isi{QP} has to move to a [Spec,\isi{FP}] position within the extended nominal projection, and whether the language has VP-\isi{ellipsis}.
