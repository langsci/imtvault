\documentclass[output=paper]{../langscibook}
\ChapterDOI{10.5281/zenodo.5643287}
\title{Multimodalidad y gramática en contacto: prácticas para interrumpir una narración colaborativa en español y en quechua}
\author{Ignacio Satti\orcid{0000-0002-9191-3627}\affiliation{Albert-Ludwigs-Universität Freiburg} and Mario Soto Rodríguez\orcid{0000-0002-6535-1392}\affiliation{Albert-Ludwigs-Universität Freiburg}}
\abstract{Este trabajo presenta un estudio comparativo de la narración colaborativa en diferentes variedades del español y en quechua, con un énfasis en la iniciación de interrupciones por parte de los potenciales conarradores. Un análisis detallado de las prácticas multimodales que utilizan los participantes en este contexto nos permite indagar sobre la interrelación entre recursos verbales y no verbales en esta posición secuencial. Los resultados cuantitativos muestran que en la región de Cochabamba se registran frecuencias de contacto visual y de uso de la gestualidad significativamente menores que en el resto de nuestros datos, provenientes de hablantes de Bogotá, Buenos Aires y migrantes en Friburgo. Los resultados cualitativos sugieren que en la zona de Cochabamba los participantes recurren a estrategias lingüísticas para realizar tareas de tipo modal e interaccional en las prácticas de interrupción, a diferencia de los participantes en el resto de las regiones, donde se favorece el uso de recursos visuales, como la mirada o la gestualidad. En el caso del quechua, encontramos que esta información está codificada en la gramática a partir de morfemas, como \textit{-qa, a/ari}, o \textit{-pis}. En el caso de los hablantes de español de esta zona, observamos que recurren a ciertas estrategias que evidencian fenómenos de contacto estudiados previamente. Nuestro estudio pretende contribuir a la discusión sobre el contacto de lenguas mostrando la relevancia de incorporar un acercamiento multimodal a las prácticas de los hablantes. }
\IfFileExists{../localcommands.tex}{
  \addbibresource{localbibliography.bib}
  \input{../localpackages}
  %Copy this to localcommands.tex

\usepackage[english]{babel}
\usepackage{amsmath}
\usepackage{amssymb,amsfonts,textcomp}
\usepackage{array}
\usepackage{hhline}
\usepackage{hyperref}

\newenvironment{styleStandard}{}{}
\newenvironment{stylelsAbstract}{}{}
\newenvironment{stylelsSectioni}{}{}
\newenvironment{stylelsSectionii}{}{}
\newenvironment{stylelsBulletList}{}{}
\newenvironment{styleBibliographyi}{}{}
\newenvironment{listWWNumxxvleveli}{}{}
\newenvironment{listWWNumxxvlevelii}{}{}
\newenvironment{listWWNumxxvleveliii}{}{}
\newenvironment{listWWNumxxvleveliv}{}{}
\newenvironment{listWWNumixleveli}{}{}
\newenvironment{listWWNumixlevelii}{}{}
\newenvironment{listWWNumixleveliii}{}{}
\newenvironment{listWWNumixleveliv}{}{}

\newcommand\textstyleListLabelxvi[1]{#1}
\newcommand\labellistWWNumxxvleveli{\thelistWWNumxxvleveli.}
\newcommand\labellistWWNumxxvlevelii{\thelistWWNumxxvlevelii.}
\newcommand\labellistWWNumxxvleveliii{\thelistWWNumxxvleveliii.}
\newcommand\labellistWWNumxxvleveliv{\thelistWWNumxxvleveliv.}
\newcommand\labellistWWNumixleveli{[F0B7?]}
\newcommand\labellistWWNumixlevelii{\textstyleListLabelxvi{o}}
\newcommand\labellistWWNumixleveliii{[F0A7?]}
\newcommand\labellistWWNumixleveliv{[F0B7?]}

\newcounter{listWWNumxxvleveli}
\newcounter{listWWNumxxvlevelii}[listWWNumxxvleveli]
\newcounter{listWWNumxxvleveliii}[listWWNumxxvlevelii]
\newcounter{listWWNumxxvleveliv}[listWWNumxxvleveliii]
\newcounter{itemize}  
  \input{../localhyphenation} 
  \togglepaper[1]%%chapternumber
}{}

\shorttitlerunninghead{Multimodalidad y gramática en contacto}

% references
% tikz

\begin{document}
\maketitle

\section{Introducción}
Cuando emerge una narración como actividad interactiva, los participantes se involucran en lo que se considera un “paquete”, es decir, una unidad construida a partir de varios turnos, con una serie de fases secuenciales, cuyo paso de una a la otra se va construyendo como un resultado interactivo por parte de todos los participantes involucrados. En este sentido, se ha demostrado que los participantes se orientan hacia una cierta secuencialidad en la narración de la historia (\citealt{Labov1972book}; \citealt{Jefferson1988}; \citealt{Lerner1992}; \citealt{Stivers2008}; \citealt{Couper-KuhlenSelting2017}). Sin embargo, cuando dos o más participantes tienen conocimiento de los hechos, la narración puede contar con más de un narrador y, potencialmente, con más de una versión de la misma. En este contexto, el participante que no está narrando puede manifestar que no está completamente de acuerdo con algún elemento de la historia en proceso (\citealt{Quasthoff1980}; \citealt{Lerner1992}; \citealt{Sacks1995}). En estos casos, se produce una secuencia lateral \citep{Jefferson1972} que interrumpe la actividad narrativa, en la cual los participantes negocian un potencial desacuerdo sobre los eventos antes de retomar la narración. El ejemplo \REF{ex:satti:1}, a continuación, ilustra el desarrollo de este fenómeno. En este fragmento, Pedro e Inés, que son pareja, están contando a un amigo en común cómo empezaron a vivir juntos en el departamento de Inés.  Sin embargo, poco tiempo después de que Pedro se había mudado, Inés se fue un mes a España. Pedro asegura que convivieron alrededor de dos semanas antes de que Inés se fuera (L577), mientras que Inés interrumpe para aclarar que convivieron, más bien, un mes (L579).

\ea\label{ex:satti:1}
\texttt{Dos semanas juntos (smor201701)}\footnote{En este trabajo adoptamos las convenciones de transcripción del sistema GAT 2. Para más información véase \citet{EhmerEtAl2019}: \url{http://www.gespraechsforschung-online.de/fileadmin/dateien/heft2019/px-gat2-espanol.pdf}. Además, hemos utilizado nombres ficticios y una anonimización parcial de las imágenes ilustrativas para proteger la identidad de los participantes.}
\begin{verbatim}
575 PED: o SEa;=yo entrÉ aquí a viVIR,
576      y me (.) yo creo que entrÉ, 
577      eh (.) vivimos !NA!da (.) dos semanas JUNtos, (0.66)           
578      tú te fuiste un MES- (0.46)     
579 INE: no:;=dos semAnas NO.
580      vivi[mos  co]mo unn mE:s o algo [aSÍ:, ] 
581 PED:     [o TRES,]
582                                      [BUEno;]=viVImos un MES.
583      tú te fuistes (.)
584      ella se fue (.) Un MES para: (.) para esPAña;
\end{verbatim}
\z

En la línea 579, Inés interrumpe la progresividad de la narración para corregir la cantidad de tiempo que pasaron juntos. Esta interrupción inicia una breve secuencia lateral que se concluye en la línea 582 con un turno iniciado por \textit{bueno} \citep{Raymond2018} con función de cierre secuencial \citep{Park2010}, en el cual Pedro acepta que vivieron un mes juntos antes de la partida de Inés. La “lateralidad” de esta secuencia puede observarse también directamente desde la perspectiva de los participantes, ya que Pedro retoma la narración recurriendo a la misma construcción que había utilizado en el momento en que esta había sido interrumpida (véanse las líneas 578 y 583). 

En este trabajo nos proponemos analizar las prácticas multimodales que utilizan los participantes para gestionar localmente este tipo de interrupciones en la narración colaborativa en español y en quechua, con un énfasis en los recursos que movilizan los conarradores para iniciar la interrupción. El análisis se centrará en la interrelación entre los recursos verbales y no verbales que componen esta práctica y en las diferencias y similitudes que registramos en hablantes de la región de Cochabamba y en otros grupos hispanohablantes (Bogotá, Buenos Aires y migrantes en Friburgo). El estudio comparativo se focalizará en la región de Cochabamba, en donde hemos registrado una frecuencia de contacto visual significativamente menor y en donde el español se encuentra en contacto con la lengua quechua, es decir, con una lengua aglutinante que dista tipológicamente del español. Para llevar a cabo el estudio, aplicamos la metodología del análisis de la conversación (\citealt{Sacks1995}; \citealt{Schegloff2007}) y de la lingüística interaccional (\citealt{Couper-KuhlenSelting2017}) con una perspectiva comparada que tiene en cuenta las dinámicas de contacto (\citealt{MartínezSperanza2009}; \citealt{Pfänder2009};  \citealt{SotoRodríguez2013}; \citealt{Blestel2015habiasido}; \citealt{Dankel2015}; \citealt{PalaciosAlcaine2017book}).

\subsection{La narración conversacional (colaborativa)}

Nuestro enfoque de la narración es un enfoque conversacional (\citealt{Jefferson1978}; \citealt{GülichQuasthoff1986}; \citealt{Sacks1995}; \citealt{Mandelbaum2012}), es decir, consideramos la narración como un resultado de la interacción entre el narrador y el receptor de la historia. Esto se debe, principalmente, a que la narración ha sido descrita como un “gran paquete” (\citealt{Jefferson1988}; \citealt{Sacks1995}; \citealt{Couper-KuhlenSelting2017}) que conlleva la suspensión de la interacción por turnos, de modo tal que uno de los participantes pueda tener derecho a la palabra durante una cierta cantidad de turnos de forma consecutiva. Para lograr esto, los participantes utilizan ciertas técnicas para mantener la unidad de la historia y hacer interpretables los elementos de la misma, posibilitando que el receptor pueda reaccionar adecuadamente y anticipar su finalización, o bien bloquear su avance. Además, el hecho de que se trate de un paquete tiene una segunda consecuencia: resulta necesario analizar las prácticas locales de la narración en relación con la estructura global de la historia, hacia la cual se ha demostrado que los participantes se orientan (\citealt{Labov1972book}; \citealt{Couper-KuhlenSelting2017}).

En caso de que haya tres o más participantes presentes, y al menos dos de ellos tengan conocimiento de los eventos, puede producirse lo que se conoce como una narración colaborativa (\citealt{Quasthoff1980}; \citealt{Goodwin1981}; \citealt{Mandelbaum1987}; \citealt{Sacks1995}). En este caso, uno de los participantes puede convertirse potencialmente en conarrador de la historia e intervenir en el desarrollo de la misma. Este tipo de narraciones, y particularmente las interrupciones por parte del conarrador, son el objeto de estudio de este trabajo.

\subsection{Un acercamiento multimodal}

Nuestro enfoque propone un acercamiento multimodal a los datos que intenta vincular la teoría de la contextualización \citep{Gumperz1982} y el análisis multimodal de la interacción (\citealt{StiversSidnell2005}; \citealt{Mondada2019}). El interés está, entonces, en la interrelación entre recursos verbales y no verbales como mecanismos de contextualización para hacer interpretables determinadas acciones en contextos secuenciales específicos. Nos concentraremos particularmente en el comportamiento de la mirada y en la movilización de recursos visuoespaciales\footnote{Utilizamos el término “visuoespacial” según lo propuesto por  \citet{StiversSidnell2005}.} como, por ejemplo, gestos y expresiones faciales, y en qué modo estos se relacionan con la movilización de recursos léxicos y gramaticales.

Los diferentes usos de la mirada en las comunidades de habla presentes en nuestro corpus son de central interés para este trabajo. Distintos estudios han resaltado la importancia de la mirada como recurso de coordinación para la organización de la toma de turno (\citealt{Kendon1967}; \citealt{Rossano2012}) y la organización de la participación \citep{Goodwin1981}. Estos estudios han identificado patrones sistemáticos en el comportamiento de la mirada en distintos grupos de hablantes. Estudios recientes que utilizan grabaciones con traqueo de la mirada han ofrecido evidencia cuantitativa que confirma muchos de estos patrones (\citealt{BrôneEtAl2017}; \citealt{ZimaEtAl2019}). Sin embargo, también se registran estudios que han cuestionado la universalidad de los mismos y han sugerido diferencias culturales en el comportamiento de la mirada (\citealt{RossanoEtAl2009}). Además de comprobar si estas diferencias se aplican a otras regiones no estudiadas, también resulta interesante observar si las mismas tienen relación con los recursos lingüísticos de los hablantes. En este sentido, la región de Cochabamba resulta de particular interés, ya que, por un lado, aún no contamos con estudios sistemáticos sobre el comportamiento de la mirada en la región andina, y, por otro lado, en esta región se presenta una situación de contacto intenso con lenguas tipológicamente distantes del español, mayoritariamente el quechua.

En la interacción cara a cara, en donde es posible establecer contacto visual, los participantes cuentan tanto con recursos vocales-auditivos como con recursos visuoespaciales (\citealt{StiversSidnell2005}). El avance en las técnicas de recopilación de datos ha posibilitado el estudio de los recursos visuoespaciales y ha evidenciado la importancia de este tipo de recursos en la comunicación. En este trabajo, entonces, intentamos abrir una nueva perspectiva que tenga en cuenta la interrelación entre las distintas modalidades en el estudio de fenómenos de contacto.

\subsection{Un acercamiento intercultural}

{Inicialmente motivados por e}l interés de constatar los principios {universales de la cortesía, diversos estudios sobre comunicación intercultural han reportado diferencias con respecto a las sociedades occidentales en las prácticas comunicativas de otras lenguas y culturas. Estos estudios han llegado a constatar una correlación entre la estructura gramatical y los códigos socio-culturales de convivencia de cada comunidad} {(\citealt{Sifianou1992}; \citealt{TannenKakava1992}; \citealt{Yamada1997}; \citealt{Mori1999}; \citealt{Watanabe2005})}{ y han remarcado la importancia de distintos aspectos indexicales en las prácticas de los hablantes} (\citealt{BlommaertEtAl2005}; \citealt{BucholtzHall2005}){. Sin embargo, aún resultan escasos los estudios sistemáticos que se han concentrado en la interrelación entre recursos verbales y no verbales en comunidades indígenas de Latinoamérica que han tenido poco contacto con la sociedad occidental. Si bien se trata de primeras aproximaciones a las prácticas comunicativas de estas comunidades, los estudios existentes ofrecen algunas pautas para la descripción de los recursos comunicativos multimodales en estas comunidades y evidencian particularidades importantes con respecto a las prácticas comunicativas, a los códigos culturales y a la manifestación de diferentes nociones conceptuales determinadas por vivencias culturales. Entre ellos destacan las referencias a las comunidades amazónicas 
Aweti} {\citep{Reiter2013}}{, 
Nheengatú} {\citep{Floyd2016}}{,
a la comunidad amazónica quechua Pastaza} {\citep{Nuckolls2015}}{,
a la comunidad quechua Ancash} {\citep{Shapero2014}}{, 
a las culturas Mayas} {\citep{Brown2014}}{, 
a la comunidad Chol May} \citep{Rodríguez2019}{ y 
al grupo Nigua en México} (\citealt{CalderónEtAl2019}){. La referencia más completa es, tal vez, el estudio de \citealt{NúñezSweetser2006} sobre la temporalidad en comunidades aymaras, en el cual se da cuenta de un sistema inverso de referencia temporal que se corrobora con gestos deícticos.}

\section{Corpus y metodología}\label{sec:satti:2}

En este trabajo hemos recurrido a dos fuentes. En primer lugar, utilizamos principalmente grabaciones del corpus en formación Freiburg SofaTalks (FST), en el cual dos personas que se encuentran en su casa, hablan sobre experiencias en común frente a un tercero. Si bien este último es quien lleva la cámara al espacio de los participantes, en todos los casos tiene una relación personal con los mismos y es, además, parte de su comunidad de habla. El corpus tiene en este momento más de 200 grabaciones en diferentes lenguas. Para los fines de este trabajo, hemos utilizado la preselección de grabaciones que son parte del proyecto de doctorado de Ignacio Satti, que reúne un total de 25 grabaciones provenientes de la provincia de Buenos Aires (Argentina), de Cochabamba (Bolivia), de Bogotá (Colombia) y de migrantes hispanohablantes residentes en la ciudad de Friburgo (Alemania). Además, en segundo lugar, hemos completado nuestros datos con una grabación en video de una narración colaborativa en quechua disponible como material de enseñanza de la lengua.\footnote {Véase \citet{AndersenDaza1994}} En ella, dos ancianos de la comunidad semi-urbana Ucuchi de la región de Cochabamba (Bolivia) narran sobre su vida a uno de sus hijos.  Para los datos provenientes del corpus FST, hemos utilizado las convenciones de transcripción GAT2 \citep{EhmerEtAl2019} mientras que para la grabación en quechua hemos recurrido, además, al alfabeto fonémico quechua oficializado a través del decreto N\textsuperscript{o} 20227 \citep[86]{Choque2005}.

Hemos codificado más de seis horas de narraciones colaborativas por parte de 44 participantes de las regiones previamente mencionadas. En primer lugar, hemos identificado un total de 129 instancias de narración colaborativa basándonos en los estudios conversacionales mencionados previamente. En segundo lugar, hemos identificado los turnos en los que el conarrador interrumpe la progresividad de la narración y posteriormente alguno de los participantes retoma la actividad narrativa. De esta manera, hemos obtenido un total de 117 secuencias laterales de este tipo. Por último, en estas mismas grabaciones, hemos realizado una codificación de la mirada utilizando el programa ELAN (\figref{fig:satti:1}), que ilustraremos a continuación.

  
\begin{figure}
\includegraphics[width=\textwidth]{figures/Satti-img001.jpg}
\caption{\label{fig:satti:1} Ilustración de la anotación de la mirada con el programa ELAN}
\end{figure}

Durante las instancias de narración colaborativa, hemos anotado los momentos en los que la persona sentada a la izquierda mira a la persona sentada a la derecha (\figref{fig:satti:2}) y los momentos en los que la persona sentada a la derecha mira a la persona sentada la izquierda (\figref{fig:satti:3}). Cada una de estas líneas ha sido codificada individualmente para favorecer la precisión de los resultados. Posteriormente, hemos creado una línea a partir de las superposiciones de las dos primeras, lo cual nos ha proporcionado los momentos de contacto visual (\figref{fig:satti:4}).

\begin{figure}
\includegraphics[width=.8\textwidth]{figures/Satti-img002.jpg}
\caption{\label{fig:satti:2}  Persona sentada a la izquierda mira a la de la derecha}
\end{figure}

\begin{figure}
\includegraphics[width=.8\textwidth]{figures/Satti-img003.jpg}
\caption{\label{fig:satti:3} Persona sentada a la derecha mira a la de la izquierda}
\end{figure}

\begin{figure}
\includegraphics[width=.8\textwidth]{figures/Satti-img004.jpg}
\caption{\label{fig:satti:4}Contacto visual}
\end{figure}

A partir de esta codificación, hemos obtenido tres datos que resultan centrales para nuestro trabajo: momentos en los que solamente uno de los participantes mira al otro, momentos de contacto visual y momentos en los que ninguno de los participantes mira al otro, ya sea porque ambos miran al receptor de la historia (\figref{fig:satti:5}) o porque uno de ellos mira al vacío (\figref{fig:satti:6}), entre otras posibilidades menos frecuentes.

\begin{figure}
\includegraphics[width=.8\textwidth]{figures/Satti-img005.jpg}
\caption{\label{fig:satti:5} Ambos hablantes miran al receptor}
\end{figure}

\begin{figure}
\includegraphics[width=.8\textwidth]{figures/Satti-img006.jpg}
\caption{\label{fig:satti:6} Persona sentada a la derecha mira hacia abajo}
\end{figure}

Una vez finalizada la codificación, hemos realizado una comparación regional entre los participantes de la región de Cochabamba y el resto de los participantes.\footnote{Si bien hemos registrado algunas diferencias de frecuencia entre los participantes de Bogotá, Buenos Aires y Friburgo, estas son mucho menores y en todos los casos contrastantes con las cifras de la región de Cochabamba.}  

\begin{table}
\caption{\label{tab:satti:1} comparación regional del comportamiento de la mirada en narraciones colaborativas.}
\begin{tabularx}{\textwidth}{Qrr}
\lsptoprule
& Cocha. & Bog., Bs. As., Fri.\\
\midrule
Uno de los dos mira al otro & 11,6\% & 56\%\\
Contacto visual & 1,4\% & 11,5\%\\
Ninguno de ellos mira al otro & 87\% & 32,5\%\\
Tiempo total (narraciones colaborativas) & 2h15m & 3h50m\\
Número de participantes & 14 & 30\\
\lspbottomrule
\end{tabularx}
\end{table}

Como se puede observar (\tabref{tab:satti:1}), la frecuencia del uso de la mirada en los participantes de la región de Cochabamba es mucho menor, tanto en lo que respecta a momentos en los que uno de los participantes mira al otro (cinco veces menor) como a momentos de contacto visual (más de ocho veces menor).

Nuestra interpretación de los resultados nos permite suponer que esta diferencia se debe a que la configuración por defecto durante las narraciones colaborativas en estas regiones es diferente. Mientras en nuestros datos de Bogotá, Buenos Aires y Friburgo el potencial conarrador mira al narrador mientras este se encuentra narrando (\figref{fig:satti:7}), en nuestros datos de Cochabamba el potencial conarrador mira al vacío (o bien al receptor de la historia) mientras el narrador está contando la historia (\figref{fig:satti:8}).

\begin{figure}
\subfigure[Bogotá, Buenos Aires, Friburgo]{\label{fig:satti:7}
\includegraphics[height=.15\textheight]{figures/Satti-img007.jpg}
}%
\subfigure[Cochabamba]{\label{fig:satti:8}
\includegraphics[height=.15\textheight]{figures/Satti-img008.jpg}
}
\caption{\label{fig:satti:78}Configuraciones}
\end{figure}

La diferencia en la configuración por defecto también tiene una influencia directa en el contacto visual, ya que el narrador se gira con frecuencia hacia el conarrador. Es decir, cuando el narrador se gira hacia el conarrador, y este lo está mirando, se establece inmediatamente contacto visual (\figref{fig:satti:9}), mientras que, si el conarrador no lo está mirando, no se establece contacto visual, o al menos no inmediatamente (\figref{fig:satti:10}).


\begin{figure}
\includegraphics[height=.15\textheight,width=\textwidth]{figures/Satti-img009.jpg}
\caption{\label{fig:satti:9}Establecimiento del contacto visual cuando el narrador se gira hacia el conarrador.}
\end{figure}

\begin{figure}
\includegraphics[height=.15\textheight]{figures/Satti-img010.jpg}
\caption{\label{fig:satti:10}Giro del narrador hacia el conarrador sin establecimiento del contacto visual.}
\end{figure}

El establecimiento del contacto visual permite la movilización de recursos visuales, ya que podemos suponer que estos serán más efectivos si el otro participante los está mirando. Esto significa que, potencialmente, en un mismo contexto secuencial los participantes no tendrían los mismos recursos disponibles para hacer interpretable una acción. Por esta razón, la hipótesis que subyace a este proyecto es que el hecho de que en un determinado contexto secuencial los participantes dispongan localmente de diferentes recursos multimodales tiene una influencia en la selección de recursos lingüísticos que utilizan para realizar la misma acción.

\section{Análisis}

El análisis llevado a cabo en este capítulo se basa en la colección de 117 instancias de interrupciones en la progresividad narrativa que generan una secuencia lateral.\footnote{En este trabajo, presentamos algunos resultados del análisis que resultan relevantes para los fines del mismo. Un análisis detallado de las mismas está siendo realizado en la tesis de doctorado de Ignacio Satti.} En primer lugar, vamos a analizar los recursos recurrentes de tipo lingüístico (\sectref{sec:satti:3.1}) y de tipo visuoespacial (\sectref{sec:satti:3.2}) en la iniciación de las interrupciones, focalizando en las similitudes y diferencias entre la región de Cochabamba y el resto de nuestros datos. A continuación, vamos a relacionar esta información con la frecuencia de contacto visual en este contexto (\sectref{sec:satti:3.3}). Por último, vamos a realizar un análisis cualitativo detallado de los recursos utilizados en la región de Cochabamba en su contexto de aparición, tanto en quechua (\sectref{sec:satti:3.4}) como en español (\sectref{sec:satti:3.5}).

\subsection{Recursos recurrentes de tipo lingüístico para la iniciación de las interrupciones}\label{sec:satti:3.1}
\largerpage
Con respecto a los recursos lingüísticos más recurrentes que se utilizan para iniciar la interrupción, sorpresivamente no hemos encontrado diferencias significativas entre los participantes en los diferentes contextos de estudio. Hemos identificado tres grupos fundamentales que aparecen en nuestros datos de forma general, incluyendo el quechua. Estos son turnos iniciados con \textit{no} como en el ejemplo \REF {ex:satti:2};\footnote{En quechua, la interrupción se produce con el equivalente \textit{mana}.} turnos iniciados con \textit{pero} como en el ejemplo \REF {ex:satti:3} ; turnos iniciados con otra partícula pero que incluyen \textit{no} o \textit{pero} como en los ejemplos \REF {ex:satti:4} y \REF {ex:satti:5} respectivamente, es decir, sin que se encuentren en posición inicial de turno; y otros más variados, como aquellos que construyen un contraste local a partir de diferentes recursos como en el ejemplo \REF {ex:satti:6}.\footnote{En este caso, una pregunta iniciada con el interrogativo \textit{qué}, similar a lo que muestra \citet{Koshik2003} para el inglés.}


\ea\label{ex:satti:2}
\begin{verbatim}
(smor201701)
77 INE: estoy enamoRAdo.
78 PED: no;=tampoco te Pases.
\end{verbatim}
\z

\ea\label{ex:satti:3}
\begin{verbatim}
(smor201701)
242 PED: y le digo al ta[xIsta]-
243 INE:                [ pEro] (.) anTOnio;
244      yo llevaba mucho esTRÉS.
\end{verbatim}
\z

\ea\label{ex:satti:4}
\begin{verbatim}
(ssat201801)
58 MAR: o sEa tu tenías pueBLItos alrededor que igual eran más- (.)
59      pero (.) nosotros siEmpre cenábamos ahÍ.
60      (0.5)
61 PAB: y: no:;=era el (.)
62      que en esa Isla eran los Únicos lugares que había para IR;
\end{verbatim}
\z

\ea\label{ex:satti:5}
\begin{verbatim}
(ssat201702)
35 STE: lula se re cagó a puteAdas con la MUjer;
36      <<a> no::=qué te voy a roba:r,=
37      =<<a> no sé qué.>
38 CLA: y que:-
39      BUEno;
40      pero estaban todos sin tomar un caFÉ,
\end{verbatim}
\z

\ea\label{ex:satti:6}
\begin{verbatim}
(ssat201802)
98 ARM: QUINce horas. (0.3)
99 LUD: <<glotal> qué va a tardAr QUINce horas;>
\end{verbatim}
\z

\begin{table}
\caption{\label{tab:satti:2}Recursos recurrentes de tipo lingüístico en las interrupciones en español}

\begin{tabularx}{\textwidth}{Qr@{~}r@{~}r}
\lsptoprule
& {  Bog., Bs As., Fri.} & {  Cocha. (español)} & {  Cocha. (quechua)}\\
\midrule
{  Interrupciones con \textit{no}}
{  Total (posición inicial)} & 44 (33) & 9 (8) & 1 (1)\\
% \tablevspace
{  Interrupciones con \textit{pero}.}
{  Total (posición inicial)} & 21 (14) & 5 (3) & 1 (0)\\
% \tablevspace
 Interrupciones con otros recursos & 32 & 8 & 1\\
 \midrule
 Total & 97 & 22 & 3\\
\lspbottomrule
\end{tabularx}
\end{table}

En la \tabref{tab:satti:2}\footnote{El número total de casos (122) es mayor al total de interrupciones (117) ya que en 5 oportunidades encontramos la combinación \textit{no,pero} o \textit{pero, no}.} observamos la cantidad total de casos con cada uno de estos recursos lingüísticos. No resulta sorprendente que los participantes movilicen recursos como \textit{no}, \textit{pero} o construyan un contraste, puesto que la motivación principal de los turnos de interrupción es corregir o reformular algo del turno del narrador. Si bien la muestra es relativamente pequeña, las tendencias generales no parecen indicar diferencias significativas respecto a los recursos que se utilizan para iniciar la interrupción. En cualquier caso, esta similitud relativa apoya nuestra idea de que se trata de una práctica comparable.

\subsection{Recursos recurrentes de tipo visuoespacial en el diseño de las interrupciones}\label{sec:satti:3.2}

Con respecto a los recursos visuoespaciales, dados los resultados de la comparación regional del comportamiento de la mirada, sí esperaríamos una diferencia de frecuencia entre los datos de Cochabamba y los datos del resto de los grupos estudiados. Por esta razón, hemos decidido observar la frecuencia en el uso de recursos visuoespaciales en el diseño del turno \citep{Drew2012} que inicia la interrupción. Para ello, nos hemos concentrado en los gestos manuales y en los cambios en la expresión facial, sobre los cuales podemos decir que, potencialmente, tienen valor comunicativo. Hemos identificado cuatro categorías: en primer lugar, turnos que están acompañados de un cambio en la expresión facial, como en \REF{ex:satti:7}, donde encontramos el ladeado de la cabeza y el arqueo de las cejas (\figref{fig:satti:12}); turnos acompañados de un gesto manual, como en \REF{ex:satti:8}, en donde encontramos un gesto de presentación con la palma vertical (\figref{fig:satti:15}); turnos acompañados tanto de una expresión facial como de un gesto manual, como en \REF{ex:satti:9}, en donde la participante acompaña el turno con arqueo de las cejas y el gesto con el dedo índice alzado (\figref{fig:satti:18}); y turnos que no están acompañados con claridad por cambios en la expresión facial o gestos manuales, como en \REF{ex:satti:10}, en donde no podemos apreciar cambios en la gestualidad durante la totalidad del turno que inicia la interrupción (\figref{fig:satti:202122}, participante a la derecha de la pantalla).

\ea\label{ex:satti:7}
\begin{verbatim}
(ssat201701)
01 PAB: ⁅a⁆no::
02      ⁅b⁆MEnos teníamos;
03      MUcho mEnos.⁅c⁆
\end{verbatim}
\z

\begin{figure}
\subfigure[no::]{\label{fig:satti:11}
\includegraphics[width=.3\textwidth]{figures/Satti-img011.jpg}
}%
\subfigure[Menos]{\label{fig:satti:12}
\includegraphics[width=.3\textwidth]{figures/Satti-img012.jpg}
}%
\subfigure[mEnos]{\label{fig:satti:13}
\includegraphics[width=.3\textwidth]{figures/Satti-img013.jpg}
}
\caption{\biberror{Secuencia correspondiente a  \REF{ex:satti:7}}}
\label{fig:satti:1112131}
\end{figure}

\largerpage[2]
\ea\label{ex:satti:8}
\begin{verbatim}
(smor201701)
18 PED: no⁅a⁆.=estaba⁅b⁆ con el juan⁅c⁆cito.
\end{verbatim}
\z


\begin{figure}
\subfigure[no]{\label{fig:satti:14}
\includegraphics[width=.3\textwidth]{figures/Satti-img014.jpg}
}%
\subfigure[estaba]{\label{fig:satti:15}
\includegraphics[width=.3\textwidth]{figures/Satti-img015.jpg}
}%
\subfigure[juancito]{\label{fig:satti:16}
\includegraphics[width=.3\textwidth]{figures/Satti-img016.jpg}
}
\caption{\biberror{Secuencia que remite a \REF{ex:satti:8}}}
\end{figure}

\clearpage

\ea\label{ex:satti:9}
\begin{verbatim}
(sorj201801)
06 ANG: no;=
07~ ~  =eso ya⁅a⁆ fuE el⁅b⁆ reGREso⁅c⁆.
\end{verbatim}
\z



\begin{figure}
\subfigure[ya fuE]{\label{fig:satti:17}
\includegraphics[width=.3\textwidth]{figures/Satti-img017.jpg}
}%
\subfigure[el]{\label{fig:satti:18}
\includegraphics[width=.3\textwidth]{figures/Satti-img018.jpg}
}%
\subfigure[reGREso]{\label{fig:satti:19}
\includegraphics[width=.3\textwidth]{figures/Satti-img019.jpg}
}
\caption{\biberror{Secuencia que remite a  \REF{ex:satti:9}}}
\end{figure}

\ea\label{ex:satti:10}
\begin{verbatim}
(ssot201701)
08 ALI: [ºh] no-=⁅a⁆era de una ⁅b⁆JOda que había Ido yo⁅c⁆.
\end{verbatim}
\z


\begin{figure}
\subfigure[era]{\label{fig:satti:20}
\includegraphics[width=.3\textwidth]{figures/Satti-img020.jpg}
}%
\subfigure[JOda]{\label{fig:satti:21}
\includegraphics[width=.3\textwidth]{figures/Satti-img021.jpg}
}%
\subfigure[yo]{\label{fig:satti:22}
\includegraphics[width=.3\textwidth]{figures/Satti-img022.jpg}
}
\caption{\biberror{Secuencia que remite a \REF{ex:satti:10}}}
\label{fig:satti:202122}
\end{figure}

En la \tabref{tab:satti:3a} encontramos un resumen de los recursos recurrentes de tipo visual en la iniciación de la interrupción en las regiones estudiadas.

\begin{table}
\caption{\label{tab:satti:3a}Recursos recurrentes de tipo visual en la iniciación de la interrupción}
\begin{tabularx}{\textwidth}{Xrrr}
\lsptoprule
& Bue. Bog. Fri. & Cocha. (Español) & Cocha. (Quechua)\\
\midrule
Expresión facial & 37 (40,2\%) & 4 (18,2\%) & 0\\
Gesto & 18 (19,6\%) & 3 (13,6\%) & 0\\
Ambos & 29 (31,5\%) & 2 (9,1\%) & 0\\
Ninguno claro & 8 (8,7\%) & 13 (59,1\%) & 3 (100\%)\\
\midrule
Total & 92 & 22 & 3\\
\lspbottomrule
\end{tabularx}
\end{table}

La comparación regional nos da la pauta de que, en las regiones de Bogotá, Buenos Aires y Friburgo, los participantes diseñan el turno de la interrupción incluyendo gestos o expresiones faciales con más frecuencia de lo que lo hacen en la región de Cochabamba. Esto resulta interesante si se tiene en cuenta que este tipo de información visual puede adoptar funciones modales (con respecto a expresiones faciales, véase  \citealt{KaukomaaEtAl2014}; \citealt{RoseanoEtAl2015}; con respecto a gestos manuales, véase \citealt{Kendon2004}; \citealt{BressemMüller2014}) y funciones de organización de la interacción (\citealt{Goodwin1986}; \citealt{Mondada2016}; \citealt{DankelSatti2019}). En este sentido, esperaríamos una alta frecuencia del uso de recursos con estas funciones en la iniciación de las secuencias laterales de negociación que representan la base de este trabajo.

\subsection{Establecimiento del contacto visual durante la iniciación de las interrupciones}\label{sec:satti:3.3}

El hecho de que un participante diseñe la interrupción con gestos o expresiones faciales no significa necesariamente que el otro participante las haya observado o que estas hayan sido relevantes para la interacción. Una forma de constatar si la movilización de gestos o de expresiones faciales tiene relevancia en el contexto de la interrupción es observar si la diferencia en la frecuencia del uso de recursos visuoespaciales se condice con la frecuencia de establecimiento de contacto visual en este mismo contexto. Para ello, hemos registrado (utilizando la misma metodología expuesta en \sectref{sec:satti:2}) si se establece contacto visual durante el par adyacente de la interrupción, es decir, si los participantes establecen contacto visual en el momento de interrumpir o de ser interrumpidos (\tabref{tab:satti:3b}).

\begin{table}
\begin{tabularx}{\textwidth}{Xrr}
\lsptoprule
& {  Cocha.}  & {  Bog. Bue. Fri.} \\
\midrule
Se establece contacto visual & 28\% (7/25) & 82,6\% (76/92)\\
\lspbottomrule
\end{tabularx}
\caption{\label{tab:satti:3b} Establecimiento del contacto visual en el par adyacente de la interrupción}
\end{table}

Los resultados reflejan la importancia de establecer contacto visual al inicio de la interrupción en nuestros datos de Buenos Aires, Bogotá y Friburgo (más del 80\% de los casos), mientras que en las interrupciones en la región de Cochabamba el resultado es significativamente menor (28\%). Este resultado va en línea con los resultados expuestos en \sectref{sec:satti:3.2}, en el sentido de que parece haber una relación entre el establecimiento de contacto visual y la movilización de recursos visuoespaciales en el turno que inicia la interrupción.  De cualquier manera, también es necesario remarcar que el solo hecho de establecer contacto visual puede tener, en sí mismo, funciones modales \citep{Kendon1967}, por lo que la relación entre ambos no es necesariamente simétrica.

Una pregunta emerge naturalmente de estos resultados. Si en nuestros datos de la región de Cochabamba los participantes no hacen uso de recursos visuoespaciales en el diseño de la interrupción, ¿qué tipo de recursos utilizan en este mismo contexto para realizar funciones modales o interaccionales? Esta pregunta va en línea con lo que plantean \citet[239-240]{RossanoEtAl2009} para el tzeltal, en donde el sistema de respuesta con repetición, por ejemplo, podría funcionar como un sistema que sustituye a la mirada como indicador de recepción. Para obtener una respuesta en nuestros datos, realizaremos un análisis detallado de las interrupciones presentes en nuestros datos en quechua (\sectref{sec:satti:3.4}) y en español de la zona de Cochabamba (\sectref{sec:satti:3.5}).

\subsection{Análisis cualitativo de las interrupciones en quechua}\label{sec:satti:3.4}

Nuestra grabación en lengua quechua corresponde a una pareja de ancianos de la localidad de Ucuchi, una zona rural a 40 km de la ciudad de Cochabamba. En la misma, hemos identificado tres interrupciones en la progresividad de la narración que resultan comparables con las que hemos registrado en nuestros datos en español. Una de ellas se encuentra en el ejemplo \REF{ex:satti:11}. En este caso, Leo está contando a su hijo sobre su retorno al pueblo después de realizar el servicio militar.  Sin embargo, Isabel, quien también conoce la historia, interrumpe la narración de Leo para recordarle que ya habían tenido un hijo antes de que partiera hacia el cuartel (L18). Esto se evidencia como un reclamo en el desarrollo posterior de la secuencia narrativa. La interrupción inicia una secuencia lateral extensa en donde los participantes negocian responsabilidades con respecto a este hijo.\footnote{La secuencia completa está siendo analizada en un proyecto paralelo por parte de Mario Soto Rodríguez. En este trabajo nos ceñimos al análisis de las estrategias de interrupción por cuestiones de espacio.}  Es importante remarcar que, durante la totalidad de la secuencia, no se produce contacto visual entre los participantes.

\ea\label{ex:satti:11}
\begin{verbatim}
Wawayux („Return to Ucuchi“)
16 LEO: cuartelman rispatax     chayman k' (0.2)
        cuartel-DIR ir-GER-CONT ahí-ABL
        y después de ir al cuartel

\end{verbatim}
\newpage
\begin{verbatim}
17      kutimuspa (0.5) kan[i]
        volver-TRANS-GER ser-1S
        me volví

18 ISA:                    [c]uartelman pero qan::: (0.4)
                            cuartel-DIR pero TÚ
                            pero cuando tú del cuartel:::

19      wawayux  kasqayman       w::awayu:x:
        hijo-COM ser-PART-1S-ABL hijo-COM
        cuando ya tuve un hijo, un hijo....

20      wawa rixsisqamantaña
        hijo conocer-ABL-INC
        después de conocer al niño...

21      qan rinki cuartelmanqa
        tú  ir-1S cuartel-DIR-TOP
        fuiste al cuartel

22 LEO: mhm ((rie))
\end{verbatim}
\z

Como podemos observar, Isabel interrumpe con una breve superposición en la línea 18 y construye la interrupción a partir de un turno de múltiples unidades.\footnote{La superposición de turnos de habla en nuestra grabación quechua es muy escasa, similar a lo que reporta \citet{Yamada1997} para el japonés. Los pocos casos de superposiciones se producen efectivamente durante las interrupciones de los conarradores, aunque son más breves que las presentes en español.}  En este ejemplo resultan interesantes dos recursos gramaticales que utiliza Isabel con función apelativa y noción de reclamo. El primero de ellos es el uso de \textit{pero} en la iniciación de la interrupción. Este uso de \textit{pero} parece coincidir con el uso de \textit{pero} en nuestros datos en español (\sectref{sec:satti:3.1}), aunque, en este caso, la conjunción se encuentra pospuesta a un sintagma adverbial (\textit{cuartelman pero qan}) a la manera de una partícula modal quechua, es decir, integrada y adaptada al sistema morfosintáctico de esta lengua. En este sentido, el empleo de esta partícula en una narración en lengua quechua evidencia la importancia del contacto lingüístico en la zona. El segundo recurso gramatical que moviliza Isabel es la inclusión del morfema \textit{{}-qa} al final de su turno. Las gramáticas del quechua consideran esta partícula como marcador de tópico y le atribuyen locaciones enunciativas iniciales (\citealt{Cerrón-Palomino2008};  \citealt{PlazaMartínez2009}). Sin embargo, en nuestro ejemplo, se emplea como una partícula modal que indica el final del turno e incluye un valor interactivo de apelación. Una muestra del valor modal e interaccional de la partícula \textit{{}-qa} es la reacción inmediata de Leo que, por un lado, responde inmediatamente (valor interaccional) y, por otro, entiende el turno previo como acusación al incluir una breve risa, potencialmente comprensible como irónica (L22) (valor modal). La posposición de ambas partículas modales refuerza el carácter apelativo de las mismas, lo cual puede resultar sustitutivo de la ausencia del contacto visual.

En el ejemplo \REF{ex:satti:12} también encontramos estrategias lingüísticas con valor apelativo ante la ausencia de contacto visual. En este ejemplo, Isabel habla sobre la gran cantidad de tareas que tenía que hacer por su cuenta, entre ellas transportar carga (L178). Leo interrumpe la narración para agregar que también contaba con la ayuda de una mujer llamada Canda, lo cual es relativizado por Isabel en la línea 182.

\ea\label{ex:satti:12}
\begin{verbatim}
Candari („Return to Ucuchi“)
178 ISA: askhata  jap'ix     kani.
         mucho-AC agarrar-AG estar-1S
         solía llevar mucha carga

179      (0.6)

180      [(wawa cha')]
           niño
           el niño…

181 LEO: [candAri;    ] (0.6) 
          Canda-inch
          Y (qué pasó con) Canda?

182 ISA:canda ma unaytAchu           kawarqaykuqa        A,
        Canda no mucho tiempo-AC-NEG ser-2>1-TEST-1s-TOP mod.INT
        Canda no se quedó con nosotros mucho tiempo
\end{verbatim}
\z

En este ejemplo observamos dos recursos que detienen el desarrollo de la narración. El primero es la toma de turno superpuesta por parte de Leo, que provoca una pausa en el turno de Isabel y una apertura de la secuencia lateral en las líneas 180 y 181. El segundo recurso que incluye Leo en su turno es una construcción interrogativa con el morfema incoativo \textit{{}-ri}, localizada al final del turno (l. 181). Esta partícula adquiere también un carácter apelativo al ubicarse en posición final del turno, y un carácter modal a partir de su aspecto continuativo. Isabel también recurre a recursos modales interaccionales mediante la combinación de las partículas \textit{{}-qa} (mencionada previamente) y \textit{a}, variante de \textit{ari}, al final de su turno (L182). Estas partículas, que no pueden traducirse fácilmente al español, agregan a la información “canda no se quedó con nosotros mucho tiempo” una manifestación del desacuerdo con el contenido (\textit{{}-qa}) y una exigencia a Leo, en el sentido de que “debería saberlo” (\textit{a}), similar a las funciones que adopta \textit{pues} en el español andino (\citealt{SotoRodríguez2013}).

Además de los recursos analizados previamente, también hemos identificado el uso de otros dos morfemas en este contexto. Por un lado, la utilización del morfema concesivo \textit{{}-pis}, considerado un morfema de adición cuando forma parte de sintagmas nominales, que en nuestros ejemplos se utiliza como estrategia para integrar al interlocutor en la tarea de resolver el problema. Por otro lado, también registramos el uso del morfema verbal testimonial \textit{{}-rqa}, empleado para casos en los que una persona ha visto o participado de un evento y puede, en consecuencia, funcionar como un recurso aseverativo como se observa en la linea L182, ejemplo \REF {ex:satti:12}.

El análisis cualitativo de las instancias de interrupción en quechua nos permite suponer que esta lengua cuenta con información codificada en la gramática a partir de diferentes morfemas que permiten a los hablantes realizar tareas de tipo modal o apelativo. Esta posibilidad que tiene la lengua quechua de incluir este tipo de morfemas durante las interrupciones vuelve potencialmente redundante movilizar recursos visuoespaciales para realizar las mismas tareas, lo cual podría ser una razón posible para explicar las diferencias presentes en la frecuencia de contacto visual.

\subsection{Análisis cualitativo de las interrupciones en el español de Cochabamba}\label{sec:satti:3.5}

En nuestros datos en español de la región de Cochabamba también encontramos diferencias significativas con respecto al comportamiento de la mirada. Sin embargo, el español no cuenta con los mismos morfemas que se encuentran presentes en la lengua quechua, por lo cual surge la pregunta sobre qué estrategias utilizan los hablantes en este contexto.

En el ejemplo \REF{ex:satti:13}, Mara y Juana, hermanas gemelas, están contando a su tío, Tato, sobre una vez que viajaron solas a la ciudad de Llallagua a visitar a su abuela. En la línea 20, Mara, que es la narradora principal en este fragmento, se dirige a Juana para pedirle confirmación sobre si, de hecho, realizaron el viaje solas.  En las líneas 22 y 23, Juana confirma la información después de una pausa extensa, proyectando una respuesta despreferida, y seguidamente interrumpe la narración con un turno iniciado por \textit{pero} (L26), agregando que ya eran “grandes”, en el sentido de adultas. Si bien parece un detalle, el hecho de que ya eran adultas bloquea, por ejemplo, una historia de “aventuras” en donde dos niñas viajan solas por Bolivia. En cualquier caso, para ambas, este detalle es lo suficientemente importante para generar una secuencia lateral en donde se especifica la evaluación “solas”. La secuencia lateral se concluye en la línea 31 y a partir de la línea 32 Mara continúa con la historia.

\ea\label{ex:satti:13}
\begin{verbatim}
Hemos ido solas (ssot201702)
15 MAR: y en Eso (.) pedimos permiso aQUÍ::-
16      que: (.) la U::-
17      y algunas COsas,
18      °hh y <<h> FUImos,> (-)
19 JUA: mhm,
20 MAR: y: fuImos:: Solas;=<<p>no?>;
21 JUA: (0.65) sí:,=
22      =hemos ido SOlas. (.)
23 MAR: SO:las;
24      (ya [ni xxx)               ]
25 JUA:     [<<all> pero ya éramos>] (.) ya somos GR[ANdes.]
26 MAR:                                             [sí:.  ]
27 JUA: ((rie)) [((rie))  ]
28 MAR:         [<<:-)> ya] eramos GRANdes.>
29 JUA: ((rie))
30 MAR: y SÍ;=
31      =hemos ido BIE:N-
32      charLA:Ndo;
33      escuchAndo MÚ:sica;
\end{verbatim}
\z

En este ejemplo, resulta relevante para nuestros intereses el cambio que se produce en la confirmación de Juana en la línea 23. Juana confirma la información, pero, en lugar repetir simplemente las palabras de Mara (“fuimos solas”, L20), realiza un cambio de tiempo verbal del pretérito perfecto simple al pretérito perfecto compuesto (“hemos ido solas”). Este cambio, naturalmente, no resulta aleatorio. En el español andino, se ha demostrado que el pretérito perfecto simple se utiliza con función evidencial para marcar un acceso directo a la información (\citealt{Escobar1997};  \citealt{ÁlvarezGarriga2012}; \citealt{PfänderPalacios2013};  \citealt{GarcíaTesoro2017}), es decir, en este contexto, funciona como una confirmación con evidencia directa de que la información que provee Mara es correcta. De esta manera, la clarificación posterior sobre el hecho de que ya eran adultas no corre el riesgo de interpretarse como una corrección a los hechos de la historia. En este tipo de contexto, también podríamos esperar un tipo de confirmación evidencial a partir de expresiones faciales o movimientos de la cabeza, como se ha demostrado, por ejemplo, en hablantes de catalán \citep{RoseanoEtAl2015}. Sin embargo, Juana recurre a un recurso de tipo gramatical para realizar esta misma acción.

Considérese ahora el ejemplo \REF{ex:satti:14}. En este fragmento, Eva está contando a un amigo, Tato, sobre la discusión que tuvo con la directora del centro educativo en el que trabajaban tanto ella como su pareja, Carlos. En la línea 55, Carlos interrumpe brevemente para agregar que también les “han hecho cometer” ciertos errores, otorgando parte de la culpabilidad a los responsables del centro educativo.

\ea\label{ex:satti:14}
\begin{verbatim}
Nos han hecho cometer (ssot201708)
49 EVA: yo en el fOndo he DIcho=
50      =yo perDOno a la hermAna,=
51      =y buEno en su momEnto habrá hecho COsas todo;=no?
52      pero YA dije.
53 TAT: mhm;
54 EVA: pero no los hemos baTIdo a la [vida porque       ]
55 CAR:                               [nos han hecho come]TER   
        también;=no?
56 EVA: sí:. (0.5)
57      no los hemos batIdo al final la VIda: entre los dO:s-
\end{verbatim}
\z

La interrupción de Carlos incluye la integración de la partícula interrogativa \textit{no} al final del turno, un fenómeno frecuente en nuestros datos de la región de Cochabamba.\footnote{Un estudio sobre la integración de este tipo de partículas en la región de Cochabamba (aplicando la misma metodología del presente estudio) se encuentra en preparación. Los primeros resultados fueron presentados el 2019-11-27 en el III simposio internacional “Desafíos en la diversidad”, en la ciudad de Quito.}  La integración de este tipo de partículas a final de turno parece adquirir funciones apelativas que en otras regiones son realizadas con la mirada. Si bien no podemos asegurar que se trate de un fenómeno de contacto, la integración y reducción de elementos léxicos al final de los turnos de habla parece ser recurrente en esta zona (véase \citealt{SotoRodríguez2013} para el caso de \textit{ps} y \textit{pues}), lo cual podría considerarse como un efecto del contacto con una lengua aglutinante. En cualquier caso, observamos que los hablantes de español de la región de Cochabamba también cuentan con estrategias lingüísticas para realizar tareas conversacionales de tipo apelativo o modal.

\section{Limitaciones y futuras investigaciones}\label{sec:satti:4}

Es necesario remarcar que el tamaño de la muestra tiene limitaciones, en especial si se pretende tener en cuenta variables como, por ejemplo, la edad o el género de los participantes. Este carácter reducido se justifica en el tiempo que conlleva la anotación y la obtención de los datos. En futuras investigaciones ampliaremos la muestra. Otro aspecto que requiere mención es el hecho de que nos hemos concentrado en los recursos léxicos, gramaticales y visuoespaciales, pero no hemos tenido en cuenta detalladamente el nivel prosódico. Los recursos prosódicos podrían ser también fundamentales en el momento de la interrupción y creemos que, en la zona andina, al utilizar con menos frecuencia los recursos visuales, el nivel prosódico puede tener un rol importante y será tenido en cuenta en trabajos futuros. Por último, las diferencias generales en la frecuencia de contacto visual durante las narraciones permiten suponer que los resultados de este estudio sobre las interrupciones podrían replicarse en otros contextos secuenciales y en otro tipo de datos. Estudios de otras prácticas en la narración, como los pedidos de confirmación y las búsquedas de palabra, se encuentran en preparación e intentarán comprobar los resultados de este trabajo en otros contextos.

\section{Resultados y discusión}\label{sec:satti:5}

Un acercamiento multimodal a los datos ha revelado diferencias considerables con respecto al comportamiento de la mirada durante las narraciones, particularmente en la región de Cochabamba. Estos resultados, en sí mismos, pueden ser de interés para distintas disciplinas. En nuestro estudio, observamos que, en las regiones de Buenos Aires, Bogotá y Friburgo, los participantes recurren con frecuencia a expresiones faciales y a gestos manuales durante las interrupciones. Esto explica la importancia de establecer contacto visual en este contexto, lo cual se produce en más del 80\% de los casos. En cambio, en la región de Cochabamba, los participantes favorecen el uso de recursos lingüísticos para realizar tareas conversacionales que en el resto de nuestros datos se realizan con la mirada o con la gestualidad. Particularmente, en el quechua encontramos información de tipo modal o interaccional que se encuentra codificada en la gramática a partir de diferentes morfemas en posición final de turno, como, por ejemplo, \textit{{}-qa}, \textit{{}-a/-ari}, \textit{{}-pis}, \textit{{}-rqa}, entre otros, y partir del uso posposicional de \textit{pero}. En el caso del español de esta zona, que no cuenta con este tipo de morfemas, observamos que los participantes también recurren a estrategias lingüísticas, como la integración de partículas interrogativas al final del turno de la interrupción (\textit{¿no?} o \textit{¿no ve?}) o la marcación de evidencialidad a partir de la oposición entre pretérito perfecto simple y pretérito perfecto compuesto. Algunas de estas estrategias también revelan la influencia del contacto lingüístico, tanto en los hablantes de español como en los hablantes de quechua, lo cual confirma lo que han revelado estudios previos que se han concentrado en estos fenómenos, como los señalados anteriormente.

Nuestro trabajo ha intentado abordar el estudio de distintas variedades a partir de un acercamiento multimodal a las prácticas comunicativas de los hablantes. Esta perspectiva nos permite preguntarnos si las diferencias en los códigos socioculturales de convivencia, como podría ser el caso de las diferencias en el comportamiento de la mirada, pueden tener una influencia en la utilización de recursos lingüísticos que hacen los participantes para realizar tareas conversacionales o, viceversa, si la estructura gramatical puede tener una influencia en la movilización de recursos visuoespaciales, como la mirada o la gestualidad. Naturalmente, una respuesta definitiva a estas preguntas requiere una investigación más amplia. Sin embargo, esperamos haber mostrado la relevancia de tener en cuenta la interrelación de recursos verbales y no verbales para analizar las prácticas lingüísticas de los hablantes.

\section*{Abreviaciones}
\begin{tabularx}{.45\textwidth}{lQ}
1 &primera~persona                                 \\
2>1 &       direccional de segunda a primera persona      \\
\textsc{abl}    &ablativo                               \\
\textsc{com}    &comitativo                             \\
\textsc{cont} ​​&    continuativo                       \\
\textsc{dir}    &direccional                            \\
\textsc{ger}    & gerundio                              \\
\end{tabularx}
\begin{tabularx}{.45\textwidth}{lQ}
\textsc{inc}    &inceptivo                              \\
\textsc{mod.int}&  modal~interaccional                  \\
\textsc{part}  ​&participio                          \\
\textsc{pl​​}   & plural                                \\
\textsc{inch}   & incoativo                            \\
\textsc{s}​​    &singular                               \\
\textsc{test​} &  ​préterito~testimonial                \\
\textsc{trans} &  translocativo                         \\
\end{tabularx}

\sloppy\printbibliography[heading=subbibliography,notkeyword=this]
\end{document}
