\documentclass[output=paper]{../langscibook}
\ChapterDOI{10.5281/zenodo.5643293}
\author{Aldo Olate Vinet\orcid{0000-0002-6926-769X}\affiliation{Universidad de La Frontera}}
\title{Contacto interétnico y experiencias sociolingüísticas: exploración intergeneracional en bilingües mapuzugun/castellano}
\abstract{A partir de la idea de que el contacto lingüístico es un fenómeno complejo que involucra a hablantes y escenarios, abordamos el problema de las experiencias sociolingüísticas de bilingües en el contacto interétnico. A través de entrevistas, se indagó en las narrativas de 20 hablantes mapuches bilingües pertenecientes a dos territorios de la región de la Araucanía, Chile, luego se constituyó una caracterización de los elementos comunes de dichos relatos experienciales. Los resultados indican que las vivencias de la niñez en torno al contacto y la instrucción formal promueven comportamientos, actitudes sociolingüísticas y representaciones tendientes al desplazamiento del mapuzugun, afectando la identidad lingüística de sus hablantes y los procesos de transmisión.

This research views linguistic contact as a complex phenomenon that involves speakers and scenarios. From this perspective, we approach the issue of sociolinguistic experiences of bilingual speakers in interethnic contact. Through the analysis of the narratives of twenty bilingual Spanish-Mapudungun speakers from the Araucanía region, Chile, we built a characterization of the common elements from these narrations. The results show that the speakers' childhood experiences with regard to intercultural contact and formal instruction promote sociolinguistic attitudes, behaviors and social representations that lead to displacement of Mapudungun, which affects the linguistic identity of speakers and language transmission.
% \keywords{Lenguas en Contacto, Experiencias sociolingüísticas, mapuzugun, Narrativas, Contacto interétnico}
% \keywords{Languages in contact, Sociolinguistics experiences,} mapuzugun\textup{, Narratives, inter-ethnical contact}
}

\IfFileExists{../localcommands.tex}{
  \addbibresource{../localbibliography.bib}
  \input{../localpackages}
  %Copy this to localcommands.tex

\usepackage[english]{babel}
\usepackage{amsmath}
\usepackage{amssymb,amsfonts,textcomp}
\usepackage{array}
\usepackage{hhline}
\usepackage{hyperref}

\newenvironment{styleStandard}{}{}
\newenvironment{stylelsAbstract}{}{}
\newenvironment{stylelsSectioni}{}{}
\newenvironment{stylelsSectionii}{}{}
\newenvironment{stylelsBulletList}{}{}
\newenvironment{styleBibliographyi}{}{}
\newenvironment{listWWNumxxvleveli}{}{}
\newenvironment{listWWNumxxvlevelii}{}{}
\newenvironment{listWWNumxxvleveliii}{}{}
\newenvironment{listWWNumxxvleveliv}{}{}
\newenvironment{listWWNumixleveli}{}{}
\newenvironment{listWWNumixlevelii}{}{}
\newenvironment{listWWNumixleveliii}{}{}
\newenvironment{listWWNumixleveliv}{}{}

\newcommand\textstyleListLabelxvi[1]{#1}
\newcommand\labellistWWNumxxvleveli{\thelistWWNumxxvleveli.}
\newcommand\labellistWWNumxxvlevelii{\thelistWWNumxxvlevelii.}
\newcommand\labellistWWNumxxvleveliii{\thelistWWNumxxvleveliii.}
\newcommand\labellistWWNumxxvleveliv{\thelistWWNumxxvleveliv.}
\newcommand\labellistWWNumixleveli{[F0B7?]}
\newcommand\labellistWWNumixlevelii{\textstyleListLabelxvi{o}}
\newcommand\labellistWWNumixleveliii{[F0A7?]}
\newcommand\labellistWWNumixleveliv{[F0B7?]}

\newcounter{listWWNumxxvleveli}
\newcounter{listWWNumxxvlevelii}[listWWNumxxvleveli]
\newcounter{listWWNumxxvleveliii}[listWWNumxxvlevelii]
\newcounter{listWWNumxxvleveliv}[listWWNumxxvleveliii]
\newcounter{itemize} 
  \input{../localhyphenation}
  \togglepaper[1]%%chapternumber
}{}
\shorttitlerunninghead{Contacto interétnico y experiencias sociolingüísticas}
\begin{document}
\maketitle 
\shorttitlerunninghead{Contacto interétnico y experiencias sociolingüísticas}
\clearpage




\section{Introducción}\label{sec:olate:1}


Las experiencias de vida y los episodios narrados por los hablantes se constituyen en valiosas fuentes que permiten reconstruir los escenarios del contacto interétnico desde una perspectiva intergeneracional. Los testimonios dejan ver las relaciones asimétricas entre las sociedades que entran en contacto y a partir de ellos pueden proyectarse prácticas, comportamientos, actitudes y representaciones a través de las cuales los hablantes han perfilado su comportamiento sociolingüístico. 

El recorrido político-histórico de la relación entre el \textit{mapuzugun} y el castellano ha implicado la transformación de los modos de vida de la sociedad mapuche, transformación que se ha visto más acelerada en el período posterior al proceso de Ocupación de la Araucanía. Junto con este proceso, la naciente sociedad chilena ha forjado representaciones de los minorizados, las cuales han sido canalizadas históricamente por diversos medios: educación formal, colonización, evangelización, circuitos de comunicación cotidianos. 

Visto desde una perspectiva histórica, las experiencias de vida narradas por los hablantes reflejan las vivencias que han tenido durante distintas etapas de su vida, evidenciando los factores que han condicionado el uso y desarrollo del idioma materno en el contexto del contacto interétnico. 

En este trabajo, nos proponemos explorar las experiencias sociolingüísticas que tuvieron en común hablantes de la lengua mapuche de dos territorios distintos y de generaciones diferentes (1910-1970) con el propósito de rescatar experiencias y prácticas comunes que nos permitan retratar la violencia colonial ejercida sobre el idioma y sus hablantes, así como sus efectos sobre el comportamiento sociolingüístico de los actores. 

Con este propósito, el presente trabajo se organiza en las siguientes secciones. En la sección \ref{sec:olate:2}, se detalla, brevemente, el contacto político-histórico entre la sociedad chilena y la mapuche;
en §\ref{sec:olate:3}, presentamos algunas ideas vinculadas con el contacto interétnico.
La investigación se presenta en §\ref{sec:olate:4};
luego, en §\ref{sec:olate:5}, se muestran los resultados;
en §\ref{sec:olate:6} se genera una discusión a partir de los resultados y las conclusiones se presentan en §\ref{sec:olate:7}.


\section{Plano político-histórico del contacto entre el \textit{mapuzugun} y el español}\label{sec:olate:2}


Este no es el espacio para detallar pormenorizadamente el plano en cuestión, baste comentar aquí algunos aspectos que permiten entender la relación asimétrica de contacto político-histórico desarrollada entre la sociedad chilena y la mapuche. 

El contacto lingüístico entre el castellano y el \textit{mapuzugun} se caracteriza tanto por su extensión temporal como por su intensidad. Ambas lenguas han estado en interacción por más de cuatrocientos años, situación que se acrecentó con la dramática anexión territorial que sufrió la sociedad mapuche desde 1860 y que redundó en un contacto multiétnico seguido de un proceso intenso de asimilación lingüístico-cultural (\citealt{PintoRodríguez2000}).

Durante este periodo de contacto intenso, como consecuencia de la introducción de diversos mecanismos colonizadores del Estado que transformaron la forma de vida tradicional mapuche, emergieron distintas formas de relaciones interétnicas condicionadas por la situación asimétrica del contacto político-cultural (\citealt{OlateVinet2017a,OlateVinet2017b}). En esta situación, hablantes de una lengua sociopolíticamente dominada debieron desarrollar competencias interculturales y comunicativas para desenvolverse en los escenarios de contacto impuestos por la sociedad chilena. Esta dinámica produjo un sinnúmero de modificaciones, no solo vinculadas a las estructuras lingüísticas (dominante y dominada) sino también en el plano de la comunicación interétnica e interpersonal (\citealt{Johanson2002}; \citealt{OlateVinet2017a}).

El contacto político-histórico implicó la incorporación obligada del castellano, surgiendo la condición de bilingüismo con fines comunicativos y de supervivencia (\citealt{Bengoa2000,Bengoa2007,Bengoa2014}). El bilingüismo, tanto social como individual, no ha sido homogéneo y ha estado determinado, en el plano interétnico, por las experiencias sociolingüísticas de los hablantes, su grado de “asimilación” a la sociedad dominante\footnote{No haremos referencia a procesos de modernización, industrialización y mediatización como recursos para la integración/asimilación por parte de los proyectos nacionales. Sin embargo, juegan un papel preponderante en el plano del contacto político-histórico actual, por lo que merecen un análisis más profundo y detenido.} y la identidad lingüística que implica el manejo del \textit{mapuzugun}. Considerando estos parámetros, habría un continuo de hablantes bilingües que va desde los más competentes lingüísticamente a los menos funcionales. Ambos extremos del continuo tienen en común la adquisición del castellano para poder desempeñarse en el mundo del colonizador.


\largerpage
Desde la perspectiva de la antropología lingüística, esta situación de contacto reviste la tensión entre fuerzas centrípetas y centrífugas del lenguaje. Visto así, la institucionalidad chilena impuso el español sobre el \textit{mapuzugun}, obligando a sus hablantes a adoptar el código homogéneo/nacional; sin embargo, la violencia colonial y la precarización de los modos de vida relegaron a la sociedad mapuche a la periferia del sistema social, provocando el surgimiento de variedades de habla alternativas y la conservación de circuitos comunicativos en \textit{mapuzugun}, factores que pueden transformarse en estrategias para la conservación de la identidad étnica o para la emergencia de la misma \citep{Duranti2000}.   

\citet{OlateVinetWittigGonzález2019} proponen diversos períodos significativos del contacto, los cuales se detallan a continuación.

El primer momento que destacan estos autores es el de la Conquista, escenario donde se produce el contacto inicial con espacios de interacción tales como: la guerra, los intérpretes, el cautiverio y los parlamentos (\citealt{DuránRamos1986}). Se configuran nuevos ámbitos de uso a partir de las instituciones culturales introducidas. Los fenómenos de bilingüismo y contacto están bien delimitados y circunscritos a grupos sociales específicos (\citealt{OlateVinetWittigGonzález2019}).

El segundo período sociolingüístico es la Frontera, que implicó la reconfiguración del territorio mapuche donde se desarrollaron nuevos procesos (comercio y misiones). El comercio tuvo una dinámica intra-territorial, regional y extra-regional que fortaleció las relaciones fronterizas y la evangelización, consolidando la convivencia entre mapuches, españoles y criollos (\citealt{PintoRodríguez2000}).

El tercer momento es la Ocupación Militar de la Araucanía (1860). La campaña militar acorrala a la sociedad mapuche en su propio territorio, derrotándola, lo cual trae consecuencias devastadoras: radicación territorial, nuevo régimen civil, colonos chilenos y extranjeros son los elementos que cambian las formas de vida de la sociedad mapuche. Se instala aquí la fuerza centrípeta que transforma al hablante, imponiéndole el castellano y el sistema ideológico del proyecto Estado-nación. Se activa el escenario de precarización y marginalización social junto con la retracción de la lengua a ámbitos intraétnicos (\citealt{OlateVinetWittigGonzález2019}).

El cuarto período es el de la Reducción. Esta etapa consolida el proyecto nacional con la ocupación de los territorios, la fundación de pueblos y ciudades. Se inicia así un intenso proceso de asimilación que tiene en paralelo otros factores que lo acentúan: vías de comunicación, producción agroindustrial, transformación del mapuche en campesino, migración (\citealt{ComunidaddeHistoriaMapuche2015}). En el contacto lingüístico, este momento implica la reducción de los contextos de uso del \textit{mapuzugun}, su exclusión de ámbitos interétnicos y la extensión del español en los circuitos comunicativos intracomunitarios. En síntesis, se da un ingreso masivo del español en la vida cotidiana mapuche.\footnote{Diversos estudios sociolingüísticos (\citealt{DuránRamos1998}; \citealt{LagosEspinoza2013}) ven en los procesos de formalización educativa un factor preponderante a la hora de discutir la situación de desplazamiento de la lengua. Lo anterior debe complementarse con las situaciones de educación no formal entendidas como espacios civilizatorios (\citealt{NahuelpánMoreno2012}) que se dan en paralelo al surgimiento de latifundios y el consiguiente intento por transformar la base social mapuche en un componente más del campesinado chileno.}

Finalmente, el momento actual implica diversos procesos que manifiestan las tensiones identitarias entre la modernidad y la tradición tanto colectiva como individualmente. Esta etapa se destaca por la reivindicación de una identidad mapuche urbana, la emergencia de intelectuales, creadores y comunicadores indígenas que reflejan una sociedad vigente en el marco de las relaciones interculturales. En la situación de contacto, este se observa en las zonas rurales de la región de La Araucanía y, a escala social, se da entre hablantes bilingües que transitan entre el campo y la ciudad (\citealt{WittigGonzálezOlateVinet2016}). El desplazamiento del \textit{mapuzugun} ha generado una respuesta activa de parte de grupos tradicionales y emergentes de la sociedad mapuche a partir de iniciativas de aprendizaje formal de la lengua, el uso de la lengua en nuevos soportes de comunicación y la creación de nuevas modalidades discursivas (\citealt{OlateVinetWittigGonzález2019}).


\section{Contacto interétnico}\label{sec:olate:3}


El contacto político-histórico descrito implica el contacto interétnico, el cual se entiende como la relación -- mediada política e históricamente -- entre individuos y sociedades que presentan tradiciones culturales e identidades distintas. La relación interétnica es dinámica y puede explorarse a través de cuestiones vinculadas con la memoria, los modos de vida, formas de socialización, la organización político--territorial, las creencias y los sistemas de comunicación. Estos componentes se vinculan con lo que se reconoce como sociolingüística histórica (\citealt{HernándezCampoyCondeSilvestre2012};  \citealt{McCollMillar2012}; \citealt{Crow2015}).

En este plano, el contacto problematizará la condición étnica e identitaria de los hablantes colonizados, la cual estará constantemente tensionada por los dispositivos “civilizatorios” o las fuerzas centrípetas de la sociedad colonizadora (\citealt{NahuelpánMoreno2012}). El hablante, sujeto que vive estas tensiones, estará inserto en un continuo que tiene en un extremo la conservación identitaria y en el otro la integración nacional (asimilación).\footnote{Este continuo incluye varios puntos que se vinculan con la identidad étnica y la identidad lingüística, en el extremo del orgullo étnico encontramos el orgullo por la lengua, también hay casos de re-identificación y emergencia, lo cual puede interpretarse, en el plano de la relación lengua e identidad, como procesos de re-activación de la lengua y la constitución de neo-hablante. Por su parte, en el polo tendiente hacia la integración nacional, tenemos procesos de invisibilización identitaria y lingüística, también de difusión de la identidad nacional y, finalmente, de cambio identitario y lingüístico. El movimiento hacia un lado u otro de la gradiente está directamente relacionado con las experiencias que tiene el hablante con los dispositivos de la sociedad colonizadora. Así, por ejemplo, la tendencia hacia el orgullo se vincula, fundamentalmente, con procesos de mantención lingüística y sentimientos de pertenencia étnica reforzados por redes sociales. Por su parte, la tendencia hacia la integración está vinculada a eventos de violencia étnica, indiferencia y desplazamiento lingüístico, cuyo agente primordial es el sujeto hegemónico.} El tránsito del hablante es oscilante, estará en función de sus experiencias en la sociedad hegemónica y será sensible a la representación que construye el sujeto hegemónico sobre el colonizado. Esta imagen articulada, tanto en el discurso como en la acción, tiene historicidad y generará consecuencias en la autorepresentación del hablante, sus creencias, valores e identidad. En otra línea, el comportamiento de los integrantes de la sociedad colonizadora (chilena) es un elemento que influye en el minorizado, así, situaciones de violencia/estigmatización o indiferencia étnico-colonial influyen sobre lealtades identitarias, interpersonales y comunicativas del violentado.


\section{El estudio}\label{sec:olate:4}


Este estudio se propone describir las experiencias de hablantes bilingües \textit{mapuzugun}{}-español con el propósito de evidenciar las vivencias y prácticas experimentadas por distintas generaciones en el marco de estereotipos y momentos significativos del contacto interétnico y político-histórico.

Las narrativas de experiencia personal pueden considerarse un género del discurso que refleja modos de conocimiento y actividades sociales. Tienen su propia organización interna que transforma las experiencias personales en procesos cognitivos complejos, implicando rememoraciones, situaciones, anticipaciones, evaluaciones y representaciones vinculadas a los episodios de vida \citep{Ochs1996}. El foco en las narrativas se desprende de la idea de que por medio de las historias de vida puede comprenderse la manera en que cada persona mueve sus conocimientos y sus valores. A través de ellas, se toma conciencia de los fenómenos sociales, son formas de expresión de la experiencia humana a cargo de un narrador, quien es actor y agente a la vez y, como tal, participa en, e interpreta, las experiencias significativas que han marcado sus trayectorias o recorridos (\citealt{JovchelovitchBauer2005}; \citealt{BolognaniNacarato2015}; \citealt{Bertaux2010}; \citealt{Delory-Momberger2015}).

En las narrativas se expresan las voces de los actores sociales, lo cual permite recuperar historias singulares narradas por sujetos históricos socioculturalmente situados con roles en la construcción de la historia individual y colectiva (\citealt{DeSouza2010}; \citealt{Duranti2009}).

El acceso a algunos episodios de las historias individuales de los hablantes permite observar: la historia sociolingüística de la lengua en el territorio, la estigmatización y los sistemas de creencias asociados, las actitudes sociolingüísticas, los eventos comunicativos, los lugares, las valoraciones sobre la lengua, la enseñanza y los procesos de socialización en torno a la lengua, entre otros. Las prácticas explicitadas en las experiencias permiten, por lo tanto, comprender los contextos sociales donde emergen, se reproducen y se transforman los hablantes y su idioma.

Las experiencias permiten internarse en las trayectorias de los hablantes, lo cual sirve para relevar los procesos de memoria de estos. Por otro lado, se instalan relatos que movilizan tanto las representaciones como las condiciones del contacto interétnico, explicitando creencias y comportamientos tendientes al uso, la transmisión, la pérdida o el mantenimiento de la lengua (\citealt{WittigGonzálezFaríasCaballero2018}).  Finalmente, también son instancias que rememoran escenarios de violencia colonial que pueden condicionar el apego o desapego a la lengua materna de manera individual y colectiva (\citealt{ComunidaddeHistoriaMapuche2012,ComunidaddeHistoriaMapuche2015}).

Esta investigación se basa en el trabajo en terreno, lo cual permite la obtención de narrativas en el propio campo. Entre los criterios de selección de la muestra se consideraron dos comunidades ubicadas en territorios distintos y cercanos al centro urbano de la región, Temuco. Esta decisión se tomó a partir de la idea de que las experiencias vividas por los hablantes tienen elementos en común, independientes del factor geográfico y de la dinámica de la comunidad, pues los dispositivos impuestos por el Estado-nación se difundieron por todo el territorio anexado.

En total, la muestra se compone de 20 entrevistas\footnote{Agradecemos la disposición y colaboración de cada uno de los entrevistados, sin su ayuda no habría sido posible generar esta reflexión y tomar conciencia de los procesos históricos que acarrea el contacto entre la sociedad chilena y la mapuche.} realizadas a hablantes bilingües \textit{mapuzugun}{}-español adscritos a los territorios de \textit{Makewe} y \textit{Txanantue} (Región de La Araucanía, Chile). El territorio de \textit{Makewe} se encuentra a 12 km al suroeste de la ciudad de Temuco, en el territorio existen varias comunidades mapuches, por lo que el sector tiene alta densidad de población mapuche rural. \textit{Txanantue}, por su lado, se halla a 16 km al noroeste de Temuco. Cada territorio tuvo 10 participantes en la entrevista. 

Las características comunes de los entrevistados se relacionan con las actividades laborales, nivel de escolaridad y bilingüismo. En cuanto al primer rasgo, los hablantes son sabios conocedores de la cultura, autoridades políticas de la comunidad, agricultores y artesanos de pequeña escala. La gran mayoría de los entrevistados tiene enseñanza básica/primaria completa o incompleta y casi todos tienen como L1 el \textit{mapuzugun}. Otros elementos que deben destacarse de estos actores tienen relación con diversas funciones políticas que cumplen y sus preferencias religiosas. De acuerdo con estas coordenadas, el conjunto de entrevistados comprende entre otras funciones: \textit{logko} (autoridad político-cultural de la comunidad), conocedores culturales, autoridades socio-espirituales (conocimiento de plantas para sanar enfermedades), entre otros. En el plano religioso, nos encontramos con varios actores que se adscriben a la religión evangélica, a la católica y otros a las creencias espirituales mapuches.

El instrumento utilizado para obtener los relatos fue la entrevista semiestructurada, aplicada por el equipo de investigación en los respectivos hogares de los hablantes. La entrevista se organizó de la siguiente manera: descripción de actividades, argumentaciones sobre temas controversiales, episodios de vida de niños y adultos y actividades/ceremonias tradicionales. La duración de estas fluctuó de 60 a 90 minutos. Las entrevistas no son estrictamente relatos autobiográficos, hay algunas secciones que reflejan algunos episodios de vida o eventos \citep{Ochs2006} que han tenido los participantes como usuarios de las lenguas, pero no se trata en sentido estricto de una narración autobiográfica.

En cuanto al vínculo para realizar las entrevistas, dentro del equipo de investigación hubo un entrevistador que pertenecía al territorio de \textit{Txanantue} que se desempeña como profesor en un colegio del sector de \textit{Makewe}. Este miembro del equipo es mapuche e hizo los contactos con las personas para realizar la actividad.

El procedimiento de análisis en el presente estudio se basó en la revisión e interpretación de 20 entrevistas realizadas a hablantes bilingües \textit{mapuzugun}{}-español. El foco se centra en algunos episodios que vivieron como hablantes de \textit{mapuzugun} en los escenarios hegemónicos del contacto interétnico y político-histórico a los que tuvieron acceso. A partir de allí, estas experiencias se vinculan con los períodos significativos del contacto reseñados en §\ref{sec:olate:2}, relevándose en la interpretación, la imagen de la lengua que vivió el hablante y sus consecuencias para la identidad lingüística, las actitudes sociolingüísticas y la transmisión.

De este modo, se consideraron los elementos coincidentes que se reflejan en los testimonios de cada grupo. Por otro lado, la narrativa se vinculará con los períodos significativos del contacto (\citealt{OlateVinetWittigGonzález2019}). Las categorías de análisis, por tanto, no son \textit{a priori}, sino emergentes y conducidas por los propios relatos de los hablantes.

En lo que atañe al contexto sociopolítico, cabe señalar que los actores están insertos en un momento del contacto, específicamente, en el período activado después de la Ocupación de la Araucanía, momento en que se observa el proceso de reducción de tierras, la precarización de la sociedad y la transformación en campesinos. Además, se proponen los períodos históricos en que nacen los hablantes, estimando un lapso temporal de 15 años para conformar las generaciones. Así, el grupo 1 tiene integrantes que nacieron entre 1910 y 1925; el grupo 2 entre 1926 y 1940; el grupo 3 entre 1941 y 1955; el grupo 4 entre 1956 y 1970 y el grupo 5 entre 1971 y 1985. Este último grupo no será analizado en este estudio por presentar solo un participante.

Esta manera de configurar la muestra nos permite relacionar los períodos históricos y las experiencias de vida de los hablantes en el marco del contacto político-histórico e interétnico. 


\section{Las experiencias con las lenguas en contacto a través de las generaciones}\label{sec:olate:5}


\subsection{Experiencias del grupo 1 (1910--1925)}



Los actores de este grupo presentan un perfil de hablantes preferentes de \textit{mapuzugun}. Son ancianos que nacieron en la década de 1910, sus edades fluctuaban entre 90 y 95 años. Dos de ellos fueron estudiantes de educación primaria, pero no completaron sus estudios.\footnote{Válido es señalar que en 1920 se promulga la Ley 3.654 de Instrucción Primaria Obligatoria, la cual establece que la Educación primaria dada bajo dirección del estado y municipalidades será gratis. Esta ley estableció que los padres estaban obligados a enviar a sus hijos, al menos cuatro años, a un establecimiento de educación primaria. Además de lo anterior, se crearon las juntas de educación en cada comuna, cuya misión fue la de realizar censos escolares para controlar el cumplimiento de la ley.} Uno no fue nunca a la escuela. La actividad principal de subsistencia fue la agricultura y el pastoreo. En el contexto del hogar hablan exclusivamente en el idioma mapuche y su vida se desarrolló principalmente en la comunidad.

Los contextos de aprendizaje del español fueron principalmente la escuela, las misiones y a través del contacto con la sociedad chilena en diferentes instituciones. El período de vida de estos hablantes se inicia 20 años después de la incorporación del territorio mapuche al proyecto nacional chileno. Esta reciente anexión muy probablemente implicó que la lengua mapuche se mantuviera como vehículo comunicativo y que el español se fuera incorporando paulatinamente en el territorio a través de las instituciones culturales promovidas por el Estado de Chile. Las más relevantes son las siguientes: instituciones gubernamentales, colonos y fundos, evangelización a través de misiones católicas y evangélicas y escuelas. Desde este período, y en adelante, se inicia un intenso proceso de asimilación provocado por el desarrollo de vías de comunicación, la producción agroindustrial, procesos migratorios y la transformación del mapuche en campesino, cuestiones que modifican diversas costumbres de la sociedad minorizada (\citealt{ComunidaddeHistoriaMapuche2012,ComunidaddeHistoriaMapuche2015}).

En términos generales, dado el reciente proceso de anexión territorial, este grupo generacional muestra su plena identificación con la lengua. Esto puede observarse en \REF{ex:olate:1}, donde se presenta una interacción con un hablante de \textit{Txanantue} (91 años):

\ea\label{ex:olate:1}
 \textsc{Entrevistador 1}: ehh abuelito, ¿usted habla mapuzugun? ¿usted habla mapuzugun?\\
 \textsc{Entrevistador 2:} (pregunta en mapuzugun), pero conteste, mai diga puh\\
 \textsc{Entrevistado:} mai mai\\
 \textsc{Entrevistador 1:} ya, ya, en wigkazugun\footnote{{\emph{wigkazungun}} ({\textit{wigka}}: extranjero/colono/chileno; {zugun}: hablar): el habla del chileno/colono (castellano).}  nomah.\\
 \textsc{Entrevistado:} soy mapuche tengo que hablar en mapuche \textup{[Corpus propio]}\\
 \z

Por otro lado, en aquellos tiempos, se observa que el drástico cambio en los modos de vida y el contacto con el español afectaba a esta generación, pues los testimonios señalan que “los padres no sabían hablar en chileno” (\textit{Makewe}, 91 años, Grupo 1). Se infiere, entonces, que el contexto de desarrollo del grupo 1 implicó la interacción frecuente en \textit{mapuzugun} en la comunidad. Estos actores vivieron la experiencia directa del proceso de reducción y la consolidación del proyecto nacional a través de la apropiación y colonización de los territorios.

Esta misma generación, en sus relatos, da cuenta de la importancia que adquirió el castellano para las generaciones posteriores, pues el manejo de este posibilitaba el acceso al estudio. El siguiente relato \REF{ex:olate:2}, eso sí, muestra cómo el contacto político-histórico modifica la valorización de las lenguas, actualiza las narrativas y moviliza nuevos estereotipos:

\ea\label{ex:olate:2}
 Hay mapuche siguen el estudio, sacan buen profesión, tengo una, una nieta, en Santiago que, eh abogá, estudió lo que hace po…..y el otro, ah, Marco Antonio se llama él, trabaja de mecánico, arreglan autoh. ese tiene pega en el Temuco, costumbrao a hablar en \textit{wigka} \textup{[Grupo 1, Makewe, 93 años, corpus propio]}\\
\z

Hay que observar el contraste entre el testimonio que afirma que debe hablarse mapuche por ser mapuche y este, el cual concede al proceso de formación profesional una importancia vital para poder desempeñarse en la sociedad chilena. Este último testimonio es expresado desde un momento que podríamos clasificar como más cercano a la realidad presente; el anterior, por otro lado, se observa más cercano a la realidad pasada, de juventud, que se vivió en la comunidad. 

En la siguiente interacción \REF{ex:olate:3}, se observan varios elementos útiles para el análisis del contexto de desarrollo de los niños. El relato es dado por un hablante de \textit{Makewe} (93 años):


\ea\label{ex:olate:3}
 \emph{Entrevistador}: antes andaban con chamal los niñitos y niñitas. Chiripa\\
 \emph{Entrevistado}: Y no sabía hablar castellano ¡Qué sabe! Usted tiene hermanos, ¿cree que sabe hablar? De viejo no sabe hablar castellano todavía. Aquí cuando llegaron los gringos ellos los vinieron a enseñar, unos gringos que llegaron.\\
 \emph{Entrevistador}: ¿unos que llegaron al hospital?\\
 \emph{Entrevistado}: Sí, a la misión. Feita (aquí), puro mapuche mapun. Llegaron los \textit{wigka} (chilenos/colonos) aquí. Puro mapuche, puro mapuche. Y llegaron los \textit{wigka} (chilenos/colonos) quizás de distintas partes, akuyi pu wigka (llegaron los chilenos/colonos), nagpai mapu egün (bajaron la tierra). Küme zugu nagpai pu fütakeche (buenas noticias bajaron para los mayores) y cedieron puh. Feita, aquí hay como no sé cuánto \textit{wigka} (chilenos/colonos) a veces por este poquito. Diez hectáreas total parece. Hay harto \textit{wigka} que compraron, otros les agarraron no más.\footnote{Entre los años 1925 y 1930 la usurpación de parte de colonos, latifundistas y campesinos chilenos fue amparada por juicios irregulares donde se aplicaron dispositivos legales, permitiendo la división y venta de tierras dentro de las comunidades (Imilan \& Álvarez 2007; \citealt{Bengoa2000}).} ¿Qué van a comprar? Ofrecieron algo y los viejos cedieron. \textup{[Grupo 1, Makewe, 93 años,} c\textup{orpus propio]}\\
\z

En \REF{ex:olate:3}, se muestran varios elementos. En primer lugar, se presenta una imagen de la niñez mapuche previa a la llegada de colonos. En segundo lugar, se explicita que en esos tiempos no se sabía hablar castellano. En tercer lugar, se observa la introducción de extranjeros que vinieron con el propósito de enseñar, vemos la introducción de una institución “civilizadora”, la misión anglicana que llegó al sector. Finalmente, se rememora la llegada de los colonos chilenos y extranjeros, desde distintos lugares, y el acceso de estos a tierras, la gran mayoría tomadas (robadas) y adquiridas mediante engaños. La escena descrita tiene varios componentes que pueden resumirse en una noción: el contacto interétnico a través de la invasión.

Se observa que el contacto intercultural y político-histórico durante esta etapa implicó la incorporación del castellano a partir de diversas instituciones: misiones, colonización; lo cual sentará las bases para la retracción del \textit{mapuzugun} y su exclusión de los nuevos escenarios de contacto. La misión, por ejemplo, se ve como un espacio de formación tendiente a “civilizar” al mapuche a partir de la enseñanza del castellano y de distintos oficios {\citet{OlateVinet2017}}



\subsection{Experiencias del grupo 2 (1925––1940)}



El perfil de los participantes de este grupo es distinto al anterior por las funciones sociales que tienen en la comunidad. Estos actores son reconocidos por su conocimiento cultural y por tener funciones políticas determinadas. Uno de ellos es \textit{logko}, jefe de la comunidad. Los otros dos son reconocidos por sus conocimientos culturales e históricos de la sociedad mapuche. Al igual que los hablantes del grupo 1, hablan preferentemente \textit{mapuzugun}, su primera lengua. Los actores que nacieron durante este período histórico tenían, cuando se realizaron las entrevistas, una edad que fluctuaba entre 74 y 80 años.  

El contexto político en que nace este grupo es agitado, en 1925 se firma y promulga una nueva constitución política del país. Durante este período hay mucha inestabilidad política; sin embargo, se avanza en la construcción de obras públicas y un desarrollo institucional que implica la creación de instituciones como la contraloría, inspección del trabajo y Carabineros de Chile. El territorio mapuche sigue en el proceso de reducción y precarización. En 1927 se crea el Ministerio de Educación, entidad encargada de administrar, fomentar y fiscalizar la educación del país. Un personaje destacado en esta etapa es Pedro Aguirre Cerda, quien bajo el lema “gobernar es educar” se preocupa de expandir la educación primaria y fomenta la educación técnica e industrial por medio de escuelas especializadas.

Los episodios expresados por el grupo 2 (1925--1940) muestran la violencia y discriminación que vivieron tanto en la escuela\footnote{\citet{DonosoRomo2008} ilustra el proceso de instalación de la escuela en el territorio mapuche. El plan educativo de Estado iba de la mano con los ideales del progreso. El objetivo de este plan fue “el de transformar a los habitantes de la zona sin importar su venia” (\citealt{DonosoRomo2008}: 30).} como en el territorio, la mayoría hablaba \textit{mapuzugun} enseñado por los abuelos, lo cual indica dos cosas: 1) que los abuelos (prácticamente monolingües) cuidaban a los niños y 2) que los padres estaban fuera del hogar, en Temuco o Santiago, posiblemente en actividades laborales. A continuación, en \REF{ex:olate:4} se presenta el testimonio de un miembro de esta generación perteneciente al territorio de \textit{Makewe} (74 años):

\ea\label{ex:olate:4}
    así como decía mijo, que no tenía ganas de ir a la escuela, y así mismo hice po oye, no tenía ganas de ir a la escuela porque los cabros me pegaban mucho. Yo era tontito en la escuela, era muy tontito, cualquiera me pasa a llevar, me quitan lápiz, muchas veces me quitan goma, salgo afuera allá me pegan, y ahí yo empezaba a llorar, y ahí tuve miedo. \textup{[Corpus propio]}
\z


Vemos en este testimonio la autorepresentación que el hablante construye a partir de las experiencias traumáticas vividas en el contexto de la educación formal de su niñez. En primer lugar, se evidencia la violencia física ejercida por los compañeros. También hay una evaluación del propio hablante al decir que “era el tontito de la escuela” porque abusaban de él. Estas situaciones provocaron miedo y, por lo tanto, le restaron ganas de ir a la escuela. El narrador recuerda que los agentes que promovieron en él estos sentimientos y emociones fueron sus compañeros de la escuela.

La importancia de hablar castellano se refleja en \REF{ex:olate:5}, testimonio de otro hablante de \textit{Makewe} (76 años):


\ea\label{ex:olate:5}
 \emph{Entrevistado}: Entonces a esa edad solamente como ya sabía, aprendí a escribir, sacar la cuenta ya, por lo menos pa poder defender\\
 \emph{Entrevistador}: ¿y ahí aprendió castellano? \\
 \emph{Entrevistado}: ahí aprendí castellano y al mismo tiempo ya pa trabajar en la casa, como no había fuera del matrimonio, no, también no tenían hijos todavía, yo era primer como hijo me tenía el padrastro. Pero una vez que aprendí todo ya está que, a trabajar en la casa. Pero lo único que no alcancé aprender la división, eso no más, pero lo demás, lo sacar la cuenta pa que no me engañen, las tablas principalmente.\\
 \emph{Entrevistador}: cuándo hablaban mapuzugun en la escuela\\
 \emph{Entrevistado}: principalmente cuando pillaban hablando mapuzugun, un varillazo al tiro, lo llamaban. Entonces casi a la fuerza, dese cuenta, obligatoriamente, pienso yo, ahora. \textup{[Corpus propio]}\\
\z

Se observa el disciplinamiento de los escenarios impuestos por la sociedad hegemónica. Precisamente, durante este período los procesos de formalización educativa fueron preponderantes y son incuestionables a la hora de analizar la situación de desplazamiento actual de la lengua. Pero, además, los escenarios implicaban situaciones de educación no formales que se dan en paralelo al surgimiento de los latifundios y la migración forzada hacia la capital. Se observa, en los testimonios, la relevancia dada a la llegada de los colonos chilenos y extranjeros y la apropiación de tierras que posteriormente se convirtieron en fundos con una organización micro donde algunos mapuches desarrollarían actividades laborales.



\subsection{Experiencias del grupo 3 (1940--1955)}



Los hablantes que nacieron durante este período histórico tenían, para cuando se realizaron las entrevistas, una edad que fluctuaba entre 59 y 73 años. Los perfiles que se observan son más diversos que los anteriores, hay actores que tienen conocimientos de medicina y de filosofía mapuche, también hay autoridades culturales y agricultores. En esta generación todavía los hablantes tuvieron como primera lengua el \textit{mapuzugun}. Varios de los entrevistados afirman pertenecer a alguna religión occidental, católica o evángélica (componente que no se observa en los perfiles anteriores). A nivel nacional, durante esta etapa hay una expansión de la educación pública, se impulsa la industrialización del país y hay planes de fortalecimiento de la clase media. 

El momento del contacto, si bien aún es la etapa histórica de la reducción indígena, ya podríamos verlo como una etapa de reducción donde la expansión del Estado y el aumento de la fuerza laboral provocan, por un lado, un proceso continuo de migración hacia la ciudad y, por otro, el fortalecimiento del sistema educativo. Los procesos de usurpación de territorios al interior de las reducciones se mantienen activos {\citet{ImilanÁlvarez2007}}.

Desde este grupo hacia adelante, se observa un quiebre en la transmisión del \textit{mapuzugun} entre padres e hijos, muchos padres dejaron de enseñar el idioma a sus hijos, tarea que realizaban los abuelos. En \REF{ex:olate:6}, observamos la experiencia de un hablante de 63 años de \textit{Txanantue}:

\ea\label{ex:olate:6}
 \emph{Entrevistador}: Y usted, ¿a qué edad empezaría a hablar castellano?\\
 \emph{Entrevistado}: Como 12 años, recién. Sí. De 10 años entré a colegio y ahí empecé a hablar, y aprendí aprendí, a esa edad recién pu. No, en el, en la casa el mapuzugun. El castellano en el colegio se hacía. Claro, sí. En el colegio recién. Entonces ahí jui aprendiendo de a poco. Pero me costó igual pu, cuesta. De primera cuesta pu. Después ya no porque se acostumbra la persona, y entonces, ya y, después fui a trabajar en Temuco, tenía como… 12 años tenía pu, e iba a trabajar diez meses. Ahí jui a practicar bien, leseando con los huinquitas puh, claro, y fui grabando igual. Aprendí a hablar bien, tonce ahí peleamos, leseamos con los cabros ¿ah? Con la edad, uta que aprendí ahí. Ahí ya por lo menos aprendí mejor hablar castellano a tiempo. Y el mapudungun sabía hablar bien pu ese no me olvidé nunca, no se puede olvidar pu, no se puede olvidar, no se puede olvidar, porque con esa lengua empezó uno, mapuzugun, entonces, pero de primera como que me había acostumbrado a hablar en \textit{wigka}, después me costaba un poco pa hablar mapuche, después que volví a Temuco puh, tonce la lengua se me enredaba después un poquito, después volví a practicar de nuevo mi… mi mapuzugun, ahí ya quedé claro otra vez, claro, y yo no entré más al pueblo tampoco, aquí no. Acá no más. Tonce ahí volví a aprender de nuevo, mapuzugun más seguido, todo el tiempo seguía hablando mapuzugun, hasta, hasta ahora y, los dos idiomas hablo, los dos idiomas.\\
 Y otra cosa antes los mayores los mismos casos porque los niños eran muy, les costaba pa hablar el castellano cuando entran al colegio, tonce los mismos mayores dijeron no, se dieron cuenta, por lo menos, toce no los dejaron hablar los… sus hijos… mapuche, ellos no más hablaban pu, los viejos, los mayores. Entonces los niños no no aprendieron más pu, aprendieron hablando castellano no más, así se fue perdiendo, perdiendo la… el mapuzugun.\\
 \emph{Entrevistador}: ¿las mamás no les enseñaron? \\
 \emph{Entrevistado}: (mapuzugun) No les enseñaron. Pa’ que no sufrieran tanto cuando el colegio. Y como había \textit{wigka} también antes, eran poco, eh, ¿cómo le dijera? Eh, un poco… egoísta eran, claro, lo mapuches huincas y eso, claro, tonce por eso entonce, hacían burla, querían, siempre querían pasar delante de los mapuches, así estaba antes pu, después ya cambió eso pu, cuando aprendió a hablar \textit{wigka} los mapuche, ahí recién quedaron tranquilos los huincas. Ahí se defendió el mapuche.  \\
 \textup{[Corpus propio]}\\
 \z

En \REF{ex:olate:6} se muestran varios elementos que deben destacarse. El castellano se aprendía en la escuela, mientras que en la casa se hablaba en \textit{mapuzugun}. Otro elemento que tiene que ver con los procesos de migración es la experiencia de aprender castellano en la ciudad en el contexto del trabajo por temporada que realizaba este actor (“iba a trabajar diez meses. Ahí jui a practicar bien, leseando con los huinquitas pu, claro, y fui grabando igual”). Se aprecia también que esta temporada de trabajo se prolongó por varios años al punto de que el hablante se acostumbró a hablar en castellano y cuando regresó a la comunidad volvió a aprender el idioma materno. Se destaca también que los mayores tenían dificultades para hablar castellano, motivo por el cual no dejaron hablar mapuche a los niños. Por último, se observa la exigencia de saber castellano para defenderse de “los \textit{wigka}” (colonos chilenos y extranjeros). Este mismo miembro del grupo afirma: “Nunca me enseñaron mmm mi mamá nunca casi no me enseñaba el mapuzungun…”

Ya los padres no estaban transmitiendo el \textit{mapuzugun}. Por otro lado, las experiencias traumáticas vividas por los hablantes alimentan un estado emocional que posiblemente reestructure el proceso de valorización de la lengua y su transmisión sobre generaciones posteriores. A esto debe sumarse la experiencia de prohibición y castigo por hablar la lengua en la escuela, que se ve observa en \REF{ex:olate:7}, testimonio de un hablante de 68 años de \textit{Txanantue}:

\ea\label{ex:olate:7}
 O sea a nosotros lo prohibieron en la escuela hablar mapuzugun. Inclusive hasta me huasquearon [golpear con una huasca] una vez. Porque no quería que nadie hablara en mapuzugun po, \textbf{y} \textbf{ahora} \textbf{lo} \textbf{prefieren} \textbf{po}. \textbf{Ese} \textbf{es…} \textbf{Como} \textbf{le} \textbf{dijera,} \textbf{es} \textbf{el} \textbf{cambio} \textbf{que} \textbf{hay}. No si lo huasquiaban como dos veces que nos pillaban hablando en mapuzugun  \textup{[Txanatue, 68 años, grupo 3, corpus propio]}
 \z

Este relato confirma que esta generación sufrió una forma de violencia colonial que implicó la prohibición y el castigo por hablar la lengua mapuche. Un detalle de este episodio hay que destacar (subrayado): la idea de que ahora prefieren el \textit{mapuzugun}, que para este actor representa un cambio que, al parecer, poco se comprende, pues se da en el mismo escenario donde se castigó la práctica de la lengua. Pero es un contrapunto que muestra la valoración que actualmente se le da a la lengua en los espacios escolares.

Como bien se plantea \citet{Ochs2006}, en las narrativas de episodios de vida, los eventos sobre los que  presta atención el narrador se muestran como inusuales, inesperados y problemáticos. Observamos en los pasajes de vida testimoniados estas características. Otro elemento que debemos destacar de los relatos es el juego de los participantes entre la experiencia pasada y la presente, hay una proyección narrativa que nos permite vislumbrar cuándo el narrador está situado en la experiencia presente y cuándo rememora los hechos pasados, que claramente son más violentos y se sitúan, la mayoría, durante la niñez.

Los dos episodios anteriores pertenecen a hombres de esta generación. A continuación, en \REF{ex:olate:8}, presentamos los relatos de una hablante mujer (67 años), perteneciente a \textit{Makewe}. Observamos casi los mismos eventos que vivieron sus co-generacionales:


\ea\label{ex:olate:8}
 \emph{Entrevistador}: ¿aprendió a hablar primero mapuzugun?\\
 \emph{Entrevistado}: Sí, yo crié con mapuzungun no más\\
 \emph{Entrevistador}: ¿En el colegio aprendió a hablar castellano? \\
 \emph{Entrevistado}: Sí, en colegio. En la casa no\\
 Entrevistador:¿cuando fue a la escuela entonces aprendió a hablar en…\\
 \emph{Entrevistado:}  Sí, eso es, sí, castellano. Cuando recién fui en la escuela, ahí hablé castellano, pero tan luego no me grabó (E: claro, le costó), me costó harto, y lloraba, qué es lo que me decían, yo no sabía qué es lo que me decían. Nada. Pero a mí no me gustaba esos cosas.\\
 \emph{Entrevistador}: ¿y los profesores los castigaban cuando hablaban\\
 \emph{Entrevistado:} Los profesores, sí… ¡Hey ustedes vengan a estudiar aquí! Dicen, ¡no (¿?) nunca!, ¡no estar peleando! Eh, nos pescan varillas, ¡varillas po! ahora ni los castigan los… (No se les puede tocar un pelo ahora a los cabros), y nos dejan hincados por ahí (ah, los hacían hincar) sí pu (sobre arvejas, trigo) uf, varillazo. A veces de repente se… flor de rosa, varilla, con ese nos pegan. \textup{[Corpus propio]}\\
\z

Desde la perspectiva del contacto interétnico, observamos que la interacción entre la sociedad nacional y la mapuche es dramática para los miembros de la última dado el proceso de transformación que se experimenta. Junto con el avance de la nación en diversas materias va con fuerza el proceso asimilatorio sobre la sociedad, la cual debe sobrevivir al trato asimétrico e indolente del Estado y sus ciudadanos.



\subsection{Experiencias del grupo 4 (1955––1970)}



Los hablantes que nacieron durante este período histórico tenían, cuando se realizaron las entrevistas, una edad que fluctuaba entre 44 y 59 años. Los perfiles que se observan son también diversos y en esta generación hay actores a los cuales no se les transmitió el \textit{mapuzugun} y que por circunstancias diversas debieron aprenderlo posteriormente. Las actividades que desempeñan estos agentes son también de agricultura y funciones dirigenciales dentro de la comunidad. En cuanto a la lengua mapuche, esta generación alterna el uso del castellano y el \textit{mapuzugun}. Del conjunto de participantes, hay dos que hablan con frecuencia la lengua, mientras que los restantes cuatro usan mayoritariamente el castellano. 

Durante este período, el énfasis en la modernización por parte del Estado se acrecienta, las migraciones mapuches por trabajo hacia los centros urbanos son cada vez más frecuentes, como se describe en \citet{ImilanÁlvarez2007}. Entre 1966 y 1992 el crecimiento de la migración de la población mapuche fue de un 60\% con respecto a décadas anteriores.\footnote{Según el estudio de \citet{ImilanÁlvarez2007}, durante la década de 1950 se registran aproximadamente 10.000 mapuches en Santiago de Chile. Hacia 1960 se reportaban 15.000 y ya en 1966 la cantidad de personas llegaba a 60.000.} El proceso de migración descrito sigue un modelo de desplazamiento directo desde el campo a los centros urbanos, que produjo un colapso de los sistemas de integración de las ciudades. Se tornan relevantes las redes establecidas entre los migrantes para poder acceder a trabajos y estadía en la ciudad. Todo este proceso se debe al empobrecimiento de la sociedad mapuche y la escasez de tierras, que impulsaron esta búsqueda de mejores opciones de vida \citealt{ImilanÁlvarez2007}.

El proceso migratorio es relevante para este grupo y para el anterior, pues, como hemos evidenciado, son los abuelos los que han transmitido la lengua a sus nietos. Los padres muy probablemente han vivido ese tránsito entre el campo y la ciudad en búsqueda de mejores condiciones de vida. 

En esta generación, la lengua se hablaba en la casa, pero ya en voz de los abuelos. En la escuela, la violencia colonial, la discriminación y la estigmatización seguían marcando las experiencias sociolingüísticas del \textit{mapuzugun} y sus procesos de transmisión. Esto se evidencia en \REF{ex:olate:9}, relato de un participante de 57 años perteneciente al territorio de \textit{Txanantue}:

\ea\label{ex:olate:9}
 esos niños de copihue se burlaban de nosotroh, porque nosotroh hablan… de repente hablabamoh en mapuzungun y esos niños de… que venían arria… del fundo loh copihue, eso noh decía a nosotro india (ya).. a mi siempre decían que era… india(Entrevistador: ya) o a veces lo peliabamos y al tiro me sacaban eso … que …(\emph{Entrevistador}: india le decían) me decían india…. \textup{[Corpus propio]}\\
\z

En este grupo, se observa el desplazamiento lingüístico y la separación del hablante y su lengua. Esta generación aprendió el \textit{mapuzugun} indirectamente “porque escuchaba a mi papá hablar mapuche y ahí fui escuchando escuchando y fui aprendiendo” (\textit{Txanantue}, 44 años, Grupo 4). No existía la voluntad de los padres para transmitir el idioma, seguramente debido a las experiencias previas vividas.

Lo primero que se aprendió en la casa, para varios miembros de esta generación, fue el castellano. En esta etapa, puede ubicarse el inicio del quiebre de la transmisión intergeneracional. Los niños hablaban menos \textit{mapuzugun}, como se observa en \REF{ex:olate:10} (participante de 51 años del territorio de \textit{Makewe}):

\ea\label{ex:olate:10}
 \emph{Entrevistador:} ¿Y qué habló primero mapuzugun o… o el castellano?\\

 \emph{Entrevistado}: El castellano. El castellano. Despué fui aprendiendo \\
 \emph{Entrevistador}: O sea lo papá no le enseñaron? \\
 \emph{Entrevistado}: No, no lo enseñaban, solo lo que uno escuchó no ma. Igual conversaban en mapuche ello, entre ello. Pero a uno no casi no le, no le conversaban. Despué fui a la escuela y ma, ma se aprendió toavía a hablar castellano. no ve que nunca tocaba el tema el profe noh dejan a un lado. No nunca se habló en la escuela… mapuche. \textup{[Corpus propio]}\\
\z

En esta experiencia se evidencia que el proceso de socialización en el núcleo familiar ha cambiado de dirección. Los padres no enseñan la lengua, los hijos solo escuchaban las interacciones de los padres, pero no había voluntad de transmitir el idioma. Tal como manifestaba un actor del grupo 3 respecto de la razón por la que no enseñaban ya la lengua: “No les enseñaron. Pa’ que no sufrieran tanto cuando el colegio” (y esta idea debe extenderse a diversos escenarios del contacto interétnico). Es cierto que este puede ser uno de los motivos más potentes que subyacen a la hora de cuestionar el quiebre de la transmisión intergeneracional y el desuso de la lengua.


\section{Discusión: experiencias transversales y las marcas de las experiencias sociolingüísticas}\label{sec:olate:6}


En los ejemplos hemos visto distintos protagonistas con experiencias comunes. En una mirada transversal, la lengua va perdiendo utilidad intergeneracionalmente, aspecto que influye en los procesos de transmisión, y los actores van siendo violentados por su condición de hablantes. Los escenarios del contacto interétnico y la violencia colonial son los factores que dan impulso a esta dinámica, que se resume en un proceso de asimilación sistemático ejercido por el Estado chileno. 

La experiencia del contacto tiene implicancias para los sistemas de creencias y los comportamientos de los hablantes, las consecuencias de esta situación se vinculan directamente con la identidad lingüística y la conservación \citep{Gundermann2014}.  

La incorporación del castellano, las campañas de deslegitimación sobre el \textit{mapuzugun}, el empobrecimiento de la sociedad y los eventos de violencia vividos por los participantes son elementos que mueven a los hablantes a dejar en un segundo plano la lengua y verse forzados a acceder al castellano para desenvolverse en la situación de asimetría político-lingüística e indiferencia social.

Otro elemento que debe tenerse en cuenta en este proceso es la autopercepción que construyen los hablantes: vemos, en varios relatos, que persiste la etiqueta “indio” dicho por el Otro hegemónico chileno. Al expresar esta referencia, los entrevistados explicitan esa construcción sobre ellos y, por supuesto, la internalizan, quedando como una marca que simboliza el sistema de representaciones y prejuicios vigentes, en ese cúmulo de experiencias, sobre la lengua y sus hablantes.

Observamos que la mayoría de los relatos han sido narrados desde la niñez, puede notarse que los procesos de socialización tradicionales que la sociedad mapuche realizaba, en el momento previo a la anexión territorial, han sido drásticamente transformados. Las experiencias a las que hemos accedido muy probablemente estaban lejos del modelo que proyectaban los padres para enseñar la lengua a sus hijos. Este escenario trajo consigo un cambio en las prácticas de socialización lingüística, cuestiones que también trastocan tanto el valor asignado como las ideologías sobre el idioma \citep{Duranti2000}.

Es válido reflexionar sobre hechos como la competencia de estos niños a partir de las experiencias testimoniadas. Si se adopta la idea de que esta competencia es el resultado, por una parte, de las actividades sociales y culturales donde participan regularmente los niños y, por otra, de las lenguas que son estimuladas para el aprendizaje, entonces, tanto el contexto de vida infantil como las prácticas sociales son claves para entender los repertorios lingüísticos desarrollados (\citealt{Duranti2000}; \citealt{OchsSchieffelin2006}). A partir de los escenarios analizados y la transformación de las prácticas de socialización puede incluso comprenderse la variedad de castellano que emerge como resultado del contacto interétnico.

Como contrapunto de estas experiencias, sabemos que, actualmente, la sociedad mapuche experimenta procesos de reivindicación identitaria a partir del trabajo de diversos agentes mapuches (intelectuales, creadores y comunicadores), lo cual da cuenta de la vigencia política e intercultural del grupo; sin embargo, en lo relacionado con la lengua, el desplazamiento lingüístico sigue su curso (\citealt{ZúñigaOlateVinet2017}) a pesar de reacciones activas por parte de agrupaciones que llevan iniciativas de aprendizaje formal de la lengua, el uso en nuevos soportes y la emergencia de nuevas modalidades discursivas (\citealt{OlateVinetWittigGonzález2019}).

El hablante y la lengua, como entidades sociohistóricas, transitan por los escenarios, son objetos de representación, construyen sus propias representaciones y viven el momento político. Las cuatro generaciones analizadas representan 60 años de contacto interétnico (1910--1970)\footnote{Cabe destacar, tal como se consigna en  \citet{OlateVinetEtAl2017} que la trayectoria del \textit{mapuzugun} durante estos períodos se ha definido por campañas de estigmatización a través de la exportación y apropiación de dogmas eurocéntricos dominantes propios de la época, los cuales se han proyectado hasta la actualidad.}, tiempo que acumula experiencias, memorias, situaciones políticas e imaginarios en uno y otro lado del contacto.

El lenguaje juega un rol fundamental en la construcción y mantención de la identidad étnica. Estudios recientes señalan que las prácticas y experiencias lingüísticas son relevantes para formar y transformar la identidad étnica \citep{Fought2011}. En este contexto, hasta qué punto los escenarios relatados refuerzan o debilitan la identidad étnica de una sociedad que ha sufrido violencia colonial.  


\section{Conclusiones}\label{sec:olate:7}


Las experiencias analizadas muestran comportamientos hacia la lengua de parte de la sociedad chilena hegemónica que repercuten en el hablante minorizado, modificando proyectivamente sus actitudes lingüísticas hacia la estigmatización del idioma. Este es un proceso intergeneracional que motiva el desplazamiento de la lengua y la adopción del castellano para poder vivir en la sociedad mayoritaria.

Las experiencias en escenarios formales se proyectan también sobre escenarios informales que son instancias más explícitas en la comunicación de las imágenes, los prejuicios y la violencia. Los escenarios formales e informales se transforman en espacios de control que, por un lado, influyen en actitudes, representaciones, ideologías, procesos de transmisión, comunicación interpersonal y procesos lingüístico-cognitivos y, por otro, moldean y alimentan los comportamientos sociolingüísticos de los hablantes (\citealt{OlateVinet2017a}).

Se entiende que la situación de contacto activa situaciones sociolingüísticamente heterogéneas en el territorio (\citealt{WittigGonzálezOlateVinet2016}). Sería interesante disponer de trabajos de otros territorios para contrastar experiencias, actitudes y efectos de los escenarios del contacto. Así también, en función de la categoría "historia de vida", sería importante contar con experiencias urbanas y rurales-urbanas para conocer algo más de las dinámicas de las lenguas en el marco del contacto interétnico.

Creemos que este trabajo puede aportar una nueva manera de ver el fenómeno del contacto para los lingüistas, este no es un asunto aislado de la compleja situación política, interétnica y lingüístico-comunicativa que viven las sociedades en interacción. Un mismo hablante es un objeto complejo, sujeto a disposiciones diversas, actuando en escenarios heterogéneos, involucrado en situaciones políticas e interétnicas dinámicas.


\section*{Agradecimientos}
\largerpage

El presente estudio se realiza gracias al apoyo del proyecto FONDECYT (1180071). Agradezco al profesor Diego Lizarralde de la Universidad de La Frontera por la revisión formal del documento y los comentarios realizados. Agradezco también los comentarios críticos de los pares evaluadores de este capítulo, fueron fundamentales para la formulación de esta versión del trabajo.



\sloppy\printbibliography[heading=subbibliography,notkeyword=this]
\end{document}
