\documentclass[output=paper]{../langscibook}
\ChapterDOI{10.5281/zenodo.5643295}
\author{Carola Mick\orcid{0000-0003-0923-1211}\affiliation{Université de Paris}}
\title{Los usos de pronombres átonos de actores de la Consulta Previa en el Perú}
\abstract{Con la implementación de la Ley de Consulta Previa de los pueblos indígenas y originarios en el año 2011 en el Perú, ya no son solo migrantes internos los que promueven el contacto entre diferentes variedades del español, sino que se dinamiza el contacto de una forma potencialmente más global, simétrica y recíproca. El presente artículo analiza estas dinámicas concentrándose en los sistemas de pronombres clíticos átonos de 3\textsuperscript{a} persona en el español de actores de la Consulta Previa. Estudios anteriores reconstruyeron sistemas diferentes para el español de hablantes provenientes de zonas bilingües quechua-español y hablantes monolingües de la capital, pero demuestran que, con más movilidad rural-capitalina, los hablantes andinos tienden a converger hacia el sistema capitalino monolingüe. Aquí se toman en consideración actores andinos y limeños de la Consulta Previa con diferentes grados de movilidad: comuneros locales, representantes indígenas a nivel regional y nacional, y actores limeños con vínculos más o menos intensos con hablantes provenientes de zonas bilingües quechua-hablantes. El análisis demuestra un acercamiento de los sistemas de pronombres clíticos en algunos aspectos -- pero no todos -- conforme aumente y se intensifique la movilidad rural-capitalina de los hablantes de ambas procedencias; en cuanto a la movilidad social, sin embargo, parece que mientras más la construcción de identidad de los hablantes andinos se apoya en la categoría institucionalmente protegida del “indígena”, más enfatizan su particularidad lingüística. Por consiguiente, el artículo argumenta que existe un uso estratégico --  no necesariamente consciente  -- de las características lingüísticas según el contexto de comunicación. El presente estudio confirma que las prácticas lingüísticas “en el sur global” \citep{Léglise2017} no son homogéneas, sino que llevan las huellas de dinámicas socioculturales heteróclitas; el análisis de los pronombres clíticos en el español de contacto permite acercarse al entendimiento de las transformaciones que conlleva la implementación de la Ley de la Consulta Previa en el Perú.
% \keywords{ variación sociolingüística, español de los Andes, pronombres átonos de 3a  persona, Consulta Previa, pueblos indígenas y originarios}
}

\IfFileExists{../localcommands.tex}{
  \addbibresource{localbibliography.bib}
  \input{../localpackages}
  %Copy this to localcommands.tex

\usepackage[english]{babel}
\usepackage{amsmath}
\usepackage{amssymb,amsfonts,textcomp}
\usepackage{array}
\usepackage{hhline}
\usepackage{hyperref}

\newenvironment{styleStandard}{}{}
\newenvironment{stylelsAbstract}{}{}
\newenvironment{stylelsSectioni}{}{}
\newenvironment{stylelsSectionii}{}{}
\newenvironment{stylelsBulletList}{}{}
\newenvironment{styleBibliographyi}{}{}
\newenvironment{listWWNumxxvleveli}{}{}
\newenvironment{listWWNumxxvlevelii}{}{}
\newenvironment{listWWNumxxvleveliii}{}{}
\newenvironment{listWWNumxxvleveliv}{}{}
\newenvironment{listWWNumixleveli}{}{}
\newenvironment{listWWNumixlevelii}{}{}
\newenvironment{listWWNumixleveliii}{}{}
\newenvironment{listWWNumixleveliv}{}{}

\newcommand\textstyleListLabelxvi[1]{#1}
\newcommand\labellistWWNumxxvleveli{\thelistWWNumxxvleveli.}
\newcommand\labellistWWNumxxvlevelii{\thelistWWNumxxvlevelii.}
\newcommand\labellistWWNumxxvleveliii{\thelistWWNumxxvleveliii.}
\newcommand\labellistWWNumxxvleveliv{\thelistWWNumxxvleveliv.}
\newcommand\labellistWWNumixleveli{[F0B7?]}
\newcommand\labellistWWNumixlevelii{\textstyleListLabelxvi{o}}
\newcommand\labellistWWNumixleveliii{[F0A7?]}
\newcommand\labellistWWNumixleveliv{[F0B7?]}

\newcounter{listWWNumxxvleveli}
\newcounter{listWWNumxxvlevelii}[listWWNumxxvleveli]
\newcounter{listWWNumxxvleveliii}[listWWNumxxvlevelii]
\newcounter{listWWNumxxvleveliv}[listWWNumxxvleveliii]
\newcounter{itemize}  
  \input{../localhyphenation} 
  \togglepaper[1]%%chapternumber
}{}

\begin{document}
\maketitle 
%\shorttitlerunninghead{}%%use this for an abridged title in the page headers





\section{La Ley de Consulta Previa y la variación del español en el Perú}\label{sec:mick:1}

Pese a los cambios significativos en las políticas lingüísticas del país y en la actitud de muchas y muchos\footnote{Salvo en casos marcados explícitamente, las formas plurales en género gramatical masculino usadas en este artículo se refieren a grupos mixtos de mujeres y hombres.} hablantes mismos, la cohabitación entre las lenguas indígenas u originarias\footnote{\label{fn:mick:2}Me alineo aquí a la terminología elaborada por la Organización Internacional del Trabajo (OIT), en colaboración con representantes de pueblos indígenas u originarios de varios países (Convenio 169), y uso las categorías “indígena” y “originario” como sinónimos. Hay que destacar que en la zona andina en el Perú, la categoría “indígena” no es muy popular (ver por ejemplo \citet{Rousseau2012}, ya que les recuerda a los afectados la categorización institucional fuertemente discriminatoria como “comunidad indígena” (explotada por terratenientes) que recién cambió (a “comunidad campesina”, propietaria colectiva de las tierras trabajadas) con la Reforma Agraria de los años 1970. Sin embargo, las poblaciones están generalmente conformes con considerarse “originarias” de la zona y conscientes de su particularidad cultural, invocando algunos de los criterios enlistados por la OIT para definir “pueblos indígenas y originarios”.} y el español sigue siendo predominantemente diglósica\footnote{Me refiero aquí al concepto de diglosia forjado por \citet{Ferguson1959}, designando situaciones de cohabitación asimétrica de diferentes variedades de una lengua existentes en una sociedad. \citet{Fishman1972} abre el concepto para incluir casos de bilingüismo social asimétrico y \citet{Lienhard1996} a otras formas de expresión cultural; ambas extensiones teóricas del término son importantes para la argumentación aquí desarrollada.} en el Perú, y esta asimetría también se reproduce en la relación entre diferentes variedades regionales y de contacto del español (ver, entre otros, \citealt{Escobar1978}; \citealt{Cerrón-Palomino1987}; \citealt{Gugenberger1994}; \citealt{TubinoZariquiey2007}):

\begin{quote}
En el curso de los siglos, cantidades de hablantes monolingües de vernáculo han ‘aprendido’ el castellano en las condiciones más adversas imaginables, con el resultado de un sistema fracturado, precario y estigmatizado socialmente, que, desde el punto de vista de una teoría comunicativa que atienda a las formas de relación que se establecen entre los comunicantes en función del dominio de las posibilidades del sistema mismo, sólo permite una comunicación asimétrica, en la que el bilingüe es desde el comienzo la instancia débil y deprimida de la relación polar, pero no sólo por razones relativas a la técnica misma del hablar sino por motivos psico-sociales, pues estos viciados procesos de bilingüización, además de implicar situaciones sociales y económicas de desventaja, han supuesto frecuentemente un desarraigo cultural y lingüístico, al haber sido orientados por una implícita ideología asimilacionista y homogeneizadora (\citealt{Rivarola1985}: 40–41).  
\end{quote}

Como explica esta cita, la asimetría entre las lenguas y variedades coincide con una repartición desigual de los recursos en la sociedad, tal y como observa \citet{Léglise2017} en las prácticas lingüísticas en el sur global en general. La Ley de Consulta Previa, aprobada e implementada en el 2011 por luchas y presión de organizaciones indígenas en el Perú, sin embargo, reaviva la promesa de remediar estas asimetrías entre comunidades lingüísticas y comunicativas heredadas del pasado. De hecho, obliga al Estado a respetar los derechos colectivos particulares de pueblos y comunidades indígenas a la integridad física, cultural y territorial, y a consultar con ellos cualquier proyecto, legislativo, de inversión pública o privada, de políticas sociales, etc, que pueda afectarlas. El reglamento de esta ley prevé un proceso de interacción en siete etapas, donde el Estado identifica la medida que somete a consulta, así como los pueblos indígenas del ámbito geográfico de la intervención o directamente afectados, publica el proyecto y los informa de él, deja cierto tiempo a los pueblos indígenas para consultar internamente y se sienta con sus representantes a dialogar antes de tomar la decisión final sobre los acuerdos alcanzados. Pese a los límites,\footnote{Los aspectos comúnmente más criticados son la imposibilidad del veto de parte de los pueblos indígenas una vez iniciado el proceso de consulta, el hecho de que la decisión final le compete al Estado, y que nada asegura que los acuerdos alcanzados se respeten (\citealt{VegaAuqui2016}). Se puede añadir, entre otros problemas, la mono-direccionalidad de la Consulta Previa que siempre se produce por iniciativa de las instituciones (véase por ejemplo \citealt{Mick2016}, no de los pueblos indígenas; problemas de representatividad en las mismas organizaciones indígenas, así como la falta de recursos de estas últimas para poder asegurar el diálogo con sus propias bases locales.} a falta de una representación indígena institucionalizada a nivel de gobierno (con algunas excepciones a nivel municipal), la Ley de Consulta Previa (de aquí en adelante: CP) es un avance significativo, ya que asegura que se incluyan opiniones de comunidades indígenas cuando se trata de tomar decisiones en el Estado que las afecte (ver también \citealt{Vargas2016}). Sin embargo, esta ley está lejos de construir relaciones completamente simétricas entre el Estado -- el centro tradicional del poder -- y los pueblos indígenas – las poblaciones estructuralmente más minorizadas del país. A largo plazo, el reglamento de la Ley de Lenguas o del Plan Nacional de Educación Intercultural Bilingüe, ambos sometidos a procesos de CP entre los años 2015 y 2016, prometen iniciar una fase de transición hacia un orden sociocultural menos marcado por la diglosia en el Perú.

El presente artículo se interesa por las dinámicas sociolingüísticas que se producen en el español de contacto entre actores provenientes de regiones andinas que se categorizan como “indígenas” u “originarios”, con representantes de las instituciones y la sociedad civil limeña. Más específicamente, analizará los sistemas de pronombres clíticos átonos de 3\textsuperscript{a} persona elaborados por estos actores, ya que, según investigaciones anteriores, en este aspecto, las variedades del “español limeño” y del “español andino” divergen en este aspecto. 

El artículo presentará el estado de la cuestión científica de la variación de los pronombres clíticos átonos en los Andes (apartado §\ref{sec:mick:2}), y detallará la metodología elaborada para el análisis de estos en relación con la Ley de CP (apartado §\ref{sec:mick:3}). El análisis empieza en el apartado §\ref{sec:mick:4} con una vista global de las tendencias pronominales encontradas en el corpus, y las secciones §\ref{sec:mick:5}, \ref{sec:mick:6} y \ref{sec:mick:7} se dedican a indagar las variaciones encontradas entre los diferentes grupos de actores, así como en los mismos idiolectos de dos hablantes andinas. Se resumen y discuten los resultados del análisis en el apartado §\ref{sec:mick:8}, y la sección §\ref{sec:mick:9} concluye con una discusión de los aspectos novedosos de estudios como el presente, y los aprendizajes que permiten en vista de la comprensión del contacto entre el español y las lenguas amerindias.

\section{Variación de los pronombres clíticos en los espacios andino y limeño}\label{sec:mick:2}

Considerando factores socio-históricos relacionados con la difusión del español a partir del siglo XVI, así como los aspectos sociolingüísticos relacionados con las características culturales de las poblaciones originarias, tradicionalmente se distinguen tres zonas dialectales en el español en el Perú: la costeña, la andina y la amazónica \citep{Caravedo1992}. En cuanto a los pronombres átonos de 3\textsuperscript{a} persona, se sostiene de manera esquemática que en la zona costeña peruana domina el llamado sistema “etimológico” no-leísta que marca en el pronombre átono las dimensiones de género, número y caso gramatical del referente; mientras que en la zona andina peruana se observa una tendencia a distinguir únicamente el caso gramatical del referente, y a generalizar para referentes de objeto directo (OD) el uso de un solo archimorfema, que – dependiendo de factores como la zona, los hablantes, el contexto de interacción, de enunciación y de predicación – puede ser: \textit{lo, Ø} (omisión del pronombre) o, menos frecuentemente \textit{le} (\citealt{Caravedo1992}; \citealt{KleeLynch2009book}). La tabla \ref{tab:mick:1} esquematiza los dos extremos del paradigma pronominal en el Perú. Ninguno de estos dos sistemas extremos se realiza de forma pura.

\begin{table}
\caption{\label{tab:mick:1} El paradigma pronominal átono de 3\textsuperscript{a} persona en el Perú}


\begin{tabularx}{\textwidth}{X cc cc ccc}
\lsptoprule
Sistema & \multicolumn{4}{c}{OD} & &{OI}&\\
\cmidrule(lr){2-5}
\cmidrule(lr){6-8}
& \multicolumn{2}{c}{ f} & \multicolumn{2}{c}{ m} & n & sg & pl\\
\cmidrule(lr){2-3}
\cmidrule(lr){4-5}
& sg & pl & sg & pl &  &  & \\
\midrule
Etimológico & \textit{la} & \textit{las} & \textit{lo} & \textit{los} & \textit{lo} & \textit{le} & \textit{les}\\
\cmidrule(lr){2-6}
\cmidrule(lr){7-8}
Archimorfemas alternativos & \multicolumn{5}{c}{ \textit{lo – Ø – le} } & \multicolumn{2}{c}{ \textit{le}}\\
\lspbottomrule
\end{tabularx}
% \todo[inline]{check lines in table}
\end{table}

Comparables declinaciones del paradigma pronominal átono de 3\textsuperscript{a} persona se encuentran también en diferentes grados y formas en otros países andinos como Ecuador y Bolivia (\citealt{KleeLynch2009book}), y los estudios lingüísticos al respecto las atribuyen al contacto del español con el quechua y/o el aimara, sin que tengan que limitarse a hablantes bilingües (\citealt{Escobar1978}; \citealt{Escobar2000,PalaciosAlcaine2015}). De hecho, el quechua no marca necesariamente un objeto directo implícito, dispone de un orden sintáctico más flexible y tiende a eliminar las redundancias a nivel de número; el género no es una categoría gramatical sino semántica opcional en el quechua \citep{Pfänder2009}. Subrayan los estudios mencionados que algunas de las variantes pronominales observadas en el español de los Andes ya están históricamente documentadas en el español peninsular. También argumentan que el contacto con lenguas amerindias conllevaría que se aprovechen, impongan y probablemente exacerben estas tendencias ya posibilitadas estructural o históricamente por la lengua española.

La mayoría de los ejemplos idiolectales estudiados en el Perú combinan de forma compleja las dos tendencias esquematizadas en la tabla \ref{tab:mick:1} y demuestran una gran variedad de soluciones encontradas en la intersección de estos sistemas pronominales, incluso entre hablantes de la misma zona. Interesándose por el español de Puno en el año 1983, \citet{Godenzzi1986} documenta las tres variantes de combinación de los pronombres átonos enumerados en la tabla \ref{tab:mick:2}, sabiendo que trata la omisión (Ø) como un caso aparte, ya que en el corpus estudiado sólo aparece en contextos de referentes de OD antepuestos.

\begin{table}
\caption{\label{tab:mick:2} Variación de los sistemas pronominales átonos de 3\textsuperscript{a} persona en el español de Puno \citep[188]{Godenzzi1986}}
\begin{tabular}{lll}
\lsptoprule
& { \textbf{OD}} & \textbf{OI}\\
\midrule
 a & lo(s), la(s) &\\
%\hhline%%replace by cmidrule{~~-}
 b & lo(s), la(s)  le(s), le(s) & le(s) \\
 c & lo(s) & \\
%\hhline%%replace by cmidrule{--~}
\lspbottomrule
\end{tabular}
\end{table}

Las variantes a, b y c mantienen la distinción de caso y el uso del pronombre \textit{le/s} para referentes de OI, pero difieren en las soluciones encontradas para el OD: la solución (a) equivale al sistema etimológico dominante, mientras que la solución (b) presenta formas de leísmo canónico con referentes de OD de personas, independientemente del género de estas, y la solución (c) elimina las distinciones de género gramatical del OD en general. Según \citet{Godenzzi1986}, la solución (a) predomina en los idiolectos de los habitantes de Puno originarios de otras ciudades (Lima o Arequipa, por ejemplo), mientras que los hablantes originarios de Puno de arraigo citadino y que tienen el español como primera lengua (L1) adoptan mayormente la solución (b). Los sistemas pronominales de los puneños de arraigo rural y que tienen el quechua o el aimara como L1 tienden más al sistema (c), pero también adoptan los sistemas (a) y (b) en contextos más específicos. 

En el español de hablantes provenientes de la zona andina de Chincheros (Cuzco), \citet{GarcíaTesoroFernándezMallat2015} no encuentran casi ocurrencias de leísmo en un corpus recogido entre los años 2011 y 2012, sino un uso pronominal átono de 3\textsuperscript{a} persona que marca las distinciones de caso de los referentes. Sin embargo, demuestran que la tendencia a usar los morfemas \textit{lo} y \textit{Ø} con referentes de OD forman parte de un mismo sistema pronominal, ya que ambas formas coocurren en porcentajes casi iguales en el corpus. Según el análisis estadístico de los enunciados, juegan un papel el nivel socioeconómico de los hablantes, si el referente lleva el rasgo semántico +animado o no, el tipo de verbo (de estado o no) y la forma en la cual aparece (conjugado o no), así como la cercanía del referente (\citealt{GarcíaTesoroFernándezMallat2015}: 146): la omisión es más frecuente en hablantes de nivel socioeconómico bajo y en contextos de enunciación con un verbo en gerundio, un referente inanimado y un verbo de estado; por su parte, el uso del pronombre \textit{lo} se ve favorecido por referentes humanos, referentes presentes en la misma predicación, así como verbos de no-estado y conjugados.  \citet{SánchezParaíso2019} confirma estos datos en la variedad del español documentada en el distrito Juliaca del departamento peruano-andino Puno, en la cual los hablantes optan por la omisión en aproximadamente el 30\% de los casos analizados.

30 años antes, sin embargo, \citet{Godenzzi1986} documentó en el departamento de Puno sistemas pronominales leístas sin distinción de género, aunque en mucho menor frecuencia que en el caso de la zona andina de Ecuador (\citealt{KleeLynch2009book}). El estudio de \citet{ParedesValdez2008} en Cajamarca, en el norte del Perú, documentó casos de leísmo en hablantes monolingües con una frecuencia similar a la de hablantes bilingües en Lima (15\%). Ambos grupos de hablantes tienden a distinguir entre referentes animados y no animados, siendo más frecuente el leísmo en el primer caso. Sin embargo, los hablantes bilingües, contrariamente a los monolingües, tienden a neutralizar el número gramatical del referente en el pronombre, es decir, generalizan el uso de \textit{le} independientemente del número gramatical del referente.

El estudio de \citet{KleeCaravedo2005} demuestra que el leísmo es un fenómeno existente entre limeños monolingües, pero en su estudio, éste constituye solo el 4\% de los casos; en los discursos de migrantes internos residentes en Lima entrevistados, este porcentaje asciende al 22\%, y en el habla de sus hijos al 20\%. En cuanto al uso del archimorfema \textit{lo} para referentes de OD plural o femenino, sin embargo, las autoras encuentran una gran diferencia entre migrantes (OD pl: 64\% – OD f: 76\%), la generación de sus hijos (OD pl: 57\% – OD f: 21\%) y los limeños (OD pl: 22\% – OD f: 2\%). Explica que “los limeños perciben las discordancias de género\footnote{El presente trabajo usa el término de la “neutralización de género gramatical” para evitar eventuales asociaciones valorativas de los diferentes sistemas pronominales alternativos.} (\textit{lo pintan la calle}) de modo sobresaliente, incluso más que las de caso”  \citep[285]{Caravedo2014}, lo cual descartaría la opción del archimorfema \textit{lo} para ellos, siendo esta última claramente la preferida por los migrantes. El uso de \textit{le} para referirse a referentes de OD humanos de género masculino (preferiblemente) es una alternativa intermedia, que les permite a los hablantes limeños acercarse ligeramente a un sistema pronominal con archimorfema promovido por los migrantes andinos, y a la vez mantener la dimensión del género gramatical que guía el uso etimológico de los pronombres \textit{lo} y \textit{la}. 

\section{El corpus}\label{sec:mick:3}

El corpus aquí analizado se constituyó en el marco de un estudio más grande, en una colaboración entre el Instituto francés de Investigación para el Desarrollo y la Pontifica Universidad Católica del Perú, acerca de las mediaciones socioculturales que se producen entre los representantes del Estado y de la sociedad civil en el marco de la Ley de CP de los pueblos indígenas y originarios.\footnote{Agradezco a ambas instituciones y en particular a Carlos Garatea por el apoyo de este proyecto.} 

Se trata de conversaciones grabadas, en parte entrevistas, con actores hispanohablantes acerca de sus experiencias con y opiniones acerca de la Ley de CP y su implementación en el Perú. Para los fines del presente artículo se analizan únicamente las grabaciones que implican hablantes provenientes de la zona andina -- mayormente de Huancavelica y Ayacucho --, o hablantes limeños con algún vínculo laboral o personal particular con hablantes de la zona andina, predominantemente con la zona de Cajamarca. La primera parte del análisis (\sectref{sec:mick:4}) incluye, a título comparativo, referencias a otros actores limeños entrevistados que no mencionan ningún contacto particular con la zona andina. Sin embargo, la variación pronominal en los discursos de estos hablantes no será objeto del presente artículo.

La selección de discursos aquí analizada proviene de 13 conversaciones entre la entrevistadora (autora del presente artículo) y 18 interlocutores,\footnote{La tabla \ref{tab:mick:8} en los anexos presenta a los interlocutores. Se tuvieron que descartar los enunciados demasiado escasos de tres actoras de proveniencia andina que intervienen de manera muy breve en estas conversaciones, ya que por falta de datos no fue posible caracterizar su sistema pronominal.} en grabaciones de un total de más de 10.25 horas. 7 de estas conversaciones se grabaron en Lima en forma de entrevista individual. 6 de los entrevistados son originarios de Lima o arraigados culturalmente en la ciudad; mientras que una entrevistada, dirigenta de una organización nacional de mujeres indígenas, permanece por razones laborales en Lima, aunque su residencia habitual estaba en el departamento andino de Ayacucho. Las otras 6 grabaciones se realizaron en la región andina. En ellas intervinieron entre uno y tres interlocutores principales, además de una o dos representantes de la misma organización de mujeres indígenas que establecieron el contacto con las/los hablantes y actuaron de intermediarias. La tabla \ref{tab:mick:9} en los anexos presenta una lista de estas seis conversaciones indicando los interlocutores respectivos.

El tiempo de habla individual varía considerablemente entre los actores limeños (aproximadamente 57 minutos de tiempo de habla por entrevistada/o) y los actores andinos (aproximadamente 15 minutos de tiempo de habla por entrevistado/a); el tiempo de habla promedio individual de estos últimos es generalmente más corto, independientemente del formato de la conversación (entrevista individual o colectiva). 

Los actores limeños entrevistados son:

\begin{itemize}
\item Dos mujeres representantes de instituciones nacionales e internacionales: Jimena,\footnote{Se trata de pseudónimos.} Liz. Ambas están interviniendo en procesos de CP y tienen vínculos familiares con la zona andina; Liz indica ser proveniente de la zona andina pero reside desde hace tiempo en Lima.
\item Una abogada y un abogado, representantes de organizaciones de la sociedad civil que vigilan la protección de los derechos colectivos de comunidades indígenas en la zona andina: Juan y Sara. Sara se categoriza como ayacuchana pero reside ya desde hace tiempo en Lima.
\item Dos hombres que trabajan de manera independiente como consultores de empresas o instituciones que interactúan con pueblos indígenas: Martín ha vivido y trabajado muchos años en la zona de Cajamarca, y Denis trabaja con empresas del sector minero, cuyas actividades se concentran en la zona minera en general, también en Cajamarca.
\end{itemize}

Estas tres entrevistas se realizaron de manera individual en o cerca de los lugares de trabajo de los entrevistados, y son semidirigidas, aunque los entrevistados son los que más manejan el desarrollo de la conversación.

Los actores grabados en la zona andina son: 

\begin{itemize}
\item Seis comuneras campesinas, socias de una organización local de mujeres en el departamento de Huancavelica, afiliadas a una de las organizaciones nacionales de mujeres indígenas más importantes: Paola, Inés, Eva, Flor, Carmen y Ana. Las conversaciones tuvieron lugar en sus comunidades respectivas durante sus trabajos cotidianos en la chacra, en el mercado, con los hijos, mientras preparaban la alimentación de la familia y/o de los animales.
\item Clara, la presidenta de la asociación local de mujeres antes mencionada; Alberto, el expresidente comunal de una comunidad en el mismo departamento; Héctor, el alcalde de una municipalidad de la zona. Las conversaciones se grabaron mientras los actores realizaban actividades cotidianas, por ejemplo en el mercado, en plena carretera o en un taxi.
\item Dos representantes regionales de una organización nacional indígena a nivel nacional, provenientes de la misma zona y residentes en un centro urbano cercano. Se trata de una madre, Ruth, y de su hija Lea; a Lea la entrevistadora la había conocido en el local de la organización indígena nacional en Lima antes de realizar las entrevistas. Lea es la actora más joven del corpus, seguida por Jimena y Sara. Ruth y Lea intervienen en 6\footnote{Lea está presente en una séptima conversación, con el alcalde Héctor, pero prácticamente no interviene, así que no se pudo reconstruir su sistema pronominal en esa conversación.}  grabaciones cada una; dado que su uso pronominal varía bastante dependiendo del contexto de conversación, el apartado §\ref{sec:mick:5} tratará sus discursos en cada una de estas interacciones de forma separada, mientras que el apartado §\ref{sec:mick:6} analiza la variación intra-idiolectal más a detalle.
\item La vicepresidenta de la organización nacional de mujeres indígenas, María. Por su labor como vicepresidenta, la entrevista se realizó en Lima, pero María proviene de y reside principalmente en la zona andina del centro-sur del país. Es la única actora andina grabada en una entrevista individual; la conversación se realizó en Lima en un restaurante después del almuerzo.
\end{itemize}

No todas las seis comuneras grabadas están informadas sobre la Ley de CP y las implicancias concretas para su vida, aunque están asociadas a una organización de mujeres que a su vez está afiliada a una de las organizaciones nacionales indígenas más influyentes y presentes en todos los procesos de CP a nivel nacional. No todos los actores grabados en los Andes están de acuerdo con la categoría “indígena”\footnote{Curiosamente, una parte de las entrevistas se realizaron en una comunidad andina que por los azares de la historia sigue manteniendo en las actas oficiales su estatus de “comunidad indígena” de antes de la Reforma Agraria. Por ello, en el pasado, sus integrantes fueron sujetos a burlas de las otras comunidades en los alrededores, que sí alcanzaron el estatus de “comunidad campesina”. Ahora, sin embargo, Ruth y Lea, que se consideran como originarias de esa comunidad, actúan de promotores del uso de la categoría “indígena” más que “originario”.}  (ver nota de página número \ref{fn:mick:2}), pero sí comparten la conciencia de la presencia histórica de su comunidad en la zona, de su particularidad sociocultural y también de la marginación y de las vulnerabilidades que experimenta históricamente ante el Estado y los intereses empresariales.

El análisis reconstruye tendencias en el uso de los pronombres clíticos átonos de los interlocutores, y describe su variación tomando el llamado “sistema etimológico” como referencia. El objetivo no es evaluar las entrevistas en base a una supuesta “norma”, sabiendo que el sistema etimológico no lo realiza prácticamente ningún hablante del español en forma pura, sino tener un punto común de comparación para caracterizar las particularidades de los usos pronominales individuales.

El análisis global del corpus permitió reconstruir, además de una influencia muy dominante del sistema etimológico, cuatro peculiaridades que interesan particularmente y que el apartado §\ref{sec:mick:4} presentará con más detalle:

\begin{itemize}
\item Omisión: se cuenta como omisión únicamente el uso del morfema Ø con un referente de objeto directo (OD) u objeto indirecto (OI) explícito.\footnote{En el corpus existe una tendencia bastante llamativa a usar una serie de verbos transitivos sin complemento de OD, cuya variación entre los hablantes merece otro estudio más detallado; aquí solamente se contabiliza como omisión pronominal cuando el referente de OD es explícito.} Resultó desafiante el tratamiento de la omisión con OD prepuestos, bastante frecuentes en las variedades andinas del español caracterizadas por un orden sintáctico más flexible (\citealt{KleeLynch2009book}). Se decidió contar todos los casos de omisión en los cuales los hablantes ponen cierto énfasis en el OD antepuesto. Por ejemplo,
generalmente se cuentan como omisión construcciones con el pronombre demostrativo
neutro antepuesto, como en los ejemplos  \REF{ex:mick:1a} y \REF{ex:mick:1b}, mientras que en una construcción menos enfatizada como en \REF{ex:mick:1c} no hay lugar para colocar un pronombre átono, así que no se cuenta como omisión:
\end{itemize}

\newpage
\ea\label{ex:mick:1}
\ea\label{ex:mick:1a} \textit{Eso Ø estamos haciendo aprobar ɽecié:n en el pacto} (Ruth\_1)\footnote{Las entrevistas se transcribieron aplicando las pautas elaboradas para el análisis de la conversación (GAT) por \citet{SeltingEtAl1998}.}
\ex\label{ex:mick:1b} \textit{eso nomá nos Ø ha hecho pue} (Eva)
\ex\label{ex:mick:1c} \textit{faena hemos hecho} (Carmen)
\z
\z

\begin{itemize}
\item Neutralización de la dimensión gramatical de género en los pronombres clíticos átonos de OD: se refiere al uso de \textit{lo/s} con referentes de OD de género gramatical femenino. La relación inversa (\textit{la/s} para ODm) es muy escasa en el corpus.
\item Leísmo: uso de los pronombres \textit{le/s} con referentes de OD. Como referencia, para la categorización de un verbo como in/transitivo en la acepción específica de cada contexto de enunciación, se usó la versión en línea del \textit{Diccionario de la Real Academia Española}; es sabido que este no considera suficientemente variedades hispanoamericanas del español, por lo cual las tasas de leísmo aquí calculadas son ciertamente mucho mayores de las que percibiría como incongruencias un/a hablante familiarizado/a con el español del Perú. De igual manera, el análisis contabiliza como leísmo los escasos casos documentados de leísmo de cortesía.
\item Loísmo y laísmo: uso de \textit{lo/s} o \textit{la/s} con referentes de OI.
\item Neutralización de la dimensión gramatical del número: no siempre se explicita el número gramatical del referente en el pronombre átono de 3\textsuperscript{a} persona con una \textit{--s} final, y una \textit{--s} final en el pronombre átono no necesariamente se refiere a un referente de número gramatical plural. Se contabilizaron como marcas de plural las variantes debilitadas, aspiradas y/o velarizadas de la \textit{--s} coexistentes en el español en el Perú \citep{Caravedo1990}.
\end{itemize}

Otro desafío para el conteo fue la pronominalización catafórica, como en los ejemplos siguientes:


\ea Pronominalización catafórica\label{ex:mick:2}
\ea\label{ex:mick:2a} \textit{no lo malogre} \textbf{\textit{nuestra} \textbf{carretera}}\textit{, no lo malogre} \textbf{\textit{nuestro} \textbf{canalito}} (Ruth\_1)
\ex\label{ex:mick:2b} \textit{pero encima lo había .h e sembrado} \textbf{\textit{cebada} \textbf{y:} \textbf{alfa}} (Paola)
\ex\label{ex:mick:2c} \textit{y eʎos aquí en=l camino todo en el camino, eʎos ya lo han hecho} \textbf{\textit{plantones}} (Paola)
\z
\z

En el ejemplo \REF{ex:mick:2a} se contabilizaron los dos casos de \textit{lo} como pronombre átono de 3\textsuperscript{a} persona, mientras que se descartaron los ejemplos \REF{ex:mick:2b} y \REF{ex:mick:2c}, porque no queda claro si el referente de \textit{lo} es el terreno sembrado en \REF{ex:mick:2b} y el camino en \REF{ex:mick:2c}, o si se trata efectivamente de una pronominalización catafórica. Solamente hay 6 casos dudosos de este tipo en el corpus, de los cuales 2 pronombres difieren en género como en el ejemplo \REF{ex:mick:2b} y uno en número como en el ejemplo \REF{ex:mick:2c} del referente. Contabilizarlos no cambia ninguna de las tendencias aquí descritas ni la argumentación aquí desarrollada.

El estudio no considera ninguno de los pronombres átonos dobles de 3\textsuperscript{a} persona como en los ejemplos \REF{ex:mick:3a} a \REF{ex:mick:3c} , puesto que en estos casos se producen fenómenos de otra índole  \citep{Fernández-Ordóñez1999} que necesitarían un análisis aparte.


\ea\label{ex:mick:3} Se le/s, se lo/s, se la/s
\ea\label{ex:mick:3a} \textit{hasta ahorita tampoco [}\textbf{\textit{el} \textbf{alcalde}}\textit{] no nos da pa TOdos} \textbf{\textit{los} \textbf{materiales}} \textit{que ha sobrADo, se lo han llevado. se la han devuelto} (Eva)
\ex\label{ex:mick:3b} \textit{se ha secado} \textbf{\textit{su} \textbf{tripa}} \textit{totalmente, se les está secando} (Ruth\_7)
\ex\label{ex:mick:3c} \textit{también} \textbf{\textit{la} \textbf{defensoría}} \textit{participa, normalmente se le invita a todas, a estas treh reuniones} (Jimena)
\z
\z

En el caso de reparaciones marcadas como tales con indicios lingüísticos, como por la hesitación en el ejemplo \REF{ex:mick:4}, se contó solamente el pronombre definitivo encontrado por la hablante (\textit{la}); sin embargo, el ejemplo \REF{ex:mick:2a}, la hablante no lo marca como autocorrección, así que contaron ambos pronombres átonos.


\ea\label{ex:mick:4} {a usted más bien lo: la agradecemos} \textbf{\textit{señorI'}} (Alberto)
\z

Por el número de interlocutores grabados, el tiempo restringido de habla (sobre todo para las hablantes provenientes de zonas andinas bilingües), así como las decisiones probablemente discutibles tomadas con respecto al conteo, queda claro que los resultados aquí presentados no tienen valor representativo. Sirven para discutir las dinámicas observadas, sin poder estimar la magnitud del fenómeno en la sociedad peruana en su totalidad.

\section{Las posibilidades de usos pronominales en el corpus}\label{sec:mick:4}

Los usos pronominales en el corpus en su conjunto confirman el paradigma del español peruano resumido en la \tabref{tab:mick:1}: coexisten varios sistemas pronominales que se encuentran en la intersección de los dos extremos, es decir, el sistema etimológico y el uso de archimorfemas.

Como subrayado arriba, el llamado “sistema etimológico”, que marca caso (ejemplo \REF{ex:mick:5a}: \textit{los} versus \textit{les}), género (ver la autocorrección en el ejemplo \REF{ex:mick:4}) y plural gramatical del referente (ejemplos \REF{ex:mick:5b}: \textit{la} versus \textit{las}) es la tendencia predominante de la gran mayoría de los discursos aquí estudiados. 


\ea\label{ex:mick:5}
\ea\label{ex:mick:5a} \textit{cuando [}\textbf{\textit{los} \textbf{sectores}}\textit{] noh pi*den capacitación a veces, loχ capacitamos y leh decimos todo,} (Jimena)
\ex\label{ex:mick:5b} \textit{pero (.) esa} \textbf{\textit{la} \textbf{defensa}} \textit{de conga; la=hemos hecho ad honorem. […] donde} \textbf{\textit{las} \textbf{decisiones}} \textit{siempre las van a tomar las organizaciones indígena loh líderes} (Sara)
\z
\z

Paralelamente, el corpus muestra una tendencia importante a la omisión del pronombre, es decir, su expresión con el morfema cero (Ø). Esta tendencia afecta a referentes de cualquier género y número gramatical, y tampoco tiene en cuenta el rasgo semántico +/–humano, como ilustran los ejemplos en \REF{ex:mick:6}:


\ea\label{ex:mick:6}
\ea\label{ex:mick:6a} OD f/sg/–h: \textbf{\textit{la} \textbf{educación} \textbf{intercultural} \textbf{bilingüe}} \textit{en artículo Ø estamos haciendo aprobar ɽeCIÉN a nivel nacional} (Ruth\_2)
\ex\label{ex:mick:6b} OD f/pl/–h: \textbf{\textit{Esas} \textbf{cosas}} \textit{Ø queremos} (Ruth\_2)
\ex\label{ex:mick:6c} OD m/sg/+h: \textit{por ejemplo} \textbf{\textit{mi} \textbf{esposo}} \textit{yo: cuánta mal el día miércoles como hoy día. le=he ʎevado a la posta. y:: no: este ↑no: nos no nos ha dado NA'↑da. y ya: me Ø he ʎevado a [la ciudad]} (Eva)
\ex\label{ex:mick:6d} OD m/pl/+h: \textit{acá en la localidad […] a to:s Ø jʰuntamos} (Eva)
\ex\label{ex:mick:6e} OI m/sg/+h, OD f/sg/–h: \textbf{\textit{ese} \textbf{agua} \textbf{potable}} \textit{habíamos pedido hasta a [APELLIDO] todavía pe y {\textrangle}{\textrangle}f{\textlangle}nada{\textlangle} de venir nada; a: a:} \textbf{\textit{al} \textbf{alcalde}} \textit{anteriOɽ todavía Ø hemos pedido} (Eva)
\z
\z

Los casos de omisión del pronombre de OI, como en el ejemplo \REF{ex:mick:6e}, son escasos en el corpus.

Paralelamente existe una tendencia leísta en el corpus, que – contrariamente al \scalebox{.9725}[1.0]{leísmo documentado en variedades peninsulares del español} \citep{Fernández-Ordóñez1999} – puede afectar a referentes de OD de cualquier género y número, como ilustran los ejemplos en \REF{ex:mick:7}: % The scalebox here is a microtypo adjustment


\ea\label{ex:mick:7}
\ea\label{ex:mick:7a} OD f/sg/+h: \textit{mi} \textbf{\textit{hermana}} \textit{se moría si no metíamos plantas prima. estaba que le supuraba TO:ða la herId. hemos mandAo tres hierbas de la: selva? de una hermana naturista. báñale con eso. en tres días sani'ta la operación} (Ruth\_2)
\ex\label{ex:mick:7b} ODn: \textit{y} \textbf{\textit{eso}} \textit{es lo que no: no se’ (.) no le entienden. no?} (Liz)
\ex\label{ex:mick:7c} OD m/pl/+h: \textit{los yach'} \textbf{\textit{los} \textbf{ancianos}} \textit{que le llamamos yachachis ante el estado ante el ministerio de cultura}. (Ruth\_4)
\ex\label{ex:mick:7d} OD f/sg/–h: \textit{yo he propuesto}\textbf{, \textbf{una} (.) \textbf{tercera} \textbf{variable}} \textit{para evaluar la viabilidA de proyectos de inversión extractivos, (-) y: le llamo la viabilidA socioambiental}. (Denis)
\z
\z

El verbo que más se usa con OI en el corpus es \textit{consultar} (13 usos leístas), seguido por otros verbos de uso leísta en parte ya documentados como tales por la literatura como \textit{llamar} (8 usos leístas en el corpus), \textit{informar} (7), \textit{ayudar} (6), \textit{afectar} (5), \textit{apoyar} y \textit{pagar} (4) \citep{Fernández-Ordóñez1999}. Se contabilizaron los sintagmas con estos verbos si bien algunos son de doble régimen, ya que las preferencias pronominales en estos casos pueden confirmar tendencias existentes.

En algunos de los enunciados aquí analizados, los pronombres átonos \textit{lo} y \textit{los} pueden también referirse a OD de género gramatical femenino. Mientras que la neutralización de género con referentes de OD no-humanos (\REF{ex:mick:8a}: no-animado, \REF{ex:mick:8b}: animado) es relativamente frecuente en el corpus, es excepcional con referentes que llevan el rasgo semántico +humano \REF{ex:mick:8c}, tal y como lo confirman los estudios mencionados en el apartado §\ref{sec:mick:2}. Probablemente, en este último ejemplo, el hablante acepta la neutralización de género (contrariamente a la reparación del mismo hablante en el ejemplo \REF{ex:mick:4} porque se refiere a un colectivo humano:


\ea\label{ex:mick:8} {Ejemplos de neutralización de género gramatical}
\ea\label{ex:mick:8a} \textit{esa}\textbf{ \textbf{información}} \textit{{\textrangle}{\textrangle}h{\textlangle}ni lo habrá leído; lo habrá hecho avionci:to así pues;{\textlangle}} (Ruth\_2)
\ex\label{ex:mick:8b} Ruth\_7: \textit{entonces} \textbf{\textit{las} \textbf{alpacas}} \textit{ya no se acercaban a tomar {\textrangle}{\textrangle}len{\textlangle}el agua,{\textlangle} a la la↑guna. porque (.) co ɽen despavoridos.}

Héctor: \textit{lo han corrido. lo han corrido.} 
\ex\label{ex:mick:8c} \textbf{\textit{a=la:} \textbf{a} \textbf{comunidad}} \textit{lo van dejando de lao} (Alberto)
\z
\z

El caso opuesto, el uso de \textit{la/s} para referentes de OD de género masculino, es excepcional en el corpus.

Otras tendencias excepcionales en el corpus incluyen el uso de \textit{lo/s} o \textit{la/s} para referentes humanos de OI, como en los ejemplos \REF{ex:mick:9} y \REF{ex:mick:10}:


\ea\label{ex:mick:9} {Ejemplo de loísmo: \textit{a quién le llega el canon minero. es a la} }
{\textit{municipalidad. […] no los llega} \textbf{\textit{a} \textbf{la} \textbf{comunidA}}. (Ruth\_2)}
\z

\ea\label{ex:mick:10} Ejemplo único de laísmo: \textit{y como yo exigía a la jefe de cáritas y y: a: por la exigencia vino} \textbf{\textit{el} \textbf{proyectista}}\textit{. de cáritas. tonces conversa:mos la digo mira:} (Clara)
\z

La gran mayoría de hablantes explotan de manera flexible las posibilidades del paradigma pronominal descritas arriba y resumidas en la tabla \ref{tab:mick:1}. Las siguientes secciones discuten la procedencia (\sectref{sec:mick:5}), la movilidad e intensidad del contacto (\sectref{sec:mick:6}) así como las condiciones de enunciación (\sectref{sec:mick:7}) como parámetros que podrían favorecer una u otra de las tendencias.

\section{Comparación de los usos pronominales por los hablantes}\label{sec:mick:5}


La tabla \ref{tab:mick:3} compara estadísticamente las tendencias en el uso de los pronombres átonos de 3\textsuperscript{a} persona, entre hablantes provenientes de la zona andina (grupo \textit{Andes}), hablantes limeños con contacto intenso con hablantes andinos (grupo \textit{Lima+}), y hablantes sin contacto particular con la zona andina (grupo \textit{Lima–}). Se observa que las cifras de \textit{Lima+} solo se sitúan entre las de los otros dos grupos en los dos primeros parámetros (sistema etimológico, omisión).

\begin{table}
\small
\caption{\label{tab:mick:3} Comparación global de los usos pronominales entre hablantes con o sin contacto más o menos intenso con la región andina.\\Esta tabla calcula los porcentajes por fenómeno aplicando las bases más pertinentes para cada factor. Los cálculos con base en todos los casos de pronombres contados se presentan en la tabla \ref{tab:mick:4}.1 en los anexos y se discuten en la sección §\ref{sec:mick:5.6}.}
\begin{tabularx}{\textwidth}{Qrrr}
\lsptoprule
Sistema pronominal & Andes & Lima+ & Lima–\\
\midrule
Legitimación “etimológica” de pronombre, por casos contados & 62.0\% & 79.0\% & 87.9\%\\
\tablevspace
\multirow{2}{5cm}{Omisión del pronombre, por casos contados} & { 20.9\%}        &  { 2.4\%}                & { 1.4\%}      \\
                                          & { (OD: 31.8\%)}  &   { (OD: 3.4\%)}        & { (OD: 2.1\%)}\\
                                          & (OI: 2.8\%) &        (OI: 0\%)             &     (OI: 0\%)\\
\tablevspace
Neutralización de género: porcentaje de ODf reemplazados por \textit{lo/s} & 22.8\% & 33.7\% & 12.9\%\\
\tablevspace
Loísmo: porcentaje de OI reemplazados por \textit{lo/s}& { 1.8\%}  &{ 2.0\%}    &     { 0\%}    \\
\tablevspace
Laísmo: porcentaje de OI reemplazados por \textit{la/s} &   0.5\% &  0\%        &       0.7\%\\
\tablevspace
Leísmo: porcentaje de OD reemplazados por \textit{le/s} & 13.6\% & 4.2\% & 5.0\%\\
\tablevspace
\multirow{2}{5cm}{Neutralización de número por casos contados} & { 3.7\%}          & { 8.5\%}     &   { 5.4\%}      \\
                                            &    { (OD: 4.2 \%)}&{ (OD: 5.4\%)}&   { (OD: 2.7\%)}  \\
                                            &    (OI: 2.8\%)   & (OI: 16.3\%)  &   (OI: 11.5\%)\\
\lspbottomrule
\end{tabularx}
\end{table}

En los tres grupos, el sistema etimológico y la omisión explican respectivamente el 81.4\% (\textit{Lima +}), el 82.9\% (\textit{Andes}) y el 89.3\% (\textit{Lima–}) de los pronombres encontrados. Los otros fenómenos (leísmo, loísmo, laísmo, neutralización de número o de género) solamente afectan el 18.6\% (\textit{Lima+}), 17.1\% (\textit{Andes}) y 10.7\% (\textit{Lima–}) de los casos. Las líneas 4 a 7 de la tabla \ref{tab:mick:3} explican los detalles de estos fenómenos más específicos y marginales.

El grupo \textit{Lima+} reproduce el sistema etimológico en el 79.0\% de los pronombres, frecuencia algo inferior a la de los limeños sin contacto particular con hablantes andinos (87.9\%). Se acerca en este aspecto ligeramente a los hablantes de la zona andina, para los que el sistema etimológico explica el 62\% de los pronombres.

\subsection{La omisión del pronombre}

En el grupo \textit{Andes}, más de la mitad (57.1\%) de los pronombres escogidos cuyo uso no se rige por el sistema etimológico se explica por la omisión del pronombre átono de 3\textsuperscript{a} persona; en el 20.9\% de todos los casos, los hablantes reemplazan el pronombre por el morfema cero (Ø). En este grupo, este último afecta el 31.8\% de los pronombres átonos de 3\textsuperscript{a} persona con referentes de OD, y en raras ocasiones también de OI (2.8\%). También con respecto a este parámetro, el grupo \textit{Lima+} se acerca ligeramente a las tendencias andinas, usando un poco más la omisión (2.4\%) que los limeños sin contacto particular con la zona andina (\textit{Lima–}: 1.4\%). Sin embargo, el uso del morfema cero afecta casi exclusivamente el OD en ambos grupos de hablantes limeños. 

Como detalla la tabla \ref{tab:mick:8} en los anexos, en los tres grupos, la omisión es más frecuente en el singular que en el plural, tanto en el OD como en el OI, pero en ambos grupos de hablantes limeños, la omisión con referentes en plural se acerca al 0\%, mientras que ocurre en un considerable 18.0\% de los casos en el grupo \textit{Andes}. Aunque en el singular, el grupo \textit{Lima+} presenta casi dos veces más omisiones que el grupo \textit{Lima}– (\textit{Lima+}: 3.0\%; \textit{Lima–}: 1.6\%), acercándose así tendencialmente a los hablantes andinos, los hablantes de este grupo mantienen el pronombre cuando se trata de referentes en plural, igual que sus pares limeños sin contacto particular.

Se observan diferencias interesantes con respecto al uso de Ø con referentes de diferentes géneros gramaticales: los hablantes de procedencia andina tienen una ligera preferencia a eliminar el pronombre cuando el referente es femenino (33.7\%, frente a 31.0\% con referentes de OD masculinos); y esto también se aplica, aunque en mucho menor medida, a los hablantes limeños sin contacto particular (3.2\% de omisiones con referentes de OD femenino, frente a 1.8\% en el OD masculino). Los limeños con contacto intenso reproducen estas relaciones solamente en el plural, omitiendo el pronombre en 6.7\% de los referentes de OD femeninos en plural (frente al 0\% en el caso del OD masculino plural), pero sólo en 1.5\% de los referentes de OD femeninos en singular (frente al 4.6\% en el caso de referentes de OD masculinos en singular). Si no se considera el factor del número gramatical del referente, sin embargo, los hablantes del grupo \textit{Lima+} omiten el pronombre de ODs femeninos en la misma proporción que con ODs masculinos (3.1\% y 3.5\%, respectivamente). Considerando únicamente la dimensión del género del referente de OD, se nota que los hablantes limeños (\textit{Lima}–) analizados no se distinguen casi en la tasa de omisión de OD femenino (3.2\%), pero sí en el OD de género gramatical masculino: omiten el pronombre con sólo el 1.8\% de los referentes de OD masculinos. Esto parece indicar que el contacto intenso con hablantes provenientes de la zona andina induce a los hablantes de procedencia y/o tradición cultural limeña a que equilibren las tasas de omisión del OD independientemente del género gramatical de este último. Incluso parece que estuvieran ligeramente exagerando esta tendencia, ya que la relación entre las tasas de omisión con referentes de OD de género femenino versus masculino es aproximadamente del 2.7\% en el grupo de los hablantes andinos, frente al 0.4\% en el grupo \textit{Lima+} y el 1.4\% en el grupo \textit{Lima–}.

\subsection{Neutralización de número}

En muchos casos en el corpus, el pronombre usado no coincide en el número gramatical con su referente. Como se aprecia en la tabla \ref{tab:mick:3}, los hablantes de procedencia andina son los que más marcan el número gramatical del referente en el pronombre, con la excepción del 3.7\% solamente. Esta tasa es del 5.4\% en el grupo \textit{Lima–}, y se eleva al 8.5\% en el grupo \textit{Lima+}. En todos los hablantes limeños, el pronombre de OI es el más afectado por este fenómeno. Las elevadas tasas de neutralización de número en los hablantes limeños podrían explicarse por una particularidad fonológica: como bien lo documentan varios estudios sociolingüísticos anteriores, los hablantes limeños tienden a la aspiración y elisión de la -s final, sobre todo en contextos pre-consonánticos \citep{Caravedo1990}.

\subsection{Neutralización de género}
\largerpage
La neutralización de género es el cuarto fenómeno más destacable en el corpus. El grupo \textit{Lima+} en particular se refiere con \textit{lo/s} al 33.7\% de los referentes de OD femeninos (ver tabla \ref{tab:mick:3}). Sorprendentemente, esta cifra es más elevada que la de los otros hablantes, inclusive los andinos (22.8\%), para los cuales la literatura documenta la neutralización de género en el pronombre átono de 3\textsuperscript{a} persona, o el desarrollo de un archimorfema \textit{lo}.

Esta última hipótesis de la existencia de un archimorfema \textit{lo} no se confirma en el presente corpus: cuando el grupo \textit{Andes} realiza un \textit{lo} como pronombre átono de 3\textsuperscript{a} persona, este se refiere en un 84.0\% a un OD masculino singular, y con esta tasa se acerca a los hablantes limeños con contacto intenso (\textit{Lima+}: 85.3\%); los discursos del grupo \textit{Lima–} reservan el uso de \textit{lo} casi exclusivamente a referentes de OD masculinos en singular (95.7\%). Y el 63.0\% (\textit{Andes}), 91.8\% (\textit{Lima+}) y 95.7\% (\textit{Lima–}) de los referentes de OD masculino singular son reemplazados por el pronombre \textit{lo} en el corpus. Ninguno de estos tres grupos confirma la generalización del uso de \textit{lo}.

Sin embargo, los hablantes de origen andino son prácticamente los únicos en usar la pronominalización catafórica del OD, fenómeno que efectivamente realizan casi exclusivamente con \textit{lo}. La pronominalización catafórica es un fenómeno marginal en el corpus. Solamente aparece en las entrevistas con 4 hablantes andinas (Carmen, Paola, Ruth\_4, Ruth\_5, Clara) – lo cual equivale al 22.7\% de los casos andinos estudiados –, y un hablante de origen limeño con contacto intenso (Juan) – lo cual corresponde al 16.7\% del grupo \textit{Lima+}. En total son 17 enunciados los que presentan este fenómeno, de los cuales 2 provienen del hablante limeño con contacto intenso, Juan. Es decir que, en promedio, las hablantes andinas producen 3 pronominalizaciones, frente a las 2 de Juan (\textit{Lima+}). Así pues, en este rasgo también se documenta un ligero acercamiento entre los hablantes limeños con contacto intenso y los hablantes andinos grabados.

Se notan diferencias en cuanto al uso de los pronombres átonos \textit{la} y \textit{las} en el corpus: en 6 de las 22 participaciones documentadas en la zona andina, no aparece \textit{la/s}, aunque sí aparecen referentes de OD femeninos (Ruth\_7, Lea\_2, Paola, Ruth\_2, Eva, María). Los ejemplos son probablemente demasiado escasos para comprobar si existen o no los pronombres \textit{la} y \textit{las} en sus idiolectos respectivos. Las diferencias entre las distintas entrevistas realizadas con Ruth y Lea muestran que que una sola entrevista no basta para caracterizar un repertorio pronominal: en la entrevista con el presidente de la comunidad, Alberto, que tiende ligeramente a neutralizar el género gramatical del referente, ambas hablantes ocultan el pronombre femenino que sí usan en otros contextos; y Ruth reproduce esta estrategia en la entrevista con el alcalde, en la que Lea prácticamente no interviene.

\subsection{Leísmo}

Solamente en una minoría de pronombres en el corpus no se reproduce el caso del referente.\footnote{Es oportuno reiterar aquí la observación mencionada en la parte metodológica: para la categorización de los verbos como in/transitivos en sus acepciones particulares en cada contexto de enunciación nos apoyamos en la versión en línea del DRAE; por lo tanto, las tasas de leísmo aquí calculadas son ciertamente mucho mayores que las incongruencias que percibiría un/a hablante promedio en el Perú.} De las tres tendencias existentes (leísmo, loísmo, laísmo), el leísmo es el más frecuente, es decir, el uso de \textit{le/s} para referentes de OD. Como indica la tabla \ref{tab:mick:3}, este fenómeno es aproximadamente tres veces más común entre los hablantes de procedencia andina que los de procedencia limeña (13.6\%), y menos frecuente en los limeños con contacto particular que sin él (4.2\% vs 5.0\%).

Como resume la tabla \ref{tab:mick:3} en los anexos, se confirma en los tres grupos que es más probable el leísmo con referentes humanos, y esto ocurre en mayor medida entre los hablantes andinos (el 91.8\% de sus leísmos implican un referente de OD \textit{+}humano) que entre los limeños (\textit{Lima+}: 80.0\%; \textit{Lima–}; 70.6\%). No en todos los grupos se confirma, sin embargo, la hipótesis de una preferencia del leísmo con referentes de OD masculinos. El grupo de hablantes de procedencia andina es el que más rompe con esta hipótesis, ya que 61.2\% de los pronombres \textit{le/s} refieren a un OD de género gramatical femenino. También es el caso de los hablantes limeños sin contacto particular con la zona andina, donde es más frecuente el leísmo con referentes de género gramatical femenino (\textit{Lima–}: 52.9\%) que los de género masculino. Los hablantes de procedencia limeña tienden al leísmo con referentes masculinos, pero dos hablantes de este grupo también usan \textit{le/s} para referirse a OD genéricos (género gramatical “neutro”), como ilustra el ejemplo \REF{ex:mick:7b} (Martín, Liz). Los hablantes procedentes de la zona andina y los del grupo \textit{Lima+} tienden más al leísmo con referentes de OD en plural que con sus equivalentes en singular: el 73.5\% de los leísmos producidos por los interlocutores de la zona andina conciernen referentes de OD en plural, mientras que esta tasa es del 53.3\% en el grupo \textit{Lima+}. En las entrevistas con hablantes limeños sin contacto particular con actores de la zona andina, la relación es inversa, ya que sólo el 29.4\% de los referentes de OD sustituidos por \textit{le/s} están en plural.

\subsection{Loísmo, laísmo}
El loísmo y el laísmo son fenómenos marginales en el corpus aquí estudiado. 

El loísmo, es decir el uso de \textit{lo/s} para referentes de OI, es poco frecuente o está completamente ausente en los tres grupos de actores (ver tabla \ref{tab:mick:3}): el grupo \textit{Lima+} es el que más presenta este rasgo (2.0\%), la tasa de loísmo de los hablantes de procedencia andina se acerca a la de estos primeros (1.8\%), mientras que este rasgo está completamente ausente entre los hablantes limeños sin contacto (0\%). 3 personas de procedencia andina (Clara, Ruth\_2, Alberto) producen un total de 4 ocurrencias loístas, frente a una ocurrencia por entrevista en 3 de las 6 grabaciones con hablantes del grupo \textit{Lima+} (Sara, Martín, Denis). El loísmo afecta casi exclusivamente a los referentes de OI en plural, salvo en el idiolecto de Alberto.

Sólo dos hablantes usan el pronombre femenino para referentes de OI (laísmo), y ambas son mujeres: una hablante, periodista en un periódico de difusión nacional, es de procedencia limeña y no tiene contacto particular con la zona andina. Mientras en su caso el referente de OI remite a colectivos humanos de género gramatical femenino (\textit{que las respeten sus derechos [a las comunidades]}), Clara presenta un caso de laísmo con un referente humano masculino (ver \REF{ex:mick:10}).\footnote{En el caso de Clara, quizá influya el hecho de que este caso de laísmo ocurre no al inicio de la conversación, pero sí al empezar la grabación.} Aunque en ambos casos se cumple la supuesta regla de que el laísmo ocurre preferentemente con referentes humanos, es sorprendente que en el caso de la periodista sea un referente en plural, y que en el caso de Clara sea un referente masculino. No hay ejemplos de laísmo en el grupo \textit{Lima+}.

\subsection{Discusión}\label{sec:mick:5.6}

En el corpus se confirma como tendencia general del conjunto de actores el uso del llamado sistema “etimológico” como referencia principal, y la presencia muy marginal de los fenómenos de loísmo y laísmo. 

A grandes rasgos, el corpus aquí analizado actualiza el paradigma de sistemas pronominales átonos de 3\textsuperscript{a} persona descrito por la literatura para el español en Lima y los Andes peruanos: mientras los actores de procedencia y/o tradición limeña tienden hacia el sistema “etimológico”, los hablantes que más se acercan geográfica y culturalmente a las zonas rurales andinas desarrollan sistemas pronominales más independientes de las restricciones de género, número y, en menor medida, del caso gramatical.

El grupo \textit{Lima+}, curiosamente, se aleja tendencialmente del sistema etimológico en el uso pronominal, pero no por ello se aleja claramente del grupo de hablantes limeños ni se acercara globalmente al grupo andino, sino que, más bien, desarrolla un sistema propio. La tabla \ref{tab:mick:4} resume las tendencias en el uso pronominal según su orden de importancia en cada uno de los tres grupos.

\begin{table}
\caption{\label{tab:mick:4} Comparación de las tendencias en el uso pronominal en los tres grupos según orden de frecuencia.\\Se indica según orden de importancia las tendencias más destacadas del uso pronominal en cada grupo (1 = tendencia predominante). La tabla \ref{tab:mick:4.1} en los anexos detalla los porcentajes respectivos.}

\begin{tabular}{|l|c|c|c|}
\hline
 &  {\textbf{\textit{Andes}}} &  {\textbf{\textit{Lima+}}} &  {\textbf{\textit{Lima–}}}\\
\hline
1 & \multicolumn{3}{c|}{ sistema “etimológico”}\\
\hline
2 & omisión & \multicolumn{2}{c|}{ neutralización de número}\\
\hline
3 & leísmo & neutr. de género & leísmo\\
\hline
4 & neutr. de género & leísmo & neutr. de género\\
\hline
5 & neutr. de número & \multicolumn{2}{c|}{ omisión}\\
\hline
6 & \multicolumn{3}{c|}{ \textit{loísmo/laísmo}}\\
\hline
\end{tabular}
\end{table}

Es únicamente en el tercer y cuarto factor donde se distinguen entre sí los dos grupos de limeños, ya que los hablantes del grupo \textit{Lima+} aceptan más la neutralización de género que el grupo \textit{Lima}–. De esta forma, se alejan también del grupo de los hablantes andinos, en cuyos usos pronominales las tendencias al leísmo y a la neutralización de género tienen el mismo rango que en el grupo \textit{Lima–}. Es en la importancia de la omisión y de la neutralización de número donde los hablantes andinos grabados se distinguen más fundamentalmente de los dos grupos limeños. Es cierto que el grupo \textit{Lima+} usa un poco más la omisión que el promedio del grupo \textit{Lima–}, pero el fenómeno sigue siendo marginal en su habla. Eventualmente, este resultado cambiaría si el conteo tomara en consideración la omisión en el caso de verbos transitivos que se usan sin ningún complemento de OD explícito. 

Seguramente cambiaría el rango de la neutralización de número en los sistemas pronominales de los hablantes limeños si se tomaran en cuenta criterios fonológicos al contar los casos de no marcación del plural. En muchos casos, la omisión de la -s final podría estar motivada por constricciones contextuales fonológicas (ver \citet{Caravedo1990}), no por particularidades sintácticas.

No obstante, llama la atención que los hablantes limeños con contacto intenso con actores de la zona andina se transformen en promotores más fervientes de la neutralización de número que los actores procedentes de la zona andina. Los estudios existentes sugieren que los hablantes andinos en general estarían más familiarizados con este fenómeno, que se percibe como particularmente marcado en Lima, según \citet{Caravedo2014}. 

\section{Influencia de la movilidad inter/nacional en los sistemas pronominales de actores andinos y limeños con contacto intenso}\label{sec:mick:6}

El presente apartado analiza de manera más detallada la variación en el uso pronominal en los dos grupos de hablantes más relacionados con la zona andina: los actores grabados provenientes de la zona andina (\sectref{sec:mick:6.2}) y los hablantes de Lima con contacto intenso (\sectref{sec:mick:6.1}).

\subsection{Movilidad inter/nacional y sistemas pronominales en el grupo \textit{Lima+}}\label{sec:mick:6.1}

Esta sección analiza más en detalle la variación de los sistemas pronominales dentro del grupo de los hablantes limeños con contacto intenso, resumidos en la tabla \ref{tab:mick:5}. Como en la tabla \ref{tab:mick:3}, el cálculo de los porcentajes escoge la base más pertinente para cada parámetro, por lo cual la suma de los promedios no corresponde al 100\%.

\begin{table}
\caption{\label{tab:mick:5} Variación en el uso de pronombres átonos en el grupo \textit{Lima+}}

\begin{tabularx}{\textwidth}{Xrrr}
\lsptoprule
& Mínimo & Máximo & Promedio\\
\midrule
“Sistema etimológico” por todos los casos & 74.8\% & 99.4\% & 79.0\%\\
Omisiones por todos los casos & 0\% & 5.0\% & 2.4\%\\
Neutr. de género por ODf & 0\% & 77.3\% & 33.7\%\\
Neutr. de número por casos & 5.0\% & 15.9\% & 8.5\%\\
leísmo por número de OD & 0\% & 7.1\% & 4.2\%\\
loísmo por número de OI & 0\%     & { 5.0\%} &  { 2.0\%}      \\
laísmo por número de OI & { 0\%}  &    0\% &                   0\%\\
\lspbottomrule
\end{tabularx}
\end{table}

De acuerdo a las posiciones que ocupa cada una/o de los hablantes en cuanto a todos estos parámetros, el grupo \textit{Lima+} se subdivide en tres: mientras que Jimena y Juan son los actores entrevistados que más se acercan al sistema “etimológico”, Sara muestra el sistema pronominal más independiente de éste, y los otros tres actores se sitúan en el intermedio superior (Martín, Liz, Denis).

Las posiciones de los actores en cuanto a los primeros tres factores considerados coinciden aproximadamente: mientras más se acercan al sistema “etimológico” los hablantes del grupo \textit{Lima+}, menos tienden a omitir pronombres o a neutralizar el género. Juan es el hablante que más se alinea al sistema etimológico en estos tres aspectos, seguido por Jimena. El sistema pronominal más independiente del sistema etimológico en estos tres aspectos lo desarrolla la hablante Sara, seguida por Liz. La variación del género del pronombre con respecto al referente -- el factor que \citet{Caravedo2014} considera como más marcado desde la perspectiva limeña -- es la que presenta la variación más importante en este grupo de actores: mientras que Juan generalmente respeta el género de los 8 referentes de OD femeninos (9.1\% en Jimena), Liz escoge el pronombre masculino en el 77.3\% de 22 referentes de OD femeninos, seguida por Sara (31.3\%). 

En los dos parámetros que se refieren al caso gramatical del referente en el pronombre átono de 3\textsuperscript{a} persona, las posiciones cambian: Jimena sigue siendo la hablante del grupo \textit{Lima+} que siempre marca el caso gramatical del referente al escoger el pronombre (0\% de leísmo y 0\% de lo/laísmo). Le sigue Juan, quien, no obstante, se encuentra en el segundo lugar del grupo \textit{Lima+} en lo que se refiere a su tendencia al leísmo (6.0\%). En cuanto al caso gramatical, Liz y Sara se posicionan en el medio del grupo \textit{Lima}+, aunque Liz es la hablante que más tiende a generalizar el uso de \textit{le/s} con referentes de OD (7.1\%), y Sara es la segunda hablante que produce más loísmos con referentes de OI (4.3\%). Martín es el hablante que menos marca el caso gramatical del referente, reemplazando el 5.0\% de los referentes de OI con los pronombres \textit{lo/s}, y el 3.3\% de los OD por \textit{lo/s}. Le sigue Denis, cuyas tendencias al leísmo (4.8\%) y loísmo (2.7\%) representan aproximadamente los promedios del grupo \textit{Lima}+. 

La tabla \ref{tab:mick:6} presenta los sistemas pronominales de las y los hablantes de manera comparativa, tomando solamente en consideración las tendencias excepcionales en el corpus. Es decir que la tabla indica el uso de los pronombres que más distinguen a cada hablante individual con respecto al promedio de hablantes del mismo grupo \textit{Lima}+. Para interpretar la tabla es importante tener en cuenta que, en todos los hablantes de este grupo, el sistema etimológico es el que explica por lo menos el 79.0\% de los usos de pronombres átonos. Conforme bajan las líneas de la tabla, los sistemas pronominales individuales tienden a independizarse de este sistema.

\begin{table}
\caption{\label{tab:mick:6} Tendencias distintivas en los sistemas pronominales de los hablantes del grupo \textit{Lima+}}
\begin{tabularx}{\textwidth}{lCCCCCCC}
\lsptoprule
& \multicolumn{3}{c}{ \textbf{OD} \textbf{sg}} & \multicolumn{2}{c}{ \textbf{OD} \textbf{pl}} & \multicolumn{2}{c}{ \textbf{OI}}\\
\cmidrule(lr){2-4}\cmidrule(lr){5-6}\cmidrule(lr){7-8}
& \textbf{m} & \textbf{f} & \textbf{n} & \textbf{m} & \textbf{f} & \textbf{sg} & \textbf{pl}\\
\midrule
 Jimena & lo & la & lo & los & { las}\newline  la & le & { les}\newline  le\\
 \tablevspace
 Juan & lo & la & lo & los & las & le & les\\
&  &  &  &  les (+h)  & les (+h)  & \\
 \tablevspace
 Martín & lo & { la}\newline  (las) & { lo}\newline  (le) & los & { las}\newline  (los) & { le} & { les/le}\newline  los (+h)\\
 \tablevspace
 Denis & lo & { la}\newline  le (+h) & { lo}\newline  Ø & los & las & le & les\\
&  &  &  &  les (+h) & les (+h)  & \\
 \tablevspace
 Liz & { lo}\newline  le (+h) & lo & lo/le & los & las/le (+h) & le & les\\
 \tablevspace
 Sara & lo & la & lo & { lo}\newline  las & { los}\newline  lo/la & le/les & { le}\newline  (les)\\
\lspbottomrule
\end{tabularx}
\end{table}

De acuerdo al grado y a las características de la movilidad de los actores, el grupo denominado \textit{Lima+} se puede subdividir en dos: 

\begin{itemize}
\item Tres hablantes vivieron prácticamente toda su vida en Lima, pero además de sus intensas interacciones laborales con actores de la zona andina, tienen vínculos familiares en la zona andina (Jimena) o nacieron allí (Liz, Sara). Jimena trabaja en una institución nacional, Liz en una organización gubernamental internacional y Sara en un instituto de investigación limeño inserto en una red internacional.
\item La vida de los otros tres actores se desarrolla geográfica y socialmente en Lima y con limeños, pero mantiene vínculos laborales más o menos intensos con la zona andina: Martín trabajó y vivió muchos años en la región de Cajamarca, los proyectos de la organización no-gubernamental que dirige Juan se desarrollan en la zona andina, y Denis trabajaba y trabaja en diferentes puestos en empresas mineras cuyas actividades principales se desarrollan en la región andina.
\end{itemize}

\largerpage
Los usos pronominales de las migrantes internas de primera generación, Sara y Liz, tienden efectivamente a autonomizarse un poco más del sistema etimológico que los de los otros hablantes. Jimena, sin embargo, pariente de migrantes, es la que más se acerca a este sistema. Mientras Liz tiende más al leísmo, Sara tiende a neutralizar la dimensión del género. Ambas actoras son bastante móviles a nivel nacional e internacional; sin embargo, mientras la actividad profesional principal de Sara se concentra en la zona andina peruana, Liz se desempeña profesionalmente en la zona andina de Latinoamérica en general. Esto podría explicar su pronunciada tendencia al leísmo – un fenómeno que existe en variedades del español andino peruano, pero que es menos frecuente que en otras regiones (por ejemplo, Ecuador, véase \citealt{KleeLynch2009book}).

El segundo grupo de hablantes – hombres con experiencias profesionales comparables en cuanto a la movilidad – se acerca al sistema etimológico. No obstante, Juan y Denis lo combinan con tendencias considerables al leísmo, mientras que hay tendencias loístas en el idiolecto de Martín. El leísmo de Juan y Denis, sin embargo, parece ser de otra índole que el de Liz, puesto que se concentra en los referentes femeninos o plurales, en vez de singulares y masculinos/neutros. El único otro caso de este leísmo extraordinario con un referente neutro lo produce Martín, quien, no obstante, tiene poca tendencia al leísmo. Por este hecho y el loísmo, se podría decir, entonces, que entre los actores del segundo subgrupo, Martín es el que más se acerca al primero. En la entrevista menciona experiencias de trabajo muy intensas con poblaciones rurales en la zona de Cajamarca durante el tiempo de su profesionalización como abogado, muy parecidas a las que sigue teniendo Sara.

Las observaciones se podrían resumir de la siguiente manera:

\begin{itemize}
\item Mientras menos móviles son geográfica y culturalmente los actores del grupo \textit{Lima}+, más se acercan al sistema etimológico (ver la hipótesis de García Tesoro, en \citealt{GarcíaTesoroMick2013}), con una diferencia que podría tener que ver con el género: los hombres relativamente menos móviles del grupo (Juan, Denis) demuestran una tendencia marcada al leísmo con referentes femeninos que no presenta Jimena.
\item Cuanto mayor sea la movilidad nacional que implique contactos intensos y prolongados con hablantes de la zona andina, más se exacerba la tendencia loísta en los discursos, como demuestran los ejemplos de Martín y Sara.
\item El ejemplo de Liz sostiene la siguiente hipótesis: con una mayor movilidad internacional en países de la zona andina se exacerba el leísmo con referentes de todos los géneros gramaticales, así como el uso de \textit{lo/s} para referentes de OD de género gramatical femenino, es decir, se neutraliza la distinción de género.
\end{itemize}

\subsection{Movilidad inter/nacional de actores en la zona andina}\label{sec:mick:6.2}

En el grupo de los hablantes de procedencia y/o tradición cultural andina, la variación individual en las tendencias analizadas es todavía más elevada que en el grupo \textit{Lima}+ (ver tabla \ref{tab:mick:7}).

\begin{table}
\caption{\label{tab:mick:7} Variación en el uso de pronombres átonos en el grupo de hablantes en los Andes.\\Entre paréntesis se indican las tasas en una entrevista en la cual solamente se contabilizaron 4 casos potenciales de pronombres átonos de 3\textsuperscript{a} persona.}
\begin{tabularx}{\textwidth}{Xrrr}
\lsptoprule
& \textbf{Mínimo} & \textbf{Máximo} & \textbf{Promedio}\\
\midrule
“Etimológico” por casos & (0\%) 29.0\% & 74.9\% & 62.0\%\\
Omisiones por casos & 3.4\% & (100\%) 81.0\% & 20.9\%\\
Neutr. de género, por ODf & 0\% & 100\% & 22.8\%\\
Neutr. de número, por casos & 0\% & 16.7\% & 3.7\%\\
leísmo por OD & 0\% & 24.6\% & 13.6\%\\
loísmo por OI  & 0\% & { 11.1\%}& { 1.8\%}\\
laísmo por OI & { 0\%} & 10.0\% & 0.5\%\\
\lspbottomrule
\end{tabularx}
\end{table}

El factor que mejor explica las divergencias del sistema etimológico en este grupo, que varían del 25.1\% al 81\%, es la omisión del pronombre átono de 3\textsuperscript{a} persona: 4 de los 6 hablantes con coincidencias con el sistema etimológico por encima del promedio del grupo andino también presentan tasas de omisión inferiores al promedio (Alberto, Lea, Héctor, Clara); y lo contrario aplica en el caso de 5 de los 6 hablantes con sistemas pronominales más autónomos frente al sistema etimológico que el promedio, que presentan también tasas de omisión más elevadas que el promedio (Flor, Eva, Carmen, Ana, María). Esta correspondencia solo se rompe en 3 hablantes (Ruth, Paola e Inés): mientras la primera se distancia más del sistema etimológico que el promedio del grupo de actores andinos, presenta una de las tasas de omisión más bajas del grupo. En su caso, el leísmo es otra tendencia que está bastante pronunciada. Paola e Inés aplican el sistema etimológico en un poco más de casos que el promedio, pero presentan tasas de omisión también ligeramente por encima del promedio. Existen también diferencias cualitativas en las omisiones producidas, ya que, en Eva, Alberto y Héctor, así como excepcionalmente en Lea, pueden afectar a referentes de OI, mientras que los otros actores sólo omiten el pronombre con el OD. 

Cabe resaltar que Alberto, Héctor y Lea también forman parte de los 6 hablantes con tasas de neutralización de género gramatical (\textit{lo/s} para referentes de OD femeninos) por encima del promedio, junto con Ruth, Clara y Ana. 


En los dos fenómenos que conciernen la relación entre el caso gramatical del pronombre átono de 3\textsuperscript{a} persona y el referente, pocos hablantes de este grupo se sitúan por encima del promedio del grupo andino: 3 hablantes (Alberto, Clara, Ruth) presentan loísmo y otros 3 (Héctor, Ruth, Carmen) presentan leísmo. En cuanto a la neutralización de número, 5 hablantes producen tasas más elevadas que el promedio de hablantes andinos grabados: Alberto, Héctor, Clara, Ruth, Ana. 

En resumen, estas observaciones permiten dividir los hablantes grabados en la zona andina en tres grupos:

\begin{itemize}
\item Un primer grupo de 4 actores usa los pronombres tendencialmente siguiendo el sistema etimológico, tiende poco a la omisión, pero presenta relativamente muchos casos de neutralización de género gramatical. Con la excepción de Héctor, este grupo tiende menos al leísmo que el promedio andino; con la excepción de Lea, este primer grupo tiende más a la neutralización de número que el promedio. La tendencia al loísmo de los otros 2 actores de este grupo, Alberto y Clara, es más marcada que en el promedio de los hablantes de procedencia andina.
\item En un segundo grupo de 4 actoras (Ana, María, Eva, Flor), la omisión del pronombre es la tendencia más destacada. En este grupo, sólo Ana tiende más a la neutralización de género y de número que el promedio.
\item Un tercer grupo de actoras es más heterogéneo en cuanto al uso de los pronombres: Inés y Paola se alinean tendencialmente al sistema etimológico, pero se acercan al segundo grupo descrito arriba por el uso relativamente frecuente del morfema cero (Ø) para referirse al OD. Ruth y Carmen aplican sistemas pronominales más independientes del sistema etimológico: Carmen también opta frecuentemente por la omisión pronominal en el OD (sus enunciados no permitieron analizar casos de OI), pero, además, junto con Ruth, presenta una tendencia más marcada al leísmo que el promedio de los actores andinos grabados. Ruth no suele usar frecuentemente el morfema cero, pero aún así presenta un sistema pronominal que se autonomiza de las restricciones de caso y número – pero no de género.
\end{itemize}

En este grupo, María es ciertamente la actora con mayor movilidad a nivel nacional e internacional, al ser la vicepresidenta de una de las organizaciones indígenas más importantes a nivel internacional, y al haber participado en una serie de formaciones organizadas por ONG internacionales en otros países, por ejemplo, en Europa. Las hablantes Ruth y Lea tienen bastante movilidad a nivel nacional, al ser las representantes regionales de una de las organizaciones indígenas más importantes a nivel nacional. Se puede considerar a Héctor y Alberto como socioculturalmente móviles, ya que, como autoridades locales – alcalde y presidente comunal respectivamente –, están familiarizados con la cultura institucional. Entre las otras actoras, Paola, Clara y Eva han ocupado u ocupan cargos de dirección en una organización local de mujeres. Ana, Carmen, Inés y Flor son las actoras más implicadas a nivel local, en cuanto comuneras que participan en la asamblea comunal, socias de la organización de mujeres y madres de hijos jóvenes.

Con la excepción de María, Ruth e Inés, se confirma la primera hipótesis establecida en la sección §\ref{sec:mick:6.1} con respecto al grupo de hablantes \textit{Lima}+: cuanto mayor la movilidad geográfica y cultural, más se acercan los sistemas pronominales adoptados al sistema etimológico. El análisis en la sección §\ref{sec:mick:6.2} nos permite, además, definir la omisión como el sistema pronominal favorecido por actores con mayor implicación local.

La segunda hipótesis -- según la cual el loísmo se ve favorecido por la movilidad nacional --, la comprueban Ruth, Alberto y Clara, pero no los otros actores móviles. El ejemplo de María destaca en el corpus, de acuerdo al análisis realizado. Aunque es relativamente móvil a nivel nacional e internacional -- no limitada a los países andinos --, mantiene un sistema pronominal con una tendencia a la omisión del pronombre, que caracteriza al grupo 2 de actoras más implicadas a nivel local. Se podría argumentar, pues, que los rasgos lingüísticos de su español están en coherencia con la función que desempeña a nivel nacional, como representante de poblaciones locales. Sin embargo, esto no aplica a otros representantes de poblaciones locales en diferentes instancias y niveles – como Lea, Alberto, Héctor y Ruth. En el caso de Lea, representante de los jóvenes de su región, el factor generacional podría influir, pero los datos no lo permiten comprobar. En los casos de los tres hablantes mayores, la presión normativa del español institucional podría ser decisiva: según Alberto, las municipalidades – y por consiguiente también los alcades como Héctor – no interactúan mucho con las poblaciones locales, sino que se concentran más en procesos institucionales internos. Los comentarios de Alberto y Ruth muestran que los representantes comunales, al contrario, tienen que mediar entre las instituciones y las comunidades para poder asegurar su influencia política. Esto podría explicar el sistema pronominal heterogéneo de Ruth que fue grabada en diferentes contextos conversacionales, y también justificaría las semejanzas entre el uso pronominal de Héctor y Alberto en la entrevista semi-formal.


\section{Influencia del contexto de interacción: Ruth y Lea}\label{sec:mick:7}

Esta última parte de análisis se interesa por la variación en los sistemas pronominales desarrollados por Ruth y Lea de acuerdo al contexto de interacción. El corpus da la ocasión de observar a ambas en 6 contextos comunicativos diferentes,\footnote{Ver la tabla \ref{tab:mick:9} en los anexos. Lea también estuvo presente en la conversación con Héctor, pero casi no participó activamente, por lo que no se pudieron caracterizar sus usos pronominales.} y sorprende la variación dentro de cada idiolecto de las hablantes: por ejemplo, Lea usa entre el 50.0\% (Lea\_1) y el 87.5\% (Lea\_3) de los pronombres átonos de 3\textsuperscript{a} persona según el sistema etimológico; en el caso de Ruth son entre el 50\% (Ruth\_4) y el 92.3\% (Ruth\_3). Sorprendió que ambas hablantes produjeran las tasas más bajas en conversación con la misma actora, Clara. Por falta de datos, el análisis trata los sistemas pronominales de los interlocutores de Ruth y Lea como constantes, sabiendo que ellos probablemente también demostrarían variaciones si hubieran sido grabados en diferentes contextos.

Una primera hipótesis es que el orden de las grabaciones podría influir en el uso pronominal, ya que las diferentes conversaciones se situaron a distancia geográfica cada vez mayor de la ciudad de residencia de Ruth y Lea y en ámbitos cada vez más rurales. Pero no hay evidencia suficiente para comprobarla. 

La segunda hipótesis, más probable, es que haya acercamiento o distanciamiento dependiendo de los interlocutores principales.

Los casos de neutralización de género producidos por Lea, por ejemplo, aumentan de acuerdo a la intensidad de esta tendencia en sus interlocutores: presenta su tasa más alta de neutralización de género (50\%) al hablar con Clara y Alberto, que usan \textit{lo/s} para respectivamente el 40.0\% y 33.3\% de los referentes de OD de género gramatical femenino. Sin embargo, Lea explicita el género de los referentes en el pronombre cuando habla con Eva y Flor o Inés y Paola, que todas también demuestran tasas bajas (del 0\% al 17.7\% ) de neutralización del género (Inés no usa ningún OD femenino). 

Sorprendentemente, la relación inversa se establece en el caso del leísmo: Lea produce las tasas más elevadas con las actoras locales que no presentan este rasgo (Lea\_4, Inés, Lea\_5, Lea\_1); mientras que no produce ningún leísmo en la entrevista con Clara, que sí presenta esta tendencia, aunque no muy marcada (9.1\%). 

En general, las omisiones no son muy características del sistema pronominal de Lea, pero las intensifica en las entrevistas con Inés, y con Eva y Flor, que son las que presentan este rasgo de forma más marcada en el corpus (33.0\%, 81.0\% y 100\% respectivamente). Sorprendentemente, Lea omite los pronombres con la misma frecuencia (25.0\%) en la entrevista con Alberto, quien casi no omite ningún pronombre (3.4\%). No obstante, solo en la entrevista con Alberto, Lea omite el pronombre con un referente OI, lo que es un rasgo distintivo del sistema pronominal de Alberto. 

Ruth, por su parte, parece servirse de manera estratégica del sistema pronominal para distanciarse o acercarse discursivamente de sus interlocutores. En la entrevista con Alberto, por ejemplo – que presenta el loísmo más marcado del corpus (11.1\%) –, ella también perfila su sistema loísta. Pero no lo actualiza con Clara que también presenta usos loístas (9.1\%). En otras entrevistas, las tasas de neutralización de género en el uso de los pronombres por Ruth suben hasta el 66.7\%; en la conversación con Clara, sin embargo – la hablante que más tiende a generalizar el uso de \textit{lo/s} independientemente del género del referente (40\%) –, Ruth no sustituye ninguno de los 3 referentes de OD femeninos por \textit{lo/s}. Las tasas de omisión del pronombre de Ruth varían del 0\% al 31.3\%, y se intensifican en las entrevistas con actoras locales que forman parte o se acercan al grupo 2 descrito en la sección §\ref{sec:mick:6.2}. (Ruth\_5, Ruth\_4, Ruth\_1). En la entrevista con Clara, que presenta una tasa de omisión del 16.7\%, sin embargo, Ruth no produce ningún caso de omisión.
Esta correlación sólo se rompe en el caso del leísmo en la entrevista con Alberto: Ruth presenta sus tasas más elevadas de leísmo en la cuarta (35.9\%) y primera conversación (26.9\%), entrevistas en las cuales participan 2 actoras  quienes más usan el leísmo. Héctor y Ruth\_7 tienen la misma tasa de leísmo (18.2\% en ambos). Alberto produce muy escasas ocurrencias de leísmo, pero en conversación con él, Ruth\_2 sustituye un considerable 20.6\% de referentes de OD por los pronombres \textit{le/s}. En resumen, Ruth parece sistemáticamente distanciarse de Clara mediante el sistema pronominal escogido; se acerca al sistema etimológico al hablar con Héctor y Alberto que forman parte del grupo 1 presentado en le sección §\ref{sec:mick:6.2}, con los cuales marca de manera más rigurosa el género de los pronombres con referentes de OD femeninos; y con Eva y Flor tanto como Ana y Carmen, representantes del grupo 2 caracterizado en la sección §\ref{sec:mick:6.2}., exacerba la omisión de pronombres.

En resumen, el análisis aquí presentado reconstruye cómo 2 actoras relativamente móviles entre los niveles local, regional y nacional, aprovechan la posibilidad que ofrece el amplio repertorio de sistemas pronominales átonos de 3\textsuperscript{a} persona para posicionarse lingüísticamente frente a sus interlocutores. Ruth tiende -- conscientemente o no -- a alinearse con sus interlocutores, con la excepción de Clara. En una ocasión grabada, Ruth critica frente a otras interlocutoras que Clara todavía tiene que aprender a defender los intereses de la organización. Esto lleva a suponer que existe cierta asimetría o competitividad entre estas actoras, que podría incitarlas a distinguir sus posiciones por medio de un distanciamiento lingüístico que se manifiesta con los sistemas pronominales. Sin embargo, con Alberto y Héctor, que representan las posiciones de poder locales tradicionales -- el alcalde por parte del Estado, y el presidente comunal por parte de la comunidad -- existe cierta presión para asegurarse de su solidaridad: Héctor es el que eventualmente podría dar acceso a fondos -- aunque escasos -- que vienen desde las instituciones; y la legitimidad del papel de Ruth en la organización indígena nacional depende de Alberto, así como de las comuneras socias de la organización de mujeres.

Lea presenta tendencias similares a las de su madre en cuanto a la alineación o al acercamiento de los usos pronominales al de sus interlocutoras, pero exacerba el leísmo, distanciándose de ellas. Puede ser una marca de diferencias generacionales o quizá también culturales, dado que su centro de vida es urbano. Con los 2 interlocutores hombres, sin embargo, no presenta el mismo tipo de acercamiento, lo cual podría ser una expresión de las asimetrías que siente. 

\section{Resumen}\label{sec:mick:8}

El análisis aquí desarrollado de la variación en los sistemas pronominales de actores más o menos afectados por la ley de CP demuestra dinámicas interesantes. Confirma las explicaciones tradicionales de la variación en el uso pronominal por procedencia geográfica y tradición cultural andina versus limeña, y la predominancia sociolingüística de la variedad limeña. Sin embargo, también muestra heterogeneidad en los sistemas pronominales de hablantes de procedencia tanto andina como limeña. Confirma la influencia del factor de la movilidad en la asimilación lingüística de los hablantes andinos al sistema etimológico. Además, la movilidad parece ser un factor relevante para el mantenimiento o la exacerbación de rasgos más independientes del sistema etimológico, en lo que podríamos llamar "factor de solidaridad" o de "identificación": los actores limeños que han tenido o tienen contactos más intensos con actores andinos, sobre todo experiencias de vida o experiencias emocionalmente impactantes, en su promedio, tienden a exacerbar la neutralización de género, incluso en mayor medida que los mismos actores andinos grabados. Solo una hablante andina destaca al presentar más rasgos considerados como “andinos” que el promedio de su grupo, y, curiosamente, se trata de la vicepresidenta de una de las organizaciones indígenas más influyentes a nivel nacional, que tiene un grado considerable de movilidad inter/nacional. Su caso es particularmente interesante para las políticas a favor de las lenguas indígenas y otros puntos de la agenda indígena: en una posición de poder más o menos asegurada, sin tener que sentir miedo por una posible discriminación, la hablante se transforma mediante su sistema pronominal en una promotora de la variedad sociolingüística de su región. Por su parte, los casos de Ruth y Lea dan más argumentos que sustentan la hipótesis de que los actores usan el posicionamiento lingüístico por medio del sistema pronominal de manera estratégica (no necesariamente consciente) para alcanzar sus fines comunicativos. El hilo rojo del corpus es la discusión sobre la Ley de CP. Los resultados aquí analizados en cuanto a la variación de los sistemas pronominales dan cuenta de que, efectivamente, es un marco que contribuye a visibilizar la heterogeneidad sociocultural, así como a democratizar el acceso a la palabra para actores minorizados. 

\section{Interpretación – Conclusión}\label{sec:mick:9}

El análisis aquí desarrollado de las variaciones de los sistemas pronominales átonos de una variedad del español en contacto con el quechua se inscribe en una larga tradición de estudios del mismo tipo, de los cuales se citó una parte a lo largo de los apartados §\ref{sec:mick:2} y §\ref{sec:mick:4}. Sin embargo, en el campo de la lingüística del contacto entre el español y las lenguas amerindias, estos estudios representan una minoría. Por lo tanto, en vista de la temática del presente volumen, es importante destacar aquí ciertos aspectos que parecen importantes en estas investigaciones, y que resuenan con las observaciones de \citet{Léglise2017}, \citet{Pennycook2006}, \citet{OtsujiPennycook2010},  \citet{GarcíaSylvan2011} y otros autores:

\begin{itemize}
\item Estos estudios analizan únicamente las dinámicas lingüísticas que se producen en el español, pero aun así contribuyen significativamente a la comprensión de las relaciones sociales, interacciones comunicativas, y dinámicas culturales en zonas de contacto entre el español y lenguas amerindias.
\item Estos trabajos no se limitan a estudiar el español de hablantes bilingües, sino que demuestran que los fenómenos de contacto conciernen a actores monolingües, e incluso los que no tienen contacto particular con hablantes bilingües. Por diglósica que sea la situación sociolingüística de la sociedad, los rasgos lingüísticos inducidos por contacto conciernen a los hablantes de toda la jerarquía social y cultural.
\item Al demostrar tal variación en el español mismo, y al visibilizar voces minorizadas en el sistema diglósico, contribuyen a desconstruir enfoques normativistas y prejuicios discriminadores, y demuestran incluso la existencia de caminos para promover dinámicas culturales de forma política.
\item Estos estudios también desconstruyen visiones lingüísticas puristas y segmentadas de los sistemas lingüísticos independientes. Demuestran que el contacto tiene la forma de un continuo cultural y lingüístico con un complejo de fenómenos de interacción heterogéneos. Para los hablantes en situaciones de contacto diglósico no se trata simplemente de optar por una u otra lengua, sino de posicionarse en este continuo de acuerdo a las características culturales y sociales de los interlocutores y los fines comunicativos respectivos.
\item En este último sentido, estos estudios también ponen de relieve las complejas micropolíticas que ejercen los hablantes en cada toma de palabra y con el pronombre escogido. Estas micropolíticas tienen importantes consecuencias tanto para los interlocutores como para sus lenguas y culturas. En este sentido, también demuestran la tarea micropolítica que tenemos todos los hablantes cuando se trata de remediar la diglosia.
\end{itemize}

Está claro que no basta con analizar sistemas pronominales u otros factores sintácticos, sino que habría que incluir todo el paradigma de las dimensiones lingüísticas – desde el elemento más pequeño de los sonidos hasta las estructuras narrativas, discursivas y comunicativas –, y combinar métodos lingüísticos con los de otras disciplinas.

\section*{Abreviaciones}
\begin{tabularx}{.475\textwidth}{@{}ll}
OI& objeto indirecto\\
OD& objeto in/directo\\
f &género gramatical femenino\\
m & género gramatical masculino\\
\end{tabularx}
\begin{tabularx}{.475\textwidth}{lQ}
sg& número gramatical singular\\
pl& número gramatical plural\\
+/–h& actualización o no del rasgo semántico \textit{humano}\\
\end{tabularx}

\section*{Convenciones de transcripción de GAT (\citealt{SeltingEtAl1998})}

Ortografía adaptada: indica mediante los símbolos del Alfabeto Fonético Internacional divergencias de la norma contextual que podrían ser significativas

\noindent
: alargamiento del sonido antepuesto\\
= eliminación de la micropausa entre dos lexemas\\
’ acortamiento de un sonido\\
(.) pausa corta\\
. ; , ? los signos de puntuación indican el movimiento prosódico al final de un segmento\\
{\textrangle}{\textrangle}f{\textlangle}  {\textlangle}, {\textrangle}{\textrangle}len{\textlangle}  {\textlangle} indicaciones prosódicos suprasegmentales en cuanto a la intensidad de la pronunciación (\textit{forte}) y el ritmo (\textit{lento}).\\
 \sloppy\printbibliography[heading=subbibliography,notkeyword=this]

\section{Anexos}

\begin{table}
\caption{\label{tab:mick:3.1} Análisis detallado de las omisiones de pronombres, por grupo (en \%).\\La base de referencia es el total de los referentes de cada categoría por grupo; por ejemplo, los hablantes de procedencia andina omiten el pronombre átono de 3\textsuperscript{a} persona en el 32.7\% de los referentes de OD.}
\begin{tabularx}{\textwidth}{Xrrrrrr}
\lsptoprule
 Ø & \textbf{OD} & \textbf{OI} & \textbf{OD} \textbf{f} & \textbf{OD} \textbf{m} & \textbf{sg} & \textbf{pl}\\
 \midrule
 Andes & 32.7 & 2.8 & 33.7 & 32.3 & 24.2 & 18.0\\
 Lima + & 3.4 & 0 & 3.1 & 3.5 & 3 & 0\\
 Lima & 2.1 & 0 & 3.2 & 1.8 & 1.6 & 0\\
\lspbottomrule
\end{tabularx}
\end{table}

\begin{table}
\caption{\label{tab:mick:3.2} Análisis detallado de los leísmos producidos por cada grupo de hablantes (en \% del total de leísmos por grupo)}
\begin{tabularx}{\textwidth}{Xrrrrr}
\lsptoprule
& \textbf{OD} \textbf{f} & \textbf{OD} \textbf{m} & \textbf{OD} \textbf{n} & \textbf{OD} \textbf{pl} & \textbf{+h}\\
\midrule 
 Andes & 61.2 & 38.8 & 0 & 73.5 & 91.8\\
 Lima + & 33.3 & 53.3 & 13.3 & 53.3 & 80.0\\
 Lima & 52.9 & 47.1 & 0 & 29.4 & 70.6\\
\lspbottomrule
\end{tabularx}
\end{table}

\begin{table}
\caption{\label{tab:mick:4.1} Tendencias porcentuales del uso pronominal según orden de frecuencia. \\Porcentajes relativos a las tendencias predominantes en el uso pronominal de los tres grupos, empezando en la primera línea por el factor más importante para el grupo respectivo.}

\begin{tabularx}{\textwidth}{lQQQ}
\lsptoprule
& \textbf{\textit{Andes}} & \textbf{\textit{Lima+}} & \textbf{\textit{Lima–}}\\
1 & etimológico: 62\% & etimológico: 79.0\% & etimológico: 87.9\%\\
2 & omisión: 20.9\% & neutr. núm.: 8.5\% & neutr. núm.: 5.4\%\\
3 & leísmo: 8.5\% & neutr. género: 6.5\% & leísmo: 3.5\%\\
4 & neutr. género: 4.0\% & leísmo: 3.0\% & neutr. género: 1.6\%\\
5 & neutr. núm.: 3.7\% & omisión: 2.4\% & omisión: 1.4\%\\
6 & loísmo/laísmo: 0.7/0.2\% & loísmo/laísmo: 0.6/0\% & loísmo/laísmo: 0/0.2\%\\
\lspbottomrule
\end{tabularx}
\end{table}

\begin{table}
\caption{\label{tab:mick:8} Categorización de los interlocutores en los dos grupos, \textit{Andes} y \textit{Lima+}}
\begin{tabularx}{\textwidth}{QQlQ}
\lsptoprule
\multicolumn{2}{c}{ \textbf{\textit{Andes}}} & \multicolumn{2}{c}{ \textbf{\textit{Lima+}}}\\
\midrule
Paola, Inés, Eva, Flor, Carmen, Ana & comuneras campesinas, socias de organización indígena local & Jimena, Liz & representan-tes de instituciones nacionales e internacionales\\
\tablevspace
Clara, Alberto, Héctor & dirigentes locales & Sara, Juan & abogados de derechos indígenas\\
\tablevspace
Ruth, Lea & madre e hija, dirigentas indígenas a nivel regional & Martín, Denis & consultores \\
\tablevspace
María & dirigente indígena a nivel nacional &  & \\
\lspbottomrule
\end{tabularx}
\end{table}

\begin{table}
\caption{\label{tab:mick:9} Interlocutores de las conversaciones grupales\\En las 6 grabaciones realizadas en la región andina intervienen varios interlocutores, cuyos nombres se encuentran aquí enlistados. Ruth y Lea intervienen en varias de ellas, por lo cual se distingue sus discursos según el número de entrevista.}
\begin{tabularx}{\textwidth}{Xl}
\lsptoprule
\textbf{Orden} \textbf{cronológico} & \textbf{Interlocutores}\\
\midrule
Conversación 1 & Eva, Flor, Lea\_1, Ruth\_1\\
Conversación 2 & Alberto, Lea\_2, Ruth\_2\\
Conversación 3 & Clara, Lea\_3, Ruth\_3\\
Conversación 4 & Ana, Carmen, Lea\_4, Ruth\_4\\
Conversación 5 & Paola, Lea\_5, Ruth\_5\\
Conversación 6 & Inés, Lea\_6\\
Conversación 7 & Héctor, Ruth\_7\\
\lspbottomrule
\end{tabularx}
\end{table}


\end{document}
