\documentclass[output=paper]{langscibook}
\ChapterDOI{10.5281/zenodo.5643279}
\author{Carol A. Klee\orcid{0000-0002-8191-6293}\affiliation{University of Minnesota}}
\title{Nuevas perspectivas sobre el lenguaje: implicaciones para el estudio del español en contacto con otras lenguas}
\abstract{En esta contribución se describen y se ejemplifican dos aproximaciones al estudio del español en contacto con las lenguas indígenas de América Latina. La primera --- el enfoque cuantitativo de la sociolingüística variacionista --- se ejemplifica en un estudio de \citet{KleeTightCaravedo2011}. En él se analiza el habla de migrantes a Lima (hablantes de quechua como L1) y la de sus hijos (hablantes de español como L1), concretamente, qué factores influyen en el uso del orden de palabras Objeto-Verbo (OV), un fenómeno que se atribuye al contacto con el quechua. Este tipo de estudio proporciona información sobre los factores lingüísticos y sociales que condicionan la variación lingüística de lenguas en contacto, lo cual permite determinar si hay un cambio en progreso en la comunidad. En contraste, la segunda aproximación, basada en las prácticas lingüísticas heterogéneas de individuos bilingües, es cualitativa e idiográfica, y se enfoca en cómo los hablantes hacen uso de sus repertorios lingüísticos para crear significado y alcanzar metas específicas de comunicación. Esta aproximación se ejemplifica con dos estudios sobre el contacto del español con el quechua: uno de \citet{Babel2014time} sobre el desarrollo de un marcador del pasado evocativo en el español andino de Bolivia, y otro de  \citet{SánchezMoreano2017}, quien analiza el orden de palabras OV en el español de ecuatorianos hablantes de quechua que han emigrado a Cali, Colombia, enfocándose en su significado social dentro de la comunidad. Las dos aproximaciones contribuyen a nuestro entendimiento del complejo fenómeno del contacto lingüístico. El desafío será descubrir las nuevas sinergías que puedan resultar de la colaboración entre ambas posturas.
% \keywords{el contacto de lenguas, el español andino, el orden de palabras, la sociolingüística variacionista, prácticas lingüísticas heterogéneas}
}

\IfFileExists{../localcommands.tex}{
  \addbibresource{../localbibliography.bib}
  \input{../localpackages}
  %Copy this to localcommands.tex

\usepackage[english]{babel}
\usepackage{amsmath}
\usepackage{amssymb,amsfonts,textcomp}
\usepackage{array}
\usepackage{hhline}
\usepackage{hyperref}

\newenvironment{styleStandard}{}{}
\newenvironment{stylelsAbstract}{}{}
\newenvironment{stylelsSectioni}{}{}
\newenvironment{stylelsSectionii}{}{}
\newenvironment{stylelsBulletList}{}{}
\newenvironment{styleBibliographyi}{}{}
\newenvironment{listWWNumxxvleveli}{}{}
\newenvironment{listWWNumxxvlevelii}{}{}
\newenvironment{listWWNumxxvleveliii}{}{}
\newenvironment{listWWNumxxvleveliv}{}{}
\newenvironment{listWWNumixleveli}{}{}
\newenvironment{listWWNumixlevelii}{}{}
\newenvironment{listWWNumixleveliii}{}{}
\newenvironment{listWWNumixleveliv}{}{}

\newcommand\textstyleListLabelxvi[1]{#1}
\newcommand\labellistWWNumxxvleveli{\thelistWWNumxxvleveli.}
\newcommand\labellistWWNumxxvlevelii{\thelistWWNumxxvlevelii.}
\newcommand\labellistWWNumxxvleveliii{\thelistWWNumxxvleveliii.}
\newcommand\labellistWWNumxxvleveliv{\thelistWWNumxxvleveliv.}
\newcommand\labellistWWNumixleveli{[F0B7?]}
\newcommand\labellistWWNumixlevelii{\textstyleListLabelxvi{o}}
\newcommand\labellistWWNumixleveliii{[F0A7?]}
\newcommand\labellistWWNumixleveliv{[F0B7?]}

\newcounter{listWWNumxxvleveli}
\newcounter{listWWNumxxvlevelii}[listWWNumxxvleveli]
\newcounter{listWWNumxxvleveliii}[listWWNumxxvlevelii]
\newcounter{listWWNumxxvleveliv}[listWWNumxxvleveliii]
\newcounter{itemize}  
  \input{../localhyphenation} 
  \togglepaper[1]%%chapternumber
}{}

\shorttitlerunninghead{Nuevas perspectivas sobre el lenguaje}
\begin{document}
\maketitle 
\shorttitlerunninghead{Nuevas perspectivas sobre el lenguaje}




\section{Introducción}


En las sociedades (post)modernas las “condiciones de vida de las lenguas”, como las denominan   \citet{ToivanenSaarikivi2016}, se han visto afectadas por una serie de fenómenos sociales que abarcan, entre otros, el aumento de las tasas de alfabetización como resultado de la educación formal; la estandarización de numerosas lenguas minoritarias; un mayor acceso a los medios de comunicación, así como a los medios sociales; la urbanización y los cambios que esta conlleva en los modos de subsistencia; y el incremento de la movilidad de las poblaciones (cf. \citealt{ToivanenSaarikivi2016}: 1). Estos cambios han ejercido un impacto notable sobre las lenguas del mundo, entre ellas las lenguas indígenas de América Latina. La urbanización masiva, aunada al aumento de la migración y la movilidad, genera oportunidades para estudiar, desde nuevas perspectivas y con enfoques innovadores, el contacto del español con las lenguas indígenas del área.

~En los últimos años, al reconsiderarse las fronteras metafóricas entre las lenguas, ha surgido una divergencia en el estudio del bilingüismo y el contacto entre lenguas. Por una parte, el enfoque que podría llamarse “tradicional” –– que parte de sistemas lingüísticos discontinuos y delimitados que se encuentran en situación de contacto –– se ha centrado, por lo general, en las consecuencias lingüísticas de dicho contacto, es decir, en los cambios léxicos, fonológicos, morfosintácticos o pragmáticos que ocurren en uno o ambos sistemas, como sucede, por ejemplo, al convivir el español durante siglos con la lengua maya en la península de Yucatán o con el quechua en la región andina. Por otra parte, durante la última década, algunos investigadores han empezado a cuestionar la existencia de lenguas independientes y definidas (\emph {named languages}), y la noción de fronteras entre las mismas, para pasar a enfocarse en el análisis de los repertorios lingüísticos de los hablantes bilingües. Es decir, en lugar de centrarse en los sistemas lingüísticos que se encuentran en situación de contacto, lo hacen en las prácticas lingüísticas heterogéneas de individuos bilingües o multilingües. Entre los diversos términos utilizados para referirse a este nuevo enfoque se encuentran los siguientes: prácticas transidiomáticas (\emph {transidiomatic practices}) \citep{Jacquemet2005}, lenguar (\emph {languaging}) o polilenguar (\emph {polylanguanging}) \citep{JørgensenEtAl2011}, translenguar (\emph {translanguaging}) (\citealt{García2009}; \citealt{GarcíaWei2014}; \citealt{CreeseBlackledge2010}) y prácticas lingüísticas heterogéneas (\textit{pratiques langagières hétérogènes}) \citep{Léglise2013}. No cabe duda de que el uso de uno u otro enfoque tendrá implicaciones metodológicas y epistemológicas para el estudio del contacto entre lenguas.

Como punto de partida para considerar la divergencia entre ambos, puede servir de referencia un artículo de  \citet{GuyHinskens2016} en el que se presenta la distinción metodológica establecida por \citet{Dilthey1883} entre \emph {Naturwissenschaften} y \emph {Geisteswissenschaften}, es decir, entre las ciencias (sensu stricto) y las humanidades:

\begin{quote}
science in the narrow sense aims at \emph {erklären}, ‘explaining’, with causality and probability as central concepts, whereas the humanities target at \emph {verstehen}, understanding, with meaningfulness as one of the central concepts (\citealt{GuyHinskens2016}: 3). 
\end{quote}

Los autores establecen una correlación con los enfoques nomotéticos e idiográficos utilizados en las ciencias: “while the former focus on recurring phenomena, inducing the regularities, laws or principles they are subject to, the latter deal with unique phenomena and their specific properties” (\citealt{GuyHinskens2016}: 3). En la lingüística moderna     sigue existiendo esta misma tensión. Los primeros en adoptar una postura crítica, entre quienes se encuentra Deborah \citet{Cameron1990}, argumentan que las teorías desarrolladas por ciertas ramas de la sociolingüística para explicar la variación se limitan a describir los fenómenos analizados (mediante el uso de correlaciones estadísticas), en lugar de explicar el comportamiento lingüístico como acto social. Otros, como Penny \citet{Eckert2018}, si bien hacen destacar que las correlaciones macrosociales resultan fundamentales para entender la variación, reconocen al mismo tiempo la importancia de enfoques sociolingüísticos que examinen el significado social de la variación que subyace a las generalizaciones cuantitativas. Este es el enfoque de la tercera ola de la sociolingüñistica, que pone más énfasis en el carácter agentivo de la variación y en la habilidad del individuo de seleccionar variantes sociolingüísticas específicas como expresión de identidad y/o de ideología. En lo que respecta al campo del contacto del español con otras lenguas, la mayoría de los estudios se han centrado en la descripción de fenómenos lingüísticos en lugar de hacerlo en los procesos comunicativos propios de la interacción social (\citealt{KleeLynch2009book}; véase también \citealt{CollinsBaynhamSlembrouck2009}).

A continuación, se describen dos enfoques distintos para el estudio del español en contacto con las lenguas indígenas de América Latina, uno según la tradición nomotética y otro según la tradición idiográfica, y se examina el aporte de cada uno a la comprensión de este fenómeno.


\section{La tradición nomotética: enfoques variacionistas para el estudio del contacto entre el español y algunas lenguas indígenas}


Como señalan Isabelle Léglise y Claudine Chamoreau (\citeyear{LégliseChamoreau2013}), los estudios del contacto entre lenguas se han dedicado tradicionalmente a explorar ambientes estables, como los que se encuentran en la región andina o en Paraguay, donde ha habido contacto interlingüístico a lo largo de varios siglos. No obstante, con las profundas transformaciones sociales de la era postmoderna han surgido nuevos cambios lingüísticos. Dentro del contexto de las lenguas en contacto, una manera de analizar el cambio lingüístico es mediante el estudio de la variación. Según  \citet[6]{LégliseChamoreau2013}: “la variación lingüística es un área opaca, un punto ciego, para la mayoría de los estudios sobre el cambio lingüístico ocasionado por el contacto de lenguas”.\footnote{“Linguistic variation is an opaque area, a blind spot, for most contact-induced language change studies”} Si bien se han realizado diversos análisis variacionistas del español en contacto con el inglés en los Estados Unidos (\citealt{Orozco2018}; \citealt{OtheguyZentella2012}; \citealt{Silva-Corvalán1994};  \citealt{TorresCacoullosTravis2011,TorresCacoullosTravis2018}; entre muchos otros), existen relativamente pocos estudios de este tipo sobre el español en contacto con lenguas indígenas en América Latina.\footnote{El libro \emph {Variación y cambio lingüístico en situaciones de contacto} (\citealt{PalaciosAlcaine2017book}) incluye algunos estudios del contacto entre el español y las lenguas indígenas en Latinoamérica desde una aproximación variacionista, en particular el de  \citet{TorresSánchez2017}.} ¿Cómo puede contribuir este tipo de estudios a entender el contacto entre lenguas? Los análisis multivariados pueden mostrar un “panorama general” de las direcciones del cambio lingüístico, en particular en las áreas urbanas, y pueden revelar además si un cambio en particular puede o no atribuirse al contacto entre lenguas.



\subsection{El español en contacto con el quechua}



El análisis variacionista también permite revelar los cambios lingüísticos que se producen a causa de la migración a las zonas urbanas. Junto con Rocío Caravedo, la autora de este artículo inició un estudio a gran escala del español limeño con objeto de determinar los cambios lingüísticos que se estaban produciendo como resultado de las migraciones masivas desde las provincias durante la segunda mitad del siglo XX. La migración de hablantes de lenguas indígenas, principalmente quechua y de español andino, tuvo como resultado el contacto entre lenguas y dialectos en Lima. La magnitud de estos cambios demográficos se refleja en la composición actual de la ciudad, donde estos migrantes, sus hijos y nietos –– a los que  \citet{ArellanoBurgos2004} denominan “neolimeños” –– constituyen la gran mayoría de la población, como puede verse en el \tabref{tab:klee:1}. Los llamados limeños “clásicos”, es decir, aquellos cuyos padres y abuelos nacieron en Lima, constituyen apenas 12,7\% de la población.


\begin{table}
\caption{\label{tab:klee:1}Composición de la población de Lima según  \citet{ArellanoBurgos2004}}

\begin{tabularx}{\textwidth}{Xr}
\lsptoprule
{Población} & {\%}\\
\midrule
{Limeños clásicos (con padres y abuelos nacidos en Lima)}  & {12.7\%}\\
{Neolimeños, migrantes de 1\textsuperscript{a} generación (nacidos en las provincias)} & {36.2\%}\\
{Neolimeños de 2\textsuperscript{a} generación (nacidos en Lima y de padres y abuelos nacidos en las provincias)}  & {43.5\%}\\
{Neolimeños de 3\textsuperscript{a} generación (ellos y sus padres nacidos en Lima y de abuelos nacidos en las provincias)}  & {  7.6\%}\\
\lspbottomrule
\end{tabularx}
\end{table}

Con la llegada de grandes masas de hablantes andinos surge una serie de preguntas acerca de la transformación del español limeño entre finales del siglo XX y principios del XXI. Entre las que tratamos de abordar (\citealt{CaravedoKlee2012}; \citealt{KleeCaravedo2005,KleeCaravedo2006,KleeCaravedo2020,KleeTightCaravedo2011,KleeEtAl2018}) se encuentran las siguientes: ¿Qué rasgos tiene el español de los inmigrantes andinos en Lima? ¿Retiene el español de estos hablantes, así como el de sus hijos y nietos, características típicas de las variedades andinas rurales, o se están adoptando rápidamente rasgos de la zona costera? Y, dado el vasto predominio demográfico de los inmigrantes, ¿han empezado a influir las características del español andino, entre ellas las que tiene por estar en contacto con el quechua o con el aimara, sobre el español costero que hablan los limeños clásicos?

Un rasgo del español andino reconocido como resultado del contacto con el quechua es la mayor frecuencia del orden de palabras OV. En algunos estudios previos del orden de palabras de esta variedad del español (\citealt{Camacho1999}; \citealt{Escobar2000}; \citealt{Klee1996}; \citealt{KleeTightCaravedo2011};  \citealt{LujanMinayaSankoff1984}; \citealt{Mendoza1991}; \citealt{Muntendam2008a,Muntendam2008b}; \citealt{Muysken1984}; \citealt{OcampoKlee1995}) se ha demostrado que existe una correlación  entre el orden OV y varias situaciones discursivas –– repetición, resumen, acuerdo y explicación ––, además de las funciones pragmáticas que se encuentran en el español que no está en una situación de contacto.

El análisis variacionista de \citet{KleeTightCaravedo2011} se basa en datos recolectados entre 34 residentes de Lima (siete de ellos inmigrantes de primera generación, seis de la generación 1,5,\footnote{La generación 1,5 se refiere a los individuos que nacieron en las provincias y llegaron a Lima antes de cumplir 12 años.} 10 inmigrantes de segunda generación y 11 limeños) y emplea las siguientes variables sociales: generación, género, lengua primaria y nivel de educación; las variables lingüísticas utilizadas son la función pragmática (sentido contrario a lo esperado, foco contrastivo, elemento focal, información, tema) y la situación discursiva (acuerdo, explicación, repetición, resumen). Como puede observarse en el \tabref{tab:klee:2}, el análisis de regresión múltiple reveló que dos de las variables independientes, la función pragmática/situación discursiva y la lengua primaria resultaron significativas para predecir el orden de palabras VO. En este cuadro se indican los valores de las variables específicas y su respectivo peso; cualquier valor por encima de ,500 revela una preferencia por el orden de palabras VO y por debajo de esa cifra por el OV.

\begin{table}
\caption{\label{tab:klee:2} Análisis multivariado de la contribución de factores internos y externos significativos al orden de palabras VO (\citealt{KleeTightCaravedo2011}: 18)}
\begin{tabularx}{\textwidth}{QQrrr}
\lsptoprule

{\textit{Variable}} & {\textit{Valor}} & {\textit{Peso del} } {\textit{factor}} & {\textit{\%}} & {\textit{Número}}\\
\midrule
{Función pragmática} & {Información} & {,741} & {99} & {415}\\
%\hhline%%replace by cmidrule{-~~~~} & {Repetición} & {,088} & { 81} & {  68}\\
& {Explicación} & {,072} & {75} & {  12}\\
& {Acuerdo} & {,062} & { 67} & {    6}\\
& {Elemento focal}  & {,041} & {68} & {  19}\\
& {Resumen} & {,025} & {53} & {  17}\\
& {Foco contrastivo} & {,004} & {17} & {    6}\\
& {Tema} & {,003} & {   8} & {   13}\\
& {Rango} & {\textit{738}} &  & \\
\tablevspace
{Lengua primaria} & {Español (o bilingüe)} & {,616} & {94} & {392}\\
& {Indígena} & {,245} & {81} & {164}\\
& {Rango} & {\textit{371}} &  & \\
\midrule
{Total N} & & & & {556}\\
\lspbottomrule
\end{tabularx}
Input 0,979; Verosimilitud logarítmica= -88,383; Chi cuadrado /celda = 0,294
\end{table}

Al igual que en estudios anteriores, con la función pragmática de información se observó una preferencia por el orden de palabras VO mientras que, con las situaciones discursivas de repetición, explicación, acuerdo, elemento focal, resumen, foco contrastivo y tema, se observaron más construcciones con el objeto antepuesto al verbo. La L1 del hablante constituye un factor significativo también. Los hablantes nativos de una lengua indígena le dan mayor preferencia al orden OV (19\%), mientras que los hablantes nativos de español se inclinan por el uso del VO y sólo utilizan el orden OV en 6\% de las instancias. Los autores llevaron a cabo un análisis multivariado de cada grupo por separado, para determinar si los hablantes de español como L1 y los hablantes de lenguas indígenas como L1 utilizan las funciones pragmáticas de forma equivalente. Los resultados revelaron que, en general, son las mismas funciones pragmáticas y situaciones discursivas las que generan el orden de palabras OV en ambos grupos, dadas las semejanzas de las variables condicionantes y el ranking interno de los factores involucrados,\footnote{La única excepción es la situación discursiva de “acuerdo”, que entre los hablantes de español como L1 no motiva el orden de palabras OV.} que parecen ser en su mayoría los mismos que se encuentran en variedades del español que no están en situación de contacto.

Este estudio demuestra que sólo en el habla de inmigrantes de primera generación cuya L1 es una lengua indígena se encuentra una frecuencia relativamente alta del orden de palabras OV. Los inmigrantes que hablan el español como L1 parecen ajustarse a las normas del español que no se encuentra en situación de contacto, puesto que no se hallaron diferencias estadísticas entre el habla de los inmigrantes de segunda generación y los limeños clásicos. Así, esta variante sintáctica parece ir paralela a ciertas características fonéticas y fonológicas del español andino que en Lima son objeto de estigmatización, en particular la lateral palatal (e.g. \emph {calle} [ˈkaʎe]); la fricativa [+estridente] [ř], también llamada r asibilada (e.g. \emph {perro} [ˈpeřo]), y la neutralización de número y género en los pronombres de complemento directo de tercera persona (e.g. \emph {¿La oveja? Lo vendí}.), características que solo se encuentran en el habla de los inmigrantes de primera generación; en la segunda generación prevalecen las variantes del limeño clásico (\citealt{KleeCaravedo2006}).

\citet{PoplackLevey2010} y \citet{PoplackEtAl2012} proponen un marco comparativo variacionista para los estudios de las lenguas en contacto, que permita comparar las restricciones lingüísticas entre la lengua fuente y la lengua receptora a través de un análisis (multivariado) cuantitativo sistemático, el cual ayuda a determinar si el contacto interlingüístico ha provocado cambios en la gramática de la lengua adoptiva. En muy pocos casos se ha aplicado el marco comparativo variacionista en estudios del contacto entre el español y las lenguas indígenas de América Latina. Algunas excepciones notables las constituyen las investigaciones sobre la expresión del pronombre de sujeto en el español yucateco de \citet{Michnowicz2015} y en el español andino del Perú de \citet{Cerrón-Palomino2019}, el estudio de \citet{Romero2015} sobre la concordancia negativa en k’ichee’ (maya) y el análisis realizado por \citet{McKinnon2020} de la aspiración de las oclusivas sordas en el español guatemalteco.

Si bien el análisis variacionista nos permite entender la variación social y el cambio lingüístico provocado por el contacto entre lenguas a nivel general, carece de los métodos necesarios para proveer detalles importantes que revelen, por ejemplo, “how speakers deploy their linguistic resources, along with other symbolic resources, to construct and reinforce the social categories to which they belong” \citep[135]{Bayley2002}. Los enfoques idiográficos, tal como se describe a continuación, se enfocan en la utilización de los distintos recursos lingüísticos con los que cuentan los hablantes y así llenan una laguna importante en el estudio del contacto del español con otras lenguas.


\section{El enfoque idiográfico: contribución al estudio de las lenguas en contacto}


A principios del siglo XXI diversos académicos, entre los cuales se cuentan  \citet{MakoniPennycook2006}, comienzan a cuestionar el concepto de lenguas independientes y definidas (\emph {named languages}) en un intento por “desinventar” y reconstituir el panorama lingüístico del mundo contemporáneo. Por su parte, \citet[102]{Blommaert2010} sugiere un cambio en “our focus from ‘languages’ (primarily as an ideological and institutional construct) to resources (the actual and observable ways of using language).” Igualmente, \citet{JørgensenEtAl2011}, en su artículo “Polylanguaging in superdiversity”, sostienen que la idea tradicional de lenguas separadas como sistemas delimitados de características lingüísticas específicas no es capaz de captar la realidad lingüística. Estos autores observan que los hablantes hacen uso de recursos lingüísticos y no de lenguas en sí, y además, que los utilizan como recursos semióticos, asociándolos con significados y valores sociales particulares.

Dentro del contexto de las lenguas en contacto, en particular del español con el inglés en los Estados Unidos, Ofelia García y Ricardo Otheguy (\citeyear{GarcíaOtheguy2015}) ofrecen una nueva perspectiva del bilingüismo hispano en respuesta a esta reconceptualización de las lenguas. Lo hacen mediante un enfoque centrado en el hablante, el cual abarca lo que denominan “a disaggregated view of language competence and a translanguaging view of bilingual practices”. Así, en un artículo publicado en 2015 junto con Wallis Reid (\citealt{OtheguyGarcíaReid2015}), definen este “translenguar” como “the deployment of a speaker’s full linguistic repertoire without regard for watchful adherence to the socially and politically defined boundaries of named (and usually national and state) languages” (\citeyear{OtheguyGarcíaReid2015}: 283).

Al mismo tiempo, García y Otheguy sostienen que tanto el enfoque del bilingüismo de índole heteroglósica y centrado en el hablante como la concepción de las prácticas multilingües basada en el translenguar son nociones que entran en conflicto con las del cambio inducido por contacto (por ejemplo, préstamos, calcos, cambio estructural) y la alternancia de código (2015: 648). Los autores señalan que en los estudios de \citet{Weinreich1953}; \citet{Haugen1953}; \citet{ThomasonKaufman1988} y \citet{Winford2003book}, la herramienta básica conceptual ha sido una sola lengua independiente y definida que introduce material -- préstamos, calcos, estructuras, etc. -- de otra lengua externamente designada. Asimismo, \citet{OtheguyGarcíaReid2015} afirman que los gramáticos no ostentan una base teórica que les permita resolver controversias sobre lo que debe separarse y designarse (p. 287) en las lenguas, aunque obviamente estén en capacidad de analizar muestras de habla y texto, y de describir sus características (p. 288). Al analizar el habla del bilingüe, García y Otheguy observan cómo cada individuo selecciona recursos a partir de un repertorio construido de forma dinámica; es decir, no se trata de préstamos, calcos ni estructuras cambiantes, ni tampoco de alternancia de lenguas. Igualmente señalan que:

\begin{quote}
[…] translanguaging will be distinguished from the notion of codeswitching in being not simply a shift between two languages, but rather the speakers’ construction and use of original and complex interrelated discursive prac\-tices that cannot be easily assigned to one or another language, but which make up the speakers’ complete language repertoire (\citealt{GarcíaOtheguy2015}: 653).
\end{quote}

Así, los autores consideran el translenguar no una simple teoría del bilingüismo sino del lenguaje en general, según la cual los individuos bilingües poseen una gramática subyacente única, sin diferenciación interna alguna que se corresponda con las descripciones estructurales de las lenguas consideradas independientes. Probablemente sea ésta la más polémica de sus afirmaciones, como puede verse en la crítica presentada por \citet{MacSwan2017}.\footnote{También véase la respuesta de  \citet{OtheguyGarcíaReid2018}.}

Pero si dejamos de lado este debate en particular, el enfoque en el translenguar puede hacer un considerable aporte al estudio y comprensión del contacto entre lenguas. Este tipo de enfoque se basa en las prácticas lingüísticas heterogéneas de individuos bilingües o multilingües, o en la forma en que el hablante recurre a diferentes recursos lingüísticos, cognitivos y semióticos para formular significados \citep{Wei2018}. \citet{OtheguyEtAl2015} sostienen asimismo que, dentro del marco del translenguar, los lingüistas deben enfocarse en los idiolectos, es decir, en la forma en que cada individuo desarrolla de manera propia una gramática mental que surge a través de la interacción con otros hablantes (2015: 289). Los autores afirman que la principal diferencia entre los idiolectos de individuos monolingües y de individuos bilingües radica en que estos últimos contienen más recursos lingüísticos, además de marcadores socioculturales más complejos para señalar cuál de los recursos usar, cuándo, dónde y con quién(es) (2015: 292). Igualmente sostienen que el concepto de translenguar resulta más adecuado para describir conductas y prácticas lingüísticas reales. Así, el idiolecto se presenta como el pilar del concepto de translenguar (2015: 294).

¿Qué impacto podría tener este enfoque sobre los estudios del contacto entre lenguas? Un cambio importante sería en el foco de las investigaciones, las cuales, como se menciona arriba, se han centrado tradicionalmente en sistemas lingüísticos en contacto, en lugar de hacerlo en las prácticas de los hablantes individuales. Cabe señalar que Gillian Sankoff anticipó este cambio de foco en un artículo publicado en 2001, titulado “Linguistic outcomes of language contact”, al final del cual se refiere a esta nueva perspectiva: 

\begin{quote}
To my way of thinking, the reintegration of the individual into the overall matrix of the speech community and the evolving languages, represents the greatest challenge and the greatest scope for advancement in the research of the next decade \citep[659]{Sankoff2001}.
\end{quote}


¿Cómo abordar entonces, desde un punto de vista metodológico, dicha perspectiva? \citet{OtheguyShinDanielEnprensa}, en un artículo por publicar, examinan la naturaleza idiolectal del contacto entre lenguas y el papel que pudo haber tenido cuando los españoles entraron en contacto con nahuas y yorubas en el Nuevo Mundo. Para estos autores resulta problemática la noción de contacto interlingüístico, puesto que el contacto se produce realmente entre idiolectos particulares y no entre lenguas discontinuas (en prensa: 9). En el artículo citan la primera página del influyente volumen de Weinreich \emph {Languages in Contact}  (\citeyear{Weinreich1953}: 1), donde señala: “The language-using individuals are at the locus of the contact”. Por otra parte, sostienen que tanto la fonología como la gramática deben basarse en el uso de la lengua (\citealt{BybeeHopper2001}; \citealt{Bybee2010}), ya que solo de esta forma es posible tener una perspectiva desagregada y específica de lo que sucede en las situaciones de contacto (en prensa: 11). Así, al analizar el uso de la sibilante sorda [s] en la zona montañosa de México, describen cómo se incorporaron al inventario léxico de los hablantes de nahuatl palabras como cristiano, escapulario y espada, donde la coda /s/ se pronunciaba al principio como [-${\int}$], por ejemplo i[ʃ]capul (< escapulario) o cri[ʃ]tiano (< cristiano), o como [-s], por ejemplo i[s]capul (< escapulario) o cri[s]tiano (< cristiano), si bien para mediados del siglo XVII tanto europeos como nahuas pronunciaban estas palabras con [-s]. En este siglo, con la llegada a la Nueva España de colonos provenientes del sur de la península ibérica, en cuyo dialecto se daba la lenición de la coda /-s/, se impuso la pronunciación de la [-s] sibilante a causa del contacto cercano entre ibéricos y nahuas, con lo que se establecieron ejemplos como ca[s]taño, cri[s]tiano y e[s]pada. Como bien señalan los autores, era la palabra cri[s]tiano, una palabra que no estaba asociada a una lengua denominada y que a la vez se encontraba en el idiolecto de muchos individuos, españoles, mestizos y nahuas, lo cual impidió que se estableciera cri[h]tiano (en prensa: 38). En la descripción que ofrecen de este proceso se destaca el papel que desempeñaron tanto los hablantes individuales como ciertas palabras específicas, y se resalta la complejidad de los eventos históricos interpersonales, la cual se pierde cuando se replantea el proceso como simplemente un préstamo del español al náhuatl (en prensa: 33). Desde la perspectiva que presentan estos autores, una aproximación que estos nombran \emph {featural-idiolectal} (es decir, un enfoque en los rasgos lingüísticos y en los idiolectos) es la clave para entender las situaciones de contacto y sus resultados lingüísticos (en prensa: 14). Esta propuesta se enfoca en los fenómenos lingüísticos que pueden observarse dentro de distintos idiolectos y además refleja la complejidad de la interacción directa entre hablantes, pero hasta el momento se ha centrado exclusivamente en una recreación hipotética del contacto histórico entre lenguas y, hasta donde sabemos, no se ha extendido a contextos contemporáneos.\footnote{\citet{Erker2017} aplica este mismo tipo de enfoque al estudio del contacto entre dialectos, en particular en el contexto del español en Estados Unidos.}

¿Qué implicaría y aportaría la adopción de un enfoque del contacto entre lenguas centrado en los rasgos idiolectales en el presente? Pueden servir de modelo, en ciertos aspectos, las investigaciones realizadas por Anna Babel (véase \citealt{Babel2009,Babel2011,Babel2014time,Babel2014stereotypes,Babel2018,Babel2021}), cuyo trabajo se enfoca en el contacto entre el español y el quechua en Bolivia. Su trabajo parte de la perspectiva de que la variación lingüística es parte de un sistema semiótico social que los hablantes utilizan estratégicamente para construir una identidad dentro de un contexto específico de interacción. Su investigación se caracteriza por períodos extendidos de trabajo de campo etnográfico, así como por una comprensión profunda de la comunidad investigada y por la grabación de conversaciones espontáneas. Este método, combinado con el microanálisis, permite interpretar las variables sociolingüísticas según el significado específico que tiene cada una al utilizarla un interlocutor en particular como parte de un evento discursivo dado. Si bien este tipo de investigación requiere mucho tiempo, el periodo invertido en la recolección y el análisis de los datos puede generar revelaciones importantes respecto a la interconexión entre los aspectos sociales y estructurales de las lenguas, que no pueden obtenerse, en cambio, mediante enfoques sociolingüísticos a nivel macro.

Por ejemplo, en un artículo publicado en 2014, “Time and reminiscence in contact”, Babel describe el desarrollo de un marcador del pasado evocativo entre dos generaciones de hablantes de una zona de contacto entre el quechua y el español en el centro de Bolivia \citep{Babel2014stereotypes}. Este proceso de potencial gramaticalización es resultado de una reinterpretación semántica/pragmática emergente de una forma verbal, impulsada por la experiencia, tanto personal como colectiva, del cambio social, como consecuencia de la migración y la urbanización. La autora señala que el proceso de cambio del sistema verbal del español conlleva tres etapas. En primer lugar, el acento de la antepenúltima sílaba de la primera persona del plural del pretérito imperfecto en español cambia a la penúltima sílaba (e.g., \emph {comprábamos} > \emph {comprabámos}), probablemente por influencia del quechua, que suele acentuar esta sílaba. En segundo lugar, el morfema con cambio de acento de la primera conjugación ( –\emph {ba}) se extiende a las conjugaciones –\emph {er} e –\emph {ir} (e.g., \emph {comía} > \emph {comiabámos}). Por último, debido a la migración y a las transformaciones sociales que esta conlleva, la autora observa que los hablantes en una zona geográfica específica empezaron a usar esta variante innovadora para referirse al espacio rural de su niñez y a los recuerdos de una época pasada \citep[312]{Babel2014time}. A continuación, se ofrece un ejemplo de los datos recogidos por Babel:

\ea
\textbf{Felipa} 

[…]

1. F: Y, a veces, lo-, ya cuando trabajaba en el Chhalar, nos \textbf{traiabámos} a las ovejas pues.

2. A: Ah hah

3. F. Ahí con nosotros. (en voz baja) Nos \textbf{traiabámos} las ovejas

4. A: Ahhhh

(pausa de 13 segundos)

5. F: Haaaarto he sufrido yo pues. Por pobre. Porque mi padre me ha dejado de bien chica. (2014: 321)
\z

En este segmento, proveniente de una narración más extensa en torno a las dificultades de haber sido criada en el campo y al sufrimiento que conllevó esa vida, Felipa, una mujer de casi 70 años, utiliza la forma innovadora \emph {traiabámos} para describir las dificultades que vivió en su niñez.

Babel descubrió que los hablantes más jóvenes, en particular aquellos que han migrado a las zonas urbanas, utilizan este pasado evocativo de manera más selectiva y en general siguen la forma normativa. Las formas innovadoras se utilizan principalmente para describir un pasado distante y rural, especialmente cuando se trata de narraciones particularmente emotivas. El siguiente segmento forma parte de una narración mucho más larga en la cual Patricia, una mujer de más de 40 años, sigue por lo general las normas de uso de los verbos:


\ea
\textbf{Patricia}

1. P: Yo me, me he casado más jovencita, y, empezaba a, desde ahí, el, ya, o sea, por ver, ¿no? Que la gente mucho \textit{veníamos} pues aquí, en semejante sol.

2. A: Mhm

3. P: \textbf{Llegabámos} con nuestras wawas [rosadas], \textit{llevábamos} en la espalda, a hacer descansar un poco, este, y después venir; y otra vez de volverse a las cinco, de ir para ir a hacer cena.

4. A: Hm (2014: 328)
\z

En este segmento Patricia utiliza el pasado evocativo para resaltar y poner de relieve la dificultad y la exigencia física de hacer este viaje cargando a sus hijos, y de ahí pasa a proveer información descriptiva (que aclara cómo llevaban a los niños) con la forma normativa \emph {llevábamos}. Babel sostiene que un cambio de acento tiene significados diferentes según el hablante. Para las mujeres mayores, quienes han pasado toda su vida en el campo, esto se relaciona con las narrativas sobre su niñez, junto con otras variantes lingüísticas que marcan una manera de hablar campesina o rural. A la vez, para sus hijas, esta forma verbal conlleva un peso simbólico relacionado a su transición de la zona rural a un centro urbano y a las emociones fuertes que sienten al recordar una época pasada. Para Babel, 

\begin{quote}
[s]peakers then exploit the potential of this stress shift as a contrastive grammatical marker in order to express a category that is relevant to them -- in this case the reminiscent past \citep[330]{Babel2014time}.
\end{quote}

Si bien Anna Babel no ubica su metodología de investigación dentro del marco del translenguar, el hecho de que se enfoca en hablantes individuales y en las distintas maneras que estos tienen de utilizar su repertorio lingüístico para comunicarse (entre las que se cuentan formas verbales innovadoras), parece colocarla dentro de la línea de investigación de \citet{OtheguyGarcíaReid2018}, quienes centran su estudio de las lenguas en contacto en idiolectos y recursos lingüísticos, así como de la de \citet{JørgensenEtAl2011}, quienes afirman que los hablantes no utilizan lenguas sino recursos lingüísticos, que asocian con significados y valores sociales específicos, es decir, como recursos semióticos. Es evidente que este tipo de análisis puede ayudarnos a discernir el significado que las características del contacto entre lenguas adquieren en los idiolectos de los hablantes.

Por su parte, Santiago  \citet{SánchezMoreano2017} utiliza también este tipo de enfoque en el análisis del orden de palabras OV en el español usado por ecuatorianos hablantes de quechua que han emigrado a Cali, Colombia. Las observaciones etnográficas realizadas por este investigador en la comunidad revelan la heterogeneidad de las prácticas y los repertorios lingüísticos de estos migrantes de primera generación. Tras realizar un detallado análisis de 64 fragmentos de conversaciones,\footnote{El artículo de Sánchez Moreano incluye una descripción de una metodología ejemplar para la transcripción de repertorios lingüísticos heterogéneos.} observa que utilizan formas del español andino que resultan marcadas en el español caleño como recurso para identificarse con sus interlocutores o distanciarse de ellos. Por ejemplo, el uso de construcciones con el orden OV, como en “puro español nomás habla”, en respuesta a una pregunta de una persona que no forma parte de la comunidad, revela un tipo de posicionamiento social que pone de relieve el sentido de pertenencia al grupo de inmigrantes ecuatorianos hablantes de quechua, al tiempo que establece una distancia respecto al interlocutor. Así, en lugar de enfocarse en el orden de palabras OV como producto del contacto entre lenguas, Sánchez Moreano provee un valioso análisis de su uso y su significado social dentro de la comunidad. En este volumen pueden encontrarse otros ejemplos de este tipo de enfoque, en el capítulo de \citetv{chapters/leglise}.


\section{Conclusión}
En resumen, las dos aproximaciones que se han presentado en este capítulo contribuyen al estudio de las lenguas en contacto de diferentes maneras. El enfoque nomotético de la sociolingüística variacionista nos provee información sobre los factores sociales que condicionan la variación lingüística de lenguas y dialectos en contacto, lo cual conlleva determinar si hay un cambio en progreso en la comunidad, como se observa en el estudio sobre el español de Lima. A la vez, este enfoque nos proporciona una manera de determinar empíricamente si un cambio lingüístico en la gramática de una lengua se debe o no al contacto con otra lengua (véase \citealt{PoplackZentzDion2012}; \citealt{Michnowicz2015};  \citealt{TorresCacoullosTravis2018} entre otros). Sin embargo, esta aproximación tiene sus limitaciones; por ejemplo, no permite un análisis pormenorizado del significado discursivo local de una variante lingüística, tal como se vio en el estudio de Anna \citet{Babel2014time} y en el de Santiago  \citet{SánchezMoreano2017}. Si conceptualizamos el lenguaje y el translenguar mediante nuevos enfoques -- centrados en la forma como los hablantes hacen uso de sus repertorios lingüísticos para crear significado y alcanzar metas específicas de comunicación -- es posible lograr una comprensión más profunda de los fenómenos que entran en juego al estar en contacto dos o más lenguas.\footnote{Cabe notar, no obstante, que este renovado interés en la expresión individual y en los recursos lingüísticos utilizados en interacciones comunicativas específicas ha sido objeto de críticas por no prestar suficiente atención a las macroestructuras de poder y de desigualdad social \citep{Mazzaferro2018}.} Los fenómenos del contacto de lenguas y del cambio social son tan complejos que, en opinión de la autora, deben analizarse desde distintas perspectivas. El desafío, mencionado igualmente por  \citet{GuyHinskens2016}, será descubrir las nuevas sinergias que puedan resultar de la colaboración entre ambas posturas.

\section*{Agradecimientos}

Agradezco mucho la invitación de Santiago Sánchez Moreano y de Élodie Blestel de colaborar en las jornadas “Variedades de español en contacto con lenguas amerindias: Sistemas de contacto o prácticas lingüísticas heterogéneas” en la Sorbonne Nouvelle. Su destreza editorial ha mejorado esta contribución. Agradezco también a Claudia Giannini-Coll, quien hizo la traducción de la versión en inglés. Los errores que puedan haber son míos.

\sloppy\printbibliography[heading=subbibliography,notkeyword=this]
\end{document}
