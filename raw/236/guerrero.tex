\documentclass[output=paper]{../langscibook}
\ChapterDOI{10.5281/zenodo.5643285}
\author{Alonso {Guerrero Galván}\orcid{0000-0003-1751-1109}\affiliation{Instituto Nacional de Antropología e Historia (INAH)} and Nadiezdha {Torres Sánchez}\orcid{0000-0002-7774-7829}\affiliation{Universidad Nacional Autónoma de México}}
\title{El \emph{habitus} lingüístico de tres redes indígenas: Otomí, chichimeca y tepehuano del sureste} 
\abstract{El presente trabajo es un ejercicio metodológico para describir los espacios de uso de las dos o más lenguas que se hablan en una comunidad multilingüe a partir de la noción de mercado y habitus lingüístico. De tal suerte, se analizan los espacios de uso del español y la lengua indígena (otomí, chichimeca jonaz y tepehuano del sureste) en tres redes comunitarias. El principal interés es, por un lado, comparar tres realidades sociolingüísticas: El Espíritu, comunidad otomí en la que la lengua indígena está entrando en un estado de obsolescencia; Misión de Chichimecas, en donde el español parece desplazar a la lengua indígena; y Santa María de Ocotán, espacio en el que la lengua indígena es ampliamente usada por los miembros de la comunidad. Por otro lado, se  propone una metodología conjunta que permita hacer dicha comparación y empezar a plantear una tipología de comunidades multilingües. 

% \keywords{bilingüismo, realidad sociolingüística, mercado lingüístico, \textit{habitus} lingüístico, ámbitos de uso }
}

\IfFileExists{../localcommands.tex}{
  \addbibresource{../localbibliography.bib}
  \input{../localpackages}
  %Copy this to localcommands.tex

\usepackage[english]{babel}
\usepackage{amsmath}
\usepackage{amssymb,amsfonts,textcomp}
\usepackage{array}
\usepackage{hhline}
\usepackage{hyperref}

\newenvironment{styleStandard}{}{}
\newenvironment{stylelsAbstract}{}{}
\newenvironment{stylelsSectioni}{}{}
\newenvironment{stylelsSectionii}{}{}
\newenvironment{stylelsBulletList}{}{}
\newenvironment{styleBibliographyi}{}{}
\newenvironment{listWWNumxxvleveli}{}{}
\newenvironment{listWWNumxxvlevelii}{}{}
\newenvironment{listWWNumxxvleveliii}{}{}
\newenvironment{listWWNumxxvleveliv}{}{}
\newenvironment{listWWNumixleveli}{}{}
\newenvironment{listWWNumixlevelii}{}{}
\newenvironment{listWWNumixleveliii}{}{}
\newenvironment{listWWNumixleveliv}{}{}

\newcommand\textstyleListLabelxvi[1]{#1}
\newcommand\labellistWWNumxxvleveli{\thelistWWNumxxvleveli.}
\newcommand\labellistWWNumxxvlevelii{\thelistWWNumxxvlevelii.}
\newcommand\labellistWWNumxxvleveliii{\thelistWWNumxxvleveliii.}
\newcommand\labellistWWNumxxvleveliv{\thelistWWNumxxvleveliv.}
\newcommand\labellistWWNumixleveli{[F0B7?]}
\newcommand\labellistWWNumixlevelii{\textstyleListLabelxvi{o}}
\newcommand\labellistWWNumixleveliii{[F0A7?]}
\newcommand\labellistWWNumixleveliv{[F0B7?]}

\newcounter{listWWNumxxvleveli}
\newcounter{listWWNumxxvlevelii}[listWWNumxxvleveli]
\newcounter{listWWNumxxvleveliii}[listWWNumxxvlevelii]
\newcounter{listWWNumxxvleveliv}[listWWNumxxvleveliii]
\newcounter{itemize} 
  \input{../localhyphenation}
  \togglepaper[1]%%chapternumber
}{}

\shorttitlerunninghead{El \textup{habitus} lingüístico de tres redes indígenas}
\begin{document}
\maketitle 
\shorttitlerunninghead{El \textup{habitus} lingüístico de tres redes indígenas}

 \section{Introducción}


La identidad lingüística es algo que se va moldeando con la práctica social del lenguaje y el uso de una o varias lenguas o dialectos, es decir, no se trata de algo fijo, pues en un contexto determinado es negociable. El hablante tiene a su disposición la elección de ciertas formas lingüísticas a las que asigna un valor dependiendo de diversos factores tales como el interlocutor, el espacio de interacción, el estigma social que puede tener la lengua en uso, el contenido del mensaje o las implicaciones sociales que puede representar enunciarlo o no.

En este sentido, consideramos a la identidad lingüística como una construcción individual, la cual se forja a partir de la negociación dentro de la comunidad o comunidades de habla en las que se participa (\citealt{Niño-MurciaRothman2008}). Esta interacción tiene su locus en los dominios de uso que conforman un determinado \textit{habitus} lingüístico, el cual es producto de las condiciones sociales e históricas de cada comunidad \citep{Bourdieu1990}.\footnote{\citet[210]{BlommaertEtAl2005} estudian las diferentes competencias multilingües a partir de un análisis crítico a las posturas de Goffman y Bourdieu y advierten que "[...] in Goffman's case, due to a lack of attention to how the social order structures and formats situations and practices; in Bordieu's case, due to an over-generalization of the case of the unified symbolic market", sugieren que "we need to address space and differences and relations between spaces as aspects of communication", pero sobre todo abogan por pensar en el espacio desde una perspectiva semiótica y escalar (2005: 212, 213). En nuestro caso, hacemos uso del concepto de mercado y \textit{habitus} lingüístico como un mecanismo que permita describir las decisiones que toma el hablante en la selección de una o más lenguas en un dominio de uso determinado (i.e. las fiestas tradicionales, la casa, las juntas, la clínica, etc.), sin profundizar en las diversas competencias comunicativas o estrategias discursivas que sigan en cada uno de estos espacios. De esta forma, enfatizamos las diferencias y semejanzas en la realidad sociolingüística de cada comunidad en las que se hablan dos o más lenguas.} El presente trabajo analiza los espacios de usos del español y la lengua indígena (otomí, chichimeca jonaz y tepehuano del sureste) en tres redes comunitarias: i) El Espíritu, comunidad otomí en la que la lengua indígena está entrando en un estado de obsolescencia; ii) Misión de Chichimecas, en donde el español parece desplazar a la lengua indígena, y iii) Santa María de Ocotán, espacio en el que la lengua indígena es ampliamente usada por los miembros de la comunidad.

En resumen, nuestro principal interés es comparar tres realidades sociolingüísticas a partir de una misma metodología, de tal suerte que, para determinar los distintos dominios de uso de las lenguas, se analizan los resultados obtenidos tanto en el trabajo etnográfico, como en el empleo de un mismo cuestionario sociolingüístico (\citealt{GuerreroGalván2009ms}), y se implementa la noción de \textit{habitus} lingüístico para describirlos y esquematizarlos.


 \section{Justificación}


La descripción sociolingüística de las diferentes comunidades en las que se hablan dos o más lenguas, en su mayoría, se ha hecho con base en el término de diglosia. Por lo tanto, es común que, al describir las comunidades bilingües, estas también se definan como diglósicas y que, de alguna manera, se homogenicen las diferentes características propias de una comunidad de habla. Sin embargo, el termino de diglosia expuesto por \citet{Ferguson1959} ha tenido un sinnúmero de modificaciones con el fin de adecuarlo a la realidad que cada investigador observa en la comunidad bajo estudio (véase \citealt{Zimmermann2010}). Esto refleja por un lado el poco dinamismo del término; es decir, que para poder mostrar las diferencias entre las comunidades es necesario modificarlo hasta el punto en que ya no se puede relacionar con su sentido inicial; y, por otro lado, si bien dichas adecuaciones muestran la diversidad de realidades sociolingüísticas de las diferentes comunidades, no permite la comparación entre ellas en aras de tener una tipología de comunidades multilingües.

  No obstante, si se considera que cada comunidad lingüística o de práctica construye una serie de disposiciones o esquemas generadores de conductas lingüísticas o \textit{habitus} \citep[114]{Bourdieu1990}, que determina que el uso de una o más lenguas es más o menos ventajoso para determinada coyuntura situacional y discursiva, como lo afirma  \citet[6, 221]{AlonsoBenito2004}.

\begin{quote}
Los actos particulares de habla [...] no se producen como actos racionalizados, individualizados y calculadores, sino como exteriorización práctica de un \textit{habitus} que aquí es un \textit{habitus} lingüístico, definido por un conjunto relacionado de disposiciones adquiridas, esquemas de percepción y de apreciación de la realidad, así como de actuación en ella, incalculados en un contexto social y una situación histórica determinada. El \textit{habitus} es simultáneamente productor de prácticas, mediador entre las relaciones socialmente objetivas y los comportamientos individuales, producto, a su vez también, de la interiorización de las condiciones objetivas y de las estrategias de adaptación de los actores de un campo.
\end{quote}

Es posible hacer uso del concepto de \textit{habitus} para la descripción de los usos diferenciados de las lenguas habladas en una comunidad. Así, Bourdieu lo considera algo implantado en los individuos de manera preconsciente, por lo que para poder describir en profundidad los diferentes \textit{habitus} es necesario contar con un registro etnográfico que nos permita develar la lógica comunitaria, las formas de organización, los ciclos rituales y las exegesis culturales que los configuran simbólicamente. En este trabajo, para este fin, nos valemos de las notas etnográficas hechas por los autores en distintas temporadas de campo en las comunidades bajo estudio, así como de las etnografías realizadas y publicadas por distintos investigadores de la región, tal y como se expone en los apartados de cada comunidad.

De esta forma, a partir de la noción de \textit{habitus}, se plantea la descripción de tres comunidades, cuya selección obedece al hecho de que, en principio, son totalmente ajenas unas a otras, pues no mantienen lazos comerciales, religiosos o parentales. Se ubican en regiones muy diferentes y han enfrentado procesos históricamente muy distintos en cuanto a su incorporación al Estado Nacional y la implantación del español y su uso en las comunidades. Además, las lenguas indígenas estudiadas pertenecen a ramas y familias distintas, de tal forma que el otomí y el chichimeca pertenecen a la familia otopame, pero el primero es de la rama central u otomiana y el segundo de la norteña o pameana, mientras que el tepehuano del sureste es de la rama pimana de la familia yutoazteca. Finalmente, y como se explica más adelante, la situación lingüística que ejemplifican estas comunidades ilustra un \textit{continuum} que va de la obsolescencia al desplazamiento y de éste al mantenimiento de la lengua indígena. De esta forma, se pondrá a prueba la eficacia del \textit{habitus} como noción explicativa de los usos de las lenguas en los distintos ámbitos, y, además, se dará cuenta de los factores a los que estas comunidades se enfrentan y de cómo han generado conductas lingüísticas tan distintas en algunos aspectos y tan parecidas en otros.

Es importante mencionar que, con el fin de ampliar la visión ética de la etnografía y contar con la participación directa de las comunidades (visión émica), se pensó en un instrumento que nos permitiera recolectar datos más homogéneos y que pudieran ser comparables, por lo que se eligió la \textit{Encuesta para peritaje lingüístico} (\citealt{GuerreroGalván2009ms}), que se describe en el apartado de metodología. Trabajamos bajo la hipótesis de que instrumentos como esta encuesta, registrada de manera oral\footnote{Siguiendo lo expuesto por   \citet[65]{Silva-Corvalán2001} para quien “la encuesta oral es la más aconsejable porque permite al encuestador investigar más a fondo las respuestas de los informantes y no ata a estos a dos o tres posibilidades preestablecidas".}, nos permiten abordar la realidad desde una perspectiva cercana a la comunidad de estudio, arrojando datos de gran valor descriptivo y comparativo.

Así pues, nuestra justificación es meramente metodológica y documental, pues consideramos que la realidad sociolingüística de México es compleja y aún no se encuentra del todo descrita, por lo que es necesario empezar a implementar metodologías unificadas que ayuden a describir dichas realidades.\footnote{Para tener una idea general de los trabajos sobre la variación en lenguas indígenas, su contacto con el español y bilingüismo en las comunidades indígenas en México véase \citet{Cifuentes1998};  \citet{FloresFarfán1998};  \citet{GuerreroGalvánSanGiacomo2014}; \citet{Hekking1995};    \citet{HillHill1986}; \citet{Levy1990}; \citet{Lastra2003}; \citet{Pfeiler1988};  \citet{SmithStark2007}; \citet{Suárez1977}; \citet{Villavicencio2006} y \citet{Zimmermann1987,Zimmermann2010}, por mencionar algunos.}


 \section{Metodología}


Los ámbitos de uso lingüístico conforman el marco de las interacciones entre hablantes, y en este sentido, sintetizan los campos en los que se desarrolla el \textit{habitus} lingüístico de la comunidad \citep[109]{Bourdieu1990}. Así, identificamos tres distintos campos en los que se han desarrollado \textit{habitus} relacionados con el uso de las lenguas indígenas y el español: i)  la familia y el ámbito de la casa son un campo nuclear donde las lenguas se transmiten, donde los individuos comparten intereses más cercanos (consanguíneos, políticos o rituales) y donde, también, suele haber una mayor solidaridad con el interlocutor; ii) la comunidad, que tiene ámbitos altamente simbólicos y culturalmente determinados, como la organización ritual (fiestas) o la forma de organización interna (juntas), y otros que los ponen en contacto directo con los intereses de otras comunidades, como la iglesia, la escuela y el mercado, que también pueden ubicarse dentro de iii) el campo extracomunitario, en donde entran en juego intereses municipales, regionales y estatales.

  Ahora bien, para intentar delinear cómo se estructuran estos diferentes campos y los ámbitos de uso que los comprenden, se exploran los resultados arrojados por la \textit {Encuesta para peritaje lingüístico}, diseñada para el proyecto de Normatividad de Variación en Lenguas Otopames de la Dirección de Lingüística del INAH. Este instrumento busca establecer una metodología específica de encuesta oral para la obtención de datos sociolingüísticos en situaciones de contacto y para el estudio del cambio lingüístico. Contempla los siguientes aspectos: (a) datos sociodemográficos, (b) datos socioeconómicos, (c) adquisición y competencia lingüística, (d) uso de las lenguas y ámbitos lingüísticos, (e) actitudes y creencias, (f) identidad y cultura y (g) variación lingüística.

Para este trabajo revisamos los resultados obtenidos en el inciso (d) uso de las lenguas y ámbitos lingüísticos, apartado que contiene dos tipos de preguntas. El primero se interesa por identificar a los interlocutores en cada una de las lenguas en relación con la familia nuclear y extensa, parentesco ritual, amigos y extraños, intentando hacer una gradación en términos de una mayor o menor solidaridad y, también, se inquiere si hay una diferencia en el uso de la lengua con respecto a los hijos, por ser mayores o menores, como se esquematiza en el \tabref{tab:guerrero:1}.

\begin{table}
\caption{\label{tab:guerrero:1} Tipo de interlocutor \citep[305]{GuerreroGalván2016}} 
¿Qué lengua hablas con tu (mamá){\dots}?
\begin{tabular}{lll|lr}
% \lsptoprule
\multicolumn{3}{c}{\textbf{Parentesco}}\\
\textbf{nuclear} & \textbf{extenso}& \textbf{ritual} & \multicolumn{2}{c}{\textbf{externo}}\\
\hline
mamá/papá    &abuela/o  &  compadres  &  amiga/o &  extraña/o\\
hermana/o    &tía/o     &  comadres\\
pareja       &suegra/o&\\
hija/o mayor &prima/o&\\
hija/o menor & nuera& \\
nieta/o      &yerno&\\
\tablevspace
\multicolumn{5}{c}{+\hfill $\longleftarrow$  (solidaridad) $\longrightarrow$ \hfill  -- }\\
% \lspbottomrule
\end{tabular}
 
\end{table} 

El segundo tipo de pregunta indaga sobre los ámbitos de uso de las lenguas y quiénes podrían ser los interlocutores en ellos (véase \tabref{tab:guerrero:2}). Se trata de explorar las relaciones sociales en el mismo eje de poder y solidaridad que se expusieron en el  \tabref{tab:guerrero:1}. El fin es observar si hay un uso determinado de una lengua por la preferencia del hablante o bien si es impuesto por el interlocutor, así como determinar si hay complementariedad de ámbitos de uso o traslape.

\begin{table}
\caption{\label{tab:guerrero:2} Espacios de uso de las lenguas \citep[305]{GuerreroGalván2016}}
% <<<<<<< HEAD
¿Qué lengua utiliza en (casa…) con (la familia…)?
   \fittable{
\begin{tabular}{p{1.8cm}|@{~}lp{2cm}@{~}|@{~}l@{}p{1.5cm}l@{}}
% \lsptoprule
\textbf{Espacios}& \multicolumn{4}{c}{\textbf{Interlocutor por espacio de uso}} \\
\textbf{de uso}   & \textbf{consanguíneo} & \textbf{ritual} & \multicolumn{3}{c}{\textbf{externo}}\\
\hline
casa  & familia & compadre/\newline comadre   & amiga/o & extraña/o\\
calle    &         &  &         &          \\
trabajo  &       &           & compañera/o&& jefa/e\\
mercado  &      &           &            && comerciante\\
ciudad   &      &           &            && autoridad    \\
comunidad&      &           &            && delegada/o   \\
escuela  &      &           &            && profesorado  \\
iglesia  &      &           &            && sacerdote    \\
clínica  &      &           &            && médico       \\
fiestas  &      &           &            && curandera/o  \\
juntas   &      &           &            && secretaria/o \\
futbol   &      &           & equipo     & equipo contrario & árbitro  \\
\tablevspace
\multicolumn{1}{l}{}       & \multicolumn{3}{l}{±             } & \multicolumn{2}{r}{±}\\
\multicolumn{1}{l}{}       & \multicolumn{3}{l}{+solidaridad  } & \multicolumn{2}{r}{+ poder}         \\
\multicolumn{1}{l}{}       & \multicolumn{3}{l}{--poder       } & \multicolumn{2}{r}{--solidaridad}    \\
\end{tabular}
   }
% =======
% \begin{tabularx}{\textwidth}{ll@{}lll@{~}l@{}Ql}
% \lsptoprule
% \multicolumn{8}{l}{¿Qué lengua utiliza en (casa…) con (la familia…)} \\
% &  & {\textit{(espacios de uso)}} & {(\textit{consanguíneo})} & {\textit{(ritual)}} &  &  & {(\textit{externo})}\\
% &  & {casa} & {familia} & {compadre} & {amigos} & {extraño} & \\
% &  & {calle} &  &  &  &  & \\
% &  & {trabajo} &  &  & {compañero} &  & {jefe}\\
% &  & {mercado} &  &  &  &  & {comerciante}\\
% &  & {ciudad} &  &  &  &  & {autoridades}\\
% &  & {comunidad} &  &  &  &  & {delegados}\\
% &  & {escuela} &  &  &  &  & {profesores}\\
% &  & {iglesia} &  &  &  &  & {sacerdote}\\
% &  & {clínica} &  &  &  &  & {médico}\\
% &  & {fiestas} &  &  &  &  & {curandero}\\
% &  & {juntas} &  &  &  &  & {secretarios}\\
% &  & {futbol} &  &  & {equipo} & {equipo \mbox{contrario}} & {árbitro}\\
% %\hhline%%replace by cmidrule{~~~-----}
% \multicolumn{8}{c}{\textit{familiaridad} (solidaridad)} \\
% &  &   {(+solidaridad)}& & &  &  & {(+ poder)}\\
% &  &           {{} ± } & &  &  &  & \hfill {{} ± } \\
% %\hhline%%replace by cmidrule{~~~-----}
% \lspbottomrule
% \end{tabularx}
% >>>>>>> d28cc0d6a2c4cc1cca41e56993a4770212d9f75f
\end{table}

Los datos que aquí se exponen pertenecen a este segundo grupo de preguntas: nos enfocamos en la identificación jerárquica de ámbitos lingüísticos con el fin de obtener un esquema de las estrategias de uso de las lenguas en un determinado espacio o situación comunicativa, para así determinar el \textit{habitus} lingüístico de cada lengua en los tres contextos sociolingüísticos aquí estudiados. Los ámbitos de uso que se encuentran en la encuesta fueron definidos a partir de los posibles interlocutores y los espacios físicos y simbólicos que pueden existir en la realidad mexicana. Aunque no pretende ser exhaustiva, busca explorar los campos reconocidos más explícitos de interacción entre los hablantes de una comunidad dada (la familia, la comunidad y el exterior de la comunidad).

Ahora bien,  \citet[2, 217]{AlonsoBenito2004} argumenta que la configuración particular del mercado lingüístico, así como los campos que abarca, dependerá de las condiciones históricas y sociolingüísticas de cada comunidad, y que, por lo tanto, "los discursos no son otra cosa que las jugadas prácticas con la que los sujetos que intervienen en un mercado lingüístico, tratando de aumentar sus beneficios simbólicos, adaptándose a las leyes de formación de los valores y a la vez poniendo en juego su capital lingüístico, social y culturalmente codificado".

De esta manera, para esquematizar los \textit{habitus} lingüísticos de cada comunidad y los mercados en que participan (cómo interactúan los campos con los ámbitos de uso de cierta lengua), es necesario representar las distintas fuerzas que participan de la interacción, a saber: (i) en un campo familiar o doméstico, que tiene como escenario la casa, se implica un nivel personal, relacionado con la familia consanguínea, y se tiene como interlocutores (actores) a los padres, tíos, padrinos, etcétera; (ii) en un campo comunitario, donde se sitúan las fiestas o las juntas, se tiene como interlocutores a los delegados, autoridades tradicionales o a los especialistas rituales; y (iii) en un campo más amplio como el municipio, la región y el Estado, se tiene interlocutores que generalmente representan la otredad, personas externas a la comunidad, como los patrones, los extraños o los desconocidos. En este sentido, los resultados se presentan siguiendo el esquema de campos de la \figref{fig:guerrero:1}, el cual está estructurado dependiendo de la comunidad de estudio y representa el \textit{habitus} lingüístico relatado por los colaboradores en cuanto a la selección y uso de una o varias lenguas en cada campo.




\begin{figure}
\caption{\label{fig:guerrero:1} Esquema de los campos y \emph{habitus} lingüísticos \citep[309]{GuerreroGalván2016}}
% \includegraphics[width=\textwidth]{figures/guerrero-img001.jpg}
\includegraphics[width=\textwidth]{figures/guerrero-img001.pdf}
\end{figure}

Así, debido a que históricamente las tres comunidades hablan una lengua indígena, nos proponemos indagar cuáles son los campos en que se desarrollan \textit{habitus} que llevan a sus hablantes a cambiar de lengua, en este caso al español, o en los que se opta por hablar indistintamente en una u otra lengua, es decir de manera bilingüe. Esta selección depende del valor de intercambio que tiene cada lengua en uso, es decir, el conjunto de intercambios de productos lingüísticos conforma un mercado lingüístico específico. Estos mercados operan en los campos ya mencionados (mercado doméstico, comunitario, municipal, regional, estatal, etcétera), yendo de lo micro a lo macro-social.



 \subsection{El muestreo en redes}



Debido a la imposibilidad de encuestar a la totalidad de la población de cada comunidad, por las cuestiones económicas y de capital humano que implica, es necesario llevar a cabo un muestreo de la población estudiada (selección de los hablantes). Hay distintas maneras de seleccionar una muestra dependiendo de los objetivos de la investigación,\footnote{“Hay básicamente dos métodos de selección de una muestra de hablantes: un método \textit{survey} en el que la selección se hace siguiendo técnicas estrictas de muestreo al azar; y un método de selección intencionada en el que: (a) las características de los hablantes han sido predeterminadas y éstos se seleccionan más o menos al azar hasta completar el número deseado de individuos en cada categoría social; o (b) los hablantes seleccionados constituyen un grupo social compacto, ya sea porque son miembros de una red social, porque viven en la misma manzana o vecindario, porque tienen un dominio común de trabajo (por ejemplo una escuela) etc. Cualquiera que sea el método que el investigador emplee, deberá decidir qué factores extralingüísticos incluirá en el análisis” (\citealt{Silva-Corvalán2001}: 51).} y en este caso, se obtuvieron los datos a partir del trabajo con redes de relaciónes sociales, retomando la noción de "el amigo de un amigo". Con este método, el investigador establece una serie de relaciones de colaboración con algunos miembros de la comunidad, quienes a su vez presentan al investigador con otras personas con las que están vinculadas, y así sucesivamente. Una vez obtenido el muestreo, se realiza una postestratificación en cuanto a las variables sociales de género, edad, educación formal, lugar de residencia, entre otros. Es importante resaltar que la red como instrumento analítico permite analizar los vínculos de interpretaciones sobre la conducta social que tienen los actores implicados en la red. Para registrar una estructura social de este tipo tenemos que determinar su localización o anclaje (los mercados en los que participa), ver la posición que ocupan los actores (central o periférica), su grado de autonomía con respecto a los demás (su accesibilidad), y el número de vínculos que mantiene con otros actores (su rango). El total de vínculos que existen dentro de la red determinan su densidad (véase \citealt[137-138]{RequenaSantos1989}). Un ejemplo de este tipo de redes se presenta en la \figref{fig:guerrero:2}, para Santa María de Ocotán. En ella se observan las distintas relaciones de amistad, laborales y de contacto, así como los diferentes núcleos de relación, como la familia, el trabajo o individuos independientes.


\begin{figure}
\caption{\label{fig:guerrero:2} Red de colaboradores de Santa María de Ocotán (\citealt{TorresSánchez2018})}
% \includegraphics[width=\textwidth]{figures/guerrero-img002.png}
\includegraphics[width=\textwidth]{figures/guerrero-img002.pdf}
\end{figure}

Debido a limitaciones de espacio no nos es posible detallar aquí todos estos aspectos, únicamente se presenta un resumen que esboza los vínculos históricos y sociolingüísticos de cada comunidad, según lo manifestaron en la encuesta los actores de las redes estudiadas.


 \section{Redes indígenas estudiadas}


Como se mencionó en los aparados anteriores, la encuesta se realizó en tres redes de distintas comunidades indígenas con diversos grados de vitalidad lingüística. Se trata de la comunidad de El Espíritu, Municipio de Alfajayucan (Hidalgo), en donde el \textit{hñähñu} u otomí prácticamente ha sido desplazado por el español; la comunidad de Misión de Chichimecas, San Luis de la Paz (Guanajuato), última localidad en donde se habla la lengua \textit{úza̱}’ o chichimeca jonaz, por lo que se considera que está en un proceso de desplazamiento; y Santa María de Ocotán, Mezquital (Durango), en donde la lengua con mayor uso es el \textit{o’dam} o tepehuano del sureste.



 \subsection{Red otomí de El Espíritu}



El otomí o \textit{hñähñu} es una lengua otopame de la rama otomiana (junto con el mazahua, el matlazinca y el ocuilteco), que se habla en ocho estados de la República Mexicana (México, Hidalgo, Querétaro, Guanajuato, Michoacán, Tlaxcala, Puebla y Veracruz). Generalmente se habla de su fragmentación en tres grandes grupos de variantes dialectales,
i) las variantes orientales o de la Sierra Oriental, que abarca los estados de Hidalgo, Puebla, Veracruz y Tlaxcala;
ii) las variantes de occidentales del norte y el Valle del Mezquital, que tiene como centro el estado de Hidalgo y se extiende al norte del Estado de México, Querétaro y el sureste de Guanajuato; y
iii) las variantes del sur de los estados de Querétaro, México y Michoacán (\citealt{Soustelle1990}; \citealt{Lastra2006};  \citealt{GuerreroGalván2013}).

Esta lengua se ha mantenido en contacto intenso con el español desde el siglo XVI, principalmente en la región del Valle del Mezquital, donde las variantes han tenido cambios más profundos y se encuentran en franco desplazamiento lingüístico. Este último es el caso del otomí de la comunidad de El Espíritu, fundada como una visita franciscana del vecino convento de San Martín Alfajayucan, que se encuentra a menos de un kilómetro de distancia. La proximidad del convento llevó a un contacto cultural intenso debido a la presencia de los frailes. Sin embargo, en relación con el contacto lingüístico, este se dio de manera intensa hasta el siglo XIX, trayendo consigo el inicio de un proceso de estigmatización y represión lingüística del otomí en los espacios públicos de la comunidad tales como la escuela y la iglesia.

Si bien la lengua ya sólo es hablada por personas mayores de 60 años, la población conserva una parte de su identidad otomí en la celebración del Carnaval, el cual reúne de manera sincrética el culto a las potencias de la naturaleza, el santoral católico y el ciclo agrícola. El carnaval de El Espíritu se articula con los de otras tres comunidades:
i) Xamange;
ii) San Antonio Corrales, y
iii) Boxtho. Los comparsas o \textit{xithás} (abuelos) visitan la iglesia y desfilan bailando por el pueblo del Alfajayucan. Su atuendo representa a los antepasados de la comunidad, seres inframundanos que fertilizan las tierras con sus batallas rituales. En El Espíritu, el momento liminal del ritual es marcado por la “barrida” del templo, pues esto ahuyentará a los malos aires, por lo que se realiza en Año Nuevo y antes del Carnaval.

No es posible saber el número exacto de hablantes de otomí en El Espíritu, pero según el censo de 2010, el 33.93 \% de los habitantes de la comunidad de 3 años y más hablan una lengua indígena, el 96.82 \% de ellos son bilingües (lengua indígena-español) y no se registró población que no hablara español. Si bien se trata de una comunidad pequeña (134 viviendas) fundada por familias otomíes, la población en hogares indígenas de 5 años y más representa sólo el 61.57 \%; es decir que el 38.43 \% de los hogares no se consideran indígenas. A nivel municipal los hablantes de lengua indígena representan tan sólo un 18.35\% de la población (véase \tabref{tab:guerrero:3} (Fuente: INEGI2010)).\footnote{El Censo de 2010 fue revisado por medio de la “Consulta interactiva de datos” y los “Datos abiertos” del INEGI (véase \url{http://www.inegi.org.mx/lib/olap/consulta/general\_ver4/MDXQueryDatos.asp?c=27781} consultado el 01 de mayo de 2020).}

% \begin{table}
% \small
% \caption{\label{tab:guerrero:3} Población de 3 años y más en El Espíritu. Fuente: \citet{INEGI2010}} \textit{mexico.pueblosdeamerica.com}
% \begin{tabularx}{\textwidth}{QrrrQ}
% \lsptoprule
% { {El}  {Espíritu} } & { {mujeres}} & { {hombres}} & { {total}} & { {alcances}  {y}  {porcentajes}}\\
% \midrule
% {Población} & {287} & {270} & {557} & {(RM/H 1.081)}\\
% \tablevspace
% {hablante de lengua indígena} & {98} & {91} & {189} & {33.93 \% (15.76 \%)}\\
% \tablevspace
% {no habla español} & {0} & {0} & {0} & {0}\\
% \tablevspace
% {habla español y lengua indígena} & {94} & {89} & {183} & {96.82 \%}\\
% \tablevspace
% {en hogares indígenas} & --- & --- & {343}  & {61.57 \% (población de 5 años y más)}\\
% \tablevspace
% {grado de escolaridad} & {7.16} & {6.88} & {7.04}  & {Adultos}\\
% \tablevspace
% {analfabetismo} & {(7.46\%)} & {(4.84\%)} & --- & {6.2\%}\\
% \tablevspace
% {viviendas} & --- & --- & {134} & {Comunidad}\\
% \tablevspace
% {Alfajayucan} & {8819} & {8304} & {17123} & {Municipio}\\
% \tablevspace
% {hablante de lengua indígena} & {1559} & {1584} & {3143} & {Municipio}\\
% \tablevspace
% {hablante de lengua indígena} & {187465} & {182084} & {369549} & {Estado}\\
% \lspbottomrule
% \end{tabularx}
% \end{table}

\begin{table}
\caption{\label{tab:guerrero:3} Datos demográficos de Alfajayucan y El Espíritu \citep{INEGI2010}}
\begin{tabularx}{\textwidth}{Xrrrr}
\lsptoprule
Localidad & \multicolumn{2}{p{2.5cm}}{Alfajayucan\newline (municipio)} & \multicolumn{2}{p{2.5cm}}{El Espíritu\newline (comunidad)}\\
\midrule
Población total &  17\,123 && 557 &\\
Hablante de lengua indígena & 3\,143 &18.35 \%& 189 & 33.93 \%\\
Habla español y lengua indígena & 1\,528 &48.61 \%& 183 & 96.82 \% \\
No habla español & 30 & 0.95 \% &  0& 0 \% \\
En hogares indígenas (5 años y más)& 6\,447 & 37.65 \%&  343 & 61.57 \% \\
Viviendas habitadas & 4\,930  &&  134 & \\
\lspbottomrule
\end{tabularx}
\end{table}


Dentro de esta primera comunidad la encuesta se realizó a una red familiar de 24 miembros, todos vecinos de El Espíritu, 11 mujeres y 13 hombres, lo que es cercano al 4\% de la población femenina y el 5\% de la masculina.


 \subsubsection{Ámbitos de uso en El Espíritu}


Los resultados de la encuesta advierten, en primer lugar, que no hay un espacio en el que la lengua otomí tenga una preferencia de uso, de tal suerte que en los espacios que está presente con mayores porcentajes, siempre está junto con el español. De igual forma, si observamos el uso del español, podemos comprobar que esta lengua es la que los colaboradores dicen usar en la mayoría de los espacios, destacando incluso que, en un ámbito como la casa en el que se podría esperar que predominara el otomí, es el español el que obtiene el 67\% frente al 25\% de uso de ambas lenguas y el 8\% del otomí (véase \figref{figtab:guerrero:4}).

\begin{figure}
\caption{\label{figtab:guerrero:4} Ámbitos de uso de El Espíritu}
% \begin{tabularx}{\textwidth}{XXXXXXXXX}
% \lsptoprule
% {{casa}} & {{fiestas}} & {{juntas}} & {{trabajo}} & {{mercado}} & {{iglesia}} & {{escuela}} & {{clínica}} & {{ciudad}}\\
% \midrule
% {{es} {67} {\%} {(16/24)}} & {{bl} {46} {\%} {(11/24)}} & {{bl} {71} {\%} {(17/24)}} & {{es} {62} {\%} {(15/24)}} & {{es} {67} {\%} {(16/24)}} & {{bl} {58} {\%} {(14/24)}} & {{es} {83} {\%} {(20/24)}} & {{es} {87} {\%} {(21/24)}} & {{es} {83} {\%} {(20/24)}}\\
% \tablevspace
% {{bl} {25} {\%} {(6/24)}} & {{es} {46} {\%}  {(11/24)}} & {{es} {29} {\%} {(7/24)}} & {{bl} {21\%}\newline {(5/24)}} & {{bl} {29} {\%} {(7/24)}} & {{es} {42} {\%} {(10/24)}} & {{bl} {12} {\%}  {(3/24)}} & {{bl} {12} {\%}  {(3/24)}} & {{bl} {12} {\%}  {(3/24)}}\\
% \tablevspace
% {{ot} {8} {\%} {(2/24)}} & {{ot} {8} {\%} {(2/24)}} &  & {{ot} {8} {\%} {(1/24)}} & {{ot} {8} {\%} {(1/24)}} &  &  &  & \\
% \lspbottomrule
% \end{tabularx}


% \pgfplotstableread{
%  %es %bl %ot
% 0 16   6   2
% 1 11  11   2
% 2 7   17   0
% 3 15   5   1
% 4 16   7   1
% 5 10  14   0
% 6 20   3   0
% 7 21   3   0
% 8 20   3   0
% }\dataset
\pgfplotstableread{
 %es %bl %ot
0 67   25   8
1 46  46   8
2 29   71   4
3 62   21   0
4 67   29   4
5 42  58   0
6 83   12   0
7 87   12   0
8 83   12   0
}\dataset
\fittable{
\begin{tikzpicture}
\begin{axis}[ybar,
        width=16cm,
        height=8cm,
        ymin=0,
        ymax=100,
        xlabel={Ámbito},
        xlabel style = {yshift=-3mm},
        ylabel={Uso \%},
        xtick=data,
        xticklabels = {
            casa\strut,
            fiestas\strut,
            juntas\strut,
            trabajo\strut,
            mercado\strut,
            iglesia\strut,
            escuela\strut,
            clínica\strut,
            ciudad\strut
        },
        nodes near coords,
        xticklabel style={rotate=0},
        major x tick style = {opacity=0},
        minor x tick num = 1,
        minor tick length=1ex,
        every node near coord/.append style={
                anchor=center,
                yshift=2mm
%                 rotate=90
        },
        legend columns=3,
        legend style={at={(0.005,.85)},anchor=south west},
        axis x line*=bottom,
        axis y line*=left,
        ]
\addplot[draw=black,fill=lsMidBlue] table[x index=0,y index=1] \dataset; %es
\addplot[draw=black,fill=black!30] table[x index=0,y index=2] \dataset; %bl
\addplot[draw=black,fill=lsMidOrange] table[x index=0,y index=3] \dataset; %ot
\legend{español, bilingüe, otomí}
\end{axis}
\end{tikzpicture}
}
\end{figure}

La \figref{fig:guerrero:3} representa los distintos ámbitos de uso en relación con los campos de la comunidad, el municipio, la región y el Estado. La \figref{figtab:guerrero:4} nos muestra una preferencia por el español en los niveles regionales y el uso del bilingüismo en los ámbitos comunitarios de la fiesta, la iglesia y las juntas, que en este caso también son espacios de interacción con la vecina cabecera municipal. Como se mencionó en el párrafo anterior, llama la atención el uso predominante del español en un espacio privado como el de la casa.


\begin{figure}
% \includegraphics[width=\textwidth]{figures/guerrero-img003.jpg}
\includegraphics[width=\textwidth]{figures/guerrero-img003.pdf}
\caption{\label{fig:guerrero:3} Esquema del \textit{habitus} lingüístico en El Espíritu (elaborada por los autores)}
\end{figure}



 \subsection{Red chichimeca de Misión de Chichimecas}



El \textit{úza̱}’ o chichimeca jonaz es una lengua de la familia otopame, de la rama pameana (junto con el pame de norte y el del sur). Los hablantes de esta lengua se mantuvieron reacios al contacto con los españoles durante la época colonial, participando activamente en la llamada “guerra chichimeca”, que se prolongó hasta mediados del siglo XVII cuando los españoles implementaron la “paz por compra” y destinaron una renta a los chichimecas para que no siguieran atacando sus recuas de mulas que circulaban con mercancías por el Camino Real de Tierra Adentro. El grupo que sobrevivió de estos indómitos guerreros fueron denominados chichimeca jonaz o \textit{eza̱’r}, como ellos se autodenominan.

A pesar de su tendencia al aislamiento y a mantenerse como un grupo relativamente pequeño, los \textit{eza̱’r} han tenido que ir abriéndose al contacto, sobre todo desde los inicios del siglo XX. Su lengua está experimentando cambios importantes, como por ejemplo el paso de /y/ a /i/ y el de /ts/ a /s/, que probablemente se han ido acelerado por su contacto y convergencia con el español \citep{GuerreroGalvánEtAl2017}.

La comunidad de Misión de Chichimecas está situada a tres kilómetros de la cabecera municipal de San Luis de Paz, Guanajuato. Esta cercanía ha hecho que la situación comunitaria empiece con un proceso de desplazamiento de la lengua indígena por el español, por ser esta última la lengua mayoritaria y de prestigio a nivel nacional.

En los años setenta, la comunidad fue dividida por libramiento carretero hacia el municipio de Victoria, lo cual separó a la Misión en Misión de Abajo y Misión de Arriba. Misión de Abajo fue el primer asentamiento \textit{eza̱’r}, además de ser el espacio donde se colocaron las primeras escuelas y servicios públicos tales como luz y drenaje. También, debido a la cercanía con la cabecera municipal, empezó a ser una zona de avecindados mestizos lo cual cambió la dinámica del uso del \textit{úza̱}. Es importante destacar que también se encuentran mestizos avecindados en Misión de Arriba, pero en menor porcentaje, por lo que la mayoría de la población es chichimeca. Ambas parcialidades celebran la fiesta de la Virgen de Guadalupe, a quien consideran patrona y ofrendan un gran \textit{chimal} o \textit{chich’á}.\footnote{El chimal o \textit{chich’á} es una estructura de madera y metal de varios metros de altura, tejida artísticamente con la planta del mismo nombre, perteneciente a la familia de los agaves como el sotol y el mezcal. Esta ofrenda es cargada en hombros por los \textit{eza’r} y acompañada por danzas de guerreros chichimecas desde la Capilla de Cerrito en Misión de Abajo hasta el Santuario de Guadalupe, que dista unos tres kilómetros.}

También es difícil determinar el número exacto de hablantes de lengua indígena en Misión de Chichimecas. El censo de 2010 registra 2011 habitantes de 3 años y más que hablan una lengua indígena, lo que representa el 29.94\% de la población total registrada. El censo presenta un monolingüismo (no habla español) del 2.73\% y un bilingüismo (lengua indígena-español) del 95.82\%. A nivel municipal se cuentan 2273 hablantes de lengua indígena, los cuales representan el 2.22\% de los habitantes del municipio. En Misión se registran 6716 habitantes, por lo que, si todos los hablantes de lengua indígena registrados para el municipio en el censo vivieran en esta comunidad, representarían el 33.84\% de la población. No obstante, en Misión los hogares indígenas representan el 51.93\% de la población de 5 años y más, en una localidad con 1157 viviendas (véase \tabref{tab:guerrero:4}).

% \begin{table}
% \caption{\label{tab:guerrero:5} Población de Misión de Chichimecas}
% Fuente: \citet{INEGI2010}
%
% \begin{tabularx}{\textwidth}{QrrrQ}
% \lsptoprule
% {{Misión} {de} {chichimecas}} & {{mujeres}} & {{hombres}} & {{total}} & {{alcances} {y} {porcentajes}}\\
% \midrule
% {Población} & {3386} & {3330} & {6716} & {(RM/H 1.008)}\\
% \tablevspace
% {hablante de lengua indígena} & {969} & {1042} & {2011} & {30.76 \%}\\
% \tablevspace
% {no habla español} & {34} & {21} & {55} & {2.73 \%}\\
% \tablevspace
% {habla español y lengua indígena} & {936} & {991} & {1927} & {95.82 \%}\\
% \tablevspace
% {en hogares indígenas} & --- & --- & {3488} & {61.41 \% (población de 5 años y más)}\\
% \tablevspace
% {grado de escolaridad} & {3.34} & {3.86} & {3.59}  & {Adultos}\\
% \tablevspace
% {analfabetismo} & {(21.83\%)} & {(13.94\%)} & --- & {17.9\%}\\
% \tablevspace
% {viviendas} & --- & --- & {1157} & {Comunidad}\\
% \tablevspace
% {Población de San Luis de la Paz} & {54241} & {47902} & {102143} & {Municipio}\\
% \tablevspace
% {hablante de lengua indígena} & {1089} & {1184} & {2273} & {Municipio}\\
% \tablevspace
% {hablante de lengua indígena} & {7026} & {8178} & {15204} & {Estado}\\
% \lspbottomrule
% \end{tabularx}
% \end{table}

\begin{table}
\caption{\label{tab:guerrero:4} Datos demográficos de San Luis  de la Paz y Misión de Chichimecas \citep{INEGI2010}}
\begin{tabularx}{\textwidth}{lrrrr}
\lsptoprule
Localidad & \multicolumn{2}{c}{\parbox{2cm}{San Luis\newline  de la Paz\newline (municipio)}} & \multicolumn{2}{c}{\parbox{2.5cm}{Misión de\newline  Chichimecas\newline  (comunidad)}}\\
\midrule
Población total &  102\,143 && 6\,716 &\\
Hablante de lengua indígena&  2\,273 &2.22 \%& 2\,011 & 29.94 \%\\
Habla español y lengua indígena & 1\,089 &47.91 \%& 1\,927 & 95.82 \% \\
No habla español & 55 & 2.41 \% &  55 & 2.73 \% \\
En hogares indígenas (5 años y más)& 4\,062 & 3.97 \%&  3\,488 & 51.93 \% \\
Viviendas habitadas & 24\,263  &&  1\,157 & \\
\lspbottomrule
\end{tabularx}
\end{table}


Los datos de esta segunda comunidad se obtuvieron de un total de 11 colaboradores, 8 hombres y 3 mujeres, lo que representaría el 0.2\% de la población masculina y el 0.8\% de la población femenina radicada en Misión, que, como mencionamos en el párrafo anterior, no necesariamente es hablante \textit{úza̱’}.


 \subsubsection{Ámbitos de uso Misión de Chichimecas}


Para el caso de Misión de Chichimecas podemos observar en la \figref{figtab:guerrero:6} que, de los nueve ámbitos, en tres el chichimeca tiene preferencia de uso: la casa, el mercado y la escuela; y en cuatro se usan las dos lenguas: fiesta, junta, trabajo e iglesia; y tan sólo en dos, la clínica y la ciudad, se destaca el uso del español. A diferencia de El Espíritu podemos observar que en la comunidad de Misión de Chichimecas la lengua indígena sigue estando en uso sin que el español esté presente.

\begin{figure}
\caption{\label{figtab:guerrero:6} Ámbitos de uso de Misión de Chichimecas}


% \begin{tabularx}{\textwidth}{XXXXXXXXX}
% \lsptoprule
% {{casa}} & {{fiestas}} & {{juntas}} & {{trabajo}} & {{mercado}} & {{iglesia}} & {{escuela}} & {{clínica}} & {{ciudad}}\\
% \midrule
% {{ch} {54} {\%} {(6/11)}} & {{bl} {45} {\%} {(5/11)}} & {{bl} {54} {\%} {(6/11)}} & {{bl} {36} {\%} {(4/11)}} & {{ch} {45} {\%} {(5/11)}} & {{bl} {36} {\%} {(4/11)}} & {{ch} {27} {\%} {(3/11)}} & {{es} {54} {\%} {(6/11)}} & {{es} {54} {\%} {(6/11)}}\\
% {{es} {18} {\%} {(2/11)}} & {{ch} {36} {\%} {(4/11)}} & {{ch} {27} {\%} {(3/11)}} & {{ch} {27} {\%} {(3/11)}} & {{bl} {27} {\%} {(3/11)}} & {{es} {36} {\%} {(4/11)}} & {{es} {18} {\%} {(2/11)}} & {{bl} {27} {\%} {(3/11)}} & {{bl} {27} {\%} {(3/11)}}\\
% {{bl} {27} {\%} {(3/11)}} & {{es} {18} {\%} {(2/11)}} & {{es} {18} {\%} {(2/11)}} & {{es} {18} {\%} {(2/11)}} & {{es} {18} {\%} {(2/11)}} & {{ch} {9} {\%} {(1/11)}} & {{bl} {9} {\%} {(1/11)}} &  & \\
% \lspbottomrule
% \end{tabularx}

\pgfplotstableread{
 %es %bl %ch
0 18 27 54
1 18 45 36
2 18 54 27
3 18 36 27
4 18 27 45
5 36 36 9
6 18 9 27
7 54 27  0
8 54 27  0
}\dataset
\fittable{
\begin{tikzpicture}
\begin{axis}[ybar,
        width=16cm,
        height=8cm,
        ymin=0,
        ymax=100,
        xlabel={Ámbito},
        xlabel style = {yshift=-3mm},
        ylabel={Uso \%},
        xtick=data,
        xticklabels = {
            casa\strut,
            fiestas\strut,
            juntas\strut,
            trabajo\strut,
            mercado\strut,
            iglesia\strut,
            escuela\strut,
            clínica\strut,
            ciudad\strut
        },
        nodes near coords,
        xticklabel style={rotate=0},
        major x tick style = {opacity=0},
        minor x tick num = 1,
        minor tick length=1ex,
        every node near coord/.append style={
                anchor=center,
                yshift=2mm
%                 rotate=90
        },
        legend columns=3,
        legend style={at={(0.005,.85)},anchor=south west},
        axis x line*=bottom,
        axis y line*=left,
        ]
\addplot[draw=black,fill=lsMidBlue] table[x index=0,y index=1] \dataset; %es
\addplot[draw=black,fill=black!30] table[x index=0,y index=2] \dataset; %bl
\addplot[draw=black,fill=lsMidOrange] table[x index=0,y index=3] \dataset; %ot
\legend{español, bilingüe, chichimeca}
\end{axis}
\end{tikzpicture}
}

\end{figure}

En la \figref{fig:guerrero:4}, se muestra como el chichimeca es usado preferentemente en tres niveles distintos: a nivel particular en la casa, a nivel comunitario en las fiestas y en la escuela, y en el mercado, que representan un nivel de interacción entre la comunidad y el municipio. En cambio, en los ámbitos con una mayor injerencia de la municipalidad tales como la clínica y la propia ciudad se prefiere el uso del español.



\begin{figure}
% \includegraphics[width=\textwidth]{figures/guerrero-img004.jpg}
\includegraphics[width=\textwidth]{figures/guerrero-img004.pdf}
\caption{\label{fig:guerrero:4} Esquema del \textit{habitus} lingüístico en Misión de Chichimecas (elaboración de los autores)}
\end{figure}



 \subsection{Red tepehuana de Santa María de Ocotán}

\largerpage[-1]
El \textit{o’dam} o tepehuano del sureste es una lengua de la familia yutoazteca, de la rama pimana (junto con el tepehuano de norte, el pima y el pápago), pertenece culturalmente a la región conocida como Gran Nayar (\citealt{ReyesValdez2006a}), donde ha mantenido un contacto histórico con coras, huicholes y mexicaneros. De manera contraria, con respecto al contacto con el español, este se mantuvo con un bajo dominio de ámbitos.

Si bien se considera la incursión de Nuño de Beltrán en el año de 1531 como una primera presencia española en la zona, fue solo hasta ocho años después, al fundarse San Francisco de Mezquital, y sobretodo en el año de 1600, que inició el contacto con la lengua española. En este momento, los frailes franciscanos entraron a la sierra a través de las visitas parroquiales, instaurando un primer contacto con el español. El contacto, no obstante, no era permanente, puesto que no se establecía un religioso de forma regular. Estas primeras incursiones se vieron interrumpidas por la conocida “rebelión tepehuana” a inicios del siglo XVII (\citealt{ReyesValdez2006a}). No obstante, al menos para Santa María de Ocotán, se tiene el registro de un capítulo provincial de 1806 en el que se especifica la presencia permanente de un religioso en la comunidad (\citealt{DelaTorreCuriel2006}: 160). Este hecho implicó un contacto más permanente con hispanohablantes, el cual no era necesariamente fuerte.

\largerpage[-1]
Una de las características culturales más importantes de este grupo es el ciclo ceremonial en el que se encuentran tanto las fiestas de origen católico como las de tradición prehispánica. Un ejemplo sería la \textit{xiotalh} o \textit{mitote},\footnote{El \textit{mitote} o \textit{xiotalh} está relacionado con los cambios estacionales y “son ceremonias que duran cinco días (aunque hay mitotes familiares de tres días); su principal característica es un baile nocturno que se realiza el último día de la ceremonia alrededor de una fogata y un músico que, colocado en el centro de la plaza, percute la cuerda de un arco sobre un tecomate que le sirve de caja de resonancia” (\citealt{ReyesValdez2006b}: 17).} la cual se realiza de manera comunal o familiar (\citealt{ReyesValdez2006a,ReyesValdez2006b}).

Actualmente, los tepehuanos suelen migrar para trabajar en los cultivos de tomate, frijol y manzana a los estados de Sinaloa, Nayarit y Chihuahua, o bien migran a la ciudad de Durango de manera temporal o definitiva, pero en la mayoría de los casos regresan al menos una vez al año a la comunidad a la que pertenecen en la sierra.

Como en los otros dos casos, no es posible determinar el número exacto de hablantes de lengua indígena en Santa María de Ocotán. El censo de 2010 registra 474 habitantes de 3 años o más, y 542 personas de 5 años o más en hogares indígenas. Es la comunidad que reporta un mayor porcentaje de monolingüismo (no habla español) del 27.84\% y un menor bilingüismo (lengua indígena-español) del 72.15\%. A nivel municipal se cuentan 23724 hablantes de lengua indígena; es decir, el 79.10\% de los habitantes del municipio. El estado de Durango alcanza los 32917 hablantes de lengua indígena, pero también hay presencia de otros grupos como wixarika y rarámuri. Santa María es una comunidad pequeña con 91 viviendas censadas, pero el 100 \% de la población de 5 años y más vive en un hogar indígena (véase \tabref{tab:guerrero:5}).

% \begin{table}[t]
% \caption{\label{tab:guerrero:7} Datos demográficos de El Mezquital y Santa María Ocotán\\Fuente: \citet{INEGI2010} }
% \begin{tabularx}{\textwidth}{l@{}rrrQ}
% \lsptoprule
% {{Sta.} {María} {de} {Ocotán}} & {{mujeres}} & {{hombres}} & {{total}} & {{alcances} {y} {porcentajes}}\\
% \midrule
% {población} & {258} & {256} & {514} & {(RM/H 1.008)}\\
% \tablespace
% {hablante de lengua indígena} & {231} & {243} & {474} & {92.21 \% (80.54 \%)}\\
% \tablespace
% {no habla español} & {89} & {43} & {132} & {25.68 \%}\\
% \tablevspace
% {habla español y lengua indígena} & {142} & {200} & {342} & {96.82 \%}\\
% \tablevspace
% {en hogares indígenas} & --- & --- & {542}  & {100 \% (población de 5 años y más)}\\
% \tablevspace
% {grado de escolaridad} & {4.22} & {6.73} & {5.37}  & {Adultos}\\
% \tablevspace
% {analfabetismo} & {(24.42\%)} & {(6.25\%)} & --- & {15.37\%}\\
% \tablevspace
% {viviendas} & --- & --- & {91} & {Comunidad}\\
% \tablevspace
% {Población del Mezquital} & {29989} & {14793} & {29989} & {Municipio}\\
% \tablevspace
% {hablante de lengua indígena} & {12062} & {11680} & {23724} & {Municipio}\\
% \tablevspace
% {hablante de lengua indígena} & {16393} & {16524} & {32917} & {Estado}\\
% \lspbottomrule
% \end{tabularx}
% \end{table}

\begin{table}
\caption{\label{tab:guerrero:5} Datos demográficos de El Mezquital y Santa María de Ocotán \citep{INEGI2010}}
\begin{tabularx}{\textwidth}{Xrrrr}
\lsptoprule
Localidad & \multicolumn{2}{p{2.5cm}}{El Mezquital\newline  (municipio)} & \multicolumn{2}{p{2.5cm}}{Santa María\newline de Ocotán\newline (comunidad)}\\
\midrule
Población total &  29\,989 && 542 &\\
Hablante de lengua indígena & 23\,742 &79.10 \%& 474 & 87.45 \%\\
Habla español y lengua indígena & 9\,157 &38.56 \%& 342 & 72.14 \% \\
No habla español & 5\,841 & 24.50 \% &  132 & 27.84 \% \\
En hogares indígenas (5 años y más)& 27\,229 & 90.79 \%&  542 & 100 \% \\
Viviendas habitadas & 6\,351  &&  91 & \\
\lspbottomrule
\end{tabularx}
\end{table}


Los datos de esta comunidad se obtuvieron de un total de 18 colaboradores pertenecientes a distintas redes familiares de la comunidad, 7 hombres y 12 mujeres, lo que implica cerca del 3\% de la población masculina y del 5\% de la femenina.


 \subsubsection{Ámbitos de uso en Santa María de Ocotán}


En Santa María de Ocotán encontramos tres espacios en los que el tepehuano tiene preferencia de uso: la casa, la fiesta y la iglesia. Sin embargo, como se puede observar en la \figref{figtab:guerrero:8}, destaca el hecho de que en ninguno de los nueve ámbitos predomina el español, pues en los seis restantes es el uso de las dos lenguas el que obtiene un mayor porcentaje incluso en espacios como la ciudad, la cual es un contexto más hispanohablante.

\begin{figure}
\caption{\label{figtab:guerrero:8} Ámbitos de uso de Santa María de Ocotán}
% \begin{tabularx}{\textwidth}{XXXXXXXXX}
% \lsptoprule
% {{casa}} & {{fiestas}} & {{juntas}} & {{trabajo}} & {{mercado}} & {{iglesia}} & {{escuela}} & {{clínica}} & {{ciudad}}\\
% \midrule
% {{te} {66.6\%} {(12/18)}} & {{te} {94.4} {\%} {(17/18)}} & {{bl} {58.8} {\%} {(10/17)}} & {{bl} {71.4} {\%} {(10/14)}} & {{bl} {55} {\%} {(10/18)}} & {{te} {68.7} {\%} {(11/16)}} & {{bl} {92.3} {\%} {(12/13)}} & {{bl} {55} {\%} {(10/18)}} & {{bl} {66} {\%} {(12/18)}}\\
% {{bl} {33.3} {\%} {(6/18)}} & {{bl} {5.8} {\%} {(1/18)}} & {{te} {41.1} {\%} {(7/17)}} & {{es} {21.4} {\%} {(3/14)}} & {{es} {44} {\%} {(8/18)}}   & {{bl} 25 \% (4/16)}        & {{te} {7.6\%} {(1/13)}}         & {{te} {28} {\%} {(5/18)}} & {{es} {33} {\%} {(6/18)}}\\
%                             &                            &                              & {{te} {7.1} {\%} {(1/14)}} &                            & {{es} {6.2} {\%} {(1/16)}} &                                 & {{es} 16 \% (3/18)} & \\
% \lspbottomrule
% \end{tabularx}


\pgfplotstableread{
 %es %bl %ch
0 0 33.3 66.6
1 0 5.8 94.4
2 0 58.8 41.1
3 21.4 71.4 7.1
4 44 55 0
5 6.2 25 68.7
6 0 92.3 7.6
7 16 55 28
8 33 66 0
}\dataset
\fittable{
\begin{tikzpicture}
\begin{axis}[ybar,
        width=16cm,
        height=8cm,
        ymin=0,
        ymax=100,
        xlabel={Ámbito},
        xlabel style = {yshift=-3mm},
        ylabel={Uso \%},
        xtick=data,
        xticklabels = {
            casa\strut,
            fiestas\strut,
            juntas\strut,
            trabajo\strut,
            mercado\strut,
            iglesia\strut,
            escuela\strut,
            clínica\strut,
            ciudad\strut
        },
        nodes near coords,
        xticklabel style={rotate=0},
        major x tick style = {opacity=0},
        minor x tick num = 1,
        minor tick length=1ex,
        every node near coord/.append style={
                anchor=center,
                yshift=2mm
%                 rotate=90
        },
        legend columns=3,
        legend style={at={(0.25,.85)},anchor=south west},
        axis x line*=bottom,
        axis y line*=left,
        ]
\addplot[draw=black,fill=lsMidBlue] table[x index=0,y index=1] \dataset; %es
\addplot[draw=black,fill=black!30] table[x index=0,y index=2] \dataset; %bl
\addplot[draw=black,fill=lsMidOrange] table[x index=0,y index=3] \dataset; %ot
\legend{español, bilingüe, tepehuano}
\end{axis}
\end{tikzpicture}
}
\end{figure}

Asimismo, en la \figref{fig:guerrero:5} observamos como el tepehuano es usado de manera preferencial en ámbitos privados como la casa y en espacios comunitarios como la iglesia y las fiestas, mientras que en todos los demás ámbitos el bilingüismo es la norma, incluso en el espacio estatal, hecho que no se presenta en las dos comunidades anteriores.




\begin{figure}
\caption{\label{fig:guerrero:5} Esquema del \emph{habitus} lingüístico de Santa María de Ocotán (elaboración de los autores)}
% \includegraphics[width=\textwidth]{figures/guerrero-img005.jpg}
\includegraphics[width=\textwidth]{figures/guerrero-img005.pdf}
\end{figure}


 \section{Resultados comparativos}


Como se mencionó al inicio de este trabajo, uno de los principales objetivos de usar tanto, un mismo instrumento de elicitación como la noción de \textit{habitus lingüístico} para describir el uso diferenciado de las lenguas habladas era con fines comparativos. Así, el analizar distintas comunidades que tiene sus propias dinámicas de uso, según sea su realidad sociolingüística, nos permite observar ciertos patrones que se cumplen o bien que se rompen. En la sección que sigue compararemos los ámbitos en los que se refleja la presencia de la lengua indígena, de ambas lenguas y del español, destacando que la forma en que estos ámbitos se van jerarquizando muestra dónde se usa qué lengua y cómo se prefiere el uso de una sobre otra.



 \subsection{Ámbitos de uso la lengua indígena}



El \tabref{tab:guerrero:9} muestra que el \textit{hñähñu} nunca obtiene el porcentaje mayor de hablantes en los espacios donde hay uso de la lengua indígena. No obstante, es importante resaltar que en los espacios de la fiesta, la casa y el mercado, el otomí obtiene sus porcentajes más altos con un 8\%. En parte, esto puede relacionarse con el hecho de que, en El Espíritu, la fiesta (en particular el Carnaval) es un espacio privilegiado para la reproducción cultural que se ha mantenido como pilar identitario para la comunidad.

\begin{table}
\caption{\label{tab:guerrero:9} Ámbitos de uso El Espíritu del otomí (OT)}


\begin{tabularx}{\textwidth}{rYYY}
\lsptoprule
{fiestas} & {{casa}} & {{trabajo}} & {{mercado}}\\
\midrule
{{ot} {8} {\%} {(2/24)}} & {{ot} {8} {\%} {(2/24)}} & {{ot} {4.1} {\%} {(1/24)}} & {{ot} {8} {\%} {(1/24)}}\\
{{es} {46} {\%}  {(11/24)}} & {{es} {67} {\%} {(16/24)}} & {{es} {62} {\%} {(15/24)}} & {{es} {67} {\%} {(16/24)}}\\
{{bl} {46} {\%} {(11/24)}} & {{bl} {25} {\%} {(6/24)}} & {{bl} {21\%} {(5/24)}} & {{bl} {29} {\%} {(7/24)}}\\
\lspbottomrule
\end{tabularx}
\end{table}
Por su parte, en Misión de Chichimecas, el \tabref{tab:guerrero:10} nos muestra que, a diferencia de El Espíritu, en Mision de Chichimecas, el chichimeca es la lengua que obtiene mayor porcentaje tanto en la casa (54\%) como en el mercado (45\%). Sin embargo, a pesar de no obtener los porcentajes más altos, la fiesta tradicional (36\%) también es un espacio de uso de chichimeca, con un papel muy importante para en la reproducción cultural e identitaria del grupo.


\begin{table}
\caption{\label{tab:guerrero:10} Ámbitos de Misión de uso Chichimecas en que se prefiere el chichimeca (CH)}
\begin{tabularx}{\textwidth}{rYYY}
\lsptoprule
{{casa}} & {{mercado}} & {{ciudad}} & {{fiestas}}\\
\midrule
{{ch} {54} {\%} {(6/11)}} & {{ch} {45} {\%} {(5/11)}} & {{ch} {27} {\%} {(3/11)}} & {{ch} {36} {\%} {(4/11)}}\\
{{es} {18} {\%} {(2/11)}} & {{es} {18} {\%} {(2/11)}} & {{es} {45} {\%} {(5/11)}} & {{bl} {45} {\%} {(5/11)}}\\
{{bl} {27} {\%} {(3/11)}} & {{bl} {27} {\%} {(3/11)}} &  & \\
\lspbottomrule
\end{tabularx}
\end{table}

Finalmente, a diferencia de las otras dos comunidades arriba mencionadas, en Santa María Ocotán, el uso de \textit{o'dam} obtiene altos porcentajes de uso en un mayor número de espacios como las fiestas con 94.4\% (17/18), seguido por la casa con 66\% (12/18) y la iglesia con 68.7\% (11/16) (véase \tabref{tab:guerrero:11}). Destaca nuevamente que tanto las fiestas como la iglesia son dos ámbitos en los que, de manera muy similar a los casos anteriores, tienen una relación con la tradición y cosmovisión tepehuana.

\begin{table}
\caption{\label{tab:guerrero:11} Ámbitos de uso Santa María de Ocotán en que se prefiere el tepehuano (TE)}
\begin{tabularx}{\textwidth}{rYYY}
\lsptoprule
{{fiestas}} & {{casa}} & {{iglesia}} & {{juntas}}\\
\midrule
{{te} {94.4} {\%} {(12/18)}} & {{te} {66.6} {\%} {(12/18)}} & {{te} {68.7} {\%} {(11/16)}} & {{te} {41.4} {\%} {(7/17)}}\\
{{bl} {5.8\%} {(6/18)}} & {{bl} {33.3} {\%} {(6/18)}} & {{bl} 25 \% (3/16)} & {{bl} {58.8} {\%} {(10/17)}}\\
&  & {{es} {6.2} {\%} {(1/16)}} & \\
\lspbottomrule
\end{tabularx}
\end{table}



 \subsection{Ámbitos del uso bilingüe}



A diferencia del uso monolingüe del \textit{hñähñu}, el \tabref{tab:guerrero:12} muestra que en El Espíritu el uso bilingüe tiene mayores porcentajes en las fiestas (46\%), la iglesia (58\%) y las juntas (71\%). Nuevamente se observa que son espacios que representan dinámicas comunitarias. No obstante, el hecho de que la lengua indígena no aparezca de manera solitaria en ningún espacio refleja el estado de obsolescencia en el que se encuentra. Sin embargo, los usos bilingües pueden tomarse como una buena señal para futuros trabajos de revitalización y mantenimiento del \textit{hñähñu.}

\begin{table}
\caption{\label{tab:guerrero:12} Ámbitos de uso El Espíritu en que se prefiere el bilingüismo (BL)}
\begin{tabularx}{\textwidth}{rYYY}
\lsptoprule
{fiestas} & {{iglesia}} & {{mercado}} & {{juntas}}\\
\midrule
{{bl} {46} {\%} {(11/24)}} & {{bl} {58} {\%} {(14/24)}} & {{bl} {29} {\%} {(7/24)}} & {{bl} {71} {\%} {(17/24)}}\\
{{es} {46} {\%} {(11/24)}} & {{es} {42} {\%} {(10/24)}} & {{es} {67} {\%} {(16/24)}} & {{es} {29} {\%} {(7/24)}}\\
{{ot} {8} {\%} {(2/24)}} &  & {{ot} {8} {\%} {(1/24)}} & \\
\lspbottomrule
\end{tabularx}
\end{table}

El \tabref{tab:guerrero:13} recoge los datos sobre el bilingüismo en Misión Chichimecas. La predominancia general del bilingüismo podría indicar un aumento en el uso del español en dominios que fueron típicamente chichimecas, como las juntas, donde se registra un 54\%, y las fiestas donde se observa un 45\%.

No es extraño el uso del español (18\%) y el uso bilingüe (54\%) en las juntas o asambleas de Misión, pues son el máximo órgano regulador de la comunidad. En ellas se eligen a los delegados municipales por sufragio universal y se discuten los asuntos trascendentales y de interés público, pero también es donde se negocia discursivamente con el mundo hispanohablante y la sociedad nacional.
  
\begin{table}
\caption{\label{tab:guerrero:13} Ámbitos de uso Misión de Chichimecas en que se prefiere el bilingüismo (BL)}
\begin{tabularx}{\textwidth}{rYYY}
\lsptoprule
{{juntas}} & {{fiestas}} & {{trabajo}} & {{iglesia}}\\
\midrule
{{bl} {54} {\%} {(6/11)}} & {{bl} {45} {\%} {(5/11)}} & {{bl} {36} {\%} {(4/11)}} & {{bl} {36} {\%} {(4/11)}}\\
{{ch} {27} {\%} {(3/11)}} & {{ch} {36} {\%} {(4/11)}} & {{ch} {27} {\%} {(3/11)}} & {{es} {36} {\%} {(4/11)}}\\
{{es} {18} {\%} {(2/11)}} &  & {{es} {18} {\%} {(2/11)}} & {{ch} {9} {\%} {(1/11)}}\\
\lspbottomrule
\end{tabularx}
\end{table}

Por último, en Santa María de Ocotán, todos los espacios que no han sido mencionados en la sección anterior, requieren un uso bilingüe. El \tabref{tab:guerrero:14}, advierte que muchos de ellos se pueden asociar a campos que están más alejados de la vida comunitaria, empezando por el desplazamiento a la ciudad de Durango con un 66\%, la clínica que se encuentra en la comunidad con un 55\% y el mercado que solo se encuentra en la ciudad (55\%). Es importante mencionar que el uso bilingüe en la clínica se debe a que, si bien el doctor solo habla español, la enfermera que lo asiste es hablante de \textit{o'dam} y la mayoría de los colaboradores afirmaron que prefieren ir cuando se encuentra ella.

Si bien las juntas constituyen un ámbito muy relacionado con la vida de la comunidad, en ellas sucede algo similar que en el caso de MIsión de Chichimecas. Es decir, en ellas se discuten tanto los temas relacionados con la comunidad como los temas externos, razón por la cual hay una presencia de las dos lenguas (\textit{o'dam} y español). Es importante destacar que fue este fue el único espacio en el que los colaboradores aseguraban mezclar las lenguas.

Asimismo, es importante destacar los porcentajes encontrados en el ámbito de la escuela, pues nuevamente es el uso bilingüe el que obtiene un mayor porcentaje (92.3\%). Este alto porcentaje muestra que, al igual que en otras situaciones de contacto, la escuela es uno de los primeros espacios en los que los colaboradores tienen un contacto con el español, y donde el uso de esta lengua se va haciendo cada vez más habitual. Sin embargo, incluso en este ámbito, el uso del español no aparece sin la presencia del tepehuano, pero sí el tepehuano sin el español, lo cual muestra el amplio uso de la lengua indígena.

\begin{table}
\caption{\label{tab:guerrero:14} Ámbitos de Santa María de Ocotán en que se prefiere el bilingüismo (BL)}
\fittable{
\begin{tabular}{rrrrr}
\lsptoprule
{{escuela}} & {{ciudad}} & {{juntas}} & {{clínica}} & {{mercado}}\\
\midrule
{{bl} {92.3} {(12/13)}} & {{bl} {66} {\%} {(12/18)}} & {{bl} {55} {\%} {(10/18)}} & {{bl} {55} {\%} {(10/18)}} & {{bl} {55} {\%} {(10/18)}}\\
{{te} {7.6} {(1/13)}} & {{es} {33} {\%} {(6/18)}} & {{te} {39} {\%} {(7/18)}} & {{te} {28} {\%} {(5/18)}} & {{es} {44} {\%} {(8/18)}}\\
&  & {{es} 16 \% (3/18)} & {{es} 16 \% (3/18)} & \\
\lspbottomrule
\end{tabular}
}
\end{table}



 \subsection{Ámbitos de uso del español}



En El Espíritu, el español tiene una fuerte presencia y, aunado a lo expuesto en los apartados anteriores, nos da pie a sugerir que los miembros de esta comunidad se encuentran en un proceso de desplazamiento del otomí en favor del español. El \tabref{tab:guerrero:15} observamos como, tanto en la clínica (con un 87\%), como en la ciudad y la escuela (ambos con un 83\%), predomina el español. Estos tres ámbitos comparten el hecho de ser espacios con una alta presencia de hispanohablantes tanto en la atención como en la solicitud del servicio, además de ser los más distantes a la comunidad.

\begin{table}
\caption{\label{tab:guerrero:15} Ámbitos de uso El Espíritu en que se prefiere el español (ES)}
\begin{tabularx}{\textwidth}{rYYY}
\lsptoprule
{clínica} & {{ciudad}} & {{escuela}} & {{juntas}}\\
\midrule
{{es} {87} {\%} {(21/24)}} & {{es} {83} {\%} {(20/24)}} & {{es} {83} {\%} {(20/24)}} & {{es} {71} {\%} {(17/24)}}\\
{{bl} {12} {\%} {(3/24)}} & {{bl} {12} {\%} {(3/24)}} & {{bl} {12} {\%} {(3/24)}} & {{bl} {29} {\%} {(7/24)}}\\
\lspbottomrule
\end{tabularx}
\end{table}

\begin{table}[b]
\caption{\label{tab:guerrero:16} Ámbitos de uso Misión de Chichimecas en que se prefiere el español (ES)}
\begin{tabularx}{\textwidth}{rYYY}
\lsptoprule
{{clínica}} & {{iglesia}} & {{ciudad}} & {{juntas}}\\
\midrule
{{es} {54} {\%} {(6/11)}} & {{es} {36} {\%} {(4/11)}} & {{es} {54} {\%} {(6/11)}} & {{es} {18} {\%} {(2/11)}}\\
{{bl} {27} {\%} {(3/11)}} & {{bl} {36} {\%} {(4/11)}} & {{bl} {27} {\%} {(3/11)}} & {{ch} {27} {\%} {(3/11)}}\\
& {{ch} {9} {\%} {(1/11)}} &  & {{bl} {54} {\%} {(6/11)}}\\
\lspbottomrule
\end{tabularx}
\end{table}

\largerpage
En Misión de Chichimecas, el \tabref{tab:guerrero:16} también exhibe el papel del español en la comunidad, siendo la clínica y la ciudad las que se destacan con un mayor porcentaje, ambas con un 54\%. Mientras, en la iglesia el español y el uso bilingüe tienen un porcentaje de 36\%. A diferencia de la comunidad de El Espíritu, entre los chichimecas se puede observar que la lengua \textit{úza̱'} -- ya sea de manera solitaria o en conjunto con el español -- sigue teniendo un valor de uso.



A diferencia de las otras dos comunidades, Santa María Ocotán se destaca porque en ninguno de los ámbitos de uso el español tiene el porcentaje más alto, en relación con el uso de la lengua indígena o bilingüe. No obstante, el \tabref{tab:guerrero:17} indica que los espacios en los que se usa el español con porcentajes menores son aquellos en los que se da una interacción con el mundo hispanohablante y que se encuentran fuera de la comunidad. Por ejemplo, en el mercado (44\%) y en la ciudad (33\%).


\begin{table}
\caption{\label{tab:guerrero:17} Ámbitos de uso de Santa María de Ocotán con presencia del español (ES)}
\begin{tabularx}{\textwidth}{rYYY}
\lsptoprule
{{mercado}} & {{ciudad}} & {{clínica}} & {{trabajo}}\\
\midrule
{{es} {44} {\%} {(8/18)}} & {{es} {33} {\%} {(6/18)}} & {{es} 16 \% (3/18)} & {{es} {11} {\%} {(2/18)}}\\
{{bl} {55} {\%} {(10/18)}} & {{bl} {66} {\%} {(12/18)}} & {{bl} {55} {\%} {(10/18)}} & {{te} {5} {\%} {(1/18)}}\\
&  & {{te} {28} {\%} {(5/18)}} & {{bl} {50} {\%} {(9/18)}}\\
\lspbottomrule
\end{tabularx}
\end{table}

En resumen, podemos observar que existe una relación del uso de la lengua indígena con ámbitos que están relacionados con la vida comunitaria tales como las fiestas y el hogar; la presencia del español y de la lengua indígena en la comunidad de El Espíritu implica que no hay un desplazamiento total de la lengua, mientras que en Misión de Chichimecas esta co-aparición de las lenguas puede sugerir que el español esté empezando a abarcar espacios comunitarios en los que se esperaría el chichimeca. Por su parte, en Santa María de Ocotán, el empleo de las dos lenguas parece ser más equilibrado y esto podría estar sujeto al interlocutor. Finalmente, son los ámbitos que están fuera de la comunidad o bien en los que se tiene una interacción con hispanohablantes los que tienen un mayor uso del español. Destaca el hecho que, mientras que en El Espíritu el español tiene los porcentajes de uso más altos, en Santa María de Ocotán presenta los más bajos, mostrándonos la estrecha relación que tienen la realidad sociolingüística de la comunidad con los usos de las lenguas.
\largerpage


 \section{Reflexiones finales}
El presente estudio, plantea el uso de la noción de mercado y \textit{habitus lingüístico} con el fin de describir los ámbitos de uso de las lenguas en una comunidad multilingüe. Estas nociones permiten esquematizar las distintas decisiones que toman los hablantes de una comunidad en relación con el uso de las lenguas en cuestión, y comparar para encontrar patrones de usos en sus diferencias y semejanzas. Asimismo, nos muestra la estrecha relación que existe entre los usos y las distintas realidades sociolingüísticas. Esto se mostró en el análisis de tres comunidades mexicanas que forman un \textit{continuum} que va de la obsolescencia (El Espíritu), al inicio de desplazamiento (Misión de Chichimecas) y a la vitalidad de la lengua indígena (Santa María de Ocotán). Por lo tanto, observamos tres mercados lingüísticos diferentes en los que el español y la lengua indígena tienen valores de cambio distintos.

Así, la comunidad de El Espíritu refleja un menor uso de la lengua indígena pero no su total desplazamiento, pues pese a que el español está presente en todos los ámbitos, el uso bilingüe en algunos de ellos implica la permanencia del otomí y su legitimidad en ciertos actos de habla. Por su parte, la comunidad de Misión de Chichimecas representaría un estado intermedio frente a las otras dos comunidades, pues en ella el chichimeca sigue teniendo una fuerte presencia a pesar que el español empieza a ser usado en contextos en los que se esperaría la lengua indígena. Finalmente, la comunidad de Santa María de Ocotán sería el extremo contrario de El Espíritu, pues los miembros de esta comunidad prefieren el uso de la lengua indígena, ya sea en solitario o en co-aparición con el español. La presencia del español es muy baja y se especializa en contextos en los que hay presencia de hispanohablantes.

  Asimismo, se exhibió que la fiesta, la casa y la iglesia (para el caso del tepehuano del sur) son los ámbitos en los que los hablantes usan más la lengua indígena, según lo mencionan los mismos hablantes. Estos son espacios en los que se destacan los vínculos de interpretación de la cosmovisión y la costumbre de los grupos indígenas estudiados. Mientras que las juntas o asambleas parecen ser espacios en el que el uso de las dos lenguas es común en las tres comunidades. Es en estos ámbitos en los que se tocan temas relacionados tanto con festividades y gobierno tradicional como con proyectos gubernamentales en los que es necesario el uso del español, o bien el uso de las dos lenguas mezcladas por la ausencia de términos necesarios.\footnote{\citet[922]{Zimmermann2010} expone esta situación para el otomí de los años ochenta en el que “situaciones sobre todo públicas se puede observar la alternancia de códigos (code switching) y la transferencia del español al otomí”.} 

\largerpage[-2]
Con respecto al español, los hablantes prefieren esta lengua en los ámbitos que están más alejados de la vida comunitaria o que incluso están fuera de ella -- como es el caso de la ciudad o el mercado. Asimismo, también destacan los espacios en los que hay interacción con hispanohablantes.\footnote{\citet[928]{Zimmermann2010} expone el caso de la comunidad zapoteca de Rincón Juárez, caracterizada por tener un bilingüismo colectivo  con un 80\% de población bilingüe a finales del siglo XX. También, menciona que el uso del español está especializado para contextos formales e institucionales, tales como la “escuela, la iglesia, el centro de salud, ocasionalmente el mercado y las asambleas municipales [...] Estos dominios son a la vez los espacios intracomunitarios de encuentro con personas provenientes de fuera de la comunidad. Mientras que, en la escuela, el factor determinante es el sistema y la orientación educativa, en el centro de salud, la iglesia y el mercado es la falta de conocimiento del zapoteco por parte de los interlocutores”.}

Queda pendiente en el análisis etnográfico de estos ámbitos en estas tres comunidades, específicamente en los usos bilingües, para observar la selección de recursos lingüísticos de las lenguas en contacto que hacen los colaboradores en relación con los interlocutores.

  Finalmente, es importante destacar la implementación de instrumentos que nos permitan hacer estudios comparativos en las diferentes comunidades bilingües del país, con el fin de poder tener un mayor conocimiento de su realidad sociolingüística y de las decisiones que toman los hablantes con base en esta última.

\sloppy\printbibliography[heading=subbibliography,notkeyword=this]
\end{document}
