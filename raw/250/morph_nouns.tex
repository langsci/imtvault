\chapter{Nouns}
\label{cpt:nouns}

The grammatical categories of \isi{nouns} and other nominals in Sanzhi are \isi{gender}, \isi{number} and case. There are three genders: masculine, feminine, and neuter. With respect to \isi{number}, \isi{nouns} distinguish singular and plural. In addition, there is an associative plural. Sanzhi Dargwa has four \isi{grammatical cases}, namely \isi{absolutive}, \isi{ergative}, \isi{dative}, and \isi{genitive}, and many more semantic cases. Most of the latter are \is{spatial case}spatial cases.

This chapter describes \isi{gender} (\refsec{sec:noungender}), \isi{number} (\refsec{sec:nounnumber}), and case (\refsec{sec:nouncase}) as well as the \isi{derivation} of \isi{nouns} (\refsec{sec:nounderivation}) and the formation of \isi{nouns} by means of \isi{compounding} and \isi{reduplication} (\refsec{sec:nounwordformation}).


%%%%%%%%%%%%%%%%%%%%%%%%%%%%%%%%%%%%%%%%%%%%%%%%%%%%%%%%%%%%%%%%%%%%%%%%%%%%%%%%
%%%%%%%%%%%%%%%%%%%%%%%%%%%%%%%%%%%%%%%%%%%%%%%%%%%%%%%%%%%%%%%%%%%%%%%%%%%%%%%%

\section{Gender}
\label{sec:noungender}

Sanzhi has the typical Dargwa \isi{gender} system of three genders that have a transparent semantic basis: masculine, feminine, and neuter. To the feminine and masculine \isi{gender} belong only those \isi{nouns} that denote humans or are perceived as humanoids or similar to humans. This means that \isi{gender} for humans follows natural \isi{gender}, and all other \isi{nouns} are neuter. Gender agreement is a major grammatical trait of East Caucasian languages, including Sanzhi. The combined \isi{gender}\tnd \isi{number} agreement affixes are given in \reftab{tab:Agreement affixes in SanzhiMorph}. All forms except the zero marking for masculine singular agreement can occur as prefixes, suffixes, and infixes (only with two words). For more information on \isi{gender} agreement see \refsec{sec:Gender/number agreement}.

\begin{table}
	\caption{Agreement affixes in Sanzhi}
	\label{tab:Agreement affixes in SanzhiMorph}
	\small
	\begin{tabularx}{0.5\textwidth}[]{%
		>{\raggedright\arraybackslash}p{46pt}
		>{\centering\arraybackslash}X
		>{\centering\arraybackslash}X
		>{\centering\arraybackslash}X}
		
		\lsptoprule
		{}		&	\tsc{sg}	 	&	\tsc{1/2pl}		&	\tsc{3pl}	\\
		\midrule 
		Masculine	&	\tit{w}\slash ∅		&	\tit{d}			&	\tit{b}\\
		Feminine	&	\tit{r}			&	\tit{d}			&	\tit{b}\\
		Neuter	&	\tit{b}			&	\multicolumn{2}{c}{\tit{d}}\\
		\lspbottomrule
	\end{tabularx}
\end{table}

Gender is normally not marked on \isi{nouns}, but there are a few \isi{nouns} that do carry overt \isi{gender} markers in word-initial position that seem to go back to \isi{gender} prefixes. They can be divided into two groups. The first group is kinship terms and the noun \sqt{owner} that differ in their form depending on the \isi{gender} of the referent \refex{ex:nounsovertgendermarkers}. 

\begin{exe}
	\ex	\label{ex:nounsovertgendermarkers}
	\begin{xlist}
		\ex	\textit{uc{\lmk}i} \sqt{brother}, \textit{ruc{\lmk}i} \sqt{sister}, \textit{buc{\lmk}i} \sqt{pair (e.g. of shoes)}
		\ex	\textit{uc{\lmk}iq'ar}, \textit{ruc{\lmk}iq'ar} \sqt{cousin} (male, female)
		\ex	\textit{rurs{\lmk}i} \sqt{girl, daughter} (no other forms available)
		\ex	\textit{wah}, \textit{rah}, \textit{bah} \sqt{owner}
		\ex	\textit{bahinte} \sqt{parents} (plural form of \sqt{owner})
	\end{xlist}
	\end{exe}
	
The second group is \isi{nouns} that express the \isi{gender} of the (implicit or explicit) possessor \refex{ex:nounsovertgendermarkers2}. Most of the words of the second group denote body parts. The first noun controls neuter plural agreement since it is morphosyntactically a compound noun (\refsec{ssec:Nounnoun compounds}). The second noun controls neuter singular agreement, independently of the \isi{agreement prefix} used. The third noun controls agreement according to the referent and therefore in accordance with the prefix it has. For instance, \textit{babq'i (χalq')} `half (of the people)' controls human plural agreement. It thus behaves similar to the \isi{nouns} in \refex{ex:nounsovertgendermarkers}.
	
\begin{exe}	
		\ex	\label{ex:nounsovertgendermarkers2}
	\begin{xlist}
		\ex	\textit{wark-maχ} \sqt{inside, inner parts, entrails} (of human beings, male and female), \textit{dark-maχ} \sqt{inside, inner parts, entrails} (of animals, in general) 
		\ex	\textit{wag}, \textit{rag},\textit{bag}, \textit{dag} \sqt{middle, waist}
		\ex	\textit{wabq'i},\textit{ rabq'i}, \textit{babq'i},\textit{ dabq'i} \sqt{half} 
	\end{xlist}
\end{exe}

There is another word \textit{daˁʡ} \sqt{face} that clearly contains a frozen \isi{gender} prefix and also occurs as part of \is{compound verb}compound verbs, e.g., in \textit{b-aˁʡ-či-aʁ-ib} \sqt{direct}, or as the root of \is{spatial adverb}spatial adverbs such as \textit{b-aˁʡ-gubal} \sqt{upside-down}. In principle, it is also possible to form the masculine singular, feminine singular, and neuter singular variants \textit{waˁʡ}, \textit{raˁʡ} and \textit{baˁʡ} that seem to have the meaning \sqt{face, muzzle} (of a man, of a woman) and \sqt{muzzle of an animal, wall, facade}, but they are not used in natural speech and speakers have trouble to find a context in which they could occur.\footnote{There is another noun \textit{baˁʡ} \sqt{leaf, sheet of paper, page} that is arguable a cognate of \textit{baˁʡ} \sqt{muzzle, wall, facade}. The two \isi{nouns} can be distinguished through their plural marking: \textit{baˁʡ} > \textit{buˁʡre} \sqt{leaves, pages} vs. \textit{baˁʡ} > \textit{baˁʡuˁrme}, \textit{baˁʡme} \sqt{muzzles, walls, facades}.}

\citet{Nichols2007} calls the overt marking of \isi{gender} on \isi{nouns} in \refex{ex:nounsovertgendermarkers} and \refex{ex:nounsovertgendermarkers2} \dqt{head gender}. \citet{Nichols2007} argues that synchronically, the initial segments of these words cannot simply be considered to be \isi{gender} \is{agreement prefix}agreement prefixes because for most \isi{nouns} head \isi{gender} does not change in the plural, in contrast to agreement affixes on verbs and other parts of speech. For instance, the plural of \textit{ucːi} \sqt{brother} and \textit{rucːi} \sqt{sister} is \textit{ucbe} and \textit{rucbe} respectively, not \textit{bucbe}. For the \isi{nouns} of the first group in \refex{ex:nounsovertgendermarkers}, the head \isi{gender} is determined by the natural \isi{gender} of the referent and not controlled by another nominal. Nichols also writes that only few such \isi{nouns} are likely of verbal origin. As for the Sanzhi words given in \refex{ex:nounsovertgendermarkers} and \refex{ex:nounsovertgendermarkers2}, I am unable to say anything about their origin.


%%%%%%%%%%%%%%%%%%%%%%%%%%%%%%%%%%%%%%%%%%%%%%%%%%%%%%%%%%%%%%%%%%%%%%%%%%%%%%%%

\section{Number}\label{sec:nounnumber}\largerpage[-1]

Most \isi{nouns} in Sanzhi can be marked for plural by means of a suffix. The singular has no special marking. Plural suffixes can be divided into three groups according to their frequency and productivity:

\begin{exe}
	\ex	\label{ex:pluralsuffixes}
	\begin{xlist}
		\ex	frequent and productive suffixes: 	\tab	\textit{-e}, \textit{-te}, \textit{-be}, \textit{-me} 
		\ex	relatively frequent suffixes: 		\tab	\textit{-re}, \textit{-ne}, \textit{-upːe}, \textit{-urbe} 
		\ex	very rare suffixes: 				\tab	\textit{-urme}, \textit{-rme}, \textit{-ube}, \textit{-de}, \textit{-une}, \textit{-(u)bne}
	\end{xlist}
\end{exe}

The first group is the only one that can be used with recent loan words from Russian (Russian loans are indicated in the lists in \refex{exPLURALinEnouns} to \refex{exPLURALinMEnouns}). The last group is restricted to one or two lexical items. Many of the \isi{nouns} undergo morphophonological processes before the plural suffix is added. Plural suffixes containing the vowel /u/ have allophones with the pharyngealized vowel \textit{uˁ}, e.g. \textit{-uˁpːe}.


% --------------------------------------------------------------------------------------------------------------------------------------------------------------------------------------------------------------------- %

\subsection{Frequent and productive plural suffixes}
\label{sec:FrequentAndProductivePluralSuffixes}
In this and the following section, I provide examples for every suffix illustrating at the same time the morphophonological processes. The morphophonological processes that are applied when forming the plural of \isi{nouns} are summarized in \refsec{ssec:MorphophonologicalrulesPluralA}.

% - - - - - - - - - - - - - - - - - - - - - - - - - - - - - - - - - - - - - - - - - - - - - - - - - - - - - - - - - - - - - - - - - - - - - - - - - - - - - - - - - - - - - - - - - - - - - - - - - - - - - - - - - - - - - - - - - - - - - - - - - - %

\subsubsection{\textit{-e}}

\begin{exe}
	\ex \label{exPLURALinEnouns}	simple suffixation:
	\begin{xlist}
		\ex	\textit{t'ult'} \textgreater\ \textit{t'ult'e} 		\sqt{bread}
		\ex	\textit{sːurrat} \textgreater\ \textit{sːurrate}		\sqt{picture}
		\ex	\textit{unc} \textgreater\ \textit{unce}			\sqt{ox}
		\ex	\textit{qːačuʁ} \textgreater\ \textit{qːačuʁe}		\sqt{bandit}
		\ex	\textit{χurejg} \textgreater\ \textit{χurejge}		\sqt{food}
		\ex	\textit{q'ampit'} \textgreater\ \textit{q'ampit'e}	\sqt{chocolate} (Russian loan)
		\ex	\textit{student} \textgreater\ \textit{studente}	\sqt{student} (Russian loan)
		\ex	\textit{praznik'} \textgreater\ \textit{praznik'e}	\sqt{holiday} (Russian loan)
	\end{xlist}

	\ex	simple suffixation, but the final consonant (stop or fricative) occurs in its underlying geminate form (see \refsec{ssec:Gemination and degemination}):
	\begin{xlist}
		\ex	\textit{juldaš} \textgreater\ \textit{juldašːe}		\sqt{friend}
		\ex	\textit{baliq} \textgreater\ \textit{baliqːe}		\sqt{fish}
		\ex	\textit{ʡuˁrus} \textgreater\ \textit{ʡuˁrusːe}		\sqt{Russian}
		\ex	\textit{ħaˁšuk} \textgreater\ \textit{ ħaˁšukːe} 	\sqt{pot}
		\ex	\textit{miriqʷ} \textgreater\ \textit{miriqːʷe}		\sqt{worm}
		\ex	\textit{t'up} \textgreater\ \textit{t'upːe}		\sqt{finger}
		\ex	\textit{tusnaq} \textgreater\ \textit{tusnaqːe}	\sqt{prison}
	\end{xlist}
    \pagebreak
	\ex	\isi{vowel mutation} (and \isi{delabialization}):
	\begin{xlist}
		\ex	\textit{halmaʁ} \textgreater\ \textit{halmuʁe}	\sqt{fiancée}
		\ex	\textit{kːazat} \textgreater\ \textit{kːazute}		\sqt{newspaper} (Russian loan)
		\ex	\textit{qːabaʁ} \textgreater\ \textit{qːabuʁe}	\sqt{pumpkin}
		\ex	\textit{q'ʷaˁl} \textgreater\ \textit{q'uˁle} 		\sqt{cow}
	\end{xlist}

	\ex	various processes (\isi{vowel deletion}, \isi{vowel mutation}, underlying geminate occurs on surface):
	\begin{xlist}
		\ex	\textit{murgul} \textgreater\ \textit{murgle}		\sqt{man}
		\ex	\textit{k'apːur} \textgreater\ \textit{k'apre}		\sqt{leaf}
		\ex	\textit{amχa} \textgreater\ \textit{umχe}		\sqt{donkey}
	\end{xlist}
\end{exe}


% - - - - - - - - - - - - - - - - - - - - - - - - - - - - - - - - - - - - - - - - - - - - - - - - - - - - - - - - - - - - - - - - - - - - - - - - - - - - - - - - - - - - - - - - - - - - - - - - - - - - - - - - - - - - - - - - - - - - - - - - - - %

\subsubsection{\textit{-te}}\largerpage

Most of the \isi{nouns} taking this suffix are disyllabic loan words ending in a resonant. This suffix is also used for the plural form of long \isi{adjectives} and, more generally, of predicates (\refsec{ssec:The -ce / -te attributive}).

\begin{exe}
	\ex	simple suffixation:
	\begin{xlist}
		\ex	\textit{šːišːim} > \textit{šːišːimte} \sqt{worry}
		\ex	\textit{tuχtur} > \textit{tuχturte} \sqt{doctor}
		\ex	\textit{qːanaw} > \textit{qːanawte} \sqt{ditch}
		\ex	\textit{ħaˁkim} > \textit{ħaˁkimte} \sqt{ruler}
		\ex	\textit{pawur} > \textit{pawurte} \sqt{cook} (Russian loan)
		\ex	\textit{mašin} > \textit{mašinte} \sqt{car} (Russian loan)
		\ex	\textit{bazar} > \textit{bazarte} \sqt{market}
		\ex	\textit{salam} > \textit{salamte} \sqt{greeting}
		\ex	\textit{kːapan} > \textit{kːapante} \sqt{shroud}
		\ex	\textit{kep} > \textit{kepte} \sqt{drinking}
		\ex	\textit{ukul} > \textit{ukulte} \sqt{injection}
		\ex	\textit{ħaˁjwan} > \textit{ħaˁjwante} \sqt{animal, livestock}
		\ex	\textit{dušman} > \textit{dušmante} \sqt{enemy}
		\ex	\textit{mahar} > \textit{maharte} \sqt{marriage}
	\end{xlist}

	\ex	\isi{vowel mutation}:
	\begin{xlist}
		\ex	\textit{χabar} > \textit{χaburte} \sqt{story, news}
		\ex	\textit{šajt'an} > \textit{šajt'unte} \sqt{devil}
		\ex	\textit{darman} > \textit{darmunte} \sqt{medicine}
		\ex	\textit{bajram} > \textit{bajrumte} \sqt{holiday}
		\ex	\textit{qːajtar} > \textit{qːajturte} \sqt{club}
		\ex	\textit{q'uˁšːem} > \textit{q'uˁšːumte} \sqt{handcuff}
	\end{xlist}

	\ex	other:
	\begin{xlist}
		\sn	\textit{admi} > \textit{adimte} \sqt{person, human being, man}
	\end{xlist}
\end{exe}


% - - - - - - - - - - - - - - - - - - - - - - - - - - - - - - - - - - - - - - - - - - - - - - - - - - - - - - - - - - - - - - - - - - - - - - - - - - - - - - - - - - - - - - - - - - - - - - - - - - - - - - - - - - - - - - - - - - - - - - - - - - %

\subsubsection{\textit{-be}}

\begin{exe}
	\ex	simple suffixation:
	\begin{xlist}
		\ex	\textit{ul} > \textit{ulbe} \sqt{eye}
		\ex	\textit{milic'a} > \textit{milic'abe} \sqt{police(man)} (Russian loan)
		\ex	\textit{pːalaženija} > \textit{pːalaženijabe} \sqt{situation} (Russian loan)
		\ex	\textit{xːun} > \textit{xːunbe} \sqt{road, way}
		\ex	\textit{ʡuˁnru} > \textit{ʡuˁnrube} \sqt{life}
	\end{xlist}

	\ex	\isi{vowel mutation}:
	\begin{xlist}
		\ex	\textit{naˁq} > \textit{nuˁqbe} \sqt{eye}
		\ex	\textit{qal} > \textit{qulbe} \sqt{house}
		\ex	\textit{nez} > \textit{nuzbe} \sqt{louse}
		\ex	\textit{qaˁr} > \textit{quˁrbe} \sqt{pear}
		\ex	\textit{ʁez} > \textit{ʁizbe} \sqt{hair}
	\end{xlist}

	\ex	\isi{vowel deletion} and \isi{degemination} (of fricative or affricate):
	\begin{xlist}
		\ex	\textit{cula} > \textit{culbe} \sqt{tooth}
		\ex	\textit{rursːi} > \textit{rursbe} \sqt{girl, daughter}
		\ex	\textit{hinci} > \textit{hincbe} \sqt{apple}
		\ex	\textit{rucːi} > \textit{rucbe} \sqt{sister}
		\ex	\textit{cːacːi} > \textit{cːacbe} \sqt{thorn}
	\end{xlist}
\end{exe}


% - - - - - - - - - - - - - - - - - - - - - - - - - - - - - - - - - - - - - - - - - - - - - - - - - - - - - - - - - - - - - - - - - - - - - - - - - - - - - - - - - - - - - - - - - - - - - - - - - - - - - - - - - - - - - - - - - - - - - - - - - - %

\subsubsection{\textit{-me}}

\begin{exe}
	\ex	\label{exPLURALinMEnouns} simple suffixation:
	\begin{xlist}
		\ex	\textit{pikru} > \textit{pikrume} \sqt{thought}
		\ex	\textit{buh} > \textit{buhme} \sqt{bundle}
		\ex	\textit{dus} > \textit{dusme} \sqt{year}
		\ex	\textit{t'uˁ} > \textit{t'uˁme} \sqt{leg}
		\ex	\textit{irk} > \textit{irkme} \sqt{threshing board}
		\ex	\textit{peč} > \textit{pečme} \sqt{oven} (Russian loan)
		\ex	\textit{šalakbluk} > \textit{šalakblukme} \sqt{building block made from concrete} (Russian loan)
	\end{xlist}

	\ex	\isi{vowel deletion} and \isi{degemination} (of stop or fricative)
	\begin{xlist}
		\ex	\textit{kːurtːi} > \textit{kːurtme} \sqt{dress, shirt}
		\ex	\textit{kːalkːi} > \textit{kːalkme} \sqt{tree}
		\ex	\textit{beretːa} > \textit{beretme} \sqt{ax}
		\ex	\textit{q'aˁli} > \textit{q'aˁlme} \sqt{branch}
		\ex	\textit{c'eltːa} > \textit{c'eltme} \sqt{gravestone}
		\ex	\textit{qulexːa} > \textit{qulexme} \sqt{bracelet}
		\ex	\textit{agarud} > \textit{agardme}\footnote{This example slightly differs from all the others because the vowel, which is deleted, occurs in the final syllable, but it is followed by consonant and thus does not represent the word-final segment.} \sqt{garden} (Russian loan)
	\end{xlist}
\end{exe}


% --------------------------------------------------------------------------------------------------------------------------------------------------------------------------------------------------------------------- %

\subsection{Modestly frequent plural suffixes}
\label{ssec:Modestlyfrequentpluralsuffixes}

The following plural suffixes are not used with recent borrowings.


% - - - - - - - - - - - - - - - - - - - - - - - - - - - - - - - - - - - - - - - - - - - - - - - - - - - - - - - - - - - - - - - - - - - - - - - - - - - - - - - - - - - - - - - - - - - - - - - - - - - - - - - - - - - - - - - - - - - - - - - - - - %

\subsubsection{\textit{-re}}

Many monosyllabic \isi{nouns} ending in a consonant take the plural suffix \textit{-re} after \isi{vowel mutation}. Disyllabic undergo \isi{vowel deletion} before the suffix is attached.

\begin{exe}
	\ex	\isi{vowel mutation} (\textit{-a} > \textit{-u}):
	\begin{xlist}
		\ex	\textit{qːap} > \textit{qːupre} \sqt{sack}
		\ex	\textit{maχ} > \textit{muχre} \sqt{wheelbarrow} (used to carry hay)
		\ex	\textit{t'at'} > \textit{t'ut're} \sqt{thread}
		\ex	\textit{k'at'} > \textit{k'ut're} \sqt{drop}
		\ex	\textit{lac} > \textit{lucre} \sqt{fence, wall}
		\ex	\textit{baˁʡ} > \textit{buˁʡre} \sqt{leaf, page}
	\end{xlist}

	\ex	\isi{vowel deletion}:
	\begin{xlist}
		\ex	\textit{migʷi} > \textit{migʷre} \sqt{large basket}
		\ex	\textit{ʡaˁmi} > \textit{ʡaˁmre} \sqt{window}
		\ex	\textit{ʁuma} > \textit{ʁumre} \sqt{block}
	\end{xlist}

	\ex	other:
	\begin{xlist}
		\sn	\textit{xːunul} > \textit{xːunre} \sqt{woman} 
	\end{xlist}
\end{exe}


% - - - - - - - - - - - - - - - - - - - - - - - - - - - - - - - - - - - - - - - - - - - - - - - - - - - - - - - - - - - - - - - - - - - - - - - - - - - - - - - - - - - - - - - - - - - - - - - - - - - - - - - - - - - - - - - - - - - - - - - - - - %

\subsubsection{\textit{-ne} }

The plural suffix \textit{-ne} is mostly found with disyllabic \isi{nouns} ending in \textit{-a} or very occasionally in \textit{-u/-uˁ}) after deleting the final vowel (and \isi{degemination} of the previous consonant if it is a \isi{tense consonant}). There are also a few monosyllabic \isi{nouns} that make use of the suffix.

\begin{exe}
	\ex	simple suffixation:
	\begin{xlist}
		\ex	\textit{ʁʷab} > \textit{ʁʷabne} \sqt{ploughshare}
		\ex	\textit{bar} > \textit{barne} \sqt{day}
	\end{xlist}

	\ex	\isi{vowel deletion} and \isi{degemination} (stops, fricatives, affricates):
	\begin{xlist}
		\ex	\textit{sːika} >\textit{ sːikne} \sqt{bear}
		\ex	\textit{ajarq'a} >\textit{ ajarq'ne} \sqt{hunter}
		\ex	\textit{žaq'a} > \textit{žaq'ne} \sqt{boar, pig}
		\ex	\textit{šuša} > \textit{šušne} \sqt{bottle}
		\ex	\textit{likːa} > \textit{likne} \sqt{bone}
		\ex	\textit{čatːa} > \textit{čatne} \sqt{cover}
		\ex	\textit{qːarqːa} > \textit{qːarqne} \sqt{stone}
		\ex	\textit{c'eltːa} > \textit{c'eltne} \sqt{gravestone}
		\ex	\textit{ečːa} > \textit{ečne} \sqt{she-goat}
		\ex	\textit{durħuˁ} > \textit{durħne} \sqt{boy, son}
	\end{xlist}
\end{exe}

% - - - - - - - - - - - - - - - - - - - - - - - - - - - - - - - - - - - - - - - - - - - - - - - - - - - - - - - - - - - - - - - - - - - - - - - - - - - - - - - - - - - - - - - - - - - - - - - - - - - - - - - - - - - - - - - - - - - - - - - - - - %

\subsubsection{\textit{-upːe}}

This suffix occurs with a handful of \isi{nouns} and by means of simple suffixation, final \isi{vowel deletion}, or surface appearance of underlying \isi{geminate consonant} (in the last noun).

\begin{exe}
	\ex	
	\begin{xlist}
		\ex	\textit{zunra} > \textit{zunrupːe} \sqt{neighbor}
		\ex	\textit{daˁqaˁ} > \textit{daˁquˁpːe} \sqt{wound}
		\ex	\textit{muza} > \textit{muzupːe} \sqt{corner}
		\ex	\textit{ʡaˁq'lu} > \textit{ʡaˁq'lupːe} \sqt{mind}
		\ex	\textit{qːuˁnq} > \textit{qːuˁnqːuˁpːe} \sqt{nose}
	\end{xlist}
\end{exe}


% - - - - - - - - - - - - - - - - - - - - - - - - - - - - - - - - - - - - - - - - - - - - - - - - - - - - - - - - - - - - - - - - - - - - - - - - - - - - - - - - - - - - - - - - - - - - - - - - - - - - - - - - - - - - - - - - - - - - - - - - - - %

\subsubsection{\textit{-urbe}}

The suffix \textit{-urbe} is mostly attested with disyllabic \isi{nouns} ending in \textit{-a} or \textit{-i}. It can be simply added to  \isi{nouns} ending in \isi{consonants}; otherwise final \isi{vowel deletion} applies.

\begin{exe}
	\ex	
	\begin{xlist}
		\ex	\textit{lac} > \textit{lacurbe} \sqt{fence, wall}
		\ex	\textit{dahag} > \textit{dahagurbe} \sqt{slope}
		\ex	\textit{uncːa} > \textit{uncːurbe} \sqt{door}
		\ex	\textit{ duʁa} > \textit{duʁurbe} \sqt{hayloft}
		\ex	\textit{mašːi} > \textit{mašːurbe} \sqt{farmstead}
		\ex	\textit{ʡaˁči} > \textit{ʡaˁčurbe} \sqt{work}
	\end{xlist}
\end{exe}


% - - - - - - - - - - - - - - - - - - - - - - - - - - - - - - - - - - - - - - - - - - - - - - - - - - - - - - - - - - - - - - - - - - - - - - - - - - - - - - - - - - - - - - - - - - - - - - - - - - - - - - - - - - - - - - - - - - - - - - - - - - %

\subsubsection{Other suffixes}

There are a few rare suffixes that only occur with a very small \isi{number} of \isi{nouns}. Some of these \isi{nouns} also make use of alternative, more common plural markers.\pagebreak

\begin{exe}
	\ex	\textit{-urme}:
	\begin{xlist}
		\ex	\textit{baˁʡ} > \textit{baˁʡuˁrme} \sqt{muzzle, wall, facade}
		\ex	\textit{ʡaˁdat} > \textit{ʡaˁdaturme} \sqt{habit, tradition}
		\ex	\textit{bek'a} > \textit{ bek'urme} \sqt{pile, heap}
		\ex	\textit{baha} > \textit{bahurme} \sqt{price}
		\ex	\textit{baʔ} > \textit{baʔurme} \sqt{end, tail, top, summit}
	\end{xlist}

	\ex	\textit{-ude}:
	\begin{xlist}
		\sn	\textit{χːʷe} > \textit{χːude} \sqt{dog}
	\end{xlist}

	\ex	\textit{-une}:
	\begin{xlist}
		\sn	\textit{ʡaˁdat} > \textit{ʡaˁdatune} \sqt{habit, tradition}
	\end{xlist}

	\ex	\textit{-(u)bne}:
	\begin{xlist}
		\sn	\textit{malla} > \textit{mallubne} \sqt{mullah}
	\end{xlist}
\end{exe}

\subsection{Morphophonological rules and other restrictions}\label{ssec:MorphophonologicalrulesPluralA}\largerpage
The distribution of the plural markers is basically lexical. For certain derived \isi{nouns} plural marking is predictable (\teg\ \isi{agent} \isi{nouns} ending in \textit{-či} take the plural suffix \textit{-be}, abstract \isi{nouns} with the suffix \textit{-dex} take the plural suffix \textit{-e} and undergo gemination of the suffix-final consonant). There are many \isi{nouns} that can attach more than one plural suffix, and both in texts and in elicitations one frequently comes across variation between speakers as well as within the speech of individuals. Examples are provided in \refex{ex:urcihorse}.

\begin{exe}
\ex \label{ex:urcihorse}
\begin{xlist}
	\ex	\tit{urči} 	>	\tit{urče\slash určme} 	\sqt{horse} 	
	\ex	\tit{ʡaˁdat} 	>	\tit{ʡaˁdaturme\slash ʡaˁdatme\slash ʡaˁdate\slash ʡaˁdatune} 	\sqt{habit, tradition}	
	\ex	\tit{dard} 	>	\tit{dardane\slash dardme}	\sqt{sorrow}		
	\end{xlist}
\end{exe}

\noindent Plural suffixation is accompanied by a \isi{number} of morphophonological processes:

\begin{itemize}
	\item	deletion of final \textit{-a}, \textit{-u} or \textit{-i} of mostly disyllabic \isi{nouns} with the suffixes \textit{-be}, \textit{-ne}, \textit{-me}, \textit{-re}, \textit{-e}, \textit{-upːe}, \textit{-urbe}, \textit{-ube}
	\item	\isi{vowel mutation} \textit{-a} > \textit{-u} and \textit{-aˁ} > \textit{-uˁ}, \textit{-e} > \textit{-u}, \textit{-e} > \textit{-i} of the vowel in the final syllable of words ending in a consonant with the suffixes \textit{-e}, \textit{-te}, \textit{-be}, \textit{-re} 
	\item	surface occurrence of geminate stop, fricative or affricate with the suffix \textit{-e} and \textit{-upːe}
	\item	\isi{degemination} of stops, fricatives and affricates in the final syllable before a vowel with the suffixes \textit{-ne}, \textit{-be} and occasionally \textit{-e}, \textit{-ube}
\end{itemize}

With a few \isi{nouns}, the last vowel shifts to \textit{u} or \textit{uˁ}, and, as a consequence, the preceding labialized consonant is automatically delabialized as in \textit{mikʷa} > \textit{mikupːe} \sqt{fingernail}, \textit{χːʷe} > \textit{χːude} \sqt{dog}, and \textit{q'ʷaˁl} > \textit{q'uˁle} \sqt{cow}. However, in most cases, no vowel shift takes place and thus labialized \isi{consonants} are not delabialized, and only the above-mentioned morphophonological processes take place (\isi{vowel deletion}, \isi{degemination}), for example \textit{mikʷ} > \textit{mikʷbe} \sqt{oak}, \textit{gʷagʷa} > \textit{gʷagʷne} \sqt{flower}, \textit{kaˁχːʷi} > \textit{kaˁχʷne} \sqt{dustpan}, and \textit{žilixʷa} > \textit{žilixʷme} \sqt{saddle}. See \refsec{sec:Phonological and morphophonological alternations} for more information on the morphophonological processes.

When case suffixes are added to \isi{nouns} overtly marked for plural, then the final vowel of the plural suffix changes from \textit{-e} to \textit{-a} (see \refsec{sec:nouncase} for examples).

\subsection{The associative plural}
\label{ssec:MorphophonologicalrulesPlural}
In addition to the normal plural, Sanzhi has an associative plural formed with the suffix \textit{-qal} that probably originates from the noun \textit{qal} \sqt{house}. The associative plural is only used with nominals that have specific reference:

\begin{itemize}
	\item	personal names\\
		\teg\ \textit{Pajt'ima-qal} \sqt{Patimat and the people associated with her}
	\item	terms denoting kinship relations\\
		\teg\ \textit{aba-qal} \sqt{mother and her relatives}, \textit{atːa-qal} \sqt{father and his relatives}
	\item	the pronoun \textit{ča} \sqt{who}, for which it is the regular means of forming the plural: \textit{ča-qal} (\refsec{ssec:ca who and ce what})
\end{itemize}

Some kinship terms (\teg\ \sqt{uncle}, \sqt{sister}) and personal names can also form the plural by means of regular plural suffixes, but there is a clear difference in meaning:

\begin{exe}
	\ex	\label{ex:Muʔminat the people associated with her}
	\begin{xlist}
		\ex	\textit{Muʔminat-be} \sqt{girls with the name Muʔminat}
		\ex	vs. \textit{Muʔminat-qal} \sqt {Muʔminat and the people associated with her}
		\ex	\textit{acːi-be} \sqt {uncles}
		\ex	vs. \textit{acːi-qal} \sqt {a specific uncle and his relatives and associates}
	\end{xlist}
\end{exe}

Cases are directly suffixed to the associative plural marker.


%%%%%%%%%%%%%%%%%%%%%%%%%%%%%%%%%%%%%%%%%%%%%%%%%%%%%%%%%%%%%%%%%%%%%%%%%%%%%%%%

\section{Gender\tnd number mismatches and exceptions}
\label{sec:Gendernumbermismatchesandexceptions}

There a couple of \isi{nouns} whose behavior deviates from the majority as described in the previous sections. This section provides examples of the different groups of divergent \isi{nouns}. There are three important parameters along which the divergent \isi{nouns} can be grouped:

\begin{itemize}
	\item availability of a morphological plural
	\item \isi{gender} agreement (only \textit{b}-agreement, only \textit{d}-agreement, or both)
	\item meaning (\teg\ mass noun interpretation)
\end{itemize}
%

The first clearly identifiable group consists of \isi{nouns} that denote liquids and other substances composed of small or minimal parts such as grains and dust-like materials. These \isi{nouns} normally control \textit{d}-agreement, but \textit{b}-agreement is possible if the noun is interpreted as denoting a specific quantity (\teg\ a bottle or a glass in case of liquids; one grain or one ear in case of sand or cereals). The specific quantity reading occurs frequently with some \isi{nouns} (\teg\ vodka, flour) and is therefore easier to obtain in elicitation. The \isi{nouns} do not have a morphological plural. Examples are:

\begin{exe}
	\ex	liquids\\
		\textit{čaˁʁir} \sqt{wine}, \textit{nejg} \sqt{milk}, \textit{kːamput'} \sqt{homemade juice}, \textit{čaˁj} \sqt{tea}, \textit{beʔe} \sqt{blood}, \textit{nerʁ} \sqt{soup}, \textit{ʡaˁraq'i} \sqt{vodka}, \textit{hin} \sqt{water}

	\ex	cereals, etc. \\
		\textit{ač'i} \sqt{wheat}, \textit{sːusːul} \sqt{rye}, \textit{ʡaˁjlač'i} \sqt{corn}, \textit{birinž} \sqt{rice} (with \textit{b}-agreement: a sack or single grain), \textit{t'ut'i} \sqt{grapes} (with \textit{b}-agreement: a single grape)
	
	\ex	other substances\\
		\textit{qːum} \sqt{sand}, \textit{cːe} \sqt{salt} (with \textit{b}-agreement: a specific quantity or one grain), \textit{pisuk'} \sqt{caster sugar} (with \textit{b}-agreement: one bowl or sack), \textit{bet'u} \sqt{flour} (with \textit{b}-agreement: one sack)
\end{exe}

The second group contains mass \isi{nouns} that control only \textit{b}-agreement and lack a morphological plural, for instance \textit{bergʷa} \sqt{smoke}, \textit{erza} \sqt{dew}, \textit{duˁħi} \sqt{snow}, and \textit{baˁqaˁla} \sqt{butter}. They can be reasonably treated as controlling neuter singular agreement.

The third group consists only of one noun \textit{χalq'} \sqt{people(s)}, which controls \textit{b}-agree\-ment and lacks a morphological plural. Because of its semantics it is classified as human plural.  

The fourth group is composed of mass \isi{nouns} that control only \textit{d}-agreement and also lack a morphological plural: \textit{mura} \sqt{hay}, \textit{pːala} \sqt{wool}, \textit{nekʷ} \sqt{straw}, \textit{qʷesːa} \sqt{ashes}, \textit{dalga} \sqt{tool, product, detail}, \textit{wajaˁħ} \sqt{thing}, and \textit{šuˁt'a} \sqt{saliva, spittle}.

The fifth group consists of \isi{nouns} that lack a singular form and only occur with what seems to be a frozen plural suffix. These \isi{nouns} control plural agreement (\textit{d}-agreement), for example \textit{mecːe} \sqt{stinging nettle}, \textit{t'alaħne} \sqt{dishes}, \textit{cːurbe} \sqt{heaven}, and \textit{susme} \sqt{throat}.

The last four groups are given in \refexrange{ex:bdagreementwithdistinctmeaningsnoplural}{ex:bdagreementwithoutdistinctmeaningswithplural}. It is not always possible to clearly identify the mass noun reading. For all words in \refex{ex:bdagreementwithdistinctmeaningsnoplural} and \refex{ex:bdagreementwithdistinctmeaningswithplural} that have the label `many' in parenthesis after the English translation, the label `many' refers to the normal (collective or distributive) plural reading, e.g. many individual apricots.\largerpage[-1]

\begin{exe}
	\TabPositions{8em,12em}
	\ex	\tit{b-} and \tit{d-}agreement with plural meaning and/or mass noun reading; no morphological plural:	\label{ex:bdagreementwithdistinctmeaningsnoplural}
	\begin{xlist}
		\ex	\tit{macːa}		\tab	\tit{b-}agr.		\tab	\sqt{sheep} (one)
		\sn	{}			\tab	\tit{d-}agr. 		\tab	\sqt{sheep} (many\slash mass noun)

		\ex	\tit{q'ar}		\tab	\tit{b-}agr.		\tab	\sqt{blade of grass}
		\sn	{}			\tab	\tit{d-}agr.		\tab	\sqt{grass, herbs} (mass noun)

		\ex	\tit{ʁaj}		\tab	\tit{b-}agr.		\tab	\sqt{word(s)} (one\slash mass noun)
		\sn	{}			\tab	\tit{d-}agr.		\tab	\sqt{language, speech} (mass noun)

		\ex	\tit{qurekːa}		\tab	\tit{b-}agr.		\tab	\sqt{apricot} (one)
		\sn	{}			\tab	\tit{d-}agr.		\tab	\sqt{apricots} (many)

		\ex	\tit{maʔ}		\tab	\tit{b-}agr.		\tab	\sqt{brain, marrow} (one\slash mass noun)
		\sn	{}			\tab	\tit{d-}agr.		\tab	\sqt{brains} (many)	

		\ex	\tit{t'ama}		\tab	\tit{b-}agr.		\tab	\sqt{voice, sound} (one\slash mass noun)
		\sn	{}			\tab	\tit{d-}agr.		\tab	\sqt{voices, sounds} (many)

		\ex	\tit{bac}		\tab	\tit{b-}agr.		\tab	\sqt{month, moon} (one)
		\sn	{}			\tab	\tit{d-}agr.		\tab	\sqt{months, moons} (many)

		\ex	\tit{tːaˁm}		\tab	\tit{b-}agr.		\tab	\sqt{trap} (one)
		\sn	{}			\tab	\tit{d-}agr.		\tab	\sqt{traps} (many)

		\ex	\tit{šːala}		\tab	\tit{b-}agr.		\tab	\sqt{light} (one\slash mass noun)
		\sn	{}			\tab	\tit{d-}agr.		\tab	\sqt{light(s)} (many\slash mass noun)

		\ex	\tit{c'a}		\tab	\tit{b-}agr.		\tab	\sqt{fire} (one\slash mass noun)
		\sn	{}			\tab	\tit{d-}agr.		\tab	\sqt{fire(s)} (many\slash mass noun)
	\end{xlist}
	\largerpage[-2]

	\ex	\tit{b-} and \tit{d-}agreement with distinct meanings; with morphologically formed plural:\label{ex:bdagreementwithdistinctmeaningswithplural}%
	\begin{xlist}
		\ex	\tit{meχ}		\tab	\tit{b-}agr.		\tab	\sqt{iron} (piece of iron)
		\sn	{}			\tab	\tit{d-}agr. 		\tab	\sqt{lock}
		\sn	plural \tit{meχbe}		\tab	\tit{d-}agr.		\tab	\sqt{pieces of iron\slash iron as mass noun}

		\ex	\tit{qːuqːu-laˁmc'}	\tab	\tit{b-}agr.		\tab	\sqt{lightning} (one\slash mass noun)
		\sn	{}			\tab	\tit{d-}agr. 		\tab	\sqt{lightning} (many\slash mass noun)
		\sn	 pl. \tit{qːuqːu-laˁmc'ne}\tab	\tit{d-}agr.		\tab	\sqt{lightnings}	   

		\ex	\tit{qːarqːa}		\tab	\tit{b-}agr.		\tab	\sqt{stone} (one)
		\sn	{}			\tab	\tit{d-}agr. 		\tab	\sqt{stones} (many\slash mass noun)
		\sn	plural \tit{qːarqne} 	\tab	\tit{d-}agr.		\tab	\sqt{stones} (many)
		
		\ex	\tit{ʁez}		\tab	\tit{b-}agr.		\tab	\sqt{hair} (one)
		\sn	{}			\tab	(\tit{d-}agr. 		\tab	\sqt{hair(s)} (many\slash mass noun))\footnote{This use not very common since there is a noun \textit{q'ačme} with the mass noun reading of `hair'.}
		\sn	plural \tit{ʁizbe} 		\tab	\tit{d-}agr.		\tab	\sqt{hairs} (many)

		\ex	\tit{čakar}		\tab	\tit{b-}agr.		\tab	\sqt{sugar} (mass noun\slash one grain, piece)
		\sn	{}			\tab	\tit{d-}agr. 		\tab	\tmd
		\sn	plural \tit{čakurte} 		\tab	\tit{d-}agr.		\tab	\sqt{pieces of sugar}
	\end{xlist}

	\ex	\tit{b-} and \tit{d-}agreement with no clearly distinct meanings (normally including mass noun interpretation); no morphological plural:		\label{ex:bdagreementwithoutdistinctmeaningsnoplural}
	\begin{xlist} 
		\ex	\tit{ims}		\tab	\tit{b-/d-}agr.	\tab	\sqt{moth(s)}   
		\ex	\tit{xʷe}		\tab	\tit{b-/d-}agr.	\tab	\sqt{seed(s)}   
		\ex	\tit{daˁʡwi}		\tab	\tit{b-/d-}agr.	\tab	\sqt{war}	   
		\ex	\tit{murhe}		\tab	\tit{b-/d-}agr.	\tab	\sqt{gold}	   
		\ex	\tit{xːamxːa}	\tab	\tit{b-/d-}agr.	\tab	\sqt{foam}   
		\ex	\tit{šaχ}		\tab	\tit{b-/d-}agr.	\tab	\sqt{hoarfrost, frost}
	\end{xlist}
   
	\ex	\tit{b-} and \tit{d-}agreement with no clearly distinct meanings; with morphological plural:\label{ex:bdagreementwithoutdistinctmeaningswithplural}%
	\begin{xlist}   
		\ex	\tit{uncːa}		\tab	\tit{b-/d-}agr.	\tab	\sqt{door}
		\sn	plural \tit{uncːurbe}	\tab	\tit{d-}agr.		\tab	\sqt{doors}

		\ex	\tit{qix}		\tab	\tit{b-/d-}agr.	\tab	\sqt{nut(s)}
		\sn	plural \tit{qixbe}		\tab	\tit{d-}agr.		\tab	\sqt{nuts}

		\ex	\tit{ʁaˁl}		\tab	\tit{b-/d-}agr.	\tab	\sqt{sleigh(s)}
		\sn	plural \tit{ʁaˁlme}		\tab	\tit{d-}agr.		\tab	\sqt{sleighs}
	\end{xlist}
\end{exe}


%%%%%%%%%%%%%%%%%%%%%%%%%%%%%%%%%%%%%%%%%%%%%%%%%%%%%%%%%%%%%%%%%%%%%%%%%%%%%%%%

\section{Case}
\label{sec:nouncase}

Sanzhi Dargwa has four \isi{grammatical cases} and 19 core semantic cases as well as one minor directional suffix. The \isi{grammatical cases} and the \isi{comitative} are given in \reftab{tab:Grammatical cases}. The 18 core \is{spatial case}spatial cases are provided in \reftab{tab:Spatial cases}. The essive is shown in the neuter singular/human plural form with the \isi{gender}\tnd \isi{number} suffix \textit{-b}. Illustrative partial paradigms of a few \isi{nouns} can be found in Tables~\ref{tab:Partial paradigms of a few nouns (vowel-final stems)}--\ref{tab:Partial paradigms of a few nouns (consonant-final stems)}.

\begin{table}
	\caption{Grammatical cases}
	\label{tab:Grammatical cases}
	\small
	\begin{tabularx}{0.35\textwidth}[]{%
		>{\raggedright\arraybackslash}p{42pt}
		>{\arraybackslash}X}
		
		\lsptoprule
		Case		&	Suffix\\
		\midrule 
		\isit{absolutive}	&	\tit{-}∅\\   
		\isit{ergative}	&	\tit{-l(i)}\\
		\isit{genitive}	&	\tit{-la} \tit{(-lla)}\\
		\isit{dative}		&	\tit{-j}\\
		\isit{comitative}	&	\tit{-cːella}\\
		\lspbottomrule
	\end{tabularx}
\end{table}

\begin{table}
	\caption{Spatial cases}
	\label{tab:Spatial cases}
	\small
	\begin{tabularx}{0.75\textwidth}[]{%
		>{\raggedright\arraybackslash}p{72pt}
		>{\raggedright\arraybackslash}X
		>{\raggedright\arraybackslash}X
		>{\raggedright\arraybackslash}X}
		
		\lsptoprule
		Meaning			&	Lative			&	Essive			&	Ablative\\ 
		\midrule
		\sqt{in, on, to} (\tsc{loc})	 	&	\tit{-le}\slash 		&	\tit{-le-b}\slash 		&	\tit{-le-r(-ka)}\slash\\
		{}				&	\tit{-ja}\slash 		&	\tit{-ja-b}\slash		&	\tit{-ja-r(-ka)}\slash\\
		{}				&	\tit{-a}		&	\tit{-a-b}		&	\tit{-a-r(-ka)}\\
		\sqt{to} (\tsc{ad})		&	\tit{-šːu}		&	\tit{-šːu-b}		&	\tit{-šːu-r(-ka)}\\
		\sqt{in, on, at} 	&	\tit{-cːe}		&	\tit{-cːe-b}		&	\tit{-cːe-r(-ka)}\\
		~~ \sqt{among} (\tsc{in})\\
		\sqt{under} (\tsc{sub})	&	\tit{-gu}		&	\tit{-gu-b}		&	\tit{-gu-r(-ka)}\\
		\sqt{in front} (\tsc{ante})	&	\tit{-sa}		&	\tit{-sa-b}		&	\tit{-sa-r(-ka)}\\
		\sqt{behind} (\tsc{post})	&	\tit{-hara}		&	\tit{-hara-b}		&	\tit{-hara-r(-ka)}\\
		\lspbottomrule
	\end{tabularx}
\end{table}

\begin{table}
	\caption{Partial paradigms of two nouns (vowel-final stems)}
	\label{tab:Partial paradigms of a few nouns (vowel-final stems)}
	\small
	\begin{tabularx}{0.75\textwidth}[]{%
		>{\raggedright\arraybackslash}p{36pt}
		>{\raggedright\arraybackslash}p{56pt}
		>{\raggedright\arraybackslash}X
		>{\raggedright\arraybackslash}X}
		
		\lsptoprule
		{}		&	Case		&	\sqt{tree}		&	\sqt{donkey}\\
		\midrule
		singular	&	\isit{absolutive}	&	\tit{kːalkːi}		&	\tit{amχa}\\
		{}		&	\isit{ergative}	&	\tit{kːalkːi-l}		&	\tit{amχa-l}\\
		{}		&	\isit{genitive}	&	\tit{kːalkːi-la}	&	\tit{amχa-la}\\
		{}		&	\isit{dative}		&	\tit{kːalkːi-(li)-j}	&	\tit{amχa-j}\\
		{}		&	\isit{comitative}	&	\tit{kːalkːi-cːella}	&	\tit{amχa-cːella}\\
		{}		&	\tsc{ad}-lative	&	\tit{kːalkːi-šːu}	&	\tit{amχa-šːu}\\[2mm]

		plural		&	\isit{absolutive}	&	\tit{kːalk-me}	&	\tit{umχ-e}\\
		{}		&	\isit{ergative}	&	\tit{kːalk-m-a-l}	&	\tit{umχ-a-l}\\
		{}		&	\isit{genitive}	&	\tit{kːalk-m-a-(l)la}	&	\tit{umχ-a-(l)la}\\
		\lspbottomrule
	\end{tabularx}
\end{table}

\begin{table}
	\caption{Partial paradigms of two nouns (consonant-final stems)}
	\label{tab:Partial paradigms of a few nouns (consonant-final stems)}
	\small
	\begin{tabularx}{0.75\textwidth}[]{%
		>{\raggedright\arraybackslash}p{36pt}
		>{\raggedright\arraybackslash}p{56pt}
		>{\raggedright\arraybackslash}X
		>{\raggedright\arraybackslash}X}
		
		\lsptoprule
		{}		&	Case		&	\sqt{friend}		&	\sqt{clothes}\\
		\midrule
		singular	&	\isit{absolutive}	&	\tit{juldaš}		&	\tit{paltar}\\
		{}		&	\isit{ergative}	&	\tit{juldaš-li}		&	\tit{paltar-ri}\slash\tit{-li}\\
		{}		&	\isit{genitive}	&	\tit{juldaš-la}	&	\tit{paltar-ra}\slash\tit{-la}\\
		{}		&	\isit{dative}		&	\tit{juldaš-li-j}	&	\tit{paltar-ri-j}\slash\tit{-li-j}\\
		{}		&	\isit{comitative}	&	\tit{juldaš-li-cːella}	&	\tit{paltar-ri-cːella}\slash\tit{-li-cːella}\\
		{}		&	\tsc{ad}-lative	&	\tit{juldaš-li-šːu}	&	\tit{paltar-ri-šːu}\slash\tit{-li-šːu}\\[2mm]

		plural		&	\isit{absolutive}	&	\tit{juldašː-e}	&	\tit{paltur-te}\\
		{}		&	\isit{ergative}	&	\tit{juldašː-a-l}	&	\tit{paltur-t-a-l}\\
		{}		&	\isit{genitive}	&	\tit{juldašː-a-(l)la}	&	\tit{paltur-t-a-(l)la}\\
		\lspbottomrule
	\end{tabularx}
\end{table}

Case suffixation is (almost) completely regular and predictable. Like in many other East Caucasian languages, including other Dargwa varieties, case suffixes in Sanzhi for the most part do not directly attach to the nominal root, but are preceded by a so-called oblique marker. For \isi{nouns} in the singular, the oblique marker is identical to the \isi{ergative} suffix \textit{-li} and will be glossed with \textsc{obl}. Demonstrative pronouns in the singular have \textit{-i} as the oblique marker; all \isi{nouns} and \is{demonstrative pronoun}demonstrative pronouns in the plural have \textit{-a}.

There are a few differences between \isi{nouns} ending in a vowel and \isi{nouns} ending in a consonant with respect to the distribution and the usage frequency of oblique markers in the singular. With \isi{nouns} ending in a vowel all case suffixes are mostly directly added to the nominal stem (\reftab{tab:Partial paradigms of a few nouns (vowel-final stems)}), but occasionally the oblique marker \textit{-l(i)} precedes suffixes of semantic cases, as in \textit{bušːukala-l-cːella} broom-\textsc{obl}-\tsc{comit} \sqt{with the broom}, \textit{q'aca-l(i)-šːu} \sqt{to the goat} (he.goat-\textsc{obl}-\tsc{ad}). With \isi{nouns} ending in a consonant (\reftab{tab:Partial paradigms of a few nouns (consonant-final stems)}), the oblique marker obligatorily precedes the \isi{dative} and the \isi{comitative}, and is normally also used before all \is{spatial case}spatial cases except for the \textsc{loc}-series with \textit{-le}. If \isi{nouns} are marked for the plural, then overt case suffixes are always added to the plural oblique marker \textit{-a}, never directly to the plural stem.

The suffixes for the \isi{ergative} and for the \isi{genitive} can and most frequently do assimilate after \textit{n} and \textit{r} (and \textit{s} in the noun \textit{dus} `year') to \textit{-ri}/\textit{-ni}/\textit{-si} (\isi{ergative}) and \textit{-ra}/\textit{-na} (\isi{genitive}). The \isi{genitive} of nominals in the plural is frequently realized as \textit{-lla} instead of \tit{-la}, but this phenomenon seems to be at least partially subject to variation. It is hard to notice in audio recordings of natural texts and speakers are not always aware of it. Therefore, it will mostly not be acknowledged in the examples. 

% --------------------------------------------------------------------------------------------------------------------------------------------------------------------------------------------------------------------- %


\subsection{Functions of grammatical cases} 
\label{ssec:Functionsofgrammaticalcases}


% - - - - - - - - - - - - - - - - - - - - - - - - - - - - - - - - - - - - - - - - - - - - - - - - - - - - - - - - - - - - - - - - - - - - - - - - - - - - - - - - - - - - - - - - - - - - - - - - - - - - - - - - - - - - - - - - - - - - - - - - - - %

\subsubsection{Absolutive}
\label{sssec:Absolutive}

The \isi{absolutive} case is zero-marked and not indicated in the glosses. It occurs in the following contexts:

\begin{enumerate}
	\item	on the sole argument of intransitive \refex{ex:My arms got tired} and extended \is{intransitive verb}intransitive verbs (\refsec{sec:Intransitive verbs}, \refsec{sec:Extended intransitive verbs}):
	\begin{exe}
		\ex	\label{ex:My arms got tired}
		\gll	di-la nuˁq-be	ʡaˁbħ-ib ca<d>i\\
			\tsc{1sg}-\textsc{gen}	arm-\textsc{pl}	get.tired.\textsc{pfv}-\textsc{pret}	\textsc{cop<npl>}\\
		\glt	\sqt{My arms got tired.}
	\end{exe}

	\item	on the \isi{patient} or \isi{theme} argument of transitive and ditransitive (extended transitive) verbs, (\refsec{sec:Transitive verbs}), (\refsec{sec:Extended transitive verbs and ditransitive verbs}):
	\begin{exe}
		\ex	\label{ex:I will bring nuts}
		\gll	du-l	ka-d-iqː-an=da	qix-be\\
			\textsc{1sg}-\textsc{erg}	\textsc{down-npl}-carry.\textsc{ipfv}-\textsc{ptcp}=\tsc{1}	nut-\textsc{pl}\\
		\glt	\sqt{I will bring nuts.}
	\end{exe}

	\item	on the \isi{stimulus} argument of \is{affective verb}affective verbs (\refsec{sec:Bivalent affective verbs}):
	\begin{exe}
		\ex	\label{ex:They did not know the Russian language}
		\gll	itːa-j		ʡuˁrus	ʁaj	d-alχ-ul	akːʷ-i=q'al\\
			those.\textsc{obl}-\textsc{dat}	Russian	language	\textsc{npl}-know.\textsc{ipfv}-\textsc{icvb}	\textsc{cop.neg}-\textsc{hab}.\textsc{pst}=\textsc{mod}\\
		\glt	\sqt{They did not know the Russian language.}
	\end{exe}

	\item	on the \isi{agent} in the \isi{antipassive} construction (\refsec{sec:Antipassive}):
	\begin{exe}
		\ex	\label{ex:I (masc.) eat fish}
		\gll	du	baliqː-a-l	∅-uk-un=da\\
			\textsc{1sg}	fish-\textsc{obl}.\textsc{pl}-\textsc{erg}	\textsc{m}-eat.\textsc{ipfv}-\textsc{icvb}=1\\
		\glt	\sqt{I (masc.) eat fish.}
	\end{exe}

	\item	on subject-like arguments and nominal predicates in \isi{copula} clauses (\refsec{sec:copulaclauses}):
	\begin{exe}
		\ex	\label{ex:The people were hungry}
		\gll	χalq'	kːuš-le=de\\
			people	hungry-\textsc{advz}=\textsc{pst}\\
		\glt	\sqt{The people were hungry.}
		
	\ex	\label{ex:‎‎‎That boy is a monster2}
	\gll	het	durħuˁ	aždaha	ca-w	\\
		that	boy	monster	\tsc{cop-m}	\\
	\glt	\sqt{‎‎‎That boy is a monster.} (E)
		
	\end{exe}

	\item	on expressions of temporal duration (\tie\ for a certain period):
	\begin{exe}
		\ex	\label{ex:(They) leave (them) for 20 days}
		\gll	d-alt-u	ʁajal	bari\\
			\textsc{npl}-let.\textsc{ipfv}-\textsc{prs.3}	twenty	day\\
		\glt	\sqt{(They) leave (them) for 20 days.}

		\ex	\label{ex:I stayed there for three years}
		\gll	ʡaˁbal	dus	kelg-un=da\\
			three	year	remain.\textsc{pfv}-\textsc{pret}=1\\
		\glt	\sqt{I stayed there for three years.}
	\end{exe}
	
	\item	in \isi{vocative} function:
		\begin{exe}
		\ex	\label{ex:What do you say, mother?}
			\gll	ce r-ik'-utːe, aba?\\
			what \textsc{f}-say.\textsc{ipfv-2sg.prs} mother			\\
		\glt	\sqt{What do you say, mother?}
	\end{exe}
\end{enumerate}


% - - - - - - - - - - - - - - - - - - - - - - - - - - - - - - - - - - - - - - - - - - - - - - - - - - - - - - - - - - - - - - - - - - - - - - - - - - - - - - - - - - - - - - - - - - - - - - - - - - - - - - - - - - - - - - - - - - - - - - - - - - %

\subsubsection{Ergative}
\label{sssec:Ergative}

The \isi{ergative} suffix is \textit{-li} (allomorphs \textit{-ni}, \textit{-ri} after \textit{n} and \textit{r} respectively, and \textit{-l}, which can only be added to vowels). The \isi{ergative} occurs in the following contexts:

\begin{enumerate}
	\item	on the \isi{agent} of transitive and ditransitive predicates, including inanimate agents:%
	\begin{exe}
		\ex	\label{ex:We did not eat boars, right?}
		\gll	žaq'-ne		a-d-uk-i	nušːa-l,	akːʷ-i=w?\\
			boar-\textsc{pl}		\textsc{neg}-\textsc{npl}-eat.\textsc{ipfv}-\textsc{hab}.\textsc{pst}	1\textsc{pl}-\textsc{erg}	\textsc{cop.neg}-\textsc{hab}.\textsc{pst}=\textsc{q}\\
		\glt	\sqt{We did not eat boars, right?}

		\ex	\label{ex:The water turns this around}
		\gll	hin-ni		heχ	lus	b-ik'-aq-u\\
			water-\textsc{erg}	\textsc{dem.down}	around \textsc{n}-move.\textsc{ipfv}-\textsc{caus}-\textsc{prs}\\
		\glt	\sqt{The water turns this around.}
	\end{exe}

	\item	 on expressions of reason/cause (though the \isi{dative} is more common in this function, see \refsec{sssec:Dative}):
	\begin{exe}
		\ex	\label{ex:Many people died of hunger}
		\gll	kːiši-l		imc'a-l		χalq'	b-ebč'-ib\\
			hunger-\textsc{erg}	additional-\textsc{advz}	people	\textsc{hpl}-die.\textsc{pfv}-\textsc{pret}\\
		\glt	\sqt{Many people died of hunger.}
	\end{exe}

	\item	 on instruments (though the \isi{comitative} is more frequent in this function, see\linebreak\refsec{sssec:Comitative}): In sentences such as \refex{ex:I cut the bread with a knife} with two ergatives it is only marginally possible to put the two \isi{ergative} items directly next to each other, presumably because this leads to processing difficulties. To overcome this problem the second \isi{ergative} is either placed in some other position, or is replaced with the \isi{comitative}:
	\begin{exe}
		\ex	\label{ex:I cut the bread with a knife}
		\gll	du-l ka-b-irčː-ul=da t'ult' dis-li / dis-li-cːella\\
			1\textsc{sg}-\textsc{erg}	\textsc{down-n}-cut.\textsc{ipfv}-\textsc{icvb}=1	bread	knife-\textsc{erg}	/ knife-\textsc{obl}-\textsc{comit}\\\\
		\glt	\sqt{I cut the bread with a knife.} (E)
	\end{exe}

	\item	in the construction with the verb \sqt{fill} \refex{ex:His yard was filled with chickens, turkeys, and geese}, \refex{ex:And they also filled the hut with rye}:
	\begin{exe}
		\ex
		\begin{xlist}
			\ex	\label{ex:His yard was filled with chickens, turkeys, and geese}
			\gll	 il-i-la		azbar	b-ic'-ib ca-b	ʡuˁrʡ-aˁ-l,	qːuz-r-a-l,	k'urk'ur-t-a-l\\
				that-\textsc{obl}-\textsc{gen}	yard	\textsc{n}-fill.\textsc{pfv}-\textsc{pret} \textsc{cop-n}	chicken-\textsc{obl}.\textsc{pl}-\textsc{erg}	goose-\textsc{pl}-\textsc{obl}-\textsc{erg} turkey.cock-\textsc{pl}-\textsc{obl}-\textsc{erg}\\
			\glt	\sqt{His yard was filled with chickens, geese, and turkeys.}
	
			\ex	\label{ex:And they also filled the hut with rye}
			\gll	b-ic'-ib	hel	buq'a=ra	sːusːul-li\\
				\textsc{n}-fill.\textsc{pfv}-\textsc{pret}	that	hut=\textsc{add}	rye-\textsc{erg}\\
			\glt	\sqt{And (they) also filled the hut with rye.}
		\end{xlist}
			\end{exe}
			
	\item	on the \isi{patient} in the \isi{antipassive} construction \refex{ex:I (masc.) eat fish};

	\item	when expressing the profession \refex{ex:He was the veterinarian}:
	\begin{exe}
			\ex	\label{ex:He was the veterinarian}
		\gll	il	ħaˁjwan	tuχtur-ri	kelg-un\\
			that	animal	doctor-\textsc{erg}	remain.\textsc{pfv}-\tsc{pret}\\
		\glt	\sqt{He was a/the veterinarian.}	
	\end{exe}
\end{enumerate}


% - - - - - - - - - - - - - - - - - - - - - - - - - - - - - - - - - - - - - - - - - - - - - - - - - - - - - - - - - - - - - - - - - - - - - - - - - - - - - - - - - - - - - - - - - - - - - - - - - - - - - - - - - - - - - - - - - - - - - - - - - - %

\subsubsection{Genitive}
\label{sssec:Genitive}

The \isi{genitive} suffix is \tit{-la} (allomorphs \tit{-na}, \tit{-ra} after \tit{n} and \tit{r}, and allomorph \tit{-lla} with many \isi{nouns} and pronouns marked for plural, and in some other contexts). It is used in the following contexts:

\begin{enumerate}
	\item	with various types of relations, \teg\ on noun modifiers denoting \isi{possession} \refex{ex:My arms got tired}, \refex{ex:in our village of Sanzhi}, \refex{ex:‎We need such a man, one with huge strength}, material \refex{ex:silver bracelet}, ingredients \refex{ex:We make chudu from all various herbs, we make it from dock}, units of measurement \refex{ex:five liter canister of wine}, properties \refex{ex:strong person}. Sanzhi does not distinguish between alienable and inalienable possessors. Some more information on constructions expressing \isi{possession} can be found in \refsec{sec:Possession}. The position of genitives at the level of the phrase is analyzed in \refsec{ssec:The structure and order of constituents within the noun phrase}.
	%
	\begin{exe}
	
		\ex	\label{ex:in our village of Sanzhi}
		\gll	nišːa-la	sungli-la	šːi-l-cːe-b\\
			\textsc{1pl-gen}	Sanzhi.person.\textsc{obl}-\textsc{gen}	village-\textsc{obl-in}-\textsc{n}\\
		\glt	\sqt{in our village of Sanzhi} (lit. \sqt{in our village of the Sanzhi people})
		
		
	\ex	\label{ex:‎We need such a man, one with huge strength}
	\gll	hel=ʁuna	admi	ʡaˁʁuni-l	ca-w	nišːi-j	ca	χːula-ce	hunar-ra w-ah\\
		that\tsc{=eq}	person	needed\tsc{-advz}	\tsc{cop-m}	\tsc{1pl-dat}		one	big\tsc{-dd.sg}	strength\tsc{-gen}	\tsc{m-}owner\\
	\glt	\sqt{‎We need such a man, one with huge strength.}
		
		
		\ex	\label{ex:silver bracelet}
		\gll	arc-la qulexa\\
			silver-\textsc{gen} bracelet\\
		\glt	\sqt{silver bracelet} (E)

		\ex	\label{ex:We make chudu from all various herbs, we make it from dock}
		\gll	har	žuralla	q'ar-ra	barcːik'ʷ	b-irq'-id,	galsːi-la	b-irq'-id\\
			every	various	herbs-\textsc{gen}	chudu	\textsc{n}-do.\textsc{ipfv}-1.\textsc{prs}	dock-\textsc{gen}	\textsc{n}-do.\textsc{ipfv}-1.\textsc{prs}\\
		\glt	\sqt{We make chudu (trad. food) from all various herbs, we make it from dock.}

		\ex	\label{ex:five liter canister of wine}
		\gll	xujal	litru-la		čaˁʁir-la	kanister\\
			five	liter-\textsc{gen}	wine-\textsc{gen}	canister\\
		\glt	\sqt{five-liter canister of wine}

		\ex	\label{ex:strong person}
		\gll	guž-la	admi\\
			strength-\textsc{gen}	person\\
		\glt	\sqt{strong person} (E)

	\end{exe}

	\item	on the arguments of most postpositions (\refsec{cpt:postpositions}):
	%
	\begin{exe}
		\ex	\label{ex:in front of the house}
		\gll	qal-la sala\\
			house-\textsc{gen}	in.front\\
		\glt	\sqt{in front of the house}

		\ex	\label{ex:behind the stone}
		\gll	qːarqːa-la	hila-b\\
			stone-\textsc{gen}	behind-\textsc{n}\\
		\glt	\sqt{behind the stone}
	\end{exe}\pagebreak

	\item	in \is{partitive construction}partitive constructions (grammatically, they represent a \isi{genitive} phrase with an omitted head noun):
	%
	\begin{exe}
		\ex	\label{ex:Drink (some) water}
		\gll	hin-na	b-erčː-a!\\
			water-\textsc{gen}	\textsc{n}-drink.\textsc{pfv}-\textsc{imp}\\
		\glt	\sqt{Drink (some) water!} (E)
	\end{exe}

\item	in the constructions of the \textit{fill}-type (for which normally the \isi{ergative} is used, see examples \refex{ex:His yard was filled with chickens, turkeys, and geese} and \refex{ex:And they also filled the hut with rye}); the \isi{genitive} is also possible in \refex{ex:Having filled (the glass) with water (he) drank it.}. In this example, the \isi{genitive} can be replaced by the \isi{ergative} without any change in meaning. As in the \isi{partitive construction}, the \isi{genitive} noun in \refex{ex:Having filled (the glass) with water (he) drank it.} is actually part of a \isi{genitive} phrase of which the head noun has not been expressed, but could be added at any time (e.g. \textit{kːuruškːa} \sqt{cup, mug}). In contexts in which no such head noun could be inserted, the \isi{genitive} is ungrammatical and the \isi{ergative} must be used instead. This applies to \refex{ex:His yard was filled with chickens, turkeys, and geese}, \refex{ex:And they also filled the hut with rye}, which would not be admissible with a \isi{genitive}.

	%
	\begin{exe}
		\ex	\label{ex:Having filled (the glass) with water (he) drank it.}
		\gll	b-ic'-ib-le, hin-na b-erčː-ib ca-b\\
			\textsc{n}-fill.\textsc{pfv-pret-cvb} water-\textsc{gen} \textsc{n}-drink.\textsc{pfv-pret} \textsc{cop-n}\\
		\glt	\sqt{Having filled (the glass) with water (he) drank it.} 
	\end{exe}
	
	\item	in the \tsc{gen} + \sqt{make} construction, there are a \isi{number} of lexicalized phrases that consist of a noun in the \isi{genitive} used together with the verb \textit{b-arq'}- \textsc{(pfv}) \sqt{do, make} (depending on the meaning there are also some other verbs allowed). This noun can usually not be described as serving any specific syntactic function in the clause, but instead forms a kind of compound together with the verb. The argument that is syntactically the \isi{direct object} and controls the \isi{gender} agreement on the verb functions as \isi{patient} or it takes over the role of the affected participant similar to a beneficiary (or maleficiary).\footnote{In the constructions in \refex{ex:(She) already married off one daughter}, \refex{ex:I married off my two sons}, \refex{ex:(I) buried my sister} and \refex{ex:Because of this you (masc.) will be beheaded} the \isi{direct object} (e.g. \textit{rursːi}) can perhaps be interpreted as the possessum and the \isi{genitive} noun (\textit{qal-la}) as possessor such that we would deal with a \isi{genitive} phrase. The phrase would, however, have the reverse case distribution of normal \isi{genitive} phrases. The possessor is normally a human referent and the possessum can be inanimate, but in the four examples it is the other way around. Furthermore, in \isi{genitive} phrases possessor and possessed usually occur next to each other in the order \isi{genitive} + noun, which is also not the case in these examples, but occasionally other orders are possible \refex{They dedicated themselves to their life (i.e. they cared for their living} (\refsec{ssec:Lexical, phrasal, and clausal modifiers in noun phrases}). In sum, an analysis in which the direct objects and the \isi{nouns} in the \isi{genitive} syntactically form \isi{genitive} phrases needs to be rejected.}  More examples of such \is{compound verb}compound verbs are given in \refsec{ssec:compoundswithnouns}.

	\begin{exe}
		\ex	\hspace{-1.31227pt}\tit{qalla + b-arq'-ij} \sqt{marry off}; \tit{qalla + ka-b-at-ij} \sqt{marry off} (house.\textsc{gen} + \textsc{down-hpl}-let.\textsc{pfv}-\textsc{inf}) \label{ex:qallabarqkabatmarryoff}
		\begin{xlist}
			\ex	 \label{ex:(She) already married off one daughter}
			\gll	ca qal-la	r-arq'-ib	cin-na	rursːi\\
				one	house-\textsc{gen} \textsc{f}-do.\textsc{pfv}-\textsc{pret}	\textsc{refl}.\textsc{sg}-\textsc{gen}	girl\\
			\glt	\sqt{(She) already married off one daughter.}
	
			\ex	\label{ex:I married off my two sons}
			\gll	di-la k'ʷel durħuˁ qal-la ka-w-at-ur=da\\
				1\textsc{sg}-\textsc{gen}	two	boy	house-\textsc{gen}	\textsc{down-m}-let.\textsc{pfv}-\textsc{pret}=1\\
			\glt	\sqt{I married off my two sons.} (E)
		\end{xlist}

		\ex	\tit{χːaˁbla + b-arq'-ij} \sqt{bury} \label{ex:(I) buried my sister}\\
		\gll	di-la	rucːi=ra	χːaˁb-la	r-arq'-ib-le=da\\
			1\textsc{sg}-\textsc{gen}	sister=\textsc{add}	grave-\textsc{gen}	\textsc{f}-do.\textsc{pfv}-\textsc{pret}-\textsc{cvb}=1\\
		\glt	\sqt{(I) buried my sister.}

		\ex	\tit{qaˁbla + b-arq'-ij} \sqt{behead}; \tit{qaˁbla + b-aˁq-ij} \sqt{behead} (neck.\textsc{gen} +\sqt{hit, strike, wound}) \label{ex:Because of this you (masc.) will be beheaded}\\
		\gll	il	bahandan	u	qaˁb-la	∅-urq-aˁn=de\\
			this	because.of	2\textsc{sg}	neck-\textsc{gen}	\textsc{m}-wound.\textsc{ipfv}-\textsc{ptcp}=2\textsc{sg}\\
		\glt	\sqt{Because of this you (masc.) will be beheaded!}
	\end{exe}

	\item	with the use of \isi{genitive} pronouns in emphatic \is{reflexive construction}reflexive constructions (\refsec{ssec:Emphatic reflexive use}):
	%
	\begin{exe}
		\ex	\label{ex:(You) yourself go away}
		\gll	ala	r-uˁq'-aˁn!\\
			2\textsc{sg}.\textsc{gen}	\textsc{f}-go-\textsc{imp}\\
		\glt	\sqt{(You) yourself (fem.) go away!} (E)
	\end{exe}
\end{enumerate}


% - - - - - - - - - - - - - - - - - - - - - - - - - - - - - - - - - - - - - - - - - - - - - - - - - - - - - - - - - - - - - - - - - - - - - - - - - - - - - - - - - - - - - - - - - - - - - - - - - - - - - - - - - - - - - - - - - - - - - - - - - - %

\subsubsection{Dative}
\label{sssec:Dative}

The \isi{dative} suffix is \textit{-j}. The \isi{dative} occurs in the following contexts:

\begin{enumerate}
	\item	\isi{experiencer} with affective predicates \xxref{ex:Why, he says, do you need to bring the boy here and there}{ex:Grandfather, are you cold} (\refsec{sec:Monovalent affective verbs and exceptional monovalent constructions}, \refsec{sec:Bivalent affective verbs})
	%
	\begin{exe}
		\ex	\label{ex:Why, he says, do you need to bring the boy here and there}
		\gll	ce	ħaˁžat-le,	∅-ik'ʷ-ar,	at	betsat	w-ič-itːaj	durħuˁ\\
			what	need-\textsc{advz}	\textsc{m}-say.\textsc{ipfv}-\textsc{prs} 2\textsc{sg}.\textsc{dat} here.there \textsc{m}-lead.\textsc{ipfv}-\textsc{subj.2}	boy\\
		\glt	\sqt{What need is there, he says, for you to bring the boy here and there.}
	
		\ex	\label{ex:anger takes me}
		\gll	dam	simi	d-ulq-u\\
			1\textsc{sg}.\textsc{dat}	anger	\textsc{npl}-direct.\textsc{ipfv}-\textsc{prs}\\
		\glt	\sqt{I am angry.} (lit. \sqt{anger directs to me}) (E)
	
		\ex	\label{ex:Grandfather, are you cold}
		\gll	χatːaj	at	b-uχːar-re=w? \\
			grandfather	2\textsc{sg}.\textsc{dat}	\textsc{n}-be.cold-\textsc{cvb=q}\\
		\glt	\sqt{Grandfather, are you cold?}
	\end{exe}
    \pagebreak
	\item	goal-like functions such as addressees \refex{ex:(They) said to me} (for this role, the \textsc{in}-lative is more common, see \refsec{sssec:in-lative -cːe, in-essive -cːe-b, and in-ablative -cːe-r}), recipients (\refsec{sec:Extended transitive verbs and ditransitive verbs}), beneficiaries\slash maleficiaries \refex{ex:(I) will make a big weeding for my daughter}, and goals of extended \is{intransitive verb}intransitive verbs with \isi{experiential} semantics \refexrange{ex:listening carefully to the sounds of the frogs}{ex:Now the boy believed it (=the mouse) and ran away} or occasionally spatial goals \refex{ex:The wasps flew after the boy and the puppy} and other types of goal-like constructions \refex{ex:This also (=the person on a picture) one resembles a woman}:
	%
	\begin{exe}
		\ex	\label{ex:(They) said to me}
		\gll	dam	/	di-cːe	b-urs-ib, \ldots\ \\
			1\textsc{sg}.\textsc{dat}	/	1\textsc{sg}.\textsc{obl-in}	\textsc{n}-say.\textsc{pfv}-\textsc{pret}\\
		\glt	\sqt{(They) said to me, \ldots}
		
		\ex	\label{ex:(I) will make a big weeding for my daughter}
		\gll	b-irq'-an=da	rursːi-j	ʡaˁħ	meq \\
			\textsc{n}-do.\textsc{ipfv}-\textsc{ptcp}=1	girl-\textsc{dat}	good	wedding\\
		\glt	\sqt{(I) will make a big wedding for my daughter.}
		
		\ex	\label{ex:listening carefully to the sounds of the frogs}
		\gll	gu-lik'-an	ʡaˁħ-le	ʡaˁt'-n-a-la	t'ama-j\\
			\textsc{down}-listen.\textsc{ipfv}-\textsc{ptcp}	good-\textsc{advz}	frog-\textsc{pl}-\textsc{obl}-\textsc{gen}	sound-\textsc{dat}\\
		\glt	\sqt{listening carefully to the sounds of the frogs}
		
		\ex	\label{ex:Murad feels very sorry for Madina}
		\gll	Murad-li-j	χʷal-le		urk'ec'i	či-d-ulq-u	Madina-j \\
			Murad-\textsc{obl}-\textsc{dat}		big-\textsc{advz}	pity	\textsc{spr}-\textsc{npl}-direct.\textsc{ipfv}-\textsc{prs}	Madina-\textsc{dat}\\
		\glt	\sqt{Murad feels very sorry for Madina.} (E)
		
		\ex	\label{ex:Now the boy believed it (=the mouse) and ran away}
		\gll	na	il-i-j	w-iχči	ag-ur-re	durħuˁ	sa-r-∅-uq-un ca-w\\
			now	that-\textsc{obl}-\textsc{dat}	\textsc{m}-believe	go.\textsc{pfv}-\textsc{pret}-\textsc{cvb}	boy	\textsc{in.front}-\textsc{abl}-\textsc{m}-go.\textsc{pfv}-\textsc{pret} \textsc{cop-m}\\
		\glt	\sqt{Now the boy believed it (= the mouse) and ran away.}
		
		\ex	\label{ex:The wasps flew after the boy and the puppy}
		\gll	žergʷ-ne	hitːi	d-uq-un-ne ca-d	durħ-a-j=ra	kac'i-j=ra \\
			wasp-\textsc{pl} after \textsc{npl}-go.\textsc{pfv}-\textsc{pret}-\textsc{cvb} \textsc{cop-npl}	boy-\textsc{obl}.\textsc{pl}-\textsc{dat}=\textsc{add}	puppy-\textsc{dat}=\textsc{add}\\
		\glt	\sqt{The wasps flew after the boy and the puppy}
	
		\ex	\label{ex:This also (=the person on a picture) one resembles a woman}
		\gll	iž=ra	xunul-li-j	miši-l	ca-r	hel\\
			this=\textsc{add}	woman-\textsc{obl}-\textsc{dat}	similar-\textsc{advz}	\textsc{cop-f}	that\\
		\glt	\sqt{This one also (= the person on a picture) is similar to a woman.}
	\end{exe}

	\item	expression of cause \refex{ex:Because of the benevolence of Kak Hussein this (man) also remained}, \refex{ex:This is exactly his deed, he says, this happened because of him, because of his mistake}, e.g. in the adverb \textit{hel-i-j} (\tsc{dem-obl-dat}) \sqt{therefore}:
	%
	\begin{exe}
		\ex	\label{ex:Because of the benevolence of Kak Hussein this (man) also remained}
		\gll	hel	qːaq	ħuˁsen-na	ʡaˁħ-dexː-li-j	hel=ra	kelg-un-il=de \\
			that	Kak	Hussein-\textsc{gen} good-\textsc{nmlz}-\textsc{obl}-\textsc{dat} that=\textsc{add} remain.\textsc{pfv}-\textsc{pret}-\textsc{ref}=\textsc{pst}\\
		\glt	\sqt{Because of the benevolence of Kak Hussein that (man) also remained (alive).}
		
		\ex	\label{ex:This is exactly his deed, he says, this happened because of him, because of his mistake}
		\gll	točno	het-i-la	ʡaˁči,	∅-ik'-ul ca-w,	het-i-cːe-r	ka-b-ič-ib	χat'a-li-j\\
			exactly that-\textsc{obl}-\textsc{gen}	work	\textsc{m}-say.\textsc{ipfv}-\textsc{icvb} \textsc{cop-m} that-\textsc{obl-in}-\textsc{abl} \textsc{down-n}-occur.\textsc{pfv}-\textsc{pret}	mistake-\textsc{obl}-\textsc{dat}\\
		\glt	\sqt{This is exactly his deed, he says, this happened because of him, because of his mistake.}
	\end{exe}

	\item	temporal duration (\sqt{for}/\sqt{in}) and points of time:
	%
	\begin{exe}
		\ex	\label{ex:(The price) for a ward for every day is 2000 (rubles)}
		\gll	palata-li-j ca-b har	bari-j	k'ʷel	azir\\
			ward-\textsc{obl}-\textsc{dat}	\textsc{cop-n}	every	day-\textsc{dat}	two	thousand\\
		\glt	\sqt{(The price) for a ward is 2000 (rubles) per day.}
		
			\ex	\label{ex:Every year he bought two hats}
			\gll	har	dusːi-j		k'ʷel	q'ap'a	isː-i \\
				every	year.\textsc{obl}-\textsc{dat}	two	hat	buy.\textsc{ipfv}-\textsc{hab}.\textsc{pst}\\
			\glt	\sqt{Every year he (usually) bought two hats.}

			\ex	\label{ex:I will do the homework in one hour}
			\gll	du-l 	b-irq'-id dars ca saˁʡaˁt-li-j \\
				1\textsc{sg}-\textsc{erg}	\textsc{n}-do.\textsc{ipfv}-1.\textsc{prs}	homework	one	hour-\textsc{obl}-\textsc{dat}\\
			\glt	\sqt{I will do the homework in one hour.} (E)
	

		\ex	\label{ex:This person needs to be there in the morning at 8 o'clock}
		\gll	ixtːu-b	čːaˁʡaˁl-li-j	sːaˁʡaˁt	kːaʔal-li-j	w-iχʷ-ij	ʡaˁʁuni-l	ca-w	hel	admi \\
			there-\textsc{n}	morning-\textsc{obl}-\textsc{dat} hour	eight-\textsc{obl}-\textsc{dat}	\textsc{m}-be.\textsc{pfv}-\textsc{inf} needed-\textsc{advz}	\textsc{cop-m} that person\\
		\glt	\sqt{That person needs to be there in the morning at 8 o'clock.}
	\end{exe}

	\item	prices:
	%
	\begin{exe}
		\ex	\label{ex:(He) sold the donkey for three rubles}
		\gll	qːuruš-li-j b-ic-ib ca-b hel	amχa \\
			ruble-\textsc{obl}-\textsc{dat}	\textsc{n}-sell.\textsc{pfv}-\textsc{pret} \textsc{cop-n} that donkey\\
		\glt	\sqt{(He) sold the donkey for three rubles.}
	\end{exe}

	\item	spatial functions (in elicitation, but not common in natural texts):
	%
	\begin{exe}
		\ex	\label{ex:The bird is sitting on the tree}
		\gll	čaˁkʷa 	kːalkːi-le /	kːalkːi-j či-ka-b-iž-ib ca-b \\
			bird	tree-\textsc{loc}	/	tree-\textsc{dat}	\textsc{spr-down}-\textsc{n}-be.\textsc{pfv}-\textsc{pret} \textsc{cop-n}\\
		\glt	\sqt{The bird sat down on the tree.}
	\end{exe}
\end{enumerate}


% --------------------------------------------------------------------------------------------------------------------------------------------------------------------------------------------------------------------- %

\subsection{Functions of semantic cases} 
\label{ssec:Functions of semantic cases}

Apart from the \isi{comitative} all semantic cases have a basic spatial meaning. \reftab{tab:Spatial cases} provides the core \is{spatial case}spatial cases. As in most other Caucasian languages the \is{spatial case}spatial cases are formally and functionally rather transparent and organized along two dimensions: location and direction (movement). There are six suffixes that express different ways of locating an item with respect to a reference point: 

\begin{itemize}
	\item \textsc{loc}-series -\textit{le}\slash\textit{ja}: in, on a reference point (\refsec{sssec:spr-lative -le/-ja/-a, spr-essive -le-b/-ja-b/-a-b and spr-ablative -le-r/-ja-r/-a-r})
	\item \textsc{ad}-series -\textit{šːu}: at, by, close to a (mostly) animate reference point (\refsec{sssec:ad-lative -šːu, ad-essive -šːu-b, and ad-ablative -šːu-r})
	\item \textsc{in}-series -\textit{cːe}: in, on, at a reference point (\refsec{sssec:in-lative -cːe, in-essive -cːe-b, and in-ablative -cːe-r})
	\item \textsc{sub}-series -\textit{gu}: under a reference point (\refsec{sssec:sub-lative -gu, sub-essive -gu-b, and sub-ablative -gu-r})
	\item \textsc{ante}-series -\textit{sa}: in front of a reference point (\refsec{sssec:ante-lative -sa, ante-essive -sa-b, and ante-ablative -sa-r})
	\item \textsc{post}-series -\textit{hara}: behind a reference point (\refsec{sssec:post-lative -hara, postessive -hara-b, and post-ablative -hara-r})
\end{itemize}


There is a semantic distinction between animate reference points (normally used together with the \textsc{ad}-series) and inanimate reference points (usually marked with the \textsc{loc}-series). Furthermore, not all conceivable spatial constellations are covered by the location suffixes. For instance, meanings such as \sqt{near} and \sqt{above} can only expressed by means of postpositions (\refcpt{cpt:postpositions}).

Furthermore, there is a three-way distinction in terms of direction (movement):

\begin{itemize}
	\item lative (zero marked): direction to a \isi{goal}
	\item essive (marked by means of the \isi{gender}/\isi{number} agreement suffixes): stative location at a reference point
	\item \isi{ablative} (-\textit{r} or -\textit{rka}): movement away from a reference point or movement through or along a reference point
\end{itemize}

The direction markers can be directly suffixed to \is{spatial postposition}spatial postpositions/adverbs and some other nominals that have inherent locational meaning (e.g. place names). With all other items, direction markers only occur in combination with the location markers. In addition to the core \is{spatial case}spatial cases given in \reftab{tab:Spatial cases} there is one minor \isi{spatial case} whose use is somewhat restricted, the directional \textit{-\textsc{gm}-a} (\refsec{sssec:Directional -gm-a}). 

%but very occasionally with a few verbs of motion or directed motion it can also express location \refex{???}. However, for the locational meaning normally the essive is used. 
%
%\begin{exe}
	%\ex	\label{ex:He kept / took them in his hands}
	%\gll	hex-tːi	nuˁq-b-a-cːe	kʷi	d-uc-ib	hek'-i-l \\
	%	this-\textsc{pl}	hand-\textsc{pl}-\textsc{obl}-\textsc{in}	in.the.hands	\textsc{npl}-keep.\textsc{pfv}-\textsc{pret} this-\textsc{obl}-\textsc{erg}\\
	%\glt	\sqt{He kept\slash took them in his hands.}
%\end{exe}
%


The lative is formally unmarked and expresses direction and movement to a reference point. The essive is formally marked through \isi{gender} agreement suffixes that agree with the item whose location is expressed. This is normally the \isi{absolutive} argument, i.e. agreement of the essive adjunct confirms to the general rules of \isi{gender} agreement. The \isi{absolutive} argument does not need to be overtly expressed in order to control the agreement, but can be left implicit \refex{ex:when he came come, he pestered his wife}. 

\begin{exe}
	\ex	\label{ex:when he came come, he pestered his wife}
	\gll	qili	sa-∅-jʁ-ib=qːella	xːunul-la	bek'-le-w	či-w	w-alkː-un-ne	k-erg-ul=de\\
		home	\textsc{hither-m}-come.\textsc{pfv}-\textsc{pret}=when	woman-\textsc{gen} head-\textsc{loc}-\textsc{m} on-\textsc{m} \textsc{m-}importune-\textsc{pret}-\textsc{cvb} \textsc{down}-sit.\textsc{ipfv}-\textsc{icvb}=\textsc{pst}\\
	\glt	\sqt{(After drinking alcohol the husband,) when he came come, he pestered his wife.} (lit. \sqt{sat on the wife's head})
\end{exe} 

However, as example \refex{ex:In the sovkhoz I worked for 40 years} shows, it is also possible for other prominent arguments to control \isi{gender} agreement on an essive adjunct. In this example, it is the implicit \isi{agent}, the first person pronoun in the \isi{ergative} case referring to the masculine speaker that controls agreement on the noun in the \textsc{in}-essive (see \refsec{ssec:Gender agreement with arguments in other than the absolutive case} for more details). 

\begin{exe}
	\ex	\label{ex:In the sovkhoz I worked for 40 years}
	\gll	hež	sawχuz-li-cːe-w	aʁʷ-c'al	dus	ʡaˁči	b-arq'-ib=da \\
		this	sovkhoz-\textsc{obl}-\textsc{in}-\textsc{m}	four-\textsc{ten}	year	work	\textsc{n}-do.\textsc{pfv}-\textsc{pret}=1\\
	\glt	\sqt{In the sovkhoz I (masc.) worked for 40 years.}
\end{exe} 

The \isi{ablative} has two meanings, \sqt{from} and \sqt{through}\slash\sqt{along}. It is most frequently expressed by the suffix -\textit{r}, but -\textit{rka} is also possible with apparently no difference in meaning. The latter suffix is morphologically complex consisting of -\textit{r} and -\textit{ka} and less frequently used than simply -\textit{r}. Diachronically, -\textit{ka} might go back to an elevation marker \textit{ka} \sqt{down} (and thus be related to the elevation \isi{preverb} \textit{ka}- \sqt{down}, see \refsec{ssec:Deixis/gravitation preverbs}).\footnote{Tanti Dargwa, a relatively closely related variety, has four orientation markers, among them -\textit{ka}, that are only suffixed to nominals inflected for the lative or the \isi{ablative}. (see \citealp[69\tnd70]{Sumbatova.Lander2014} and \citealp{ForkerLTSanzhi}).} 

The \is{spatial case}spatial cases are functionally and partially also formally close to \is{spatial adverb}spatial adverbs (\refsec{ssec:SpatialAdverbsDerivedFromPostpositions}) and postpositions (\refsec{sec:Spatialpostpositions}) and can be used alone or together with them. Furthermore, there are semantic and formal resemblances with spatial \is{preverb}preverbs (\refsec{sec:Preverbs}). Spatial cases are also used for non-spatial purposes, e.g. as part of valency frames, in certain constructions such as comparison or to express non-canonical \isi{agent} constructions. In the following, spatial and non-spatial functions will be described in more detail. Microtoponyms, some other place names (\refsec{cpt:morph-placenames}) and \is{spatial adverb}spatial adverbs (\refsec{sec:spatialadverb}) diverge from ordinary common \isi{nouns} when inflected for \is{spatial case}spatial cases. In a nutshell, they have an inherent locational meaning and are only inflected for direction (lative, essive, and \isi{ablative}).


% - - - - - - - - - - - - - - - - - - - - - - - - - - - - - - - - - - - - - - - - - - - - - - - - - - - - - - - - - - - - - - - - - - - - - - - - - - - - - - - - - - - - - - - - - - - - - - - - - - - - - - - - - - - - - - - - - - - - - - - - - - %

\subsubsection{Comitative}
\label{sssec:Comitative}

The suffix of the \isi{comitative} is -\textit{cːella}. Diachronically it is probably complex consisting of the \textsc{in}-lative -\textit{cːe} and the \isi{genitive} -\textit{(l)la}. It is used with nominals having animate referents in the \isi{comitative} function \refex{ex:we (were) going by foot there with my friends} as well as with inanimate \isi{nouns} in the instrumental function \refex{ex:striking with the crock}, \refex{ex:When we go back, we go by car} and to express manner \refex{ex:when he came here with sorrow} or experiencers \refex{ex:as if it happened with you} as well as in other contexts roughly corresponding to the use of English \textit{with} \refex{ex:Inside the pot there are 2, 3, 4, things like this with holes on the lower side}.

\begin{exe}
	\ex	\label{ex:we (were) going by foot there with my friends} 
	\gll	xural	d-ax-ul	hej-ka=či-d-a	di-la	juldašː-a-cːella\ldots\\
		by.foot 1/2.\textsc{pl}-go.\textsc{ipfv}-\textsc{icvb} this-\textsc{down}=on-1/2.\textsc{pl}-\textsc{dir} 1\textsc{sg}-\textsc{gen} friend-\textsc{obl}.\textsc{pl}-\textsc{comit}\\
	\glt	\sqt{We (were) going by foot there with my friends\ldots}
	
	\ex	\label{ex:striking with the crock}
	\gll	paˁq ik'-ul q'isːa-l-cːella\\
		strike say\textsc{.ipfv.m-icvb} crock-\textsc{obl-comit}\\
	\glt	\sqt{striking with the crock}
	
	\ex	\label{ex:When we go back, we go by car}
	\gll	saˁ-q'-aˁn	zamana=q'ar	nušːa	mašin-ni-cːella	saˁ-q'-un=da \\
		\textsc{hither}-go-\textsc{ptcp}	time=\textsc{mod}	1\textsc{pl}	car-\textsc{obl}-\textsc{comit}	\textsc{hither}-go-\textsc{icvb}=1\\
	\glt	\sqt{When we go back, we go by car.}
	
	\ex	\label{ex:when he came here with sorrow}
	\gll	hiχ	bala-cːella	ha-∅-jʁ-ib-le	heštːu\\
		\textsc{dem.down}	misfortune-\textsc{comit}	\textsc{up}-\textsc{m}-come.\textsc{pfv}-\textsc{pret}-\textsc{cvb}	here\\
	\glt	\sqt{when he came here with sorrows ...}
	
	\ex	\label{ex:as if it happened with you}
	\gll	nu,	a-cːella	ag-ur-il	daˁʡle \\
		well	2\textsc{sg}-\textsc{comit}	go.\textsc{pfv}-\textsc{pret}-\textsc{ref} as\\
	\glt	\sqt{as if it happened to you}
		
	\ex	\label{ex:Inside the pot there are 2, 3, 4, things like this with holes on the lower side}
	\gll	ħaˁšukː-a-d	d-i-d	k'ʷel	ʡaˁbal	aʁʷal	ʁunab-te	χe-d	xari-gu-d	ʡaˁm-r-a-cːella\\
		pot-\textsc{loc}-\textsc{npl}	\textsc{npl}-in-\textsc{npl} two	three	four	\textsc{eq}-\textsc{dd.pl} 	exist.\textsc{down-npl}	down-\textsc{sub}-\textsc{npl	} hole-\textsc{pl}-\textsc{obl}-\textsc{comit}\\
	\glt	\sqt{Inside the pot there are 2, 3, 4 things like this with holes on the lower side.}
\end{exe}


% - - - - - - - - - - - - - - - - - - - - - - - - - - - - - - - - - - - - - - - - - - - - - - - - - - - - - - - - - - - - - - - - - - - - - - - - - - - - - - - - - - - - - - - - - - - - - - - - - - - - - - - - - - - - - - - - - - - - - - - - - - %

\subsubsection{\textsc{loc}-lative -\textit{le}/-\textit{ja}/-\textit{a}, \textsc{loc}-essive -\textit{le}-\textit{b}/-\textit{ja}-\textit{b}/-\textit{a}-\textit{b} and \textsc{loc}-ablative -\textit{le}-\textit{r}/-\textit{ja}-\textit{r}/-\textit{a}-\textit{\texttt{r}}}
\label{sssec:spr-lative -le/-ja/-a, spr-essive -le-b/-ja-b/-a-b and spr-ablative -le-r/-ja-r/-a-r} 

The \textsc{loc}-series is together with the \textsc{in}-series (\refsec{sssec:in-lative -cːe, in-essive -cːe-b, and in-ablative -cːe-r}) by far the most frequently used series among the \is{spatial case}spatial cases. It can be broadly described as a general location marker that expresses the most common location of a figure with respect to the ground. Both its formal make-up and its functions are rather complex and deserve a future study. In this section, I can only provide a sketch of its formation and its meaning. Formally, the \textsc{loc}-series is very heterogeneous (in contrast to all other \isi{spatial case} suffixes). It is either expressed by dedicated suffixes \textit{-a} and \textit{-ja} or by a change of the final vowel \textit{i > e}. The vowel change almost exclusively affects \isi{nouns} that take \textit{-li} as their \isi{ergative} suffix and \isi{oblique stem} marker, and thus we get \textit{-li} > \textit{-le}. However, not all \isi{nouns} that have the \isi{ergative}\slash oblique suffix \textit{-li} undergo the vowel change, but some of those \isi{nouns} take a suffix as \textsc{loc}-series marker. The occurrences of the allomorphs (suffixes or vowel change) can only partially be predicted. I will first give the usage constraints for each marker, describe its functions by means of examples and in the end compare it to similar markers from other Dargwa varieties. In addition to the relatively regular ways of forming the \textsc{loc}-series by means of the just listed allomorphs, there is a special class of \isi{nouns} that has inherent locational meaning and can be said to semantically express the \textsc{loc}-series as well although synchronically no case suffix can be identified. This class consists of native place names (mostly names of villages and microtoponyms, \refcpt{cpt:morph-placenames}).

The default way of forming the \textsc{loc}-series for \isi{nouns} in the singular is the vowel change of the \isi{ergative}\slash oblique suffix \textit{-li} > \textit{-le}. The combinations \textit{*-li-le} \linebreak (\textsc{-obl-loc}) or \textit{*-l-le} (with a deletion of the vowel of the  \isi{ergative}\slash oblique suffix) are ungrammatical. The suffix \textit{-le} is used after \isi{consonants} and vowels. There is one noun, \textit{neqːi} \sqt{cave}, which has \textit{neqːe} as the \textsc{loc}-lative (in addition to the regular form \tit{neqːi-le}), i.e. we have again the vowel change \textit{i > e}. 

The suffix -\textit{ja} is only used after the vowels \textit{a}, \textit{i} and \textit{u}. It occurs with most \isi{nouns} in the plural, personal and \is{demonstrative pronoun}demonstrative pronouns, plural \is{reflexive pronoun}reflexive pronouns as well as with very few other \isi{nouns} in the singular, e.g. \textit{qu-ja} (field-\textsc{loc}), \textit{aba-ja} (mother-\textsc{loc}). 

The suffix \textit{-a} is used with a \isi{number} of \isi{nouns} of which at least some make use of \textit{-li} as \isi{ergative}\slash oblique suffix. Examples of such \isi{nouns} are \textit{ħaˁšak} \sqt{pot} \refex{ex:Inside the pot there are 2, 3, 4, things like this with holes on the lower side} (\isi{comitative} \textit{ħaˁšak-li-cːella}), \textit{mistːik'} \sqt{mosque}, \textit{qːatːa} \sqt{canyon}, \textit{musːa} \sqt{place}, \textit{daˁrqʷ} \sqt{barn, cattle-shed}, \textit{šːi} \sqt{village}. With a few of these \isi{nouns} the suffix is simply added to the noun, e.g. \textit{mistːik'-a} \sqt{to the mosque} and \textit{ħaˁšuk-a} \sqt{into the pot}. With those \isi{nouns} that have stem-final \textit{a}, the locative case differs from the base stem in the pitch accent that switches to the final vowel, e.g. \textit{musːá} \sqt{place\slash to the place} \refex{ex:In once place, there are}.\footnote{This refers only to the nominals that take the suffix \textit{-a}. It is not the case that every nominal ending in \textit{a} takes the suffix \textit{-a}, e.g. \textit{ʡaˁrmija-le} `in the army'.} Thus, we can assume that \textit{a + a > aː > á}. The noun \sqt{village} has the special locative form \tit{šːa} \sqt{(in)to the village}. This form is not the \isi{oblique stem} because it does not serve as the base form for the formation of other cases. Furthermore, the more common way of saying \sqt{in the village} is to use the \textsc{in}-essive \refex{ex:in our village of Sanzhi}. The suffix \textit{-a} is also attested for some plural \isi{nouns} that make use of -\textit{ne} as the plural suffix, e.g. \textit{mus-n-a} (place-\textsc{pl}-\textsc{obl}.\textsc{loc}) \sqt{to the places}, \textit{kis-n-a-b} (pocket-\textsc{pl}-\textsc{obl}.\textsc{loc}-\textsc{n}) \sqt{in the pockets}, \tit{buruš-n-a-r} (mattress\tsc{-pl-obl.loc-f}) \sqt{on the mattresses}. These examples can perhaps be analyzed as undergoing a vowel change \textit{e > a} for the formation of the \textsc{loc}-series.

The meaning of the \textsc{loc}-series is rather broad. It has a basic general spatial and directional meaning indicating movement to a \isi{goal}, static location at a reference point and movement away from a reference point. Usually the location is the most typical location. The reference points can be places \refex{ex:In once place, there are}, place names, villages, cities, buildings, institutions \refex{ex:Is he in prison or in the army}, body parts, vehicles and other means of transport, containers \refex{ex:putting (the pears) into a basket}, and so on. The \textsc{loc-}series translates into English as \sqt{to, in, on}. Its meaning includes vertical location, e.g. on a wall \refex{ex:pic on wall_1}, and also location inside a reference point \refex{ex:putting (the pears) into a basket}, \refex{ex:We pour (the xinkal) into a pot with boiling (water) and cook it}. Note that instead of the \textsc{loc}-series it is possible to use the  \textsc{in}-series in examples \xxref{ex:pic on wall_1}{ex:We pour (the xinkal) into a pot with boiling (water) and cook it} with no difference in meaning (\refsec{sssec:in-lative -cːe, in-essive -cːe-b, and in-ablative -cːe-r}). However, it seems that with certain locations there are conventionalized uses of the one or the other suffix. For instance, with names of settlements the \textsc{loc}-series occurs \refex{ex:He came back from Makhachkala and had the operation}, whereas with the noun \textit{šahar} ‎‎\sqt{town} the \textsc{in}-series form \textit{šahar-ri-cːe} is used. With \textit{ħaˁšak} \sqt{pot} the \textsc{loc}-series is clearly preferred \refex{ex:We pour (the xinkal) into a pot with boiling (water) and cook it}, but the \textsc{loc}-series form \textit{ħaˁšak-li-cːe} is also attested \refex{You pour (them) into a pot with boiling (water).}. Further research is needed in order to arrive at a more detailed picture about the semantic similarities and differences between these two \isi{spatial case} series.


\begin{exe}
	\ex	\label{ex:Is he in prison or in the army}
	\gll	tusnaq-le-w=uw	iž	ʡaˁrmija-le-w=uw? \\
		prison-\textsc{loc}-\textsc{m}=\textsc{q}	this		army-\textsc{loc}-\textsc{m}=\textsc{q}\\
	\glt	\sqt{Is he in prison or in the army?}

	\ex	\label{ex:In once place, there are}
	\gll	ca	musːa-d	k'e-d,	∅-ik'-ul ca-w,	kːalk-me\\
		one	place.\textsc{loc}-\textsc{npl}	exist.\textsc{up-npl}	\textsc{m}-say.\textsc{ipfv}-\textsc{icvb} \textsc{cop-m}	tree\tsc{-pl}\\
	\glt	‎‎\sqt{In one place, there are, he says, trees.}

	\ex	\label{ex:pic on wall_1}
	\gll	sːurrat baˁʡ-le-b sa-r-h-aq-ib ca-b\\
picture	wall-\textsc{loc-n}	\textsc{ante-abl-up}-hang.\textsc{pfv-pret}	\textsc{cop-n} \\
	\glt	\sqt{‎‎The picture is hanging on the wall.} (E)
	
	\ex	\label{ex:putting (the pears) into a basket}
	\gll	t'unnuˁq-le	d-i-ka-d-irxː-ul	\ldots\\
		basket-\textsc{loc} \textsc{npl-in-down}-\textsc{npl}-put.\textsc{ipfv}-\textsc{icvb}\\
	\glt	\sqt{putting (the pears) into a basket ...}

	\ex	\label{ex:We pour (the xinkal) into a pot with boiling (water) and cook it}
	\gll	d-i-h-ax-ub-le	rurčː-an	ħaˁšukː-a	luxː-id\\
		\textsc{npl-in-up}-pour.\textsc{pfv}-\textsc{pret}-\textsc{cvb}	boil-\textsc{ptcp}	pot-\textsc{loc}		cook.\textsc{ipfv}-1.\textsc{prs}\\
	\glt	\sqt{We pour (the khinkal) into a pot with boiling (water) and cook it.}
\end{exe}

When used together with the postposition \textit{či-b}, the \textsc{loc}-series can also express the meaning \sqt{above} (see \refsec{ssec:postposition ci} on postpositions for an example).

\hspace*{-0.22562pt}When the locative suffix is followed by the \isi{ablative} case, the meaning is \sqt{from, through} \refex{ex:He came back from Makhachkala and had the operation}, \refex{ex:We take up (the trousers) and go across the river}. 


\begin{exe}
	\ex	\label{ex:He came back from Makhachkala and had the operation}
	\gll	Maˁħaˁčqːala-le-r	sa-∅-jʁ-ib-le,	hel	apiracːija	b-arq'-ib ca-b\\
		Makhachkala-\textsc{loc}-\textsc{abl}	\textsc{hither-m}-come.\textsc{pfv}-\textsc{pret}-\textsc{cvb} that operation	\textsc{n}-do.\textsc{pfv}-\textsc{pret} \textsc{cop-n}\\
	\glt	\sqt{He came back from Makhachkala and had the operation.}

	\ex	\label{ex:We take up (the trousers) and go across the river}
	\gll	aq	d-arq'-ib-le,	heštːi	či-r-d-ax-ud	erk'ʷ-le-r\\
		high	\textsc{npl}-do.\textsc{pfv}-\textsc{pret}-\textsc{cvb}	these	\textsc{spr}-\textsc{abl}-1/2\textsc{pl}-go.\textsc{ipfv}-1.\textsc{prs}	river-\textsc{loc}-\textsc{abl}\\
	\glt	\sqt{We roll up (the trousers) and go across the river.}
\end{exe}


In the basic meaning, the \textsc{loc}-series is only used with inanimate \isi{nouns} and can be opposed to the \textsc{ad}-series (\refsec{sssec:ad-lative -šːu, ad-essive -šːu-b, and ad-ablative -šːu-r}), which is used with animate \isi{nouns} to express the same general meaning. Thus, compare \refex{ex:I did not go to school at all} and \refex{ex:They go to the qadis, to the mullahs}:

\begin{exe}
	\ex	\label{ex:I did not go to school at all}
	\gll	du	wabše	uškul-le	ag-ur-il	akːʷa-di \\
		1\textsc{sg}	at.all	school-\textsc{loc}	go.\textsc{pfv}-\textsc{pret}-\textsc{ref}	\textsc{cop.neg}-1\\
	\glt	\sqt{I did not go to school at all.}
\end{exe}

In \refex{ex:First begin with these who drink}, the \isi{participle} bearing the \textsc{loc-}\isi{ablative} refers to a picture showing people who drink and the speaker is asked to begin his story with this picture.

\begin{exe}
	\ex	\label{ex:First begin with these who drink}
	\gll	hel-tːi	b-učː-an-t-a-ja-rka	w-aʔ-∅-ač'-e	caj-na! \\
		that-\textsc{pl}	\textsc{n}-drink.\textsc{ipfv}-\textsc{ptcp}-\textsc{pl}-\textsc{obl}-\textsc{loc}-\textsc{abl}	\textsc{m}-begin-\textsc{m}-come.\textsc{pfv}-\textsc{imp} one-\textsc{time}\\
	\glt	\sqt{First begin with these who drink!} (said to a man)
\end{exe}

The \textsc{loc}-essive can also be used for the expression of metaphorical location and direction, e.g. \textit{di-la} \textit{ʡuˁnru-le-b} (\tsc{1sg-gen} life\tsc{-loc-n}) \sqt{in my life}, and for a \isi{number} of constructions denoting feelings and emotions that are located in body parts or in persons \refex{ex:when I remember you,}, \refex{ex:forgive me}.

\begin{exe}
	\ex	\label{ex:when I remember you,}
	\gll	u	urk'i-le	r-ak'-utːe, \ldots\\
		2\textsc{sg}	heart-\textsc{loc}	\textsc{f}-move.\textsc{pfv}-\textsc{cond}.2\textsc{sg}\\
	\glt	\sqt{if I remember you (fem.), \ldots}

	\ex	\label{ex:forgive me}
	\gll	di-ja-r	či-r-ka-d-erχːʷ-aja! \\
		1\textsc{sg}-\textsc{loc}-\textsc{abl}	\textsc{spr}-\textsc{abl-down}-\textsc{npl}-apologize.\textsc{pfv}-\textsc{imp}.\textsc{pl}\\
	\glt	\sqt{Forgive me!}
\end{exe}

The \textsc{loc-}series can also be used with animate \isi{nouns}. In this case the spatial meaning is \sqt{on, onto} and thus more specific than when used with inanimate \isi{nouns}:

\begin{exe}
	\ex	\label{ex:these wounds on the mother}
	\gll	heχ-tːi	daˁquˁ-pːe	aba-ja-d	či-d \\
		\textsc{dem.down}-\textsc{pl}	wound-\textsc{pl}	mother-\textsc{loc}-\textsc{npl}	on-\textsc{npl}\\
	\glt	\sqt{the wounds on the mother}

	\ex	\label{ex:(he was) riding on a horse and shaking}
	\gll	w-ax-ul	urči-le-w	tːamqːar	sa-∅-jk'-ul	\ldots\\
		\textsc{m}-go.\textsc{ipfv}-\textsc{icvb}	horse-\textsc{loc}-\textsc{m}	stagger	\textsc{hither}-\textsc{m}-move.\textsc{ipfv}-\textsc{icvb}\\
	\glt	\sqt{(he was) riding on a horse and shaking \ldots}
\end{exe}

Occasionally, one can find expressions for points in time marked by the \textsc{loc}-essive, e.g. \textit{sːaˁʡaˁt} \textit{kːaʔal-le-b} (hour eight\tsc{-loc-n}) \sqt{at eight o'clock}. However, other cases such as the \isi{dative} are more common in this function. The \textsc{loc}-\isi{ablative} is regularly used in phrases with the meaning \sqt{after (time)}, e.g. \textit{k'ʷel ʡaˁbal minut'-le-r} (two three minute-\textsc{loc-abl}) \sqt{after two, three minutes}.

Finally, the \textsc{loc}-\isi{ablative} occurs in \is{comparative construction}comparative constructions \refex{ex:Yours (i.e. your story) is better than mine} marking the \isi{standard of comparison} (\refsec{sec:Comparative constructions}):

\begin{exe}
	\ex	\label{ex:Yours (i.e. your story) is better than mine}
	\gll	ala	qːuʁa-l	arg-ul ca-b	di-la-ja-rka \\
		2\textsc{sg}.\textsc{gen}	beautiful-\textsc{advz}	go.\textsc{ipfv}-\textsc{icvb} \textsc{cop-hpl}	1\textsc{sg}-\textsc{gen}-\textsc{loc}-\textsc{abl}\\
	\glt	\sqt{Yours (i.e. your story) is better than mine.}
\end{exe}


Synchronically, all markers are allomorphs of the same case. For instance, in \textit{tusnaq-le-b} (prison-\textsc{loc-hpl}) \sqt{in the prison} vs. \textit{tusnaqːa-ja-b} (prison.\textsc{obl.pl-loc-hpl}) \sqt{in the prisons} the only difference is the \isi{number} of the noun to which the case suffix is attached. Furthermore, it is ungrammatical to have both the vowel change and a suffix \textit{-a} or \textit{-ja} with one and the same nominal, e.g. *\textit{ša-ja} (village.\textsc{loc-loc}). Normally each nominal can apply only one operation to form the \textsc{loc}-series, but there are a few examples that prove that there is some variation, e.g. \tit{neqːe} vs. \tit{neqːi-le} `into the cave' (cave.\textsc{loc} vs. cave-\textsc{loc}). As the above description has shown, the distribution of the allomorphs used is at least in part lexically determined and needs further study.
 
Diachronically, all markers go back to formally and functionally unrelated markers. This becomes clear when we compare Sanzhi to other Dargwa varieties. In her comparative paper on \is{spatial case}spatial cases in Dargwa, \citet{vandenBerg2003c} provides \isi{spatial case} paradigms of ten Dargwa varieties from north to south. For the analysis of the Sanzhi locative marker three groups of suffixes are relevant:

\begin{itemize}
	\item some varieties have a suffix \textit{-j} or \textit{-ja} with the meaning \sqt{on}
	\item some varieties have \textit{-n(a)} or \textit{-le} (or variants thereof) with the meaning \sqt{in a hollow space}
	\item some varieties have \textit{-n(a)}, \textit{-la}, or \textit{-le} (or variants thereof) with a general locative meaning\footnote{Van den Berg (\citeyear{vandenBerg2003c}) further hypothesizes that there is a connection between the markers for \sqt{in a hollow space} and the general locative markers.} 
\end{itemize}

The Sanzhi locative case marker seems to be a mixture of all three groups. Formally its exponents correspond to suffixes from all three groups, and functionally the marker unifies the three different meanings. Other Dargwa varieties show a similar picture. For instance, Tanti Dargwa, another south Dargwa variety, has a \textsc{super}-series expressed with the suffix \textit{-ja} that is part of the regular paradigm of \is{spatial case}spatial cases. In addition, it has a category \sqt{location} (\textit{lokalizatsija}) that is only formed from the direct stem of \isi{nouns} in the singular \citep[66\tnd68]{Sumbatova.Lander2014}. This special form is used when expressing the most natural location of a figure with respect to the ground. As the Sanzhi locative, its formation is very heterogeneous by means of unproductive suffixes (\textit{-na}, \textit{-ni}), vowel change (\textit{i > e}) or a switch of the pitch accent to the word final vowel \textit{a}. The Tanti examples parallel the examples of the Sanzhi locative given above.

Mekegi, a northern Dargwa variety, has a general locative suffix \textit{-le} that is directly added to the nominal stem. This suffix is mentioned in \citeb{vandenBerg2003c}, but unfortunately she does not provide examples or a description of its meaning. In the same paper, van den Berg suggests that this marker has cognates in Akusha Dargwa (\textit{-la}) and Urakhi Dargwa (\textit{-la}) and notes that its precise meaning requires further investigation. The Akusha Dargwa grammar by the same author provides a few examples of the suffix \textit{-la}, which is only added to inanimate \isi{nouns} \citep[24]{vandenBerg2001}. On the same page, the grammar also mentions a couple of \isi{nouns} with irregular locative forms that have shapes analogous to some of the Sanzhi words discussed in this section.



% - - - - - - - - - - - - - - - - - - - - - - - - - - - - - - - - - - - - - - - - - - - - - - - - - - - - - - - - - - - - - - - - - - - - - - - - - - - - - - - - - - - - - - - - - - - - - - - - - - - - - - - - - - - - - - - - - - - - - - - - - - %

\subsubsection{\textsc{ad}-lative -\textit{šːu}, \textsc{ad}-essive \textit{-šːu-b}, and \textsc{ad}-ablative -\textit{šːu-r}}
\label{sssec:ad-lative -šːu, ad-essive -šːu-b, and ad-ablative -šːu-r}

The series of \is{spatial case}spatial cases formed with the suffix -\textit{šːu} denotes movement to a \isi{goal} (the moving item is not further specified for precise location with respect to the \isi{goal}), general location that can be broadly translated with \sqt{at, by, with}, and movement away from a source. The \isi{goal}, location or source, i.e. the noun bearing the \isi{spatial case} suffix, mostly has an animate referent. Thus, the \textsc{loc}-series and the \textsc{ad}-series are in a kind of animacy opposition. However, as \refex{ex:They went to the cave of the bear} and \refex{ex:Then Amirhamza took a rifle} show, inanimate reference points are also allowed.

\begin{exe}
	\ex	\label{ex:He mingled with his drinking friends}
	\gll	na	istikan-na	juldašː-a-šːu	ʁudur	∅-ič-ib ca-w	hel\\
		now	glass-\textsc{gen}	friend-\textsc{obl}.\textsc{pl}-\textsc{ad}	mix	\textsc{m}-occur.\textsc{pfv}-\textsc{pret} \textsc{cop-m}	that\\
	\glt	\sqt{He mingled with his drinking friends.}
	
	\ex	\label{ex:They go to the qadis, to the mullahs}
	\gll	na	iltːi	q'ad-n-a-šːu	mallu-bn-a-šːu	b-ax-ul ca-b \\
		now	these	qadi-\textsc{pl}-\textsc{obl}-\textsc{ad}	mullah-\textsc{pl}-\textsc{obl}-\textsc{ad}	\textsc{hpl}-go.\textsc{ipfv}-\textsc{icvb} \textsc{cop-hpl}\\
	\glt	\sqt{They are going to the qadis, to the mullahs.}
	
	\ex	\label{ex:They went to the cave of the bear}
	\gll	ag-ur-re ca-b	sːika-la	mergʷ-li-šːu \\
		go.\textsc{pfv}-\textsc{pret}-\textsc{cvb} \textsc{cop-hpl}	bear-\textsc{gen}	lair-\textsc{obl}-\textsc{ad}\\
	\glt	\sqt{They went to the cave of the bear.}
	
	\ex	\label{ex:Then Amirhamza took a rifle}
	\gll	c'il	il	Amirħaˁmza	uq-un-ne ca-w	tupang-li-šːu\\
		then	that	Amirhamza	go.\textsc{pfv.m}-\textsc{pret}-\textsc{cvb} \textsc{cop-m} rifle-\textsc{obl}-\textsc{ad}\\
	\glt	\sqt{Then Amirhamza took a rifle.} (lit. \sqt{went to the rifle})
	
	\ex	\label{ex:(We) left the car with the milkmaids}
	\gll	dajark'a-b-a-šːu-b	b-at-ur=da	mašin\\
		milkmaid-\textsc{pl}-\textsc{obl}-\textsc{ad}-\textsc{n}	\textsc{n}-let.\textsc{pfv}-\textsc{pret}=1	car\\
	\glt	\sqt{(We) left the car with the milkmaids.} (in the place where the milkmaids used to work)
	
	\ex	\label{ex:From him (they) brought a permission}
	\gll	hel-i-šːu-rka	k-aqː-ib-le	kaʁar\\
		that-\textsc{obl}-\textsc{ad}-\textsc{abl}	\textsc{down}-carry-\textsc{pret}-\textsc{cvb}	letter\\
	\glt	\sqt{From him (they) brought a permission (lit. \sqt{letter}).}
\end{exe}


% - - - - - - - - - - - - - - - - - - - - - - - - - - - - - - - - - - - - - - - - - - - - - - - - - - - - - - - - - - - - - - - - - - - - - - - - - - - - - - - - - - - - - - - - - - - - - - - - - - - - - - - - - - - - - - - - - - - - - - - - - - %

\subsubsection{\textsc{in}-lative -\textit{cːe}, \textsc{in}-essive -\textit{cːe-b}, and \textsc{in}-ablative -\textit{cːe-r}}
\label{sssec:in-lative -cːe, in-essive -cːe-b, and in-ablative -cːe-r}

The suffix of the \textsc{in}-series is -\textit{cːe}. The locational meaning can be roughly translated as \sqt{in} \refex{In one village lived a big family}, or \sqt{on, at} \refex{Then, on the threshold, he took the flour}, and its directional meaning is \sqt{to} \refex{No person at all should go to the graveyard}.

\begin{exe} 
	\ex	\label{In one village lived a big family}
	\gll	ca	šːi-l-cːe-b	ca	kulpat,	χːula	kulpat	b-už-ib ca-b \\
		one	village-\textsc{obl-in}-\textsc{hpl} one	family	big	family	\textsc{hpl}-stay-\textsc{pret} \textsc{cop-hpl}\\
	\glt	\sqt{In one village lived a big family.}
	
		\ex	\label{Then, on the threshold, he took the flour}
	\gll	c'ili heba, burma-cːe-w, h-asː-ib-le bet'u ...\\
		then then threshold.\textsc{obl-in-m}, \textsc{up}-take.\textsc{pfv-pret-cvb} flour\\
	\glt	\sqt{Then, on the threshold, (he) took the flour, ...}
	
	\ex	\label{No person at all should go to the graveyard}
	\gll	χːuˁrba-cːe	wabše	admi	w-ax-an	akːu=q'al\\
		graveyard-\textsc{in}	at.all	person	\textsc{m}-go.\textsc{ipfv}-\textsc{ptcp}	\textsc{cop.neg}=\textsc{mod}\\
	\glt	\sqt{No person at all should go to the graveyard.}
\end{exe}

Note that in contexts such as \sqt{in a settlement}, \sqt{in a container-like object} or \sqt{on a vertical surface}, it is possible to use the \textsc{loc}-series instead of the \textsc{in}-series with no semantic differences between the two variants. Thus, compare \refex{In one village lived a big family} with \refex{ex:He came back from Makhachkala and had the operation}, \refex{ex:pic on wall_2} with \refex{ex:pic on wall_1}, and \refex{You pour (them) into a pot with boiling (water).} with \refex{ex:We pour (the xinkal) into a pot with boiling (water) and cook it}. 

\begin{exe} 
	\ex	\label{ex:pic on wall_2}
	\gll	sːurrat baˁʡ-li-cːe-b sa-r-h-aq-ib ca-b\\
picture	wall-\textsc{obl-in-n}	\textsc{ante-abl-up}-hang.\textsc{pfv-pret}	\textsc{cop-n} \\
	\glt	\sqt{‎‎The picture is hanging on the wall.} (E)
	
		\ex	\label{Murad knocked at the door.}
	\gll	Murad uncːa-l-cːe q'ut' ik'-ul=de\\
 Murad door-\textsc{obl-in} knock say\textsc{.ipfv.m-icvb=pst}\\
	\glt	\sqt{Murad knocked at the door.} (E)
	
			\ex	\label{You pour (them) into a pot with boiling (water).}
	\gll	rurq-aˁn ħaˁšak-li-cːe d-i-h-erxʷ-itːe \\
boil-\textsc{ptcp}	pot-\textsc{obl-in}	\textsc{npl-in-up}-pour.\textsc{ipfv-2sg} \\
	\glt	\sqt{You pour (them) into a pot with boiling (water).} 
	
\end{exe}

The \textsc{in}-\isi{ablative} does not only translate as \sqt{from} \refex{Ali took away my bike.} and more specifically as \sqt{from within, out of} \refex{At this time that boy came out of the sack}, but is also used to denote \sqt{among, along, through} \refex{from all, from them (you) need to make one story}. Thus, we find it in superlative constructions \refex{Was grandfather the oldest among his brothers} (\refsec{sec:Comparative constructions}):

\begin{exe}
	\ex	\label{Ali took away my bike.}
	\gll ʡaˁli-l		b-erqː-ib			di-cːe-r welesepet\\
	Ali-\textsc{erg} \textsc{n}-take.\textsc{pfv-pret} \textsc{1sg-in-abl} bike\\
	\glt	\sqt{Ali took away my bike.} (E)
	
	\ex	\label{At this time that boy came out of the sack}
	\gll	il	durħuˁ	tːura	ha-w-q-un ca-w	hel	zamana	qːap-li-cːe-r\\
		that	boy	outside	\textsc{up-m}-go.\textsc{pfv}-\textsc{pret} \textsc{cop-m}	that	time	sack-\textsc{obl-in}-\textsc{abl}\\
	\glt	\sqt{At this time that boy came out of the sack.}

	\ex	\label{from all, from them (you) need to make one story}
	\gll	li<b>il-li-cːe-r	hel-tː-a-cːe-r	ca	χːabar	b-irq'-an ca-b \\
		all<\textsc{hpl}>-\textsc{obl-in}-\textsc{abl} that-\textsc{pl}-\textsc{obl-in}-\textsc{abl}	one	story	\textsc{n}-do.\textsc{ipfv}-\textsc{ptcp} \textsc{cop-n}\\
	\glt	\sqt{From all, from them (i.e. from all pictures on the table) (you) need to make one story.}
	
	\ex	\label{Was grandfather the oldest among his brothers}
	\gll	bah	χːula-ce	w-irχ-i=w	χatːaj	ču-la	ucː-b-a-cːe-r? \\
		most	big-\textsc{dd}	\textsc{m}-be.\textsc{ipfv}-\textsc{hab}.\textsc{pst}=\textsc{q} grandfather	\textsc{refl}.\textsc{pl}-\textsc{gen}	brother-\textsc{pl}-\textsc{obl-in}-\textsc{abl}\\
	\glt	\sqt{Was grandfather the oldest among his brothers?}
\end{exe}

The \textsc{in}-lative has also more metaphorical uses when marking the goal-like argument of the verbs \textit{aq}- \sqt{go through} \refex{They dedicated themselves to their life (i.e. they cared for their living} or \textit{b-arχː}- \sqt{be engaged in} \refex{The wife is engaged in her work} and other predicates \refex{Other (people) considered myself as rotten}.

\begin{exe}
	\ex	\label{They dedicated themselves to their life (i.e. they cared for their living}
	\gll	jašaw-li-cːe	qar	aq-ib ca-b	uže	ču-la\\
		being-\textsc{obl-in}	up	go.through.\textsc{pfv}-\textsc{pret} \textsc{cop-hpl}	already	\textsc{refl}.\textsc{pl}-\textsc{gen}\\
	\glt	\sqt{They dedicated themselves to their life (i.e. they cared for their living).}

	\ex	\label{The wife is engaged in her work}
	\gll	xːunul	ha-r-arχː-ib ca-r	cin-na	ʡaˁči-l-cːe \\
		woman	\textsc{up-f}-be.engaged.in-\textsc{pret} \textsc{cop-f}	\textsc{refl}.\textsc{sg}-\textsc{gen}	work-\textsc{obl-in}\\
	\glt	\sqt{The wife is engaged in her work.}

	\ex	\label{Other (people) considered myself as rotten}
	\gll	du	hi-j	r-erʔib	cik'al-li-cːe	ħaˁsib	r-arq'-ib-il=da\\
		1\textsc{sg}	who.\textsc{obl}-\textsc{dat}	\textsc{f}-rotten	thing-\textsc{obl-in}	test	\textsc{f}-do.\textsc{pfv}-\textsc{pret}-\textsc{ref}=1\\
	\glt	\sqt{Other (people) considered myself as rotten.}
\end{exe}

Young speakers use the \textsc{in}-\isi{ablative} alone \refex{From over there a little boy with a bike came} or in combination with the postposition \textit{b-alli} \sqt{together} to express the \isi{comitative}. Older speakers reject such a usage by pointing out that the \isi{comitative} case -\textit{cːella} that can be optionally combined with the same postposition (\refsec{sssec:Comitative}) is the only grammatical variant.

\begin{exe}
	\ex	\label{From over there a little boy with a bike came}
	\gll	hel-ka	sa-∅-jʁ-ib	nik'a	durħuˁ	welisipjed-li-cːe-r\\
		that-\textsc{down}	\textsc{hither-m}-come.\textsc{pfv}-\textsc{pret}	small	boy	bike-\textsc{obl}-\textsc{in}-\textsc{abl}\\
	\glt	\sqt{From over there a little boy with a bike came.}
\end{exe}

The \textsc{in}-essive is used in the temporal expression \sqt{in the year X} \refex{I married in 1960}.

\begin{exe}
	\ex	\label{I married in 1960}
	\gll	xːunul	ka-r-iž-ib=da	urek-c'a-ra-ib	dusːi-cːe-b\\
		woman	\textsc{down}-\textsc{f}-be.\textsc{pfv}-\textsc{pret}=1	six-\textsc{ten-num}-\textsc{ord}	year.\textsc{obl}-\textsc{in}-\textsc{n}\\
	\glt	\sqt{I married in 1960.}
\end{exe}

There are a \isi{number} of non-spatial functions that the \textsc{in}-series fulfills. The \textsc{in}-essive expresses temporarily limited possessors \refex{He has a pen (in his hands}.

\begin{exe}
	\ex	\label{He has a pen (in his hands}
	\gll	hel	ručka=ra	le-b	iž-i-cːe-b \\
		that	pen=\textsc{add}	exist-\textsc{n}	this-\textsc{obl-in}-\textsc{n}\\
	\glt	\sqt{He also has a pen (in his hands).}
\end{exe}

The \textsc{in}-lative denotes temporarily limited recipients \refex{the shoes, the trousers, the shirt is handed over to him}, addressees \refex{He is telling the stories to his family} and causees \refex{Mother made Madina eat porridge} (see \refsec{sec:Reported speech constructions} for \isi{reported speech} constructions and \refsec{sec:Causativization} for causativization).

\begin{exe}
	\ex	 \label{the shoes, the trousers, the shirt is handed over to him}
	[a man arrives in prison and receives the clothes of prisoners]\\
	\gll	heχ-tːi	tːapri=ra	wačag=ra	kːurtːi=ra	heχ-i-cːe	kʷi-lukː-un ca-d\\
		\textsc{dem.down}-\textsc{pl}	shoe=\textsc{add}	trousers=\textsc{add}	shirt=\textsc{add}	\textsc{dem.down}-\textsc{obl-in}	\textsc{into.the.hands}-give.\textsc{ipfv}-\textsc{icvb} \textsc{cop-npl}\\
	\glt	\sqt{‎‎The shoes, the trousers, the shirt is handed over to him.}

	\ex	\label{He is telling the stories to his family}
	\gll	hel	kulpat-li-cːe	χːabar-t-a-l	∅-ux-ul ca-w \\
		that	family-\textsc{obl-in}	story-\textsc{pl}-\textsc{obl}-\textsc{erg}	\textsc{m}-tell.\textsc{ipfv}-\textsc{icvb} \textsc{cop-m}\\
	\glt	\sqt{He is telling the stories to his family.}
	
	\ex	\label{Mother made Madina eat porridge}
	\gll	aba-l	Madina-cːe	kaš	b-erk-aq-un\\
		mother-\textsc{erg}	Madina-\textsc{in}	porridge	\textsc{n}-eat.\textsc{pfv}-\textsc{caus}-\textsc{pret}\\
	\glt	\sqt{Mother made Madina eat porridge.} (E)
\end{exe}

The \textsc{in}-\isi{ablative} marks causers and causes \refex{the mistakes that he made that happened through him}, involuntary agents \refex{I cannot do it well (lit. it will not go well from me} and other non-canonical agents \refex{He had enough of his wife}. In the \isi{involuntary agent construction}, the verb cannot be transitive, i.e., it cannot have a genuine \isi{agent} argument, but must be intransitive or labile. The added involuntary \isi{agent} is thus rather an adjunct than an argument.

\begin{exe}
	\ex	\label{the mistakes that he made that happened through him}
	\gll	cin-ni-cːe-r	ka-d-ič-ib	χat'a \\
		\textsc{refl}.\textsc{sg}-\textsc{obl-in}-\textsc{abl}	\textsc{down}-\textsc{npl}-occur.\textsc{pfv}-\textsc{pret} mistake\\
	\glt	\sqt{the mistakes that he made (that happened through him)}
	
	\ex	\label{I cannot do it well (lit. it will not go well from me}
	\gll	di-cːe-r	a-arg-u	ʡaˁħ-le \\
		1\textsc{sg-in}-\textsc{abl}	\textsc{neg}-go.\textsc{ipfv}-\textsc{prs}	good-\textsc{advz}\\
	\glt	\sqt{I cannot do it well (lit. it will not go well from me).}
	
	\ex	\label{He had enough of his wife}
	\gll	heχ	xːunul-li-cːe-r	w-elqː-un ca-w \\
		\textsc{dem.down}	woman-\textsc{obl-in}-\textsc{abl}	\textsc{m}-satiate.\textsc{pfv}-\textsc{pret} \textsc{cop-m}\\
	\glt	\sqt{He had enough of his wife.}
\end{exe}


% - - - - - - - - - - - - - - - - - - - - - - - - - - - - - - - - - - - - - - - - - - - - - - - - - - - - - - - - - - - - - - - - - - - - - - - - - - - - - - - - - - - - - - - - - - - - - - - - - - - - - - - - - - - - - - - - - - - - - - - - - - %

\subsubsection{\textsc{sub}-lative -\textit{gu}, \textsc{sub}-essive -\textit{gu-b}, and \textsc{sub}-ablative -\textit{gu-r}}
\label{sssec:sub-lative -gu, sub-essive -gu-b, and sub-ablative -gu-r}

The suffix of the \textsc{sub}-series is \textit{-gu}. The spatial meaning of the \textsc{sub}-series is \sqt{under}. It is added to the oblique form of the noun, but for many \isi{nouns} the oblique form can be identical to the citation form \refexrange{(She) went to pee under the bushes}{In ancient times when we were in Sanzhi we baked (bread) in the community oven}. 

\begin{exe}
	\ex	\label{(She) went to pee under the bushes}
	\gll	ag-ur ca-r	qːʷat'a-gu	dacːi	d-arq'-ar-aj \\
		go.\textsc{pfv}-\textsc{pret} \textsc{cop-f}	bush-\textsc{sub}	urine	\textsc{npl}-do.\textsc{pfv}-\textsc{prs}-\textsc{subj}.3\\
	\glt	\sqt{(She) went to pee under the bushes.}

	
	\ex	\label{when (the worms) dissolved in the sun}
	\gll	bari-gu-d	d-ac'-ib-le	\ldots\\
		sun-\textsc{sub}-\textsc{npl}	\textsc{npl}-thaw.\textsc{pfv}-\textsc{pret}-\textsc{cvb}\\
	\glt	\sqt{(after having put the worms into a bottle of vodka) (they) dissolved in the sun, \ldots}
\end{exe}	

The \textsc{sub}-series has some more lexicalized \refex{In ancient times when we were in Sanzhi we baked (bread) in the community oven} and metaphorical uses \refex{he says, You go home following the words of your wife, yes}, \refex{Here also (i.e. in this case) I am guilty in front of my daughter-in-law}.

\begin{exe} 
	\ex	\label{In ancient times when we were in Sanzhi we baked (bread) in the community oven}
	\gll	kari-gu-b	b-uc'-a-di	dešːa	Sanži-d	d-el-le\\
		oven-\textsc{sub}-\textsc{n}	\textsc{n}-bake.\textsc{ipfv}-\textsc{hab}-1	ancient	Sanzhi-\tsc{1/2pl}	\tsc{1/2pl}-remain.\textsc{pfv}-\textsc{cvb}\\
	\glt	\sqt{In ancient times when we were in Sanzhi we (usually) baked (bread) in the community oven (\tit{kari}).}
	
	\ex	\label{he says, You go home following the words of your wife, yes}
	\gll	``heχ	xːunul-la ʁaj-li-gu	aq-ib-le	qili	arg-ul=de=w?''	∅-ik'-ul ca-w	hana \\
		\textsc{dem}	woman-\textsc{gen}	word-\textsc{obl}-\textsc{sub}	go.through.\textsc{pfv}-\textsc{pret}-\textsc{cvb}	home	go.\textsc{ipfv}-\textsc{icvb}=2\textsc{sg}=\textsc{q}	\textsc{m}-say.\textsc{ipfv}-\textsc{icvb} \textsc{cop-m}	now\\
	\glt	\sqt{He says, ``Do you go home following the words of your wife?''}
	
	\ex	\label{Here also (i.e. in this case) I am guilty in front of my daughter-in-law}
	\gll	heštːu-r=ra	hetː-a-la	bunah-li-gu	r-ič-ib=da	hel	durħuˁ-la	xːunul-la \\
		here-\textsc{f}=\textsc{add}	those-\textsc{obl}-\textsc{gen}	sin-\textsc{obl}-\textsc{sub}	\textsc{f}-occur.\textsc{pfv}-\textsc{pret}=1	that	boy-\textsc{gen}	woman-\textsc{gen}\\
	\glt	\sqt{Here also (i.e. in this case) I am guilty in front of my daughter-in-law.} (lit. \sqt{I occurred under their sins, of that son's wife})
\end{exe}

There is a spatial \isi{preverb} that has the same form and the same meaning as the case marker (\refsec{ssec:Location preverbs and spatial cases expressing direction}) and is often used in clauses that contain \isi{nouns} bearing the \isi{spatial case} suffix \refex{He is sitting with his hands holding his head}. There is also a formally and semantically identical \isi{spatial postposition}\slash adverbial that recurrently appears after the \isi{spatial case} marker \refex{Their, these (vegetables) are probably (growing) under the ground}, \refex{The river with the name Uluchaj is passing by down from our village}.
	
\begin{exe}	
	\ex	\label{He is sitting with his hands holding his head}
	\gll	qajqaj-li-gu	nuˁq-be=ra	gu-ha-d-uc-ib-le	ka-∅-jž-ib ca-w\\
		jaw-\textsc{obl}-\textsc{sub}	hand-\textsc{pl}=\textsc{add}	\textsc{sub-up-npl}-catch.\textsc{pfv}-\textsc{pret}-\textsc{cvb}	\textsc{down-m}-remain-\textsc{pret} \textsc{cop-m}\\
	\glt	\sqt{He is sitting with his hands holding his head (lit. \sqt{catching the hands under the jaw}).}
	
	\ex	\label{Their, these (vegetables) are probably (growing) under the ground}
	\gll	ču-la	d-urkː-ar	iχ-tːi	ganza-l-gu-d	gu-d	daˁʡle	ca-d\\
		\textsc{refl}.\textsc{pl}-\textsc{gen}	\textsc{npl}-find.\textsc{ipfv}-\textsc{prs}	\textsc{dem.down-pl}	ground-\textsc{obl}-\textsc{sub}-\textsc{npl}	under-\textsc{npl}	like	\textsc{cop.npl}\\
	\glt	\sqt{Their, these (vegetables) are probably like (growing) under the ground.}

	
	\ex	\label{The river with the name Uluchaj is passing by down from our village}
	\gll	Ulučaj	b-ik'-ul	erk'ʷ	ca-b	ka-b-ax-an	nišːa-la	šːi-l-gu-r	gu-b-a\\
		Uluchaj	\textsc{hpl}-say.\textsc{ipfv}-\textsc{icvb}	river	\textsc{cop-n}	\textsc{down-n}-go.\textsc{ipfv}-\textsc{ptcp}	1\textsc{pl}-\textsc{gen}	village-\textsc{obl}-\textsc{sub}-\textsc{abl}	down-\textsc{n}-\textsc{dir}\\
	\glt	\sqt{The river with the name Uluchaj is passing by down from our village.}
\end{exe}

% - - - - - - - - - - - - - - - - - - - - - - - - - - - - - - - - - - - - - - - - - - - - - - - - - - - - - - - - - - - - - - - - - - - - - - - - - - - - - - - - - - - - - - - - - - - - - - - - - - - - - - - - - - - - - - - - - - - - - - - - - - %

\subsubsection{\textsc{ante}-lative -\textit{sa}, \textsc{ante}-essive -\textit{sa-b}, and \textsc{ante}-ablative -\textit{sa-r}}
\label{sssec:ante-lative -sa, ante-essive -sa-b, and ante-ablative -sa-r}

The broad meaning of the \textsc{ante}-series is location in front of a reference point or on flat surfaces. It can be translated into English with \sqt{in front, by, at}, but also with \sqt{on}. In the latter meaning it is functionally equivalent to the \textsc{loc}-series and the \textsc{in}-series. Thus, instead of \textit{baˁʡ-li-sa-b} in example \refex{There is also a rifle (hanging) on the wall} it is possible to use \textit{baˁʡ-le-b} \refex{ex:pic on wall_1} or \textit{baˁʡ-li-cːe-b} \refex{ex:pic on wall_2}. Similarly, a more common alternative to \textit{xːun-ni-sa-b} \refex{On the way she met a wolf} is \textit{xːun-ne-b} (way-\textsc{loc-n}), but \textit{xːun-cːe-r} (way-\textsc{in-abl}) is also attested, and instead of \textit{burma-sa} \refex{Then she sat down at the entrance of the house and was crying} also \textit{burma-cːe-b} (threshold-\textsc{in-b}) can be found. I leave the more precise analysis of the semantic similarities and differences between the three cases for future research.

\begin{exe} 
	\ex	\label{Then she sat down at the entrance of the house and was crying}
	\gll	c'il	qal-la	burma-sa	ka-r-iž-ib-le	r-isː-ul	r-už-ib ca-r\\
		then	house-\textsc{gen}	threshold-\textsc{ante}	\textsc{down}-\textsc{f}-be.\textsc{pfv}-\textsc{pret}-\textsc{cvb}	\textsc{f}-cry-\textsc{icvb}	\textsc{f}-be-\textsc{pret} \textsc{cop-f}\\
	\glt	\sqt{Then she sat down at the threshold of the house and was crying.}
	
	%\ex	\label{By God, they say that he was hit by a car,}
	%\gll	wallah	b-ik'ʷ-ar,	mašin-ni-sa	sa-w-erh-ib-le, \ldots\\
		%by.God	\textsc{hpl}-say.\textsc{ipfv}-\textsc{prs}	car-\textsc{obl}-\textsc{ante}	\textsc{hither}-\textsc{m}-beat.\textsc{pfv}-\textsc{pret}-\textsc{cvb}\\
	%\glt	\sqt{By God, they say that he was hit by a car, \ldots}
	
	\ex	\label{There is also a rifle (hanging) on the wall}
	\gll	tupang=ra	le-b	baˁʡ-li-sa-b \\
		rifle=\textsc{add}	exist-\textsc{n}	wall-\textsc{obl}-\textsc{ante}-\textsc{n}\\
	\glt	\sqt{There is also a rifle (hanging) on the wall.}
	
	\ex	\label{On the way she met a wolf}
	\gll	xːun-ni-sa-b	suk	b-ič-ib ca-b	bec' \\
		way-\textsc{obl}-\textsc{ante}-\textsc{n}	meet	\textsc{n}-occur.\textsc{pfv}-\textsc{pret} \textsc{cop-n}	wolf\\
	\glt	\sqt{On the way (she) met a wolf.}
	
	\ex	\label{When they ran away from them, the boy saw a large stone}
	\gll	hel-tː-a-sa-r	sa-r-d-ulq-an	zamana	suk	b-ič-ib-le ca-b	durħuˁ-j	χːula	qːarqːa\\
		that-\textsc{pl}-\textsc{obl}-\textsc{ante}-\textsc{abl}	\textsc{in.front-abl}-\textsc{npl}-direct.\textsc{ipfv}-\textsc{ptcp}	time	meet	\textsc{n}-occur.\textsc{pfv}-\textsc{pret}-\textsc{cvb} \textsc{cop-n}	boy-\textsc{dat}	big	stone\\
	\glt	\sqt{When they ran away from them, the boy came across a large stone.}
\end{exe}

The \textsc{ante}-\isi{ablative} is also used with a \isi{number} of \isi{experiential} predicates such as \sqt{be afraid, fear}, \sqt{long for, miss}, \sqt{be embarrassed}, and \sqt{be ashamed}, with which it denotes the source-like \isi{stimulus} of the experience \refex{They say, are you afraid of your wife}, \refex{ex:I miss youMorph}. When an animate noun bearing the \textsc{ante}-essive or the \textsc{ante}-lative is used together with a verb of movement or a locative predicate the meaning is \sqt{herd, pasture; look after, care for} \refex{I went after the sheep}.

\begin{exe}
	\ex	\label{They say, are you afraid of your wife}
	\gll	``xːunul-li-sa-r	uruχ	∅-ik'-ul=de=w?''	b-ik'-ul \\
		woman-\textsc{obl}-\textsc{ante}-\textsc{abl} fear \textsc{m}-say.\textsc{ipfv}-\textsc{icvb}=2\textsc{sg}=\textsc{q} \textsc{hpl}-say.\textsc{ipfv}-\textsc{icvb}\\
	\glt	\sqt{They say, ``Are you afraid of your wife?''}

	\ex	\label{ex:I miss youMorph}
	\gll	dam b-et'-ib ca-b a-sa-r \\
	\textsc{1sg.dat} \textsc{n}-miss.\textsc{pfv-pret} \textsc{cop-n} \textsc{2sg-ante-abl}	\\
	\glt	\sqt{I miss you, I long for you.} (E)
	
	\ex	\label{I went after the sheep}
	\gll	macːa-l-sa	w-aš-ib=da\\
		sheep-\textsc{obl}-\textsc{ante}	\textsc{m}-go.\textsc{ipfv}-\textsc{pret}=1\\
	\glt‎‎	\sqt{I (masc.) went after the sheep.} (i.e. worked as a shepherd)
\end{exe}


% - - - - - - - - - - - - - - - - - - - - - - - - - - - - - - - - - - - - - - - - - - - - - - - - - - - - - - - - - - - - - - - - - - - - - - - - - - - - - - - - - - - - - - - - - - - - - - - - - - - - - - - - - - - - - - - - - - - - - - - - - - %

\subsubsection{\textsc{post}-lative -\textit{hara}, \textsc{post}-essive -\textit{hara-b}, and \textsc{post}-ablative -\textit{hara-r}}
\label{sssec:post-lative -hara, postessive -hara-b, and post-ablative -hara-r}

The meaning of the \textsc{post}-series is location behind \refex{Then the fox peed behind the wolf}, \refex{when we small children ran behind it} and next to a reference point \refex{they are sitting at the table}, \refex{Will her husband allow her (i.e. my daughter) to throw away the family and (sit) with me?}, although this is not always reflected in the English translation. Sometimes it occurs together with the postposition \tit{hitːi} that roughly has the same meaning \refex{when we small children ran behind it}, \refex{Will her husband allow her (i.e. my daughter) to throw away the family and (sit) with me?}, see \refsec{ssec:postposition hiti}.

\begin{exe} 
	\ex	\label{Then the fox peed behind the wolf}
	\gll	c'il	kːurtːa-l	bec'-li-hara	dacːi	hitːi	d-arq'-ib ca-d	 \\
		then	fox-\textsc{erg}	wolf-\textsc{obl}-\tsc{post}	urine	behind	\textsc{npl}-do.\textsc{pfv}-\textsc{pret} \textsc{cop-npl}\\
	\glt	\sqt{Then the fox peed behind the wolf.}
	
	\ex	\label{when we small children ran behind it}
	\gll	nik'a	durħ-ne	nušːa	hel-i-hara	hitːi	d-uq-un-ne, \ldots\\
		small	boy-\textsc{pl}	1\textsc{pl}	that-\textsc{obl}-\tsc{post}	behind	1/2\textsc{pl}-go.\textsc{pfv}-\textsc{pret}-\textsc{cvb}\\
	\glt	\sqt{When we small children ran behind it (i.e. a pig), \ldots}
	
	\ex	\label{they are sitting at the table}
	\gll	heχ	ust'ul-li-hara-b	ka-b-iž-ib-le	heχ-tːu-b, \ldots\\
		\textsc{dem.down}	table-\textsc{obl}-\tsc{post}-\textsc{hpl} \textsc{down-hpl}-be-\textsc{pret}-\textsc{cvb}	\textsc{dem.down}-\textsc{loc}-\textsc{n}\\
	\glt	\sqt{They are sitting at the table there \ldots}
	
	\ex	\label{Will her husband allow her (i.e. my daughter) to throw away the family and (sit) with me?}
	\gll	rursːi-la	kulpat	lak' či-r-ka-b-arq'-ib-le	heχ-i-la sub-li	r-alt-u=w	di-hara	hitːi-r?\\
		girl-\textsc{gen}	family	throw \textsc{spr}-\textsc{abl-down}-\textsc{hpl}-do.\textsc{pfv}-\textsc{pret}-\textsc{cvb}	\textsc{dem.down}-\textsc{obl}-\textsc{gen}	husband-\textsc{erg}	\textsc{f}-let.\textsc{ipfv}-\textsc{prs}=\textsc{q}	1\textsc{sg}-\tsc{post}	behind-\textsc{f}\\
	\glt	\sqt{Will her husband allow her (i.e. my daughter) to throw away the family and (sit) with me?}
\end{exe}


% - - - - - - - - - - - - - - - - - - - - - - - - - - - - - - - - - - - - - - - - - - - - - - - - - - - - - - - - - - - - - - - - - - - - - - - - - - - - - - - - - - - - - - - - - - - - - - - - - - - - - - - - - - - - - - - - - - - - - - - - - - %

\subsubsection{Directional \textit{-b-a}}
\label{sssec:Directional -gm-a}

There is a further directional suffix containing a \isi{gender}/\isi{number} agreement marker, namely \textit{-\textsc{gm}-a}. This suffix is almost exclusively used with \is{spatial adverb}spatial adverbs \refex{ex:we (were) going by foot there with my friends}, \refex{The river with the name Uluchaj is passing by down from our village} (see \refsec{sec:spatialadverb} for more examples). But it can occasionally be used with \isi{nouns} \refex{This is also growing along the façade of the house}, \refex{They went in the direction of pakh-pakh (microtoponym), across the side of the graveyard,}.

\begin{exe}
	\ex	\label{This is also growing along the façade of the house}
	\gll	il=ra	χːula	b-irʁ-an ca-b	hel	qal-la	baˁʡ-li-cːe-r-b-a\\
		that=\textsc{add}	big	\textsc{n}-come.\textsc{ipfv}-\textsc{ptcp} \textsc{cop-n}	that	house-\textsc{gen}	façade-\textsc{obl-in}-\textsc{abl}-\textsc{n}-\textsc{dir}\\
	\glt	\sqt{That is also growing along the façade of the house.}
	
	\ex	\label{They went in the direction of pakh-pakh (microtoponym), across the side of the graveyard,}
	\gll	paˁχ.paˁχ-la	heχtːu-d-a	ag-ur-re	hej	χːuˁrba-la	šːal-le-r žaˁr-ra	qːatːa-d-a \\
		pakh.pakh-\textsc{gen}	there.\textsc{down-npl}-\textsc{dir}	go.\textsc{pfv}-\textsc{pret}-\textsc{cvb}	this	graveyard-\textsc{gen}	side-\textsc{loc}-\textsc{abl}	lime-\textsc{gen}	canyon.\textsc{loc}-\textsc{npl}-\textsc{dir}\\
	\glt	\sqt{They went in the direction of pakh-pakh (\isi{microtoponym}), across the side of the graveyard, through the lime canyon.}
\end{exe}


%%%%%%%%%%%%%%%%%%%%%%%%%%%%%%%%%%%%%%%%%%%%%%%%%%%%%%%%%%%%%%%%%%%%%%%%%%%%%%%%

\section{Derivation of nouns}
\label{sec:nounderivation}


% --------------------------------------------------------------------------------------------------------------------------------------------------------------------------------------------------------------------- %

\subsection{Agent nouns with \textit{-či}}
\label{subsection:Agent nouns with -či}

This suffix is found in a great \isi{number} of East Caucasian languages and originates from Turkic. It productively derives \isi{agent} \isi{nouns} from \isi{nouns} denoting objects or places. The base \isi{nouns} are partially loans from Turkic or Arabic, but also include more recent loans from Russian and native words. The derived \isi{nouns} can refer to men as well as to women and form the plural with \tit{-be}.

\begin{exe}
	\ex	\label{masqaranjoke}
	\begin{xlist}
		\TabPositions{14em,16em}
		\ex	\tit{masqar(an)-či} \sqt{jester, jokester}		\tab	<	\tab	\tit{masqaran} \sqt{joke}
		\ex	\tit{bajraq-či} \sqt{standard bearer}			\tab	<	\tab	\tit{bajraq} \sqt{flag, standard}
		\ex	\tit{tuken-či} \sqt{shop assistant}			\tab	<	\tab	\tit{tuken} \sqt{shop}		
		\ex	\tit{isklad-či} \sqt{person responsible for the community storehouse}
		\sn	~\hspace*{1em}						\tab	<	\tab	\tit{iskald} \sqt{storehouse}
		\ex	\tit{anq-či} \sqt{gardener}				\tab	<	\tab	\tit{anq} \sqt{garden}
		\ex	\tit{qːalaj-či} \sqt{s.o. who solders, brazes}		\tab	<	\tab	\tit{qːalaj} \sqt{tin, tin-solder}
	\end{xlist}
\end{exe}


% --------------------------------------------------------------------------------------------------------------------------------------------------------------------------------------------------------------------- %

\subsection{Agent nouns with and \textit{-kar}, \textit{-q'aˁ}, and \textit{-uˁq'}}\largerpage[-1]
\label{ssec:Agent nouns with -q'aˁ, -uˁq' and -kar}

These suffixes, which are not productive, derive \isi{agent} \isi{nouns} from other \isi{nouns}, infinitives, short \isi{adjectives}, parts of \is{compound verb}compound verbs and one postposition. The suffix \textit{-kar}, which also exists in Standard Dargwa, is predominantly attested with borrowed \isi{nouns} \refex{sinner}. Nouns with \textit{-q'aˁ} mostly form their plural with \textit{-ne} (after deletion of the stem-final vowel) \refex{haymaker}; \isi{nouns} with \textit{-uˁq'} use \textit{-e} as plural marker \refex{lierdrummer} and \isi{nouns} with \textit{-kar} employ \textit{-te} or \textit{-ne}.

\begin{exe}
\ex	\label{sinner}
	\begin{xlist}
		\TabPositions{14em,16em,18em}
		\ex	\tit{ʡaˁjib-kar} \sqt{convict, guilty person}	\tab	<	\tab	\tit{ʡaˁjib} \sqt{guilt, blame}
		\ex	\tit{bunah-kar} \sqt{sinner}			\tab	<	\tab	\tit{bunah} \sqt{sin}
		\ex	\tit{ʡaˁmal-kar} \sqt{trickster}			\tab	<	\tab	\tit{ʡaˁmal} \sqt{talent, trick}
		\ex \textit{ʡaˁsi-kar} \sqt{mean, evil person} \tab	<	\tab \textit{ʡaˁsi} \sqt{evil}
		\ex \textit{pitne-kar} \sqt{intriguer, gossiper} \tab	<	\tab \textit{pitne} \sqt{gossip}
	\end{xlist}


	\ex	\label{haymaker}
	\begin{xlist}
		\TabPositions{14em,16em}
		\ex	\tit{aršːi-q'aˁ} \sqt{mower, hay-maker}		\tab	<	\tab	\tit{aršːi} \sqt{mature crops}
		\ex	\tit{ajar-q'aˁ} \sqt{hunter}			\tab	<	\tab	\tit{ajar} \sqt{hunt}
		\ex	\tit{utːi-q'aˁ} \sqt{mower, hay-maker}		\tab	<	\tab	\tit{utː-ij} \sqt{mow} (\tsc{ipfv})
		\ex	\tit{mušlu-q'aˁ} \sqt{wrestler}			\tab	<	\tab	\tit{mušlu w-iħ-ij} \sqt{wrestle, fight}\\
		\tab \tab\tab (\tsc{ipfv})
		\ex	\tit{sːiħru-q'aˁ} \sqt{witch, sorceress} 		\tab	<	\tab	\tit{sːiħru} \sqt{hypnosis}
		\ex	\tit{walli-q'aˁ} \sqt{best man}, \tit{ralli-q'aˁ} \sqt{maid of honor}
		\sn	~\hspace*{1em}					\tab	<	\tab	\tit{b-alli} \sqt{together}
	\end{xlist}

\ex	\label{lierdrummer}
	\begin{xlist}
		\TabPositions{14em,16em}
		\ex	\tit{sːunk-uˁq'} \sqt{liar}				\tab	<	\tab	\tit{sːunk} \sqt{lie}
		\ex	\tit{tːamr-uˁq'} \sqt{drummer}			\tab	<	\tab	\tit{tːam} (plural \tit{tːamre}) \sqt{drum}
		\ex	\tit{gapn-uˁq'} \sqt{show-off, braggart}	\tab	<	\tab	\tit{gap b-arq'-ij} \sqt{praise} (\tsc{pfv})
	\end{xlist}

\end{exe}

% --------------------------------------------------------------------------------------------------------------------------------------------------------------------------------------------------------------------- %

\subsection{Abstract nouns with \textit{-dex}}
\label{ssec:Abstract nouns with -dex}

This very productive suffix derives abstract \isi{nouns} from open class words (verbs, \isi{adjectives}, \isi{nouns}, adverbs, bound stems). The \isi{nouns} derived by means of the suffix \textit{-dex} control either neuter singular \refex{They did not forgive (each other)} or occasionally neuter plural agreement. Some of the derived \isi{nouns} can form the plural, usually by means of gemination of the last segment plus suffixation of \textit{-e}, e.g. \textit{ʁʷabzadexːe} \sqt{heroism, bravery, courage}. Thus, it is also possible to analyze the underlying form of this suffix as containing a geminate as final consonant that is degeminated in the singular because of the impossibility of having \is{geminate consonant}geminate consonants in syllable-final position. A similar analysis regarding geminate stem-final \isi{consonants} in singular vs. plural forms of other \isi{nouns} that take the plural suffix \textit{-e} has been suggested in \refsec{sec:FrequentAndProductivePluralSuffixes}. 

Base \isi{adjectives} are mostly underived short \isi{adjectives}, i.e. bare roots \refex{kːuš-dexhunger}. The numeral \tit{ca} \sqt{one}, but no other numeral, can also serve as base: 

\begin{exe}
	\ex	\label{kːuš-dexhunger}
	\begin{xlist}
		\TabPositions{14em,16em}
		\ex	\tit{kːuš-dex} \sqt{hunger}	\tab	<	\tab	\tit{kːuš} \sqt{hungry}
		\ex	\tit{gʷana-dex} \sqt{warmth}	\tab	<	\tab	\tit{gʷana} \sqt{warm}
		\ex	\tit{ʡaˁsi-dex} \sqt{anger}	\tab	<	\tab	\tit{ʡaˁsi} \sqt{angry}
		\ex	\tit{ca-dex} \sqt{unity}		\tab	<	\tab	\tit{ca} \sqt{one}
	\end{xlist}
\end{exe}

Base \isi{nouns} either denote people or abstract items. The derived \isi{nouns} refer to abstract items or properties associated with the thing or the person that the base noun denotes.

\begin{exe}
	\ex	\label{ʡaˁχːuˁl-dexhospitality}
	\begin{xlist}
		\TabPositions{2em,4em}
		\ex	\tit{ʁʷabza-dex} \sqt{heroism, bravery, courage} \\
\tab	<	\tab	\tit{ʁʷabza} \sqt{dzhigit, hero}
		\ex	\tit{ʡaˁχːuˁl-dex} \sqt{hospitality}		\\
		\tab	<	\tab	\tit{ʡaˁχːuˁl} \sqt{kunak, guest, host}
		\ex	\tit{dajark'a-dex} \sqt{duties of a milkmaid}\\
		\tab					<	\tab	\tit{dajark'a} \sqt{milkmaid}
		\ex	\tit{uruχ-dex} \sqt{fearfulness, anxiousness} \\
		\tab					<	\tab	\tit{uruχ b-iχʷ-ij} \sqt{get afraid}
		\ex	\tit{saniʡaˁt-dex} \sqt{skillfulness}			\\
			<	\tab	\tit{saniʡaˁt} \sqt{skill}
	\end{xlist}
\end{exe}

With verbs only the preterite (predominantly from perfective stems) and the \tit{-an} \isi{participle} (usually from imperfective stems) can function as base for the \isi{derivation} of \tit{-dex}-\isi{nouns} \refex{They did not forgive (each other)}. The only regular exceptions are forms of the \isi{copula} verbs, which have defective paradigms \refex{all my existing (things)}.

\begin{exe}
	\ex	\label{They did not forgive (each other)}
	\gll	ca	ha-b-erχː-ur-dex	a-b-irχʷ-i\\
		one	\textsc{up-hpl}-apologize.\textsc{pfv}-\textsc{pret}-\textsc{nmlz}	\textsc{neg}-\textsc{n}-become.\textsc{ipfv}-\textsc{hab}.\textsc{pst}\\
	\glt	\sqt{They did not forgive (each other).}

	\ex	\label{all my existing (things)}
	\gll	di-la	li<b>il=ra	le-b-dex\\
		1\textsc{sg}-\textsc{gen}	all<\textsc{n}>=\textsc{add}	exist-\textsc{n}-\textsc{nmlz}\\
	\glt	\sqt{all my existing (things)}
\end{exe}

The following example shows a compound noun derived with \textit{-dex}. There is no independent base noun \textit{urk'i} \textit{hitːi} and the complex can also not be regarded as a postpositional phrase because \textit{hitːi} requires the dependent noun to be marked with the \isi{genitive}. When \textit{-dex} is added the complex functions as a nominal that controls agreement on the clause-final verb, and the \isi{reciprocal pronoun} preceding it functions as a modifier of it (or of the noun \textit{urk'i}).

\begin{exe}
	\ex	\label{There is no pity between them}
	\gll	ca-lla	ca-lla	urk'i	hitːi-dex	b-akː-u\\
		one-\textsc{gen}	one-\textsc{gen}	heart	behind-\textsc{nmlz}	\textsc{n}-\textsc{cop.neg}-\textsc{prs} \\
	\glt	\sqt{There is no pity between them (one for the other).}
\end{exe}

The nominalized verbs retain their arguments, but since they are nominalized they occur in argument position and can be modified, e.g. by personal pronouns \refex{because of my inability to walk}. Thus, the subject-like arguments can either occur as preserved arguments of the nominalized verb \refex{because of our embarrassment, because of our fear of the teacher}, \refex{Because you are there I feel bad} or they can occur as possessors \refex{because of my inability to walk}.

\begin{exe} 
	\ex	\label{because of my inability to walk}
	\gll	di-la	w-aš-ij	erχʷ-an-dex-li-j \\
		1\textsc{sg}-\textsc{gen}	\textsc{m}-go.\textsc{ipfv}-\textsc{inf} be.able.\textsc{ipfv}.\textsc{neg}.\textsc{m}-\textsc{ptcp}-\textsc{nmlz}-\textsc{obl}-\textsc{dat}\\
	\glt	\sqt{because of my inability to walk}

	\ex	\label{because of our embarrassment, because of our fear of the teacher}
	\gll	nušːa	uruc	d-iχ-ub-dex-li-j,	učitil-li-sa-rka	uruχ	d-iχ-ub-dex-li-j\\
		1\tsc{pl} embarrassed 1/2\tsc{pl}-become\tsc{.pfv-pret-nmlz-obl-dat}	teacher\tsc{-obl-ante-abl}	fear	1/2\tsc{pl-}become\tsc{.pfv-pret-nmlz-obl-dat}\\
	\glt	\sqt{because of our embarrassment, because of our fear of the teacher}
\end{exe}

Many of the derived abstract \isi{nouns}, especially those derived from verbs, have the semantic role of cause or reason and therefore bear the \isi{dative} suffix. Thus, deverbal nominals occur as nominalized \is{adverbial clause}adverbial clauses with the meaning \sqt{because of X}. When inflected for the \isi{dative} more words are admissible as base for the \isi{derivation} than would be possible without the case suffix. For instance, there is no noun \textit{itːu-r-dex}, but if this word is inflected for the \isi{dative}, it can occur as an adverbial denoting the cause \refex{Because you are there I feel bad}.

\begin{exe}
	\ex	\label{Because you are there I feel bad}
	\gll	u itːu-r-dex-li-j dam wahi-l ca-d \\
		2\textsc{sg}	there-\textsc{f-nmlz-obl}-\textsc{dat}	1\textsc{sg}.\textsc{dat}	bad-\textsc{advz} \textsc{cop-npl}\\
	\glt	\sqt{Because you (fem.) are there I feel bad.} (E)
\end{exe}


% --------------------------------------------------------------------------------------------------------------------------------------------------------------------------------------------------------------------- %

\subsection{Action nouns\slash event nouns and tools with \textit{-ala}}
\label{ssec:Action and event nouns way of V-ing with -ala}

The suffix \textit{-ala} is used for the formation of \isi{nouns} that denote tools and of action or event-denoting \isi{nouns} with the meaning \sqt{way of V-ing} from verbal stems. It is morphologically productive and rather resembles inflectional suffixes such as the \isi{masdar}, but it is not commonly used in texts. The resulting \isi{nouns} keep their arguments. They can be marked for plural (suffixes \textit{-e} or \textit{-me}) and control neuter singular agreement. Those \isi{nouns} that have been derived from \is{transitive verb}transitive verbs with a \isi{gender} \isi{agreement prefix} have a frozen \isi{gender} prefix \textit{b-} (e.g. \textit{biχala} \sqt{lace, shoelace}, \textit{bušːukala} \sqt{broom}). However, the verb given in \refex{irʁ-ala understandingA} originates from an extended \isi{intransitive verb} that agrees with the human \isi{agent} in \isi{gender}. As \refex{(Look at) the way of the girl's holding the child in her hands} shows, this noun has variable \isi{gender} depending on the \isi{gender} of the referent and thus behaves like a small \isi{number} of underived  \isi{nouns} with \isi{gender} exponents \refex{ex:nounsovertgendermarkers}.
%
%\footnote{According to \citep[e.g. Z.][90]{AbdullaevEtAl2014}, in Standard Dargwa the noun is analyzed as being derived from a verb \textit{bušk-es} \sqt{sweep up} that, however, does not seem to exist in Sanzhi.}

\largerpage[2]
\begin{exe}
	\ex	\label{irʁ-ala understanding}
	\begin{xlist}
		\TabPositions{12em,14em}
		\ex	\tit{biχ-ala} \sqt{lace, shoelace}		\tab	<	\tab	\tit{b-iχ-ij} \sqt{tie, bind, fasten} (\tsc{pfv})
		\ex	\tit{bušːuk-ala} \sqt{broom}		\tab	<	\tab  (\textit{b-ušk-}	\sqt{sweep up} (\tsc{pfv}))\footnote{Synchronically, this verb is no longer used in Sanzhi Dargwa and has been replaced by a \isi{compound verb} \textit{qʷaˁrš b-arq'-ij} `wipe, sweep, stroke'. However, it is attested, e.g., in the South Dargwa variety Tanti. Thus, I suggest that Sanzhi lost the verb, but kept the noun. An alternative explanation suggested by my main language assistant is that Sanzhi borrowed the noun from Standard Dargwa.}
		\ex	\tit{icː-ala} \sqt{illness, disease, pain}		\tab	<	\tab	\tit{icː-ij} \sqt{hurt, ache} (\tsc{ipfv})
		\ex	\tit{berkʷ-ala} \sqt{dish, food}		\tab	<	\tab	\tit{b-erkʷ-ij} \sqt{eat} (\tsc{pfv})
		\ex	\tit{irʁ-ala} \sqt{understanding}		\tab	<	\tab	\tit{irʁ-ij} \sqt{understand} (\tsc{ipfv})
		\ex	\tit{kaxʷ-ala} \sqt{way of killing}		\tab	<	\tab	\tit{kaxʷ-ij} \sqt{kill} (\tsc{pfv})
		\ex	\tit{haʔ-ala} \sqt{proverb, saying}	\tab	<	\tab	\tit{haʔ-ij} \sqt{say} (\tsc{pfv})
		\ex	\tit{bet'-ala} \sqt{boredom}		\tab	<	\tab	\tit{b-et'-ij} \sqt{be bored, long for} (\tsc{pfv})
		\ex	\tit{dalaj r-ikʷ-ala} \sqt{way of singing songs} \label{irʁ-ala understandingA}
		\sn	~\hspace*{1em}				\tab	<	\tab	\tit{dalaj r-ikʷ-ij} \sqt{sing songs}
	\end{xlist}
\end{exe}

\begin{exe}
	\ex	\label{(Look at) the way of the girl's holding the child in her hands}
	\gll	iχ	rursːi-la	qːuʁa-l	kʷi	sa-b-uc-ala=le	nik'a-ce\\
		\textsc{dem.down}	girl-\textsc{gen}	beautiful-\textsc{advz}	in.the.hands	\textsc{hither-n}-keep.\textsc{pfv}-\textsc{nmlz}=\textsc{emph}	small-\textsc{dd}.\textsc{sg}\\
	\glt	\sqt{(Look at) the beautiful way of the girl's holding the child in her hands!}\footnote{The \isi{agreement prefix} in this example is neuter singular, expressing the \isi{gender} of the nominalized verb form itself. The noun \textit{rursːi-la} is a \isi{genitive} modifier of the nominalized verb.}
\end{exe}


% --------------------------------------------------------------------------------------------------------------------------------------------------------------------------------------------------------------------- %

\subsection{Action and event nouns with \textit{-utːi} and \textit{-a}}
\label{ssec:Action and event nouns with -uti and -a}

The suffixes \textit{-utːi} and \textit{-a} (allomorph \textit{-aˁ}) derive action \isi{nouns} from verbs, whereby they are almost exclusively added to the perfective stem \refex{belč'-utːistudy}, \refex{bebk'-adeath, funeral}. They are not productive. Most of the \isi{nouns} can be marked for plural, normally with the suffix \textit{-ne} (e.g. \tit{bebk'-ne} \sqt{funerals}). They control either neuter singular \refex{his acting} or occasionally neuter plural \refex{bebk'-adeath, funeralA}, \refex{bebk'-adeath, funeralC} or masculine singular \refex{bebk'-adeath, funeralB} agreement. The \isi{gender} prefixes of the deverbal \isi{nouns} are petrified. 

\begin{exe}
	\ex	\label{belč'-utːistudy}
	\begin{xlist}
		\TabPositions{12em,14em}
		\ex	\tit{belč'-utːi} \sqt{study}			\tab	<	\tab	\tit{b-elč'-ij} \sqt{read, learn, study} (\textsc{pfv})
		\ex	\tit{berkː-utːi} \sqt{eating}		\tab	<	\tab	\tit{b-erkːʷ-ij} \sqt{eat} (\textsc{pfv})
		\ex	\tit{berc-utːi} \sqt{rescue, salvation}	\tab	<	\tab	\tit{b-erc-ij} \sqt{save, rescue} (\textsc{pfv})
		\ex	\tit{barq'-utːi, birq'-utːi} \sqt{work, deed, action} \\
		<	\tit{b-arq'-ij} (\tsc{pfv}), \tit{b-irq'-ij} (\tsc{ipfv}) \sqt{do, make}
		
	\end{xlist}

	\ex	\label{his acting}
	\gll	b-arq'-utːi	iž-i-la \\
		\textsc{n}-do.\textsc{pfv}-\textsc{nmlz}	this-\textsc{obl}-\textsc{gen}\\
	\glt 	\sqt{her\slash his acting}

	\ex	\label{bebk'-adeath, funeral}
	\begin{xlist}
		\TabPositions{12em,14em}
		\ex	\tit{bebk'-a} \sqt{death, funeral}		\tab	<	\tab	\tit{b-ebk'-ij} \sqt{die} (\tsc{pfv})
		\ex	\tit{buq-a} \sqt{societal, community help}
		\sn	~\hspace*{1em}					\tab	<	\tab	\tit{b-uq-ij} \sqt{go} (\tsc{pfv})
		\ex	\tit{dikː-a} \sqt{love}				\tab	<	\tab	\tit{b-ikː-ij} \sqt{want, like, love} \label{bebk'-adeath, funeralA}
		\ex	\tit{daˁq-a} \sqt{wound}				\tab	<	\tab	\tit{b-aˁq-ij} \sqt{wound, hit, strike} (\tsc{pfv}) \label{bebk'-adeath, funeralC}
		\ex	\tit{arχː-aˁ} \sqt{deed}				\tab	<	\tab	\tit{b-arχː-ij} \sqt{be engaged in} \label{bebk'-adeath, funeralB}
		\ex	\tit{baˁħ-aˁ berʁʷ-a} \sqt{wet-dry} (name of a game)
		\sn	~\hspace*{1em}					\tab	<	\tab	\tit{b-aˁħ-ij} \sqt{get wet}, \textit{b-erʁʷ-ij} \sqt{dry} 

	\end{xlist}
\end{exe}


% --------------------------------------------------------------------------------------------------------------------------------------------------------------------------------------------------------------------- %

\subsection{Other derived nouns}
\label{ssec:Some more derived nouns}

There are four \isi{nouns} that are formed by means of \isi{reduplication} of the first syllable and the suffix \textit{-aj}. For most of them the base is bound and unclear, i.e. it does not exist as an independent word. The last three of the four words are toys and/or names of games. 

\begin{exe}
	\ex	\label{ex:rugrugajsmall circles}
	\begin{xlist}
		\TabPositions{12em,14em}
		\ex	\tit{rugrugaj} \sqt{small circles}		\tab	<	\tab	\tit{rugrug}	\sqt{round}
		\ex	\tit{diʁdiʁaj} \sqt{spinning top}		\tab	<	\tab	? (\tit{*diʁdiʁ})
		\ex	\tit{muqluqaj} \sqt{spinning top}		\tab	<	\tab	? (\tit{*muqluq})
		\ex	\tit{cicinaj-te} \sqt{retractor, name of a game}
		\sn	~\hspace*{1em}				\tab	<	\tab	? (\tit{*cicinaj})
	\end{xlist}
\end{exe}

There are a fair \isi{number} of Sanzhi words that end in \textit{-aj} \refex{ex:confusionMix}, \refex{ex:gossiper} and it seems reasonable to suggest that the language once had a similar derivational suffix. This has been suggested for Standard Dargwa \citep[90]{AbdullaevEtAl2014}. Around half of the words are clearly morphologically complex. Two of them are only used in child-directed speech, and five of the words refer to human beings, more particularly, kinship relations or social roles \refex{ex:gossiper}.

\begin{exe}
	\ex	\label{ex:confusionMix}
	\begin{xlist}
		\TabPositions{12em,14em}
		\ex	\tit{ʁudurmaj} \sqt{confusion, mess}		\tab	<	\tab	\tit{ʁudur}	\sqt{mixed}
		\ex	\tit{aqanaj} \sqt{top, summit}		\tab	<	\tab	\tit{aq}	\sqt{tall, high, loud}		
		\ex	\tit{zanʁaraj} \sqt{beanbag, rattle}		\tab	<	\tab	\tit{zanʁ}	\sqt{ringing}
		\ex	\tit{pːapːaj} \sqt{bread}		(child-directed speech)
		\ex	\tit{maˁmmaj} \sqt{scary}		(child-directed speech)
		\ex	\tit{tːapːaraj} \sqt{zany, moony, erratic}		
		\ex	\tit{sːurkːupːaj} \sqt{swing} (i.e. hanging seat)			
		\ex	\tit{dalaj} \sqt{song}		
	\end{xlist}
\end{exe}


\begin{exe}
	\ex	\label{ex:gossiper}
	\begin{xlist}
		\TabPositions{12em,14em}
		\ex	\tit{wanaħaˁj} \sqt{mama's boy, lazy person}		
		\ex	\tit{kʼukʼaj} \sqt{gossiper}		
		\ex	\tit{χːabaj} \sqt{grandmother}		\tab	<	\tab	\tit{χːula}	\sqt{big} + \tit{aba}	\sqt{mother}
		\ex	\tit{χːatːaj} \sqt{grandfather}		\tab	<	\tab	\tit{χːula}	\sqt{big} + \tit{atːa}	\sqt{father}
		\ex	\tit{w-ikː-an-aj} \sqt{beloved}		\tab	<	\tab	\tit{b-ikː-}	\sqt{love, like, want} + modal \\
		\tab	\tab	\tab \isi{participle} \tit{-an}
	\end{xlist}
\end{exe}


There are three kinship terms with the suffix \textit{-q'ar} \refex{ex:ucːi-q'armale cousin}. 

\begin{exe}
	\ex	\label{ex:ucːi-q'armale cousin}
	\begin{xlist}
		\TabPositions{12em,14em}
		\ex	\tit{ucːi-q'ar} \sqt{male cousin}		\tab	<	\tab	\tit{ucːi}	\sqt{brother}
		\ex	\tit{rucːi-q'ar} \sqt{female cousin}	\tab	<	\tab	\tit{rucːi}	\sqt{sister}
		\ex	\tit{k'uni-q'ar} \sqt{nephew}		\tab	<	\tab	?
	\end{xlist}
\end{exe}

A few words are derived by means of the \isi{spatial case} \textit{-gu} plus a further suffix \textit{-(l)la}, which is formally identical to the \isi{genitive} \refex{ex:qajqaj-li-gu-lachin}, and with the postposition \textit{sala} \sqt{in front} or the \isi{spatial case} suffix \textit{-sa} (\tsc{ante}) \refex{ex:kːuma-sala the village square in Sanzhi}.

\begin{exe}
	\ex	\label{ex:qajqaj-li-gu-lachin}
	\begin{xlist}
		\TabPositions{12em,14em}
		\ex	\tit{qajqaj-li-gu-la} \sqt{chin}		\tab	<	\tab	jaw\tsc{-obl-sub-nmlz}
		\ex	\tit{ʡaˁmi-gu-la} \sqt{window-sill}	\tab	<	\tab	window\tsc{-sub-nmlz}
		\ex	\tit{šːal-li-gu-la} \sqt{lateral upper part of the body}
		\sn	~\hspace*{1em}				\tab	<	\tab	side\tsc{-obl-sub-nmlz}		
	\end{xlist}

	\ex	\label{ex:kːuma-sala the village square in Sanzhi}
	\begin{xlist}
		\TabPositions{12em,14em}
		\ex	\tit{kːuma-sala} \sqt{the village square in Sanzhi}
		\sn	~\hspace*{1em}				\tab	<	\tab	kuma-\textsc{in.front}
		\sn	~\hspace*{1em}				\tab	{}	\tab	(\tit{kuma} = place name)
		\ex	\tit{ul-be-sala-nte} \sqt{glasses}		\tab	<	\tab	eye-\textsc{pl}-\textsc{in.front-pl}
		\ex	\tit{qaˁb-li-sa kajqan} \sqt{necklace}	\tab	<	\tab	neck-\textsc{obl}-\textsc{ante} + hang.\textsc{pfv}.\textsc{ptcp} 
	\end{xlist}
\end{exe}

Place names are derived from verbs by means of the \isi{locative participle} \textit{-an} (see\linebreak\refsec{sssec:The locative participle}). Names for ethnic groups, inhabitants, etc. are derived by adding the suffix \tit{-(a)n} to a root that might be the place name or some other root related to it (see \refsec{cpt:morph-placenames}).

%%%%%%%%%%%%%%%%%%%%%%%%%%%%%%%%%%%%%%%%%%%%%%%%%%%%%%%%%%%%%%%%%%%%%%%%%%%%%%%%

\section{Reduplication and compounding}
\label{sec:nounwordformation}
Neither \isi{compounding} nor \isi{reduplication} are productive or frequent ways of forming new nominals in Sanzhi. In this section, I list the majority of reduplicated and compound \isi{nouns} that I have been able to identify so far. 

%--------------------------------------------------------------------------------------------------------------------------------------------------------------------------------------------------------------------- %

\subsection{Reduplication}
\label{ssec:Reduplication}

A \isi{number} of \isi{nouns} have the structure CV.CV(V) or CVC.CVC(V) and are composed of two (almost) identical segments following each other. Some of the \isi{nouns} are clearly onomatopoetic \refex{ex:qːaˁqːaˁshook2}, others are not \refex{ex:qːaˁqːaˁshook}.

\begin{exe}

	\ex	\label{ex:qːaˁqːaˁshook2}
		\TabPositions{15em,12em}
		\tit{k'urk'ur} \sqt{turkey cock}		\tab		\tit{qːut'qːut'i} \sqt{woodpecker} \\
		\tit{t'at'ar} \sqt{fly}			\tab		\tit{baˁtbaˁt} \sqt{duck} \\
		\tit{čaˁχčaˁχ} \sqt{waterfall}		\tab		\tit{ħaˁħaˁ} \sqt{laughter} \\
		\tit{qːuqːu}	\sqt{thunder}		\tab		\tit{xʷixʷit'} \sqt{pipe}
	
	\ex	\label{ex:qːaˁqːaˁshook}
		\TabPositions{15em,12em}
		\tit{qːaˁqːaˁs} \sqt{hook}			\tab		\tit{gumgum} \sqt{small metal pitcher}\\
		\tit{majmaj} \sqt{condemnation}		\tab		\tit{gʷagʷa} \sqt{flower}\\
		\textit{ʁaˁʁaˁ} \sqt{pebble, gravel}
	
\end{exe}


% --------------------------------------------------------------------------------------------------------------------------------------------------------------------------------------------------------------------- %

\subsection{N + N compounds}
\label{ssec:Nounnoun compounds}

Sanzhi has a few noun plus noun compounds. Occasionally, the origin of one of the compound members is unclear. Because Sanzhi has also nominal \isi{apposition} it is not always easy to differentiate between juxtaposed \isi{nouns} in an \isi{apposition} construction that syntactically form a phrase (\refsec{sec:Noun phrases}) and compounds that function as one word. There are several criteria that need to be applied in order to identify compounds. First, a few words show a greater phonological cohesion (e.g. \textit{ababa} \sqt{grandmother}) or make use of word forms that differ from the base stems or inflected forms (e.g. \textit{kːalkːa} \sqt{tree}in \textit{kːalkːa zize} \sqt{strawberry}).

\begin{exe}
	\ex	\label{ex:kːalkːa zizeraspberry}
	\begin{xlist}
		\TabPositions{10em,12em}
		\ex	\tit{ababa}\footnote{My main language assistant G. Gadzhimuradov said that this and the following word in \refex{ex:ATTABAphon} are used by the younger generations, but not by elderly speakers, and may have been borrowed from other Dargwa varieties.} \sqt{grandmother} (on mother's side)
		\sn	~\hspace*{1em}					\tab	<	\tab	\tit{aba-la aba} \sqt{mother-\tsc{gen} mother}
				\ex	\label{ex:ATTABAphon} \tit{atːaba} \sqt{grandmother} (on father's side)
		\sn	~\hspace*{1em}	\tab	<	\tab	\tit{atːa-la aba} \sqt{father-\tsc{gen} mother}
		
		
		\ex	\tit{kːalkːa zize} \sqt{raspberry}			\tab	<	\tab	\tit{kːalkːi} \sqt{tree} + \tit{zize} \sqt{strawberry}
		\ex	\tit{waqːa ʡaˁt'a, waqːaˁt'a} \sqt{turtle}
		\sn	~\hspace*{1em}					\tab	<	\tab	\tit{waqːa} \sqt{skull} + \tit{ʡaˁt'a} \sqt{frog}
	\end{xlist}
\end{exe}

Second, a few compounds are of the dvandva type. This means that they are coordinative compounds, but in contrast to \isi{noun phrase} \isi{coordination} (\refsec{sec:Coordination of noun phrases and other phrases}) dvandva compounds do not make use of the \isi{additive enclitic} and function morphosyntactically as one word, i.e., they take only one inflectional suffix.

\begin{exe}
	\ex	\label{ex:nominalcompounds2}
	\begin{xlist}		
		\TabPositions{10em,12em}	
				\ex	\tit{atːa aba} \sqt{parents}	\tab	<	 \tit{atːa} \sqt{father} + \tit{aba} \sqt{mother}
		\ex	\tit{urk'i muqer} \sqt{lungs}			\tab	<	\tit{urk'i} \sqt{heart} +\tit{muqer} \sqt{breast}
		\ex	\tit{qːuqːu laˁmc'} \sqt{thunderstorm, tempest}
		\sn	~\hspace*{1em}					\tab	<	\tit{qːuqːu} \sqt{thunder} + \tit{laˁmc'} \sqt{lightning}				
		\end{xlist}
\end{exe}

Two more \isi{nouns} are classified as noun + noun compounds because they are semantically not transparent. They are neither coordinative compounds nor do they have an identifiable head noun such that they cannot be analyzed as appositions (i.e. noun phrases). The second noun in \refex{ex:tamaHamaPhon} seems to have been formed by \isi{reduplication}.

\begin{exe}
	\ex	\label{ex:nominalcompounds3}
	\begin{xlist}	
		\TabPositions{10em,12em}		
		\ex	\tit{dučːi laˁmc'} \sqt{firefly}			\tab	<	\tab	\tit{dučːi} \sqt{night} + \tit{laˁmc'} \sqt{lightning}
		\ex	\label{ex:tamaHamaPhon} \tit{t'ama hama} \sqt{gossip, hub-hub}
		\sn	~\hspace*{1em}					\tab	<	\tab	\tit{t'ama} \sqt{sound, noice, voice} + ?
	\end{xlist}
\end{exe}

		%\ex	\tit{silmik qix} \sqt{hazelnut}			\tab	<	\tab	\tit{silmik} \sqt{hazelnut} + \tit{qix} \sqt{nut}
		
		%\ex	\tit{ugaj aba} \sqt{stepmother}			\tab	<	\tab	 ? + \tit{aba} \sqt{mother}
				%\ex	\tit{ʡaˁχul atːa} \sqt{stepfather}			\tab	<	\tab	\tit{ʡaˁχːul} \sqt{kunak, host-guest}
		%\sn	~\hspace*{1em}					\tab	{}	\tab	+ \tit{atːa} \sqt{father}
				%\ex	\tit{lut'i t'uˁ} \sqt{sole of foot}			\tab	<	\tab	\tit{lut'i} \sqt{horseshoe} + \tit{t'uˁ} \sqt{ foot}
				
%--------------------------------------------------------------------------------------------------------------------------------------------------------------------------------------------------------------------- 

\subsection{Other compounds}
\label{ssec:AdjN}

There are a few more X + noun combinations that have been lexicalized as compound \isi{nouns} and show phonological and morphological cohesion. The first part X can be an adjective (including numerals) or a verb \refex{ex:χːulabamother-in-law}. The order of both parts corresponds to the usual order of phrases (modifier + noun) or clauses (object + verb). 



\begin{exe}
	\ex	\label{ex:χːulabamother-in-law} 
	\begin{xlist}
		\TabPositions{10em,12em}
		\ex	\tit{χːulaba}	\sqt{mother-in-law}		\tab	<	\tab	 \tit{χːula} \sqt{big, old} + \tit{aba} \sqt{mother}
		\ex	\tit{χːulatːa}	\sqt{father-in-law}		\tab	<	\tab	 \tit{χːula} \sqt{big, old} + \tit{atːa} \sqt{father}
	\ex	\tit{dumbat} \sqt{food} 			\tab	<	\tab	\tit{dum} \sqt{edge} + ? \tit{b-at-ij} \sqt{leave, let} (\textsc{pfv})
			
	\end{xlist}
\end{exe}	

%

The compound \isi{nouns} that contain numerals as their first part can be divided into two groups. The first group in \refex{ex:CompoundNounsNumeralsNouns} has \isi{nouns} as the second part and additionally what looks like unproductive derivational suffixes (\textit{-lan, -ar, -an}).\footnote{The suffixes -\textit{ar} and -\textit{an} both contain the same vowel \textit{a}, which is identical to the vowel in oblique plural forms of \isi{nouns} (e.g. \textit{tʼuˁ-m-a-lla} leg-\tsc{pl-obl-gen}) such that one could perhaps suggest that the vowel is actually not part of the suffix. However, \isi{oblique stem} forms of \isi{nouns} are normally only used for case formation and we would need an additional motivation for using the oblique form as the basis of derivational processes. Both suffixes are also used for the \isi{derivation} of \isi{adjectives}, see \refsec{sec:Derivation of adjectives}.} The base \isi{nouns} are usually in the plural, which indicates that these are compound \isi{nouns} and not phrases. In noun phrases with numerals, \isi{nouns} normally occur in the singular. Furthermore, the numerals appear in their basic stem form that cannot be used independently, but only as the basis for other word formation processes. 

\begin{exe}
	\ex	\label{ex:CompoundNounsNumeralsNouns} 
	\begin{xlist}
		\TabPositions{2em,4em}
			\ex	\tit{azirt'uˁlan\slash azirtʼuˁmar}	\sqt{centipede}		\\
			\tab	<	\tab	 \tit{azir} \sqt{thousand} + \tit{t'uˁ-m-ar} (leg\tsc{-pl-nmlz})										
			\ex	\tit{azirkumran\slash azirk'apran}	\sqt{abomasum}		\tab	<	\tab	 \tit{azir} \sqt{thousand} + \tit{kam-r-an} (layer-\tsc{pl-nmlz})\slash\textit{kʼap-r-an} \\
			(leaf-\tsc{pl-nmlz})
			\ex	\tit{ʡaˁbk'apːrar q'ar}	\sqt{clover}		\tab	<	\tab	 \tit{ʡaˁb-k'apːr-ar} (three-leaf-\tsc{pl-nmlz}) + \tit{q'ar} \sqt{herbs}
			\ex	\tit{ʡaˁbqigar}	\sqt{three bundles of grain bound together}		\tab	<	\tab	\tit{ʡaˁb} \sqt{three} + ?	
			\ex	\tit{ʡaˁbt'uˁmar}	\sqt{tripod}		\tab	<	\tab	\tit{ʡaˁb} \sqt{three} + \tit{tʼuˁ-m-ar} (leg-\tsc{pl-nmlz})						
	\end{xlist}
\end{exe}	

The second group of compound \isi{nouns} with numerals has only two members \refex{ex:CompoundNounsNumerals2} that denote offspring born in a multiple birth. The first part is again the basic stem form of the numeral whereas the second part looks like a noun derived from a verb by means of an otherwise unattested suffix \textit{-i}.

\begin{exe}
	\ex	\label{ex:CompoundNounsNumerals2} 
	\begin{xlist}
		\TabPositions{9em,11em}					
			\ex	\tit{k'ʷidarq'i}	\sqt{twins}		\tab	<	\tab	 \tit{k'ʷi} \sqt{two} + \tit{d-arq'-i} (\tsc{npl}-make.\tsc{pfv-?}) 
			\ex	\tit{ʡaˁbdarq'i}	\sqt{triplet}			\tab	<	\tab	 \tit{ʡaˁb} \sqt{three} + \tit{d-arq'-i} (\tsc{npl}-make.\tsc{pfv-?}) 
	\end{xlist}
\end{exe}	


%---------------------------------------------------

\section{Phrasal compounds}
\label{ssec:N-genN}
Sanzhi has a very productive way of forming short noun phrases that have the structure of short participial clauses of the form (noun) + verb and semantically strongly resemble derived action \isi{nouns} (\refsec{ssec:Action and event nouns way of V-ing with -ala}, \refsec{ssec:Action and event nouns with -uti and -a}). In the minimal case a verb alone to which the modal/future \isi{participle} \textit{-an} is suffixed is enough. If the verb is transitive, it is often accompanied by a patientive argument. Morphosyntactically, these combinations are headless \is{relative clause}relative clauses with a transparent semantics (\refsec{sec:Headless relative clauses}), but the phrases are lexicalized and perceived as complex \isi{nouns} by the speakers. They denote agents, professions or tools.

	\begin{exe}
	\ex	\label{ex:AN-nouns} 
	\begin{xlist}
		\TabPositions{2em,4em}
	\ex \textit{umzan} \sqt{sieve, colander} <	 \textit{umz-an} filter-\textsc{ptcp}
\ex \textit{sarruˁrqaˁn} \sqt{kidnapper} \\
\tab	<	\tab \textit{sa-r-r-uˁrq-aˁn} \textsc{ante-abl-f}-drag.\textsc{ipfv-ptcp}
\ex \textit{paltar durχan}	\sqt{tailor}  \\
\tab	<	\tab \textit{paltar} \sqt{clothes} + \textit{d-urχ-an} \textsc{npl}-sew.\textsc{ipfv-ptcp}
\ex \textit{qʼʷaˁl icːan} \sqt{milkmaid} \tab	 <	\tab   \textit{qʼʷaˁl} \sqt{cow} + \textit{icː-an} milk.\textsc{ipfv-ptcp}
\ex \textit{ʡuˁnze sarirtʼan} \sqt{handkerchief} \\
\tab	 <	\tab  \textit{ʡuˁnze} \sqt{snot, slime} + \textit{sa-r-irtʼ-an} \textsc{ante-abl}-take.away.\textsc{ipfv-ptcp}
\ex \textit{sːurrat heltʼan} \sqt{photographer, camera} \\
\tab	 <	\tab  \textit{sːurrat} \sqt{picture} + \textit{ha-eltʼ-an} \textsc{up}-take.out.\textsc{ipfv-ptcp}
\ex \textit{ʁaj hadulqʼan} \sqt{intriguer} \\
\tab	 <	\tab    \textit{ʁaj} \sqt{word, language} + \textit{ha-d-ulqʼ-an} \textsc{up-npl}-lock.\textsc{ipfv-ptcp}
\ex \textit{kːʷiš iʁan} \sqt{dough scraper} \\
\tab	 <	\tab   ?\textit{kːʷiš} \sqt{dough}\footnote{The word \textit{kːʷiš} is not the regular word for \sqt{dough} in Sanzhi, but it exists in the derived noun \textit{kːʷiš-a} \sqt{wooden board for making dough}.} + \textit{iʁ-an} chase.\textsc{ipfv-ptcp}
\ex \textit{pal kerxʷan} \sqt{fortune-teller} \\
\tab	 <	\tab   \textit{pal} \sqt{prediction}\footnote{The word \textit{pal} does not seem to be used in Sanzhi, but it is part of the derived noun \textit{pal-či} \sqt{fortune-teller}} + \textit{ka-erxʷ-an} \textsc{down}-pour.\textsc{ipfv-ptcp}
 	\end{xlist}
\end{exe}


There is a range modifier + noun combinations that have idiomatic meanings that are not transparently predictable from the meanings of the parts, but syntactically are rather phrases and not compound \isi{nouns}. The modifiers used are \isi{nouns} in the \isi{genitive} case \refex{ex:qːaˁqːaˁshook3} or \isi{adjectives} \refex{ex:ADJNOund}. Sometimes the origin of \isi{nouns} used as modifiers in these combinations is unclear. Most examples denote animals or plants. In fact, sometimes names for herbs, healing plants or other edible plants seem to be made up on the spot and are rather descriptive.

\begin{exe}
	\ex	\label{ex:qːaˁqːaˁshook3}
	\begin{xlist}
		\TabPositions{13em,15em}
		\ex	\tit{ħaˁž-i-la žatːa} \sqt{swallow}	\tab	<	\tab	 Hajj-?-\textsc{gen} + swallow
		\ex	\tit{jarma-la č'imi} \sqt{squirrel}		\tab	<	\tab	 ear?-\textsc{gen} + tail
				\ex	\tit{alax-la q'ʷaˁl} \sqt{ladybird}		\tab	<	\tab	 Allah?-\textsc{gen} + cow
		\ex	\tit{kːaˁta-la maˁmre} \sqt{blackberry}	\tab	<	\tab	 cat-\textsc{gen} + female breast
		\ex	\tit{kːaˁta-la q'ar} \sqt{valeriana}	\tab	<	\tab	 cat-\textsc{gen} + herbs
		\ex	\tit{šajt'an-na q'ap'a} \sqt{mushroom}	\tab	<	\tab	 devil-\textsc{gen} + hat
		\ex	\tit{ħaˁž-la qːara} \sqt{bean}		\tab	<	\tab	 Hajj-\textsc{gen} + pea
		\ex	\tit{birikːa-lla ʁut'} \sqt{cow-parsnip}	\tab	<	\tab	 cow-parsnip-\textsc{gen} + edible root
		\ex	\tit{čaˁj-la q'ar} \sqt{Saint John's wort}	\tab	<	\tab tea-\textsc{gen} + herbs
		\ex	\tit{daˁqaˁ-lla q'ar} \sqt{plantain}	\tab	<	\tab wound-\textsc{gen} + herbs
		\ex	\tit{p'aˁlc'ik'ʷ-la q'ar} \sqt{linen}	\tab	<	\tab nut butter-\textsc{gen} + herbs (\textit{urbech})
		\ex	\tit{ca-t'uˁ-la xːun} \sqt{path}		\tab	<	\tab	 one-leg-\textsc{gen} + way
		\ex	\tit{ul-la hin} \sqt{healing spring}		\tab	<	\tab eye-\textsc{gen} + water
		\ex	\tit{suʁat-la hin} \sqt{mortar, whitewash}	
		\sn	~\hspace*{1em}	 \tab	<	\tab lime-\textsc{gen} + water	 
	\end{xlist}
\end{exe}

	\begin{exe}
	\ex	\label{ex:ADJNOund} 
	\begin{xlist}
		\TabPositions{12em,14em}	
		
		\ex	\tit{bicːi mura} \sqt{nut grass}		\tab	<	\tab	\textit{bicːi} \sqt{tasty} + \textit{mura} \sqt{grass, hay}
		\ex	\tit{buχːari qati} \sqt{papakha}		\tab	<	\tab	 \tit{buχːar} \sqt{cold} +\tit{qati} \sqt{hat}
		\ex	\tit{k'ant'i nisːe} \sqt{cottage cheese}	\tab	<	\tab	 \tit{k'ant'i} \sqt{soft} + \tit{nisːe} \sqt{cheese}
		\ex	\tit{χːula barne} \sqt{holiday at the end of Ramadan}
		\sn	~\hspace*{1em}	\tab	<	\tab	 \tit{χːula} \sqt{big, old} + \textit{barne} \sqt{days}
	\end{xlist}
\end{exe}




%
%\begin{exe}
	%\ex	\label{ex:puχ diducniconstipation}
	%\begin{xlist}
		%\TabPositions{14em,16em}
		%\ex	\tit{puχ diducni} \sqt{constipation}	
		%\sn	~\hspace{2em}<\hspace{1em}	\tit{puχ d-i-d-uc-ni} excrement \textsc{npl-in}-\textsc{npl}-keep.\textsc{pfv}-\textsc{msd}
		%\ex	\tit{kːurcːum daqni} \sqt{earthquake} 
		%\sn	~\hspace*{1em}\hfill(the base \isi{nouns} do not exist independently; the second noun has the form of the \isi{masdar}, i.e. of a deverbal noun, but I am not able to identify the verb stem)
		%\ex	\tit{luxːan qːabaʁ} \sqt{a kind of pumpkin}
		%\sn	~\hspace{2em}<\hspace{1em}	\tit{luxː-an qːabaʁ} cook.\textsc{ipfv}-\textsc{ptcp} pumpkin
		%\ex	\tit{dircan cik'al} \sqt{article of trade}
		%\sn	~\hspace{2em}<\hspace{1em}	\tit{d-irc-an cik'al} \tsc{npl}-sell.\textsc{ipfv}-\textsc{ptcp} thing
		
		%\ex	\tit{sːurrat helt'an} \sqt{camera}
		%\sn	~\hspace{2em}<\hspace{1em}	\tit{sːurrat h-elt'-an} picture \textsc{up}-take.out.\textsc{ipfv}-\textsc{ptcp}
	%\end{xlist}
%\end{exe}





	
