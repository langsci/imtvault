\chapter{Constituent order and information structure}
\label{cpt:Constituent order and information structure}

This chapter addresses \isi{constituent order} at the phrase level, in particular within the \isi{noun phrase} (\refsec{sec:Constituent order at the phrase level}); \isi{constituent order} at the clause level in main clauses and subordinate clauses; as well as the information-structural patterns that are associated with certain orders (\refsec{sec:Constituent order at the clause level and information structure}). It also provides a short overview of other ways of manipulating the \isi{information structure}, most notable term focus\is{focus} constructions and cleft-like constructions and focus-sensitive \is{focus-sensitive particle}particles (\refsec{sec:Other types of focus constructions}--\refsec{sec:Focus-sensitive particles}).


%%%%%%%%%%%%%%%%%%%%%%%%%%%%%%%%%%%%%%%%%%%%%%%%%%%%%%%%%%%%%%%%%%%%%%%%%%%%%%%%

\section{Constituent order at the phrase level}
\label{sec:Constituent order at the phrase level}

\subsection{General remarks on the constituent order at the phrase level}
\label{ssec:General remarks on the constituent order at the phrase level}

Noun phrases are strictly head-final. A template of the structure of noun phrases and the internal order of modifiers is given in \refsec{ssec:The structure and order of constituents within the noun phrase}. Noun phrases can be quite complex, but mostly they have between zero and two modifiers. Example \refex{ex:‎‎‎I go to my woman who I know} illustrates a \isi{noun phrase} consisting of a demonstrative, a \isi{genitive} pronoun, a short \isi{relative clause}, and the head noun. All modifiers occur in the order which is most common in the Sanzhi corpus (i.e. demonstrative before \isi{genitive} before \isi{relative clause}).

\begin{exe}
	\ex	\label{ex:‎‎‎I go to my woman who I know}
	\gll	het	di-la	[r-alχ-an]	xːunul-li-šːu	r-ax-ul=da\\
		that	\tsc{1sg-gen}	\tsc{f-}know\tsc{.ipfv-ptcp}	woman\tsc{-obl-ad}	\tsc{f-}go\tsc{-icvb=1}\\
	\glt	\sqt{‎‎‎I go to my woman who I know.}
\end{exe}

Within postpositional phrases and adjective phrases, the \isi{word order} is also head-final. Thus, adverbial modifiers always precede and never follow \isi{adjectives} \refex{ex:That is a very good medicine}, and postpositions always follow their complements \refex{ex:He is sitting in front of the house}. However, since some postpositions also occur as adverbs, one might come across examples that seem to contradict this claim because they contain postpositions used adverbially without a complement, or with what seems to be a postpositional complement, but does not occur in the expected position (see \refsec{sec:Postpositional phrases} for example \refex{ex:He stayed in prison}).\largerpage[2]

\begin{exe}
	\ex	\label{ex:That is a very good medicine}
	\gll	c'aq'-le	ʡaˁħ	darman	ca-b	hel\\
		very\tsc{-advz}	good	medicine	\tsc{cop-n}	that\\
	\glt	\sqt{That is a very good medicine.}

	\ex	\label{ex:He is sitting in front of the house}
	\gll	ka-jž-ib	ca-w	qal-la	sala\\
		\tsc{down}-remain\tsc{.m.pfv-pret}	\tsc{cop-m}	house\tsc{-gen}	front\\
	\glt	\sqt{He is sitting in front of the house.}
\end{exe}

All modifiers except for \is{demonstrative pronoun}demonstrative pronouns and numerals occasionally occur in positions detached from the \isi{noun phrase} and in such cases they syntactically do not belong to the \isi{noun phrase} anymore, but form a separate constituent (\refsec{ssec:The structure and order of constituents within the noun phrase}). I will use the term \dqt{floating} for these items. In the Sanzhi corpus, \is{floating modifier}floating modifiers sometimes immediately follow the head noun (\refex{ex:when I gave birth to my daughter} below), but mostly they occur after the verb \refex{ex:There was a bad road in our (area)}. There are only few examples of \is{floating modifier}floating modifiers in positions to the left of the \isi{noun phrase} (see \refsec{ssec:Floating modifiers} for one example with a \isi{genitive} and \refsec{ssec:Extraposed adjectives, postpositional phrases, and relative clauses} below for two examples with \isi{adjectives}). In general, the position after the verb can have special properties with regard to its contribution to the \isi{information structure} of the sentence (\refsec{sec:Constituent order at the clause level and information structure}), and it seems that when modifiers are extraposed to this position they often receive a contrastive interpretation. This is not unique for \is{floating modifier}floating modifiers, but also regards other constituents that occur after the verb. 

Floating modifiers in the East Caucasian languages have been analyzed by a \isi{number} of researchers. \citet{Kazenin2009} investigates their syntactic structure in Lak and the question of whether they form one constituent with the \isi{noun phrase} or not (see \refsec{ssec:Floating modifiers}). Other authors such as \citet{Testelec1998a}, \citet{Creissels2013}, \citet{Komen2014}, and \citet{Lander2014, Lander2016} treat their impact on the \isi{information structure}, in particular contrastivity. In the following two sections, I will present the accounts that can be found in the literature and discuss their applicability to Sanzhi.

% --------------------------------------------------------------------------------------------------------------------------------------------------------------------------------------------------------------------- %

\subsection{Floating genitives}
\label{ssec:Extraposed genitives}

The most common modifiers occurring in an extraposed position are \isi{genitive} modifiers \xxref{ex:There was a bad road in our (area)}{ex:when I gave birth to my daughter}. In the majority of cases, these postponed genitives are personal pronouns or demonstratives used as personal pronouns. This means that their referents are human and highly topical. As was just mentioned, the floating \isi{genitive} most frequently follows the verb \xxref{ex:There was a bad road in our (area)}{ex:‎We had a good conversation at that time}. However, genitives occupying the position directly after the noun are also attested. In \refex{ex:when I gave birth to my daughter} this can be explained by the fact that the \isi{genitive} phrase is part of a subordinate clause for which verb-final \isi{constituent order} is strongly preferred (\refsec{sec:The syntax of adverbial clauses}).

\begin{exe}
	\ex	\label{ex:There was a bad road in our (area)}
	\gll	wahi-ce	xːun	b-irχʷ-i	nišːa-lla\\
		bad\tsc{-dd.sg}	road	\tsc{n-}be\tsc{.ipfv-hab.pst}	\tsc{1pl-gen}\\
	\glt	\sqt{There was a bad road in our (area).}

	\ex	\label{ex:(She) is the daughter of my cousin}
	\gll	rucːiq'ar-ra	rursːi	ca-r	di-la\\
		first.cousin\tsc{-gen}	girl	\tsc{cop-f}	\tsc{1sg-gen}\\
	\glt	\sqt{(She) is the daughter of my cousin.}

	\ex	\label{ex:‎We had a good conversation at that time}
	\gll	ca	ʡaˁħ	iχtilat	ag-ur	nišːa-lla	hel	zamana\\
		one	good	talk	go\tsc{.pfv-pret}	\tsc{1pl-gen}	that	time\\
	\glt	\sqt{‎We had a good conversation at that time.}

	\ex	\label{ex:when I gave birth to my daughter}
	\gll	rursːi	di-la	r-irq'-an=qːel ...\\
		girl	\tsc{1sg-gen}	\tsc{f-}do\tsc{.ipfv-ptcp=}when\\
	\glt	\sqt{when I gave birth to my daughter ...}
\end{exe}

There are various explanations for why \isi{genitive} modifiers can follow the head noun, but contrastiveness does not seem to be the best one. \citet{Creissels2013} analyzes floating genitives in the East Caucasian language Akhvakh. Like in Sanzhi, the floating genitives predominantly follow the verb, denote human referents, and occur in one of the three functions that are also attested for prenominal genitives:

\begin{itemize}
	\item	person-body parts
	\item	person-relatives \refex{ex:(She) is the daughter of my cousin}, \refex{ex:when I gave birth to my daughter}, \refex{ex:‎My mother had seven cousins}, or other social relations \refex{ex:‎At the time when I run to pull his legs, I fell (lit. hit) directly on the lap of our teacher}
	\item	person-objects, including abstract objects \refex{ex:There was a bad road in our (area)}, \refex{ex:‎We had a good conversation at that time}
\end{itemize}

As in Akhvakh, possessive predications in Sanzhi, i.e. \isi{copula} clauses with nominals in the \isi{genitive} functioning as predicates, can be treated as a subtype of floating genitives when they have \isi{genitive} nominals that follow the \isi{copula} \refex{ex:‎My mother had seven cousins}.

\begin{exe}
	\ex	\label{ex:‎My mother had seven cousins}
	\gll	hetːi	weral	ucːiq'ar	le-b=de	di-la	aba-la\\
		those	seven	cousin	exist\tsc{-hpl=pst}	\tsc{1sg-gen}	mother\tsc{-gen}\\
	\glt	\sqt{‎My mother had seven cousins.}
\end{exe}

In contrast to Akhvakh, however, the floating \isi{genitive} in Sanzhi can also have a head noun in a case other than the \isi{absolutive}, though only one such example has been found so far in the Sanzhi corpus. Thus, example \refex{ex:‎At the time when I run to pull his legs, I fell (lit. hit) directly on the lap of our teacher} contains the postponed \isi{genitive} \tit{nišːala} \sqt{our}, which functions as possessor of the noun \tit{učitilla} \sqt{teacher\tsc{.gen}}. Furthermore, the sentence contains a \isi{relative clause} that follows its head noun \tit{zamana} \sqt{time}.

\begin{exe}
	\ex	\label{ex:‎At the time when I run to pull his legs, I fell (lit. hit) directly on the lap of our teacher}
	\gll	hel	zamana	[duc'	uq-un-il	du	[heχ-i-la	t'uˁ-me	gu-r-sa-d-ertː-ij]],	hek'	učitil-la	w-arx-le	nišːa-la	[\ldots]	dukla	w-i-w-aˁq-ib=da\\
		that	time	run	go\tsc{.m.pfv-pret-ref}	\tsc{1sg}	\tsc{dem.down}\tsc{-obl-gen}	leg\tsc{-pl}	\tsc{sub-abl-hither}\tsc{-npl-}tear\tsc{.pfv-inf}	\tsc{dem.up}	teacher\tsc{-gen}	\tsc{m-}direct\tsc{-advz}	\tsc{1pl-gen} {}	onto.lap	\tsc{m-in}\tsc{-m-}strike\tsc{.pfv=pret=1}\\
	\glt	\sqt{‎At the time when I (masc.) ran to pull down his legs, I fell (lit. hit) directly on the lap of our teacher.}
\end{exe}

\citet[346]{Creissels2013} describes the semantic properties of floating genitives for Akh\-vakh by noting that postposed genitives have an \dqt{empathy effect} and \dqt{consider the situation from the point of view of the possessor.} He writes that the floating \isi{genitive} construction \dqt{has a possessive framing function, in the sense that the floating \isi{genitive} identifies the personal sphere of its referent as the frame within which the predication expressed by the clause holds} \citep[333]{Creissels2013}. He further compares them to other framing adjuncts such as spatial and temporal expressions and external possessors. The account given by \citea{Creissels2013} fits the Sanzhi data well. Like the genitives in Akhvakh, floating genitives in Sanzhi denote affected participants similar to beneficiaries or maleficiaries that are not functioning as arguments, but whose referents are either strongly involved in the situation expressed by the verb and/or are in physical proximity to that situation (see also \citealp{Shibatani1994} and \citealp{Serzhant2016} on external possessor constructions). Thus, the floating genitives can often be interpreted as referring to a location \refex{ex:There was a bad road in our (area)}, and it is relatively common for adjuncts denoting locations to follow the verb in a \isi{locational copula} construction (\refsec{ssec:Location, existence, and possession}). The following minimal pair illustrates the difference between preposed and postposed \isi{genitive} modifiers \refex{ex:sheep eat hay minimal pair}.

\begin{exe}
	\ex	\label{ex:sheep eat hay minimal pair}
	\begin{xlist}
		\ex	\label{ex:‎My five sheep ate all the hay@A}
		\gll	heštːi	di-la	xujal	macːa-l	li<d>il	mura	d-erkː-un\\
			these	\tsc{1sg-gen}	five	sheep\tsc{-erg}	all\tsc{<npl>}	hay	\tsc{npl-}eat\tsc{.pfv-pret}\\
		\glt	\sqt{‎My five sheep ate all the hay.} (E)

		\ex	\label{ex:‎The five sheep of mine ate all the hay}
		\gll	heštːi	xujal	macːa-l	di-la	li<d>il	mura	d-erkː-un\\
			these		five	sheep\tsc{-erg}	\tsc{1sg-gen} all\tsc{<npl>}	hay	\tsc{npl-}eat\tsc{.pfv-pret}\\
		\glt	\sqt{‎The five sheep of mine ate all the hay.} (E)
	\end{xlist}
\end{exe}

In general, preposed genitives are definite \refex{ex:‎(He) is my (real) grandfather. For you he is only an old man extraposed genitive}.

\begin{exe}
	\ex	\label{ex:‎(He) is my (real) grandfather. For you he is only an old man extraposed genitive}
	\gll	di-la	χatːaj	ca-w,	ala	itwaj	χatːaj	ca-w\\
		\tsc{1sg-gen}	grandfather	\tsc{cop-m}		\tsc{2sg.gen}	like.that	grandfather	\tsc{cop-m}\\
	\glt	\sqt{‎(He) is my (real) grandfather. For you he is only an old man.} (lit. He is like a grandfather of yours.)
\end{exe}

In the following example \refex{ex:the house filled well with people} with the verb \tit{b-ic'-} (\tsc{pfv}) \sqt{fill}, the noun denoting with what the house is filled has human reference and bears the \isi{genitive} case. Although the \isi{genitive} in \refex{ex:the house filled well with people} cannot be analyzed as a modifier of the preceding noun \tit{qal} \sqt{house}, the construction expresses the affectedness and the involvement of the referent of the noun \tit{χalq'} \sqt{people} similar to the other extraposed genitives discussed so far. By contrast, if the filler is inanimate, the \isi{ergative} case has to be used instead of the \isi{genitive}. The \isi{ergative} also denotes instruments, and instruments are, in general, not affected by an action or a situation (\refsec{sssec:Ergative}).

\begin{exe}
	\ex	\label{ex:the house filled well with people}
	\gll	ʡaˁħ-le	qal	b-ic'-ib	χalq'-la\\
		good\tsc{-advz}	house	\tsc{n-}fill\tsc{.pfv-pret}	people\tsc{-gen}\\
	\glt	\sqt{the house filled well with people}
\end{exe}

Similarly, sentence \refex{ex:‎‎‎(She) did not put the two of us like this} shows a personal pronoun in clause-final position following the verb. This pronoun can be considered an \isi{floating modifier} of the quantified \isi{noun phrase} \tit{k'ʷel admi} \sqt{two persons} in clause-initial position. Again it has a human referent that is depicted as being affected by the situation.

\begin{exe}
	\ex	\label{ex:‎‎‎(She) did not put the two of us like this}
	\gll	k'ʷel admi	itwaj	ka-d-at-ur-te	akːʷa-di	nušːa\\
		two	person	like.this	\tsc{down-1/2pl-}let\tsc{.pfv-pret-dd.pl} 	\tsc{cop.neg-1}	\tsc{1pl}\\
	\glt	\sqt{‎‎‎(She) did not put the two of us like this (i.e. in vain).}
\end{exe}

In a few cases the use of floating genitives may have other pragmatic reasons. One factor is probably to avoid interpretative ambiguities. If the \isi{noun phrase} contains other nominals as modifiers, then the modifiers of the head noun that precede the \isi{genitive} could be interpreted as modifiers of the \isi{genitive} (see \refsec{ssec:The structure and order of constituents within the noun phrase} for more examples) instead of belonging to the head noun. In some examples the postponed \isi{genitive} might be a kind of afterthought, providing more information about the referent. This could be the case in the following example \refex{ex:the hat of him, of the boy}.

\begin{exe}
	\ex	\label{ex:the hat of him, of the boy}
	\gll	hel	qːatːi	hel-i-la,	hel	durħuˁ-la\\
		that	hat	that\tsc{-obl-gen}	that	boy\tsc{-gen}\\
	\glt	\sqt{the hat of him, of the boy}
\end{exe}

Finally, contrast might occasionally be a reason when the extraposed noun is inanimate and cannot be analyzed as an affected participant of the situation. In the dialogue from which \refex{ex:She said, come now in August, to prescribe the cure of injections} was taken, the speaker is talking about two types of medical treatments that were proposed to her, and she perhaps uses a postponed \isi{genitive} in order to express contrast.

\begin{exe}
	\ex	\label{ex:She said, come now in August, to prescribe the cure of injections}
	\gll	``hana	awgust-le-r	sa-r-eʁ-e,''	r-ik'ʷ-ar,	``lečenie	b-elk'-ij ukul-t-a-lla!''\\
		now	August\tsc{-loc-f}	\tsc{hither-f-}go\tsc{.pfv-imp}	\tsc{f-}say\tsc{.ipfv-prs}	treatement	\tsc{n-}write\tsc{.pfv-inf}	injection\tsc{-pl-obl-gen}\\
	\glt	\sqt{She said ``Come now in August, to prescribe the treatment with injections!''}
\end{exe}


% --------------------------------------------------------------------------------------------------------------------------------------------------------------------------------------------------------------------- %

\subsection{Floating adjectives, postpositional phrases, and relative clauses}
\label{ssec:Extraposed adjectives, postpositional phrases, and relative clauses}

Floating \isi{adjectives}, postpositional phrases, and \is{relative clause}relative clauses are less frequent than floating genitives, but they are also attested in the corpus. They obligatorily bear the \isi{cross-categorical suffix} \tit{-ce} (\isi{adjectives}, postpositional phrases, and occasionally \is{relative clause}relative clauses, \refsec{ssec:The -ce / -te attributive}) or the \isi{cross-categorical suffix} \tit{-il} (\is{relative clause}relative clauses, \refsec{ssec:The -il attributive}), which both are used for nominalizations. As with genitives, the \is{floating modifier}floating modifiers often do not follow the noun immediately, but occur after the verb. In most examples they occur in \isi{copula} constructions, but other clause types can also be found.

For other East Caucasian languages, particularly Avar-Andic and Tsezic languages, \citet{Testelec1998a} claims that \dqt{modifiers [\ldots] may be postponed only under contrast.} This statement and the examples, which follow it, suggest that it is a bi-\isi{conditional} connection, to the extent that every modifier that is postponed is necessarily contrastive. Similar claims have also been made for Tanti Dargwa (\citealt{Lander2014, Lander2016}, \citealt{Sumbatova.Lander2014}) but with the caveat that in Tanti Dargwa modifiers bearing a cognate of the Sanzhi suffix \tit{-ce} are said to be always restrictive, but not necessarily contrastive, and that contrastive modifiers are less likely to follow the head noun.

The Sanzhi sentences containing \is{floating modifier}floating modifiers seem to support the hypothesis that those modifiers normally receive a restrictive interpretation, which can be contrastive, but does not have to be. Thus, in \refex{ex:‎There was also a river there that was small}, the speaker adds more information about the river without contrasting it to other rivers. Similarly, the stories mentioned in \refex{ex:About them there were, are many stories, interesting@19c} are not contrasted to other stories that were not interesting. The man in \refex{ex:and then (he) saw the man on the tree@19c} is also not contrasted to another man that was not on the tree, but the speaker simply mentions a feature of the man that unambiguously identifies him. The fact that the man is standing in the crown of a tree is encoded by means of a postpositional phrase that modifies the noun \textit{admi} `man' and is marked with -\textit{ce}.

\begin{exe}
	\ex	\label{ex:‎There was also a river there that was small}
	\gll	erk'ʷ=ra	le-b=de	nik'a-ce\\
		river\tsc{=add}	exist\tsc{-n=pst}	small\tsc{-dd.sg}\\
	\glt	\sqt{‎There was also a river there that was small.}

	\ex	\label{ex:About them there were, are many stories, interesting@19c}
	\gll	il-tː-a-la	d-aqil	χabur-te	k'e-d=de	ca-d,	{} interesni-te\\
		that\tsc{-pl-obl-gen}	\tsc{npl-}much	story\tsc{-pl}	exist.\tsc{up-npl=pst}	\tsc{cop-npl}	{} interesting\tsc{-dd.pl}\\
	\glt	\sqt{About them there were, are many stories, interesting.}

	\ex	\label{ex:and then (he) saw the man on the tree@19c}
	\gll	či-w-až-ib-le admi	kːalkːi-cːe-w	či-w-ce\\
		\tsc{spr-m-}see\tsc{.pfv-pret-cvb}	person	tree\tsc{-in-m}	on\tsc{-m-dd.sg}\\
	\glt	\sqt{and then (he) saw the man on the tree}
\end{exe}

Sometimes it is unclear if the \isi{floating modifier} has only a restrictive meaning or if the existence of another, contrasting referent is also implied. For instance, the shampoo in \refex{ex:There is shampoo for children, expensive@19c} is identified by means of its description as being expensive, but from the context it remains unclear whether it is also contrasted with other shampoos that are not expensive.

\begin{exe}

		\ex	\label{ex:There is shampoo for children, expensive@19c}
		\gll	detski	šampun	le-b	durqa-ce\\
			children's	shampoo	exist\tsc{-n}	expensive\tsc{-dd.sg}\\
		\glt	\sqt{There is shampoo for children, expensive.}
\end{exe}


Unambiguous examples of contrastive postposed modifiers can be found. In \refex{ex:‎‎I took one, a big one} the speaker compares the watermelon that he bought with the other watermelons that were sold but were smaller, and \refex{ex:Then we make khinkal, the thin one} contrasts the thin khinkal to other types of khinkal. In \refex{ex:Then we make khinkal, the thin one} the contrastive interpretation results from the postnominal position, not from the use of the suffix \tit{-ce}. 

\begin{exe}

	\ex	\label{ex:‎‎I took one, a big one}
	\gll	ca	hel	s-asː-ib=da	χːula-ce\\
		one	that	\tsc{hither}-buy\tsc{.pfv-pret=1}	big\tsc{-dd.sg}\\
	\glt	\sqt{‎‎I took one, the big one.} 

		\ex	\label{ex:Then we make khinkal, the thin one}
		\gll	c'il	χːink'-e	d-irq'-id	d-uk'ul-te\\
			then	khinkal\tsc{-pl}	\tsc{npl-}do\tsc{.ipfv-1.prs}	\tsc{npl-}thin\tsc{-dd.pl}\\
		\glt	\sqt{Then we make khinkal, the thin one.}
		
\end{exe}


It can be argued that the examples in \xxref{ex:‎There was also a river there that was small}{ex:and then (he) saw the man on the tree@19c} represent instances of right \isi{dislocation} in which the dislocated element is a nominalized expression conveying an afterthought that provides more information about the referent of the head noun. A similar case can be made for examples in which the nominalized \isi{adjectives} precede the \isi{nouns} to which they refer. In those sentences it is the nominal that is right-dislocated \xxref{ex:There turned out to be a similar tree@20a}{ex:This is a black shirt. And here this is also a black shirt@20b} (see also \refex{ex:‎We Sanzhi people have a story} in \refsec{ssec:Floating modifiers}).

\begin{exe}
	\ex	\label{ex:There turned out to be a similar tree@20a}
	\gll	miši-ce	b-už-ib	ca-b	urcul\\
		similar\tsc{-dd.sg}	\tsc{n-}stay\tsc{-pret}	\tsc{cop-n}	tree\\
	\glt	\sqt{There turned out to be a similar tree.}

	\ex	\label{ex:This is a black shirt. And here this is also a black shirt@20b}
	\gll	c'utːar	kːurtːi	ca-b.	heštːu-b=ra	c'utːar-ce	ca-b	kːurtːi\\
		black	shirt	\tsc{cop-n}	here\tsc{-n=add}	black\tsc{-dd.sg}	\tsc{cop-n}	shirt\\
	\glt	\sqt{This is a black shirt. And here this is also a black shirt.}
\end{exe}

Clear examples of floating \is{relative clause}relative clauses are even less frequently found. They mostly have the form of identificational \isi{copula} clauses and thus resemble \isi{cleft} constructions. Usually \is{relative clause}relative clauses contain a gap in the position of the head of the \isi{relative clause} (\refcpt{cpt:Relative clauses}), but the floating \is{relative clause}relative clauses can also be free \is{relative clause}relative clauses with a pronoun that is co-referential to a nominal outside of the \isi{relative clause}. The \is{relative clause}relative clauses are restrictive, although two of them have personal names functioning as heads. In examples \refex{ex:‎‎That is the Patimat who had come to us. (She is the daughter} and \refex{ex:One was the son of my Cibac, his name is Rasul, the one that lived in Sanzhi, do you know him} the speaker helps the addressee to identify the referents by giving more information about them. Since Patimat is the most common female name in Dagestan and Rasul is also a common name, the hearer cannot be expected to immediately know about whom the speaker is talking.

\begin{exe}
	\ex	\label{ex:‎This also and this also is probably the man who hit the woman on the jaw}
	\gll	iž=ra	het=ra,	het	ʡaˁχːuˁl	∅-iχʷ-ij	[xːunul-la	qajqaj-li-cːe	b-aˁq-ib-il]\\
		this\tsc{=add}	that\tsc{=add}	that	guest	\tsc{m-}be\tsc{.pfv-inf}	woman\tsc{-gen}	jaw\tsc{-obl-in}	\tsc{n-}hit\tsc{.pfv-pret-ref}\\
	\glt	\sqt{‎This also and this also is probably the man who hit the woman on the jaw.}

	\ex	\label{ex:‎‎That is the Patimat who had come to us. (She is the daughter}
	\gll	hel	Pat'imat	ca-r,	[hel	nišːi-šːu	ka-r-eʁ-ib-il]\\
		that	Patimat	\tsc{cop-f}	that	\tsc{1pl}-\tsc{ad} \tsc{down-f}-go\tsc{.pfv-pret-ref}\\
	\glt	\sqt{‎‎That is the Patimat who had come to us. (She is the daughter of her, of Aminat.)}

	\ex	\label{ex:One was the son of my Cibac, his name is Rasul, the one that lived in Sanzhi, do you know him}
	\gll	ca	di-la	C'ibac-la	durħuˁ=de,	Rasul	b-ik'-ul,	[hek'	Sːanži-w er	∅-iχ-ub-il],	w-alχ-atːe=w	\\
		one	\tsc{1sg-gen}	Cibac\tsc{-gen}	boy\tsc{=pst}	Rasul	\tsc{hpl-}say\tsc{.ipfv-icvb}	\tsc{dem.up}	Sanzhi\tsc{-m}	life	\tsc{m-}be\tsc{.pfv-pret-ref}	\tsc{m-}know\tsc{.ipfv-prs.2sg=q}\\
	\glt	\sqt{One was the son of my Cibac, his name is Rasul, the one that lived in Sanzhi, do you know him?}
\end{exe}

In \refex{ex:‎He gives respect to anybody who comes here, whom he knows} the head is an \isi{indefinite pronoun} from Russian that is followed by two \is{relative clause}relative clauses, restricting the reference of the pronoun. Note that the \isi{indefinite pronoun} functions as \isi{recipient} in the clause. It is a borrowing from Russian and unmarked for case (both in Sanzhi and in Russian), although a Sanzhi nominal in this position would have required \isi{dative} case. This is remarkable because it is one of the few corpus examples of \is{floating modifier}floating modifiers with a head noun in a position, which normally requires case marking (see also \refex{ex:‎At the time when I run to pull his legs, I fell (lit. hit) directly on the lap of our teacher} for another examples with a floating \isi{genitive}). Almost all examples of \is{floating modifier}floating modifiers discussed in this and other sections belong to \isi{nouns} that are in the \isi{absolutive} case (e.g. S and P arguments).

\begin{exe}
	\ex	\label{ex:‎He gives respect to anybody who comes here, whom he knows}
	\gll	ʡaˁħ-le	χːula	ħuˁrmat	b-irq'-u	lubuj		[hej-ka	sa-jʁ-ib-il]	[cin-na	w-alχ-an]\\
		good\tsc{-advz}	big	respect	\tsc{n-}do\tsc{.ipfv-prs}	any	this-\tsc{down}	\tsc{hither}-come\tsc{.pfv-pret-ref}	\tsc{refl.sg-gen}	\tsc{m-}know\tsc{.ipfv-ptcp}\\
	\glt	\sqt{‎He gives respect to anybody who comes here, whom he knows.}
\end{exe}

\citet{Komen2014} analyzes floating \is{relative clause}relative clauses in the East Caucasian language Chech\-en. Like in Sanzhi, floating (also called ``extraposed'' by him) \is{relative clause}relative clauses are rare. Furthermore, if they are restrictive then their head always occurs in the preverbal position, which is the focus\is{focus} position in Chechen. In Sanzhi, almost all examples are \isi{copula} clauses with the head of the \isi{relative clause} immediately preceding the \isi{copula}, a position which is normally used for focal items \refex{ex:‎‎That is the Patimat who had come to us. (She is the daughter}. However, in \refex{ex:‎He gives respect to anybody who comes here, whom he knows} the head follows the verb. Due to the lack of more examples it is impossible to clarify at the moment whether the condition that \citet{Komen2014} established for Chechen also holds for Sanzhi.


%%%%%%%%%%%%%%%%%%%%%%%%%%%%%%%%%%%%%%%%%%%%%%%%%%%%%%%%%%%%%%%%%%%%%%%%%%%%%%%%

\section{Constituent order at the clause level and information structure}
\label{sec:Constituent order at the clause level and information structure}

In this section, the \isi{constituent order} of clauses, including that of \is{interrogative clause}interrogative clauses, will be analyzed. I will also discuss how this order reflects the \isi{information structure} status of the constituents.

I adopt the definition of focus\is{focus} given by \citet{Diketal1981} as \dqt{what is relatively the most important or salient information in the given setting.} Focus is opposed to topic, which is defined as \dqt{the entity \sqt{about} which the predication predicates something in the given setting.} Sentences can be fully-focused when the entire sentence conveys the most important information. However, normally only a sub-part of the utterance carries focus, which means that we have \is{focus!constituent}constituent focus (or \dqt{term focus} in the terminology of \citealp{Diketal1981}). Constituent focus can be divided into (i) completive focus\is{focus!completive} and (ii) \is{focus!contrastive}contrastive focus. Completive \isi{focus} fills a gap in the pragmatic information of the addressee, as is the case with answers to content \isi{questions} (\refsec{sec:Content questions}). Contrastive focus constitutes a reply to the addressee's contrary belief, for instance by correcting and replacing it. \citea{Diketal1981} distinguish three subtypes of \is{focus!contrastive}contrastive focus:\largerpage

\begin{itemize}
	\item	selective
	\item	corrective (expanding, restricting, or replacing; \refsec{ssec:Corrections})\is{focus!corrective}
	\item	parallel (in parallel structures)
\end{itemize}

Predicate-centered \isi{focus}, in which the verb is in focus, represents a type of \is{focus!constituent}constituent focus that can also be either \is{focus!completive}completive (in answers to \isi{questions} about the kind of action that was carried out) or contrastive (e.g. when the truth value is at issue).

Topical information and topical referents are usually familiar in the given context, and as such they can be used as anchors for providing new information. Therefore, they are usually unmarked and often omitted from the utterance in Sanzhi Dargwa. However, topics can also be new or contrastive when they differ from the topic(s) in the previous utterances.

The third notion relevant for \isi{information structure} is contrast. It is independent of the notions ``topic'' and ``focus''\is{focus} (e.g. \citealt{Vallduvi.Vilkuna1998}; \citealt{Neeleman.etal2009}). Contrast is relational because there must be a relation between the contrasted item and at least one other identifiable alternative in the context, and both the contrasted item and the alternative must be explicitly verbally mentioned (\citealt{Malchukov2004}; \citealt{Izutsu2008}; \citealt{Repp2010}). The two items must be comparable to each other with respect to a shared domain. At the same time they are different and can therefore be contrasted with each other.


% --------------------------------------------------------------------------------------------------------------------------------------------------------------------------------------------------------------------- %

\subsection{Declarative clauses}
\label{ssec:Declarative clauses}

The \isi{constituent order} in main clauses is basically free, at least in the sense that every logically possible order can be found in texts and it can easily be elicited. The most frequent order is SOV, but SVO is also very common. Verb-initial orders (VSO and VOS) are rare, but not ungrammatical. For intransitive clauses we find SV and occasionally VS. In general, the \isi{constituent order} is heavily influenced by pragmatics and the \isi{information structure} of the utterance. Contextually retrievable arguments are frequently omitted. This includes subject-like arguments as well as object-like arguments. Therefore, many monovalent predicates in the corpus occur without a subject and most bivalent predicates have only one overt argument. For clauses with one (\is{monovalent verb}monovalent verbs) and two (\isi{bivalent verbs}) overt arguments, the tendencies for linking certain constituent orders with specific pragmatic values of the constituents listed in \reftab{tab:The relationship between constituent order and information structure in main clauses} can be observed. With the label ``neutral'', I refer to predicate \isi{focus}, which is commonly assumed to represent a universally unmarked type of focus in which the subject is the topic and the predicate is focused.

\begin{table}
	\caption{The relationship between constituent order and information structure in main clauses}
	\label{tab:The relationship between constituent order and information structure in main clauses}
	\small
	\begin{tabularx}{\textwidth}{lQ}
		\lsptoprule
			\isit{constituent order}		&	 common distribution of topic and \isit{focus}\\
		\midrule
			\multicolumn{2}{c}{{monovalent predicates}}\\\midrule
			SV		&	neutral or subject focus or presentational focus\\
			VS		&	fully-focused (presentational) or topicalizing of subject\\\midrule
			\multicolumn{2}{c}{{bivalent predicates}}\\\midrule
			SOV	&	neutral or object focus\\
			SVO	&	neutral or subject focus or occasionally presentational structure with object focus\\
			OVS	&	object focus (+ predicate focus) and\slash or subject topicalization (= S is a contrastive topic)\\
			OSV	&	subject focus + predicate focus and/or object topicalization (contrastive topic)\\
			VSO	&	verb focus and topical subject and topical object\\
			VOS	&	[too rare in texts]\\
		\lspbottomrule
	\end{tabularx}
\end{table}

The pragmatically neutral order, in which none of the constituents is particularly emphasized, is SV in intransitive clauses and SOV/SVO in transitive clauses, including in elicited sentences without any context. In such clauses, the subject usually represents given information, while the predicate alone or the predicate together with the object carries the new information.

\begin{exe}
	\ex	SV: neutral \isi{information structure}\label{ex:She is crying SV}\\\relax%
		[talking about a woman whose husband is taken away by the police]\\
	\gll	hel	r-isː-ul	ca-r\\
		that \tsc{f-}cry\tsc{-icvb}	\tsc{cop-f}\\
	\glt	\sqt{She is crying.}

	\ex	SOV: neutral \isi{information structure}\label{ex:‎My mother protected me (kept me dear) SOV}\\\relax
		[autobiographical narration about personal experiences]\\
	\gll	di-la	aba-l	du	jaħaˁra-l ha-r-iq'-un=da\\
		\tsc{1sg-gen}	mother\tsc{-erg}	\tsc{1sg}	dear\tsc{-advz}	\tsc{up-f-}bring.up\tsc{-pret=1}\\
	\glt	\sqt{‎My mother protected me (kept me dear).}

	\ex	SOV: neutral \isi{information structure}\label{ex:But he also, not in the last year but two years ago, made a mistake SOV}\\\relax
		[talking about a doctor in a neighbouring village]\\
	\gll	ik'-i-l=ra	irig	akːu=n	sala-b	dus	ʡaˁšibkːa	b-arq'-ib\\
		\tsc{dem.up}\tsc{-obl-erg=add}	last.year	\tsc{cop.neg=prt}	front\tsc{-hpl}	year	mistake	\tsc{n-}do\tsc{.pfv-pret}\\
	\glt	\sqt{But he also, not in the last year but two years ago, made a mistake.}

	\ex	SOV: object \isi{focus}\label{ex:(They) built it in the canyon, (they) apparently built a water mill SOV}\\\relax
		[narration about how Sanzhi people built a water mill on a mountain that turned out not to work there]\\
	\gll	qːatːa-b	b-arq'-ib-le,	hin-na	urχːab	b-arq'-ib	b-už-ib	ca-b\\
		canyon\tsc{-n}	\tsc{n-}do\tsc{.pfv-pret-cvb}	water\tsc{-gen}	mill	\tsc{n-}do\tsc{.pfv-pret}	\tsc{n-}stay\tsc{-pret} \tsc{cop-n}\\
	\glt	\sqt{(They) built it in the canyon, (they) apparently built a water mill.}\pagebreak

	\ex	SVO: neutral \isi{information structure}\label{ex:‎‎‎I like garlic, fried tomato sauce, and also well dried-meat SVO}\\\relax
		[finishing the description of how to prepare \textit{khinkal}]\\
	\gll	dam	b-ičː-aq-id	burt	sːerži	ʡaˁħ-le	b-erʁ-ub	dig=ra\\
		\tsc{1sg.dat} \tsc{n-}want\tsc{.ipfv-caus-1.prs} sour.cream	garlic	good\tsc{-advz} \tsc{n-}dry\tsc{.pfv-pret}	meat\tsc{=add}\\
	\glt	\sqt{‎‎‎I like garlic, sour cream, and also well dried-meat.}
\end{exe}

For a \isi{number} of SOV languages such as Urdu, Turkish, Armenian, Georgian (e.g. \citealt{ButtKing1996}, \citealt{Comrie1984},  \citealt{Testelec1998b}) and also Chechen \citep{Komen2007}, a very strong association between focus\is{focus} and the preverbal position has been observed. In many if not all East Caucasian languages, there is also a clear tendency for putting focused items immediately before the verb  (\citealt{Testelec1998a, Testelets.1998c, Forker.Belyaev2016}). In general, Sanzhi behaves alike, but focused constituents are not always and exclusively placed directly in front of the verb. The two neutral orders SOV and SVO can be used to focus\is{focus} the subject \refex{ex:‎‎In Ashti Rasul makes cures SVO} or the object \refex{ex:(They) built it in the canyon, (they) apparently built a water mill SOV}.


\begin{exe}
	\ex	SVO: subject focus\is{focus}\label{ex:‎‎In Ashti Rasul makes cures SVO}\\\relax
		[(talking about doctors and treatments) I go to the woman who I know.]\\
	\gll	Eštːa	Rasul-li	b-irq'-u	lečenie	ca	ca=qːel\\
		Ashti.\tsc{loc}	Rasul\tsc{-erg}	\tsc{n-}do\tsc{.ipfv-prs}	cure	one	one=when\\
	\glt	\sqt{‎In Ashti, Rasul makes treatments from time to time.}
\end{exe}

However, subject or object focus\is{focus} is mostly expressed by other constituent orders that cannot be considered pragmatically neutral. The order OVS has been noticed to be used when the object or the object together with the predicate is in focus\is{focus} (\citealt{Testelec1998a, Testelets.1998c, Forker.Belyaev2016}), and this is confirmed by the following Sanzhi examples \xxref{ex:What do you (masc.) say1}{ex:‎‎Then at this time in order to show his masculinity, while she had the child in her arms, he hit his wife OVS}. In \refex{ex:they are taking her husband, took him away OVS} the speaker uses two clauses to describe basically the same event, the arrest of her husband by the police. The first clause has the \isi{constituent order} OV and no overt subject, and the second clause has VS without an overt object.

\begin{exe}
	\ex	OVS: object focus\is{focus} \label{ex:What do you (masc.) say1}\\
	\gll	ce	∅-ik'-ul=de	u?\\
		what	\tsc{m-}say\tsc{.ipfv-icvb=2sg}	\tsc{2sg}\\
	\glt	\sqt{What do you (masc.) say?}
	
	\ex	OVS: object + predicate focus\is{focus}\label{ex:they are taking her husband, took him away OVS}\\\relax
		[talking about a woman holding her little son in her hands while her husband is arrested by the police]\\
	\gll	sub	uk-ul	ca-w,	kʷi-r-s-asː-ib-le	ix-tː-a-l ...\\
		husband	gather\tsc{.m.ipfv-icvb}	\tsc{cop-m}	\tsc{in.hands-abl-}\tsc{hither}-take\tsc{.pfv-pret-cvb}	\tsc{dem.up}\tsc{-pl-obl-erg}\\
	\glt	\sqt{(They) are taking her husband, they took him away from her hands, \ldots}

	\ex	OVS: object + predicate focus\is{focus}\label{ex:‎‎Then at this time in order to show his masculinity, while she had the child in her arms, he hit his wife OVS}\\\relax
	[‎‎Then at this time in order to show his masculinity, while she had the child in her arms, ...]\\
	\gll	xːunul-li-j	b-aˁq-ib	ca-b	hel-i-l\\
		woman\tsc{-obl-dat}	\tsc{n-}hit\tsc{.pfv-pret}	\tsc{cop-n}	that\tsc{-obl-erg}\\
	\glt	\sqt{He hit his wife.}
\end{exe}

OVS order can also be used when the subject is a contrastive topic, which, in principle, does not need to exclude the possibility of \is{focus}focusing the object. In \refex{ex:I had taken the five-litre canister of wine in the morning OVS} the object is an aboutness topic, in addition to the subject being a contrastive topic.

\begin{exe}
	\ex	OVS: contrastive topic (I vs. they)\label{ex:I had taken the five-litre canister of wine in the morning OVS}\\\relax
		[A: They had a five-liter canister of wine.\\\hphantom{[}B: Are you saying that you had the canister?]\\
	\gll	xujal	litru-la	kanister	čaˁʁir-la	b-alli	b-erqː-ib=da	du-l	čːaˁʡaˁl-la\\
		five	liter\tsc{-gen}	canister	wine\tsc{-gen}	\tsc{n-}together	\tsc{n-}carry\tsc{.pfv-pret=1}	\tsc{1sg-erg}	morning\tsc{-gen}\\
	\glt	\sqt{‎C: I had taken the five-liter canister of wine in the morning.}
\end{exe}

OSV \isi{constituent order} is not particularly common in texts, but its pragmatic value seems to be relatively clear. It is mainly used for topicalizing objects, in particular for contrastive topics \refex{ex:I will never forget you OSV}, \refex{ex:The frog he put into a can OSV}. It is also used to focus\is{focus} the subject together with the predicate \refex{The water spins it around.}, \refex{ex:Well, I know her; [that Salikhat who married Rasul}.


\begin{exe}
	\ex	OSV: topical object + predicate focus\is{focus}\label{ex:I will never forget you OSV}\\\relax
		[One student whom I took with me in the car said at the end of the journey]\\
	\gll	u	du-l	nikagda	qum.a.art-an=de\\
		\tsc{2sg}	\tsc{1sg-erg}	never	forget\tsc{.ipfv.neg-ptcp=2sg}\\
	\glt	\sqt{You, I will never forget!}

	\ex	OSV: object topicalization (contrastive topic)\label{ex:The frog he put into a can OSV}\\\relax
		[The boy had a dog and a frog.]\\
	\gll	ʡaˁt'a	il-i-l	b-i-ka-b-at-ur	ca-b\\
		frog	that\tsc{-obl-erg}	\tsc{n-in-down}\tsc{-n-}leave\tsc{.pfv-pret}	\tsc{cop-n}\\
	\glt	\sqt{The frog he put (into a can).}

\ex	OSV: subject + predicate focus\is{focus}\label{The water spins it around.}\\\relax
		[The water drops in that thing.]\\
	\gll	heχ	hin-ni	lus	b-irq'-u\\
		\tsc{dem.down} water-\tsc{erg} around \tsc{n}-do.\tsc{ipfv-prs}\\
	\glt	\sqt{The water spins it around.}\pagebreak
	
	\ex	OSV: object topicalization, subject + predicate focus\is{focus}\label{ex:Well, I know her; [that Salikhat who married Rasul}\\
	\gll	na	it	du-l	r-alχ-an=q'al\\
		now	that	\tsc{1sg-erg}	\tsc{f-}know\tsc{.ipfv-ptcp=mod}\\
	\glt	\sqt{Well, I know her [that Salikhat who married into my family.]}
\end{exe}

Although in principle two different verb-initial orders can be elicited, in texts only VSO is attested. As stated in \citet{Forker.Belyaev2016}, VSO consistently expresses verb focus\is{focus} in combination with topical subjects and topical objects \xxref{ex:‎I already told these stories VSO}{ex:They sit and drink wine.IS}. The first item in \refex{ex:and the dogs didn't hear his sound or whistling VSO}, \textit{t'am}, forms a \isi{compound verb} together with the following verbal lexeme and thus does not function as argument. Example \refex{ex:They sit and drink wine.IS} shows an \isi{antipassive} construction in which the subject appears as pronoun in the \isi{absolutive} and the object as ergative-marked noun 

\begin{exe}
	\ex	VSO: verb \isi{focus}, topical subject and object\label{ex:‎I already told these stories VSO}\\
	\gll	b-urs-ib=da	du-l	iltːi	χabur-te\\
		\tsc{n-}tell\tsc{.pfv-pret=1}	\tsc{1sg-erg}	those	story\tsc{-pl}\\
	\glt	\sqt{‎I already TOLD these stories.}

	\ex	VSO: verb \isi{focus}, topical subject and object\label{ex:‎‎The husband hit the wife VSO}\\\relax
		[Husband and wife fought and a scandal happened and]\\
	\gll	b-aˁq-ib	ca-b	sub-li	xːunul-li-j\\
		\tsc{n-}hit\tsc{.pfv-pret}	\tsc{cop-n}	husband\tsc{-erg}	woman\tsc{-obl-dat}\\
	\glt	\sqt{‎‎The husband HIT the wife.}

	\ex	VSO: verb \isi{focus}, topical subject and object\label{ex:and the dogs didn't hear his sound or whistling VSO}\\\relax
		[The boy began to whistle and to yell at the dogs in the village. The tree fell down,]\\
	\gll	t'am	d-aq'-ib-le=kːu	χu-d-a-j	il-i-la	xʷit'=ra\\
		sound	\tsc{npl-}hear\tsc{.pfv-pret-cvb=}\tsc{cop.neg}	dog\tsc{-pl-obl-dat}	that\tsc{-obl-gen}	whistle\tsc{=add}\\
	\glt	\sqt{and the dogs didn't hear his sound or whistling.}

	\ex	VSO: verb \isi{focus}, topical subject\label{ex:They sit and drink wine.IS}\\\relax
		[Then they went to drink with the money they made from the theft.]\\
	\gll	b-učː-ul	ka-b-iž-ib	ca-b	hetːi	čaˁʁir-li\\
		\tsc{hpl-}drink\tsc{.ipfv-icvb}	\tsc{down-hpl-}be\tsc{.pfv-pret}	\tsc{cop-hpl}	those	wine\tsc{-erg}\\
	\glt	\sqt{They sit and drink wine.} (lit. \sqt{Drinking they sit down with wine.})
\end{exe}

The answer to the following question illustrates verb \isi{focus} with an intransitive predicate and a topical subject \refex{ex:What are (they) doing They went for a walk, to relax}:

\begin{exe}
	\ex	VS: verb focus, topical subject	\label{ex:What are (they) doing They went for a walk, to relax}
	\begin{xlist}
		\ex	\label{ex:What are (they) doing Q}
		\gll	ce	b-irq'-ul=e?\\
			what	\tsc{n-}do\tsc{.ipfv-icvb=q}\\
		\glt	\sqt{‎Q: What are (they) doing?}

		\ex	\label{ex:They went for a walk, to relax A}
		\gll	šːatːir	tːura	b-uq-un ca-b	hex-tːi,		či-b-b-axʷ-araj\\
			walk	outside	\tsc{hpl-}go\tsc{.pfv-pret}	\tsc{cop-hpl}	\tsc{dem.up}\tsc{-pl}	\tsc{spr-hpl-hpl-}relax\tsc{.pfv-subj.3}\\
		\glt	\sqt{‎A: They went for a walk, to relax.}
	\end{xlist}
\end{exe}

To sum up what has been observed so far, we can state that focal arguments and adjuncts with various semantic functions and \is{grammatical role}grammatical roles most commonly precede the verb, occurring immediately before it.  The only regular exceptions are presentational sentences with newly introduced arguments, which follow the verb (\refsec{ssec:Thetic sentences and presentational constructions}). Topical constituents appear to the left of focal constituents, but contrastive topics also normally occur at the right edge of the clause. This is in line with the studies by \citet{Testelec1998a, Testelec1998b} and \citet{Forker.Belyaev2016}.

In Sanzhi, there is no clear tendency for the relative placement of \isi{direct object} (\isi{theme}~= T) vs. \isi{indirect object} (\isi{recipient}\slash \isi{goal} = G). Both G-T-V and T-G-V are found. Which order is chosen depends on the pragmatic value of the arguments within the \isi{information structure} of the utterance, and there does not seem to be a pragmatically neutral order. Thus, the G arguments in \refex{ex:‎‎From there he ran away, gave me his bag TGV}, \refex{ex:Do not put that thing on me!} seem to be contrastive, and together with the verb form part of the new information.  Moreover, like other arguments, T and G arguments can also occur after the verb.

\begin{exe}
	\ex	G-T-V\label{ex:‎‎He had also to give something to somebody, I don't know GTV}\\%
	\gll	hi-cːe-k'al	cik'al	lukː-an-te=de=w,	aχːu,	hel-i-l=ra\\
		who\tsc{.obl-in-indef}	something	give\tsc{.ipfv-ptcp-dd.pl=pst=q}	not.know that\tsc{-obl-erg=add}\\
	\glt	\sqt{‎‎(He) also had to give something to somebody, I don't know.}

	\ex	T-G-V\label{ex:‎‎From there he ran away, gave me his bag TGV}\\\relax
		[Did it die, he asked. No, I said. ‎‎From there he ran away.]\\
	\gll	sumk'a	di-cːe	b-ičː-ib\\
		bag	\tsc{1sg-in}	\tsc{n-}give\tsc{.pfv-pret}\\
	\glt	\sqt{(He) gave me his bag.}

	\ex	T-G-V\label{ex:Do not put that thing on me!}\\\relax
		‎[May your beloved stay alive, dear brother!]\\
	\gll	hana	hel	cik'al	dam	či-ma-sa-b-iršː-itːa!\\
		now	that	something	\tsc{1sg.dat}	\tsc{spr-proh-hither}\tsc{-n-}put\tsc{.ipfv-proh.sg}\\
	\glt	\sqt{Do not put that thing on me!} (i.e. that piece of work)
\end{exe}

Goal-like arguments (recipients, addressees), just like subjects and objects, most commonly occur before the verb (G-T-V, T-G-V) \xxref{ex:‎‎He had also to give something to somebody, I don't know GTV}{ex:Do not put that thing on me!}. However, they also seem to have a relatively high probability to follow the verb, which does not depend on their status within the \isi{information structure} of the utterance \refex{ex:‎His wife begged him, Do not go constituents}, \refex{ex:He said, Give him an empty cartridge!}. The tendency includes goal-like adjuncts (directional adverbials, and possibly also beneficiaries, see below). It has been observed for verb-final languages of other language families, most notably Western Iranian languages, but also Iraqi Turkmen and Azerbaijani spoken in the wider area (Anatolia, south Caucasus; see \citealp{Haig2014}). Since the postverbal placement does not extend to objects, it cannot be explained by Russian influence, but might be due to contact with Kumyk (Turkic) speakers. However, before we can attribute the use of postverbal goals in Sanzhi to the impact of Kumyk it needs to be clarified if Kumyk belongs to the Turkic languages with postverbal goals and if other East Caucasian languages that are not in contact with Kumyk do not have postverbal goals to the same extent as Sanzhi. An alternative explanation might resort to iconicity. The \isi{goal} is the spatial endpoint of the situation. Thus, a postverbal \isi{goal} is iconic in that the destination of the transfer follows the entity to be transferred (the T argument) and it also follows the action expressed by the predicate.

\begin{exe}
	\ex	\label{ex:‎His wife begged him, Do not go constituents}
	\gll	xːunul-li	tiladi	b-arq'-ib	ca-b	hel-i-cːe	``ma-ax-utːa!''	r-ik'-ul\\
		woman\tsc{-erg}	request	\tsc{n-}do\tsc{.pfv-pret}	\tsc{cop-n}	that\tsc{-obl-in}	\tsc{proh-}go\tsc{-proh.sg}	\tsc{f-}say\tsc{.ipfv-icvb}\\
	\glt	\sqt{‎His wife begged him ``Do not go!''}

	\ex	\label{ex:He said, Give him an empty cartridge!}
	\gll	``xalastuj	pat'run	b-ikː-a,''	∅-ik'ʷ-ar,	``hel-i-j!''\\
		empty	cartridge	\tsc{n-}give\tsc{.pfv-imp}	\tsc{m-}say\tsc{.ipfv-prs}	that\tsc{-obl-dat}\\
	\glt	\sqt{``Give him an empty cartridge!'', he said.}
\end{exe}

Adjuncts like \isi{comitative} noun phrases, instruments, and manner adverbials are most frequently positioned after the subject, if there is one, and before the verb, but sometimes they can be found after the verb \refex{ex:I had taken the five-litre canister of wine in the morning OVS}, \refex{ex:‎At that time he embellished his story even more}, \refex{ex:‎‎‎They made injections into the hand for ten days}. 

Temporal and locational adjuncts normally occur at the beginning of clauses and precede any arguments and other adjuncts; especially the short \is{temporal adverb}temporal adverbs \tit{ha} and \tit{hana} \sqt{now} \refex{ex:She said, come now in August, to prescribe the cure of injections}, \refex{ex:Do not put that thing on me!}, \refex{ex:‎At that time he embellished his story even more}, and locational adverbs such as \textit{heštːu} \sqt{here} and \textit{hetːu} \sqt{there} have a strong tendency to occur clause-initially. However, they can be placed postverbally when representing new information or when they are contrastive. By contrast, directional adjuncts are goal-like and behave similarly to goal-like arguments \refex{ex:‎‎‎They made injections into the hand for ten days}. Though a position left of the verb at the beginning of the clause is common for directional adjuncts, postverbal placement is roughly equally common, not only when they encode new information. This behavior is part of a general tendency for all goal-like arguments and adjuncts, and can be explained by means of iconicity.

Examples of temporal, locational, and directional adverbs are \refex{ex:‎‎In Ashti Rasul makes cures SVO}, \refex{ex:‎At that time he embellished his story even more}, and  \refex{ex:‎Now such customs as here were not there, Rasul}. If adjuncts are \is{focus}focused, they immediately precede the verb \refex{ex:‎‎‎They made injections into the hand for ten days}, \refex{ex:‎With Tawlu always such stories happen}.

\begin{exe}
	\ex	\label{ex:‎At that time he embellished his story even more}
	\gll	hel	zamana	il-i-l	hati=ra	qːuʁa-l	d-al d-ič-aq-ib-le, \ldots\\
		that	time	that\tsc{-obl-erg}	more\tsc{=add}	beautiful\tsc{-advz}	\tsc{npl-}together	\tsc{npl-}occur\tsc{.pfv-caus-pret-cvb}\\
	\glt	\sqt{‎At that time he embellished (his story) even more, \ldots}

	\ex	{[A. She came to learn about the Sanzhi customs. B answers:]}\label{ex:‎Now such customs as here were not there, Rasul}\\%
	\gll	hana	heštːu-d	ʁuna	ʡaˁdat-urme	akːʷ-i	hetːu-d,	wa	Rasul\\
		now	here\tsc{-npl}	\tsc{eq}	custom\tsc{-pl}	\tsc{cop.neg-hab.pst}	there\tsc{-npl}	hey	Rasul\\
	\glt	\sqt{‎Now such customs as here were not there, Rasul.} (i.e. the customs that exist now in Druzhba did not exist in earlier times in Sanzhi, so how can she learn about the Sanzhi customs here?)
	
	
	\ex	{[They did the treatment; they prescribed injections for one month.]}\label{ex:‎‎‎They made injections into the hand for ten days}\\
	\gll	ukul-te=q'al	wec'al	bari	d-arq'-ib	naˁq-li-cːe\\
		injection\tsc{-pl=mod}	ten	day	\tsc{npl-}do\tsc{.pfv-pret}	hand\tsc{-obl-in}\\
	\glt	\sqt{‎‎‎They made injections into the hand for ten days.}


	\ex	\label{ex:‎With Tawlu always such stories happen}
	\gll	heχ	Tawlu-la	har	zamana	d-irχʷ-ar	t'amahama\\
		\tsc{dem.down}	Tawlu\tsc{-gen}	every	time	\tsc{npl-}become\tsc{.ipfv-prs}	story\\
	\glt	\sqt{‎With Tawlu always such stories happen.}
\end{exe}

As discussed in \refsec{ssec:Extraposed genitives}, Sanzhi has floating genitives for which the referents of the genitives are highly topical and affected. In the great majority of cases, those genitives take over the clause-final position and can possibly be regarded as framing adjuncts.

Interjections and addressee \is{particle}particles occur at the edge of clauses \refex{ex:‎Now such customs as here were not there, Rasul}, either preceding all other items in the clause or following them (see \refsec{sec:Pause fillers, address particles, exclamatives, and interjections} for examples).

Clauses fulfilling argument positions in complement constructions can precede or follow the verb. For more information see \refsec{sec:The syntactic properties of complement clauses}. In a complex sentence consisting of a main clause and at least one subordinate \isi{adverbial clause}, the neutral order is for the \isi{adverbial clause} to precede the main clause or to be center-embedded within the main clause, though the order in which the main clause precedes the \isi{adverbial clause} is also attested (\refsec{sec:The syntax of adverbial clauses}).

Subordinate clauses have a strong tendency to be verb-final \refex{ex:‎At that time he embellished his story even more}. This is true especially for complement clauses (\refsec{sec:The syntactic properties of complement clauses}) and \is{relative clause}relative clauses (\refsec{sec:Other syntactic properties of relative clauses}). It is easier to find \is{adverbial clause}adverbial clauses, in particular those headed by the general converbs, which show other than verb-final orders. Due to the rather fixed \isi{word order} and the readiness with which arguments are dropped, which for subordinate clauses is higher than for main clauses, the manipulation of the \isi{information structure} in subordinate clauses by means of the \isi{constituent order} is not readily available. However, it is possible to switch the order of subject and object, and occasionally postverbal arguments can be found. In such cases, the same connections between focal or topical elements and certain positions in the clause can be observed that were summarized for main clauses in \reftab{tab:The relationship between constituent order and information structure in main clauses}. For instance, \refex{ex:‎‎‎That (journalist) was a nationalist} shows an \isi{adverbial clause} with the \isi{copula} complement preceding the verb because it represents the focus\is{focus} and the topical subject following the verb. Example \refex{ex:‎The boy brought another (bottle of) wine from the car} illustrates VSO order in an \isi{adverbial clause} and resembles the finite VSO clauses in \refex{ex:‎‎The husband hit the wife VSO} because the verb carries the new information and the subject is topical.

\begin{exe}
	\ex	\label{ex:‎‎‎That (journalist) was a nationalist}
	\gll	nacijonalist	gu-r-sa-jč-ib-le	hel, \ldots\\
		patriot	\tsc{sub-abl-hither-}occur\tsc{.m.pfv-pret-cvb}	that\\
	\glt	\sqt{‎‎‎That (journalist) was a patriot, \ldots}

	\ex	{[the protagonists ran out of alcohol and sent a boy to the car that was filled with bottles]}\label{ex:‎The boy brought another (bottle of) wine from the car}\\
	\gll	mašina-l-cːe-r	k-aqː-ib-le	durħuˁ-l	cara=ra	čaˁʁir, \ldots\\
		car\tsc{-obl-in-abl}	\tsc{down}-carry\tsc{-pret-cvb}	boy\tsc{-erg}	other\tsc{=add}	wine\\
	\glt	\sqt{‎The boy brought another (bottle of) wine from the car, \ldots}
\end{exe}


% --------------------------------------------------------------------------------------------------------------------------------------------------------------------------------------------------------------------- %

\subsection{Information structure in interrogative clauses and question-an\-swer pairs}
\label{ssec:Information structure in interrogative clauses and question-answer pairs}

Polar \isi{questions} are obligatorily marked by the \isi{enclitic} \tit{=w}. The \isi{enclitic} is added to the verb or another constituent, e.g. noun, pronoun, adjective. Polar \isi{questions} seem to have a greater tendency for verb-final \isi{constituent order} than declarative utterances have \refex{ex:‎Did they take you somewhere to work}, \refex{ex:Among the graves, in the grass, who finds (him)}, but nevertheless postverbal topical constituents can be found \refex{ex:‎Then did they build it in another place Q}. Answers to polar \isi{questions} consist mostly only of the verb, which can be the affirmative \isi{copula} \tit{ca-b}, the negative \isi{copula} \tit{akːu}, or a full verb form \refexrange{ex:‎Did they take you somewhere to work}{ex:Did you go to the goat canyon Yes, we went}. Polar \isi{questions} are analyzed in detail in \refsec{sec:Simple polar questions and disjunctive polar questions}.

\begin{exe}
	\ex	\label{ex:‎Did they take you somewhere to work}
	\gll	čina-k'al	ʡaˁči-le	r-uc-ib-il=de=w?\\
		where\tsc{-indef}	work\tsc{-loc}	\tsc{f-}keep\tsc{.pfv-pret-ref=pst=q}\\
	\glt	\sqt{‎Did they take you (fem.) somewhere to work?}

	\ex	\label{ex:‎Then did they build it in another place In another place, they built it in a completely other place}
	\begin{xlist}
		\ex	\label{ex:‎Then did they build it in another place Q}
		\gll	c'il	cara	musːa-b	b-arq'-ib-le=de=w il?\\
			then	other	place\tsc{-n}	\tsc{n-}do\tsc{.pfv-pret-cvb=pst=q} that\\
		\glt	\sqt{‎Q: Then did they build it in another place?}

		\ex	\label{ex:In another place, they built it in a completely other place A}
		\gll	cara	musːa-b,	absalut'na cara	musːa-b	b-arq'-ib-il=de\\
			other	place\tsc{-n}	completely	other	place\tsc{-n}	\tsc{n-}do\tsc{.pfv-pret-ref=pst}\\
		\glt	\sqt{‎A: In another place, they built it in a completely other place.}
	\end{xlist}

	\ex	\label{ex:Did you go to the goat canyon Yes, we went}
	\begin{xlist}
		\ex	\label{ex:Did you go to the goat canyon Q}
		\gll	q'aca-la	neqːe	ixtːu	ag-ur-il=de=w	u?\\
			he.goat\tsc{-gen}	canyon.\tsc{loc}	there.\tsc{up}	go\tsc{.pfv-pret-ref=pst=q} \tsc{2sg}\\
		\glt	\sqt{Q: Did you go to the goat canyon?}

		\ex	\label{ex:Yes, we went A}
		\gll		e,	ag-ur=da\\
			 yes	go\tsc{.pfv=pret=1}\\
		\glt	\sqt{A: Yes, we went.}
	\end{xlist}
\end{exe}

Content \isi{questions} contain interrogative words and are also marked by a special \isi{enclitic} \tit{=e}\slash\tit{=ja} that is normally attached to the predicate (\refsec{sec:Content questions}). If there is no predicate, then the interrogative pronoun functions as head. The interrogative pronoun mostly appears immediately before the verb, that is, in the same position in which most focus\is{focus} items occur in declarative utterances. Sentence topics regularly precede the interrogative pronoun \refex{ex:Among the graves, in the grass, who finds (him)}, \refex{ex:Which work was uncle Mahammadhazhi doing Q} or, more rarely follow it \refex{ex:Then from where did you go Q}. It is also possible to have one clause-initial topic and another clause-final topic.

\begin{exe}
	\ex	\label{ex:Among the graves, in the grass, who finds (him)}
	\gll	c'elt-m-a-cːe-w,	q'ar-ri-cːe-w,	hi-l	urkː-ul=e?\\
		gravestone\tsc{-pl-obl-in-m}	herbs\tsc{-obl-in-m}	who\tsc{.obl-erg}	find\tsc{.m.ipfv-icvb=q}\\
	\glt	\sqt{Among the graves, in the grass, who finds (him)?}

	\ex	\label{ex:Which work was uncle Mahammadhazhi doing He was the village head, brigadier, then he stopped being village head}
	\begin{xlist}
		\ex	\label{ex:Which work was uncle Mahammadhazhi doing Q}
		\gll	Maħaˁmmadħaˁži	acːi-l		ce	ʡaˁči	b-irq'-ul		kelg-un-il=de?\\
			Mahammadhazhi	uncle\tsc{-erg}	what	work	\tsc{n-}do\tsc{.ipfv-icvb}	remain\tsc{.pfv-pret-ref=pst}\\
		\glt	\sqt{Q: Which work was Uncle Mahammadhazhi doing?}

		\ex	\label{ex:He was the village head, brigadier, then he stopped being village head A}
		\gll	iχ	selsawet-le	kelg-un	birgadir-le	kelg-un c'ili	selsawet-le-r	ag-ur\\
			\tsc{dem.down}	village.head\tsc{-loc}	remain\tsc{.pfv-pret}	brigadier\tsc{-loc}	remain\tsc{.pfv-pret}	then	village.head\tsc{-loc-abl}	go\tsc{.pfv-pret}\\
		\glt	\sqt{‎A: He was the village head, brigadier, then he stopped being village head (lit. left).}
	\end{xlist}

	\ex	\label{ex:Then from where did you go We put up these (trousers) and go through the river}
	\begin{xlist}
		\ex	\label{ex:Then from where did you go Q}
		\gll	c'il	čina-r	d-ax-utːa=ja	ušːa?\\
			then	where\tsc{-abl}	\tsc{1/2pl-}go\tsc{-prs.2pl=q} \tsc{2pl}\\
		\glt	\sqt{‎Q: Then from where did you go?}

		\ex	\label{ex:We put up these (trousers) and go through the river A}
		\gll	aq	d-arq'-ib-le	heštːi,	či-r-d-ax-ud	erk'ʷ-le-r\\
			high	\tsc{npl-}do\tsc{.pfv-pret-cvb}	these	\tsc{spr-abl-1/2pl-}go\tsc{-1.prs}	river\tsc{-loc-abl}\\	%
		\glt	\sqt{A: We put up these (trousers) and go through the river.}
	\end{xlist}
\end{exe}

Fronting of interrogative words is also possible, but very rare in natural texts. The only pronoun that is repeatedly fronted is \tit{cellij}\slash\tit{cel} \sqt{why} \refex{ex:‎Why did I leave him in the room and did not take him away} (see \refsec{sssec:cel and celij why} for more examples).\footnote{As one reviewer pointed out, the fronting of a pronoun with the meaning `why' is common. For instance, in Hungarian it is the only interrogative word that can be found in a position other than the focus\is{focus} position immediately to the left of the verb. In the Austronesian language Pohnpeian ‘why' must be initial; other interrogative phrases do not have to be, see \citet{Dryer2005} for examples and references.} The pronoun \tit{ceqːel} \sqt{when} also occasionally occurs in clause-initial position \refex{ex:‎‎‎When did grandfather go to Sanzhi}, but usually it is preceded by \is{spatial adverb}spatial adverbs. Thus, sentence \refex{ex:‎‎‎When did grandfather go to Sanzhi} was repeated by the speaker and in the second occurrence the order of the interrogative pronoun and the directional adverbial were swapped.

\begin{exe}
	\ex	\label{ex:‎Why did I leave him in the room and did not take him away}
	\gll	cellij	het	qili-w	w-at-ur-re,	laˁk'	a-arq'-ib=da=jal?\\
		why	that	home\tsc{-m}	\tsc{m-}let\tsc{.pfv-pret-cvb}	away	\tsc{neg-}do\tsc{.pfv-pret=1=indq}\\
	\glt	\sqt{‎Why did I leave it (the photo) in the room and did not take it away (into the other room)?}

	\ex	\label{ex:‎‎‎When did grandfather go to Sanzhi}
	\gll	ceqːel	Sanži-le	ag-ur-il=de	χatːaj?\\
		when	Sanzhi\tsc{-loc}	go\tsc{.pfv-pret-ref=pst}	grandfather\\
	\glt	\sqt{‎‎‎When did grandfather go to Sanzhi}
\end{exe}

For other pronouns corpus examples are non-existent, but available in elicitation. Sentence \refex{ex:What will Batichaj bring now} can be uttered in a situation in which we know that the shop is empty and we wonder what Batichaj can bring if there is nothing to buy. The utterance in \refex{ex:Who should Rukijat marry} represents the pragmatically neutral \isi{constituent order} for this type of question with a postverbal topical subject, see also \refex{ex:What do you (masc.) say1}.

\begin{exe}
	\ex	\label{ex:What will Batichaj bring now}
	\gll	ce	Bat'ičaj-li	ha-b-iqː-an-ne hana?\\
		what	Batichaj\tsc{-erg}	\tsc{up-n-}carry\tsc{.ipfv-ptcp-fut.3}	now\\
	\glt	\sqt{What will Batichaj bring now?} (E)

	\ex	\label{ex:Who should Rukijat marry}
	\gll	hij	xadi	r-ax-an=e	Ruq'ijat?\\
		who\tsc{.dat}	married	\tsc{f-}go\tsc{-ptcp=q}	Rukijat\\
	\glt	\sqt{Whom should Rukijat marry?} (E)
\end{exe}

Topic-comment sentences are answers to \isi{questions} such as \tit{What is X doing?} Example \refex{ex:What is Isakadi doing? He dragged the watermelon and put it in front of the women} shows this type of question and the respective answer that has a neutral pragmatic structure, that is, the subject is topical and has been omitted in the answer. The predicate together with the object and the \isi{goal} represents the focal information and occurs in front of the verb (object) and directly after the verb (\isi{goal}), the typical positions for focal objects and goals. The \isi{constituent order} in the question is SOV, with the interrogative pronoun occurring \textit{insitu}.

\begin{exe}
	\ex	\label{ex:What is Isakadi doing? He dragged the watermelon and put it in front of the women}
	\gll	Isaq'adi-l	ce	b-irq'-ul=e?	qus	ha-b-aˁq-ib-le,	hel	qːalpuz	ka-b-išː-ib	het	xːun-r-a-j	naprotiw\\
		Isakadi\tsc{-erg} what	\tsc{n-}do\tsc{.ipfv-icvb=q}	slip	\tsc{up-n-}drag\tsc{.pfv-pret-cvb}	that watermelon	\tsc{down-n-}put\tsc{.pfv-pret}	that	woman\tsc{-pl-obl-dat} in.front\\
	\glt	\sqt{What is Isakadi doing? He dragged the watermelon and put it in front of the women.}
\end{exe}

In sum, the \isi{information structure} of the vast majority of content \isi{questions} is \textit{(topic)-question word-verb-(topic)} with the interrogative \isi{enclitic} attached to the verb. Answers to content \isi{questions} can mirror this structure by placing the item that answers the question in the preverbal position as well, with optional topical elements placed at the edges of the clause \refex{ex:He was the village head, brigadier, then he stopped being village head A}. Alternatively, they can also contain the item in focus\is{focus} in another position, as \refex{ex:We put up these (trousers) and go through the river A} shows, in which the relevant noun follows the verb. Short answers consisting only of the focus\is{focus} are also common.

It is possible to use the \is{focus!constituent}constituent focus construction in \is{interrogative clause}interrogative clauses. In this construction the interrogative \isi{enclitic} is attached to the item in focus\is{focus} and the verb must take the form of a \isi{participle}. See \refsec{ssec:Contrastive focus and floating predicative particles} below for more details and examples.


% --------------------------------------------------------------------------------------------------------------------------------------------------------------------------------------------------------------------- %

\subsection{Right and left dislocation}\label{ssec:Right and left dislocation}
\largerpage

Because of its free \isi{constituent order}, it is not always easy to identify \isi{dislocation} in Sanzhi. Topicalization by means of placing arguments or adjuncts at the edges of clauses cannot be equated with \isi{dislocation}. We can be sure we are dealing with \isi{dislocation} when we find a pronoun in the clause that is co-referential with the dislocated \isi{noun phrase} and when the dislocated \isi{noun phrase} does not correspond to any arguments or adjunct of the clause and is therefore unlinked.

Dislocation is not particularly frequent, but when it occurs it has the same structure and the same functions that have been attested for \isi{dislocation} in other East Caucasian languages \citep{Forker.Belyaev2016}. Thus, left \isi{dislocation} is a topicalization strategy. The dislocated \isi{noun phrase} occurs in the \isi{absolutive} case. It can be linked or unlinked \refex{ex:Well, also those own relatives, and those who do not have (relatives), one has to look after them a bit, think of them}, \refex{ex:‎The fingers, the hands, I do not see that far}. In the following two examples, the dislocated items are given in square brackets and they are unlinked to the following clauses.

\begin{exe}
	\ex	\label{ex:Well, also those own relatives, and those who do not have (relatives), one has to look after them a bit, think of them}
	\gll	[nu	ix-tːi	w-ah-la	tuχum-te=ra		il-tːi	akːʷ-ar-te=ra]	q'ʷila	bek'lal	er	∅-ik'ʷ-an	ca-w,	pikri	b-ik'ʷ-an	ca-b\\
		well	\tsc{dem.up}\tsc{-pl}	\tsc{m-}owner\tsc{-gen}	relative\tsc{-pl=add}		that\tsc{-pl}	\tsc{cop.neg-ptcp-dd.pl=add}	a.little	at.all	look	\tsc{∅-}look.at\tsc{.ipfv-ptcp} \tsc{cop-m}		thought	\tsc{n-}say\tsc{.ipfv-ptcp}	\tsc{cop-n}\\
	\glt	\sqt{Well, also those own relatives, and those who do not have (relatives), one has to look after them a bit, think of them.}

	\ex	\label{ex:‎The fingers, the hands, I do not see that far}
	\gll	[t'upː-e=ra	nuˁq-be=ra=q'al]	het-itːe	haraq-le	či-b-ig-ul	akːʷa-di\\
		finger\tsc{-pl=add}	arm\tsc{-pl=add=mod}	that\tsc{-advz}	far\tsc{-advz}	\tsc{spr-n-}see\tsc{.ipfv-icvb}	\tsc{cop.neg-1}\\
	\glt	\sqt{‎The fingers, the hands, I do not see that far.}
\end{exe}

Right \isi{dislocation} expresses afterthoughts that either extend the reference of the doubled item, make it more explicit, or re-phrase it in order to help the addressee to arrive at a correct understanding. The dislocated item bears the same case marking as its doubled counterpart in the clause. In \refex{ex:He gave them pears, to the boys_IS} the \isi{recipient}, which is encoded as \isi{reflexive pronoun} in the clause, is also expressed as a full \isi{noun phrase} after the clause. In \refex{ex:‎For one month I was cured only with pills, for 30 days_IS} the temporal adjunct has been repeated (though it is not an exact repetition).

\begin{exe}
	\ex	\label{ex:He gave them pears, to the boys_IS}
	\gll	it-i-l	ču-j	quˁr-be=ra	d-ičː-ib,	hel-tːi	duˁrħ-n-aˁ-j\\
		that\tsc{-obl-erg}	\tsc{refl.pl-dat}	pear\tsc{-pl=add}	\tsc{npl-}give\tsc{.pfv-pret}	that\tsc{-pl}	boy\tsc{-pl-obl-dat}\\
	\glt	\sqt{He gave them pears, to the boys.}

	\ex	\label{ex:‎For one month I was cured only with pills, for 30 days_IS}
	\gll	ca	bac	darman-t-a-lla=cun	lečenie	b-arq'-ib=da,	ʡaˁb-c'al	bari\\
		one	month	medicine\tsc{-pl-obl-gen=}only	cure	\tsc{n-}do\tsc{.pfv-pret=1} three-\tsc{ten}	day\\
	\glt	\sqt{‎For one month I was treated only with pills, for 30 days.}
\end{exe}


%%%%%%%%%%%%%%%%%%%%%%%%%%%%%%%%%%%%%%%%%%%%%%%%%%%%%%%%%%%%%%%%%%%%%%%%%%%%%%%%

\section{Other types of focus constructions}
\label{sec:Other types of focus constructions}


% --------------------------------------------------------------------------------------------------------------------------------------------------------------------------------------------------------------------- %

\subsection{Thetic sentences and presentational constructions}
\label{ssec:Thetic sentences and presentational constructions}

Thetic sentences are fully-focused sentences that can be uttered out of the blue or as answers to the question \tit{What happened?} In Sanzhi, they have the same structure with respect to the division of the utterance into topic and focus\is{focus} as utterances with the default \isi{information structure} \refex{ex:‎‎In one village there lived a family, a big family} (\refsec{sec:Constituent order at the clause level and information structure}).

\begin{exe}
	\ex	{[the beginning of a story]}\label{ex:‎‎In one village there lived a family, a big family}\\%
	\gll	ca	šːi-l-cːe-b	ca	kulpat,		χːula	kulpat	b-už-ib	ca-b\\
		one	village\tsc{-obl-in-hpl}	one	family	big	family	\tsc{hpl-}stay\tsc{-pret}	\tsc{cop-hpl}\\
	\glt	\sqt{‎‎In one village there lived a family, a big family.}
\end{exe}

Moreover, there is a specialized construction for presentational and existential sentences that introduce new and mostly animate referents into discourse. This construction can be either a thetic sentence if it occurs, e.g., at the very beginning of a story \refex{ex:‎‎The bear, the wolf, the fox, and the camel were friends}, but it can also contain some topical material. In all cases the newly introduced referent is an argument that follows the verb and occurs as the subject if the verb is intransitive and as the object if the verb is transitive. Thus, the \isi{constituent order} is VS \refex{ex:‎‎The bear, the wolf, the fox, and the camel were friends}, \refex{ex:We had a game} or VO \refex{ex:Allah sent snow during the night}.\footnote{This sentence contains the \isi{particle} \textit{ma}, which used when giving things to other people and inviting them to take the things (\refsec{sec:Pause fillers, address particles, exclamatives, and interjections}). In this examples, it is Allah who gave snow to the people who had to `take' it, i.e., live with it.} The optional topical items precede or follow the verb with its argument.

\begin{exe}
	\ex	\label{ex:‎‎The bear, the wolf, the fox, and the camel were friends}
	\gll	daršːi	ka-b-iž-ib	ca-b	sːika=ra	bec'=ra	 kːurtːa=ra	waˁlur=ra\\
		friendship	\tsc{down-n-}be\tsc{.pfv-pret}	\tsc{cop-n}	bear\tsc{=add}	wolf\tsc{=add}	fox\tsc{=add} camel\tsc{=add}\\
	\glt	\sqt{‎‎The bear, the wolf, the fox, and the camel were friends.}

	\ex	\label{ex:We had a game}
	\gll	nišːa-la	b-irχʷ-i	ħaˁz\\
		\tsc{1pl-gen}	\tsc{n-}be\tsc{.ipfv-hab.pst}	game\\
	\glt	\sqt{(When we were little) we had a game.}

	\ex	\label{ex:Allah sent snow during the night}
	\gll	Allah-li	ma	hana	dučːi-la	paˁq	či-ka-b-arq'-ib	duˁħi\\
		Allah\tsc{-erg}	take	now	night\tsc{-gen}	strike	\tsc{spr-down}\tsc{-n-}do\tsc{.pfv-pret}	snow\\
	\glt	\sqt{Allah sent snow during the night.}
\end{exe}


% --------------------------------------------------------------------------------------------------------------------------------------------------------------------------------------------------------------------- %

\subsection{Contrastive focus and floating predicative particles}
\label{ssec:Contrastive focus and floating predicative particles}

Selective and corrective \is{focus!corrective} focus, which belongs to the category of \is{focus!contrastive}contrastive focus, is expressed by placing the predicative \is{particle}particles (\refsec{sec:Predicative particles}) or the \isi{copula} (\refsec{sec:The copula}) immediately after the focused constituent. This construction exists in many East Caucasian languages, but its frequency in texts seems to greatly differ from language to language. In this section, I will analyze the Sanzhi construction by first presenting data about the floating predicative \is{particle}particles and then about the floating \isi{copula}. All predicative \is{particle}particles have the (verbal) predicate as their default host. In \is{focus!contrastive}contrastive focus constructions, the \is{particle}particles float off from their normal host and appear on arguments and adjuncts, on phrasal heads and on modifiers of heads. The lexical verb must take the form of a \isi{participle} \xxref{ex:Now I am cleaning the dishes@3}{ex:It is ME who is washing the dishes now@6}. It can be either a complex \isi{participle} with the suffixes \tit{-il} or \tit{-ce}, or it can simply be the \isi{modal participle}. In contrast to sentences without \is{focus!contrastive}contrastive focus, it cannot be a converb or the simple \isi{preterite participle}.

\begin{exe}
	\ex	\label{ex:Now I am cleaning the dishes@3}
	\gll	du-l 	hana 	tʼalaˁħ-ne 	ic-ul=da\\
		\tsc{1sg-erg}	now	dishes\tsc{-pl}	wash\tsc{.ipfv-icvb=1}\\
	\glt	\sqt{Now I am cleaning the dishes.} (E)

	\ex	\label{ex:Now I am cleaning the dishes@4}
	\gll	du-l 	hana	tʼalaˁħ-ne=da	ic-an,	cʼil	\ldots\\
		\tsc{1sg-erg}	now	dishes\tsc{-pl=1}	wash\tsc{.ipfv-ptcp}	then\\
	\glt	\sqt{Now I am cleaning THE DISHES, \ldots\ (e.g. I will clean the windows later)} (E)

	\ex	\label{ex:NOW I am washing the dishes@5}
	\gll	du-l	hana=da	tʼalaˁħ-ne	ic-an\\
		\tsc{1sg-erg}	now\tsc{=1}	dishes\tsc{-pl}	wash\tsc{.ipfv-ptcp}\\
	\glt	\sqt{NOW I am washing the dishes.} (E)

	\ex	\label{ex:It is ME who is washing the dishes now@6}
	\gll	du-l=da	hana	tʼalaˁħ-ne	ic-an\\
		\tsc{1sg-erg=1}	now	dishes\tsc{-pl}	wash\tsc{.ipfv-ptcp}\\
	\glt	\sqt{It is ME who is washing the dishes now.} (E)
\end{exe}

As examples \refex{ex:I had to read the newspaper that ZAMIR bought yesterday@25a} and \refex{ex:Now I will / have to start to make KHINKAL@6} show, floating predicative \is{particle}particles can even occur in certain types of subordinate clauses such as infinitival complements.

\begin{exe}
	\ex	\label{ex:I had to read the newspaper that ZAMIR bought yesterday@25a}
	\gll	du-l	b-uč'-an	[Zamir-ri=de	sːa	asː-ib-il]	kːazat\\
		\tsc{1sg-erg}	\tsc{n-}read\tsc{.ipfv-ptcp}	Zamir\tsc{-erg=pst}	yesterday	buy\tsc{.pfv-pret-ref}	newspaper\\
	\glt	\sqt{I had to read the newspaper that ZAMIR bought yesterday.} (E)

	\ex	\label{ex:Now I will / have to start to make KHINKAL@6}
	\gll	hana	du	r-aʔ-r-irxː-an	[χːink'-e=da	d-arq'-ij]\\
		now	\tsc{1sg}	\tsc{f-}begin\tsc{-f-}begin\tsc{.ipfv-ptcp}	khinkal\tsc{-pl=1}	\tsc{npl-}do\tsc{.pfv-inf}\\
	\glt	\sqt{Now I will\slash have to start to make KHINKAL.} (E)
\end{exe}

Most notably, with transitive and \is{affective verb}affective verbs it is possible to drop the \isi{ergative} or \isi{dative} marking of the subject if the predicative \isi{particle} is attached to it.

\begin{exe}
	\ex	\label{ex:I will / have to tell Sanijat the story@8a}
	\gll	du=da	Sanijat-li-j	χabar	b-urs-an\\
		\tsc{1sg=1}	Sanijat\tsc{-obl-dat}	story	\tsc{n-}tell\tsc{.pfv-ptcp}\\
	\glt	\sqt{It is me who will\slash have to tell Sanijat the story.} (E)

	\ex	\label{ex:I will not forget this lesson@8b}
	\gll	du=da	it	dars	qum.ert-an\\
		\tsc{1sg=1}	\tsc{dem}	lesson	forget\tsc{.ipfv.neg-ptcp}\\
	\glt	\sqt{It is me who will not forget that lesson.} (E)
\end{exe}

When Sanzhi speakers are asked to formulate \isi{questions} to which sentences with floating predicative \is{particle}particles are suitable answers, they produce content \isi{questions} in which the relevant item that is \is{focus}focused in the answer is replaced by a question word serving as the host for the \is{particle}particles \refex{ex:WHEN will you / do you have to go to Druzhba We will / have to go TODAY@9}. 

\begin{exe}
	\ex	\label{ex:WHEN will you / do you have to go to Druzhba We will / have to go TODAY@9}
	\begin{xlist}
		\ex	\label{ex:WHEN will you / do you have to go to Druzhba@9a}
		\gll	ušːa	ceqːel=da=ja	Družba-le	d-ax-an?\\
			\tsc{2pl}	when\tsc{=2pl=q}	Druzhba\tsc{-loc}	\tsc{1/2pl-}go\tsc{.ipfv-ptcp}\\
		\glt	\sqt{WHEN will you\slash do you have to go to Druzhba?} (E)

		\ex	\label{ex:We will / have to go TODAY@9b}
		\gll	nušːa	ižal=da	d-ax-an\\
			\tsc{1pl}	today\tsc{=1}	\tsc{1/2pl-}go\tsc{.ipfv-ptcp}\\
		\glt	\sqt{We will\slash have to go TODAY.} (E)
	\end{xlist}
\end{exe}

In an \isi{interrogative clause} with a floating person \isi{enclitic} the interrogative \isi{enclitic} \tnd\ which is also a \isi{focus-sensitive particle} \tnd\ must be added to the same item that serves as the host for the person \isi{enclitic}. Encliticizing the two markers to two different items \refex{ex:WHEN will you / do you have to go to DruzhbaWRONG} is ungrammatical.

\begin{exe}
	\ex[*]{	\label{ex:WHEN will you / do you have to go to DruzhbaWRONG}
	\gll		ušːa	ceqːel=da	Družba-le	d-ax-an=e?\\
			\tsc{2pl}	when\tsc{=2pl}	Druzhba\tsc{-loc}	\tsc{1/2pl-}go\tsc{.ipfv-ptcp=q}\\
	\glt	(Intended meaning: \sqt{WHEN will you\slash do you have to go to Druzhba?})}
\end{exe}

Person enclitics or the past \isi{enclitic} used as \is{focus!contrastive}contrastive focus markers are not very frequent in the corpus although the constructions are readily available in elicitation as many examples in this section prove. In example \refex{ex:But you were the one who said that they had taken only Alibatir to the ditch@26b} the past \isi{enclitic} occurs in combination with another focus-sensitive predicative \is{focus-sensitive particle}particle, the modal \isi{particle} =\textit{q'al}.

\begin{exe}
	\ex	{[discussing the viewpoint that a speaker has to take for a narration]}\label{ex:I am now her, right? Yes, YOU are now}\\%
	\gll	du=da	hana	heχ,	akːu=w? 	e,	u=de	hana	\ldots\\
		\tsc{1sg=1}	now	\tsc{dem.down}	\tsc{cop.neg=q}	yes	\tsc{2sg=2sg}	now\\
	\glt	\sqt{I am now her, right? Yes, YOU are now \ldots}

	\ex	{[talking about the places where the speaker had been]}\label{ex:It was Latvia that seemed to me to be really the most beautiful country constituents}\\%
	\gll	bah	qːuʁa-ce	dunja	ka-b-icː-ur-il	dam	dejstvitelna	Latwija=de\\
		most	beautiful\tsc{-dd.sg}	world	\tsc{down-n-}stand\tsc{.pfv-pret-ref}	\tsc{1sg.dat}	really Latvia\tsc{=pst}\\
	\glt	\sqt{It was Latvia that seemed to me to be really the most beautiful country.}

	\ex	{[Before I came my mother died. At home there was nobody.]}\label{ex:Again it was me who remained alone@26a}\\%
	\gll	apjat	ca	du=gina=de	kelg-un-il	ca\\
		again	one	\tsc{1sg=}only\tsc{=pst}	remain\tsc{.pfv-pret-ref}	one\\
	\glt	\sqt{Again it was me who remained alone.}

	\ex	\label{ex:But you were the one who said that they had taken only Alibatir to the ditch@26b}
	\gll	c'il	u=de=q'al	qːanaw-t-a-j	ca	ʡaˁlibatir 	akːʷ-ar		a-w-erč-ib 	∅-ik'ʷ-an\\
		then	\tsc{2sg=pst=mod}	ditch\tsc{-pl-obl-dat}	one	Alibatir 	\tsc{cop.neg-prs.3}	\tsc{neg-m-}lead\tsc{.pfv-pret}	\tsc{m-}say\tsc{.ipfv-ptcp}\\
	\glt	\sqt{But you were the one who said that they had taken only Alibatir to the ditch.} 
\end{exe}

It is slightly easier to find examples with interrogative enclitics \xxref{ex:I asked Bahama, who is it that is wrapped there}{ex:‎‎Or is this her who is stealing} or the modal \isi{particle} \tit{=q'al} \xxref{ex:‎‎‎This is my son?!‎ ‎‎Yes, he says}{ex:‎It was him himself who had agreed to get off (the car) in Kala (place name)}; see also the example \refex{ex:But you were the one who said that they had taken only Alibatir to the ditch@26b} above. All corpus examples discussed so far can be classified as selective \isi{focus}, because the focus selects one item from among a presupposed set of possible alternative values \citep[62]{Diketal1981}.

\begin{exe}
	\ex	\label{ex:I asked Bahama, who is it that is wrapped there}
	\gll	``ča=ja,''	haʔ-ib=da,	``il	itːu-w	k'ap	k-arq'-ib-il?'' Baħaˁmma-cːe\\
		who\tsc{=q}	say\tsc{.pfv-pret=1}	that	there\tsc{-m} wrap	\tsc{down}-do\tsc{.pfv-pret-ref} Bahamma\tsc{-in}\\
	\glt	\sqt{I asked Bahama ``Who is it that is wrapped there?''}

	\ex	\label{ex:Who is it who said this}
	\gll	ča=ja	hel	∅-ik'ʷ-an?\\
		who\tsc{=q}	that	\tsc{m-}say\tsc{.ipfv-ptcp}\\
	\glt	\sqt{Who is it who said that?}

	\ex	\label{ex:‎‎Or is this her who is stealing}
	\gll	ja=ra	hej	r-ilʡ-aˁn	hež=uw?\\
		or\tsc{=add}	this	\tsc{f-}steal\tsc{.ipfv-ptcp}	this\tsc{=q}\\
	\glt	\sqt{‎‎Or is this her who is stealing?}

	\ex	\label{ex:‎‎‎This is my son?!‎ ‎‎Yes, he says}
	\gll	``di-la	durħuˁ=q'al?!''	``ca<w>i,''	∅-ik'ʷ-ar\\
		\tsc{1sg-gen}	boy\tsc{=mod}	\tsc{cop<m>}	\tsc{m-}say\tsc{.ipfv-prs}\\
	\glt	\sqt{``‎‎‎Is he my son?!''‎ ``‎‎(Yes) he is,'' he says.}

	\ex	\label{ex:Here (i.e. on this picture) is it that the boy is together with them@26b}
	\gll	il-tːu-w=q'al	durħuˁ	w-alli	le-w\\
		that\tsc{-loc-m=mod}	boy	\tsc{m-}together	exist\tsc{-m}\\
	\glt	\sqt{It is here (i.e. on this picture) that the boy is together with them.}

	\ex	{[We are thinking where to meet Abdulkhalik.]}\label{ex:‎It was him himself who had agreed to get off (the car) in Kala (place name)}\\%
	\gll	ik'=q'al	waˁʡda	b-arq'-ib-il	ca-w	Qːala-j	či-r-ka-w-q-an-aj\\
		\tsc{dem.up}\tsc{=mod}	agreement	\tsc{n-}do\tsc{.pfv-pret-ref}	\tsc{refl.sg-m}	Mamedkala\tsc{-dat}	\tsc{spr-abl-down}\tsc{-m-}go\tsc{-ptcp-subj.3}\\
	\glt	\sqt{‎It was he himself who had agreed to get off (the car) in Mamedkala.}
\end{exe}

Constituent focus\is{focus} constructions with the \isi{copula} have the same pragmatics as constructions with predicative \is{particle}particles, that is, they express \is{focus!contrastive}contrastive focus, most commonly selective \isi{focus}. Constituent focus constructions with the \isi{copula} are a bit more flexible in the sense that the lexical verb can not only occur as a \isi{participle}, but appears occasionally in the form of the general converbs, which are normally used to form analytic tenses \refex{ex:This is a BUNDLE of something. This BUNDLE also has been put down (there)@27a}. In most of the examples, there is a weak pitch accent on the item preceding the auxiliary. In \refex{ex:There is the one that does not burn, the cow-parsnip@25} the \isi{existential copula} serves as host for the modal \isi{particle} \tit{=q'al}, which also belongs to the class of predicative \is{particle}particles.\largerpage[2]


\begin{exe}
	\ex	\label{ex:There is the one that does not burn, the cow-parsnip@25}
	\gll	a-rurg-an	le-b=q'al	it,	birikːalla.ʁut'	le-b=q'al	het\\
		\tsc{neg-}burn\tsc{-ptcp}	exist\tsc{-n=mod}	that	cow.parsnip	exist\tsc{-n=mod}	that\\
	\glt	\sqt{There is the one that does not burn, the cow-parsnip.}

	\ex	{[A: They are sitting, enjoying themselves, passing the time. B replies:]}\label{ex:Yes, they resolve their sorrows BEHIND THE GLASS@25b}\\
	\gll	e,	ču-la	dard-ane	šišːim-te	istikan-na	hila-cːe-d	ca-d 		d-irq'-an-te\\
		yes	\tsc{refl.pl-gen}	sorrow\tsc{-pl}	suffering\tsc{-pl}	glass\tsc{-gen}	behind\tsc{-in-npl}		\tsc{cop-npl} 	\tsc{npl-}do\tsc{.ipfv-ptcp-dd.pl}\\
	\glt	\sqt{It is behind the glass that resolve their sorrows.} (i.e. by drinking alcohol)
	
		\ex	\label{ex:‎Your pupil is spoiled, she said}
	\gll	``ala	gawhar	ca-b,''	r-ik'ʷ-ar, ``zaja	b-iχ-ub-il''\\
		\tsc{2sg.gen}	pupil	\tsc{cop-n}	\tsc{f-}say\tsc{.ipfv-prs}	spoil	\tsc{n-}be\tsc{.pfv-pret-ref}\\
	\glt	\sqt{\dqt{It is your pupil that is spoiled,} she said.}
\end{exe}

It is not only the standard \isi{copula} \textit{ca-b} that can be used in \isi{focus} constructions, but also the existential/locational copulas (\refsec{sec:Locational copulae}):

\begin{exe}
	\ex	\label{ex:‎Up there the graveyard of ours is visible}
	\gll	χːuˁrbe	k'e-d	heχtːu-d	či-d-ig-ul	nišː-ala\\
		graveyard	exist.\tsc{up-npl}	there.\tsc{down-npl}	\tsc{spr-npl-}see\tsc{.ipfv-icvb}	\tsc{1pl-gen}\\
	\glt	\sqt{‎Up there the graveyard of ours is visible.}
\end{exe}

In the next two examples, the constituents followed by the \isi{copula} are already mentioned and thus given in the context. These examples can count as confirmative or approving \isi{focus}. The speakers do not correct some previous statements, but they acknowledge the selected alternative and reassure the selection. For instance, \refex{ex:This is a BUNDLE of something. This BUNDLE also has been put down (there)@27a} is part of a description of a picture. The speaker is at first unsure about one of the depicted objects. With the utterance of \refex{ex:This is a BUNDLE of something. This BUNDLE also has been put down (there)@27a} he affirms that those objects must be bunches or bundles of something unidentified. With the sentence in \refex{ex:This is it what has happened to him@27b} the speaker sums up his argumentation about a possible sequence of events. The \isi{demonstrative pronoun} and the verb bearing the \isi{participle} suffix were uttered with a rising tone, and there was no noticeable accent on the pronoun preceding the \isi{copula}.

\begin{exe}
		\ex	{[What are these, not bundles? Bundles.]}\label{ex:This is a BUNDLE of something. This BUNDLE also has been put down (there)@27a}\\%
		\gll	cel-la=jal	χːala	ca-b	b-iχ-un-il. 	hež=ra	χːala	ca-b	ka-b-išː-ib-le\\
			what\tsc{.obl-gen=indef}	bundle	\tsc{cop-n}	\tsc{n-}be\tsc{.pfv-pret-ref} 	this\tsc{=add}	bundle	\tsc{cop-n}	\tsc{down-n-}put\tsc{.pfv-pret-cvb}\\
		\glt	\sqt{This is a BUNDLE of something. This BUNDLE also has been put down (there).}

		\ex	{[Discussing which events pictures might show]}\label{ex:This is it what has happened to him@27b}\\
		\gll	hel	ca-b	hel-i-j	ag-ur-il\\
			that	\tsc{cop-n}	that\tsc{-obl-dat}	go\tsc{.pfv-pret-ref}\\
		\glt	\sqt{[This exactly, this has happened]. This is it what has happened to him.}
\end{exe}

Contrastive focus constructions that make use of the \isi{copula} or of other \is{particle}particles such as person enclitics are found in a \isi{number} of other East Caucasian languages. In particular, floating person enclitics have been investigated for Udi \citep{Harris2001, Harris2002}, Lak \citep{Kazenin2002}, Standard Dargwa \citep{Xajdakov1986} and Tanti Dargwa \citep{Sumbatova2013}. The impact of the modal \isi{particle} on the \isi{information structure} and its use in focus\is{focus} constructions has been analyzed by \citet{ForkerSubmittedc}. \citet{Kazenin2002} proposes an analysis of such constructions as synchronic \isi{cleft} constructions because they have the pragmatic value of \isi{cleft} construction: the predicative \isi{particle} or \isi{copula} identifies or specifies the argument in a presupposed open proposition. The structure can be simplified as follows:

\begin{exe}
	\ex	(topic) [FOCUS]=\isi{particle}/\isi{copula} (topic) \label{ex:topic focus particlecopula topic}
\end{exe}

Clefts functionally and formally resemble \isi{copula} clauses (\refsec{sec:copulaclauses}) when the focal part is followed by a clause headed by a \isi{participle} that has the properties of a \isi{relative clause} \refex{ex:I will / have to tell Sanijat the story@8a}, \refex{ex:This is it what has happened to him@27b} \refsec{sec:Other syntactic properties of relative clauses}. The constructions express term \isi{focus}. They can be formed not only with the help of the modal \isi{particle}, but also with all other predicative \is{particle}particles or auxiliary verbs that function as copulas.

If we adopt the definition of \isi{cleft} proposed by \citet{Lambrecht2001}, we also notice the morphosyntactic similarities between the constructions in Sanzhi and other Dagestanian languages and clefts in European languages. \citea{Lambrecht2001} defines a \isi{cleft} as a syntactically biclausal structure consisting of two \is{simple clause}simple clauses, that is, a main clause headed by an auxiliary, and a subordinate clause that is a (free) \isi{relative clause} or relative-like clause. Thus, the auxiliary is the predicative \isi{particle} or \isi{copula}, i.e. items that are also used in \isi{copula} clauses, and the lexical verb appears as \isi{participle}. In most corpus examples, the \is{focus}focused constituent occurs at the left or right edge of the clause, so that the construction looks like a biclausal construction with a \isi{copula} clause and a preceding or following \isi{relative clause} that provides the information needed to identify the \isi{copula} subject \refex{ex:This is a BUNDLE of something. This BUNDLE also has been put down (there)@27a}, \refex{ex:This is it what has happened to him@27b}  (for more details see \citealt{Kazenin2002}; \citealt{Forker2016a}). However, the topical part is also frequently simply a \isi{demonstrative pronoun} \refex{ex:There is the one that does not burn, the cow-parsnip@25}.

Since it is possible to place the \is{focus}focused item in the middle of the clause, between what is supposed to represent a free \isi{relative clause}, a synchronic \isi{cleft} analysis as proposed by \citet{Kazenin2002} becomes questionable. For a detailed discussion of the \isi{cleft} approach and its problems see \citet{Forker2016a}.

% %--------------------------------------------------------------------------------------------------------------------------------------------------------------------------------------------------------------------- %

\subsection{Corrections}
\label{ssec:Corrections}

Corrective focus is found in utterances that correct a previous statement, e.g. by replacing the relevant information. This is frequently done by means of the negative \isi{copula} \tit{akːu} that indicates \is{focus!constituent}constituent focus on the preceding item \xxref{ex:‎My husband is not Sanzhi, he is Chakhri}{ex:‎This is not a water, this is a wind mill}. The adversative \isi{particle} \tit{=n(u)} is optionally encliticized to the \isi{copula}.

\begin{exe}
	\ex	\label{ex:‎My husband is not Sanzhi, he is Chakhri}
	\gll	di-la	murgul	sunglan	akːu,		čːuˁħrugan	ca-w\\
		\tsc{1sg-gen}	man	Sanzhi	\tsc{cop.neg}	Chakhri.person	\tsc{cop-m}\\
	\glt	\sqt{‎My husband is not Sanzhi, he is Chakhri.}

	\ex	{[A: He carries him home. B replies:]}\label{ex:No, not home, they take him to the sobering-up station@28}\\%
	\gll	qili	akːu=n,	witrezwitel-le	uqː-ul	ca-w\\
		home	\tsc{cop.neg=prt}	sobering-up.station\tsc{-loc}	carry\tsc{.m.pfv-icvb}	\tsc{cop-m}\\
	\glt	\sqt{No, not home, they are carrying him to the sobering-up station.}

	\ex	\label{ex:‎This is not a water, this is a wind mill}
	\gll	iž	hin-na	akːu=n,	iž	č'an-na	urχːab	ca-b\\
		this	water\tsc{-gen}	\tsc{cop.neg=prt}	this	wind\tsc{-gen}	mill	\tsc{cop-n}\\
	\glt	\sqt{‎This is not a water, this is a wind mill.}
\end{exe}

In \refex{ex:It is your car that is spoiled, not Ali's} the order of rejecting clause and correcting clause is reversed in comparison with the above examples. The first clause, which represents a cleft-like constructions as discussed in \refsec{ssec:Contrastive focus and floating predicative particles}, expresses the correction and the second clause the rejection.

\begin{exe}
	\ex	\label{ex:It is your car that is spoiled, not Ali's}
	\gll	ala mašin ca-b	zaja	b-iχ-ub-il,	ʡaˁli-la	akːu\\
		\tsc{2sg.gen}	car	\tsc{cop-n}	spoil	\tsc{n-}be\tsc{.pfv-pret-ref} Ali\tsc{-gen}	\tsc{cop.neg}\\
	\glt	\sqt{It is your car that is spoiled, not Ali's.} (E)
\end{exe}

%%%%%%%%%%%%%%%%%%%%%%%%%%%%%%%%%%%%%%%%%%%%%%%%%%%%%%%%%%%%%%%%%%%%%%%%%%%%%%%%

\section{Verb doubling}
\label{sec:Verb doubling}

Another construction that is found in Sanzhi and in other East Caucasian languages is verb doubling. This construction consists of a verb that is repeated. The first part occurs either in the \isi{infinitive} or in the form of the stem without any inflectional suffix, but preserving derivational morphology, spatial \is{preverb}preverbs, or morphemes used in \isi{compounding}. To this part the \isi{additive} \tit{=ra} is encliticized. The second part consists of a finite verb form of the same lexical verb \xxref{ex:‎She herself said it in that manner}{ex:As for drinking, they sit and drink with you, and (we) had nice conversations}. All corpus examples from Sanzhi make use of the verbal stem, but examples with the \isi{infinitive} that have the same pragmatics can be elicited (see below).

\begin{exe}
	\ex	{[‎‎Be sick for three days and then die, she said.]}\label{ex:‎She herself said it in that manner}\\%
	\gll	hel	ca-r	r-ik'ʷ-an	purma-lla,	r-ebč'=ra	r-ebč'-ib\\
		that	\tsc{refl-f}	\tsc{f-}say\tsc{.ipfv-ptcp}	uniform\tsc{-gen}	\tsc{f-}die\tsc{.pfv=add}	\tsc{f-}die\tsc{.pfv-pret}\\
	\glt	\sqt{‎She herself said it in that manner (=uniform), and as for dying she died [after three days as she had wanted].}

	\ex	{[He is thinking a lot. And there he is also thinking.]}\label{ex:‎And as for crying, he is crying a lot here}\\%
	\gll	w-is=ra	w-isː-ul	ca-w	ik'	heštːu-w	ca-w\\
		\tsc{m-}cry\tsc{=add}	\tsc{m-}cry\tsc{-icvb}	\tsc{cop-m}	\tsc{dem.up}	here\tsc{-m}	\tsc{cop-m}\\
	\glt	\sqt{‎And as for crying, he is crying a lot here.}

	\ex	{[But the Estonians, I think, are very similar to the Icari people, no difference, how often I went there]}\label{ex:As for drinking, they sit and drink with you, and (we) had nice conversations}\\%
	\gll	b-uč=ra	b-učː-u	a-cːella	ka-b-iž-ib-le, qːuʁa-ce	iχtilat=ra b-irq'-i\\
		\tsc{n-}drink\tsc{.ipfv=add}	\tsc{n-}drink\tsc{.ipfv-prs}	\tsc{2sg-comit}	\tsc{down-hpl-}be\tsc{.pfv-pret-cvb} beautiful\tsc{-dd.sg}	talk\tsc{=add}	\tsc{n-}do\tsc{.ipfv-hab.pst}\\
	\glt	\sqt{As for drinking, they sit and drink with you, and (we) had nice conversations.}
\end{exe}

It is also possible to employ the \isi{particle} \tit{=q'ar} instead of the \isi{additive}. This \isi{particle} is frequently used to topicalize the items to which it is added (\refsec{ssec:The enclitic =q'ar}). In \refex{ex:By God, he says, lay down (in the hospital)} only the \isi{preverb} in a slightly modified form is repeated, not the verbal stem.

\begin{exe}
	\ex	\label{ex:‎As for drinking, he would have drunk the tea, but they did not let him}
	\gll	d-erčː-ij=q'ar	it-i-l	čaˁj	d-učː-an=de,	amma	itːa-l	a-alt-ur	ca-w\\
		\tsc{npl-}drink\tsc{.pfv-inf=mod}	that\tsc{-obl-erg}	tea	\tsc{npl-}drink\tsc{.ipfv-ptcp=pst}	but	those\tsc{.obl-erg}	\tsc{neg-}let\tsc{.ipfv-pret}	\tsc{cop-m}\\
	\glt	\sqt{‎As for drinking, he would have drunk the tea, but they did not let him.} (E)
	
		\ex	\label{ex:By God, he says, lay down (in the hospital)}
	\gll	wallah	∅-ik'ʷ-ar		ka-r-isː-en!	kat=q'ar	ka-r-ilsː-a-di\\
		by.God	\tsc{m-}say\tsc{.ipfv-prs}		\tsc{down-f-}sleep\tsc{.pfv-imp} \tsc{pvb=prt}	\tsc{down-f-}sleep\tsc{.ipfv-hab-1}\\
	\glt	\sqt{``By God,'' he says, ``lay down (in the hospital)!'' ‎‎As for laying, I lie, (but there are stairs from here and there.)}
	
\end{exe}

\citet{Maisak2010} discusses verb doubling constructions in a \isi{number} of Dagestanian languages including Xuduc Dargwa and Icari Dargwa under the heading of ``predicate topicalization''. He concludes that the constructions cannot be used in thetic utterances, but only in reference to previously mentioned or otherwise already known situations, and that they mostly involve polarity \isi{focus}, because there is an associated adversative or \isi{concessive} clause that is following the verb doubling construction or implicitly present in the context. \citea{Maisak2010}'s interpretation seems to fit the Sanzhi data well, since in all examples at least part of the information conveyed in the verb doubling construction has been provided in the previous context. As for the adversative or \isi{concessive} clause following the verb doubling construction, there is one corpus example in which the following clause has an adversative \refex{ex:By God, he says, lay down (in the hospital)} marking, but the other examples \xxref{ex:‎She herself said it in that manner}{ex:As for drinking, they sit and drink with you, and (we) had nice conversations} lack adversative clauses, and there is no example with a following \isi{concessive} clause. Again, such examples can be elicited, but due to the general absence of associated adversative or \isi{concessive} clauses in the corpus, the expression polarity focus cannot be regarded as the main function of verb doubling in Sanzhi. Instead, the main function is predicate topicalization.


%%%%%%%%%%%%%%%%%%%%%%%%%%%%%%%%%%%%%%%%%%%%%%%%%%%%%%%%%%%%%%%%%%%%%%%%%%%%%%%%

\section{Focus-sensitive particles}
\label{sec:Focus-sensitive particles}

Sanzhi Dargwa has a \isi{number} of focus-sensitive \is{focus-sensitive particle}particles. The most common \is{particle}particles are:

\begin{itemize}
	\item	the \isi{additive} \tit{=ra} \sqt{also, too, as well as} (\refsec{ssec:The additive enclitic})
	\item	the modal \is{particle}particles \tit{=q'al} and \tit{=q'ar} (\refsec{ssec:The enclitic =q'al}, \refsec{ssec:The enclitic =q'ar})
	\item	the adversative \tit{=n(u)} \sqt{but} (\refsec{ssec:The enclitic =n(u)})
\end{itemize}

Further \is{particle}particles are \tit{=cun} \sqt{only}, \tit{=gina} \sqt{alone, only}, \tit{malle} \sqt{even}, \tit{arrah} \sqt{at least}, and \tit{akːʷar} \sqt{except, without, only}. In addition, all predicative \is{particle}particles, not only \tit{=q'al}, can be used in focus\is{focus} construction and thus also belong to the class of focus-sensitive \is{focus-sensitive particle}particles. The general functions of these \is{particle}particles are analyzed in more details in \refsec{sec:Predicative particles} and \refsec{sec:Discourse and modal enclitics} including examples for every \isi{particle}.

The focus-sensitive \is{focus-sensitive particle}particles can be encliticized to focal items in term focus-con\-struc\-tions as was described in \refsec{ssec:Contrastive focus and floating predicative particles} above. Focal items are frequently nominals or adverbials, but verbs can also be \is{focus}focused. If the focal item is a verb, the enclitics are either added to the finite inflected verb or to some other part of the predicate such as the \isi{preverb} \refex{ex:He guarded me well, he also guarded me safely, although without brothers and sisters}, or the verb appears in a non-finite form such as the \isi{infinitive} \refex{ex:‎Coming home is not bad}.

\begin{exe}
	\ex	{[Thanks to Allah for the life he gave]}\label{ex:He guarded me well, he also guarded me safely, although without brothers and sisters}\\%
	\gll	ʡaˁħ	r-al	r-iχː-ib=da,	qːuʁa	r-al=ra	r-iχː-ib=da, ucːi	rucːi	akːʷ-ar=xːar\\
		good	\tsc{f-}fit	\tsc{f-}guard\tsc{-pret=1}	beautiful	\tsc{f-}fit\tsc{=add}	\tsc{f-}guard\tsc{-pret=1} brother	sister	\tsc{cop.neg-prs.3=conc}\\
	\glt	\sqt{He guarded me (fem.) well, he also guarded me safely, although (I was) without brothers and sisters.}

	\ex	\label{ex:‎Coming home is not bad}
	\gll	qili	w-iteʁ-ij=q'ar	wahi-l	akːu\\
		home	\tsc{m-}reach\tsc{.pfv-inf=mod}	bad\tsc{-advz}	\tsc{cop.neg}\\
	\glt	\sqt{‎Coming home is not bad.}
\end{exe}

For example, when \tit{=q'al} is attached to the finite verb, it is the verb phrase or the whole clause that is in its scope. It is possible that the verb alone is in the scope of the \isi{particle} and represents the focus\is{focus} \refex{ex:Well, by God, we do have them, he said@17c}, but it can also be the verb with its overtly expressed arguments \refex{ex:There is the one that does not burn, the cow-parsnip@25}, which sometimes amounts to the entire clause. If the entire clause is in its scope, we can have verum \isi{focus}: e.g. in \refex{ex:These are NOT fish, right@8e} the polarity of the verb is at stake, not the lexical meaning conveyed by the verb.

\begin{exe}
	\ex	{[A said: You do not have musical instruments in your village?]}\label{ex:Well, by God, we do have them, he said@17c}\\%
	\gll	c'il	wallah	χe-b=q'al	χe-b,	∅-ik'ʷ-ar\\
		then	by.God	exist.\tsc{down-n=mod}	exist.\tsc{down-n}	\tsc{m-}say\tsc{.ipfv-prs}\\
	\glt	\sqt{Well, by God, we do have them, he said.}

	\ex	\label{ex:These are NOT fish, right@8e}
	\gll	baliqː-e	akːu=q'al,	ca-d=uw?\\
		fish\tsc{-pl}	\tsc{cop.neg=mod}	\tsc{cop-npl=q}\\
	\glt	\sqt{These are NOT fish, right?}
\end{exe}

Particles can be combined and are either encliticized to the same host \refex{ex:Again it was me who remained alone@26a}, \refex{ex:But you were the one who said that they had taken only Alibatir to the ditch@26b} or two different hosts. Example \refex{ex:‎Although I was little, I myself went behind the oxen, this is also work} contains the \isi{additive} and both modal \is{particle}particles occurring in one and the same utterance.

\begin{exe}
	\ex	\label{ex:‎Although I was little, I myself went behind the oxen, this is also work}
	\gll	nik'a-ce=xːare,	du-l=q'ar, it=ra	ʡaˁči=q'alle,	unc-e	sa-r-d-uc-ij	uˁq'-ni-lla\\
		small\tsc{-dd.sg=conc}	\tsc{1sg-erg=mod}	that\tsc{=add}	work\tsc{=mod}	ox\tsc{-pl}	\tsc{hither}\tsc{-abl-npl-}keep\tsc{.pfv-inf}		go\tsc{-msd-gen}\\
	\glt	\sqt{‎Although I (masc.) was little, I myself went behind the oxen, this is also work.}
\end{exe}

The \isi{particle} \tit{akːʷar} \sqt{except, without, only} differs from the other \is{particle}particles because it is morphosyntactically the negative \isi{copula} inflected for the third person of the \isi{habitual present}. It follows the item in focus\is{focus} and occurs in a clause with negative polarity. Its literal meaning is \sqt{(X) not being} and the negative clause refers to a situation that only takes place when X is present. For instance, \refex{ex:At that time there were only trucks constituents} can literary be translated as \sqt{There were no cars not being trucks.}

\begin{exe}
	\ex	\label{ex:At that time there were only trucks constituents}
	\gll	gurzawuj-te	akːʷ-ar	mašin-te	a-d-irχʷ-i=q'al	it=qːella\\
		truck\tsc{-pl} 	\tsc{cop.neg-prs.3}	car\tsc{-pl}	\tsc{neg-npl}-be\tsc{.ipfv-hab.pst=mod}	that=when\\
	\glt	\sqt{At that time there no cars except for trucks.} 

	\ex	\label{ex:‎Nobody knew it, only Mahammad and me}
	\gll	hil-k'al-li-j	b-alχ-ul	akːʷ-i	il	Maħaˁmmad-li=ra	du-l=ra	akːʷ-ar\\
		who\tsc{.obl-indef-obl-dat} \tsc{n-}know\tsc{.ipfv-icvb}	\tsc{cop.neg-hab.pst}	that	Mahammad\tsc{-erg=add}	\tsc{1sg-erg=add}	\tsc{cop.neg-prs}\\
	\glt	\sqt{‎Nobody knew it, except for Mahammad and me.}
\end{exe}

Modal \is{particle}particles such as \tit{=q'al} or \tit{=q'ar}, but also additives, scalar additives, or exclusive \is{particle}particles are often compared to or sometimes even equated with \isi{focus} \is{particle}particles. For instance, the Standard Dargwa \isi{particle} \textit{q'alli} is cognate with Sanzhi \tit{=q'al}. It has been called a ``sentence focus \isi{particle}'' by \citet[74\tnd75]{vandenBerg2001}.

However, all \is{particle}particles discussed in this section participate in the pragmatic structuring of the utterance, but they are not plain markers of \isi{focus}. They all have a lexical meaning, although the semantics of modal \is{particle}particles is comparatively hard to access. Whenever they are used, this meaning is expressed, but their use is not restricted to focus constructions. There are a \isi{number} of corpus examples in which the referent of the item bearing the \isi{particle} is not only not new, but has been mentioned in the immediately preceding utterance \refex{ex:‎‎‎They made injections into the hand for ten days}, \refex{ex:‎The fingers, the hands, I do not see that far}, \refex{ex:There is milk soup and soup from other things@21}. It is taken up again in the utterance containing a \isi{focus-sensitive particle}. In such examples the host is frequently not a verb, but a nominal, and occurs in clause-initial position. The host is not focal, but topical, and the construction is used to topicalize patients \refex{ex:‎‎‎They made injections into the hand for ten days}, \refex{ex:There is milk soup and soup from other things@21} or other semantic roles that are typically non-topical \refex{ex:‎The fingers, the hands, I do not see that far}. For a more thorough discussion of the functions of \tit{=q'al}, see \citet{ForkerSubmittedc}.

\begin{exe}
	\ex	{[Talking about Sanzhi dishes, the daughter of the speaker reminds her mother that there is also milk soup. Then the mother continues to talk about this topic.]}\label{ex:There is milk soup and soup from other things@21}\\%
	\gll	nejg-la	nerʁ-be=q'al,	cara-la=ra	d-irχ-u\\
		milk\tsc{-gen}	soup\tsc{-pl=mod}	other\tsc{-gen=add}	\tsc{npl-}be\tsc{.ipfv-prs}\\
	\glt	\sqt{There is milk soup and soup from other things, (we are also preparing milk soup\ldots)}
\end{exe}
