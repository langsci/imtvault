\documentclass[output=paper]{langsci/langscibook}
\ChapterDOI{10.5281/zenodo.4280641}

\author{Neil Myler\affiliation{Boston University}}

\title{Rethinking the reach of categorical constraints: The final-over-final
constraint and combinatorial variability}

% \chapterDOI{} %will be filled in at production

\abstract{This squib argues that categorical rules and constraints of the sort
    traditionally found in generative syntax can, in principle, make
    interesting and testable quantitative predictions about surface frequencies
    in language use, despite occasional claims to the contrary. Specifically,
    the final-over-final constraint (\glsunset{FOFC}\gls{FOFC},
    \citealt{BibHolRob2014,BibNewShee2009b,Holmberg2000,Walkden2009}; many
    others) is predicted to exert a specific influence on the likelihood of OV
    vs.\ VO word order in the language use of a speaker that allows both, given a
    \textsc{combinatorial variability} approach to intra-speaker syntactic
    variation (\citealt{Adger2006} et seq.).}

\maketitle

\rohead{\thechapter\hspace{0.5em}Rethinking the reach of categorical constraints}

\begin{document}\glsresetall

\section{Introduction}

Generative linguistics has traditionally employed categorical rules and\linebreak
constraints in its quest to understand the properties of the syntax of
particular languages and the properties of the syntactic component of the
language faculty more generally.  For this reason, its theoretical postulates
have often been taken to be either irrelevant to or at odds with the inherent
variability of language use (see \citealt{Guy2005}; \citealt{Newmeyer2005};
inter alia).

In this squib, I will argue that categorical constraints can, in fact, make
interesting and testable quantitative predictions about surface frequencies,
given a certain theory of how intra-speaker syntactic variation is to be
modeled.  More specifically, I will show that the \emph{final-over-final}
\emph{constraint}\footnote{Note that \glsunset{FOFC}\gls{FOFC} is referred to as
the \emph{\glsdesc{FOFC}} in some more recent work,
including~\textcite{SheeBibRobHol2017}.}
(\gls{FOFC} -- \citealt{BibHolRob2014,BibNewShee2009b,Holmberg2000,Walkden2009},
many others) should exert a specific influence on the likelihood of OV vs.\ VO
word order in the language use of a speaker that allows both, given a
\emph{combinatorial variability} approach to intra-speaker syntactic
variation (\citealt{Adger2006} et seq.).

The squib is structured as follows.  In \Cref{sec-30:key:1}, I introduce the
combinatorial variability approach, showing how it might be used to generate
predictions concerning the expected baseline surface frequencies of OV\,vs.\,VO
order in the speech of \ili{Quechua}--\ili{Spanish} bilinguals, focusing on DP complements
and the head-di\-rec\-tion\-a\-lity of VP and TP. In \Cref{sec-30:key:2}, I introduce
FOFC and demonstrate that the surface frequencies predicted by the
combinatorial variability approach change if \gls{FOFC}\is{final-over-final condition} is held to be valid.
In \Cref{sec-30:key:3}, I outline the prospects and challenges for testing
these predictions in a sociolinguistic study of actual Quechua--Spanish
bilinguals in Cochabamba, Bolivia. \Cref{sec-30:key:4} is a brief conclusion.

\section{Quechua--Spanish contact and combinatorial
variability}\label{sec-30:key:1}

To make the discussion of combinatorial variability more concrete, I will frame
this section around the specific example of language contact between speakers
of \ili{Quechua} and \ili{Spanish}.  Speakers of these two languages are in contact in
Peru, Bolivia, Ecuador, parts of Colombia, and parts of northern Chile and
northern Argentina. Many \ili{Quechua} speakers in these places are bilingual in
Spanish.  As is well-known, \ili{Quechua} and \ili{Spanish} are almost typological
opposites in terms of their basic word order. \ili{Quechua} is predominantly
head-final, as shown in the example from Cochabamba Quechua\il{Cochabamba Quechua} (a Bolivian
variety) in (\ref{bkm:Ref341518195}).  \ili{Spanish}, on the other hand, is a
head-initial language, as shown in (\ref{bkm:Ref341518219}).

\ea\label{bkm:Ref341518195}Cochabamba Quechua\il{Cochabamba Quechua}\\
    \gll Kay  runa  Cochabamba-man  ri-q ka-rqa.\\
            This man Cochabamba-to go-\Nmlz{} be-\Pst{}\\
    \glt    \enquote*{This man used to go to Cochabamba.}
\ex \label{bkm:Ref341518219}\ili{Spanish}\\
    \gll Este hombre ha ido a Cochabamba.\\
            This  man      has   gone    to  Cochabamba\\
    \glt    \enquote*{This man has gone to Cochabamba.}
\z

Pre-theoretically, one might expect contact between \ili{Quechua} speakers and
Spanish speakers to give rise to mutual influence on word order, such that
head-initial orders increase in \ili{Quechua} usage, and/or head-final ones increase
in \ili{Spanish} usage, depending on the degree of bilingualism of the speaker,
attitudes towards each language, and so on.  Indeed, such has been reported in
the literature on Andean \ili{Spanish} (e.g., \citealt{Muntendam2008};
\citealt{Muysken1984}; \citealt{Sanchez2003}) and in studies of the influence
of \ili{Spanish} on \ili{Quechua} (\citealt{Camacho1999}; \citealt{Hintz2009};
\citealt{Sanchez2003}, 2012).  Let us now turn to the combinatorial variability
approach, and how it might analyze such variation.

Comparative syntax research within the Minimalist program has pursued the idea
that syntactic variation across languages\slash dialects should be analyzed only in
terms of variation in the featural needs of functional items (the so-called
Borer-Chomsky conjecture, as it is dubbed by \citealt{Baker2008b}; see
\citealt{Borer1984}; \citealt{Chomsky1995}).  This presents a generativist
pathway to \emph{orderly} \emph{heterogeneity} in the sense of
\textcite{WeiLabHer1968}: Suppose that an individual’s lexicon contains
function morphemes with the same categorial feature and the same contribution
to truth conditions (and thus roughly the same distribution), but which differ
in one or more of their morphosyntactic features.  Then, the choice of one or
the other lexical item in a derivation will result in somewhat different
outputs, but with no difference in meaning.  Thus, there will be an appearance
of syntactic optionality, but in reality the only optionality is in lexical
choice: once particular lexical items have been chosen, the syntactic
derivation is fully determined.  This is the essence of
\citeauthor{Adger2006}'s (\citeyear{Adger2006} et seq.) proposed
reconciliation of Minimalist syntax with sociolinguistic
variation.\largerpage[2]

As \citet{Adger2006} points out, it is possible to calculate quantitative
predictions about variability which arise from the combinatorics of the
relevant syntactic elements (hence the name \emph{combinatorial variability} 
for the overall approach).  Take lexical items A, B, and
C; all with identical truth-conditional meaning but with distinct syntactic
features.   A and B, when chosen, give rise to a series of derivational steps
S\textsubscript{1}.  C, on the other hand, differs in some aspect of its
feature content from A and B, and thus gives rise to a distinct derivation
S\textsubscript{2}, whose output differs on the surface from
S\textsubscript{1}.  This will give the appearance of syntactic variability.
All else held equal, a prediction is made about the nature of that variability.
Since two out of a possible three lexical choices give rise to
S\textsubscript{1}, but only one choice yields S\textsubscript{2}, the
prediction is that the output corresponding to S\textsubscript{1} should appear
in usage two thirds of the time, and the output of S\textsubscript{2} should
appear one third of the time.\footnote{This follows only if no other factors
favor A, B, or C over the others, so that the choice is determined by chance.
In actual use, of course, the probability distribution predicted by purely
syntactic combinatorics will be modulated by sets of factors influencing
lexical choice itself, including sociolinguistic factors.  I return to this
issue below.}

Returning to our example from Quechua--Spanish contact, we will now examine the
baseline frequencies of OV and VO word order that a combinatorial variability
approach would predict.  First, we need an inventory of the syntactic
microparameters that are relevant to analyzing word-order differences between
the two languages.%\largerpage[2]

The first is \textsc{head-directionality of the vp}.\footnote{For simplicity I
    will assume the traditional head parameter in the ensuing discussion, but
    nothing I have to say is incompatible with an antisymmetric approach to the
    relationship between structure and linearization (see \citealt{Kayne1994}).
Since Kayne’s linear correspondence axiom is a key component of many existing
approaches to deriving FOFC, this is good news.}  In \ili{Spanish}, the head of VP is
on the left (this value will be denoted \enquote{L} for short).   In \ili{Quechua}, the head
of the VP is on the right (\enquote{R} for short).

The second parameter is \textsc{head-directionality of the tp}.  This
parameter, of course, is directly analogous to the first.  \ili{Spanish} T is on the
left, and \ili{Quechua} T is on the right.  This parameter has a direct influence on
where the verb surfaces relative to its complement, because T in these
languages attracts the verb (i.e., there is \is{verb movement}V-to-T movement).  \is{verb movement}V-to-T movement
is known to apply in \ili{Spanish} because of the placement of VP-peripheral \isi{adverbs}
relative to the verb and the direct object (\citealt{Pollock1989};
\citealt{Zagona2002}).\footnote{I assume here that T is the relevant landing
    site in all cases, but this is certainly an oversimplification.  See
    \textcite{Schifano2015,Schifano2018} for evidence that considerably more
    granularity is needed, with \isi{verb movement} targeting different positions in
    the Cinquean extended IP (\citealt{Cinque1999} et seq.) in different
    languages.  This does not affect the main point here, so long as
    \isi{verb movement} is to a landing site higher in the structure than the final position
    of the direct object.  Thanks to an anonymous reviewer for raising this issue.}\largerpage[2]

\ea \ili{Spanish}\\
    \gll    Juan abrió     cuidadosamente la    puerta.\\
            Juan opened  carefully             the door\\
    \glt    \enquote*{Juan carefully opened the door.}\\[-.5em]
    \begin{tikzpicture}[baseline, scale=.9]

        \Tree   [.TP
                    [.DP \edge[roof]; {Juan} ]
                    [.T$'$
                        [.T
                            \node (goal) {V\\abrió};
                            T
                        ]
                        [.VP
                            [.AdvP \edge[roof]; {cuidadosamente} ]
                            [.VP
                                \node (t) {\sout{V}\\\sout{abrió}};
                                [.DP \edge[roof]; {la puerta} ]
                            ]
                        ]
                    ]
                ]

        \draw [overlay,arrow, bend left=40] (t.south) to (goal.south);
    \end{tikzpicture}
\z

It is much more difficult to ascertain whether or not there is \is{verb movement}V-to-T movement
in \ili{Quechua}, since both VP and TP are head-final in that language, and this
makes it impossible to check whether the verb “crosses over” \isi{adverbs} at the
edge of VP.  The empirical evidence we have to hand is therefore compatible
with \is{verb movement}V-to-T movement being present or absent in \ili{Quechua}.  However, there is one
typological consideration which weighs in favor of assuming that \ili{Quechua} does
have \is{verb movement}V-to-T movement.  The syntactic literature has found that VO languages
with rich agreement inflection on the finite verb always have \is{verb movement}V-to-T movement
(\citealt{Kosmeijer1986}; \citealt{Pollock1989}; see \citealt{KoeZei2012} for a
recent reaffirmation of this correlation).  Since \ili{Quechua} has extremely rich
agreement inflection on its finite verbs, we may assume it has \is{verb movement}V-to-T movement
also.\footnote{An anonymous reviewer points out that there remain a number of
    potential problems for this conclusion (referring to \citealt{Vikner2005,%
    HanLidzMus2007,HanMusLidz2016}).  This must be borne in mind, because if it
    turns out that \ili{Quechua} lacks \is{verb movement}V-to-T, then another test-bed for the
quantitative predictions of \gls{FOFC}\is{final-over-final condition} would need to be found.  The broader
point of this squib, that such predictions are formulable and testable in
principle, stands regardless.}

To see why this matters for surface word-order, consider the case of a
derivation in which VP-headedness has the \ili{Quechua} \enquote{R} value, but TP-headedness
has the \ili{Spanish} \enquote{L} value.  In such a case, the surface word order will be VO
in spite of the fact that the structure is “underlyingly” OV, because of \is{verb movement}V-to-T
movement.

\ea \is{verb movement}V-to-T movement obscures head-finality of VP\\
    \begin{tikzpicture}[baseline, scale=.9]
        \Tree 	[.TP
                    [.DP \edge[roof]; {Juan} ]
                    [.T$'$
                        [.T
                            \node (goal) {V\\abrió};
                            T
                        ]
                        [.VP
                            [.AdvP \edge[roof]; {cuidadosamente} ]
                            [.VP
                                [.DP \edge[roof]; {la puerta} ]
                                \node (t) {\sout{V}\\\sout{abrió}};
                            ]
                        ]
                    ]
                ]
        \draw [arrow, bend left=60] (t.south) to (goal.south);
    \end{tikzpicture}
\z\largerpage

Given these basic assumptions about clause structure and the points of
parametric variation which differentiate \ili{Spanish} and \ili{Quechua}, we can now ask
about the predictions of combinatorial variability for the baseline frequencies
of OV vs.\ VO order.

Let us assume that a bilingual speaker is able to represent syntactic objects
from each language in much the same way as a monolingual speaker.  That is, a
bilingual speaker has access to a left-headed VP structure much as a
monolingual \ili{Spanish} speaker does, and also has access to a right-headed VP
structure in the same way that a monolingual \ili{Quechua} speaker does.  Similarly,
the bilingual’s functional lexicon will contain a lexical item T which takes
its complement to its right, Spanish-style, and another lexical item T which
takes its complement to the left, Quechua-style, and so on for other syntactic
objects.  Of course, in making utterances, bilingual speakers will have to make
a choice between these options.  It turns out that the different parameter
settings discussed above, simply through the nature of their logically possible
combinations, give rise to quantitative predictions about what the baseline
frequencies of these different choices should be.

For the purposes of simplicity, I will concentrate on DP direct objects only.
The calculations below would have to be somewhat different for QP and CP
complements.  In the case of QPs, the fact that \ili{Quechua} allows overt scrambling
for scope would somewhat increase the chance of OV order surfacing, relative to
non-quantificational DPs.  For CPs, the possibility of clausal extraposition in
both languages would boost the predicted baseline frequency of VO order.

There are $2*2=4$ possible combinations of parameter settings relevant here,
shown below.

\ea\label{bkm:Ref341539124}Combinations of parameter settings: DPs\\
    \begin{tabularx}{.4\textwidth}{ll}
    \lsptoprule
    Combination A & Output: VO\\
    \midrule
    Parameter     & Setting\\
    VP-headedness & L\\
    TP-headedness & L\\
    \lspbottomrule
    \end{tabularx}
    \quad
    \begin{tabularx}{.4\textwidth}{ll}
    \lsptoprule
    Combination B & Output: OV\\
    \midrule
    Parameter     & Setting\\
    VP-headedness & R\\
    TP-headedness & R\\
    \lspbottomrule
    \end{tabularx}\\
    \begin{tabularx}{.4\textwidth}{ll}
    \lsptoprule
    Combination C & Output: VO\\
    \midrule
    Parameter     & Setting\\
    VP-headedness & R\\
    TP-headedness & L\\
    \lspbottomrule
    \end{tabularx}
    \quad
    \begin{tabularx}{.4\textwidth}{ll}
    \lsptoprule
    Combination D & Output: OV\\
    \midrule
    Parameter     & Setting\\
    VP-headedness & L\\
    TP-headedness & R\\
    \lspbottomrule
    \end{tabularx}
\z

Hence, the logically possible combinations predict a 50/50 split between VO
orders and OV orders for DPs.\pagebreak

\ea\label{bkm:Ref342038579}VO vs. OV order with DP complements
    \sn VO = 2/4 outputs = 50\%
    \sn OV = 2/4 outputs = 50\%
\z

\section{Bringing in the final-over-final constraint (FOFC)}\label{sec-30:key:2}

The \emph{final-over-final constraint} of \textcite[171]{BibHolRob2014} has an
interesting effect on this calculation.\glsunset{FOFC}

\ea \emph{The final-over-final constraint} (FOFC)\\
    A head-final phrase αP cannot dominate a head-initial phrase βP, where α
    and β are heads in the same extended projection.
\z

This constraint will, of course, make the categorical prediction that V-O-Aux
orders will be absent from compound tenses in the \ili{Spanish} and the \ili{Quechua} of
bilinguals.  In addition, however, \gls{FOFC}\is{final-over-final condition} has a quantitative effect.  In
particular, it rules out combination D in (\ref{bkm:Ref341539124}), because
that combination involves a head-final TP dominating a head-initial VP.  In
terms of the predicted baseline surface frequencies, we thus obtain the
following results instead of the ones we saw in (\ref{bkm:Ref342038579}):

\ea VO vs.\ OV order with DP complements (if \gls{FOFC}\is{final-over-final condition} is valid)
    \sn VO = 2/3 outputs = 67\%
    \sn OV = 1/3 outputs = 33\%
\z

This is an exciting finding, because it shows that categorical constraints can
give rise to stochastic effects, meaning that such constraints \emph{are} of
potential relevance to variationist work after all.  This result emerges from
the fact that combinatorial variability derives quantitative predictions by
looking at the interaction of different parameter settings, and universal
constraints like \gls{FOFC}\is{final-over-final condition} take certain combinations of parameter settings out of
the picture.  Another intriguing consequence of this result is that it becomes
possible, in principle, to use variationist data to test the predictions of
such universal constraints.  Since the baseline frequencies predicted are
different if \gls{FOFC}\is{final-over-final condition} holds than they are if it does not, in principle it becomes
possible to test \gls{FOFC}\is{final-over-final condition} by seeing how the variationist data pan out.  In the next
section, I examine the prospects for doing this.

\section{Testing the predictions: Prospects and challenges}\label{sec-30:key:3}\largerpage

It is clear what the signature of \gls{FOFC}\is{final-over-final condition} should be in quantitative data: because
FOFC bars one of the logically possible routes to OV word order, OV should be
less common than VO all else held equal if \gls{FOFC}\is{final-over-final condition} is valid.  If \gls{FOFC}\is{final-over-final condition} is not
valid, then OV and VO should be equally frequent, all else held equal.

The challenge in testing predictions of this sort, of course, is that all else
is seldom equal, and a range of social factors that have been discussed in the
sociolinguistics literature will also influence the actual surface frequencies
of the orders.  These must be controlled for or accommodated somehow if the
signature of \gls{FOFC}\is{final-over-final condition} is to be detected.  Most obviously, although the literature
reports mutual influence between \ili{Spanish} and \ili{Quechua} word orders, it still
might be the case that speakers have some (presumably subconscious) sense that
Quechua exhibits more head-finality. If so, language mode would be expected to
favor OV when the speaker is talking in \ili{Quechua}, and VO when the speaker is
talking in \ili{Spanish}.  Such an effect would be especially likely if the VO vs.\ OV
difference turned out to be a socially salient linguistic variable.

The issue of social salience raises the possibility that speakers might use OV
vs.\ VO order as a way of indexing particular identity categories, including
attitudes to \ili{Quechua} and \ili{Spanish}, orientation towards or away from indigenous
culture, and so on.  Since exposure to standard \ili{Spanish} will favor VO order,
degree of education is another factor to be considered. In addition, of course,
degree of bilingualism/proficiency in each language would be expected to be
relevant.

Finally, there is a presupposition of the combinatorial variability approach
which itself has yet to be tested;  namely, the idea that the probability that
a given variable will be used is determined by chance if no other factor
intervenes.  This assumption is not unreasonable, but nor is it certain to be
correct -- we still await an empirical demonstration that it is on the right
track.

In an ongoing collaboration, the sociolinguist Daniel Erker\ia{Daniel Erker}
and I have carried out a pilot study involving demographic/attitudinal surveys,
sociolinguistic interviews, reading passage data, and grammaticality judgments
on both \ili{Spanish} and \ili{Quechua} as spoken in Cochabamba, Bolivia.  The data set
includes 19 speakers: 4 monolingual \ili{Spanish} speakers, and 15 Quechua--Spanish
bilinguals.  For the bilinguals, we have interview data, reading passage data,
and grammaticality judgment data on both languages.     The analysis of this
data is still in progress.  As well as addressing a number of issues in the
sociolinguistics of language contact, we hope that a full version of this study
(including monolingual \ili{Quechua} speakers, and many more speakers overall) will
allow us to test the quantitative predictions of FOFC, and the predictions of
the combinatorial variability approach more generally.

\section{Conclusion}\label{sec-30:key:4}

This squib has shown that categorical principles and constraints can make
predictions about apparently non-categorical phenomena.  Testing those
predictions, however, is a difficult and delicate task, one that is not yet
within our reach from a practical standpoint.  Bringing it within our reach
will require the collaboration of formal linguists and sociolinguists.

\printchapterglossary{}

\section*{Acknowledgements}

For Ian Roberts, the syntactic Eric Cantona.

This squib is an outgrowth of ongoing joint work with Daniel Erker on
Que\-chua--Spanish contact, including a poster we jointly presented at the annual
meeting of the Linguistic Society of America in 2016. I would like to thank the
audience of that poster presentation, a colloquium audience at UPenn, anonymous
reviewers from the National Science Foundation, David Adger, Byron Ahn, Carol
Neidle, Cathy O’Connor, and Ian Roberts for their feedback on various aspects
of my and Daniel’s joint work. Thanks also to two anonymous reviewers and to
Daniel Erker for their comments on an earlier version of this squib. Errors and
infelicities are on my head alone.

{\sloppy\printbibliography[heading=subbibliography,notkeyword=this]}

\end{document}
