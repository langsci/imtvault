\documentclass[output=paper]{langsci/langscibook}
\ChapterDOI{10.5281/zenodo.4280633}
\author{István Kenesei\affiliation{Research Institute for Linguistics,
Budapest, \& University of Szeged}}
\title{Life without word classes: On a new approach to categorization}

% \chapterDOI{} %will be filled in at production

\abstract{This is an attempt to redefine word classes, or more precisely, to
    replace the concept of word class with clusters of properties much like the
    notion of the phoneme is dissolved into the various combinations of
    distinctive features. It is claimed that word classes are but comfortable
    generalizations not supported by hard evidence as seen in examples from a
    select group of languages and illustrated in detail by the list of
    auxiliaries in Hungarian.}

\maketitle

\begin{document}\glsresetall

\section{Introduction and overview}\label{sec:27.1}

The problem of the definition of word classes has been with us since the very
beginnings of linguistics. The first grammars already provided terms according
to which to classify words. Dionysius Thrax (BCE 170--90) lists the following
eight classes: noun, verb, participle, article, pronoun, preposition, adverb,
conjunction. The definitions are simple, familiar, and of course mostly
notional, e.g.,

\begin{quote}
A Noun is a declinable part of speech, signifying something
either concrete or abstract (concrete, as stone; abstract, as education);
common or proper (common, as man, horse; proper, as Socrates, Plato). It has
five accidents: gender, species, forms, numbers, and cases.\\
\hbox{}\hfill\hbox{(\emph{The grammar of Dionysios Thrax}, this citation from \citealt[331]{Davidson1874})}
\end{quote}
The classical definitions have followed us well into the 20th century. To
quote another example, this is what the Port-Royal philosophers had to say
about parts of speech in the 17th century:

\begin{quote}
Les objets de nos pensées, sont ou les choses, comme \emph{la terre}, \emph{le
Soleil}, \emph{l’eau}, \emph{le bois}, ce qu’on appelle ordinairement
\emph{substance}. Ou la manière des choses; comme d’estre \emph{rouge}, d’estre
\emph{dur}, [\dots] \& c.\ ce qu’on appelle \emph{accident}. [\dots] Car ceux
qui signifient les substances, ont esté appellez \emph{noms substantifs}; \&
ceux qui signifient les accidens [\dots], \emph{noms} \emph{adjectifs}.\\
\hbox{}\hfill\hbox{\parencite[30--31]{LancelotArnauld1660}}
\end{quote}

This type of definition was widespread until about the middle of the 20th
century. In his otherwise highly original \emph{Grammar of spoken English},
\citet{Palmer1924} lists more or less the same eight classes, viz., nouns,
pronouns and determinatives, qualificatives (i.e., adjectives), verbs, \isi{adverbs},
prepositions, connectives (“together with interrogative words”), and
interjections and exclamations. In the “logical classification of nouns”, for
instance, he gives an inventory of subtypes, rather than a classical
definition, namely, concrete nouns (including proper and common nouns, with the
latter further divided into class, i.e.\ countable, and material nouns, etc.)
and abstract nouns \citep[28--32]{Palmer1924}.

However, due to the influence of Saussure’s \emph{Cours}
(\citeyear{Saussure1916}), American descriptive linguists, and in particular
Leonard Bloomfield, who was the first of them to appreciate Saussure’s
achievements (cf., e.g., \citealt{Koerner1995}), started to concentrate on the
formal features of parts of speech. “The noun is a word-class; like all other
form-classes, it is to be defined in terms of grammatical features [\dots{}] When it
has been defined, it shows a class-meaning which can be roughly stated as
\enquote{object of such and such a \emph{species}}; examples are \emph{boy},
\emph{stone}, \emph{water}, \emph{kindness}.” \citep[202]{Bloomfield1935} One
of Bloomfield’s more dogmatic followers had this to say in his widely used
textbook:

\begin{quote}\relax
[The pattern of interchangeability] defines a form-class which
includes \emph{she}, \emph{he}, \emph{it}, \emph{John}, \emph{Mary}, \emph{the
man at the corner}, \emph{my friend Bill}, and so on endlessly, but which by no
means includes all forms, since we can name many which are excluded:
\emph{her}, \emph{him}, \emph{them}, \emph{me}, \emph{yes}, \emph{no},
\emph{ripe}, \emph{find her}, \emph{go with us tomorrow}.\\\hbox{}\hfill%
\hbox{\citep[162]{Hockett1958} }
\end{quote}
Note that Hockett’s form-classes include not only
words proper, but entire\linebreak phrases, and there is no \enquote{class-meaning}
mentioned, since the most important feature is mutual substitutability.\largerpage[2]

But if distributional analysis is closely observed, its negative consequences
are unavoidable, as was seen as early as the 1960s. According to one British
linguist \enquote{as many classes are set up as words of different formal
behaviour are found} \parencite[174]{Robins1980}, and another maintains in an
article on the definition of word classes that “[\dots{}] very few words have
an overall identical formal behaviour [\dots{}]. One would end up with a
multitude of single member classes” \citep[28]{Crystal1967}. Or to cite a more
recent article: “Whatever identifying criteria we use for parts of speech –
meaning, syntactic function, or inflection – the relationship between
particular criteria and particular parts of speech is typically many-to-many”
\citep[3]{Anward2000}.

Neither do alternative approaches fare better in this respect. Functionalist
linguists, as shown by Simon \citet{Dik1989} or Kees \citet{Hengeveld1992},
differentiate word classes by two prototypical functions or \isi{parameters}, such as
predication vs.\ referentiality, and head vs.\ modifier, with the resulting
four classes arranged in an implicational hierarchical order\is{implicational
relations} in \REF{ex:27.1}
that corresponds to the sequence verb~> noun > adjective > adverb
\citep{Hengeveld1992}.

\ea\label{ex:27.1}%
    \begin{tabularx}{\linewidth}[t]{QcQcQcQ}
        Head of\newline pred. phrase & > & Head of ref. phrase & > &
        Modifier of head of ref. phrase & > & Modifier of head of pred. phase\\
    \end{tabularx}
\z

The \enquote{radical constructionist} William \citet{Croft2005} also notes the
futility of the distributional method, and, instead of language specific word
classes, proposes restricted typological universals based on “propositional
acts”, such as reference, predication, and modification, that define “lexical
semantic classes” like objects, actions, and properties, respectively
\parencite[438]{Croft2005}.

As I will try to show, neither the approach based on the introduction of a new
or different set of criteria for the same small number of word classes nor the
opposing view stemming from otherwise well-established criticism based on the
failure of distributional analysis is viable. Instead, I will suggest a
compromise solution that benefits from both without their possible
drawbacks.\largerpage[2]

Research into the typology of word classes has come up with observations
differentiating between part-of-speech systems depending on whether or not the
categories of lexical items are fixed or not. Languages can thus be grouped
into one of three sets: (a) differentiated, as English, in which all four word
classes are clearly displayed, and two subtypes in which such dedicated lexical
items are missing: (b) flexible, like \ili{Turkish}, in which non-verbs can
belong to any one of the three classes nouns, adjectives, and \isi{adverbs}, and (c)
rigid, like \ili{Krongo} (Kadu, Sudan), in which there are nouns and verbs, but the
rest of the lexical categories are rendered by syntactic means, e.g., relative
clauses \parencite[32ff.]{Hengeveld2013}.\footnote{Due credit must be given here
    to the polyglot phonologist and theoretical linguist Ferenc Mártonfi
    (1945--1991), who had expressed similar thoughts well ahead of the recent
    upsurge of interest in word class typology, as illustrated in the following
    passage.  \enquote{From the point of view of parts-of-speech this means
        that there are languages in which syntactic features like ’verbal’ or
        ’nominal’ must be marked for all or most of the words (e.g., in
        \ili{Hungarian}, \ili{German}, etc.), and there are languages where this would be
        redundant, non-distinctive marking, which is omissible (and this holds
        for the large majority of words in, e.g., Chinese, Vietnamese, etc.
[\dots]).  In other words, this means that lexical word classes are not
universal.} (\citealt[201]{Martonfi1973}; my translation)}

It is true that \isi{Distributed Morphology} offers an attractive solution to
the problem of word classes by merging a functional category with an
unspecified root (cf.\ \citealt{HalleMarantz1993,Marantz1997,Arad2003,Panagiotidis2015},
among others). In this approach,
categorization\is{lexical categories} is a syntactic process. Items, whether
heads or phrases, have no categories of their own determined by their lexical
characterization, but acquire them, as it were, by becoming complements of
functional heads, such as the nominalizer \emph{n}, the verbalizer \emph{v}, or
the adjectivizer \emph{a} \parencite[17]{Panagiotidis2015}. However,
\citegen[266ff.]{Baker2003} arguments are persuasive in attributing syntactic
categories\is{lexical categories} to roots or stems, particularly, as I would
focus on his proposal in the light of the above typology, in the case of a
number of languages in the \enquote{differentiated} type, which will be the
subject of our discussion below. Baker claims that “where there is less
functional structure, we find more categorial distinctiveness”
\parencite[268]{Baker2003}.

\section{Properties rather than definitions}\label{sec:27.2}

Traditional part-of-speech characterizations usually list the most general
properties and illustrate them by prototypical examples, which serve
practically as ostensive definitions, thus rendering the characterization
itself redundant since the examples are a sufficient ground for any competent
native speaker by means of which to classify the words of the language in
question. The criteria, which usually rely on distributional and/or semantic
factors, are usually too soft or porous, and the classes set up do not directly
follow from the definitions.

At the same time these very definitions preclude the establishment of, for
example, the uniform class of verbs in English or in other languages of the
differentiated type since intransitive verbs are as a rule incapable of
substituting for transitive ones, or mass nouns for countable nouns, and so
forth. If, however, we are satisfied with partial overlapping, then the class
of adjectives will in part coincide with that of nouns, cf. \emph{Italian} or
\emph{(the) blind}, or even adjectives will subsume two partially overlapping
subsets, relational and qualitative ones, cf. \emph{(*more) naval (exercise)}
vs.\ \emph{(more) interesting exercise}. In addition to flexible word classes
(cf.\ \citealt{RijkhoffvanLier2013}), some dispute the distinction between
inflection and derivation as well, positing a continuum for them
\citep{Dressler1989}. What is to blame in this state of affairs is the metric
applied; if we have a single scale, the difficulties will inevitably resurface
again.

Moreover, it follows from a unidimensional system of criteria that whenever
some word class is defined by a set of characteristics, then a given item
belongs to that word class if it has precisely those characteristics. If any
item has some property that it shares with another item, the property will
serve to determine the class formed by them. This is clearly circular and if we
insist on this approach the circle cannot be broken.

Note that the notion of word class applies only to linguistic items that can
combine with other such items. Utterance-sized words, such as interjections,
greetings, etc., even though they may be listed and categorized\is{lexical
categories} in dictionaries, do not partake in syntactic constructions (except
in citation forms), thus, theoretically speaking they have no properties
comparable to those of \enquote{ordinary} word classes, while the labels
attached to them certainly have a practical advantage for users of these
dictionaries.

It is precisely the (morphological, syntactic, semantic, or pragmatic)
properties of combinable lexical items relevant from the viewpoint of
categorization\is{lexical categories} that control their cooccurrence with other lexical items.
Consequently, there will be as many classes as there are properties, thus
vindicating \citegen{Robins1980}, \citegen{Crystal1967}, or
\citegen{Anward2000} views of a multitude of word classes. But these
definitions will no longer be circular since the criteria they are based on
will figure in various levels of grammar in determining the combination of
items, that is, in morphology, syntax, semantics, and pragmatics.

Consequently, what we understand by a word class will be a set of instructions
specifying what other lexical or syntactic objects, whether affixes, words or
syntactic phrases, a given word can combine with. \enquote{Traditional} word
classes, i.e., nouns, verbs, adjectives, \isi{adverbs}, satisfy various clusters of
properties.  In effect, the unidimensional category of word class has been
replaced by multidimensional matrices of sets of properties.

A similar suggestion is inherent in \citeauthor{Crystal1967}’s
(\citeyear{Crystal1967}: 46) list of criteria for
nouns in English, reproduced in \figref{fig:27.1}.

\begin{figure}
    \centering
    \begin{tikzpicture}[baseline, font=\smaller, align=left, scale=.66]

        \node (1) at (0, 4.5) {1};
        \node (2) at (0, 3.5) {2};
        \node (3) at (-1.25, 1.5) {3};
        \node (4) at (-1.25, 0.15) {4};

        \node (news) at (-7, 2.5) [fill=black!20]
            {$+$1\\$-$2\hphantom{\hspace{1em}\enquote{information}}\\\hspace*{1em}e.g.\\$+$3\hspace{1em}\enquote{news}\\$-$4};

        \node (info) [below=.5cm of news.south west, anchor=north west,
            fill=black!20]
            {$+$1\\$-$2\\\hspace*{1em}e.g.\\$+$3\hspace{1em}\enquote{information}\\+4};

        \node (cent) [below=.5cm of info.south west, anchor=north west,
            fill=black!20]
        {central class\\%
        $+1$\hphantom{\hspace{1em}\enquote{information}}\\%
        $+2$\hspace{1em}\enquote{hardship}\\%
        \hspace*{1em}e.g.\\%
        $+3$\hspace{1em}\enquote{peroration}\\+4};

        \node (girl) at (7, 2.0) [fill=black!20]
            {$+$1\hphantom{\hspace{1em}\enquote{information}}\\%
                $+$2\hspace{1em}\enquote{boy}\\%
                \hspace*{1em}e.g.\\%
                $+$3\hspace{1em}\enquote{girl}\\%
            $-$4};

        \node (phon) [below=.5cm of girl.south west, anchor=north west,
            fill=black!20]
            {$+$1\hphantom{\hspace{1em}\enquote{information}}\\%
                $-$2\\%
                \hspace*{1em}e.g.\\%
                $-$3\hspace{1em}\enquote{phonetics}\\%
            $-$4};

% %         \node (legend) [below=.5cm of phon.south west, anchor=north west]
% %             {Legend:\\
% %              1. May act as subject\\%
% %              2. Inflect for number\\%
% %              3. Co-occur with article\\%
% %              4. Morphological indication};

        \draw [->] (news.east) to (-2, -1.0);
        \draw [->] (info.east) to (0, -2);
        \draw [->] (cent.east) to (0, 0);
        \draw [->] (girl.west) to (4);
        \draw [->] (phon.west) to (0, -3.5);

        \draw (0, 0) circle (4cm);
        \draw (0, 1) circle (2cm);
        \draw (1, -1) circle (2cm);
        \draw (-1, -1) circle (2cm);
        \draw [clip] (0, 1) circle (2cm);
        \draw [clip] (1, -1) circle (2cm);
        \draw [clip] (-1, -1) circle (2cm);
        \draw (1, -1) [fill=black!20] circle (2cm);

        \draw [->] (cent.east) to (0, 0);
        \draw [->] (girl.west) to (4);
        \draw [->] (phon.west) to (0, -3.5);

    \end{tikzpicture}
    \caption{\citegen{Crystal1967} criteria for nouns. \emph{Legend}:
             1 -- May act as subject;
             2 -- Inflect for number;
             3 -- Co-occur with article;
             4 -- Morphological indication.\label{fig:27.1}}
\end{figure}

\citet{Gross1986} gives a classification of \ili{French} verbs according to the types
of subjects, complements and the properties of their complements, based on 4
subject and 32 complement types, setting up a matrix of 36 verb types.

In a discussion of the problems of universal and language specific
classification \citet[94]{Haspelmath2012} presents the overlapping system of
word classes in \ili{Chamorro}, following \citet{Topping1973} and \citet{Chung2012},
according to the properties and classes as in \tabref{tab:27.1}.

\begin{table}
\fittable{\begin{tabular}{lcccc}
\lsptoprule
                                & \multicolumn{4}{c}{Word type}\\\cmidrule(lr){2-5}
Property                        & ‘see’ & ‘go’ & ‘big’ & ‘person’\\\midrule
passive                         & $+$                & $-$               & $-$                & $-$\\
\emph{Yo’-}type pronoun subject & $-$                & $+$               & $+$                & +\\
Infinitive                      & $+$                & $+$               & $+$                & +\\
Incorporation                   & $-$                & $-$               & $-$                & +\\
Prefixation with \emph{mi-}     & $-$                & $-$               & $-$                & +\\
Subject-predicate agreement     & $+$                & $+$               & $+$                & $-$\\
Specific external agreement     & $+$                & $+$               & $-$                & $-$\\
Person-number agreement (realis) & $+$                & $-$               & $-$                & $-$\\
                                & Transitival      & Intransiverb     & Adjectival      & Nominal\\
\lspbottomrule
\end{tabular}}
\caption{\citegen{Haspelmath2012} extension of \citegen{Chung2012} table of
\mbox{grammatical} properties and clauses in Chamorro}\label{tab:27.1}
\end{table}

In contrast with more \enquote{regular} languages like \ili{Latin}, which has
the two major classes of verbs and nouns, with the two subclasses nouns
(\emph{nomen substantivum}) and adjectives (\emph{nomen adjectivum}) in the
latter group as distinguished by properties of having case and (in)variable
gender, Haspelmath argues that \ili{Chamorro} has six possible word class
systems in view of the properties in \tabref{tab:27.1}, as illustrated in
\figref{tab:27.2}.

\begin{figure}
\fbox{\resizebox{\textwidth - 3\fboxsep}{!}{\begin{tabular}{l ccc}
A & Transitival $+$ Intransiverb & \multicolumn{2}{c}{Adjectival $+$ Nominal}\\\cmidrule(lr){2-2}\cmidrule(lr){3-4}
B & Transitival & \multicolumn{2}{c}{Intransiverb $+$ Adjectival $+$ Nominal}\\\cmidrule(lr){2-2}\cmidrule(lr){3-4}
C & \multicolumn{2}{c}{Transitival $+$ Intransiverb $+$ Adjectiverb} & Nominal\\\cmidrule(lr){2-3}\cmidrule(lr){4-4}
D & Transitival $+$ Intransiverb & Adjectival & Nominal\\\cmidrule(lr){2-2}\cmidrule(lr){3-3}\cmidrule(lr){4-4}
E & Transitival & Intransiverb & Adjectival $+$ Nominal\\\cmidrule(lr){2-2}\cmidrule(lr){3-3}\cmidrule(lr){4-4}
F & Transitival & Intransiverb $+$ Adjectival & Nominal\\\cmidrule(lr){2-2}\cmidrule(lr){3-3}\cmidrule(lr){4-4}
\end{tabular}}}
\caption{The six possible word class systems of Chamorro according to
\textcite{Haspelmath2012}}\label{tab:27.2}
\end{figure}

The properties in question can be of various ranks and significance, as claimed
by \citet{Crystal1967}, since some may extend to more items than others, e.g.,
whether or not it can be a subject, take a definite article, etc. Then there
are classes that can easily adopt new items, whereas others do not – a familiar
distinction between open and closed classes. But closed classes, i.e.,
grammatical words or functional categories, do not form unified classes at all.

This was shown, for example, by \citet{Radford1976} in classifying English
auxiliaries by listing six properties distinguishing auxiliaries from verbs,
such as the ability to take negative clitics, to take \emph{do}-support, to
nominalize, to occur in untensed clauses, to occur in untensed clauses, to take
\emph{to} before a following infinitive, and to display concord, all of which,
except for the first, are properties characterizing verbs.

\citet{Aarts2007} differentiates between subjective and intersective gradience,\linebreak
where the former is a case of “categorial shading in prototypicality from a
central core to a more peripheral boundary” in a single category, while in the
latter “there are two categories on a cline” (p. 97). Rendered in the framework
presented here, it is the relevance and/or number of features from one or the
other word class that determine to what degree the item in question belongs
to one or the other category in Aarts’ intersective gradience.\is{lexical
categories}

If we examine auxiliaries in \ili{Hungarian}, we can identify the following
properties that distinguish them from main verbs that also take infinitives as
their complements.\footnote{Note that the first two properties \REF{ex:27.2}
    and \REF{ex:27.3} below lump together subclasses of main verbs with (some)
auxiliaries.}\largerpage[-2]

\begin{exe}\label{ex:27.2}
    \ex Split complex verbs, i.e., a combination of particle $+$ verb
    construction, see (\ref{ex:27.2}b,c)\footnotemark
    \begin{xlist}
    \exi{i.} \emph{utál} ‘hate’, \emph{szégyell} ‘be ashamed to’, \dots{}
    \exi{ii.} \emph{akar} ‘want’, \emph{próbál} ‘try’, \emph{tud} ‘know, can’,
        \dots{}
    \exi{iii.} \emph{fog} ‘will’, \emph{szokott} ‘usually does’, \emph{kell}
        ‘must’, \emph{szabad} ‘may, is allowed to’, \emph{talál} ‘happen to’,
        passive \emph{van} $+$ V-\emph{va/ve}
    \exi{a.}[*]{%
        \gll    be   utál-sz     jön-ni \\
                in    hate-\Ssg{}  come-\Inf{}\\}
    \exi{a$'$.}[]{%
        \gll    utál-sz    be jön-ni \\
                hate-\Ssg{}  in   come-\Inf{}\\
        \glt    ‘you hate to come in’}
    \exi{b.}[]{%
        \gll    be   akar-sz     jön-ni\\
                in    want-\Ssg{}  come-\Inf{}\\
        \glt    ‘you want to come in’}
    \exi{c.}[]{%
        \gll    be   fog-sz    jön-ni\\
                in    want-\Ssg{}  come-\Inf{}\\
        \glt    ‘you will come in’}
    \end{xlist}
\end{exe}\footnotetext{As is illustrated in (\ref{ex:27.2}i) and
    (\ref{ex:27.2}a,a$'$), not all verbs can split the complex verbs in
    their complement infinitivals. Those that do are listed in
    (\ref{ex:27.2}ii--iii) and illustrated in (\ref{ex:27.2}b,c), where
    (\ref{ex:27.2}ii) are examples of main verbs and (\ref{ex:27.2}iii)
    those of auxiliaries, as seen in~\Cref{tab:27.3}. The phenomenon was
    first described by \textcite{Proszekyetal1984} and in more detail by
    \textcite{KalmanCetal1989}, though their conditions are not followed
here, cf.\ also \textcite{Kenesei2000}.}

\ea\label{ex:27.3}
    Has tense/modal meaning\\
    \emph{tud} ‘be.able’, \emph{bír} ‘can’, \emph{fog} ‘will’, \emph{kell}
    ‘must’, \emph{szabad} ‘may, is allowed to’, \emph{lehet} ‘may, is possible’
\ex\label{ex:27.4}
    Has no present or past tense forms\\
    \emph{szokott} ‘usually does’; \emph{fog} ‘will’, \emph{szabad} ‘may, is
    allowed to’
\ex\label{ex:27.5}
    Has no person or number agreement\\
    \emph{kell} ‘must’, \emph{szabad} ‘may, is allowed to’, \emph{lehet} ‘may,
    is possible’
\ex\label{ex:27.6}
    Has no infinitival form\\
    \emph{fog} ‘will’, \emph{szokott} ‘usually does’, \emph{szabad} ‘is allowed
    to’, \emph{lehet} ‘is possible’
\ex\label{ex:27.7}
    Has no thematic subject (external argument)\\
    \emph{fog} ‘will’, \emph{szokott} ‘usually does’, \emph{talál} ‘happen to’;
    passive \emph{van} $+$ V\emph{-va/ve}, \emph{kell} ‘must’,
    \emph{szabad} ‘is allowed to’, \emph{lehet} ‘is possible’
\ex\label{ex:27.8}
    Has no potential inflection (i.e., missing \emph{-hat/het} ‘may’ affixation)\\
    \emph{fog} ‘will’, \emph{szokott} ‘usually does’, \emph{talál} ‘happen to’;
    \emph{szabad} ‘is allowed to’
\ex\label{ex:27.9}
    Has no conditional inflection (i.e., missing \emph{-na/ne} ‘would’
    affixation)\\
    \emph{fog} ‘will’, \emph{szokott} ‘usually does’
\ex\label{ex:27.10}
    Has no imperative/subjunctive forms\\
    \emph{fog} ‘will’, \emph{szokott} ‘usually does’, \emph{talál} ‘happen to’
\ex\label{ex:27.11}
    Has person-marked infinitival complements, see (\ref{ex:27.11}a)\\
    \emph{kell} ‘must’, \emph{szabad} ‘is allowed to’, \emph{lehet} ‘is
    possible’
    \ea \ili{Hungarian}\\
        \gll jön-ni-ük kell\\
            come-\Inf{}-\Tpl{} must\\
        \glt ‘they must come’
    \z
\z

Moreover, the above list is augmented by restrictions on syntactic positions,
i.e., what complement VPs each verb in the list can take, cf.\ \REF{ex:27.12}.

\ea\label{ex:27.12} \ili{Hungarian}
    \ea[]{%
    \gll    be  fog   kell-eni     tud-ni       jön-ni\\
            in  will  must-\Inf{}  be.able-\Inf{}  come-\Inf{}\\
    \glt    ‘it will be necessary to be able to come in’}
    \ex[*]{%
    \gll    be  fog   tud-ni       kell-eni       jön-ni\\
    in  will  be.able-\Inf{}  must-\Inf{}  come-\Inf{}\\
    \glt    intended: ‘(someone) will be able to have to come in’}
    \ex[*]{%
    \gll    be fog/szokott    talál-ni      jön-ni\\
            in  will/usually    happen-\Inf{}  come-\Inf{}\\
    \glt    intended: ‘(s/he) will/usually happen/s to come in’}
    \z
\z

These properties set apart main verbs (in bold type, with each exemplifying a
large array) and the single items of auxiliaries (in normal type). And, what is
more important, there are no two auxiliaries that are characterized by the same
set of features, as shown in \tabref{tab:27.3}, in which the lack of a
property is marked by a minus sign.\footnote{The star in the last cell
    indicates the irrelevance of the property. The $\pm$ sign in column~2 shows that
    some verbs in this group have modal meanings, and in column~3 that speakers
    vary as to the acceptability of the past tense form of \emph{szabad}.}

\begin{table}
\begin{tabular}{l *{10}{c}}
\lsptoprule
\rotatebox{0}{Verb} & \rotatebox{90}{Splitting} & \rotatebox{90}{Modal meaning} & \rotatebox{90}{Past tense} & \rotatebox{90}{Person/number marking} & \rotatebox{90}{Infinitive}
& \rotatebox{90}{Thematic subject} & \rotatebox{90}{Potential} & \rotatebox{90}{Conditional} & \rotatebox{90}{Imperative} & \rotatebox{90}{Person-marked infinitive}\\\midrule
\textbf{\emph{utál}} ‘hate’ & $-$ & $-$ & $+$ & $+$ & $+$ & $+$ & $+$ & $+$ & $+$ & $-$\\
\textbf{\emph{akar}} ‘want’ & $+$ & $\pm$ & $+$ & $+$ & $+$ & $+$ & $+$ & $+$ & $+$ & $-$\\
\textbf{\emph{bír}} ‘can’ & $+$ & $+$ & $+$ & $+$ & $+$ & $+$ & $+$ & $+$ & $+$ & $-$\\
\emph{fog} ‘will’ & $+$ & $+$ & $-$ & $+$ & $-$ & $-$ & $-$ & $-$ & $-$ & $-$\\
\emph{szokott} ‘usually’ & $+$ & $+$ & $+$ & $+$ & $-$ & $-$ & $-$ & $-$ & $-$ & $-$\\
\emph{talál} ‘happen’ & $+$ & $-$ & $+$ & $+$ & $+$ & $-$ & $-$ & $+$ & $-$ & $-$ \\
\emph{kell} ‘must’ & $+$ & $+$ & $+$ & $-$ & $+$ & $-$ & $-$ & $+$ & $+$ & $+$\\
\emph{szabad} ‘may’ & $+$ & $+$ & $\pm$ & $-$ & $-$ & $-$ & $-$ & $+$ & $+$ & $+$ \\
\emph{lehet} ‘may’ & $+$ & $+$ & $+$ & $-$ & $-$ & $-$ & $-$ & $+$ & $+$ & $+$\\
\textsc{passive} \emph{van} ‘be’ & $+$ & $-$ & $+$ & $+$ & $+$ & $-$ & $+$ & $+$ & $+$ & *\\
\lspbottomrule
\end{tabular}
% % % {\smaller
% % % \begin{tabularx}{\textwidth}{lXXXXX}
% % % \lsptoprule
% % % \textbf{Verb}                    & \textbf{\emph{utál}}\newline ‘hate’ & \textbf{\emph{akar}} ‘want’ & \textbf{\emph{bír}}\newline ‘can’ & \emph{fog}\newline ‘will’ & \emph{szokott}\newline ‘usually’ \\
% % % \textbf{Property}                &   &   &   &   & \\\midrule
% % % Splitting                        & $-$ & $+$ & $+$ & $+$ & $+$ \\
% % % Modal meaning                    & $-$ & $\pm$ & $+$ & $+$ & $+$ \\
% % % Past Tense                       & $+$ & $+$ & $+$ & $-$ & $+$ \\
% % % Person/number marking            & $+$ & $+$ & $+$ & $+$ & $+$ \\
% % % Infinitive                       & $+$ & $+$ & $+$ & $-$ & $-$ \\
% % % Thematic subject                 & $+$ & $+$ & $+$ & $-$ & $-$ \\
% % % Potential                        & $+$ & $+$ & $+$ & $-$ & $-$ \\
% % % Conditional                      & $+$ & $+$ & $+$ & $-$ & $-$ \\
% % % Imperative                       & $+$ & $+$ & $+$ & $-$ & $-$ \\
% % % Person-marked infinitive         & $-$ & $-$ & $-$ & $-$ & $-$ \\\midrule
% % % \textbf{Verb}                    & \emph{talál}\newline ‘happen’ & \emph{kell}\newline ‘must’ & \emph{szabad}\newline ‘may’ & \emph{lehet}\newline ‘may’ & \textsc{passive}\newline \emph{van} ‘be’   \\
% % % \textbf{Property}                &   &   &   &   &   \\\midrule
% % % Splitting                        & $+$ & $+$ & $+$ & $+$ & $+$ \\
% % % Modal meaning                    & $-$ & $+$ & $+$ & $+$ & $-$ \\
% % % Past Tense                       & $+$ & $+$ & $\pm$ & $+$ & $+$ \\
% % % Person/number marking            & $+$ & $-$ & $-$ & $-$ & $+$ \\
% % % Infinitive                       & $+$ & $+$ & $-$ & $-$ & $+$ \\
% % % Thematic subject                 & $-$ & $-$ & $-$ & $-$ & $-$ \\
% % % Potential                        & $-$ & $-$ & $-$ & $-$ & $+$ \\
% % % Conditional                      & $+$ & $+$ & $+$ & $+$ & $+$ \\
% % % Imperative                       & $-$ & $+$ & $+$ & $+$ & $+$ \\
% % % Person-marked infinitive         & $-$ & $+$ & $+$ & $+$ & * \\
% % % \lspbottomrule
% % % \end{tabularx}
% % % }
\caption{Feature matrix for \ili{Hungarian} verbs and auxiliaries}\label{tab:27.3}
\end{table}

Starting with the fourth column there are only \enquote{classes} containing single
items, and it is precisely these words that qualify as auxiliaries, which
points at property \REF{ex:27.7} as the one distinguishing them from main
verbs, or more precisely, main verbs that take infinitival clauses as
complements.\footnote{See \citet{Kenesei2006} for a full set of arguments.}
Note, however, that the lack of a thematic subject/external argument is a
property found also in unaccusative verbs, but they, in turn, do not take
infinitival complements, and \tabref{tab:27.3} was set up to include verbs
with infinitival complements only. Again, it is another instance of
cross-classification, as is generally the case with the open class of (main)
verbs, but the ultimate lesson is that the word class of auxiliaries does not
seem to emerge, because the rest of the features are not shared by any two of
the items listed in \tabref{tab:27.3}.

\section{Conclusion: Life without word classes}\largerpage[2]

We could go on to demonstrate similar one-member classes in case of articles,
conjunctions, and other functional categories, but, as was seen above,
categories\is{lexical categories} in open classes are also prone to a limitless
multiplication of classes. The way out of this impasse is at hand: word class
is an epiphenomenon, it is not a basic concept but a derivative notion in
linguistics. There are no word classes; what we have to do with is properties
and their combinations, clusters, or matrices. The morphological and syntactic
environment, including the complements of individual functional or notional
items, can be determined also by various combinations of properties, spelling
them out as the characterizations of individual items as we have seen in the
case of the auxiliaries.

Morphological or syntactic processes rely and work on properties rather than
(classes of) words or morphemes, which renders the discussion on whether word
classes are universal or language-specific irrelevant
(\citealt{Hengeveld1992,Croft2005,Haspelmath2012} etc.). What can be universal
is not some word class but a set of distinctive properties, some of which were
illustrated above. Since there are probably no languages without subjects,
\citegen{Crystal1967} feature of \enquote{May act as subject} is probably
universal.\footnote{One anonymous reviewer contests my reliance on this
    property, cf.: \enquote{The author says ‘there are probably no languages
        without subjects’ but that is a statement which has frequently been
    contested by those who work on so-called ‘topic prominent’ languages}. My
    studies of \isi{topic}-prominent languages, which include \ili{Hungarian}, among
    others, do not, however, confirm this statement, but cf.\ also e.g.,
    \citet{EKiss2002} for a more complete overview. This reviewer also
    maintains that \enquote{various theories do without a core concept of
        \enquote{subject} (including most if not all versions of generative
        grammar), while others such as \gls{LFG}\is{Lexical Functional Grammar}
        and \isi{Relational Grammar} make it a
theoretical primitive.} While this is indeed the case, the fact that `subject’
is a derived notion, rather than a core concept, in generative grammars does
not preclude reference to it by the properties invoked here.} It is likely that
all languages have a property of \enquote{May have a complement}, and if there
are cases in a language, then it makes sense to posit the feature
\enquote{Assigns (structural) case}. But just as the consonantal phonological
feature for clicks may be relevant only in \ili{Bantu} languages, it is possible that
the syntactic feature of \isi{incorporation}, which is significant in
\ili{Chamorro}, is
missing in a large number of languages. And with reference to the languages
with \enquote{flexible word classes}, as well as to the decomposition of
categories\is{lexical categories} in \isi{Distributed Morphology}, it may very
well be the case that the syntactic categorizing heads, i.e., the
\enquote{categorizers} that merge with categorially unspecified lexical items,
are themselves bundles of properties along the lines discussed here.

There is hardly anything surprising in this development, especially if we take
into account the fact that it is no longer the phoneme that is the basic unit
in phonology but distinctive features and the term phoneme is but shorthand for
sets of distinctive features, as seen in the following passage:\largerpage

\begin{quote}

In recent years it has become widely accepted that the basic units of
phonological representation are not segments but features, the members of a
small set of elementary categories\is{lexical categories} which combine in
various ways to form the speech sounds of human languages.
\parencite[245]{ClementsHume1995}\footnote{See also \citet{Siptár2006}.}

\end{quote}

And finally, just as phonologists have not got rid of the term
\enquote{phoneme}, so syntacticians or morphologists need not throw out the
notion of \enquote{word class} -- if they are aware that it is a convenient
abbreviation without any consequence or theoretical relevance.

\printchapterglossary{}

\section*{Acknowledgements}

This article has grown out of a number of presentations to various audiences,
e.g., at the 14th \emph{morphology meeting}, and the \emph{beyond dichotomies} conference,
both in Budapest, 2010, the Research Institute for Linguistics, and the
Linguistics and Literature Section of the Hungarian Academy of Sciences. I am
grateful to the audiences there, and in particular to László Kálmán and Péter
Siptár. My special thanks go to the two anonymous reviewers of the current
version.  Research reported here was supported by Grant NKFIH K120073
\enquote{Open access book series on the syntax of Hungarian}.

{\sloppy
\printbibliography[heading=subbibliography,notkeyword=this]
}

\end{document}
