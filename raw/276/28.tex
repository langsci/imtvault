\documentclass[output=paper]{langsci/langscibook}
\ChapterDOI{10.5281/zenodo.4280635}

\author{Andrea Moro\affiliation{University School for Advanced Studies IUSS, Pavia}}
\title{The matrix: Merge and the typology of syntactic categories}

% \chapterDOI{} %will be filled in at production

\abstract{In recent works
    \parencite{Moro2000,Moro2009,Chomsky2013,Chomsky2017,ChoGalOtt2019,Rizzi2015,Rizzi2016}
    a new type of phrasal structure has been assumed resulting from Merging two
    XPs where neither XP projects: the unlabelled [XP YP]. This structure
    stands out as an exception with respect to the typical
    X\textsuperscript{0}s and XPs. I will show that by considering some basic
    properties of \isi{Merge} in an abstract combinatorial framework the
    stipulative character of this category is absorbed along with some
potential redundancies of \glsunset{UG}\gls{UG}.}

\maketitle

\begin{document}\glsresetall

\section{The X\textsuperscript{0} vs.\ XP distinction and the lexicon}

A basic opposition is manifested in syntax between X\textsuperscript{0}s and
XPs. A traditional way of distinguishing between these two categories is to
refer to the lexicon: an X\textsuperscript{0} directly comes from the lexicon,
whereas an XP does not. In fact, this opposition can also be captured by
referring to \isi{Merge} by reasoning as follows.\largerpage[-1]

\section{The matrix or beyond the X\textsuperscript{0}--XP taxonomy}

An X\textsuperscript{0} cannot be targeted by \gls{IM} whereas an XP can; call
this property “atomicity”. Interestingly, this not the only way to cast
X\textsuperscript{0} and XPs into two disjoint classes by referring to
\isi{Merge}.  An X\textsuperscript{0} cannot appear as a specifier whereas an
XP can.  Since a specifier is an XP which is Merged to another XP without
projecting, one can say that an XP is an optional projector whereas an
X\textsuperscript{0} is not; call this property “incapsulation”.\footnote{This
    operation can in principle be reiterated generating “multiple specifiers”
    or one specifier and multiple adjuncts; I will maintain \citegen{Kayne1994}
    \glsunset{LCA}\gls{LCA}-based principle according to which there can be
    only one element merged with a phrase to preserve the possibility of
    linearization. This is only partially true since there could be multiple
    subjects provided that only one is spelled-out at \gls{PF}. The existence
    of these configurations is provided by inverse copular\is{copulas} sentences
    in \ili{Italian}. In this case, the preverbal phonologically overt DP is mutually
    c-commanding \emph{pro} without violating the \gls{LCA}\is{linear correspondence axiom} since \emph{pro} is
    not visible to linearization. Clear support for this analysis comes from
    cases where the preverbal subject is singular and the postverbal one
    plural: in this case, the copula anomalously agrees with the postverbal DP
    showing that there must be a \emph{pro} (in fact a “null predicate”)
    mediating the \isi{agreement} relation as in \emph{la causa sono Pietro e
    Giovanni} (the cause-sing.fem.  are Peter and John). The intervening
    subject is \emph{pro} as proposed in \citet{Moro1997} as in \emph{la causa
    pro sono io} (the cause pro am I; `the cause is me') or just \emph{sono io}
    (am I; `it's me'). Indeed, if more than one adjunct\slash subject is generated:
all but one must move, as a consequence of the principle of dynamic
antisymmetry.}

\ea A syntactic entity S is:
\label{ex:key:28.1}
	\ea atomic ([$+$a]) iff no parts of it can be targeted by \gls{IM}.
	\ex incapsulable ([$+$i]) iff it can be merged to an XP without projecting.
	\z
\z

Let us now construe a combinatorial square matrix based on these two
independent properties displaying both positive and negative polarities and
start by representing the two opposite and already recognized entities, namely
an X\textsuperscript{0} as [$+$a, $-$i] and  an XP as [$-$a, $+$i]:\footnote{Matrices
    are typical structuralist tools that have their origin in phonological
    models.  In syntax, they have been used less massively; two major examples
    are \citegen{Chomsky1970} and \citet{Jackendoff1977} -- both incorrectly
    assuming that noun phrases cannot be predicates -- and
    \citealt{MuyskenvanRiemsdijk1986} relying on features pertaining to X-bar
    levels. In fact, perhaps the first use of derivative categories in
    linguistics can be traced to at least the Hellenistic models of grammar,
witness the term “participium” (lit: that takes part) related to a verbal form
which displays adjectival morphology.}

\ea\label{ex:key:28.2}\leavevmode\\[-1\baselineskip]
    \begin{tabular}{ccc}
    \lsptoprule
    	        & $+$i & $-$i\\
    \midrule
    $+$a &             & X\textsuperscript{0}\\
    $-$a & XP          & \\
    \lspbottomrule
    \end{tabular}
\z

This matrix raises a new question, namely whether there exist any [$+$a, $+$i] and
[$-$a, $-$i] syntactic entities, i.e. homopolar syntactic entities, or whether
there exist only the heteropolar ones. I will show that the answer is
affirmative and this matrix solves the problem raised by unlabeled [XP YP]
structures.  Let us first consider the case of a syntactic entity with all
negative polarity features.

\subsection{Bare small clauses}

A natural candidate to occupy the [$-$a, $-$i] slot is the so-called “bare small
clause” (\glsunset{BSC}\gls{BSC}), prototypically represented by the complement
of the copula. Two separate issues must be addressed here: a preliminary one is
whether there is any empirical reason to assume that such non-atomic
constituents exist; the other is whether there is any empirical reason to
exclude them from the specifier position. In fact, they have both already be
answered positively.  I will just sketchily remind here the data upon which the
answer is built.\largerpage

Originally, the complement of the copula\is{copulas} was considered to be the same as the
complement of \emph{believe}-type verbs and labelled “small clause”
(\glsunset{SC3}\gls{SC3}): namely, a non-inflected predicative\is{predication} structure (see
\citealt{Williams1975} and \citealt{Stowell1978} for the first proposals and
\citealt{Graffi2001} for a critical survey). It has been later proposed that
these two types of complements have two distinct structures (see
\citealt{Moro1997} for the original proposal; and \citealt{Moro2017a,Moro2017b}
for a synthetic update): the complement of \emph{believe}-type verbs is a
phrase headed by a predicational\is{predication} head – whose precise categorical nature is
still under discussion – whereas the complement of the copula\is{copulas} is an unlabeled
phrase resulting from the direct merge of two phrases. The minimality of the
latter structure is what justifies the term “bare”; accordingly, these phrases
are represented as [XP YP] merged without any intervening head.\footnote{This
    analysis revives \citegen{Williams1980} original proposal for the analysis
of \glspl{SC3}\is{small clauses} which was abandoned partially because of the influential proposal by
\citegen{Chomsky1986} to uniform clause structures to the XP format,
normalizing all phrases to endocentric structures.}  The specificity of this
construction is not the merging of two phrases but rather the fact that
\emph{neither} phrase project, unlike the case of specifiers that yield
[\textsubscript{α} XP YP] where the label α coincides with
either phrase and the specifier is the phrase which does not
project.\footnote{Notice that in this analysis of predicative\is{predication} structures both
    the subject and the predicated\is{predication} are incapsulated; this independent fact
    shows that incapsulation is more general than “specifierhood” which is
inherently asymmetrical.}

The empirical reasons supporting the distinction between \gls{SC3}\is{small clauses} and
\gls{BSC} are based on several distinct domains. For the sake of simplicity,
three distinct types of domains can be reminded here and exemplified in
\eqref{ex:key:28.3}: the distribution of predicative\is{predication} markers
(\ref{ex:key:28.3}a,b);\footnote{The presence of a predicative\is{predication}
    marker in the complement of \emph{believe}-type verbs was taken by
    \citet{Moro1988} as the spell-out of an abstract
    predicative\is{predication} head (Pred\textsuperscript{0}); its absence in
    copular\is{copulas} constructions, instead, led to hypothesis that the
    clausal constituent was better analyzed as an AgrP and -- correspondingly --
    the copula\is{copulas} as the expression of tense (and aspect) features
    (T\textsuperscript{0}) yielding a first version of the so-called
    “Split-Infl” hypothesis. This analysis preceded and was empirically
    distinct from the influential version proposed by \citet{Pollock1989} and
was later partially abandoned in favor of the unheaded \gls{BSC} hypothesis,
while maintaining the idea that IPs were in fact to be analyzed as TPs.}
intervening effects on cliticization, more specifically violations of
\citegen{Rizzi1990} relativized minimality (\ref{ex:key:28.3}c,d);\footnote{I
    have simplified the representation in (\ref{ex:key:28.3}d): for locality
    reasons, a \gls{BSC} can never be completely evacuated (see
    \citet{Moro1993} elaborating on \citegen{Rizzi1990} notion of
    head-government. The clitic\is{clitics} is rather sub-extracted from a DP as an
    N\textsuperscript{0}. The same D\textsuperscript{0}/N\textsuperscript{0}
    distinction holds for wh-elements where \emph{which} corresponds to
    D\textsuperscript{0} while \emph{what} to N\textsuperscript{0}, witness
    cases like \emph{what a party}! where the wh-element co-occurs with an
    overt D\textsuperscript{0}; this also explains the possibility to extract
    \emph{what} but not \emph{which} in existential sentences (see
    \citealt{Moro1997} revising \citegen{Heim1987} semantic account of this contrast
    and the locality conditions on extraction; see also \citealt{Moro1993} for
locality issues within a Minimalist framework).} instability, i.e. the
necessity of \isi{movement} out of the embedded clausal structure both in \ili{English}
(\ref{ex:key:28.3}e--g) and in \emph{pro}-drop languages
(\ref{ex:key:28.3}g):\footnote{Notice that the \emph{pro}-drop parameter is
    totally irrelevant here: \isi{movement} in required in \ili{Italian} on a par with in
    \ili{English}. No “expletive” can rescue the structure where neither phrase
    moves, not even \emph{ci} (there), reinforcing the hypothesis that \isi{movement}
    is required to solve the instability of the lower \gls{BSC} rather than
    satisfy some specific condition of the subject position; for the impact of
    this phenomenon on discharging the \isi{extended projection principle} see
    \citet{Moro1997,Moro2000} and, in particular, \citet{Moro2009} for a
    detailed discussion involving the role of Focus\textsuperscript{0} in
post-verbal positions.}

\ea\label{ex:key:28.3}
    \ea {Mary considers [John (as) the culprit~] (cf.\ also \emph{John is considered \emph{t} (as) the culprit})}
    \ex {John is [ \emph{t} (*as) the culprit~]}
    \ex {\ili{Italian}}\\
        \gll \llap{*}lo ritengo [ Maria H\textsuperscript{0} \emph{t} ]\\
            so-\Cl{} believe {} Maria {} {}\\
    \ex \ili{Italian}\\
        Maria lo è [\emph{t} \emph{t}]
    \ex Mary considers [ John stupid ]
    \ex \makebox[0pt][r]{*}is [ John stupid ]
    \ex Italian\\
        \gll \llap{*}è [ Gianni stupido ]\\
                is {} Gianni stupid\\
    \z
\z\largerpage[-2]

All these facts converge toward the analysis according to which the complement
of the copula\is{copulas} consist of merging two phrases without the
intervention of a head. This analysis has proved to be consistent across
languages; a strong support to the existence of \glspl{BSC}\is{small clauses}
along with \glspl{SC3}\is{small clauses} comes from Pereltsvaig's analysis of \ili{Russian}
\citep{Pereltsvaig2007}. Moreover, it has also been proposed that
\glspl{BSC}\is{small clauses} also occur in nominal domains, as complements of
P\textsuperscript{0} heads playing the same role as the copula\is{copulas} in
that they provide a landing site for either the subject or the
predicative\is{predication} phrase (\citealt{Moro2000}; see also
\citealt{Kayne1994,denDikken1997,Zamparelli2000}). Simple
examples are pairs like \emph{these types of books} vs.\ \emph{books of this
type} which are generated by the same underlying structure containing a
\gls{BSC}, namely [ of [\textsubscript{BSC} [books] [this type]]], by raising
either the subject [books] or the predicative\is{predication} nominal [this
type] to the specifier of P\textsuperscript{0} (cf. \emph{books are of these
types}). We can now turn to the second issue, namely as to why
\glspl{BSC}\is{small clauses} cannot be specifiers.

One of the special properties of \glspl{BSC}\is{small clauses} – witness
examples like (\ref{ex:key:28.3}f,g) -- is that they force \isi{movement} of either XP: if the two XPs constituting the \gls{BSC}
are both noun phrases then either \isi{movement} is possible, yielding a canonical
vs.\ inverse copular\is{copulas} sentence depending on whether the subject or the predicate
raises (and similarly, mutatis mutandis, in nominal constructions); if
the predicate of the copular\is{copulas} sentence is not a noun phrase -- say an adjectival
phrase -- then the only viable rescue strategy is for the subject to raise,
because of the morphological restrictions imposed on the landing site (arguably
related to Case assignment). The reason of the instability of this structure is
inherently related to the symmetrical nature of this configuration; there are
two alternative explanations, one based on the \gls{LCA}\is{linear correspondence axiom} \citep{Moro2000}~– \isi{movement}
is necessary to allow linearization of two mutually c-commanding phrases -- the
other on labeling\is{labelling} algorithm \citep{Moro2009}~– \isi{movement} is necessary to provide
a label to the \gls{BSC} (see also
\citealt{Moro2000,Moro2009,Chomsky2013,Chomsky2017,ChoGalOtt2019,Rizzi2015,Rizzi2016}
for further support to this explanation and in general for the principle of
dynamic anti-symmetry). It could well be that both explanations are valid and
that this phenomenon reveals a twofold nature of instability depending on the
test adopted. Duality is not to be avoided per se in empirical science
if it is grounded and impinges on separate empirical reasons.\largerpage[-1]

However, for what matters here, even if only one explanation will turn out to
be true, still the instability – hence, the necessity of \isi{movement} out of a \gls{BSC}
– remains as an undisputed fact. And it is this very fact that offers a
straightforward explanation for the second issue addressed in this section,
namely as to why \glspl{BSC}\is{small clauses} cannot be specifiers. An obvious case study is the
impossibility for \gls{BSC} to be clausal subjects, i.e. specifiers of TP.  The
crucial fact is that \isi{movement} is banned from within this position unless some
specific conditions are realized which do not apply here (for the locality
conditions on the subject position see in particular the discussion in
\citealt{Rizzi2015}, \citealt{Stepanov2007} and references cited there). All
in all, the impossibility for a \gls{BSC} to occur as a subject follows for principle
reasons without ad hoc stipulations: on the one hand its instability requires
movement; on the other, \isi{movement} is impossible for locality
conditions.\footnote{Interestingly notice the following contrast:

\begin{exe}
    \exi{(i)}
    \begin{xlist}
        \ex[*]{[John the culprit] is strange}
        \ex[]{[for John to be the culprit] is strange}
    \end{xlist}
\end{exe}

This shows that what prohibits for a clausal structure to be clausal subject is
not related to the finiteness of tense and aspects features. As for the
possibility of a local \isi{movement} to a focal position to solve instability (see
\citealt{Moro2009}). Notice also that being \gls{BSC} [$-$i] it must project when merged
with an XP\@: this is consistent and in fact it derives the solution to the
instability of these constituents as predicted by the principle of dynamic
anti-symmetry (see
\citealt{Moro2000,Moro2009,Chomsky2013,Chomsky2017,ChoGalOtt2019,Rizzi2015,Rizzi2016}).}

Eventually, the homopolar negative slot [$-$a, $-$i] generated by the matrix in
\eqref{ex:key:28.2} can then be filled in by \glspl{BSC}\is{small clauses}:

\ea\label{ex:key:28.4}\leavevmode\\[-1\baselineskip]
    \begin{tabular}{ccc}
    \lsptoprule
                & $+$i & $-$i\\
    \midrule
    $+$a &             & X\textsuperscript{0}\\
    $-$a & XP          & BSC\\
    \lspbottomrule
    \end{tabular}
\z

The matrix, in fact, completely eliminates the stipulative character of \glspl{BSC}\is{small clauses}:
these acentric phrases are not exceptions as they are now framed in the same
two property based grid generating the other two categories, namely words and
endocentric phrases. The exception would now rather be if they did \emph{not}
exist.

\subsection{Expletives}

There is a residual empty slot in the matrix in \eqref{ex:key:28.4}, namely the
homopolar positive syntactic entity: [$+$a, $+$i]. Is there a reason for assuming
that there exist atomic entities that can occur as the specifiers of a phrase,
that is that can be incapsulated? I would like to suggest that this category
exists and coincides with \isi{expletives}.\footnote{I refer to “expletives” in
    general but a more fine-grained terminology would distinguish between
    subject-\isi{expletives} as in \emph{it was clear that John left} and
    predicative-\isi{expletives} as in \emph{it's that John left}, just to remain to
    pro-CPs, along the lines of \citet{Moro1997}.} In a sense, this assumption
    is trivially proved. Elements like \emph{there} in \ili{English} existential
    sentences, for example, are clearly atomic but they cannot further project
    when merged with a phrase -- in fact, they  prototypically end up occupying
    the position canonically reserved to clausal subjects -- hence [$+$i].
    Nevertheless, they do qualify as exceptions since atomic entities, i.e.
    X\textsuperscript{0}s, do project and they cannot occupy the subject
    position: expletive appear like “inert heads”. One possibility would of
    course be to assume that \isi{expletives} are not real heads but rather
    “monolithic” phrases which exceptionally contain no parts visible to
    Internal \isi{Merge} but this would of course be a way just to rephrase the
    situation.  On the other hand, however, the capacity of \isi{expletives} to share
    \emph{some} properties with heads can indeed be independently supported, by
    considering more fine-grained and hidden empirical data, such as those
    manifested in copular\is{copulas} constructions.  Consider the following contrast taken
    from \citeauthor{Moro1997} (\citeyear{Moro1997}; see also \citealt{Stepanov2007} for an analysis of
    the same data in (\ref{ex:key:28.5}a):\footnote{This contrast was also
        discovered with respect to quantifier raising:

\begin{exe}
    \exi{(i)}
    \begin{xlist}
    \ex there weren't pictures of many girls
	\ex the cause of every riot wasn't pictures of many girls
    \end{xlist}
\end{exe}

The embedded quantifier \emph{many} can have scope over negation, hence be
extracted from the subject DP at \gls{LF}, only in a \emph{there}-sentence (ia).
Notice that the example in (ia) falsifies \citegen{Williams1984} analysis of
\emph{there} as a scope marker: for a full discussion, see \citet[Ch.\
2]{Moro1997}.}

\ea%5
    \label{ex:key:28.5}
    \ea[]{which wall do you think there was [a picture of \emph{t}]}
	\ex[*]{which wall do you think the cause of the riot was [a picture of \emph{t}]}
	\z
\z

Following \textcite{Moro1988,Moro1997}, I will assume that \emph{there} is a
not a subject expletive which is inserted late in the derivation; this element
is rather a pro-predicate expletive raised from a lower position or,
equivalently, that existential sentences like (\ref{ex:key:28.5}b) belong to the
more general class of inverse copular\is{copulas} sentences: cf. [there was [ [a picture of
the wall] \emph{t} ]]. In (\ref{ex:key:28.5}b), instead, the phrasal predicate
\emph{the cause of the riot} is raised to the pre-verbal position. The major
difference between the two sentences, then, is that the head of the predicate
is embedded in (\ref{ex:key:28.5}b) (namely, \emph{cause}) whereas it edges the TP
phrase in (\ref{ex:key:28.5}a) (namely, \emph{there}).

This distinction allows to explain this contrast by appealing to the notion of
L-marking. More specifically, \citet{Moro1997} adopted the version of L-marking
as formulated in \citet{Cinque1990} which differed from \citegen{Chomsky1986}
original proposal: Cinque's version is based on the selectional capacities of a
head rather than its theta-marking ones. Synthetically, a phrase is an
island\is{islands}
(or a barrier to \isi{movement}) unless it enters into a local relationship with a
head selecting it, where by “local relationship” a minimal dominance relation
is intended canonically expressed in terms of c-command. An interesting remark
on L-marking highlights its persistence in Minimalist frameworks: “Though
varieties of government would be `imperfections', to be avoided if possible,
the closer-to-primitive notion of L-marking should pass muster, hence also
notions of barrier that are based on nothing more than L-marking”
(\citealt{Chomsky2000}, 117; for a critical review of the notion of L-marking
and the empirical and historical reasons behind it see \citealt{Roberts1988}).\largerpage[2]

All in all, the impossibility to extract from within the post-verbal subject in
(\ref{ex:key:28.5}b) is immediately explained by the fact that it is not
L-marked: the element selecting it is the predicative\is{predication} head \emph{cause} and it
fails to c-command it; the only other head c-commanding the subject is the
copula: although it qualifies in terms of local configuration, it does not
select the subject: thus the subject is not L-marked and  extraction from it
yields an ungrammatical sentence. This parallels the case of a preverbal
subject of an embedded sentence: it is in a proper local configuration with a
complementizer c-commanding it but it is not selected by it (see
\citealt{Rizzi1990,Rizzi2015}; see also again \citealt{Stepanov2007} for
critical considerations on extractions from the subject position). In
(\ref{ex:key:28.5}a), instead, the head \emph{there} (locally) c-commands the
lower subject and it selects it in its capacity as a pro-predicate: thus, the
subject is L-marked and extraction is viable. The special head-like relation
between the expletive \emph{there} in subject position and the
copula\is{copulas} is also
manifested in the fact that the copula\is{copulas} anomalously shows rightward \isi{agreement},
reasonably a sign that the number features of the subject have been transmitted
by the pro-predicative\is{predication} element selecting it:\footnote{That there are cases
    where the nominal head of a predicate \emph{must} agree with its subject is
    independently attested in cases like:

\begin{exe}
    \exi{(i)} I consider John and Peter my best friend*(s)
\end{exe}

However, \isi{agreement} is by no means obligatory in all cases. In fact, there can
be a complete mismatch in gender and number as in:

\begin{exe}
    \exi{(ii)}
        \gll considero i libri la mia passione\\
        	consider-\Fsg{} the-\M.\Pl{} books-\M.\Pl{} the-\glossF.\Sg{} my-\glossF.\Sg{} passion-\glossF.\Sg{}\\
        \glt
\z

See \citet{Moro1988,Moro1997,Moro2017a} for further considerations.}

\ea\label{ex:key:28.6}
	\ea there were many pictures of the wall
	\ex the cause of the riot was/*were many pictures of the wall
	\z
\z

Similar considerations concerning \emph{there} would hold for pre-verbal
\emph{it} in quasi-copular\is{copulas} sentences such as \emph{it seems that Mary left} as
well as in inverse copular\is{copulas} sentences with clausal subjects like \emph{it's that
Mary left}, whose common structure is: [ it V\textsuperscript{0} [ [that Mary
left] \emph{t}]. There are also other occurrences of \emph{there} with other
verbs than the copula\is{copulas} which would lead to the same conclusion, namely
unaccusative constructions but illustrating them here would take us too far
(see \citealt{Moro1997} and the crucial extensions suggested in the
comprehensive theory of argument structure proposed in
\citealt{HaleKeyser2002}).

Crucially, for what matters here, there is a further piece of evidence in favor
of the fact that \isi{expletives} have a twofold nature. In the previous examples, I
have provided evidence that they share the same selectional properties as
\emph{heads}; it can be also proved that they do behave like \emph{phrases} by
reasoning as follows. \isi{expletives} are only merged with other phrases; as [$+$i]
elements they cannot project, thus the resulting phrase can either be a full
endocentric phrase (where the other element projects) as in [\textsubscript{TP}
Expl TP ] or it can be a \gls{BSC} (where neither phrase projects) as in the
[\textsubscript{BSC} DP Expl ] generating (\ref{ex:key:28.5}a) where neither
phrase projects. In the latter case, either phrase must be further moved as
predicted by dynamic anti-symmetry\@:\footnote{For the reasons why the expletive
    raises and the impact it has on semantic structure see
    \textcites[Ch.\ 3]{Moro1997}{Moro2000}{Moro2009};
    \citet{Chomsky2013,Chomsky2017,ChoGalOtt2019,Rizzi2015,Rizzi2016}
    if the expletive did not have phrasal properties and they were just like
    heads, it would be hard to explain why the structure is unstable and it
requires \isi{movement}. All in all, \isi{expletives} appear to share some properties with
both X\textsuperscript{0} and XPs.}

The very existence of atomic and incapsulated \isi{syntactic categories} (expletives)
is ultimately well-grounded empirically and this allows us to fill in the last
available slot in the two property based grid:\footnote{Notice that the
    \gls{BSC} analysis originally proposed for existential sentences,
    quasi-copular sentences, and unaccusative constructions has been extended
    to cover previously unrelated constructions. In particular, the same
    analysis has been proposed to include wh-phrases to explain split
    interrogatives, including the classic “\emph{was-für} split phenomena” and
    its equivalent in \ili{Romance} languages (see \citealt{Moro2000} and
    \citealt{Ott2012} for a further and original extensions of this proposal).
    In \ili{Italian}, for example, we get the following case study where the
    particle \emph{di} (`of') plays the same role as a nominal copula\is{copulas}
    in \emph{questi tipi di libri} (`these types of books') forcing \isi{movement} of
    the wh-element \emph{cosa} (`what') to the specifier position of the proper
    CP-slot:

\begin{exe}
    \exi{(i)}
	\gll Cosa legge [ \emph{t} di [\textsubscript{BSC}  libri  \emph{t} ]]?\\
		what reads-\Tsg{} {} {} of {} books {} {}\\
	\glt `What books does s/he read?'
\end{exe}

For what matters here, examples like (i) show that the twofold nature of
elements like \emph{there} is not isolated to canonical expletives: it is
rather unexpectedly shared by wh-elements like \emph{cosa} (`what') which
constitute an unstable structure with another full phrase, revealing their
phrasal nature, but do not contain any part accessible to Internal \isi{Merge}, i.e.\
they behave like X\textsuperscript{0}. We should perhaps speak of “generalized
expletives” to include clausal and non-clausal ones.}

\ea\label{ex:key:28.8}\leavevmode\\[-1\baselineskip] %8
    \begin{tabular}{ccc}
    \lsptoprule
                & $+$i & $-$i\\
    \midrule
    $+$a & Expl        & X\textsuperscript{0}\\
    $-$a & XP          & BSC\\
    \lspbottomrule
    \end{tabular}
\z

\section{On evaluating the matrix: Suggestions for the future agenda}

The fourfold taxonomy generated by the matrix absorbs the exceptionality of \gls{BSC}
and \isi{expletives} framing them along X\textsuperscript{0} and XP in a natural way
within the same grid generated by two syntactic properties formulated by
referring to \isi{Merge}.

In principle, this may not be the only welcome result: the matrix could also be
exploited to capture further empirical generalizations. For example, it reveals
natural classes – i.e. \isi{agreement} is possible only with a [$+$i] category – or it
allows to identify grammatical functions in a more comprehensive way – i.e.
predicative\is{predication} structures coincide with [$-$a,$-$i] category (see
\citealt{Moro2000,Moro2004} for further discussion) or simplifications – i.e.
two homopolar entities (namely, \isi{expletives} and \glspl{BSC}\is{small clauses}) cannot be
merged. Whether or not this matrix will be theoretically useful for formulating
new questions is left for future research to answer.

\printchapterglossary{}

\section*{Acknowledgments}

My special thanks go to Robert Frank, Raffaella Zanuttini, Giorgio Graffi,
Cristiano Chesi, Andrew Nevins and Alessandra Tomaselli for their illuminating
remarks and two anonymous reviewers: the errors remain all mine. I wish I wrote
this paper in \ili{Italian} for if Ian translated it for me it would have become much
better.\largerpage

{\sloppy\printbibliography[heading=subbibliography,notkeyword=this]}

\end{document}
