\documentclass[output=paper]{langsci/langscibook}
\ChapterDOI{10.5281/zenodo.4280655}
\author{Jamie Douglas\affiliation{University of Cambridge}}
\title{Rethinking relatives}

% \chapterDOI{} %will be filled in at production

\abstract{This chapter is concerned with the syntactic size of finite and infinitival
relative clauses in \ili{English}. I claim that these fall into three (or even four)
distinct structural sizes. Assuming a cartographic descriptive framework, I
provide evidence for this claim from novel observations concerning the
(un)availability of adverbial and argument fronting in the different types of
relative clause (following \citealt{Haegeman2012}). Specifically, some relative
clauses permit both adverbial and argument fronting, some permit adverbial
fronting only, whilst others do not permit fronting at all. Additional support
for my claim comes from three instances of \isi{categorial distinctness} effect (in
the sense of \citealt{Richards2010}), which I argue instantiate a distinctness
effect between elements in SpecTopP and SpecFocP\@.}

\maketitle

\begin{document}\glsresetall

\section{Introduction}\label{sec:02.1}

\Glspl{RC} have been a subject of study within generative frameworks for
decades. It is probably fair to say that the syntactic literature has been
primarily concerned with how the \gls{RC}\is{relative clauses} head (the noun
modified by the RC) is related to the RC-internal gap, with reconstruction
effects playing a prominent role in discussions and analyses. However, rather
than focussing on the \gls{RC} head, I will consider the \gls{RC}\is{relative
clauses} itself.  More specifically, I will investigate the syntactic structure
and the structural size of \ili{English} \glspl{RC}.

The literature typically recognises two distinct structural sizes as far as RCs
are concerned: clausal RCs, as in \eqref{ex:2.1}, and reduced RCs, as in \eqref{ex:2.2}.

\ea\label{ex:2.1} Clausal RCs\\
    the man [(who(m)/that) I met yesterday]
\ex\label{ex:2.2} Reduced RCs\\
    the man [(being) arrested by police yesterday]
\z
I will not discuss reduced \glspl{RC}\is{relative clauses} here (for recent discussion, see
\citealt{Douglas2016,Harwood2017}) but will \isi{focus} exclusively on clausal RCs,
simply calling them \glspl{RC}\is{relative clauses} from now on. I argue that \glspl{RC}\is{relative clauses} are not
homogeneous in their structural size, i.e.\ they vary in terms of how much
syntactic structure they contain. The different types of \gls{RC}\is{relative clauses} that I will
investigate are exemplified below:

\ea\label{ex:2.3} Finite \emph{wh}-\glspl{RC}
    \ea    The man [who saw me] is John.
    \ex    The house [which I lived in] fell down.
    \ex    The house [in which I lived] fell down.
    \z
\ex\label{ex:2.4} (Finite) \emph{that}-\glspl{RC}
    \ea    The man [that saw me] is John.
    \ex    The man [that I saw] is John.
    \ex    The house [that I lived in] fell down.
    \z
\ex\label{ex:2.5} Finite $\varnothing$-\glspl{RC}
    \ea    The man [I saw] is John.
    \ex    The house [I lived in] fell down.
    \z
\ex\label{ex:2.6} Infinitival \emph{wh}-\glspl{RC}
    \ea    The man [to whom to speak] is John.
    \ex    The house [in which to live] is that one.
    \ex    For a beginner, the course will likely provide a good atmosphere [in
        which for you to fire your first shots].\footnote{This example is from:
        \url{http://hunting.about.com/od/hunting-for-beginners/a/Hunting-For-Beginners.htm}.
        Such examples are not acceptable to all speakers (see, e.g., the
        judgements in \citealt{ChomskyLasnik1977,Huddleston2002}), though there
        are speakers for whom they are acceptable.}
    \z
\ex\label{ex:2.7} Infinitival \emph{for}-\glspl{RC}
    \ea    The man [for you to see] is John.
    \ex    The man [for her to speak to] is John.
    \z
\ex\label{ex:2.8} Infinitival $\varnothing$-\glspl{RC}
    \ea    The man [to see] is John.
    \ex    The man [to speak to] is John.
    \z
\z
The names for the different types of \gls{RC}\is{relative clauses} should be reasonably transparent. I do
not refer to \emph{wh}-\glspl{RC} with and without preposition \is{pied-piping}pied-piping as
different types. Furthermore, I classify examples like (\ref{ex:2.6}c) as infinitival
\emph{wh}-\glspl{RC} rather than infinitival\is{relative clauses!infinitival} \emph{for}-\glspl{RC} since the
\emph{wh}-phrase is further to the left. $\varnothing${-\glspl{RC} are those without an
overt} \emph{wh}{-relative pronoun,} \emph{that} or \emph{for}.

The idea that \glspl{RC}\is{relative clauses} might vary in structural size is not new, with a number of
authors claiming a size difference between finite \glspl{RC}\is{relative clauses} introduced by an overt
relative pronoun or complementiser and those not \parencite{Boskovic1994,Boskovic1996,%
Boskovic1997,Boskovic2016,Weisler1980,Doherty1993,Doherty2000},
or between infinitival\is{relative clauses!infinitival} \glspl{RC}\is{relative clauses} relativising on subjects and those
relativising on non-subjects \citep{Bhatt1999}. However, previous studies tend
not to consider finite and infinitival\is{relative clauses!infinitival} \glspl{RC}\is{relative clauses} together, nor to consider the
issue from a serious cartographic perspective (though
see~\citealt{Haegeman2012} for the application of such an approach to a range
of clause types in \ili{English}).

My more specific aim is thus to determine the structure and size of the left
periphery of full clausal \glspl{RC}. To investigate this question, I test
whether full clausal \glspl{RC}\is{relative clauses} of the various types illustrated above are
compatible with adverbial and argument fronting\is{fronting!of
arguments}\is{fronting!adverbial} (including negative preposing),
as done in~\textcite{Haegeman2012} for a range of clause types following the
cartographic tradition (\citealt{Rizzi1997} et seq.\ among many others).
Unlike \textcite{Haegeman2012}, I \isi{focus} exclusively on RCs, demonstrating that
there is a lot more to say about \glspl{RC}\is{relative clauses} and fronting possibilities in their
left peripheries. This is largely a result of empirical differences. Haegeman
writes:

\begin{quote}In the following discussion judgments are based on the literature and on a
number of informants, all speakers of British \ili{English}. There is, however,
interspeaker variation, and some speakers are much more liberal when it comes
to the distribution of \is{fronting!of arguments}fronted arguments in \ili{English}. These speakers may well
find that their judgments deviate systematically from those discussed here.
Given that the divergence is systematic, I tentatively conclude that their
grammar must differ from that of the speakers on whom this work is based.
\parencite[54]{Haegeman2012} \end{quote}

I, and some that I have informally consulted, seem to belong to the
\enquote{much more liberal} speakers of British \ili{English} (others that I have
consulted seem to belong to Haegeman’s \enquote{not-so-liberal}
group).\footnote{\textcite{Haegeman2012} notes where some authors seem to be
more liberal, e.g.\ \textcite{Radford2009}.} The biggest difference between
\citegen{Haegeman2012} reported judgements and those to be reported below is
that Haegeman essentially rejects argument fronting in all \glspl{RC}\is{relative clauses} (a
long-standing and widespread claim in the literature,
see~\citealt{Chomsky1977} and \citealt{Bak1984}), whilst I (and some of my
consultants) accept it in some (but not all) RC-types. Nonetheless, even when
it is permitted, argument fronting is constrained. I will argue that argument
fronting is subject to what will be called a \emph{categorial distinctness
effect} (see \citealt{Richards2010}), i.e.\ an argument that is fronted inside
an \gls{RC}\is{relative clauses} must be of a different phrasal category from
whatever is relativised.  This will become apparent in \Cref{sec:02.3}.

The structure of this chapter is as follows. The adverbial
fronting\is{fronting!adverbial} data is laid out in \Cref{sec:02.2}, whilst
the argument fronting data and the aforementioned \isi{categorial distinctness}
effect are presented in \Cref{sec:02.3}. My analysis is laid out in
\Cref{sec:02.4} and suggests a close formal relation between relativisation
and topicalisation\is{topicalization} (at least in finite \gls{RC}\is{relative clauses}
contexts). \Cref{sec:02.5} concludes.

\section{Adverbial fronting}\label{sec:02.2}

\subsection{Finite \emph{wh}-RCs}\label{sec:02.2.1}

Adverbial fronting and adverbial negative preposing seem to behave in more or
less the same way, except that adverbial negative preposing triggers so-called
subject--auxiliary inversion. In this section, I will show that adverbial
fronting\is{fronting!adverbial} is permitted in finite \emph{wh}- and \emph{that}-\glspl{RC} and in
infinitival \emph{wh}{-RCs, but is not permitted in the other RC-types.}

Adverbial fronting is permitted in \emph{wh}{-RCs, b}oth in non-subject RCs, as
in \eqref{ex:2.9}, and in subject RCs, as in \eqref{ex:2.10} (see also
\citealt{Doherty1993,Doherty2000}). The same
applies to adverbial negative preposing, as in \eqref{ex:2.11} (non-subject RCs) and \eqref{ex:2.12}
(subject RCs).

\ea\label{ex:2.9}
    \ea    I met a man who \textit{next year} Mary might (actually) date.
    \ex    I bought a dress which \textit{next year} Mary might (actually) wear.
    \z
\ex\label{ex:2.10}
    \ea    I met a man who \textit{next year} might (actually) date Mary.
    \ex    I bought a dress which \textit{next year} might (actually) make Mary popular.
    \z
\ex\label{ex:2.11}
    \ea    I met a man who \textit{under no circumstances} would Mary ever date.
    \ex    I bought a dress which \textit{under no circumstances} would Mary ever wear.
    \z
\ex\label{ex:2.12}
    \ea    I met a man who \textit{under no circumstances} would ever go out with Mary.
    \ex    I bought a dress which \textit{under no circumstances} would ever make Mary popular.
    \z
\z
The \emph{wh}-relative pronoun may or may not \is{pied-piping}pied-pipe a preposition.
Adverbial fronting is compatible with either option, as in \eqref{ex:2.13}. The same
applies to adverbial negative preposing, as in \eqref{ex:2.14}.

\ea\label{ex:2.13}
    \ea    I met a man who \textit{next year} Mary might (actually) grant a second date to.
    \ex    I met a man to whom \textit{next year} Mary might (actually) grant a second date.
    \z
\ex\label{ex:2.14}
    \ea    I met a man who \textit{under no circumstances} would Mary ever grant a first date   to.
    \ex    I met a man to whom \textit{under no circumstances} would Mary ever grant a first   date.
    \z
\z
\subsection{Finite \emph{that}-RCs}

Adverbial fronting is permitted in \emph{that}-RCs, both in non-subject RCs,
as in \eqref{ex:2.15}, and in subject RCs, as in \eqref{ex:2.16} (see also
\citealt{Doherty1993,Doherty2000}). The
same applies to adverbial negative preposing, as in \eqref{ex:2.17} (non-subject RCs) and
\eqref{ex:2.18} (subject RCs).

\ea\label{ex:2.15}
    \ea    I met a man that \textit{next year} Mary might (actually) date.
    \ex    I bought a dress that \textit{next year} Mary might (actually) wear.
    \z
\ex\label{ex:2.16}
    \ea    I met a man that \textit{next year} might (actually) date Mary.
    \ex    I bought a dress that \textit{next year} might (actually) make Mary popular.
    \z
\ex\label{ex:2.17}
    \ea    I met a man that \textit{under no circumstances} would Mary ever date.
    \ex    I bought a dress that \textit{under no circumstances} would Mary ever wear.
    \z
\ex\label{ex:2.18}
    \ea    I met a man that \textit{under no circumstances} would ever go out with Mary.
    \ex    I bought a dress that \textit{under no circumstances} would ever make Mary popular.
    \z
\z
\emph{that}-\glspl{RC} do not permit \is{pied-piping}pied-piping of prepositions at all so (\ref{ex:2.19}b) and
(\ref{ex:2.20}b) are ungrammatical independently of adverbial fronting\is{fronting!adverbial} and adverbial
negative preposing respectively.

\ea\label{ex:2.19}
    \ea[]{I met a man that \textit{next year} Mary might (actually) grant a second date to.}
    \ex[*]{I met a man to that (\textit{next year}) Mary might (actually) grant a second date.}
    \z
\ex\label{ex:2.20}
    \ea[]{I met a man that \textit{under no circumstances} would Mary ever grant a first date to.}
    \ex[*]{I met a man to that (\textit{under no circumstances}) would Mary ever grant a first date.}
    \z
\z

\subsection{Finite ∅-RCs}

Unlike in finite \emph{wh}{-\glspl{RC} and finite} \emph{that}{-RCs, adverbial
fronting\is{fronting!adverbial} is not permitted in finite}
$\varnothing$-\glspl{RC} (see also \citealt{Doherty1993,Doherty2000}). This applies to both
non-subject RCs, as in \eqref{ex:2.21}, and subject RCs, as
in~\eqref{ex:2.22}. Note, however, that finite subject
$\varnothing${-\glspl{RC} are} generally impossible in (standard)
\ili{English}.\footnote{There are apparent counterexamples, such as (i):

    \begin{exe}
        \exi{(i)}  There’s a man sells vegetables at the market.
    \end{exe}

However, there is good reason to believe that these are not instances of
genuine $\varnothing$-\glspl{RC} (see
\citealt{denDikken2005,Harris1980,Henry1995,Lambrecht1988,McCawley1998}),
so I set these aside (pace \citealt{Doherty1993,Doherty2000}).} In other
words, the examples in \eqref{ex:2.22} are ungrammatical independently of adverbial
fronting. Exactly the same holds of adverbial negative preposing, as in \eqref{ex:2.23}
(non-subject RCs) and \eqref{ex:2.24} (subject RCs).

\ea\label{ex:2.21}
    \ea[*]{I met a man \textit{next year} Mary might (actually) date.}
    \ex[*]{I bought a dress \textit{next year} Mary might (actually) wear.}
    \z
\ex\label{ex:2.22}
    \ea[*]{I met a man (\textit{next year}) might (actually) date Mary.}
    \ex[*]{I bought a dress (\textit{next year}) might (actually) make Mary popular.}
    \z
\ex\label{ex:2.23}
    \ea[*]{I met a man \textit{under no circumstances} would Mary ever date.}
    \ex[*]{I bought a dress \textit{under no circumstances} would Mary ever wear.}
    \z
\ex\label{ex:2.24}
    \ea[*]{I met a man (\textit{under no circumstances}) would (ever) go out with Mary.}
    \ex[*]{I bought a dress (\textit{under no circumstances}) would (ever) make Mary popular.}
    \z
\z
\emph{Ø-}RCs do not permit \is{pied-piping}pied-piping of prepositions in general. Hence (\ref{ex:2.25}b)
and (\ref{ex:2.26}b) are ungrammatical independently of adverbial fronting\is{fronting!adverbial} or adverbial
negative preposing respectively.

\ea\label{ex:2.25}
    \ea[*]{I met a man \textit{next year} Mary might (actually) grant a second date to.}
    \ex[*]{I met a man to (\textit{next year}) Mary might (actually) grant a second date.}
    \z
\ex\label{ex:2.26}
    \ea[*]{I met a man \textit{under no circumstances} would Mary ever grant a first date.}
    \ex[*]{I met a man to (\textit{under no circumstances}) would Mary ever grant a first    date.}
    \z
\z

\subsection{Infinitival \emph{wh}-RCs}

In \ili{English}, infinitival\is{relative clauses!infinitival} \emph{wh}{-\glspl{RC} obligatorily involve a \is{pied-piping}pied-piped
preposition.} Subject infinitival\is{relative clauses!infinitival} \emph{wh}{-\glspl{RC} are consequently impossible
because subjects do not have any prepositions to \is{pied-piping}pied-pipe. All of the examples
therefore involve non-subject relativisation. As can be seen, adverbial
fronting\is{fronting!adverbial}} and adverbial negative preposing is permitted, as in \eqref{ex:2.27} and \eqref{ex:2.28}
respectively.

\ea\label{ex:2.27}
    Mary’s the woman to whom \textit{next week} to hand these documents.
\ex\label{ex:2.28}
    Mary’s the woman to whom \textit{under no circumstances} to ever hand these documents.
\z
Some speakers allow the complementiser \emph{for} and an overt subject in
infinitival \emph{wh}{-RCs, though even then it is typically judged as somewhat
degraded. Other} speakers judge it ungrammatical (see
\citealt{ChomskyLasnik1977}; \citealt[1067]{Huddleston2002}). For those that do accept such
structures, adverbial fronting\is{fronting!adverbial} is permitted in such cases. The fronted
adverbial obligatorily precedes \emph{for}, as in \eqref{ex:2.29}.

\judgewidth{??}
\ea\label{ex:2.29}
    \ea[??]{Mary’s the woman to whom \textit{next week} for you to hand these documents.}
    \ex[*]{Mary’s the woman to whom for you \textit{next week} to hand these documents.}
    \z
\z
The same seems to be true for adverbial negative preposing, as in \eqref{ex:2.30}.

\ea\label{ex:2.30}
    \ea[??]{Mary’s the woman to whom \textit{under no circumstances} for you to ever hand these documents.}
    \ex[*]{Mary’s the woman to whom for you \textit{under no circumstances} to ever hand these documents.}
    \z
\z

\subsection{Infinitival \emph{for}-RCs}

Unlike in infinitival\is{relative clauses!infinitival} \emph{wh}{-\glspl{RC} (with and without} \emph{for}), adverbial
fronting is not permitted in infinitival\is{relative clauses!infinitival} \emph{for}{-RCs, i.e.\ infinitival\is{relative clauses!infinitival} RCs
with overt} \emph{for} but no \emph{wh}{-relative pronoun}, as in \eqref{ex:2.31}. The
same applies to adverbial negative preposing, as in \eqref{ex:2.32}.

\ea\label{ex:2.31}
    \ea[?*]{I met a man \textit{next year }for you to bring to the party.}
    \ex[*]{I met a man for you \textit{next year }to bring to the party.}
    \z
\ex\label{ex:2.32}
    \ea[*]{I met a man \textit{under no circumstances} for you to ever bring to the party.}
    \ex[*]{I met a man for you \textit{under no circumstances} to ever bring to the party.}
    \z
\z
Infinitival \emph{for}-\glspl{RC} do not permit \is{pied-piping}pied-piping of prepositions in
general. Hence (\ref{ex:2.33}) and \eqref{ex:2.34} are ungrammatical independently of adverbial
fronting and adverbial negative preposing.
\judgewidth{*}

\ea[*]{Mary’s the woman to (\textit{next week}) for you to hand these documents.}\label{ex:2.33}
\ex[*]{Mary’s the woman to (\textit{under no circumstances}) for you to ever hand these documents.}\label{ex:2.34}
\z

\subsection{Infinitival ∅-RCs}

Like in infinitival\is{relative clauses!infinitival} \emph{for}{-RCs, adverbial fronting\is{fronting!adverbial} is not permitted in
infinitival} $\varnothing${-RCs, i.e.\ infinitival\is{relative clauses!infinitival} \glspl{RC}\is{relative clauses} with neither} \emph{for} nor
a \emph{wh}{-relative pronoun}, as in \eqref{ex:2.35}. The same applies to adverbial
negative preposing, as in \eqref{ex:2.36}.

\ea[*]{I met a man \textit{next year} to bring to the party.}\label{ex:2.35}
\ex[*]{I met a man \textit{under no circumstances} to ever bring to the party.}\label{ex:2.36}
\z
Infinitival $\varnothing${-\glspl{RC} do not permit \is{pied-piping}pied-pi}ping of prepositions in
general, hence \eqref{ex:2.37} and \eqref{ex:2.38} are ungrammatical independently of adverbial
fronting and adverbial negative preposing.

\ea[*]{Mary’s the woman to (\textit{next }\textit{week}) to hand these documents.}\label{ex:2.37}
\ex[*]{Mary’s the woman to (\textit{under no circumstances}) to ever hand these documents.}\label{ex:2.38}
\z

\subsection{Summary}

Adverbial fronting and adverbial negative preposing are permitted in finite
\emph{wh}{-RCs, finite} \emph{that}{-RCs, and infinitival} \emph{wh}{-\glspl{RC} (with
and without} \emph{for}). They are not permitted in finite $\varnothing${-RCs,
infinitival} \emph{for}{-RCs, and infinitival} $\varnothing${-RCs. Furthermore,}
they do not seem to interact with preposition \is{pied-piping}pied-piping in any way.

\section{Argument fronting}\label{sec:02.3}

\subsection{Finite \emph{wh}-RCs}

I turn now to argument fronting. As I will show, argument fronting is more
constrained than adverbial fronting.\is{fronting!adverbial} Indeed, as pointed
out in \Cref{sec:02.1}, \citegen{Haegeman2012} analysis is based on cases where
argument fronting in \glspl{RC}\is{relative clauses} is generally impossible.
This seems to be true for some of the speakers I have consulted as well.
However, other speakers are \enquote{more liberal}. Nevertheless, even for
these more liberal speakers it is not the case that \is{fronting!of
arguments}fronted arguments are freely permitted in all types of \gls{RC}. As
will be seen, argument fronting exhibits a \emph{categorial distinctness
    effect}. Anticipating the findings, argument fronting is permitted in
    finite \emph{wh}- and \emph{that}-\glspl{RC} but not in the other
    RC-types.%\largerpage[3]

Let us first consider non-subject \glspl{RC}. Fronted arguments are acceptable
to \enquote{more liberal} informants, as in
\eqref{ex:2.39}.\footnote{Similarly, \textcite[282]{Radford2009} judges the
    following example as acceptable:

\begin{exe}
    \exi{(i)}  A university is the kind of place in which, that kind of behaviour, we cannot tolerate.
\end{exe}} The \is{fronting!of arguments}fronted argument
obligatorily follows the relative pronoun\is{relative pronouns}, as shown by the ungrammaticality of
\eqref{ex:2.40}.

\ea\label{ex:2.39}
    \ea    I met a man to whom, \textit{a second date}, Mary might actually grant.
    \ex    I bought a car in which, \textit{muddy shoes}, I would never allow.
    \z
\ex\label{ex:2.40}
    \ea[*]{I met a man, \textit{a second date}, to whom Mary might actually grant.}
    \ex[*]{I bought a car, \textit{muddy shoes}, in which I would never allow.}
    \z
\z
However, argument fronting is restricted. Observe that in \eqref{ex:2.39} the
\emph{wh}-relative pronouns have \is{pied-piping}pied-piped a preposition. Interestingly,
without such \is{pied-piping}pied-piping, the examples become degraded or unacceptable, as in
\eqref{ex:2.41}.

\ea\label{ex:2.41}
    \ea[?*]{I met a man who(m), \textit{a second date}, Mary might actually grant to.}
    \ex[?*]{I bought a car which, \textit{muddy shoes}, I would never allow in.}
    \z
\z
The same effect can be seen when it is the \is{fronting!of arguments}fronted argument rather than the
relative pronoun that has the option of \is{pied-piping}pied-piping a preposition. In \eqref{ex:2.42}, the
fronted argument has \is{pied-piping}pied-piped a preposition and the result is acceptable,
whilst in \eqref{ex:2.43}, it has not \is{pied-piping}pied-piped a preposition and the result is
unacceptable.

\ea\label{ex:2.42}
    I witnessed the second date which, \textit{to that man}, Mary should never have granted.
\ex[*]{I witnessed the second date which, \textit{that man}, Mary should never have granted to.}\label{ex:2.43}
\z
What these data tell us is that the relative pronoun\is{relative pronouns} and \is{fronting!of arguments}fronted argument
cannot both be nominal phrases (DPs). If one is a DP, the other must \is{pied-piping}pied-pipe
a preposition, i.e.\ be a prepositional phrase (PP). To my knowledge, this is a
novel empirical generalisation. Adopting \citegen{Richards2010} terminology, I refer
to this as a \emph{categorial distinctness effect}.

This raises the question of what happens when both the relative pronoun\is{relative pronouns} and
fronted argument \is{pied-piping}pied-pipe a preposition. The result is grammatical (example
adapted from \textcite{Totsuka2014}.

\ea\label{ex:2.44}
    I met a man \textit{with whom}, \textit{about linguistics}, I could talk all day.
\z
However, there is an issue about whether the fronted PP in such examples is
actually an argument (see \citealt[294, 322--325]{Rizzi1997}). I leave such
examples aside for now but will return to them in \Cref{sec:02.4.4}.\largerpage

The \isi{categorial distinctness} effect is particularly important when it comes to
argument fronting in subject \glspl{RC}. It has been claimed that fronted topics, or
fronted arguments more generally, are impossible in subject RCs
\parencites[58]{Haegeman2012}[307]{Rizzi1997}. The following examples, taken
from \textcite[307]{Rizzi1997}, are intended to show that \is{fronting!of arguments}fronted arguments are
possible in non-subject RCs, as in (\ref{ex:2.45}a) and (\ref{ex:2.46}a), but impossible in subject
RCs, as in (\ref{ex:2.45}b) and (\ref{ex:2.46}b) (judgements as in the
original).\footnote{\textcite[Ch.\ 2, note 6]{Haegeman2012} notes via personal
    communication with Andrew Radford that he accepts the following:

    \begin{exe}
        \exi{(i)}  He’s the kind of person who, a noble gesture like that, would simply not appreciate.
    \end{exe}
    I, and others, find this example odd. We feel that it needs a subject
    resumptive pronoun to be even marginally acceptable, as in (ii).
    Interestingly, an object resumptive does not seem even marginally possible,
    as in (iii). See \Cref{sec:02.4.4} for discussion.

    \begin{exe}
        \exi{(ii)}[?]{He’s the kind of person who, a noble gesture like that, he would simply not appreciate.}
        \exi{(iii)}[*]{He’s the kind of person who, a noble gesture like that, would simply not appreciate it.}
\end{exe}\label{fn:02.5}}

\ea\label{ex:2.45}\judgewidth{??}
    \ea[??]{the man to whom, \textit{that book}, I gave}
    \ex[*]{the man who, \textit{that book}, gave to me}
    \z
\ex\label{ex:2.46}
    \ea[?]{a man to whom, \textit{liberty}, we should never grant\footnotemark}
    \ex[*]{a man who, \textit{liberty}, should never grant to us}
    \z
\z
\footnotetext{(\ref{ex:2.46}a) is adapted from \textcite[17]{Baltin1982}. Baltin
judges it as acceptable, but notes that not all speakers find it totally
acceptable.}
%
However, observe that the non-subject \gls{RC}\is{relative clauses} examples in (\ref{ex:2.45}a) and (\ref{ex:2.46}a) satisfy
categorial distinctness whilst the subject \gls{RC}\is{relative clauses} examples in (\ref{ex:2.45}b) and (\ref{ex:2.46}b) do
not. If the \isi{categorial distinctness} effect is responsible for the
ungrammaticality of (\ref{ex:2.45}b) and (\ref{ex:2.46}b), the prediction is that \is{fronting!of arguments}fronted arguments
will be allowed in subject \glspl{RC}\is{relative clauses} provided that the \is{fronting!of arguments}fronted argument \is{pied-piping}pied-pipes a
preposition. This prediction is borne out as the contrast between \eqref{ex:2.47} and \eqref{ex:2.48}
shows.\largerpage[1.5]

\ea\label{ex:2.47}
    \ea[*]{I met a man who, \textit{Mary}, might actually grant a second date to.}
    \ex[*]{I bought a car which, \textit{children}, can give hours of entertainment to.}
    \z
\ex\label{ex:2.48}
    \ea[?]{I met a man who, \textit{to Mary}, might actually grant a second date.}
    \ex[]{I bought a car which, \textit{to children}, can give hours of entertainment.}
    \z
\z
These data thus show that argument fronting \emph{is} permitted in subject RCs
but that the \is{fronting!of arguments}fronted argument must be a PP in line with the categorial
distinctness effect.

The same effect can be seen with argument negative preposing. As the contrasts
below show, if the relative pronoun\is{relative pronouns} has not \is{pied-piping}pied-piped a preposition, the
fronted argument must do so. This applies to both non-subject and subject \glspl{RC}.

\ea\label{ex:2.49}
    \ea     I met a man who(m), \textit{to no woman}, would I ever recommend (as a date).
    \ex     I bought a dress which, \textit{to no woman}, would I ever give (as a present).
    \ex     I met a man who, \textit{to no woman}, would ever give roses.
    \ex     I bought a dress which, \textit{to no woman}, would ever be given (as a present).
    \z
\ex\label{ex:2.50}
    \ea [*]{I met a man who(m), \textit{no woman}, would I ever recommend to (as a date).}
    \ex [*]{I bought a dress which, \textit{no woman}, would I ever give to (as a present).}
    \ex [*]{I met a man who, \textit{no woman}, would ever give roses to.}
    \ex [*]{I bought a dress which, \textit{no woman}, would ever be given to (as a present).}
    \z
\z
The negative preposed argument can only be a DP if the relative pronoun
pied-pipes a preposition.

\ea\label{ex:2.51}\judgewidth{\textsuperscript{??}}
    \ea    I met a man to whom, \textit{no advice} would I ever give.
    \ex    I met a woman to whom, \textit{no roses} would a man ever give.
    \z
\ex\label{ex:2.52}
    \ea[*]{I met a man who(m), \textit{no advice} would I ever give to.}
    \ex[\textsuperscript{??}]{I met a woman who(m), \textit{no roses} would a man ever give to.}
    \z
\z
To summarise, I have shown that argument fronting is permitted in finite
\emph{wh}{-\glspl{RC} but is subject to a \isi{categorial distinctness} effect. The
categorial distinctness effect says that a relative pronoun\is{relative pronouns} and fronted
argument cannot both be} DPs. If one is a DP, the other must be a PP\@. This is
schematised in \tabref{tab:15:1}.%\footnote{\citet[306]{Rizzi1997} claims that,
%in French, intervening CLLD phrases induce subject-object asymmetries.
%
%\begin{exe}
%    \exi{(i)} \ili{French}
%    \ea[?]{Je ne sais pas à qui, ton livre, je pourrais le donner.\\
%        \enquote*{I don’t know to whom, your book, I could give it.}}
%    \ex[*?]{Je ne sais pas qui, ton livre, pourrait l’acheter.\\
%        \enquote*{I don’t know who, your book, could buy it.}}
%    \z
%\end{exe}
%
%\begin{exe}
%    \exi{(ii)} \ili{French}
%    \ea[?]{un homme à qui, ton livre, je pourrais le donner\\
%        \enquote*{a man to whom, your book, I could give it}}
%    \ex[*?]{un homme qui, ton livre, pourrait l’acheter\\
%        \enquote*{a man who, your book, could buy it}}
%    \z
%\end{exe}
%He contrasts the French data with data from Italian, which does not exhibit a
%subject-object asymmetry in such cases \citep[306]{Rizzi1997}:
%
%\begin{exe}
%    \exi{(iii)} \ili{Italian}
%    \ea     un uomo a cui, il tuo libro, lo potremmo dare\\
%            \enquote*{a man to whom, your book, we could give it}
%    \ex     un uomo che, il tuo libro, lo potrebbe comprare\\
%            \enquote*{a man who, your book, could buy it}
%    \z
%\end{exe}
%Since Italian does not exhibit \emph{that}-trace effects, he concludes that the
%French data are showing a subject-object asymmetry. However, given what we have
%found in \ili{English}, we need to test whether this is correct, or whether a PP CLLD
%structure would improve the judgement for the French examples (ib) and (iib).
%(iiib) shows that Italian does not exhibit the distinctness effect seen in
%English.}

\begin{table}
\begin{tabular}{ccc}
\lsptoprule
Relative pronoun & Fronted argument & Combination allowed?\\\midrule
DP & DP & *\\
DP & PP & ✔\\
PP & DP & ✔\\
PP & PP & See \Cref{sec:02.4.4}\\
\lspbottomrule
\end{tabular}
\caption{Categorial distinctness effect\label{tab:15:1}}
\end{table}

\subsection{Finite \emph{that}-RCs}

Argument fronting is permitted in finite \emph{that}-\glspl{RC} and is subject to the
categorial distinctness effect. However, for whatever reason, preposition
pied-piping is not possible with \emph{that}, which rules out PP--DP and
PP--PP, and I predict from the \isi{categorial distinctness} effect that option
DP--DP is not available either. Consequently, I predict that DP--PP is the only
option, i.e.\ the \is{fronting!of arguments}fronted argument can only be a PP. This prediction is borne
out and applies to both non-subject and subject \glspl{RC}.

\ea\label{ex:2.53}
    \ea[]{I bought a dress that, \textit{to Mary}, I might consider giving (as a present).}
    \ex[]{I bought a dress that, \textit{to Mary}, could be given (as a present).}
    \ex[]{I bought a car that, \textit{to children}, would give hours of entertainment.}
    \z
\ex\label{ex:2.54}
    \ea[*]{I bought a dress that, \textit{Mary}, I might consider giving to (as a present).}
    \ex[*]{I bought a dress that, \textit{Mary}, could be given to (as a present).}
    \ex[*]{I bought a car that, \textit{children}, would give hours of entertainment to.}
    \z
\z
The same applies to argument negative preposing.

\ea\label{ex:2.55}
    \ea    I bought a dress that, \textit{to no woman}, would I ever give (as a present).
    \ex    I bought a dress that, \textit{to no woman}, would ever be given (as a present).
    \z
\ex\label{ex:2.56}
    \ea[*]{I bought a dress that, \textit{no woman}, would I ever give to (as a present).}
    \ex[*]{I bought a dress that, \textit{no woman}, would ever be given to (as a present).}
    \z
\z
If there is no preposition for the \is{fronting!of arguments}fronted argument to \is{pied-piping}pied-pipe in the first place, we predict that argument fronting will simply be unavailable. This prediction is also borne out as the following examples show.

\ea\label{ex:2.57}\judgewidth{??}
    \ea[*]{I bought a car that, \textit{muddy shoes}, I would never allow in.}
    \ex[*]{I bought a car that, \textit{hours of entertainment}, would give to children.}
    \ex[*]{I bought a car that, \textit{the children}, can keep entertained.}
    \z
\ex\label{ex:2.58}
    \ea[??]{I bought a car that, \textit{not a single muddy shoe} would I ever allow in.}
    \ex[*]{I bought a car that, \textit{not a single hour of entertainment}, would ever give to any child.}
    \ex[*]{I bought a car that, \textit{no child}, can keep entertained.}
    \z
\z

\subsection{Finite ∅-RCs}

Unlike finite \emph{wh}{-\glspl{RC} and finite} \emph{that}{-RCs, argument fronting is
not permitted in finite} $\varnothing${-\glspl{RC} at all, even if the} \is{fronting!of arguments}fronted argument is
a PP\@. Since subject $\varnothing${-\glspl{RC} are generally impossible in \ili{English}, only
non-subject} $\varnothing${-\glspl{RC} are illustrated.}

\ea\label{ex:2.59}
    \ea[*]{I met a man, \textit{Mary}, I might recommend to (as a date).}
    \ex[*]{I bought a dress, \textit{Mary}, I could give to (as a present).}
    \z
\ex\label{ex:2.60}
    \ea[*]{I met a man, \textit{to Mary}, I might recommend (as a date).}
    \ex[*]{I bought a dress, \textit{to Mary}, I could give (as a present).}
    \z
\z
Pied-piping of prepositions is not permitted with $\varnothing$. Therefore, if
argument fronting were possible at all, we would expect PP \is{fronting!of arguments}fronted arguments to
be possible, as they were with \emph{that}{-RCs. Since PP \is{fronting!of arguments}fronted arguments are
impossible,} I conclude that argument fronting is generally impossible in
finite $\varnothing${-RCs.}

Argument negative preposing behaves in exactly the same way.

\ea\label{ex:2.61}
    \ea[*]{I met a man, \textit{no woman} would I ever recommend to (as a date).}
    \ex[*]{I bought a dress, \textit{no woman} would I ever give to (as a present).}
    \z
\ex\label{ex:2.62}
    \ea[*]{I met a man, \textit{to no woman} would I ever recommend (as a date).}
    \ex[*]{I bought a dress, \textit{to no woman} would I ever give (as a present).}
    \z
\z

\subsection{Infinitival \emph{wh}-RCs}

Argument fronting is not permitted in infinitival\is{relative clauses!infinitival} \emph{wh}{-\glspl{RC} (regardless of
whether} \emph{for} is present or not), even if the \is{fronting!of arguments}fronted argument is a DP\@.
Since infinitival\is{relative clauses!infinitival} \emph{wh}{-\glspl{RC} obligatorily involve \is{pied-piping}pied-piping of a
preposition,} if argument fronting were possible at all, we would expect DP
fronted arguments to be possible. Since they are not, I conclude that argument
fronting is generally impossible in infinitival\is{relative clauses!infinitival} \emph{wh}{-RCs.}

\ea\label{ex:2.63}
    \ea[*]{I found an ideal venue in which, \textit{Mary}, to propose to.}
    \ex[*]{I found an ideal venue in which, \textit{Mary}, for you to propose to.}
    \ex[*]{I found an ideal venue in which for you, \textit{Mary}, to propose to.}
    \z
\ex\label{ex:2.64}
    \ea[*]{I found an ideal venue in which, \textit{to} \textit{Mary}, to propose.}
    \ex[*]{I found an ideal venue in which, \textit{to} \textit{Mary}, for you to propose.}
    \ex[*]{I found an ideal venue in which for you, \textit{to} \textit{Mary}, to propose.}
    \z
\z
Similarly, argument negative preposing is not permitted (regardless of whether
\emph{for} is present or not, and regardless of whether the \is{fronting!of arguments}fronted argument is
a PP or a DP).

\ea\label{ex:2.65}
    \ea[*]{This is a place in which, \textit{no man}, to ever give one’s real name to.}
    \ex[*]{This is a place in which, \textit{no man}, for you to ever give your real name to.}
    \ex[*]{This is a place in which for you, \textit{no man}, to ever give your real name to.}
    \z
\ex\label{ex:2.66}
    \ea[*]{This is a place in which, \textit{to no man}, to ever give one’s real name.}
    \ex[*]{This is a place in which, \textit{to no man}, for you to ever give your real name.}
    \ex [*]{This is a place in which for you, \textit{to no man}, to ever give your real name.}
    \z
\z

\subsection{Infinitival \emph{for}-RCs}

As with infinitival\is{relative clauses!infinitival} \emph{wh}{-RCs, a}rgument fronting is not permitted in infinitival\is{relative clauses!infinitival} \emph{for}{-\glspl{RC} at all, regardless of whether the} \is{fronting!of arguments}fronted argument is a DP or a PP\@.

\ea\label{ex:2.67}
    \ea[*]{I found an ideal venue, \textit{Mary}, for you to propose to in.}
    \ex[*]{I found an ideal venue for you, \textit{Mary}, to propose to in.}
    \z
\ex\label{ex:2.68}
    \ea[*]{I found an ideal venue, \textit{to Mary}, for you to propose in.}
    \ex[*]{I found an ideal venue for you, \textit{to Mary}, to propose in.}
    \z
\z
The same applies to argument negative preposing.

\ea\label{ex:2.69}
    \ea[*]{I saw a venue, \textit{no woman}, for one to propose to in.}
    \ex[*]{I saw a venue for one, \textit{no woman}, to propose to in.}
    \z
\ex\label{ex:2.70}
    \ea[*]{I saw a venue, \textit{to no woman}, for one to propose in.}
    \ex[*]{I saw a venue for one, \textit{to no woman}, to propose in.}
    \z
\z

\subsection{Infinitival ∅-RCs}

Finally, as with all other infinitival\is{relative clauses!infinitival} \glspl{RC}\is{relative clauses} so far, argument fronting is not
permitted in infinitival\is{relative clauses!infinitival} $\varnothing${-RCs, regardless of whether the} fronted
argument is a DP or a PP\@.

\ea[*]{I found an ideal venue, \textit{Mary}, to propose to in.}\label{ex:2.71}
\ex[*]{I found an ideal venue, \textit{to Mary}, to propose in.}\label{ex:2.72}
\z
The same applies to argument negative preposing.

\ea[*]{I saw a venue, \textit{no woman}, for one to propose to in.}\label{ex:2.73}
\ex[*]{I saw a venue, \textit{to no woman}, for one to propose in.}\label{ex:2.74}
\z

\subsection{Summary}

Argument fronting is permitted in finite \emph{wh}-\glspl{RC} and \emph{that}-RCs, and
is prohibited in finite $\varnothing$-\glspl{RC} and all infinitival\is{relative clauses!infinitival} \glspl{RC}. Where
argument fronting is permitted, it is subject to a categorial distinctness
effect. The relative pronoun\is{relative pronouns} (or relative operator in the case of
\emph{that}-\glspl{RC}) and \is{fronting!of arguments}fronted argument cannot both be DPs. If one is a
DP, the other must be a PP\@.  Exactly the same pattern is found with argument
negative preposing.

\section{Analysis and discussion}\label{sec:02.4}

\subsection{The distribution of adverbial and argument fronting}

Putting the conclusions from \Cref{sec:02.2} and \Cref{sec:02.3} together, we have the empirical
situation regarding the distribution of adverbial and argument
fronting\is{fronting!adverbial}\is{fronting!of arguments} in
English \glspl{RC}\is{relative clauses} shown in \tabref{tab:15.2} (note that the terms \emph{adverbial
fronting} and \emph{argument fronting} will now be used to cover their negative
preposing counterparts as well).

\begin{table}
\fittable{\begin{tabular}{lcccccc}
\lsptoprule
 & \multicolumn{3}{c}{Finite} & \multicolumn{3}{c}{Infinitival}\\\cmidrule(lr){2-4}\cmidrule(lr){5-7}
& \emph{wh}-\glspl{RC} & \emph{that}-\glspl{RC}  & $\varnothing$-\glspl{RC} & \emph{wh}-\glspl{RC} & \emph{for}-\glspl{RC} & $\varnothing$-RCs\\
\midrule
Adverbial fronting &  ✔  &  ✔  & * &  ✔ & * & *\\
Argument fronting  & (✔) & (✔) & * &  * & * & *\\
\lspbottomrule
\end{tabular}}
\caption{Distribution of adverbial and argument fronting in full clausal
\glspl{RC}\is{relative clauses} in \ili{English}. ✔: allowed; (✔): allowed
subject to restrictions; *: not allowed.\label{tab:15.2}}
\end{table}

I propose that this distribution can be captured by positing (at least) three
distinct sizes of \gls{RC}\is{relative clauses} in \ili{English}, which I will
describe in cartographic terms.  \textcite[242]{Rizzi2004} proposes the
following articulation of the C-domain (* here means \enquote{iterable}):

\ea\label{ex:2.75}
    Force $>$ Top* $>$ Int $>$ Top* $>$ Focus $>$ Mod* $>$ Top* $>$ Fin $>$ IP
\z
SpecTopP hosts \isi{topic} phrases, SpecFocusP hosts \isi{focus} phrases, SpecIntP hosts
high \emph{wh}{-elements such as Italian} \emph{perché} ‘why’, and SpecModP
hosts fronted adverbials\is{fronting!adverbials} in all but \enquote{very special discourse contexts}
\parencite{Rizzi2004}. I will adopt the simplified version in \eqref{ex:2.76}.

\ea\label{ex:2.76}
    Force $>$ Top $>$ Foc $>$ Mod* $>$ Fin $>$ IP
\z
The reasons for this simplification are: (i) I am not concerned with Int; (ii)
English does not permit multiple topics (see \citealt{Haegeman2012} and
references therein); and (iii) \ili{English} topics can never follow foci (see
\citealt{Haegeman2012} and references therein). Fronted arguments can be topics
or foci.  Below, I will address the issue of whether the \is{fronting!of
arguments}fronted argument in \glspl{RC}\is{relative clauses} is a \isi{topic}
or a \isi{focus}.

I am now in a position to account for the distribution of argument fronting and
adverbial fronting\is{fronting!adverbial} in \glspl{RC}. In brief, I propose that finite \emph{wh}{-\glspl{RC} and}
\emph{that}{-\glspl{RC} are TopPs, infinitival} \emph{wh}-\glspl{RC} are FocPs, and finite
$\varnothing${-RCs, infinitival} \emph{for}{-\glspl{RC} and infinitival} $\varnothing${-\glspl{RC} are
FinPs (or alternatively, unsplit CPs). This proposal is summarised} in
\tabref{tab:15.3}.

\begin{table}
\small
\begin{tabularx}{\textwidth}{QCCCCCC}
\lsptoprule
& Finite \emph{wh}-\glspl{RC} & Finite \emph{that}-\glspl{RC} &  Infinitival \emph{wh}-\glspl{RC} &  Finite $\varnothing$-\glspl{RC} &  Infinitival \emph{for}-\glspl{RC} &  Infinitival $\varnothing$-RCs\\
\midrule
Adverbial fronting & ✔ & ✔ & ✔ & * & * & *\\
Argument fronting & (✔) & (✔) & * & * & * & *\\
Structural size &
\multicolumn{2}{c}{$\overbrace{\hphantom{2.5cm}\text{TopP}\hphantom{2.5cm}}$} &
$\overbrace{\text{FocP}}$ &
\multicolumn{3}{c}{$\overbrace{\hphantom{1.5cm}\text{FinP (or unsplit CP)}\hphantom{1.5cm}}$}\\
\lspbottomrule
\end{tabularx}
\caption{RC structures}\label{tab:15.3}
\end{table}

FinPs are too small to contain TopP, FocP or ModP\@. Consequently, they permit
neither argument nor adverbial fronting. In finite $\varnothing${-RCs, Fin is}
$\varnothing$, whilst in infinitival\is{relative clauses!infinitival} \emph{for}-RCs, Fin is lexicalised as
\emph{for}, in line with previous proposals
(\citealt{Haegeman2012,Radford2009a,Rizzi1997}). If infinitival\is{relative clauses!infinitival} $\varnothing${-RCs
are FinP}s, Fin is also $\varnothing$ in these cases. FocPs contain ModP, so
permit adverbial fronting.\is{fronting!adverbial} Argument
fronting\is{fronting!of arguments} is not permitted because FocP is too small
to contain TopP and because relativisation in infinitival\is{relative
    clauses!infinitival} \emph{wh}{-RCs targets SpecFocP\@. Finally, TopPs
contain FocP and ModP\@. Consequently, they permit argument fronting (focus
fronting) and adverbial fronting.} I assume that Top is lexicalised as
\emph{that} in \emph{that}{-RCs, but is} $\varnothing$ in finite
\emph{wh}{-\glspl{RC} (where the} \emph{wh}{-relative pronoun occupies
SpecTopP).} In the following subsections, I will expand on and discuss various
aspects of this proposal.

\subsection{FinP RCs}\largerpage

There is potentially a size difference between finite $\varnothing$-\glspl{RC}
and infinitival \emph{for}-\glspl{RC} on the one hand, and infinitival
$\varnothing$-\glspl{RC} on the other. The evidence comes from accessibility in
the sense of \textcite{KeenanComrie1977}, i.e.\ the \isi{grammatical functions} that
can be relativised. Finite $\varnothing$-\glspl{RC} and infinitival
\emph{for}-\glspl{RC} can relativise any argument (except the subject),
including arguments embedded inside (finite) clauses. Infinitival
$\varnothing$-\glspl{RC} can also relativise any argument (including the
subject), but cannot relativise out of an embedded finite clause
\citep{Longenbaugh2016}, at least for some speakers.\footnote{I have found the
judgements of (\ref{ex:2.79}e,f) to be somewhat variable.} This is shown in the following
examples (the (e) and (f) examples in \eqref{ex:2.77} to
\eqref{ex:2.79} are taken or adapted from
\citealt{Longenbaugh2016}).

\ea\label{ex:2.77}  Finite $\varnothing$-\glspl{RC}\judgewidth{?*}
    \ea[*]{I found a man can fix the sink.}
    \ex[]{I found a sink you can fix.}
    \ex[]{I found a woman you can give a present to.}
    \ex[]{I found a boy you can force to run faster.}
    \ex[]{I found a play you can prove was written by Shakespeare.}
    \ex[]{I found a play you can prove (that) Shakespeare wrote.}
    \z
\ex\label{ex:2.78}  Infinitival \emph{for}-\glspl{RC}
    \ea[*]{I found a man for to fix the sink.}
    \ex[]{I found a sink for you to fix.}
    \ex[]{I found a woman for you to give a present to.}
    \ex[]{I found a boy for you to force to run faster.}
    \ex[]{I found a play for you to prove was written by Shakespeare.}
    \ex[]{I found a play for you to prove (that) Shakespeare wrote.}
    \z
\ex\label{ex:2.79}  Infinitival $\varnothing$-\glspl{RC}
    \ea[]{I found a man to fix the sink.}
    \ex[]{I found a sink to fix.}
    \ex[]{I found a woman to give a present to.}
    \ex[]{I found a boy to force to run faster.}
    \ex[*]{I found a play to prove was written by Shakespeare.}
    \ex[?*]{I found a play to prove (that) Shakespeare wrote.}
    \z
\z
If this is correct, infinitival\is{relative clauses!infinitival}
$\varnothing${-\glspl{RC} seem to exhibit A$'$-properties in that arguments can
    be relativised without higher arguments intervening with such
    \isi{movement}, as well as A-properties in that such \isi{movement} is
clause-bound (at least for some speakers)}, as shown by the ungrammaticality of
relativising an element from an embedded finite clause in (\ref{ex:2.79}e, f). In contrast,
finite $\varnothing${-\glspl{RC} and infinitival} \emph{for}-\glspl{RC} exhibit
A$'$-properties.  \textcite{Longenbaugh2016} suggests that the hybrid
A$'$/A-properties are the result of a composite probe, i.e.\ one seeking both A-
and A$'$-related features. One could hypothesise that, if a C-domain is absent,
both A- and A$'$-features are present on T, whilst if a C-domain is present, the
A-features are on T and the A$'$-features in the C-domain. If this is correct,
this suggests the following three things. First, finite $\varnothing${-RCs,
infinitival} \emph{for}{-\glspl{RC} and infinitival} $\varnothing$-\glspl{RC}
all lack the requisite structure to host fronted adverbials and fronted
arguments, i.e.\ their C-domains contain no structure higher than FinP. Second,
finite $\varnothing$-\glspl{RC} and infinitival \emph{for}-\glspl{RC} do have
at least some portion of the C-domain. Third, infinitival
$\varnothing$-\glspl{RC} may lack a C-domain altogether.

\subsection{FocP RCs}

According to my proposal, infinitival\is{relative clauses!infinitival}
\emph{wh}{-\glspl{RC} do not permit argument fronting because relativisation
and argument fronting} would be competing for the same position, namely
SpecFocP\@. However, it has also been claimed in the literature that argument
fronting is generally impossible in infinitival clauses (see
\citealt[206--208]{Bianchi1999}). Evidence comes from the impossibility of
argument fronting in \isi{raising} and \isi{control} infinitivals
(\citealt[67--68]{Haegeman2012}; see also
\citealt[484--485]{HooperThompson1973}).

\ea\label{ex:2.80} \textcite[485]{HooperThompson1973}
    \ea[*]{My friends tend, the more liberal candidates, to support.}
    \ex[*]{I have decided, your book, to read.}
    \z
\z
Argument fronting is also prohibited in \glsunset{ECM}\gls{ECM}\is{exceptional
case marking}\is{ECM|see{exceptional case marking}} complements \parencite[Ch.\ 2, note 20]{Haegeman2012}.

\ea\label{ex:2.81} \textcite[297, fn.\ 14]{CulicoverLevine2001}
    \ea[*]{I really want, that solution, Robin to explore thoroughly.}
    \ex [*]{Police believe, the London area, the suspect to have left.}
    \z
\z
However, this evidence does not rule out structural size being relevant since
these infinitival\is{relative clauses!infinitival} clauses could themselves be too small to host fronted
arguments. Instead, we need to test an infinitival\is{relative clauses!infinitival} clause that is independently
considered to be quite large. If argument fronting is impossible in such cases,
this is evidence that argument fronting is simply impossible in infinitival
clauses regardless of their size. However, if argument fronting is possible, it
suggests that structural size does play a role in the availability of argument
fronting. In this respect, consider embedded questions. It is typically said
that \emph{wh}{-phrases in embedded} finite contexts target a higher
position in the left periphery (SpecForceP) than in matrix contexts (SpecFocP)
(see \citealt{Haegeman2012,Pesetsky1995}), thereby capturing the observation
that matrix \emph{wh}{-phrases follow topics but embedded} \emph{wh}{-phrases
precede them. The high position of} \emph{wh}{-phrases in embedded clauses is}
potentially related to clause-typing \citep{Cheng1991}. Now, assuming that
\emph{wh}-phrases in embedded infinitival\is{relative clauses!infinitival} questions also occupy a high left
peripheral position for clause-typing, observe that argument fronting seems to
be possible. The examples may not be perfect, but they certainly seem better
than those in \eqref{ex:2.80} and \eqref{ex:2.81}.\largerpage

\ea\label{ex:2.82}
    \ea[?]{John didn’t know what, to Mary especially, to say at a time like that.}
    \ex[?]{I asked to whom, this particular form, to give so that it would be processed promptly.}
    \z
\z
Therefore, it seems that argument fronting is not incompatible with infinitival
contexts per se (pace \citealt{Bianchi1999}), and I thus conclude that
infinitival \emph{wh}{-\glspl{RC} do not permit argument fronting because they are
structurally too small} and not because they are infinitival.

Finally, a potential problem is that infinitival\is{relative clauses!infinitival} \emph{wh}{-\glspl{RC} do not seem
to be necessarily associated with \isi{focus} interpretations (Luigi Rizzi, p.c.).}
This may be due to us erroneously associating the lowest position for fronted
arguments in the C-domain with SpecFocP\@. The crucial proposal that I am making
is that infinitival\is{relative clauses!infinitival} \emph{wh}{-\glspl{RC} have only a single} position for fronted
arguments in their left-periphery. This is targeted by relativisation and hence
blocks all other argument fronting. If it turns out that there is a position
for \is{fronting!of arguments}fronted arguments below FocP (see \citealt[83,
fn.\ 15]{Douglas2016}), what I have
been calling FocP \glspl{RC}\is{relative clauses} would actually be slightly smaller than FocP\@. However,
the essence of the present proposal would remain unaffected.

\subsection{TopP RCs}\label{sec:02.4.4}

I now return to finite \emph{wh}{-\glspl{RC} and} \emph{that}{-RCs, which I have
proposed are TopPs.} This proposal makes several (correct) predictions. First,
if relativisation targets SpecTopP, we predict that there is only a single
position left for argument fronting. Thus, we expect multiple argument fronting
to be permitted in non-RC contexts, but only single argument fronting in RC
contexts. This prediction is borne out. \ili{English} permits multiple fronted
arguments in non-RC contexts always in the order topic--focus
(\citealt{Culicover1991,Haegeman2012}).\footnote{The standard claim is that
multiple topics are not permitted in \ili{English} (\citealt{Haegeman2012} and
references therein), and that multiple foci are not permitted generally
(\citealt{Haegeman2012,Rizzi1997}).}

\ea\label{ex:2.83}
    That book, \emph{to John} Mary gave in 1979.
\z
However, it is extremely difficult if not impossible to have multiple fronted
arguments within \glspl{RC}.

\ea[*]{the year in which, that book, \emph{to John} Mary gave}\label{ex:2.84}
\z
Alternatively, the difficulty with multiple argument fronting in
\glspl{RC}\is{relative clauses} may be due to the \isi{categorial distinctness}
effect, i.e.\ it may simply be too difficult to front two arguments and
relativise an element whilst simultaneously respecting categorial distinctness.
To tease these two options apart, I will consider a second prediction made by
the present analysis.

My analysis predicts that \is{fronting!of arguments}fronted arguments in finite \emph{wh-} and
\emph{that}-\glspl{RC} will target SpecFocP, i.e.\ the \is{fronting!of arguments}fronted argument will be a
\isi{focus} rather than a \isi{topic}. On a hypothetical alternative analysis, multiple
argument fronting is allowed in principle but ruled out by categorial
distinctness. This means that a single \is{fronting!of arguments}fronted argument could be either a \isi{focus}
or a \isi{topic} in principle. To distinguish these two hypotheses, we must thus ask
whether the \is{fronting!of arguments}fronted argument behaves like a \isi{topic} at all. The empirical
situation is difficult, but overall the \is{fronting!of arguments}fronted argument in \glspl{RC}\is{relative clauses} seems to be a
\isi{focus} rather than a \isi{topic}, as will be shown below, thereby supporting our
analysis rather than the hypothetical alternative.

I will apply two of \citegen{Rizzi1997} topic/focus diagnostics. Rizzi shows
that foci exhibit \gls{WCO}\is{weak crossover} whilst topics\is{topic} do not. As the following data show,
the \is{fronting!of arguments}fronted argument always seems to be sensitive to
\gls{WCO} suggesting that it must be a \isi{focus} and cannot be a \isi{topic} (the
judgements may be quite subtle in some cases).

\ea\label{ex:2.85}\judgewidth{?*}
    \ea[?*]{the school to which, John Smith\tss{i} his\tss{i} mother is planning to send}
    \ex[]{the school to which, John Smith Mary is planning to send}
    \z
\ex\label{ex:2.86}
    \ea[?]{the person to whom, this book\tss{i} its\tss{i} author is happy to give for free}
    \ex[]{the person to whom, this book Mary is happy to give for free}
    \z
\z
As a second diagnostic, \textcite{Rizzi1997} notes that topics can be resumed by
resumptive pronouns, but foci cannot (at least in Italian). Although \ili{English}
does not typically make use of resumptive pronouns (unless with hanging topics
or to repair certain island violations\is{islands}), it seems that the \is{fronting!of arguments}fronted argument is
not very readily resumed by a resumptive pronoun. In fact, it seems more
acceptable to resume the \gls{RC}\is{relative clauses} head (or relative pronoun) than the fronted
argument (recall \cref{fn:02.5}). This suggests that the \is{fronting!of arguments}fronted argument must be a
focus and cannot be a \isi{topic}. Consider the following contrasts:

\ea\label{ex:2.87}\judgewidth{?*}
    \ea[?*]{a man to whom, \textit{unfettered liberty} we would never grant \textit{it}}
    \ex[?]{a man \textit{to whom}, unfettered liberty we would never grant \textit{to him}}
    \z
\ex\label{ex:2.88}
    \ea[?*]{a man to whom, \textit{this book} Mary would happily give \textit{it}}
    \ex[?]{a man \textit{to whom}, this book Mary would happily give \textit{to him}}
    \z
\z
Although none of these considerations are conclusive in isolation, they
nevertheless both seem to converge on the conclusion that argument fronting in
English \glspl{RC}\is{relative clauses} is always focalisation\is{focalization} and never topicalisation\is{topicalization}. This in turn
suggests that the ban on multiple argument fronting in \glspl{RC}\is{relative clauses} in \ili{English}, as
in~\eqref{ex:2.84}, is due to the idea that SpecTopP is targeted by
relativisation and so cannot be targeted by topicalisation\is{topicalization} as well. This thus
suggests that relativisation and topicalisation\is{topicalization} compete for the same position,
i.e.\ SpecTopP\@.  This formally captures the long-standing intuition that
relativisation and topicalisation\is{topicalization} are intimately related (see
\citealt{Abels2012a,Bianchi1999,Kuno1973,Kuno1976,Williams2011}) and could in
fact suggest that topicalisation\is{topicalization} feeds relativisation in \ili{English} and other
languages (see \citealt{Douglas2016} for discussion of \ili{English} and
\ili{Malagasy} in this respect).

The third prediction made by our analysis concerns the categorial distinctness
effect. As seen above, this effect holds between the \is{fronting!of arguments}fronted argument and the
relative pronoun/operator, i.e.\ between the constituents in SpecFocP and
SpecTopP\@. If this is correct, we might also expect to find the categorial
distinctness effect between foci and topics more generally. This is indeed what
we find.

\ea\label{ex:2.89}
    \ea[]{This present, \emph{to Mary} I would give.}
    \ex[*]{This present, \emph{Mary} I would give to.}
    \z
\ex\label{ex:2.90}
    \ea[]{To Mary, \emph{this present} I would give.}
    \ex[*]{Mary, \emph{this present} I would give to.}
    \z
\z
(\ref{ex:2.89}) shows that, if the \isi{topic} phrase is a DP, the
\isi{focus} phrase cannot be a DP, as in (\ref{ex:2.89}b), and must be a
PP, as in (\ref{ex:2.89}a). \eqref{ex:2.90} shows that, if the
\isi{focus} phrase is a DP, the \isi{topic} phrase cannot be a DP, as in
(\ref{ex:2.90}b), and must be a PP, as in (\ref{ex:2.90}a). As far as I
am aware, this is a novel observation and lends independent and important
support to our proposal.

Finally, our analysis is able to incorporate \citegen{Richards2010} idea of why
the relative pronoun in infinitival\is{relative clauses!infinitival}
\emph{wh}-\glspl{RC} obligatorily \is{pied-piping}pied-pipes a preposition in
\ili{English}.

\ea\label{ex:2.91} Infinitival \emph{wh}-\glspl{RC}
    \ea[*]{the man \textit{whom} to talk to}
    \ex[]{the man \textit{to whom} to talk}
    \z
\z
\textcite{Richards2010} proposes that this is due to a \isi{categorial
distinctness} effect between the \emph{wh}{-relative pronoun and the external
determiner of the RC head.} \textcite[35]{Richards2010} provides the following
schematic structures:

\ea\label{ex:2.92} Infinitival \emph{wh}-\glspl{RC}
    \ea[*]{[\tss{DP} \textit{D} [\tss{NP} N=RC head [\tss{CP} [\textit{\tss{DP}} \textit{\emph{wh}-relative pronoun} ] [\tss{C’} C [\tss{TP} \dots{}]]]]]}
    \ex[\hphantom{*}]{[\tss{DP} \textit{D} [\tss{NP} N=RC head [\tss{CP}
        [\textit{\tss{PP}} \textit{P} [\textit{\tss{DP}}
    \textit{\emph{wh}-relative pronoun} ]] [\tss{C’} C [\tss{TP} \dots]]]]]}
    \z
\z
According to Richards, D and N are not phase heads.\is{phases} Consequently, the DP
relative pronoun and the external determiner D in (\ref{ex:2.92}a) are
linearised in the same spellout domain. This yields the linearisation statement
\tuple{D,D} (amongst others). However, because the two D’s are non-distinct,
\tuple{D,D} is uninterpretable at the interfaces by hypothesis. This is the
categorial distinctness effect and accounts for the ungrammaticality of
(\ref{ex:2.91}a). In (\ref{ex:2.92}b), however, the DP relative pronoun
is embedded in a PP (where P is a phase head).  Consequently, the external
determiner D and the DP relative pronoun\is{relative pronouns} are linearised in separate spellout
domains so the problematic \tuple{D,D} statement never arises and
(\ref{ex:2.91}b) is grammatical.

\textcite{Richards2010} highlights that his structures in \eqref{ex:2.92}
simply serve to illustrate his proposal; they are not integral to it.
Consequently, I adapt the structures in \eqref{ex:2.92} to those in
\eqref{ex:2.93} to be more consistent with our conclusions and assumptions.

\ea\label{ex:2.93} Infinitival \emph{wh}-\glspl{RC}
    \ea[*]{[\tss{DP} D [\tss{TopP} \textit{[\tss{DP} RC head]} Top [\tss{FocP} [\textit{\tss{DP}} \textit{\emph{wh}}\textit{{-relative pronoun ]}} Foc [\tss{FinP} Fin [\tss{TP} \dots{}]]]]]}
    \ex[\hphantom{*}]{[\tss{DP} D [\tss{TopP} \textit{[\tss{DP} RC head]} Top
        [\tss{FocP} \textit{[\tss{PP} P [\tss{DP} \emph{wh}-relative pronoun ]]} Foc [\tss{FinP} Fin [\tss{TP}
\dots{}]]]]]}
    \z
\z
Following \textcite{Borsley1997} and \textcite{Bianchi2000}, I analyse the
\gls{RC}\is{relative clauses} head as a DP phrase (rather than as an N head, as
in \ref{ex:2.92}). In this way, the categorial distinctness effect arises
because the DP relative pronoun\is{relative pronouns} and the DP \gls{RC}\is{relative clauses} head
are linearised in the same spellout domain, i.e.\ the categorial distinctness
effect is a relation between two phrases rather than between two heads, as in
\eqref{ex:2.92}.

Now, recall that I argued independently on the basis of the distribution of
adverbial and argument fronting\is{fronting!of
arguments}\is{fronting!adverbial} that infinitival\is{relative clauses!infinitival} \emph{wh}-\glspl{RC} are FocPs. In
\eqref{ex:2.93}, I have shown the \gls{RC}\is{relative clauses} head as being in SpecTopP\@. This
can be interpreted under the raising analysis\is{relative clauses!raising analysis} of \glspl{RC}\is{relative clauses} (see especially
\citealt{Bianchi1999,Bianchi2000}) if one assumes that the \gls{RC}\is{relative clauses} head is
subextracted out of the relative pronoun\is{relative pronouns} DP, or under the matching analysis if
one assumes that the \gls{RC}\is{relative clauses} head can be base-generated in SpecTopP (see
\citealt{Douglas2016} for discussion). What is interesting for present purposes
is that, once again, the \isi{categorial distinctness} effect holds between the
constituents in SpecFocP and SpecTopP\@.  According to \citegen{Richards2010}
account, this would mean that Top is not a phase head.\is{phases} If it were, the
constituent in SpecTopP and the one in SpecFocP would be in different spellout
domains and we would not expect any \isi{categorial distinctness} effect, contrary to
fact.\footnote{Note that, if this is correct, it would suggest that the
    C-domain is not a dynamic phase domain (in the sense of
    \citealt{Boskovic2014}; \citealt{Harwood2015}), i.e.\ it cannot be the case that the highest head
    in the C-domain (whatever it may be) is necessarily phasal (in fact,
    \citealt{Boskovic2014} explicitly leaves the C-domain out of his discussion of
    dynamic \isi{phases}). If it were, we would expect the Top head in infinitival
\emph{wh}-\glspl{RC} to be a phase head.}

Why does the \gls{RC}\is{relative clauses} head in finite \emph{wh}-\glspl{RC} not exhibit
categorial distinctness effects with the relative pronoun?

\ea\label{ex:2.94} Finite \emph{wh}-\glspl{RC}
    \ea     the man \textit{whom} I should speak to
    \ex    the man \textit{to whom} I should speak
    \z
\z
The answer that our analysis provides is that the relative pronoun\is{relative pronouns} is located
in SpecTopP in such cases and the \gls{RC}\is{relative clauses} head is higher, i.e.\ in SpecForceP, as
schematised in \eqref{ex:2.95}.

\ea\label{ex:2.95} Finite \emph{wh}-\glspl{RC}
    \ea {}[\tss{DP} D [\tss{ForceP} \textit{[\tss{DP} RC head]} Force [\tss{TopP} [\textit{\tss{DP} \emph{wh}-relative pronoun}] Top [\tss{FocP} Foc [\tss{FinP} Fin [\tss{TP} \dots{}]]]]]
    \ex {}[\tss{DP} D [\tss{ForceP} \textit{[\tss{DP} RC head]} Force
    [\tss{TopP} \textit{[\tss{PP} P [\tss{DP} \emph{wh}-relative pronoun]]} Top [\tss{FocP} Foc [\tss{FinP} Fin [\tss{TP}  \dots{}]]]]]
    \z
\z
In other words, whilst there is a \isi{categorial distinctness} effect between
constituents in SpecFocP and SpecTopP, there is no such effect between
constituents in SpecTopP and SpecForceP\@. Again, on \citegen{Richards2010}
account, this would suggest that Force \emph{is} a phase head.\is{phases} As a
result, the constituents in SpecForceP and SpecTopP would be in different
spellout domains and no \isi{categorial distinctness} effect would arise
between them, i.e.\ if the constituent in SpecForceP is a DP, the constituent
in SpecTopP can be either a PP, as in (\ref{ex:2.95}b), or a DP, as in
(\ref{ex:2.95}a).

I have thus argued that infinitival\is{relative clauses!infinitival} \emph{wh}{-\glspl{RC} are FocPs with the \gls{RC}\is{relative clauses} head
being located in SpecTopP, and the finite} \emph{wh}{-\glspl{RC} are TopPs with the RC
head being located in SpecForceP\@.} This analysis allows us to give a uniform
analysis of the \isi{categorial distinctness} effects in the three contexts
identified above: (i) between topics and foci in non-RC contexts, (ii) between
relative pronouns and fronted foci in finite \emph{wh}-\glspl{RC}, and (iii)
between the \gls{RC}\is{relative clauses} head and relative pronouns in
infinitival\is{relative clauses!infinitival} \emph{wh}{-RCs.} This brings our
proposal very close to the configurations proposed by
\textcite{Bianchi1999,Bianchi2000,Bianchi2004}. However, whilst Bianchi
proposes that the \gls{RC}\is{relative clauses} head moves into SpecTopP or
SpecForceP, i.e.\ a head raising analysis\is{relative clauses!raising analysis}
of RCs, I believe that there are various reasons for adopting a matching
analysis of \glspl{RC}\is{relative clauses} instead whereby the RC head is
base-generated in SpecTopP or SpecForceP rather than moving into these
positions (see \citealt[Ch.\ 2]{Douglas2016} for details and discussion).
Although it might be unorthodox to posit that the \gls{RC}\is{relative clauses}
head in a matching analysis is base-generated in a high C-domain position,
\textcite{Chierchia2016} has recently proposed that the crucial property of
A-positions is that they are positions that introduce discourse markers.  This
applies to theta-positions and the \glsunset{EPP}\gls{EPP}-subject
position,\is{extended projection principle} but also to certain discourse-based
positions such as topic positions. This potentially provides a rationale for
why the \gls{RC} head may be base-generated in SpecTopP (self-evidently a
\isi{topic} position).  Whether it can be extended to SpecForceP is a matter I
leave for future research.\is{EPP|see{extended projection principle}}

I have proposed that the restrictions on argument fronting found in finite
\emph{wh}{-}RCs manifest the \isi{categorial distinctness} effect found more
generally between the constituents in SpecTopP and SpecFocP in \ili{English}. Recall
that the \isi{categorial distinctness} effect I have been considering effectively
restricts the distribution of fronted DP arguments, i.e.\ I have said that two
fronted arguments cannot both be DPs. What about PPs? If the effect is really
one of categorial distinctness, we would predict that two \is{fronting!of arguments}fronted arguments
cannot both be PPs either. However, PPs do not seem to be sensitive to the
categorial distinctness effect. Recall \eqref{ex:2.44}, repeated as \eqref{ex:2.96} below:

\ea\label{ex:2.96}
    I met a man \textit{with whom}, \textit{about linguistics}, I could talk all day.
\z
In \eqref{ex:2.96}, the relative pronoun\is{relative pronouns} and \emph{linguistics} have both \is{pied-piping}pied-piped a
preposition resulting in two fronted PPs in the C-domain. \textcite{Totsuka2014}
concludes on the basis of such examples that there is \emph{no} categorial
distinctness effect between the relative pronoun\is{relative pronouns} and the \is{fronting!of arguments}fronted argument,
contrary to what I have demonstrated for DPs (Totsuka does not discuss the data
I have been concerned with though). However, there is a serious question about
whether \emph{about linguistics} is an argument PP as opposed to a fronted
adverbial PP (see \citealt[294, 322--325]{Rizzi1997}). Although it is difficult
to front a lot of material simultaneously in \ili{English}, it at least seems
marginally possible to front the \gls{RC}\is{relative clauses} subject in an example like \eqref{ex:2.97}.

\ea[?]{I met a man with whom, \emph{Mary}, about linguistics, could talk all day.}\label{ex:2.97}
\z
Crucially, both the focussed\is{focus} subject DP and the PP \emph{about linguistics} can
co-occur suggesting that they are not competing for the same position (by
hypothesis, SpecFocP). This suggests that the PP \emph{about linguistics} is
lower than FocP, plausibly in SpecModP. In fact, given the difficulty of
finding multiple PP arguments with any single predicate in \ili{English}, it may be
that the fronted \enquote{argument} PP in all examples like \eqref{ex:2.96} is in fact a
fronted adverbial PP\@.\is{fronting!adverbial}\is{fronting!PP fronting}

Finally, I return to the issue of \isi{categorial distinctness} effects in finite
\emph{that}{-RCs,} illustrated in \eqref{ex:2.53} and \eqref{ex:2.54},
repeated below.

\ea\label{ex:2.98}
    \ea    I bought a dress that, \textit{to Mary}, I might consider giving (as a present).
    \ex    I bought a dress that, \textit{to Mary}, could be given (as a present).
    \ex    I bought a car that, \textit{to children}, would give hours of entertainment.
    \z
\ex\label{ex:2.99}
    \ea[*]{I bought a dress that, \textit{Mary}, I might consider giving to (as a present).}
    \ex[*]{I bought a dress that, \textit{Mary}, could be given to (as a present).}
    \ex[*]{I bought a car that, \textit{children}, would give hours of entertainment to.}
    \z
\z
This pattern can be straightforwardly assimilated to the pattern in finite
\emph{wh}{-\glspl{RC} if} \emph{that} is analysed as a relative
pronoun\is{relative pronouns} rather than a complementiser, except that unlike
the \emph{wh}{-relative pronouns it cannot pied-pipe\is{pied-piping} a
preposition} (see, e.g.,~\citealt{Kayne2014}). However, there are dialects of
\ili{English} where both a relative pronouns and \emph{that} can co-occur (see
\citealt[6]{Trotta2004}) suggesting that \emph{that} is not a relative pronoun
and is in fact a complementiser.

If \emph{that} is a complementiser, we can hypothesise that the fronted
argument is interacting with the null relative operator in finite
\emph{that}-RCs, which (for whatever reason) is always a DP, never a PP\@. This
is potentially problematic for \citegen{Richards2010} approach to categorial
distinctness, according to which \isi{categorial distinctness} effects arise when
linearisation statements involve two non-distinct categories. If one of those
elements does not require linearisation, e.g.\ if it is unpronounced, Richards
suggests that there will be no distinctness effect. For example, Richards
proposes that traces (or the unpronounced copies in a \isi{movement} chain) do not
count for linearisation because the system can tell \emph{pre-linearisation}
that such elements will be null. If we wish to maintain Richards’s system for
finite \emph{that}{-RCs, the system must not be able to tell that the relative
    operator in finite} \emph{that}{-\glspl{RC} is going to be null until after the
linearisation statements have been calculated. The raising analysis\is{relative clauses!raising analysis} would have
trouble accounting for this under Richards’s system} since the relative operator
is a trace/copy, whilst the matching analysis could capture this if the
relative operator becomes null post-syntactically (see \citealt{Douglas2016}
for further discussion of the raising and matching analyses).

\section{Conclusion}\label{sec:02.5}

I have reached the conclusion that the different types of clausal \glspl{RC}\is{relative clauses} in
English systematically differ in structural size. This accounts for the various
fronting possibilities. Finite \emph{wh}- and \emph{that}-\glspl{RC} are the largest:
they can host fronted adverbials and fronted focussed\is{focus} arguments.  Infinitival
\emph{wh}{-\glspl{RC} are the next largest: they can host fronted adverbials but not
fronted arguments. Finite} $\varnothing${-, infinitival} \emph{for}{- and
infinitival} $\varnothing${-\glspl{RC} are the smallest: they do} not permit fronting of
any kind. This is summarised in \Cref{tab:4}.

\begin{table}
\small
\begin{tabularx}{\textwidth}{QCCCCCC}
\lsptoprule
&  Finite \emph{wh}-\glspl{RC} &  Finite
    \emph{that}-\glspl{RC} &  Infinitival \emph{wh}-\glspl{RC} &
     Finite \newline$\varnothing$-\glspl{RC} &  Infinitival \emph{for}-\glspl{RC} &  Infinitival $\varnothing$-RCs\\
\midrule
Adverbial fronting & ✔ & ✔ & ✔ & * & * & *\\
Argument fronting & (✔) & (✔) & * & * & * & *\\
Structural size &
\multicolumn{2}{c}{$\overbrace{\hphantom{1.5cm}\text{TopP}\hphantom{1.5cm}}$} & $\overbrace{\text{FocP}}$ & \multicolumn{2}{c}{$\overbrace{\text{FinP (or unsplit CP)}}$} & $\overbrace{\text{FinP/TP}}$\\
\lspbottomrule
\end{tabularx}
\caption{Summary of \gls{RC}\is{relative clauses} structures}\label{tab:4}
\end{table}

I argued that argument fronting in finite \emph{wh}- and \emph{that}-\glspl{RC}
is focalisation\is{focalization}, not topicalisation\is{topicalization}. I suggested that topicalisation\is{topicalization} in these
RCs is ruled out because relativisation and topicalisation\is{topicalization} compete for the same
structural position. Similarly, I suggested that focalisation\is{focalization} in infinitival
\emph{wh}-\glspl{RC} is ruled out because focalisation\is{focalization} and relativisation compete for
the same structural position. I thus concluded that finite \emph{wh}- and
\emph{that}-\glspl{RC} are TopPs, whilst infinitival\is{relative clauses!infinitival} \emph{wh}-\glspl{RC} are FocPs. I also
proposed that the other types of \gls{RC}\is{relative clauses} are FinPs (or unsplit CPs), i.e.\
they have a C-domain with a single C head, or, in the case of infinitival
$\varnothing${-R}Cs, perhaps no C-domain at all.

I also observed that \ili{English} exhibits a \isi{categorial distinctness} effect in the
C-domain in (at least) three environments: (i) between the relative
pronoun\slash operator and fronted (focussed)\is{focus} argument in finite \emph{wh}- and
\emph{that}-RCs; (ii) between \isi{topic} and \isi{focus} in non-RC contexts; and (iii)
between the \gls{RC}\is{relative clauses} head and relative pronoun\slash operator in infinitival
\emph{wh}-\glspl{RC} (following \citealt{Richards2010}). I proposed that these are three
instances of a single effect, namely the \isi{categorial distinctness} effect between
topic and \isi{focus} in \ili{English}, and that relativisation and topicalisation\is{topicalization} are
formally similar (at least in finite RCs).

\label{sec:abbreviations}

\printchapterglossary{}

\section*{Acknowledgements}

I wish to thank Ian Roberts, Theresa Biberauer, Luigi Rizzi, Guglielmo Cinque,
David Willis, and the audience of SynCart 1 (Chiusi, 2016) for helpful feedback
and comments on this work. This paper is adapted from Chapter 3 of my thesis,
which was supervised by Ian Roberts at the University of Cambridge. I
gratefully acknowledge the AHRC grant no. 04271 and the ERC grant no. 269752
(\emph{Rethinking comparative syntax}).

{\sloppy\printbibliography[heading=subbibliography,notkeyword=this]}

\end{document}
