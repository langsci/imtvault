\documentclass[output=paper]{langscibook} 
\ChapterDOI{10.5281/zenodo.5524290}
\author{Artemis Alexiadou\affiliation{Humboldt-Universität zu Berlin} and Elena Anagnostopoulou\affiliation{University of Crete}}
\title{Greek aspectual verbs and the causative alternation}
\abstract{In this paper we examine a particular type of causative construction in Greek, 
    built on the basis of the verb \textit{matheno} ‘learn’ and aspectual verbs like \textit{arhizo} ‘start’. 
    Focusing on the latter and building on \citet{Amberber1996} and \citet{Anagnostopoulou2001}, 
    we will analyze causative constructions involving aspectual verbs as a sub-case of the (anti-)causative alternation. 
    We will further propose to correlate this with the fact that aspectual verbs in Greek have been shown 
    to be ambiguous between control and raising interpretations, following \citet{MourounasWilliamson2019}. 
    Finally, we speculate that the cross-linguistic variation between Greek and English can be attributed 
    to the cross-linguistic availability of the conative alternation.}

\begin{document}
\maketitle 

\section{Introduction}

In this paper, we investigate a certain type of causative construction in Greek, recently discussed in \citet{AnagnostopoulouSevdali2020}. 
These are built on the basis of the verb \textit{matheno} `learn' and aspectual verbs like \textit{arhizo} `start', 
\textit{ksekinao} `start' and \textit{sinehizo} `continue' and are illustrated in \REF{alexiadouex:key:1} and \REF{alexiadouex:key:2}:

\ea
    \label{alexiadouex:key:1}
    \ea[]{
        \label{alexiadouex:key:1a}
        \gll  I Maria  emathe    dhisko/dhiskovolia.\\
        The Mary.\textsc{nom}  learned  discus.\textsc{acc}\\
        \glt `Mary learned discus.'
    }
    \ex[]{
        \label{alexiadouex:key:1b}
        \gll O proponitis  emathe   tis Marias / tin Maria  dhisko/dhiskovolia.\\
        The trainer.\textsc{nom} learned the Mary.\textsc{gen} {} the Mary.\textsc{acc} discus.\textsc{acc}\\
        \glt `The trainer taught Mary discus.'
    }
    \z
\ex
    \label{alexiadouex:key:2}
    \ea[]{
        \label{alexiadouex:key:2a}
        \gll I Maria    arhise    Aglika.\\
        The Mary.\textsc{nom}  started    English.\textsc{acc}\\
        \glt `Mary started (to learn) English.'
    }
    \ex[]{
        \label{alexiadouex:key:2b}
        \gll Tha  \{ tis / tin \} arhiso  \{ tis Marias / tin Maria \} Aglika.\\
        \textsc{fut} {} \textsc{cl.gen} {} \textsc{cl.acc} {} start.\textsc{1sg} {} the Mary.\textsc{gen} {} the Mary.\textsc{acc} {} English\\
        \glt `I will make Mary start (to learn) English.'
    }
    \z
\z 

In this paper, we will focus on \REF{alexiadouex:key:2}, the examples with aspectual verbs. 
Building on insights in \citet{Amberber1996} and \citet{Anagnostopoulou2001} on ingestive predicates, e.g. \textit{eat} but also \textit{learn}, 
we will analyze the alternation in \REF{alexiadouex:key:2} as a sub-case of the (anti-)causative alternation, cf. \citet{Levin1993}, and \citet{MourounasWilliamson2019}. 
This will straightforwardly explain why such examples encode causative semantics. 
Specifically, we will consider \REF{alexiadouex:key:2a} a dyadic anticausative predicate, and \REF{alexiadouex:key:2b} the causative variant thereof. 
According to \citet{Amberber1996} and \citet{Anagnostopoulou2001}, the unexpected behavior of ingestive verbs has to do with the fact 
that the goal argument is interpreted as an Agent, when no external argument is present. 
We will show that \REF{alexiadouex:key:2} is an ingestive structure and thus subject to the same principle. 
In \REF{alexiadouex:key:2a} the goal argument is interpreted as an Agent, as there is no external argument present. 
This is not the case in \REF{alexiadouex:key:2b}, where the external argument is present. 
While \REF{alexiadouex:key:2a} is transitive on the surface, it does not behave like a typical transitive verb, as it cannot undergo passivization. 
On this view, \REF{alexiadouex:key:2b} is a causative construction in which the subject is the cause of the initial sub-event of a Mary learning English event, 
and \REF{alexiadouex:key:2a} is its anticausative variant. 
Following \citet{MourounasWilliamson2019}, we will propose to correlate this with the fact that aspectual verbs in Greek have been shown 
to be ambiguous between control and raising interpretations 
\citep{alexiadouanagnostopoulou1999, Roussou2009, AlexiadouAnagnostopoulouIordachioaiaMarchis2010, AlexiadouAnagnostopoulouIordachioaiaMarchis2012, AlexiadouAnagnostopoulouWurmbrand2014}, 
just as in English. 
In previous work, we argued that aspectual verbs in Greek form restructuring-type biclausal domains via a Long Distance Agree chain 
between the matrix and the embedded, semantically null, T with fully specified $\varphi$-features. 
This forces coindexation between the matrix and the embedded subject and Obligatory Backward or Forward Control and Raising/Long Distance Agreement phenomena. 
Following \citet{MourounasWilliamson2019}, we will propose that aspectual verbs have a single lexical entry for both subjunctive and nominal complements. 
\REF{alexiadouex:key:2b} is in fact similar to \citegen{Grano2016} example \textit{John started Bill smoking} and provides evidence against the claim 
that aspectual verbs do not permit overt subjects in their non-finite complements (cf. \citegen{Grano2016} \textit{overt embedded subjects} generalization). 
As we take the examples in \REF{alexiadouex:key:2} to involve ingestive predicates, we will conclude that the cross-linguistic variation between Greek and English 
can be attributed to the cross-linguistic availability of the conative alternation. 
English allows the counterpart of \REF{alexiadouex:key:2b} if the theme argument is introduced via a PP; 
Greek does not have a systematic conative alternation and, therefore, it does not require a PP in constructions comparable to that in \REF{alexiadouex:key:2b}. 

\section{The anticausative alternation with Greek aspectual verbs}

As is well known, in English and in Greek verbs like \textit{break} or \textit{open} undergo the causative alternation:

\ea%3
    \label{alexiadouex:key:3}
    \ea John broke the window.
    \ex The window broke.
    \z 
\ex%4
    \label{alexiadouex:key:4}
    \ea
        \gll O Janis anikse    to parathiro.\\
        The John opened.\textsc{3sg}  the window.\textsc{acc}\\
        \glt ‘John opened the window.’
    
    \ex
        \gll To parathiro    anikse.\\
        The window.\textsc{nom} opened.\textsc{3sg}\\
        \glt ‘The window opened.’
    \z 
\z
One diagnostic to distinguish anticausatives from passives discussed at length in \citet{AlexiadouAnagnostopoulouSchafer2015}, 
building on \citet{LevinRappaportHovav1995}, is the availability of the \textit{by-itself} modifier. 
While anticausatives allow the \textit{by-itself} phrase, passives disallow it. 
\citet{AlexiadouAnagnostopoulouSchafer2015} argue that this relates to the \textit{no particular cause} interpretation 
associated with the \textit{by-itself} phrase in English and its counterparts across languages. 
This is incompatible with the interpretation of the passive, which implies the presence of an external argument. 
By contrast, English passives, but not anticausatives, allow agentive \textit{by}-phrases:

\ea%5
    \label{alexiadouex:key:5}
    \ea The window was broken *by itself/by John.
    \ex The window broke by itself/*by John.
    \z 
\z 
\citet{MourounasWilliamson2019} argue that aspectual verbs undergo the causative alternation in English, 
as they do not tolerate agentive \textit{by} phrases as opposed to the passive variant, see \REF{alexiadouex:key:6}:

\ea%6
    \label{alexiadouex:key:6}
    \ea The official began the London marathon.
    \ex The London marathon began.
    \ex The London marathon was begun by the official.
    \ex The London marathon began (*by the official).
    \z 
\z
Greek aspectual verbs behave similarly. 
They form actively marked anticausatives and can be modified by \textit{by-itself}. 
While \textit{begin} does not have a non-actively marked passive variant, 
the non-actively marked variant of \textit{stop} is marginally acceptable and is interpreted as passive \REF{alexiadouex:key:7c}, 
similarly to non-actively marked intransitive variants of Greek de-adjectival verbs.%
\footnote{As \citet{AlexiadouAnagnostopoulouSchafer2015} and references therein discuss at length, 
    Greek also has several anticausatives which bear Non-Active morphology. 
    In the case of de-adjectival verbs, the authors point out that the anticausative bears active morphology and the further intransitive variant, 
    which bears Non-Active, is interpreted solely as a passive.}
\ea%7
    \judgewidth{\%}
    \label{alexiadouex:key:7}
    \ea[]{
        \gll O astinomikos  stamatise  tin  kikloforia.\\
        The policeman   stopped   the   traffic\\
    }
    \ex[]{
        \gll I kikloforia  stamatise  apo {moni tis}.\\
        The traffic   stopped   by itself\\
    }
    \ex[\%]{\label{alexiadouex:key:7c}
        \gll I kikloforia  stamatithike  apo tus astinomikus.\\
        The traffic  {was stopped}   by the policemen\\
    }    
    \z 
\z 
We argue that the examples in \REF{alexiadouex:key:2}, repeated below, are a further instantiation of the causative alternation, 
the difference being that \REF{alexiadouex:key:2a} is a dyadic anticausative.%
\footnote{An anonymous reviewer asks if all aspectual verbs behave alike. 
    In our judgement, they do, but they differ with respect to the realization of the theme argument. 
    With \textit{stamatao} ‘stop’, \textit{sinexizo} `continue’, the theme argument must be a DP, 
    and it can't be a bare NP, unlike the complement of \textit{arhizo} `start' in \REF{alexiadouex:key:8}.    
    This is an interesting difference which relates to the fact that there is a presupposition associated with these verbs that a particular event has started. 
    Entities that are known both to the speaker and the hearer are DPs in Greek, see also Footnote~\ref{alexiadouftn:key:6}. 
    With \textit{teliono} ‘finish’, the theme is introduced via  the preposition \textit{me} ‘with’.
}

\ea \label{alexiadouex:key:8}
    \ea
        \gll I Maria    arhise    Aglika. \\
        The Mary.\textsc{nom}  started    English.\textsc{acc} \\
        \glt ‘Mary started (to learn) English.’
    \ex
        \gll Tha  \{ tis / tin \} arhiso  \{ tis Marias / tin Maria \} Aglika.\\
        \textsc{fut} {} \textsc{cl.gen} {} \textsc{cl.acc} {} start.\textsc{1sg} {} the Mary.\textsc{gen} {} the Mary-\textsc{acc} {} English\\
        \glt `I will make Mary start (to learn) English.'
    \z 
\z

Support for this comes from the observation that \REF{alexiadouex:key:2a} resists passivization:

\ea[*]{
    \gll Ta Aglika    arhistikan  apo ti Maria.\\
    The English.\textsc{nom} started.\textsc{nact}   by the Mary.\textsc{acc} \\
    \glt ‘English was started by Mary.’
}
\z 

Building on \citet{Anagnostopoulou2001}, in \REF{alexiadouex:key:2b}, the DP argument is interpreted as a goal as there is a higher agent present. 
A characteristic property of \REF{alexiadouex:key:2b} is that the embedded verb is necessarily interpreted as ‘learn’, 
which describes acquisition of information that may be viewed as a type of ingestion. 
The existence of examples where the embedded verb can also be ‘eat’ or ‘drink’ in \REF{alexiadouex:key:9} supports the claim 
that these constructions belong to the broader class of ingestives \citep[213--217]{Levin1993}, 
construed as `taking something into the body or mind (literally or figuratively)' \citep[46]{Masica1976}:

\ea%9
    \label{alexiadouex:key:9}
    \ea
        \gll Tha  \{ tis / tin \} arxiso \{ tis Marias / tin Maria \} fruta.\\
        \textsc{fut} {} \textsc{cl.gen} {} \textsc{cl.acc} {} start.\textsc{1sg} {} the Mary.\textsc{gen} {} the Mary.\textsc{acc} {} fruit\\
        \glt  ‘I will make Mary start (to eat) fruit.’
    \ex
        \gll Tha \{ tis / tin \} arxiso \{ tis Marias / tin Maria \} gala.\\
        \textsc{fut} {} \textsc{cl.gen} {} \textsc{cl.acc} {} start.\textsc{1sg} {} the Mary.\textsc{gen} {} the Mary.\textsc{acc} {} milk\\
        \glt ‘I will make Mary start (to drink) milk.’
    \z 
\z 

Ingestive verbs are known in the literature to display exceptional behavior across languages, 
a fact which has been related to the observation that the person 
that consumes e.g. food, liquids (as in \textit{eat} or \textit{drink}) or knowledge (as in \textit{learn}, \textit{study})
not only controls but is also affected by the consumption event. 
Cross-linguistic evidence suggests that languages treat ingestive verbs differently from ordinary transitive verbs 
(see \citealt{Jerro2019} for a recent summary, cf. \citealt{Amberber1996,Jackendoff1990}). 
In e.g. Amharic these verbs pattern with unaccusatives rather than with transitives with respect to causativization \citep{Amberber1996}. 
This in turn can be related to the fact that in the presence of an external argument the DP is interpreted as a goal, 
while in the absence of an external argument, the DP is interpreted as an agent, as suggested in \citet{Anagnostopoulou2001} for \textit{learn}.%
\footnote{Different implementations of this have been put forth in the literature. \citet{Anagnostopoulou2001} argues that the interpretation of the DP as an agent or a goal depends on the presence of an external argument. 
    \citet{Amberber1996} proposes that in the anticausative structure the Agent and the Goal role are coindexed. 
    \citet{Krejci2012} claims that ingestive verbs are inherent reflexives, an analysis adopted in \citet{Jerro2019}. 
    He argues that the subject of \textit{eat} is associated with various entailments that are split across two arguments in \textit{feed}.}
Because of this, \REF{alexiadouex:key:2a} is in principle compatible with agentive adverbials, 
a fact that we attribute to the particular interpretation associated with ingestive structures, 
despite the fact that this argument is not introduced by Voice, the head canonically introducing agents.%
\footnote{Many thanks to an anonymous reviewer for pointing this out to us.}

With respect to the case patterns exhibited in \REF{alexiadouex:key:2b}, 
\citet{AnagnostopoulouSevdali2020} extensively argue that the optionality in the case of the causee argument is only apparent. 
When the lower direct object is definite, as in examples \REF{alexiadouex:key:10}, only the genitive causee is licit; 
the accusative one is ungrammatical. 

\ea
    \label{alexiadouex:key:10}
    \ea[]{
        \gll Pjos \{ \textsuperscript{ok}tis / *tin \}  emathe \{ \textsuperscript{ok}tis Marias / *tin Maria \} ta Aglika?\\
        Who {} \textsuperscript{ok}\textsc{cl.gen} {} *\textsc{cl.acc} {} learned {} \textsuperscript{ok}the Mary.\textsc{gen} {} *the Mary.\textsc{acc} {} the English?\\
        \glt ‘Who taught Mary the English language?’
    }
    \ex[]{
        \gll Tha \{ \textsuperscript{ok}tis / *tin \} arhiso \{ \textsuperscript{ok}tis Marias / *tin Maria \} ta Aglika.\\
        \textsc{fut} {} \textsuperscript{ok}\textsc{cl.gen} {} *\textsc{cl.acc} {} start.\textsc{1sg} {} \textsuperscript{ok}the Mary.\textsc{gen} {} *the Mary.\textsc{acc} {} the English\\
        \glt ‘I will make Mary start (to learn) English.’
    }
    \z 
\z

The case of the causee argument is thus sensitive to the realization of the lower object: 
when this object is a definite DP, the causee must be genitive.
It is only in the presence of a lower bare NP, as in \REF{alexiadouex:key:1} and  \REF{alexiadouex:key:2} that both cases are possible.%
\footnote{An anonymous reviewer asks if it is the DP vs. NP distinction that is crucial here or the definite/non-definite distinction, 
    as one could think of English as definite (proper name like) even in the absence of a determiner. 
    In Greek, unlike in English, proper names necessarily appear with a determiner. 
    \citet{AlexopoulouFolli2011,AlexopoulouFolli2019} have argued that Greek definite determiners are not expletive when they appear with proper names, 
    but rather have a semantic effect. 
    It brings about an interpretation, according in which the noun is known both  to the speaker and the hearer. 
    The same reviewer asks if the anticausative of \REF{alexiadouex:key:11} is possible in Greek, which it is.\label{alexiadouftn:key:6}}
\citet{AnagnostopoulouSevdali2020} argue at length that the above described case distribution can be naturally accounted for 
if genitive case in Greek is dependent case upward which is assigned in the vP domain in opposition to a lower DP 
while accusative case is dependent case downward assigned in the TP domain in opposition to a higher DP.  
When the lower object is a bare NP it only optionally counts as a case competitor for the assignment of dependent genitive. 
Genitive is assigned when the lower NP counts as a case competitor 
and accusative (dependent case in opposition to the external argument) is assigned when it doesn’t. 
This conclusion is reinforced by the observation that when the lower argument is a PP, which does not count as a case competitor, 
the higher one must bear accusative case and cannot have dependent genitive, as shown in \REF{alexiadouex:key:11}.

\ea%11
    \label{alexiadouex:key:11}
    \gll Pjos \{ *tis / \textsuperscript{ok}tin \}  emath-e \{ *tis Maria-s / \textsuperscript{ok}tin Maria \} s-ta  narkotika?\\
    Who {} *\textsc{cl.gen} {} \textsuperscript{ok}\textsc{cl.acc} {} learn-\textsc{pst.3sg} {} *the Maria-\textsc{gen} {} \textsuperscript{ok}the Maria.\textsc{acc} {} to-the drugs.\textsc{acc}? \\
    \glt ‘Who got Maria addicted to drugs?’
\z 

The final point that we would like to make with respect to aspectual verbs is that 
they can also take subjunctive complements and in this case they have been argued to be ambiguous between control and raising interpretations, 
see \citet{alexiadouanagnostopoulou1999} and \citet{Roussou2009}. 
Unlike English, Greek lacks infinitival complements: 
sentences that correspond to infinitivals in English are introduced by the subjunctive particle \textit{na}. 
Agent-oriented adverbs are possible with aspectual verbs and they necessarily have matrix scope, as shown in \REF{alexiadouex:key:12}. 
Moreover, they form imperatives, as shown in \REF{alexiadouex:key:13}:

\ea%12
    \label{alexiadouex:key:12}
    \ea[]{
        \gll Epitidhes  arhisa  na  magirevo  stis 5.00.\\
        {on purpose} started.\textsc{1sg} \textsc{subj} cook.\textsc{1sg} at 5.00\\
       \glt `I started on purpose tocook at 5:00.' }
    \ex[]{
        \gll Epitidhes  stamatisa  na  perno     ta farmaka.\\
        {on purpose} stopped.\textsc{sg} \textsc{subj}  take.\textsc{1sg} the medicine\\
    	\glt `I stopped on purpose to take medication.'}
    \z 
\ex%13
    \label{alexiadouex:key:13}
    \ea[]{
        \gll Arhise    na  diavazis!\\
        Start.\textsc{2sg}   \textsc{subj} read.\textsc{2sg}\\
        \glt ‘Start reading!’
    }
    \ex[]{
        \gll Stamata  na  kapnizis!\\
        Stop.\textsc{2sg}  \textsc{subj}  smoke.\textsc{2sg}!\\
        \glt ‘Stop smoking!’
    }
    \z 
\z 

On the basis of idiomatic expressions, \citet{alexiadouanagnostopoulou1999} show that aspectual verbs can be raising verbs. 
In Greek, fixed nominatives as part of idiomatic expressions occur in postverbal position. 
        
\ea%14
    \label{alexiadouex:key:14}
    \ea[]{
        \label{alexiadouex:key:14a}
        \gll Mu     bikan    psili    st'aftia.\\
        \textsc{cl.1sg.gen} entered.\textsc{3pl} fleas.\textsc{nom}  {in the ears}\\
        \glt ‘I became suspicious.’
    }
    \ex[*]{
        Psili mu bikan st'aftia.
    }
    \z 
\z

Examples like \REF{alexiadouex:key:14a} can be embedded under \textit{arhizo} and \textit{stamatao}. 
The subject in the embedded clause agrees with the embedded and the matrix verb:

\ea%15
    \label{alexiadouex:key:15}
    \gll Stamatisan / arhisan    na   mu     benun psili    st'aftia.\\
    Stopped.\textsc{3pl} {} started.\textsc{3pl}  \textsc{subj}  \textsc{cl}{}.\textsc{1sg.gen} enter.\textsc{3pl} fleas-\textsc{nom.pl} {in the ears}\\
    \glt  ‘I stopped being/started becoming suspicious.’
\z

In \REF{alexiadouex:key:15} the nominative depends on the lower verb for its interpretation and yet it agrees with both verbs obligatorily. 
Lack of agreement, leads to ungrammaticality, as shown in \REF{alexiadouex:key:16}: 

\ea%16
    \label{alexiadouex:key:16}
    \gll *Stamatise / arhise    na     mu     benun psili    st'aftia.\\
    Stopped-\textsc{3sg} {} started-3\textsc{3sg}  \textsc{subj}    \textsc{cl.1sg.gen}   enter-\textsc{3pl} fleas-\textsc{nom}  {in the ears}   \\
    \glt ‘I stopped being/started becoming suspicious.’
\z

\citet{alexiadouanagnostopoulou1999} point out that the fact that agreement between the subject and the matrix verb is obligatory, 
is an argument that these constructions display Agree without movement. 
They conclude that aspectual verbs are ambiguous between a control and a raising interpretation, see also \citet{Roussou2009}.%
\footnote{
    An anonymous reviewer points out that the behavior of \textit{arhizo} that we describe here 
    is reminiscent of other embedding verbs that have been argued to alternate between a causative and a non-causative meaning, 
    depending on whether the embedded verb is controlled or not, e.g. \textit{prospatho} ‘try’. 
    An attempt to relate the behavior of \textit{prospatho} to our alternation here would bring us too far afield.
}

\section{Towards an analysis}

Following \citet{MourounasWilliamson2019}, we propose that there is a single lexical entry associated with both subjunctive and nominal complements of aspectual verbs. 
Adopting the analysis proposed in \citet{AlexiadouAnagnostopoulouSchafer2015}, 
we assign the structures in \REF{alexiadouex:key:17} to anticausative and causative variants of aspectual verbs in Greek. 
Greek sentences like \REF{alexiadouex:key:2a} have an anticausative analysis, \REF{alexiadouex:key:17a}. 
The subject DP originates in the ResultP, which can be seen as a small clause consisting of the subject and a DP which has a coerced event interpretation 
(‘English’ understood as ‘learn English’). 
The subject of the small clause undergoes ‘raising’ entering Agree with T. 
On the other hand, \REF{alexiadouex:key:17b} is the causative counterpart which projects a Voice above the v+Root combination introducing an external argument. 
The subject DP in \REF{alexiadouex:key:17b} enters Agree with T and ‘Mary’ receives either dependent genitive or dependent accusative 
depending on the nature of the lower DP (NP or DP or PP). 

\ea%17
    \label{alexiadouex:key:17}
    \ea
        \label{alexiadouex:key:17a}
        \textit{anticausative} \textit{begin}:
        Greek \textit{Mary started English} 
        (comparable  to ‘Mary started the journey’, ‘Mary started smoking’ in  English) \\
        \begin{forest}
            [vP 
                [\phantom{xxxx}]
                [v'
                    [v-Root]
                    [Result
                        [{Mary English},roof ]
                    ]
                ]
            ]
        \end{forest}
    \ex 
        \label{alexiadouex:key:17b}
        \textit{causative} \textit{begin}:
        Greek \textit{I started Mary English}
        (comparable to `I started John smoking' in English) \\
        \begin{forest}
            [VoiceP
                [DP]
                [vP 
                    [v-Root]
                    [Result
                        [{Mary English}, roof ]
                    ]
                ]
            ]
        \end{forest}
    \z 
\z

Building on \citet{MourounasWilliamson2019}, we correlate the anticausative structure of aspectual verbs with the raising interpretation, 
while the causative structure with the control interpretation, as in \REF{alexiadouex:key:18}:

\ea%18
    \label{alexiadouex:key:18}
    \ea 
        \label{alexiadouex:key:18a}
        \textit{anticausative} \textit{begin}, 
        \textit{TP} \textit{compl.} \textit{raising} \\
        \ob
           T$\varphi_k$ 
           \ob\textsubscript{vP}
               \ob\textsubscript{RootP} 
                   start/ stop 
                   \ob\textsubscript{MoodP}
                       na
                       \ob\textsubscript{TP}
                           T$\varphi_k$
                           DP$\varphi_k$  
                       \cb
                   \cb
               \cb
            \cb
         \cb        
    \ex 
        \label{alexiadouex:key:18b}
        \textit{causative} \textit{begin}, 
        \textit{TP} \textit{compl,} \textit{control} \\
        \ob
            T
            \ob\textsubscript{VoiceP} 
                DP 
                \ob\textsubscript{vP}
                    \ob\textsubscript{RootP} 
                        start/ stop 
                        \ob\textsubscript{MoodP} 
                            na
                            \ob\textsubscript{TP}
                                \ob\textsubscript{VoiceP} 
                                    PRO 
                                \cb
                            \cb
                        \cb  \cb \cb \cb \cb
    \z 
\z 

In \REF{alexiadouex:key:18a}, the raising structure, no Voice is projected above matrix VP (the Root + v combination) 
and the embedded subject undergoes Raising or enters Long Distance Agreement with the matrix T. 
On the other hand, Voice is present above the matrix DP introducing a matrix subject which enters an obligatory control relation with a null PRO embedded subject. 
\citet{MourounasWilliamson2019}, building on \citet{wurmbrand2001,Wurmbrand2002,Wurmbrand2014}, assume that in languages with infinitives like English, complements of aspectual verbs are vPs which lack a TP component. 
This is not the case in Greek which provides evidence for the presence of a semantically empty T head 
and a Mood head occupied by the subjunctive particle \textit{na}, see \citet{AlexiadouAnagnostopoulouToapp}. 

In the above sketched system, the control analysis of aspectuals is captured by the presence of VoiceP in the matrix clause. 
By contrast, the raising analysis is captured by the fact that these verbs undergo the causative alternation and their intransitive variants lack Voice. 
This naturally provides an explanation for the causative interpretation associated with aspectual verbs observed in \REF{alexiadouex:key:2b} 
and for the alternation between \REF{alexiadouex:key:2a} and \REF{alexiadouex:key:2b} 
which originates in the presence of an external argument in the causative construction \REF{alexiadouex:key:2b} 
and its absence in the (anti-)causative \REF{alexiadouex:key:2a}. 

Before closing this squib, we briefly address two questions. 
First, why is it that aspectuals in Greek may license ECM with small clauses of the type illustrated in \REF{alexiadouex:key:2b} 
but not with full clausal complements \REF{alexiadouex:key:19b}, and why is it that \REF{alexiadouex:key:19a} is grammatical but \REF{alexiadouex:key:19b} is not?

\ea%19
    \label{alexiadouex:key:19}
    \ea[]{
        \label{alexiadouex:key:19a}
        \gll I Maria    arhise    na  matheni Aglika.\\
        The Mary.\textsc{nom}  started-\textsc{3sg}  \textsc{subj} learn.\textsc{3sg} English\\
        \glt ‘Mary started to learn English.’
    }
    \ex[*]{
        \label{alexiadouex:key:19b}
        \gll Arhisa  tin Maria  na  matheni  Aglika.\\
        started.\textsc{1sg} the Mary.\textsc{acc} \textsc{subj} learn.\textsc{3sg} English\\
    }
    \z
\z 

Second, what explains the fact that constructions like \REF{alexiadouex:key:2a} and \REF{alexiadouex:key:2b} are possible in Greek but not in English?

With respect to the first question, we will follow \citet{Grano2016} and \citet{MourounasWilliamson2019}, 
who propose that the semantics of subject-introducing infinitives are interpretably incompatible with the lexical semantics of aspectual verbs. 
ECM infinitives (whether they are CPs introduced by ‘for’ or TPs) necessarily encode modality 
\citep{kratzer2006,Moulton2009,Grano2016}, 
and they are uninterpretable when combined with non-modal eventualities such as those introduced by aspectual verbs. 
As a result of this, only non-modal properties of eventualities may serve as interpretable restrictors of the event variable introduced by aspectual verbs. 
We will adopt this analysis and will assume that it also applies to ECM subjunctives. 
In the Greek small clause constructions under discussion of the type seen in \REF{alexiadouex:key:2b} as well as in examples like ‘I started John smoking’ in English, 
there is no modal operator blocking embedding under aspectuals, and the relevant constructions are licit. Simiarly, raising infinitives as in \REF{alexiadouex:key:19a} do not encode modality.

With respect to the second question, we note that even in English it is possible to construct \REF{alexiadouex:key:2}, 
however in the transitive variant the DP argument is introduced by \textit{on}, see \citet{Levin1993}:%
\footnote{
    We are grateful to an anonymous reviewer for bringing these examples to our attention.
}

\ea%20
    \label{alexiadouex:key:20}
    \ea 
        \label{alexiadouex:key:20a}
        Mary started English in the third grade.
    \ex 
        \label{alexiadouex:key:20b}
        John started Mary on English.
    \z 
\z

We tentatively propose that \textit{on} is required to license an aspectual interpretation signaling continuation 
and that this should be linked to the conative alternation in English 
which, according to \citet[42]{Levin1993} “expresses an “attempted” action without specifying this action was actually carried out”. 
Usually the PP employed in the intransitive conative variant is headed by \textit{at} 
but, interestingly, sometimes \textit{on} surfaces with certain verbs of ingesting, as pointed out by \citet{Levin1993}:

\ea%21
    \label{alexiadouex:key:21}
    \ea The mouse  nibbled  the cheese. 
    \ex The mouse  nibbled  at/on the cheese. 
    \z 
\z 
    
We would like to speculate that the \textit{on} seen in \REF{alexiadouex:key:20b} is a trace of the conative construction. 
Greek does not have a systematic conative alternation and, therefore, it does not require a PP in constructions comparable to \REF{alexiadouex:key:20b}. 
The issue awaits further research.

\section*{Acknowledgments}

We are indebted to two anonymous reviewers for their insightful comments. 
Many thanks to Susi for friendship and inspiration through the years. 
AL 554/8-1 (Alexiadou) and a Friedrich Wilhelm Bessel Research Award 2013 and HFRI-F17-44 (Anagnostopoulou) are hereby acknowledged.

{\sloppy\printbibliography[heading=subbibliography,notkeyword=this]}

\end{document} 
