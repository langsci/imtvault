\documentclass[output=paper]{langscibook}
\ChapterDOI{10.5281/zenodo.5524284}
\author{Hiroaki Saito\affiliation{Mie University; University of Connecticut}}
\title{Size of sentential complements in Japanese}
\abstract{This chapter investigates sentential complements in Japanese, focusing on those selected by two semantically similar particles, \emph{teki} ‘like, -ish’ and \emph{ppoi} ‘like, -ish’. I show that, despite their apparent similarities, the clausal complements taken by \emph{teki} and \emph{ppoi} behave differently in a number of respects. I argue that their differences are due to a difference in size of their clausal complements; \emph{teki} takes a syntactically larger sentential complement than \emph{ppoi}. I also show that the same contrast as with \emph{teki} and \emph{ppoi} is found with another pair of particles; the evidential markers \emph{mitai} and \emph{yoo}, which indicates that the suggested size difference is not idiosyncratic to the pair of \emph{teki} and \emph{ppoi}. The difference in the size of the sentential complements of the elements in question is argued to provide evidence that syntactic selection is needed independently of semantic selection.}

\begin{document}
\maketitle

\section{Introduction}
In this chapter, I discuss two semantically similar morphemes \emph{teki} ‘like, -ish’ and \emph{ppoi} ‘like, -ish’ in Japanese.  \emph{Teki} and \emph{ppoi} usually take a nominal complement, forming an AP, as shown in \REF{saito1} (e.g. \citealt{KaiserYamamoto2001}). 

\begin{multicols}{2}
\begin{exe}
\ex \label{saito1}
\begin{xlist}
\ex \label{saito1a}
\gll {seizi}\\ 
politics\\
\glt `politics'\\

\ex \label{saito1b}
\gll {seizi}-\{{teki/ppoi}\}\\
politics-\{\textsc{teki/ppoi}\}\\
\glt `political/politics-ish'
\end{xlist}
\end{exe}
\end{multicols}\pagebreak

The distribution of \emph{teki} and  \emph{ppoi} is not limited to the use found in (\ref{saito1}).  \emph{Teki} and  \emph{ppoi} can also be attached to a sentential complement, as illustrated in (\ref{saito2}) (see e.g. \citealt{Yamashita2000}, \citealt{Saito2017} for \emph{teki},  \citealt{Kojima2003}, \citealt{Yamada2014} for  \emph{ppoi}).\footnote{As shown in \REF{saitoi} and \REF{saitoii}, \textit{teki} and \textit{ppoi} can also appear in the sentence-final position, which can be considered as a case where they take a nominal or clausal complement (see (\ref{saito1}) and (\ref{saito2}) in the text). 

\begin{exe}
\ex \label{saitoi}
\gll John-no sigusa-wa [\textsubscript{NP} zyosee]-\{\emph{teki-da/ppoi}\}.\\
John-\textsc{gen} gesture-\textsc{top} {} woman-\{\textsc{teki-cop/ppoi}\}\\ 
\glt `John’s gesture is womanly.' (Lit. ‘John’s gesture is woman-ish/like.’)

\ex \label{saitoii}
\gll {Ano} {karee-wa} [\textsubscript{S} {okaasan-ga} {tuku-tta}]-\{{teki-da/ppoi}\}.\\
that curry-\textsc{top} {} mother-\textsc{nom} cook-\textsc{past}-\{\textsc{teki-cop/ppoi}\}\\ 
\glt ‘That curry is like (the) one the mother cooked.’ (Lit. ‘That curry is [the mother cooked]-ish/like.’)

\end{exe}

Notice, however, that when \emph{teki} and \emph{ppoi} attach to a matrix clause, there is a contrast, as pointed out by an anonymous reviewer: \emph{ppoi}, but not \emph{teki}, can attach to the matrix clause. In this paper, I will focus on their occurrence in prenominal (modifier) clauses, putting aside the contrast found in (\ref{saitoiii}).
\begin{exe}
\ex \label{saitoiii}
\gll [{Okaasan-ga} {karee-o} {tuku-ru}]-\{$^{?*}${teki-da/ppoi}\}.\\
mother-\textsc{nom} curry-\textsc{acc} cook-\textsc{pres}-\{\textsc{teki-cop/ppoi}\}\\ 
\glt ‘It seems that the mother will cook curry.’ (Lit. ‘[The mother will cook curry]-ish/like.’)
\end{exe}
}

\begin{exe}
\ex \label{saito2}
\gll [[\textsubscript{S} {okaasan-ga}   {tuku-ru}]-\{{teki-na/ppoi}\}] {karee}\\
{} mother-\textsc{nom} cook-\textsc{pres}-\{\textsc{teki-cop/ppoi}\} curry\\ 
\glt ‘curry like (the) one the mother cooks’ (Lit. ‘[the mother cooks]-ish/like curry’)
\end{exe}

The selectional property of the morphemes \emph{teki} and \emph{ppoi} is thus similar in that both can take a nominal or a sentential complement. It should also be noted that the present tense marker \emph{ru} appears in the clause selected by \emph{teki/ppoi}, which indicates that their “clausal” complements are really clausal, involving at least a TP-layer.

This chapter investigates syntactic properties of the clausal complements of \emph{teki} and \emph{ppoi}. In the next section, I show that, despite the similarities we have observed in (\ref{saito1}) and (\ref{saito2}), \emph{teki} and \emph{ppoi} show differences regarding their clausal complements. In \sectref{saitos3}, I argue that the contrast found with \emph{teki} and \emph{ppoi} is due to a difference in the size of their clausal complements; I suggest that \emph{teki} selects a larger sentential complement than \emph{ppoi}. In \sectref{saitos4}, I demonstrate that another pair of particles (the evidential markers \emph{mitai} and \emph{yoo}) shows exactly the same contrast as \emph{teki} and \emph{ppoi}, indicating that the suggested size difference in clausal complements is not idiosyncratic to the specific pair of \emph{teki} and \emph{ppoi}. \sectref{saitos5} concludes this chapter.



\section{\textit{Teki} vs. \textit{ppoi}} \label{saitos2}

In this section, I will demonstrate that the clausal complements selected by \emph{teki} and \emph{ppoi} behave differently in a number of respects, suggesting that selectional properties of \emph{teki} and \emph{ppoi} are not identical despite the apparent similarities. Specifically, I will investigate the distribution of imperatives, volitionals, the politeness marker, and nominative-genitive conversion with respect to the clausal complements of \emph{teki} and \emph{ppoi}.

The first difference concerns the imperative marker. Japanese allows embedded imperatives (\citealt{Oshima2006}, \citealt{Schwager2006}). Thus, as shown in (\ref{saito3}), the imperative can appear in the clausal complement of the matrix verb ‘say’. Notice that (\ref{saito3}) involves true embedding, not a direct quote. Under the intended interpretation, the pronoun \emph{kare} ‘he’ refers to the matrix subject John; this interpretation is not available if \emph{kare} is contained in a direct quote (see e.g. \citealt{anand2006}, \citealt{crnictrinh2009}).



\begin{exe}
\ex \label{saito3}
\gll {John$_{\textnormal{i}}$-ga} {Mary-ni} [{kare$_{\textnormal{i}}$-no} {hon-o} {ka-e} {to}] {i-tta}.\\
John-\textsc{nom} Mary-\textsc{dat} he-\textsc{gen} book-\textsc{acc} buy-\textsc{imp} \textsc{c} say-\textsc{past}\\ 
\glt `John$_{\textnormal{i}}$ told Mary to buy his$_{\textnormal{i}}$ book.’     
\end{exe}

What is important for us is that imperatives can appear in the clausal complement of \emph{teki}, as shown in (\ref{saito4}) (see \citealt{Saito2017}).

\begin{exe}
\ex \label{saito4}
\gll [[{asita} {a-e}]-{teki-na}] {hito}\\
tomorrow meet-\textsc{imp}-\textsc{teki}-\textsc{cop} person\\ 
\glt ‘(a) person like (the) one who you should meet tomorrow’
\end{exe}

Note that the clause selected by \emph{teki} is truly embedded; in (\ref{saito5}), the embedded pronoun \emph{kare} ‘he’ refers to John (the matrix subject). This interpretation should be impossible if (\ref{saito5}) involved a direct quote of John’s utterance or thought.\footnote{\emph{Teki} can also introduce a direct quote (\citealt{Saito2017}), and so can \emph{ppoi}, at least for some speakers (see e.g. \citealt{Ohara2010}). In the following, I will focus on clausal complements of \emph{teki} and \emph{ppoi} which are truly embedded.}

\begin{exe}
\ex \label{saito5}
\gll {John$_{\textnormal{i}}$-wa}  {Mary-ni} [[{kare$_{\textnormal{i}}$-no} {mise-de} {tabe-ro}]-{teki-na}] {mono-o} {tutae-ta}.\\
John-\textsc{top} Mary-\textsc{dat} he-\textsc{gen} store-at eat-\textsc{imp-teki-cop} thing-\textsc{acc} tell-\textsc{past}\\ 
\glt `John$_\textnormal{i}$ told Mary something like what she should eat at his$_\textnormal{i}$ restaurant.’ \jambox*{(\citealt{Saito2017}: 167)}
\end{exe}

The clausal complement of \emph{ppoi} behaves differently in this regard. Imperatives cannot appear in the clause selected by \emph{ppoi}, as in (\ref{saito6}).

\begin{exe}
\ex \label{saito6}
\gll [[{asita} {a-e}]-{ppoi}] {hito}\\
tomorrow meet-\textsc{imp}-\textsc{ppoi} person\\ 
\glt ‘(a) person like (the) one who you should meet tomorrow’
\end{exe}

The volitional marker \emph{yoo} shows the same distribution as imperatives. It can appear in the clausal complement of \emph{teki}, not in that of \emph{ppoi}:\footnote{The volitional \emph{oo} in (\ref{saito7}) is a realization of the volitional marker in question, \emph{yoo}. The distribution of \emph{yoo} and \emph{oo} is phonologically conditioned.}


\begin{exe}
\ex \label{saito7}
\gll [[{asita} {a-oo}]-\{{teki-na/*ppoi}\}] {hito}\\
tomorrow meet-\textsc{vol}-\{\textsc{teki-cop/ppoi}\} person\\ 
\glt ‘(a) person like (the) one who I/we will meet tomorrow’
\end{exe}

The same holds for the politeness marker. As illustrated in (\ref{saito8}), the politeness marker can occur in the clause selected by \emph{teki} while it cannot appear in the clause taken by \emph{ppoi}.

\begin{exe}
\ex \label{saito8}
\gll [[{asita} {ai-mas-u}]-\{{teki-na/*ppoi}\}] {hito}\\
tomorrow meet-\textsc{pol-pres}-\{\textsc{teki-cop/ppoi}\} person\\ 
\glt ‘(a) person like (the) one who I will meet.\textsubscript{\textsc{polite}} tomorrow’
\end{exe}

The clausal complements of \emph{teki} and \emph{ppoi} thus show differences regarding imperatives, volitionals, and the politeness marker. All of them can appear in the clausal complement of \emph{teki}, but not in that of \emph{ppoi}, which indicates that \emph{teki} and \emph{ppoi} take different kinds of clausal complements.

There is a further difference between the clausal complements of \emph{teki} and \emph{ppoi}, which concerns nominative-genitive conversion. It is well known that the subject in prenominal clauses (relative clauses) can be optionally marked by genitive case, instead of nominative, in Japanese, as illustrated in (\ref{saito9}) (nominative-genitive conversion, NGC henceforth; see e.g. \citealt{Harada1971}, \citealt{Watanabe1996a}, \citealt{Hiraiwa2000,hiraiwa2005}, \citealt{MakiUchibori2008}, \citealt{miyagawa2011}).

\begin{exe}
\ex \label{saito9}
\gll [{okaasan}-\{{ga/no}\} {tuku-ru}] {karee}\\
mother-\{\textsc{nom/gen}\} cook-\textsc{pres} curry\\ 
\glt ‘curry the mother cooks’
\end{exe}

Crucially, NGC is impossible in the clausal complement of \emph{teki}, as in (\ref{saito10}) (\citealt{Saito2017}).

\begin{exe}
\ex[*]{\label{saito10}
\gll [[{hudan} {okaasan-no} {tuku-ru}]-{teki-na}] {karee}\\
usually mother-\textsc{gen} cook-\textsc{pres-teki-cop} curry\\ 
\glt ‘curry like (the) one the mother usually cooks’}
\end{exe}

This contrasts with the clausal complement of \emph{ppoi}, where NGC is allowed.

\begin{exe}
\ex \label{saito11}
\gll [[{hudan} {okaasan-no} {tuku-ru}]-{ppoi}] {karee}\\
usually mother-\textsc{gen} cook-\textsc{pres-ppoi} curry\\ 
\glt ‘curry like (the) one the mother usually cooks’
\end{exe}

We thus observe a difference between the clausal complements of \emph{teki} and \emph{ppoi} with respect to the availability of NGC. NGC is allowed in the complement clause of \emph{ppoi} while it is disallowed in that of \emph{teki}.

To sum up, I have shown that, despite their apparent similarities, \emph{teki} and \emph{ppoi} show differences in their clausal complements regarding the distribution of imperatives, volitionals, and politeness marking. They also behave differently regarding the availability of NGC; NGC is possible in the clausal complement of \emph{ppoi}, not in that of \emph{teki}. In the next section, to account for this contrast, I will suggest that \emph{teki} and \emph{ppoi} select a sentential complement of different sizes; \emph{teki} takes a larger complement than \emph{ppoi}.

\section{The size of things: CP vs. TP} \label{saitos3}

In the previous section, we have seen that the clausal complements of \emph{teki} and \emph{ppoi} behave differently regarding the distribution of imperatives, volitionals, politeness marking, and NGC. The observations from the previous section are summarized in Table \ref{saitotab1}.


\begin{table}
\caption{Clausal complements of \emph{teki} and \emph{ppoi}}
\label{saitotab1}
 \begin{tabular}{l rr}
  \lsptoprule
  \multicolumn{1}{r@{}}{Clausal complements of:}     &     \emph{teki} & \emph{ppoi}\\
  \midrule
  Imperatives can appear  &   Yes  &    No     \\
  Volitionals can appear &   Yes &   No      \\
  The politeness marker can appear & Yes & No \\
  NGC is possible & No & Yes\\
  \lspbottomrule
 \end{tabular}
\end{table}

In this section, I will suggest that the contrasts found in Table \ref{saitotab1} are due to the difference in the size between the clausal complements of \emph{teki} and \emph{ppoi}. Specifically, I will argue that \emph{teki} takes a larger sentential complement than \emph{ppoi}. 

First, consider the (un)availability of imperatives, volitionals, and the politeness marker. To explain the fact that these elements can appear in the clausal complement of \emph{teki}, but not in that of \emph{ppoi}, I suggest that the former involves richer structure than the latter (see \citealt{wurmbrand2001} et seq. on selection of clausal complements of different sizes). Independently, imperatives, volitionals, and politeness marking have been claimed to involve the CP-domain (or some projection above TP, e.g. \citealt{rizzi1997}, \citealt{Han1998}, \citealt{cinque1999}, \citealt{Haegeman2006}, see also \citealt{Ueda2007,Ueda2007}, \citealt{Endo2009}, \citealt{Hasegawa2010}, \citealt{miyagawa2012a}, \citealt{Yoshimoto2017} for volitionals, imperatives, and the politeness marker in Japanese).\footnote{But see \citealt{shimamura2021} [this volume] for \emph{yoo}.} I thus assume that the imperative, volitional, and politeness morphemes (or corresponding operators) are located in a C-head in Japanese, and argue that \emph{teki} takes CP as its complement, as shown in (\ref{saito12}). The presence of the CP-layer, which is the locus of imperatives, volitionals, and politeness marking, ensures the availability of these elements in the clausal complement of \emph{teki}.\footnote{If we assume a more fine-grained structure of CP in Japanese, \emph{teki} would take ReportP in \citeauthor{Saito2012}'s (\citeyear{Saito2012}) sense, which is usually selected by a verb of saying/thinking. I leave for future research investigations of the clausal complements of \emph{teki} and \emph{ppoi} in terms of the cartographic approach to the Japanese right periphery. See also \citet{Saito2017} for similarities between clausal complements of \emph{teki} and those of verbs of saying.}

\begin{exe}
\ex \label{saito12}
[$_{\textnormal{CP}}$ [$_{\textnormal{TP}}$ ...]\hspace{0.5mm}]-\emph{teki}
\end{exe}

I further suggest that \emph{ppoi} takes a smaller complement than CP. Recall that tense markers can appear in the clausal complement of \emph{ppoi} (while the imperative/volitional/politeness morphemes cannot), as shown in (\ref{saito2}), repeated below. 

\begin{exe}
\ex \label{saito13}
\gll [[$_{\textnormal{S}}$ {okaasan-ga}   {tuku-ru}]-\{{teki-na/ppoi}\}] {karee}\\
{} mother-\textsc{nom} cook-\textsc{pres}-\{\textsc{teki-cop/ppoi}\} curry\\ 
\glt ‘curry like (the) one the mother cooks’ (Lit. ‘[the mother cooks]-ish/like curry’)
\end{exe}

Given this, I suggest that \textit{ppoi} takes a TP complement, as in (\ref{saito14}) (see also \citealt{Yamada2014}).

\begin{exe}
\ex \label{saito14}
[$_{\textnormal{TP}}$ ...]-\emph{ppoi}
\end{exe}

Tense markers like \emph{ru} ‘\textsc{pres}’ can occur in the clausal complement of \emph{ppoi} since  the TP-layer is present. The imperative, volitional, and politeness markers, however, cannot appear due to the lack of the C-domain, which is necessary to host these elements; there is no syntactic position for them.

I furthermore suggest that the (un)availability of NGC in the clausal complements of \emph{teki} and \emph{ppoi} is also due to their size difference. I here assume that genitive case in Japanese is assigned by an N (or D) head through a syntactic dependency (e.g. \citealt{Bedell1972}, \citealt{miyagawa1993}, \citealt{miyagawa2011}). To be more specific, I assume that N licenses genitive case through an Agree relation (\citealt{miyagawa2011}). (\ref{saito15}) shows the standard case of NGC, where the subject in a relative clause is marked with genitive case, like (\ref{saito9}) above. In (\ref{saito15}), the N head enters an Agree relation with the subject in the relative clause, licensing the genitive case on it. I also assume that relative clauses are TPs in Japanese, following \citet{murasugi1991}, \citet{Taguchi2008}, and \citet{ParkYoo2017} (see also \citealt{Saito1985}).

\begin{exe}
\ex \label{saito15}
\begin{tikzpicture}[baseline]
\node[baseline] (A) at (0,0) {[\textsubscript{NP} [\textsubscript{TP (Relative clause)}};
\node[right] (B) at (A.east) {Subject};
\node[right] (C) at (B.east) {\ldots{} ]};
\node[right] (D) at (C.east) {N};
\node[right] (E) at (D.east) {]};
\draw[-] (D) -- ($ (D) - (0,0.5) $) -- ($ (B) - (0,0.5) $) -- (B);
\end{tikzpicture}
\end{exe}

Let us then consider NGC in the clausal complement of \emph{ppoi} first. As schematically shown in (\ref{saito16}), the N head licenses the genitive subject in the clausal complement of \emph{ppoi}, like the standard case of NGC.

\begin{exe}
\ex \label{saito16}
\begin{tikzpicture}[baseline]
\node[] (A) at (0,0) {[\textsubscript{NP} [ [\textsubscript{TP}};
\node[right] (B) at (A.east) {Subject};
\node[right] (C) at (B.east) {\ldots{} ]-\emph{ppoi} ]};
\node[right] (D) at (C.east) {N};
\node[right] (E) at (D.east) {]};
\draw[-] (D) -- ($ (D) - (0,0.5) $) -- ($ (B) - (0,0.5) $) -- (B);
\end{tikzpicture}
\end{exe}

It should be noted that (\ref{saito16}) is slightly different from the standard case of NGC in an unmarked relative clause like (\ref{saito9})/(\ref{saito15}) because \emph{ppoi} appears between the prenominal clause and the head noun. If we assume that Agree is subject to the Phase Impenetrability Condition (and assuming \citeauthor{Chomsky2000}'s (\citeyear{Chomsky2000,Chomsky2001}) approach to phases), \emph{ppoi} (or \emph{teki}), being an adjectival head, is not a phasal head, hence its presence does not block the Agree relation between the embedded subject and the head noun.\footnote{In this paper, I treat \emph{teki(-na)} and \emph{ppoi} as simply an A head, leaving aside investigations of the exact structure involved with adjectives, including adjectival inflection and the copula on (nominal) adjectives (see e.g. \citealt{nishiyama1999}, \citealt{Yamakido2005,Yamakido2013} for relevant discussion). What is important for the current discussion is the size difference between the clausal complements of \emph{teki} and \emph{ppoi}, not the structure of these elements themselves.}

In the previous section, we have observed that NGC is not allowed in the clausal complement of \textit{teki}, as in (\ref{saito10}), repeated below. 

\begin{exe}
\ex[*]{ \label{saito17}
\gll [[{hudan} {okaasan-no} {tuku-ru}]-{teki-na}] {karee}\\
usually mother-\textsc{gen} cook-\textsc{pres-teki-cop} curry\\ 
\glt ‘curry like (the) one the mother usually cooks’}
\end{exe}

I suggest that the unavailability of NGC here is due to the extra layer the clausal complement of \emph{teki} involves, namely, the C-domain. Since CP is a phase, I claim that C prevents the N head from licensing genitive case, disallowing genitive subjects in CP (see also \citealt{miyagawa2011}, \citealt{ParkYoo2017}). This is schematically illustrated in (\ref{saito18}). 

\begin{exe}
\ex \label{saito18}
\begin{tikzpicture}[baseline]
\node[] (A) {[\textsubscript{NP} [ [\textsubscript{CP} [\textsubscript{TP}};
\node[right] (B) at (A.east) {Subject};
\node[right] (C) at (B.east) {\ldots{} ] ]-\emph{teki} ]};
\node[right] (D) at (C.east) {N};
\node[right] (E) at (D.east) {]};
\draw[-] (D) -- ($ (D) - (0,0.5) $) to node {$\not$} ($ (B) - (0,0.5) $) -- (B);
\end{tikzpicture}
\end{exe}

Therefore, the contrast between the clausal complements of \emph{teki} and \emph{ppoi} regarding the availability of imperatives, volitionals, politeness marking, and NGC can be captured under the current analysis, where \emph{teki} and \emph{ppoi} take sentential complements of different sizes; \emph{teki} takes a larger complement than \emph{ppoi}.

One may wonder if the suggested size difference between the clausal complements of \emph{teki} and \emph{ppoi} is only found with these two specific elements. As we will see in the next section, the contrast in question is in fact found with other particles, indicating that the size difference I have suggested in this section is not idiosyncratic to \emph{teki} and \emph{ppoi}.

\section{\textit{Mitai} and \textit{yoo}} \label{saitos4}

In the previous section, I have argued that the contrast between the clausal complements of \emph{ppoi} and \emph{teki} is due to their syntactic size difference. In this section, I will show that the same contrast is found with another pair; the evidential particles \emph{mitai} and \emph{yoo}.

\emph{Mitai} and \emph{yoo} are used to mark inferential evidentiality, as shown in (\ref{saito19}). \emph{Mitai} and \emph{yoo} have the same or at least very similar meaning. In fact, \citet[169]{Narrog2009} notes that they are “stylistic variant[s]”.

\begin{exe}
\ex \label{saito19}
\gll {Okaasan-ga}  {karee-o} {tuku-ru-}\{{mitai/yoo}\}-{da.}\\
mother-\textsc{nom} curry-\textsc{acc} cook-\textsc{pres}-\{\textsc{mitai/yoo}\}-\textsc{cop}\\ 
\glt ‘It seems that the mother will cook curry.’
\end{exe}

Like \emph{teki} and \emph{ppoi}, these elements can occur in prenominal clauses, as in (\ref{saito20}).

\begin{exe}
\ex \label{saito20}
\gll [[{okaasan-ga} {tuku-ru}]-\{{mitai/yoo}\}-{na}] {karee}\\
mother-\textsc{nom}  cook-\textsc{pres}-\{\textsc{mitai/yoo}\}-\textsc{cop} curry\\ 
\glt ‘curry like (the) one the mother cooks’
\end{exe}

While \emph{mitai} and \emph{yoo} are semantically the same, there are a number of differences regarding properties of the clausal complements \emph{mitai} and \emph{yoo} take. What is crucial for us is that the clausal complements of \emph{mitai} and \emph{yoo} behave just like those of \emph{teki} and \emph{ppoi}, respectively; the clausal complements of \emph{mitai} and \emph{yoo} show the same contrasts as the ones we have observed for \emph{teki} and \emph{ppoi} in \sectref{saitos2}.

First, the imperative marker can occur in the clausal complement of \emph{mitai}, but not in that of \emph{yoo}, as shown in (\ref{saito21}).

\begin{exe}
\ex \label{saito21}
\gll [[{asita} {a-e}]-\{{mitai/*yoo}\}-{na}] {hito}\\
tomorrow meet-\textsc{imp}-\{\textsc{mitai/yoo}\}-\textsc{cop} person\\ 
\glt ‘(a) person like (the) one who you should meet tomorrow’
\end{exe}

It should be noted here that \emph{mitai} can truly embed its clausal complement. In (\ref{saito22}), the pronoun \emph{kare} in the clausal complement of \emph{mitai} refers to the matrix subject John. This reading would be impossible if \emph{mitai} could not truly embed a clause  (see (\ref{saito5}) above for \emph{teki}).\footnote{\emph{Mitai} can also introduce a direct quote. In (\ref{saitoi2}), the pronoun \emph{ore} ‘I’ refers to John under the intended interpretation. \emph{Ore} would refer to the speaker of (\ref{saitoi2}) if (\ref{saitoi2}) could involve only an indirect quote. I will focus here on cases where the clausal complement of \emph{mitai} is truly embedded. 
\begin{exe}
\ex \label{saitoi2}
\gll {John$_{\textnormal{i}}$-wa}  {Mary-ni} [[{ore$_{\textnormal{i}}$-no} {ie-ni} {ko-i}]-{mitai-na]-koto]-o} {tutae-ta}.\\
John-\textsc{nom} Mary-\textsc{dat} I-\textsc{gen} home-to come-\textsc{imp-mitai-cop}-thing-\textsc{acc} tell-\textsc{past}\\ 
\glt ‘John$_{\textnormal{i}}$ told Mary something like: “Come to my$_{\textnormal{i}}$ home!”’

\end{exe}
}

\begin{exe}
\ex \label{saito22}
\gll {John$_{\textnormal{i}}$-wa}  {Mary-ni} [[{kare$_{\textnormal{i}}$-no} {mise-de} {tabe-ro}]-{mitai-na}] {mono-o} {tutae-ta}.\\
John-\textsc{top} Mary-\textsc{dat} he-\textsc{gen} store-at eat-\textsc{imp-mitai-cop} thing-\textsc{acc} tell-\textsc{past}\\ 
\glt ‘John$_{\textnormal{i}}$ told Mary something like what she should eat at his$_{\textnormal{i}}$ restaurant.’
\end{exe}

The same holds for the volitional marker and the politeness marker. As (\ref{saito23}) and (\ref{saito24}) show, they can appear in the clausal complement of \emph{mitai}, but are disallowed in that of \emph{yoo}.


\begin{exe}
\ex \label{saito23}
\gll [[{asita} {a-oo}]-\{{mitai/*yoo}\}-{na}] {hito}\\
tomorrow meet-\textsc{vol}-\{\textsc{mitai/yoo}\}-\textsc{cop} person\\ 
\glt  ‘(a) person like (the) one who I/we will meet tomorrow’

\ex \label{saito24}
\gll [[{asita} {a-mas-u}]-\{{mitai/*yoo}\}-{na}] {hito}\\
tomorrow meet-\textsc{pol-pres}-\{\textsc{mitai/yoo}\}-\textsc{cop} person\\ 
\glt ‘(a) person like (the) one who I will meet.\textsubscript{\textsc{polite}} tomorrow’
\end{exe}

Therefore, the clausal complement of \emph{mitai} shows the same syntactic properties as that of \emph{teki} regarding the distribution of the imperative, volitional, and politeness morphemes: these elements can appear in the clausal complements of \emph{mitai} and \emph{teki}. Furthermore, \emph{yoo} and \emph{ppoi} behave in the same way in this regard. In their clausal complements, the imperative, volitional, and politeness markers are all disallowed (see \sectref{saitos2} for \emph{teki} and \emph{ppoi}).

Furthermore, the clausal complements of \emph{mitai} and \emph{yoo} show the same contrast as those of \emph{teki} and \emph{ppoi} regarding the availability of NGC. As observed in (\ref{saito20}) above, as well as in (\ref{saito25}) below, the subject in the clausal complement of \emph{mitai} and \emph{yoo} is usually marked with nominative case, just like regular subjects.

\begin{exe}
\ex \label{saito25}
\gll [[{hudan} {okaasan-ga}   {tuku-ru}]-\{{mitai/yoo}\}-{na}] {karee}\\
usually mother-\textsc{nom} cook-\textsc{pres}-\{\textsc{mitai/yoo}\}-\textsc{cop} curry\\ 
\glt ‘curry like (the) one the mother usually cooks’
\end{exe}

Let us then look at NGC in the sentential complements of \emph{mitai} and \emph{yoo}. In the clausal complement of \emph{mitai}, NGC is disallowed, as illustrated in (\ref{saito26}). The subject in the \emph{mitai}-clause cannot be marked with genitive case.\footnote{It should be noted that some speakers find (\ref{saito26}) better than NGC in the clausal complement of \emph{teki} (=\,\ref{saito17}). I put this speaker variation aside in this paper.} 

\begin{exe}
\ex[*]{ \label{saito26}
\gll [[{hudan} {okaasan-no} {tuku-ru}]-{mitai-na}] {karee}\\
usually mother-\textsc{gen} cook-\textsc{pres-mitai-cop} curry\\ 
\glt ‘curry like (the) one the mother usually cooks’}
\end{exe}

In contrast, in the clausal complement of \emph{yoo}, like that of \emph{ppoi}, NGC is possible, as in (\ref{saito27}).

\begin{exe}
\ex \label{saito27}
\gll [[{hudan} {okaasan-no} {tuku-ru}]-{yoo-na}] {karee}\\
usually mother-\textsc{gen} cook-\textsc{pres-yoo-cop} curry\\ 
\glt ‘curry like (the) one the mother usually cooks’
\end{exe}

The clausal complements of \emph{mitai} and \emph{yoo} thus show the same contrasts as those of \emph{teki} and \emph{ppoi}; in \emph{mitai/teki}-clauses, NGC is disallowed while in \emph{yoo/ppoi}-clauses, NGC is possible.

To account for the contrast between the clausal complement of \emph{mitai} and \emph{yoo}, I suggest that \emph{mitai} and \emph{yoo} take clausal complements of different sizes; \emph{mitai} selects a CP complement while \emph{yoo} selects a TP complement, just like \emph{teki} and \emph{yoo}, respectively.

\begin{exe}
\ex \label{saito28}
\begin{xlist}
\ex \label{saito28a}
[$_{\textnormal{CP}}$[$_{\textnormal{TP}}$ ...]]-\emph{mitai}
\ex \label{saito28b} [$_{\textnormal{TP}}$ ...]-\emph{yoo}
\end{xlist}
\end{exe}

We can then obtain a parallel explanation for the contrasts between \emph{mitai} and \emph{yoo} as for the contrasts between \emph{teki} and \emph{ppoi}. In the clausal complement of \emph{mitai}, the CP-layer, which provides syntactic positions for imperatives, volitionals, and politeness marking, is present. Hence, these elements can appear. In the clausal complement of \emph{yoo}, however, there is no syntactic position for these elements due to the lack of the C-domain.

Regarding the (un)availability of NGC, there is a phasal head C present in the clausal complement of \emph{mitai}, which blocks genitive case licensing from the N head.

\begin{exe}
\ex \label{saito29}
\begin{tikzpicture}[baseline]
    \node[] (A) {[\textsubscript{NP} [ [\textsubscript{CP} [\textsubscript{TP}};
    \node[right] (B) at (A.east) {Subject};
    \node[right] (C) at (B.east) {\ldots{} ] ]-\emph{mitai} ]};
    \node[right] (D) at (C.east) {N};
    \node[right] (E) at (D.east) {]};
    \draw[-] (D) -- ($ (D) - (0,0.5) $) to node {$\not$} ($ (B) - (0,0.5) $) -- (B);
    \end{tikzpicture}
\end{exe}
In the clausal complement of \emph{yoo}, on the other hand, due to the absence of the C-layer, there is no intervener for the Agree relation between the subject and the N head. Thus, NGC is possible.

Before concluding this section, a note on syntactic (c-) and semantic (s-) selection is in order. There has been a controversy whether syntactic selection and semantic selection are independent or one can be derived from the other (e.g. \citealt{Grimshaw1979}, \citealt{pesetsky1982}, \citealt{PollardSag1987}, \citealt{ChomskyLasnik1993}, \citealt{Bedell1972},  \citealt{Odijk1997}). In this section, we have observed that, while \emph{mitai} and \emph{yoo} are semantically the same (or at least very similar, recall that \citet[169]{Narrog2009} states that they are stylistic variants), there are a number of differences regarding syntactic properties of their clausal complements, which can be captured under the current analysis. (Recall also that the morphemes \emph{teki} ‘like, -ish’ and \emph{ppoi} ‘like, -ish’ are also semantically similar.) If selection of clausal complements of \emph{mitai} and \emph{yoo} were solely semantically determined, the contrast we have observed in this section for \emph{mitai} and \emph{yoo} would be difficult to capture, as the lexical semantics of \emph{mitai} and \emph{yoo} are (almost) the same. Thus, the contrast between the type of clausal complements of \emph{mitai} and \emph{yoo} provides evidence for the independence of syntactic selection from semantic selection, as argued by \citet{Grimshaw1979} among others.

To wrap up this section, we have observed that the clausal complements of \emph{mitai} and \emph{yoo} behave in the same way as those of \emph{teki} and \emph{ppoi}, respectively, regarding the distribution of the imperative, volitional, and politeness morphemes and the availability of NGC. I have suggested that the contrasts in question are due to the size difference between the clausal complements of \emph{mitai} and \emph{yoo}, just like the contrast between \emph{teki} and \emph{ppoi}. The size differences between the clausal complements of \emph{teki} and \emph{ppoi} are not idiosyncratic to these items. The observed differences in the size of clausal complements have also been argued to provide evidence that syntactic and semantic selection are independent mechanisms.

\section{Conclusion} \label{saitos5}
In this chapter, we have observed that the clausal complements of the semantically similar particles \emph{teki} and \emph{ppoi} show differences regarding the distribution of imperatives, volitionals, politeness marking, and nominative genitive conversion. Imperatives, volitionals, and politeness marking can appear in the clausal complement of \emph{teki}, but not in that of \emph{ppoi}. NGC is allowed in the clausal complement of \emph{ppoi}, but not in that of \emph{teki}. I have suggested that this contrast is due to the difference in size of their clausal complements, arguing that \emph{teki} takes a larger sentential complement than \emph{ppoi}. I have also shown that the same contrast is found with the evidential particles \emph{mitai} and \emph{yoo}, which indicates that the size difference in clausal complements I have argued for is not idiosyncratic to the pair of \emph{teki} and \emph{ppoi}. The difference in the size of the clausal complements of the elements in question provides evidence that syntactic selection (c-selection) is needed independently of semantic selection (s-selection).

\section*{Acknowledgements}
It is a great pleasure and honor to dedicate this paper to Susi Wurmbrand, who has made significant contributions to theoretical linguistics, and who has been always encouraging as a wonderful teacher. For valuable comments, I thank Željko Bošković and two anonymous reviewers. I am also grateful to the audience at the 22nd Seoul International Conference on Generative Grammar (SICOGG 22) and the 28th Japanese/Korean Linguistics Conference (JK28), where parts of this paper were presented. This work was partially supported by the Japan Society for the Promotion of Science (JSPS) KAKENHI Grant Number JP20K13003.

{\sloppy\printbibliography[heading=subbibliography,notkeyword=this]}

\end{document}
