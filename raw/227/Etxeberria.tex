\documentclass[output=paper,
modfonts
]{langscibook}

\title{Definiteness, partitivity, and domain restriction: A fresh look at definite reduplication}

\author{Urtzi Etxeberria\affiliation{CNRS-IKER}\lastand Anastasia Giannakidou\affiliation{University of Chicago}}

\ChapterDOI{10.5281/zenodo.3252032}
% \epigram{}


%ABSTRACT
\abstract{We propose that the phenomenon of \isi{definite reduplication} in \ili{Greek} involves using the \is{definite determiners}definite determiner D as domain restrictor in the sense of \citet{etxeberria-giannakidou2009}. The use of D as a domain-restricting function with \isi{quantifiers} has been well documented for \ili{European languages} such as \ili{Greek}, Basque\il{Basque}, \ili{Bulgarian} and \ili{Hungarian} -- and typically results in a \is{partitives}partitive-like interpretation of the \is{quantifier phrases}QP. We propose a unifying analysis that treats \isi{domain restriction} and D-reduplication as the same phenomenon; and in our analysis, D-reduplication emerges semantically as similar to a partitive structure, a result resonating with earlier claims to this end by \citet{kolliakou2004}. None of the existing accounts of definites can capture the correlations in the use of D with quantifiers and in reduplication that we establish here.}

\shorttitlerunninghead{Definiteness, partitivity, and domain restriction: Definite reduplication}

\begin{document}
\maketitle

\section{Quantifiers, domain restriction, and D} 
One of the most fruitful ideas in the formal semantics tradition has been the thesis that \is{quantifier phrases}quantifier phrases (QPs) denote \isi{generalized quantifiers} (GQs; see \citealt{montague1974, BarwiseCooper1981, Westerstahl1984, Partee1987, zwarts1986, keenan1987, keenan1996, keenan-westerstahl1997}; among many others). Classical GQ theory posits that there is a natural class of expressions in language, called \is{quantificational determiners|(}quantificational determiners (Qs), which combine with a nominal constituent (an NP of type \textit{et}, a first order predicate) to form a quantifier nominal (QP). This \is{quantifier phrases}QP denotes a \is{generalized quantifiers}GQ, a set of sets. In a language like \ili{English}, the syntax of a QP like \textit{every woman} is as follows:

\ea\label{ex:etxeberria:1}
	\ea\label{ex:etxeberria:1a}
	$[\![ $\textit{every}$ \hspace{1mm} $\textit{woman}$ ]\!]$  = $\lambda$Q. $\forall$x. woman(x) $\rightarrow$ Q(x)
	\ex\label{ex:etxeberria:1b} 
	$[\![ $\textit{every}$ ]\!]$  = $\lambda$P. $\lambda$Q. $\forall$x. P(x) $\rightarrow$ Q(x)
	\ex\label{ex:etxeberria:1c} \Tree[.QP\hspace{1mm}\textit{ett} [.Q\hspace{1mm}\textit{et,ett} \textit{every} ] [.NP\hspace{1mm}\textit{et} \textit{woman} ] ]
	\z
\z

The Q \textit{every} combines first with the NP argument \textit{woman}, and this is what we have come to think of as the ``standard'' QP-internal syntax. The NP argument provides the domain of the Q, and the Q expresses a relation between this domain and the set denoted by the VP. Qs like \textit{every, most,} etc. are known as \textsc{strong}, and they contrast with the so-called \textsc{weak} \isi{quantifiers} like e.g. \textit{some, few, three, many} \citep{Milsark1977}.

It has also long been noted that the domain of strong quantifiers is contextually (explicitly or implicitly) restricted \citep[see \textit{inter alia}][]{reuland-termeulen1987}. Contemporary work agrees that we need to encode contextual restriction in the QP, but opinions vary as to whether contextual restriction is part of the \is{syntax-semantics interface}syntax/semantics \citep{Partee1987, Fintel1994, vonfintel1998, stanley-szabo2000, stanley2002, matthewson2001, marti2003, Giannakidou2004, etxeberria2005, etxeberria2008, etxeberria2009, gillon2006, gillon2009, etxeberria-giannakidou2009, etxeberria-giannakidou2014, giannakidou-rathert2009}, or not (\citealt{recanati1996, recanati2004, recanati2007} and others in the strong contextualism tradition). In the syntax-semantics approach, it is  assumed that the \is{domain restriction}domains of Qs are contextually restricted by covert domain variables at \is{logical form}LF (which are usually free, but can also be bound, and they can be  either atomic, e.g. \textit{C}, or complex of the form \textit{f(x)}, corresponding to selection functions; see \citealt{vonfintel1998,stanley2002,marti2003}). Below, we employ C:

\begin{exe}
\ex \label{ex:etxeberria:2} \textit{Many people came to the concert last night; every student got drunk.}
\ex \label{ex:exteberria:3}
$\forall$x {\ob} student (x) $\cap$ C(x) {\cb} $\rightarrow$ got drunk (x).
\end{exe}

Here, the nominal argument of the \isi{universal quantifier} \textit{every}, i.e. \textit{student}, is the set of students who came to the concert last night, not the students in the whole world. This is achieved by the domain variable C, which is an \isi{anaphor} and will look back in the \isi{discourse} for a salient property, in this case the set of people who came to the concert last night. \textit{Every student} then will draw values from the intersection of \textit{student} with C.

Another element that combines with a domain to give a nominal argument is the \is{definite determiners}definite determiner, i.e. the \ili{English} \textit{the} and its equivalents (including \isi{demonstratives}), designated as D (\citealt{abney1987}; see \citealt{AlexiadouEtAlii2008} for an extensive overview). The demonstrative is generated in \ili{English} under the same head (thus *\textit{this the book}). The \is{determiner phrases|(}DP has a structure parallel to \REF{ex:etxeberria:1c}, only  we have D, and the constituent is called DP (though some authors call the Q uniformly D; see \citealt{matthewson1998}; \citealt{gillon2009}). As indicated below, the DP produces a referential expression, a (\is{maximality}maximal or \is{uniqueness|(}unique) individual, indicated here with \is{iota operator}\textit{iota}:

\ea\label{ex:exteberria:4} 
\Tree[.DP\textit{e}:\hspace{1mm}$\iota$($\lambda$x.woman(x)) [.D\hspace{1mm}\textit{et,e} $\{$\textit{the}/\textit{this}$\}$ ] [.NP\hspace{1mm}\textit{et} \textit{woman}:\hspace{1mm}$\lambda$x.woman(x) ] ]
\z

\ea\label{ex:etxeberria:5}
	\ea \textit{the}/\textit{this woman} = $\iota$ ($\lambda$x.woman (x))
	\ex \textit{the}/\textit{these women} = max ($\lambda$x.woman (x))
	\z
\z

The DP produces the most basic argument \textit{e} which can be lifted up to the \is{generalized quantifiers}GQ type when necessary. Both D and \is{quantificational determiners|)}Q are functions that need a domain, and it is the NP that provides this domain. Contextual presuppositions are indicated above in the indexing with C. The DP denotes the unique or maximal individual presupposed to exist in the common ground. \citet{CoppockBeaver2015} use $\theta$-notation to capture the presupposition of \isi{uniqueness} as the argument of the $\theta$ operator:

\ea\label{ex:exteberria:6}
Lexical entry: \textit{the}\\
\textit{the} $\rightarrow$ $\lambda$P.$\lambda$x {\ob}$\theta$($\vert$P$\vert$ $\leq$ 1) $\wedge$ P(x){\cb} 
\z

Notice that, contrary to all other approaches, for \citet{CoppockBeaver2015} \textit{the} is a \is{non-saturating functions}non-saturated constituent in the referential use. We come back to this assumption later. We take it here that the use of D creates a morphologically definite argument, it is thus the core of what can be understood as ``definiteness''.

DP has been argued to exhibit different types of referentiality. For one thing, a \is{generic definites}DP can be generic and \is{kind reference}refer to a \is{kinds}kind which is itself a very different ``object'' than a concrete unique entity in the world. Observe, in addition, the following:

\ea\label{ex:exteberria:7}
	\ea\label{ex:exteberria:7a}
		\textit{John got these data from \textbf{the student of a linguist}}.
	\ex\label{ex:exteberria:7b}
		\textit{John went \textbf{to the store}}.
	\ex\label{ex:exteberria:7c}
		\textit{I read \textbf{the newspaper} every day}.
	\ex\label{ex:exteberria:7d}
		\textit{I raised \textbf{my hand}}.
	\z
\z

In the examples here the DPs do not make \isi{reference} to unique entities: the linguist in \REF{ex:exteberria:7a} possibly has more than one student; in \REF{ex:exteberria:7b} the particular identity of the store to which John has gone is not important, and the store is certainly not unique; \REF{ex:exteberria:7c} can be used in a context in which no newspaper has been mentioned or in which multiple newspapers are read; in \REF{ex:exteberria:7d} \textit{my hand} is used to make reference to one of my two hands. \citet{Poesio1994} introduced the term \is{weak definites}``weak definite'' to refer to such ``non-uniquely referential'' uses of D (see among others \citealt{CarlsonSussman2005, Schwarz2009, Aguilar-GuevaraZwarsts2011, corblin2013}). More recent relevant work identifies \is{sloppy identity}``sloppy'' identity, \isi{narrow scope} interpretation, lexical restrictions (\textit{John took the bus} vs \#\textit{John took the coach}), restrictions on modification, \isi{number} restrictions, and meaning enrichment (\textit{John went to the store} means that John went to a store to do some shopping) for such non-unique DPs (see \citealt{CarlsonSussman2005, Aguilar-GuevaraEtAlii2014}).

In some languages, the referential strength of DP is reflected in a difference between weak and strong forms of D itself (\citealt{Cieschinger2006, Waldmuller2008, Schwarz2009}). In \il{German}Standard German, for example, a \is{prepositions}preposition and the definite article can be contracted (\textit{zum} vs. \textit{zu dem}). \citet{Schwarz2009} proposes that the \is{strong definite articles}strong/non-contracted D is used when the \is{noun phrases}noun phrase is \is{anaphora}anaphoric (a \is{definites}pragmatic definite) and it picks up a unique/given referent from the \isi{discourse}; the \is{weak definite articles}weak/contracted article is used when the noun phrase has \is{uniqueness|)}unique \isi{reference} on the basis of its own description.\is{strong definite articles}\is{weak definite articles} 

In the present paper, we discuss two puzzles of D in \ili{Greek} and Basque\il{Basque} that cannot be described by the existing approaches in terms of non-\isi{uniqueness} or weak/strong D. The D in the case we focus on appears in a non-canonical position: (a) on a \is{quantificational determiners}quantificational determiner; and (b) multiple D structures. Let us illustrate the first, which holds also in \ili{Salish} languages, \ili{Hungarian} and \ili{Bulgarian}. D can be an independent head (\ili{Greek}, \il{Salish!St'\'at'imcets Salish}St'\'at'imcets),\footnote{The \il{Salish!St'\'at'imcets Salish}St'\'at'imcets D has a proclitic part (\textit{ti} for singulars; \textit{i} for plurals) encoding deictic and \isi{number} morphology, and an enclitic part \textit{…a} adding to the first lexical item in the \is{determiner phrases|)}DP (\citealt{matthewson1998}).} or suffixal D (Basque\il{Basque}, \ili{Bulgarian}): \newpage

\ea\label{ex:etxeberria:8}
\ili{Greek} \citep[121]{Giannakidou2004}
\ea[]{
	\gll {o} {kathe} {fititis}\\
	\textsc{det.sg} every student\\
	\glt `each student'}
\ex[*]{
	\gll {kathe} {o} {fititis}\\
	every \textsc{det.sg} student\\
	\glt {(`each student')}}
\z
\z

\ea\label{ex:etxeberria:9}
Basque\il{Basque} \citep[41--42]{etxeberria2005}
\ea[]{\label{ex:etxeberria:9a} 
	\gll {mutil} {guzti-ak}\\
	boy all-\textsc{det.pl}\\
	\glt {`all the students'}}
\ex[]{\label{ex:etxeberria:9b}
	\gll {mutil} {bakoitz-a}\\
	boy each-\textsc{det.sg}\\
	\glt {`each student'} }
\ex[*]{\label{ex:etxeberria:9c}
	\gll {mutil} {guzti} \textnormal{/} \textnormal{*}{mutil} {bakoitz}\\
	boy all / \phantom{*}boy each\\
	\glt {(`all students / each student')} }
\ex[*]{\label{ex:etxeberria:9d}
	\gll {mutil-ak} {guzti}\\
	boy-\textsc{det.pl} all\\
	\glt {(`all the students')} }
\ex[*]{\label{ex:etxeberria:9e}
	\gll {mutil-a} {bakoitz}\\
	boy-\textsc{det.sg} each\\
	\glt {(`each boy')}}
\z
\z

\ea\label{ex:etxeberria:10}
\il{Salish!St'\'at'imcets Salish}St'\'at'imcets Salish (\citealt{matthewson1999}; \citeyear{matthewson2001})
\ea[]{\label{ex:etxeberria:10a} 
\gll {i} {t\'akem-a} {sm'ulhats}\\
\textsc{det.pl} all-\textsc{det} woman\\
\glt `all of the women'}
\ex[]{\label{ex:etxeberria:10b}
\gll {i} {z\'i7zeg'-a} {sk'wemk'\'uk'wm'it}\\
\textsc{det.pl} each-\textsc{det} child(\textsc{pl})\\
\glt `each of the children'}
\z
\z \newpage 

\ea\label{ex:etxeberria:11}
\ili{Hungarian} \citep{szabolcsi2010}
\ea[]{\label{ex:etxeberria:11a} 
	\gll {minden} {di\'ak}\\
	every student\\
	\glt {`every student'}}
\ex[]{\label{ex:etxeberria:11b}
	\gll {az}  {\"osszes} {di\'ak}\\
	the all student\\
	\glt {`all the students'}}
\ex[*]{\label{ex:etxeberria:11c}
	\gll {\"osszes} {az} {di\'ak}\\
	all the student\\
	\glt {(`all the students')}}
\z
\z 

\ea\label{ex:etxeberria:12}
\ili{Bulgarian} \citep{schurcks2014}
\ea\label{ex:etxeberria:12a}
\gll {vsjako} {momče}\\
every boy\\
\glt {`every boy'} 
\ex\label{ex:etxeberria:12b}
\gll {vsički-te} {momčeta}\\
every-\textsc{det.pl} boy.\textsc{pl}\\
\glt {`all the boys'} 
\z
\z

These data, where the D combines with a \is{quantificational determiners|(}Q are unexpected under the standard analysis of \is{determiner phrases}DP because D combines with a Q and not an NP. Hence D above does not have the proper input \textit{et}, and instead combines with the wrong type, a Q (type \textit{et,ett}). That should be ruled out, as it indeed happens in \ili{English} *\textit{the every boy}. In \ili{Greek}, Basque\il{Basque}, \il{Salish!St'\'at'imcets Salish}St'\'at'imcets, \ili{Hungarian}, or \ili{Bulgarian} the mismatch is ``salvaged'', we argued in earlier work, by the ability of D to function as a \is{domain restriction}domain restrictor (\citealt{Giannakidou2004, etxeberria2005, etxeberria-giannakidou2009, etxeberria-giannakidou2014}).

In the present paper, we will argue that the \isi{domain restriction} function of D is key to understand the phenomenon of \is{definite reduplication|(}definite reduplication in \ili{Greek}. This phenomenon includes multiple occurrences of D within the same DP: 

\ea\label{ex:etxeberria:13}
\ili{Greek} 
\ea\label{ex:etxeberria:13a}
\gll {to} {kalo} {to} {paidi}\\
the good the child\\
\glt {`the good child'}
\ex\label{ex:etxeberria:13b} 
\gll {to} {kalo} {paidi}\\
the good child\\
\glt `the good child'
\z
\z

The D-reduplicated structure is puzzling because there is only one referent (just like with the simple definite \textit{to kalo paidi} `the good child'); and, just like with D on Q, one of the two Ds combines with an \is{adjectives}adjective, a \textit{prima facie} non-canonical combination. Definite reduplication occurs in other languages, e.g.\linebreak\il{Swedish}Swedish (but not in \ili{Danish}, a related language), although in this paper we will only concentrate on \ili{Greek} D-reduplication:

\ea\label{ex:etxeberria:14}
	\ili{Swedish} \\
	\gll {den} {gamla} {mus-en}\\ 
	the old mouse-\textsc{def}\\
	\glt `the old mouse'
\z

Although \ili{Greek} definite reduplications, or \isi{polydefinites}, as \citet{kolliakou2004} calls them, have received lots of attention in the literature (see \citealt{AlexiadouWilder1998b, CamposStravrou2004, kolliakou2004, ioannidou-dendikken2006, lekakou-szendroi2007}), there is no consensus on what exactly the proper treatment is, with accounts ranging from vacuity of D to close apposition. In addition, polydefinites have never been linked to the use of D with \isi{quantifiers}. 

In our paper, we will connect the two phenomena and argue that they are both manifestations of the function of D as domain restriction. The only difference between the two is that in one case D applies on Q, but with \isi{polydefinites} D applies on a predicate. At the same time, it is important to note that neither of the two phenomena can be captured by the concepts of ``\isi{weak definiteness}'' or ``determinacy'' \citep{CoppockBeaver2015} used in the literature. Importantly, our analysis of the two phenomena renders them akin to \isi{partitives} semantically, and from this it follows that partitive structures, domain restriction, and definite reduplication are different, but related strategies for partitivity. 

The discussion proceeds as follows. We illustrate first, in \sectref{sec:etxeberria:2}, the theory of D as domain restrictor developed in our earlier work, specifically when D applies to Q. In \sectref{sec:etxeberria:3}, we present the option of D as domain restriction on the NP, an option observed in \ili{Salish} languages. We point out that this option is a direct equivalent to a partitive semantically, and then focus on multiple definites (\sectref{sec:etxeberria:4}). We suggest here that multiple definites are the \ili{Greek} equivalent to the \ili{Salish} strategy. Our analysis is most related to \citet{kolliakou2004}, and predicts a number of behaviors consistent with partitivity.

Our overall conclusion is that ``definiteness'' is a family of phenomena revealing the following functions of D:\pagebreak 

\ea\label{ex:etxeberria:15}
{Types for D}
\begin{itemize}
	\item {Saturating:} 
	\begin{itemize}
		\item \textit{et} $\rightarrow$ \textit{e} (\textit{iota}); intensionalized version \is{genericity}(generic)
	\end{itemize} 
	\item \is{non-saturating functions}{Non-saturating:}
	\begin{itemize}
		\item \textit{et,ett} $\rightarrow$ \textit{et,ett} (D\textsubscript{DR} on Q)
		\item \textit{et} $\rightarrow$ \textit{et} (D\textsubscript{DR} on NP or AP)
	\end{itemize}
\end{itemize}
\z

``Weak definiteness'' D, in contrast to domain restriction, is a saturating function, and determinacy \citep{CoppockBeaver2015} only relates to the b-version of \is{non-saturating functions}non-saturating D. 

\section{D as a domain restrictor} \label{sec:etxeberria:2}\largerpage[2]

In recent work, \citet{Giannakidou2004}, \citet{etxeberria2005}, and \citet{etxeberria-giannakidou2009, etxeberria-giannakidou2014} proposed that supplying C is a function that D heads can perform cross-linguistically. We based this idea on \citet{Westerstahl1984, Westerstaahl1985}, who argued that the \is{definite articles}definite article supplies a context set C; our proposal was that supplying C actually happens as an overt syntactic strategy in some languages. \is{domain restriction}Domain restricting D is a \is{non-saturating functions}non-saturating, type-preserving (i.e. modifier) function that applies to the Q and adds the C variable to the nominal argument of Q. This is akin to \isi{property anaphora}, since C is \is{property anaphora}anaphoric to a property present in the context, as we said earlier. Domain restricting D comes in two forms: as a Q modifier or as a predicate modifier, found in \il{Salish!St'\'at'imcets Salish}St'\'at'imcets and similar languages \citep{matthewson2001, gillon2006, gillon2009}. \is{definite reduplication|)}Definite reduplication, we will argue, is the manifestation of the predicate modifier strategy in \ili{Greek}. 

%SUBSECTION
\subsection{D on Q and property anaphora}\is{property anaphora}

Recall the examples mentioned in the introduction. We repeat here only the \ili{Greek} and Basque\il{Basque} data for simplicity. \citet{etxeberria-giannakidou2009, etxeberria-giannakidou2014} propose that D here is a modifier function D\textsubscript{DR}, defined it as in \REF{ex:etxeberria:18}:

\ea\label{ex:etxeberria:16}
\ili{Greek} \citep{Giannakidou2004}
\ea[]{\label{ex:etxeberria:16a}
	\gll {o} {kathe} {fititis}\\
	\textsc{det.sg} every student\\
	\glt `each student'}
\ex[*]{\label{ex:etxeberria:16b} 
	\gll {kathe} {o} {fititis}\\
	every \textsc{det.sg} student\\
	\glt {(`each student')}}
\z
\z\pagebreak

\ea\label{ex:etxeberria:17} 
Basque\il{Basque} \citep{etxeberria2005}
\ea[]{\label{ex:etxeberria:17a} 
	\gll {mutil} {guzti-ak} \textnormal{//} {mutil} {bakoitz-a}\\
	boy all-\textsc{det.pl} \textnormal{//} boy each-\textsc{det.sg}\\
	\glt {`all the students // each student'} }
\ex[*]{\label{ex:etxeberria:17b}
	\gll {mutil} {guzti/bakoitz}; \textnormal{*}{mutil-ak} {guzti}; \textnormal{*}{mutil-a} {bakoitz}\\
	boy all/each boy-\textsc{det.pl} all boy-\textsc{det.sg} each\\
	\glt {(`all students / each student; all the students; each student')} }
\z
\z

\ea\label{ex:etxeberria:18}
	D to D\textsubscript{DR} type-shifting:
	\begin{enumerate}
		\item D\textsubscript{DR} rule: When D composes with Q, use D\textsubscript{DR}.
		\item D\textsubscript{DR} = $\lambda$Z\textsubscript{\textit{et,ett}} $\lambda$P\textsubscript{\textit{et}} $\lambda$Q\textsubscript{\textit{et}} Z (P $\cap$ C) (Q);\\ Z is the relation denoted by Q
	\end{enumerate}
\z

D\textsubscript{DR} is a \is{non-saturating functions}non-saturating function that definite heads can type-shift to. Above, we formulate it as a combinatorial rule D\textsubscript{DR}. When D functions as D\textsubscript{DR} it introduces the context set variable C. D\textsubscript{DR} does not create a referential expression, but is simply a modifier of Q, apparently emerging to fix the mismatch since D is fed the wrong type of argument. By supplying C, which is an \isi{anaphor}, D\textsubscript{DR} triggers the presupposition that the common ground contains a property to be picked as the value for C. Application of D\textsubscript{DR}, in other words, creates a presuppositional, \is{anaphora}anaphoric domain for Q, necessitating a \isi{discourse} \is{familiarity}familiar property to be anchored to. This renders the interpretation of the \is{quantifier phrases}QP akin to a \is{partitives}partitive, although it is not morphologically a partitive (for more details, see \citealt{etxeberria-giannakidou2009, etxeberria-giannakidou2014}). 

Syntactically, we assume that D attaches to Q, so the result is a QP with the following structure:

\ea\label{ex:etxeberria:19}
	\ea 
		{\ob}\textsubscript{QP} o\textsubscript{D} + kathe\textsubscript{Q} {\ob}\textsubscript{NP} fititis\textsubscript{N}{\cb}{\cb}
	\ex
		\textit{o kathe fititis} = {\ob}(C) kathe{\cb} (student) \hspace{5mm}`each student'
	\z
\z

\ea\label{ex:etxeberria:20}
	\ea
		\Tree[.QP [.Q [.D \textit{o} ] [.Q \textit{kathe} ] ] [.NP \textit{fititis} ] ]
	\ex
		\ili{Greek}: \textit{o kathe fititis} = {\ob}(C) kathe{\cb} (fititis)
	\ex
		Basque\il{Basque}: \textit{ikasle guzti-ak} = (ikasle) {\ob}guzti (C){\cb}
	\ex
		$[\![Q]\!]$ = $\lambda$P$\lambda$R. $\forall$x P(x) $\rightarrow$ R(x)
	\ex
		$[\![D]\!]$ = $\lambda$Z\textsubscript{\textit{et,ett}} $\lambda$P\textsubscript{\textit{et}} $\lambda$R\textsubscript{\textit{et}} Z (P $\cap$ C) (R);\\ Z is the relation denoted by Q 
	\ex
		$[\![D(Q)]\!]$ = $\lambda$P$\lambda$R. $\forall$x (P(x)$\cap$C(x)) $\rightarrow$ R(x)
	\z
\z

\textit{O kathe} `each' and \textit{guzti-ak} `all' end up being presuppositional Qs since their domain will always be anaphoric to C, as a consequence of them being \is{domain restriction}D-restric\hyp{}ted. Crucially, Etxeberria and Giannakidou argue that the composition of \textit{each} (and similar D-universals cross-linguistically) involves a structure parallel to the \ili{Greek}\fshyp{}Basque\il{Basque}: {\ob}D-every{\cb}; only, in contrast to \ili{Greek}\fshyp{}Basque\il{Basque}, with \textit{each}, D is covert. Typologically, D with Qs in \ili{Greek}, Basque\il{Basque}, \ili{Hungarian}, \ili{Bulgarian}, and St'\'at'imcents shifts to D\textsubscript{DR}, but \ili{English} \textit{the} does not, so whether D can function as D\textsubscript{DR} in a given language is subject to parametrization.\footnote{But why do we have this contrast in the ability of D to perform D\textsubscript{DR}? Could it be a random fact about Ds across languages? Could it relate to availability of repair strategies more generally? Clearly, whether a D can perform D\textsubscript{DR} cannot be due to the morphological status of D since, as shown earlier, \ili{Greek} \textit{o} and \ili{English} \textit{the} are similar, independent heads and monosyllabic. \ili{Greek} \textit{o}, however, is phonologically weaker than \ili{English} \textit{the}, so perhaps phonological weakness is a factor. Suffixal Ds like the Basque\il{Basque} D are phonologically weaker too, clitic-like Ds.} In a \is{article-less languages}language lacking a \is{definite articles}definite article, the shift to D\textsubscript{DR} will be done by the closest approximant of definiteness, e.g. \ili{Chinese} \textit{dou} \citep{Cheng2009}, and \ili{Korean} \textit{ku} which is a morphological demonstrative \citep{kang2015}. 

In introducing D\textsubscript{DR}, we enrich definiteness to include this possibility of D not saturating its argument. NPs preceded by the definite article (definite descriptions) are referential expressions, which, since the classical treatments of \citet{Russell1905}, \citet{strawson1952}, and \citet{Heim1982} are known to denote \is{familiarity|(}familiar \is{uniqueness}unique entities. In many accounts, \isi{reference} and familiarity are considered the core properties of a definite description, while \isi{uniqueness} is a derived one (\isi{informational uniqueness} in \citealt{Roberts2003}; see also \citealt{WardBirner1995}; \citealt{elbourne2005}; \citealt{Ludlow2007} for counterexamples to uniqueness, and \citealt{Schwarz2009} suggesting that in \il{German}German familiarity and uniqueness can be distinguished). In other theories, uniqueness is the core, as in the account by \citet{CoppockBeaver2015} who argue that ``definiteness is a morphological category which, in \ili{English}, marks a (weak) uniqueness presupposition, while determinacy consists in denoting an individual'' \citep[377]{CoppockBeaver2015}.

Like us, \citet{CoppockBeaver2015} propose a \is{non-saturating functions}non-saturating denotation for \textit{the}, with the uniqueness presupposition designated by the $\theta$ operator:

\ea\label{ex:etxeberria:21}
	Lexical entry: \textit{the}\\
	\textit{the} $\rightarrow$ $\lambda$P.$\lambda$x {\ob}$\theta$($\vert$P$\vert$ $\leq$ 1) $\wedge$ P(x){\cb} 
\z

\ea\label{ex:etxeberria:22}
	\Tree [.$\lambda$x{\ob}$\theta$($\vert$\textsc{moon}$\vert$$\leq$1)$\wedge$\textsc{moon}(x){\cb} [.$\lambda$P$\lambda$x{\ob}$\theta$($\vert$P$\vert$$\leq$1)$\wedge$P(x){\cb} \textit{the} ] [.$\lambda$x\textsc{moon}(x) \textit{moon} ] ] 
\z

\textit{The moon} denotes the property of being a moon, defined only if there is no more than one moon. This analysis, like our D\textsubscript{DR}, does not saturate the NP argument, and referential closure happens on top of that, by a covert \is{type shifting}type shifter. This amounts to saying that D itself is not referential in this basic use. 
Our D plus \is{quantificational determiners|)}Q data remain mysterious under this analysis. (Also mysterious remain \is{weak definites}weak definite data where \isi{uniqueness} appears to be systematically violated). Roberts's theory of definiteness, on the other hand, seems to provide a more appropriate frame for \isi{domain restriction}. 

\citet{Roberts2003} argues that definites conventionally trigger two presuppositions: one of \is{weak familiarity|(}weak familiarity, and a second one called \is{informational uniqueness}\textsc{informational uniqueness}.\enspace These are the informational counterparts of Russellian existence and\linebreak uniqueness, respectively. 

\citet{roberts2004} argues that the same presuppositions characterize the meaning of pronouns and \isi{demonstratives} \citep{Roberts2002}. In more recent work \citep{Roberts2010b} a Gricean view is developed which permits a simplification of her earlier theory in that the uniqueness effect observed in certain contexts follows from retrievability, with no need to stipulate even informational uniqueness. The resulting theory stands in contrast to a number of other recent treatments of definites (\citealt{Neale1990}, as well as those that treat definites as E-type or D-type implicit descriptions \citealt{heim1990}; \citealt{elbourne2005}; \textit{inter alia}; \citealt{CoppockBeaver2015}, see also \citealt{fara2001}). For the purposes of this paper, it is not necessary to dwell in the details of this discussion; we will concentrate on the main theses of Roberts's theory that are essential to our analysis of D\textsubscript{DR}:

\ea\label{ex:etxeberria:23}
	\ea
		{\ili{English} Definite NPs:} definite descriptions, personal pronouns, demonstrative descriptions and pronouns, \isi{proper names}.
	\ex
		{Semantic Definiteness:} A \is{determiner phrases}DP is definite if it carries an \is{anaphora}anaphoric presupposition of weak familiarity.
	\ex
		{Weak familiarity:}  Weak familiarity requires that the existence of the relevant entity be entailed in the common ground.  Existence entailments alone are sufficient to license introduction of a \is{discourse}discourse referent into the context. Weak familiarity does not mean previous mention. Previous mention is strong familiarity.
	\ex
		The antecedent of an anaphoric expression is the discourse referent which satisfies its anaphoric presupposition.
	\ex
		\is{anaphora}Anaphora and weak familiarity do not presuppose a linguistic antecedent.
	\ex
		Pronouns, unlike definite descriptions, carry the additional presupposition that the discourse referent which satisfies their presupposition is maximally salient at that point in the discourse. This  explains why uniqueness effects do not arise with pronouns.
	\z
\z

In other words, 

\begin{displayquote}
The notion of familiarity involved {\ob}in definites{\cb} is not that more commonly assumed, which I will call strong familiarity, where this usually involves explicit previous mention of the entity in question. Rather, I define a new notion, that of weak familiarity wherein the existence of the entity in question need only be entailed by the (local) context of interpretation. {\ob}...{\cb} Gricean\ia{Grice, H. Paul} principles and the epistemic features of particular types of context are invoked to explain the \isi{uniqueness} effects observed by Russell and others. \citep[288]{Roberts2003}
\end{displayquote}

The notions of \is{hearer knowledge}\textit{hearer old} versus \is{discourse}\textit{discourse old} have also been used \citep{Prince1981,WardBirner1995} to distinguish different ``shades'' of familiarity.

The definiteness criterion is thus the \is{anaphora}anaphoric presupposition of \is{weak familiarity|)}weak familiarity, and some definites will further need prior mention (strong familiarity).  Our idea that D in D\textsubscript{DR} supplies a context set C, renders D\textsubscript{DR} a case of \isi{property anaphora}, since C targets a \is{familiarity|)}familiar property in the common ground. In D\textsubscript{DR}, D is a signal that such a property exists in the common ground. This renders the D-restricted \is{quantifier phrases}QP similar to a \is{partitives}partitive (\textit{every one of the students}), since this is the typical structure where the NP domain is presupposed. 

We move on now to provide some syntactic arguments for our direct composition of D with \is{quantificational determiners}Q.

%SUBSECTION
\subsection{D\textsubscript{DR} does not produce a syntactic DP}

The application of D\textsubscript{DR}, as we envision it, is a \is{type shifting}type shifting rule; but we could also think of it as a lexical modification of Q. In either case, a type shifting or lexical rule would not make us expect that the product will alter the category of Q: we have a QP and not a \is{determiner phrases|(}DP. However, one could ask: how do we know that \ili{Greek} \textit{o kathe} or Basque\il{Basque} \textit{guzti-ak} (and the rest of Basque\il{Basque} strong Qs that can be modified by D; \citealt{etxeberria2005, etxeberria2009}) do not create DPs? These are certainly attested structures:

\ea\label{ex:etxeberria:24}
	\ea
		\ili{Greek}\\
		\gll {\ob}{I} {\ob}{tris} {fitites} {pu} {irthan} {sto} {parti}{\cb}{\cb}, {itan} {endelos} {methismeni}.\\
		{\ob}the {\ob}three students that came to.the party{\cb}{\cb} were completely drunk\\
		\glt {`The three students that came to the party were completely drunk.'}
	
	\ex
		Basque\il{Basque}\\
		\gll {\ob}{Festara} {etorri} {ziren} {hiru} {ikasle}{\cb} {-ak}{\cb} {erabat} {mozkortuta} {zeuden}.\\
		{\ob}to.the.party  came \textsc{aux.pl} three student{\cb} -\textsc{det.pl}{\cb} completely drunk were\\
		\glt `The three students that came to the party were completely drunk.'
	\z
\z

These are referential DPs. The output is of type \textit{e}, and not a \is{generalized quantifiers}GQ, which is the output of the D\textsubscript{DR} structure, as we argued. What are the arguments that our D\textsubscript{DR} structure is not a DP of this kind? \citet{etxeberria-giannakidou2014} offer a number of arguments which we summarize here.\footnote{\citet{etxeberria2005, etxeberria2009} excludes the hypothesis that Basque\il{Basque} \is{quantificational determiners}Qs that combine with the D are \is{adjectives}adjectives. The reader is referred to these works for extensive discussion on this point.}

Apart from the obvious fact that \textit{to kathe agori} `each boy' is a \is{quantifiers}quantificational expression, evidence that D in \textit{o-kathe} does not create a DP comes from two facts. First, [\textit{o-kathe} NP] cannot co-occur with the demonstrative pronoun (\textit{aftos} `this', \textit{ekinos} `that') -- which in \ili{Greek}, like in many other languages, must embed DPs \citep{stavrou1983, stavrou-horrocks1989, AlexiadouEtAlii2008}:\footnote{The \ili{Greek} test on the impossibility of demonstratives and the D-restricted \textit{o kathe} \ili{Greek} cannot be used in Basque\il{Basque} because the D and the \isi{demonstratives} appear in the same syntactic position D (we exemplify in (i) only with the singular). \newpage 

\ea \label{ex:etxeberria:i} \textnormal{Basque\il{Basque}}
	\ea[]{
		\gll {ikasle-a}\\
		{student-\textsc{det.sg}}\\
		\glt {{`the student'}}}
	\ex[]{
		\gll {ikasle} {hau/hori/hura}\\
		{student} {\textsc{dem.sg.proximal/medial/distal}}\\
		\glt {{`this/that/that student'}}}
	\ex[*]{ 
		\gll {ikasle-a} {hau/hori/hura}\\
		{student-\textsc{det.sg}} {\textsc{dem.sg.proximal/medial/distal}}\\
		\glt {{(`this/that/that student')}}}
	\z
\z} \newpage 

\ea\label{ex:etxeberria:25} \ili{Greek}\\
\ea
\gll {aftos} \textnormal{*}{\op}{o}{\cp} {fititis}\\
this \phantom{*(}the student\\
\glt `this student'
\ex
\gll {ekinos} \textnormal{*}{\op}{o}{\cp} {fititis}\\
that \phantom{*(}the student\\
\glt `that student'
\z
\z

\ea\label{ex:etxeberria:26} \ili{Greek}\\
\ea
\gll {afti} \textnormal{/} {ekini} {i} {tris} {fitites}\\
these / those the three students\\
\glt {`these / those three students'}
\ex
\gll {aftos} \textnormal{/} {ekinos} {o} {enas} {fititis}\\
this / that the one student\\
\glt {`this / that one student'}
\z
\z

\ea\label{ex:etxeberria:27} \ili{Greek}\\
\gll \textnormal{*}{aftos} \textnormal{/} \textnormal{*}{ekinos} {o} {kathe} {fititis}\\
\phantom{*}this / \phantom{*}that the every student\\
\glt {(Lit. `This / that each student')}
\z

The \isi{demonstratives} \textit{aftos/ekinos} are not D heads in \ili{Greek}, but phrases in {\ob}Spec, DP{\cb} \citep{stavrou-horrocks1989}. Since the demonstrative cannot occur with \textit{o kathe}, we must conclude that the phrase headed by the D-\textit{kathe} is not a DP.

The second piece of evidence that \textit{o kathe} NP does not behave syntactically as a DP comes from the fact that it cannot reduplicate. \is{polydefinites}Polydefinites, as we mentioned in \S1, are pervasive in \ili{Greek} \citep[see][]{AlexiadouWilder1998b, CamposStravrou2004, kolliakou2004, ioannidou-dendikken2006, lekakou-szendroi2007}:

\ea\label{ex:etxeberria:28} \ili{Greek}\\
\gll {o} {kokinos} {o} {tixos}\\
the red.\textsc{nom} the wall.\textsc{nom}\\
\glt `the wall that is red'
\z

\is{definite reduplication}Reduplication is not possible with \textit{o kathe}, but it is with a \is{numerals}numeral:

\ea\label{ex:etxeberria:29} \ili{Greek}\\
\ea[*]{
	\gll {o} {kathe} {o} {fititis}\\
	the each the student\\
	\glt {(`each student')}}
\ex[]{
	\gll {o} {enas} {o} {fititis}\\
	the one the student\\
	\glt {`the one student'}}
\ex[]{
	\gll {i} {tris} {i} {fitites}\\
	the three the students\\
	\glt {`the three students'}}
\z
\z

These are, in fact, equivalent semantically to \isi{partitives}, a point to which we return:

\ea\label{ex:etxeberria:30} \ili{Greek}\\
\ea[]{
	\gll{enas} {apo} {tus} {fitites}\\
	one of the students\\
	\glt {`one of the students'} }
\ex[]{
	\gll{tris} {apo} {tous} {fitites}\\
	three of the students\\
	\glt {`three of the students'}}
\z
\z

In a language where DPs duplicate easily, the impossibility of \is{definite reduplication}reduplication with \textit{o kathe} suggests again that \textit{o kathe} is not a DP.

A third argument against the DP analysis comes from Basque\il{Basque}, where it is possible to conjoin two NPs or two APs under the same single D, as shown as shown in \REF{ex:etxeberria:31} and \REF{ex:etxeberria:32} (in \ili{Greek} this is not possible, so we cannot apply this test).\newpage 

\ea\label{ex:etxeberria:31}
	Basque\il{Basque}: {NP conjunction}\\
	\gll {\ob}\textsubscript{\textnormal{DP}} {\ob}\textsubscript{\textnormal{NP}} {Ikasle}{\cb} {eta} {\ob}\textsubscript{\textnormal{NP}} {irakasle}{\cb} -{ak}{\cb} {azterket-a} {garai-a-n} {daude}.\\
	{\ob} {\ob} student{\cb} and {\ob} teacher{\cb} -D.\textsc{pl.abs}{\cb} exam-D.\textsc{sg} period-D.\textsc{sg-in} \textsc{aux.pl}\\
	\glt `The students and teachers are in exams period.'
\z

\ea\label{ex:etxeberria:32}
	Basque\il{Basque}: {AdjP conjunction}\\
\gll {Maiak} {\ob}\textsubscript{\textnormal{DP}} {\ob}\textsubscript{\textnormal{AdjP}} {zaldi} {haundi}{\cb} {eta} {\ob}\textsubscript{\textnormal{AdjP}} {elefante} {txiki}{\cb} -{ak}{\cb} {ikusi} {ditu}.\\
Maia.erg {\ob} {\ob} horse big{\cb} and {\ob} elephant small{\cb} -\textsc{det.pl.abs}{\cb} see \textsc{aux.pl}\\
\glt `Maia has seen the big horses and small elephants.'
\z

If Basque\il{Basque} strong \is{quantificational determiners}Qs created DPs, we  predict that we should be able to conjoin two strong Qs under the same D; but this is impossible as shown by the following examples:

\ea\label{ex:etxeberria:33} 
Basque\il{Basque} \\
\ea[*]{
	\gll {\ob}\textsubscript{\textnormal{DP}} {\ob}\textsubscript{\textnormal{QP}} {Ikasle} {gehien}{\cb} {eta} {\ob}\textsubscript{\textnormal{QP}} {irakasle} {guzti}{\cb} -{ak}{\cb} {goiz} {iritsi} {ziren}.\\
	{\ob} {\ob} student most{\cb} and {\ob} teacher all{\cb} -\textsc{det.pl.abs}{\cb} early arrive \textsc{aux.pl}\\
	\glt Intented: `Most of the students and all of the teachers arrived early.'}
\ex[*]{
	\gll {\ob}\textsubscript{\textnormal{DP}} {\ob}\textsubscript{\textnormal{QP}} {Neska} {bakoitz}{\cb} {eta} {\ob}\textsubscript{\textnormal{QP}} {mutil} {guzti}{\cb} -{ek}{\cb} {sari} {bat} {irabazi} {zuten}.\\
	{\ob} {\ob} girl each{\cb} and {\ob} boy all{\cb} -\textsc{det.pl.erg}{\cb} prize one win \textsc{aux.pl}\\
	\glt Intended: `Each girl and all of the boys won a prize.'}
\z
\z

These sentences show that Basque\il{Basque} \is{quantificational determiners}strong Qs create \is{quantifier phrases}QPs and not DPs headed by D (see \citealt{etxeberria2005, etxeberria2009} for extensive discussion; for \ili{Greek} \textit{o-kathe}, more recent discussions are found in \citealt{lazaridou2012}, \citealt{margariti2014}). 

We thus conclude that D-restricted Qs do not create referential DPs, unlike the combination of D with a weak \is{numerals}numeral. Since D in D\textsubscript{DR} is a modifier and a head, the simplest thing to assume is, as we do, that D adjoins to Q. Recall that, as we said, we can envision this as a lexical or morphological operation. Another option would be to move D from a lower position and adjoin it to Q in a structure like {\ob}QP{\ob}DP{\ob}NP{\cb}{\cb}{\cb}: 

\ea\label{ex:etxeberria:34}
\begin{forest}
	[QP
		[Q,name=quant
		]
		[DP
			[D,name=det
			] {\draw[->,overlay] () to[in=south,out=west] (!us);}
			[NP
			]
		]
	]	
\end{forest}

\z
%	\begin{tikzpicture}
%	\Tree [.QP [.\node(goal){Q}; ] [.DP [.\node(probe) {D}; ] [.NP ] ] ]
%	\draw[->] (probe)..controls +(south west:1.7) and +(west:1.5)..(goal);
%	\end{tikzpicture}
	
In this case, we get again a QP since Q would be in a structurally higher position; hence both movement of D from a lower to a higher position and our direct adjunction analysis allow D to function as a Q-modifier. In \isi{definite reduplication}, as we shall see, we clearly observe instances of D in lower position. In this analysis, therefore, a structural parallelism with partitivity is more observable. Given that the lower D position is indeed for D\textsubscript{DR} in \ili{Greek}, as we will argue next, it seems reasonable to keep it as an analytical option.   

We move on now to the \il{Salish!St'\'at'imcets Salish}St'\'at'imcets Salish data which illustrate the other incarnation of D\textsubscript{DR} applying to a predicate. This is a lower D, and will be the variant needed for \ili{Greek} D reduplication, we will argue.

\section{D\textsubscript{DR} on the NP: Partitive meaning} \label{sec:etxeberria:3}\is{partitives|(}

\is{article-less languages}\il{Salish!St'\'at'imcets Salish}St'\'at'imcets Salish does not have a \is{definite articles}definite article, but possesses a morphologically deictic D (\citealt{matthewson1998,matthewson2008}; see \citealt{gillon2006,gillon2009} for \il{Squamish}Squamish, another \ili{Salish} language). This D, \citet{etxeberria-giannakidou2009,etxeberria-giannakidou2014} argue, functions as the \ili{Greek} and Basque\il{Basque} D in D\textsubscript{DR}, but can also function as D\textsubscript{DR} when applied to the NP argument. The result is again introducing the \is{property anaphora}anaphoric variable C, yielding a contextually salient set of individuals characterized by the [NP$\cap$C] property:

\ea\label{ex:etxeberria:35}
	D to D\textsubscript{DR} type-shifting:
\begin{enumerate}
	\item D\textsubscript{DR} rule: When D composes with NP under Q, use D\textsubscript{DR}.
	\item $[\![\text{\textit{D}\textsubscript{\textit{DR}}}]\!]$ = $\lambda$P\textsubscript{\textit{et}} $\lambda$x (P(x) $\cap$ C(x))
\end{enumerate}
\z

\ea\label{ex:etxeberria:36}
	\textit{i...a} in D\textsubscript{DR}\\
	$[\![\text{\textit{i...a}}]\!]$ = $\lambda$P\textsubscript{\textit{et}} $\lambda$x (P(x) $\cap$ C(x))
\z

As noted in \citet{Giannakidou2004}, D\textsubscript{DR} works in this case like \citet{chung-ladusaw2003}'s Restrict: it does not saturate the NP argument (i.e. it does not close it under \is{iota operator}\textit{iota}), but only restricts it via C. It works like a modifier, as in D\textsubscript{DR} on the Q:

\ea\label{ex:etxeberria:37} \il{Salish!St'\'at'imcets Salish}St'át'imcets Salish \\
\ea[]{
	\gll {L\'exlex} {\ob}{t\'akem-a} {i} {smelhm\'ulhats-a}{\cb}.\\
	intelligent {\ob}all \textsc{det.pl} woman.\textsc{pl}-\textsc{det}{\cb}\\
	\glt `All of the women are intelligent.'}
\ex[*]{
	\gll {L\'exlex} {\ob}{t\'akem-a} {smelhm\'ulhats}{\cb}.\\
	intelligent {\ob}all woman.\textsc{pl}{\cb}\\
	\glt {(`All of the women are intelligent.')}}
\z
\z

\ea[*]{\textit{every the woman}}
\z

\ea[]{Greek \\ \gll \llap{*~}kathe i gynaika\\
		every the woman\\
		\glt {(`every woman')}}
\z 

Having D\textsubscript{DR} as an NP modifier is consistent with the idea of a lower \is{determiner phrases|)}DP layer, as we mentioned earlier (see \citealt{szabolcsi1987,szabolcsi2010}, and works cited in \citealt{AlexiadouEtAlii2008}). If \il{Salish!St'\'at'imcets Salish}St'\'at'imcets D is D\textsubscript{DR}, the \ili{Salish} structures are not as peculiar as initially appearing, but illustrate a systematic \isi{grammaticalization} of \isi{domain restriction} via D. However, D on NP is generally not allowed in \ili{English}, \ili{Greek} and Basque\il{Basque}:

\ea\label{ex:etxeberria:38}
	\ea[*]{
		\textit{every the boy}}
	\ex[*]{
		\textit{most the boys}}
	\ex[*]{
		\textit{many the boys}}
	\ex[*]{
		\textit{three the boys}}
	\z
\z

\ea\label{ex:etxeberria:39} \ili{Greek} \\
\ea[*]{
	\gll {kathe} {to} {aghori}\\
	every the boy\\
	\glt {(`every boy')}}
\ex[*]{
	\gll {merika} {ta} {aghoria}\\
	several the boys\\
	\glt {(`several boys')}}
\ex[*]{
	\gll {tria} {ta} {aghoria}\\
	three the boys\\
	\glt {(`three boys')}}
\z
\z

When D is fed an NP, it functions referentially in \ili{European languages}; hence the need for the partitive \is{prepositions}preposition (\ili{Greek} \textit{apo}, Basque\il{Basque} ablative \textit{-tik}, etc.) to give back the right input (\textit{et}) for composition with Q, e.g. \textit{ikasle-eta-tik asko}, lit.: students-D-of many; `many of the students':

\ea\label{ex:etxeberria:40}
Greek 
\ea 
\gll {merika} {apo} {ta} {aghoria} \\
several of the boys\\
\glt {`several of the boys'}
\ex
\gll {tria} {apo} {ta} {aghoria}\\
three of the boys\\
\glt {`three of the boys'}
\z
\z

As \citeauthor{matthewson1998} notes, the \ili{Salish} \is{determiner phrases}DP structures are equivalent to the partitive PPs semantically. In \ili{Greek} (and Basque\il{Basque}) then, the morphological partitive is the way to do \isi{domain restriction} on the NP argument (inside \is{quantifier phrases}quantifier phrases); and we correlated this in our earlier work with the observation that \il{Salish!St'\'at'imcets Salish}St'\'at'imcets lacks partitive constructions. In \ili{European languages}, we argued, the partitive is the analogue of the \il{Salish!St'\'at'imcets Salish}St'\'at'imcets \is{quantificational determiners}Q with the D\textsubscript{DR} restricted NP. This correlation between partitivity and D\textsubscript{DR} is key, as we show in the next section, to understanding the nature of multiple definites. 

We close this section with a few typological remarks. We have added D\textsubscript{DR} as a possible functions of definites. \textsc{Definiteness} thus emerges as a family of functions of D:\largerpage

\ea\label{ex:etxeberria:41}
	{Types for D}
	\begin{itemize}
	\item {Saturating:} 
		\begin{itemize}
			\item \textit{et} $\rightarrow$ \textit{e} (iota); intensionalized version \is{genericity}(generic)
		\end{itemize} 
		\item \is{non-saturating functions}{Non-saturating:}
		\begin{itemize}
			\item \textit{et,ett} $\rightarrow$ \textit{et,ett} (D\textsubscript{DR} on Q)
			\item \textit{et} $\rightarrow$ \textit{et} (D\textsubscript{DR} on NP or AP)
		\end{itemize}
	\end{itemize}
\z

The main division is between saturating (referential) and non-saturating types. D\textsubscript{DR} belongs to the later, as shown. \is{weak definites}\textsc{Weak definites} discussed in the literature are saturated thus referential, and determinacy, as understood in \citet{CoppockBeaver2015} only relates to the b-version of non-saturating D. Our point about D\textsubscript{DR} is that D functions as a generalized modifier, applying not to just \isi{nouns} but also \isi{quantifiers} and, as we will show with \is{definite reduplication|(}D reduplication, \is{adjectives}adjectives. 

Finally, it is not even necessary in our analysis that D\textsubscript{DR} be performed strictly speaking by the definite article. \ili{Greek}, Basque\il{Basque}, \ili{Bulgarian} and \ili{Hungarian}, are all languages that have a definite article and employ it for D\textsubscript{DR}. Why the definite article and not a demonstrative? Because the \is{definite articles}definite article is phonologically weak (a suffix in Basque\il{Basque} and \ili{Bulgarian}, and monosyllabic in \ili{Greek}, \ili{Hungarian}), whereas the demonstrative is typically a strong head (it is heavier lexically, it can stand alone as a phrase, compare \textit{the} and \textit{this}: *\textit{read the} versus \textit{read this}). In languages like \il{Salish!St'\'at'imcets Salish}St'\'at'imcets and \ili{Korean} (\citealt{kang2015}) that have deictic D but no article distinction, the demonstrative performs D\textsubscript{DR} (see more arguments in \citealt{etxeberria-giannakidou2014} that \il{Salish!St'\'at'imcets Salish}St'\'at'imcets D is deictic). In case, finally, that a language lacks D altogether, if there is some element that encodes \isi{familiarity}, that element will function as D\textsubscript{DR}. The data reported in \citet{Cheng2009} about \ili{Chinese} \textit{dou} confirm this prediction: \textit{dou} is not a D, but according to Cheng it functions as D\textsubscript{DR}, while also functioning as the \is{iota operator}\textit{iota} operator when used with free choice items (\citealt{giannakidou-cheng2006}).\is{partitives|)}


\section{Definite reduplication as involving D\textsubscript{DR}} \label{sec:etxeberria:4}

\subsection{Multiple Ds with single reference}\is{reference}

The phenomenon of definite reduplication is pervasive in \ili{Greek} \citep{AlexiadouWilder1998b, CamposStravrou2004, kolliakou2004, ioannidou-dendikken2006, lekakou-szendroi2007}:

\ea\label{ex:etxeberria:42} \ili{Greek} \\
\ea[]{ 
	\gll {to} {kalo} {paidi}\\
	the good child\\
	\glt `the good child'}
\ex[*]{
	\gll {to} {paidi} {kalo}\\
	the child good\\
	\glt {(`the good child')}}
\ex[]{\gll {to} {kalo} {to} {paidi}\\
	the good the child\\
	\glt `the good child'}
\ex[]{
	\gll {to} {paidi} {to} {kalo}\\
	the child the good\\
	\glt `the good child'}
\ex[*]{
	\gll {paidi} {to} {kalo}\\
	child the good\\
	\glt {(`the good child')}}
\z
\z

In the simple monadic \is{definites}definite, the \is{adjectives}adjective must precede the \is{nouns}noun; this is the canonical structure. In the \is{polydefinites|(}polydefinite construction, one D appears combined with the noun whereas a second D combines with the adjective. The order now is free, as we see. The major puzzle posed by these \is{determiner phrases}{\ob}DP+DP{\cb} structures is: why have them if they are equivalent to simple definites? We will argue here that they are not.

The polydefinite structures are sometimes thought to express a predication relation between the two DPs, and the sentence would be translated as something like `the child who/that is good' \citep{AlexiadouWilder1998b, CamposStravrou2004}. But it has generally been quite difficult in the literature to disentangle the pragmatic differences between monadic and polydefinites.

The order of the elements inside these polydefinites is quite free as we saw, and observe further the following examples:

\ea\label{ex:etxeberria:43} \ili{Greek} \\
\ea 
\gll {to} {palio} {to} {spiti} {to} {petrino}\\
the old the house the stone-made\\
\glt {`the old house made of stone'}
\ex
\gll {to} {palio} {to} {petrino} {to} {spiti}\\
the old the stone-made the house\\
\glt {`the old house made of stone'}
\ex
\gll {to} {spiti} {to} {palio} {to} {petrino}\\
the house the old the stone-made\\
\glt `the old house made of stone'
\z
\z

The definite reduplication phenomenon only happens with D; the \is{indefinite articles}indefinite article results in ungrammaticality: 

\ea\label{ex:etxeberria:44} \ili{Greek} \\
\ea[*]{
	\gll {ena} {kalo} {ena} {paidi}\\
	a good a child\\
	\glt {(`a good child')}}
\ex[*]{
	\gll {ena} {palio} {ena} {spiti} {ena} {petrino}\\
	a old a stone-made a house\\
	\glt {(`an old house made of stone')}}
\z
\z

The D with the \is{nouns}noun seems to form the referential core of the structure, i.e.  the \is{determiner phrases}DP that refers to an object. The combinations of D with the additional \is{adjectives}adjectives are non-referring, and perform D\textsubscript{DR}, we will claim. Crucially, the phenomenon cannot be reduced to \isi{weak definiteness} as we know it from the literature.

\subsection{Multi-D structures, partitives, and D\textsubscript{DR}}\is{partitives|(}

Our analysis will be that the secondary, \is{adjectives}adjectival uses of D are applications of D\textsubscript{DR} on a predicate, with the ensuing partitive interpretation. \citet{kolliakou2004}, as far as we know, is the first to make a clear connection between definite reduplication and partitive interpretation: 

\begin{displayquote}
Though in both \textit{to kokino podilato} {\ob}the red bike{\cb} and \textit{to kokino to podilato} {\ob}the red the bike{\cb} the same property `red bike' is uniquely instantiable {\ob}in the resource situation{\cb}, \textit{only in the latter case is the index anchored to an entity that is a proper subset of a previously introduced set}. (\citealt{kolliakou2004}: 308, emphasis ours)
\end{displayquote}

Kolliakou continues that: 

\begin{displayquote}
The polydefinite \textit{to kokino to podilato}, is, therefore, semantically identical to the monadic \textit{to kokino podilato}, whereas \textit{the special pragmatic import of the former originates from an additional contextual restriction on the anchoring of the index} that interacts with the common \is{morphosyntax}morphosyntactic and semantic basis. (\citealt{kolliakou2004}: 265, emphasis ours).
\end{displayquote}

Our take of this idea is that one D is referential, the other(s) perform D\textsubscript{DR}. While the D plus NP introduces a referent, the additional D combining with \is{adjectives}adjectives performs \isi{domain restriction}, and the multi-D structure is akin to a partitive.

To understand that the multi-D structure picks out a proper subset of a set introduced in \isi{discourse}, consider a \isi{uniqueness} context where there is only one bike and it is red. In this context, reduplication is odd: \newpage 

\ea \label{ex:etxeberria:45} \ili{Greek} \\
\ea[\#]{
	\gll {To} {kokkino} {to} {podhilato} {mou} {aresei} {poli}!\\
	the red the bike me like.\textsc{3sg} much\\
	\glt `I like the red bike a lot!'}
\ex[]{
	\gll {To} {kokkino} {podhilato} {mou} {aresei} {poli}!\\
	the red bike me like.\textsc{3sg} much\\
	\glt `I like the red bike a lot!'}
\z
\z 

Consider now \is{maximality}maximal contexts where there is no subset:

\ea{\label{ex:etxeberria:46} \ili{Greek} \citep{kolliakou2004} \\
	\gll {Idame} {tis} {dilitiriodis} {\op}\textnormal{\#}{tis}{\cp} {kobres}.\\
	saw.\textsc{1pl} the poisonous \phantom{{\op}\#}the cobras\\
	\glt `We saw the poisonous cobras.'}
\z

\ea{\label{ex:etxeberria:47} \ili{Greek} \citep{CamposStravrou2004} \\
	\gll \llap{\#~}{Tous} {epikindinous} {tous} {kakopious} {prepi} {na} {tous} {apofevgeis}.\\
	the dangerous the criminals must \textsc{subj} them avoid\\
	\glt `You must avoid the dangerous criminals.'}
\z

The polydefinites are odd because all cobras are poisonous and all criminals are dangerous. In both the \is{uniqueness}unique and the maximal context partitive readings are impossible, and reduplication is impossible too.

\citet{CamposStravrou2004} suggest that polydefinites only have intersective readings, see \REF{ex:etxeberria:48b}. Compare them with regular DPs in \REF{ex:etxeberria:48a}:

\ea\label{ex:etxeberria:48} \ili{Greek} \\
\ea\label{ex:etxeberria:48a}
\gll {Gnorises} {tin} {orea} {tragoudistria}?\\
met.\textsc{2sg} the beautiful singer\\
\glt `Did you meet the beautiful singer?'\\
$\surd$ the singer who sings beautifully\\
$\surd$ the singer who is beautiful

\ex\label{ex:etxeberria:48b}
\gll {Gnorises} {tin} {orea} {tin} {tragoudistria}?\\
met.\textsc{2sg} the beautiful the singer\\
\glt `Did you meet the beautiful singer?'\\ 
* the singer who sings beautifully\\
$\surd$ the singer who is beautiful
\z
\z

This fact can be interpreted as further supporting the partitive interpretation because the non-intersective reading requires either intensionalization or \is{quantifiers}quantification over events, in either case going beyond the set of physically beautiful singers.

Finally, consider that partitives with \is{adjectives}adjectives in \ili{Greek} are generally quite odd. Compare the adjectival partitives with the \is{numerals}numeral partitive (which we encountered before). It is fair to generalize that adjectival partitives are odd in \ili{English} too: 

\ea\label{ex:etxeberria:49} \ili{Greek} \\
Context: In front of us there are red, blue and yellow bikes.
\ea[]{
	\gll {Dyo} \textnormal{/} {Merika} {apo} {ta} {podhilata} {einai} {gallika}.\\
	two / several of the bikes are French\\
	\glt {`Two / several of the bikes are French.'}}
\ex[??]{\label{ex:etxeberria:49b}
	\gll {Ta} {kokkina} {apo} {ta} {podhilata} {einai} {gallika}.\\
	the red~(ones) of the bikes are French\\
	\glt {`The red ones of the bikes are French.'}}
\ex[]{ 
	\gll {Ta} {kokkina} {ta} {podhilata} {einai} {gallika}.\\
	the red the bikes are French.\\
	\glt {`The red bikes are French.'}}
\z
\z

The definite reduplication looks like a strategy in \ili{Greek} to try to form a partitive with an \is{adjectives}adjective, an option not available with the partitive \is{prepositions}preposition. The inability of \REF{ex:etxeberria:49b}, which holds in \ili{English} too, is in fact quite interesting, indicating that an adjective, unlike a \is{numerals}numeral, is not a very good device to establish the part-of relation. Notice that \ili{Greek} licenses \isi{nominal ellipsis} with adjectives (\textit{ta kokkina} = `the red ones', see \citealt{giannakidou-merchant1997,giannakidou-stavrou1999}), and the \textit{ones} version is still odd in \ili{English}. Hence, the problem with potential \is{adjectives}adjectival partitives seems to be not with ellipsis or its equivalents; it is rather of a semantic nature. An adjective is not a good device to be used in the partitive structure because it is not a quantity expression and therefore cannot designate a proper subset (as required by partitivity). Quantity expressions such as numerals and \isi{quantifiers} are the best devices because they are indeed quantity expressions.

Our proposal is that definite reduplication involves the D\textsubscript{DR} function on a predicate, just like in \ili{Salish}. And given that with \is{adjectives}adjectives there is no partitive alternative, the structural parallel is exactly the same (recall the \ili{Salish} lacks partitives). The structure is as follows:

\ea\label{ex:etxeberria:50} \ili{Greek}\\
	\ea 
		\gll \textit{to} \textit{kokkino} \textit{to} \textit{podhilato}\\
		the red the bike\\ %\largerpage[3]
	\ex
		\Tree [.DP{\ob}$\iota$($\lambda$x(bike(x))$\cap$C(x)$\cap$red(x)){\cb} [.D \textit{to} ] [.AP{\ob}$\lambda$x(bike(x))$\cap$C(x)$\cap$red(x){\cb} [.Adj \textit{kokkino} ] [.DP{\ob}$\lambda$x(bike(x))$\cap$C(x){\cb} [.D\textsubscript{DR}{\ob}$\lambda$P\textsubscript{\textit{et}}$\lambda$x(P(x))$\cap$C(x){\cb} \textit{to} ] [.NP{\ob}$\lambda$x(bike(x)){\cb} \textit{podhilato} ] ] ] ]
	\z
\z

As we see, the top D functions referentially, to saturate the predicate, now \is{domain restriction}domain restricted via D\textsubscript{DR} coming from below. Since the order permutates syntactically, and since intersection is commutative, it doesn't matter which predicate (the \is{adjectives}adjective or the noun) undergoes D\textsubscript{DR}. In fact, the free permutability of the structure can be seen as an argument in favour of the modifier analysis. The top D saturates, while any lower Ds perform D\textsubscript{DR}. If we have more than \is{determiner phrases}two DP layers (as in \textit{to spiti to palio to petrino} (lit. `the house the old the stone-made')) we assume that there will be an identity relation between the Cs contributed by each application of D\textsubscript{DR}. C, finally, as is typically the case, will have to refer to a non-singleton set, hence the partitivity effect. 

The simple monadic \is{definites}definite, on the other hand, lacks C and there is no partitive effect.

\ea\label{ex:etxeberria:51}
	\textit{to kokkino podhilato} (`the red bike') = $\iota$ (red(x) $\cap$ bike (x)). 
\z

The partitive effect can be reinforced by \isi{focus} as discussed further in \citet{kolliakou2004}, e.g. in contrastive contexts: \textit{to kokkino to podhilato, oxi to ble} `the red bike not the blue one'. 

What we are suggesting here, namely application of D\textsubscript{DR} at the lower level(s), renders, as we said, the reduplication structure of \ili{Greek} akin to the \ili{Salish} \is{determiner phrases}DP strategy. Crucially, as in \ili{Salish}, the structure of reduplication is not that of a partitive, i.e. it does not involve a PP, just like in \ili{Salish}. There must be agreement in case and \isi{number}, just like with all nominals in \ili{Greek} (we thank a reviewer for asking this question). 

D\textsubscript{DR} has been suggested further for certain D+\is{adjectives}adjective combinations found in \ili{Slavic} (\citealt{schurcks2014}, \citealt{marusic-zaucer2014}). In Slavic languages, so-called \is{adjectives}long-adjectives are usually interpreted as definites with D \textit{i} combining only with the \is{adjectives}adjective, not the \is{nouns}noun:

\ea\label{ex:etxeberria:52}
\il{Serbo-Croatian!Serbian}Serbian
\ea[]{
	\gll {lep} {grad}\\
	beautiful town \\
	\glt `a beautiful town'}
\ex[]{
	\gll {lep}-{i}- {grad} \\
	beautiful-\textsc{def} town\\
	\glt `the beautiful town'}
\ex[*]{
	\gll {lep} {grad-i}\\
	beautiful town-\textsc{def}\\
	\glt {(`the beautiful town')}}
\z
\z

In \ili{Slovenian}, there are similar phenomena. We will not delve into more detail here, but simply want to note that the strategy of D\textsubscript{DR} on the \is{adjectives}adjective is possible in other \il{Balkan Sprachbund languages}Balkan Sprachbund languages.\is{partitives|)}

%SUBSECTION
\subsection{Comparison with other approaches}

The D\textsubscript{DR} analysis we proposed seems to be an adequate and simple enough analysis of the polydefinite structure. Other alternatives such as for instance the close apposition analysis proposed by \citet{lekakou-szendroi2007} cannot capture some of the key properties of the structure:

\ea\label{ex:etxeberria:53} \ili{Greek}\\
\ea
\gll {o} {aetos} {o} {puli}\\
the eagle the bird\\ \largerpage[2]
\ex
\Tree [.DefP [.Def ∅ ] [.DP\textsubscript{1,2} [.DP\textsubscript{1} [.D \textit{o} ] [.NP \textit{aetos} ] ] [.DP\textsubscript{2} [.D \textit{o} ] [.NP \textit{puli} ] ] ] ]
\z
\z

Reduplication as close apposition:

\ea\label{ex:etxeberria:54} \ili{Greek}\
\ea
\gll {o} {spiti} {to} {petrino}\\
the house the stone\\ %\largerpage[3]
\ex
\Tree [.DefP [.Def ∅ ] [.DP\textsubscript{1,2} [.DP\textsubscript{1} [.D \textit{o} ] [.NP \textit{spiti} ] ] [.DP\textsubscript{2} [.D \textit{o} ] [.NP [.AP \textit{petrino} ] [.N ∅ ] ] ] ] ]
\z
\z

For this analysis to work, a number of assumptions must be made. First, we need to assume definiteness ``concord'' (\textit{\`a la} \citealt{zeijlstra2004}); but there is no explanation why reduplication is optional whereas concord is obligatory. And a concord analysis would render the difference between a monadic definite and a polydefinite semantically vacuous, missing the \is{partitives}partitive and anti-\isi{uniqueness} effects observed, as well as the correlation with the impossibility of the partitive with \is{adjectives}adjectives that we noted. The concord/apposition account, finally, fails to unify reduplication with the D on \is{quantificational determiners}Q. 

Our analysis does precisely that. It unifies definite reduplication with the D\textsubscript{DR} strategy on a predicate and says that \is{polydefinites|)}polydefinites fall under the phenomenon of \isi{domain restriction}, which involves a modifier function of D.  It turns out, then, very interestingly, that \ili{Greek} has both options of D\textsubscript{DR}. Two open questions are: (a) why Basque\il{Basque} doesn't exhibit the D-reduplication strategy, and (b) whether our D\textsubscript{DR} analysis can extend to capture D-reduplication in other languages (e.g. in \ili{Swedish}, noted earlier). We will leave the latter as a prediction of our theory, to be tested in future research. 


%SECTION
\section{Conclusions}

As a summary of our discussion, we proposed here a modifier analysis D\textsubscript{DR} of D heads cross-linguistically that includes the following two options:

\ea
	D to D\textsubscript{DR} type-shifting:
	\begin{enumerate}
		\item D\textsubscript{DR} rule: When D composes with Q, use D\textsubscript{DR}.
		\item D\textsubscript{DR} = $\lambda$Z\textsubscript{\textit{et,ett}} $\lambda$P\textsubscript{\textit{et}} $\lambda$Q\textsubscript{\textit{et}} Z (P $\cap$ C) (Q);\\ Z is the relation denoted by Q
	\end{enumerate}
\z

The \is{domain restriction}domain restricting function is a non-saturating use of D as a modifier (D\textsubscript{DR}); and if our analysis of \ili{Greek} definite reduplication is correct, \ili{Greek} also has the option of D\textsubscript{DR} on the predicate, just like \ili{Salish}. 

Clearly, given the data from \ili{Greek}, Basque\il{Basque} and \ili{Salish} languages in contrast to \ili{English}, a fair question to ask is what determines, in each language, whether the available D will have the option to function as a modifier or not. As we suggested already, the difference doesn't follow  from  the morphological status of D since \ili{Greek} \textit{o} and \ili{English} \textit{the} are both independent heads and monosyllabic. \ili{Greek} \textit{o}, however, is phonologically weaker than \textit{the}, therefore phonological weakness may be a factor, as we noted earlier. Suffixal Ds are phonologically weaker too since they are clitic Ds; hence, if phonological weakness is a decisive factor, we expect to find more D\textsubscript{DR} in languages with suffixal Ds. 

Finally, our analysis of \is{definite reduplication|)}D reduplication as D\textsubscript{DR} strengthens our initial link between D\textsubscript{DR} and partitivity, and suggests that it is actually quite general. By introducing C, D\textsubscript{DR} creates partitivity in all cases, since NP intersected with C will be as subset of NP. The domain after D\textsubscript{DR} is therefore always a subset of a larger domain. Hence, partitivity is present even in the case of application of D\textsubscript{DR} to \is{quantificational determiners}Q. 

%\section*{Abbreviations}

\section*{Acknowledgements}
We thank Alda Mari and Marika Lekakou for their comments on earlier material related to this paper. We also want to thank Klaus von Heusinger for his insightful comments, most of which we will continue thinking about. Thanks also to the audiences of \textit{New Ideas in Semantics and Modeling} (NISM) 2016 as well as the \textit{Linguistics Seminar} of the University of Hamburg. We would also like to thank the two anonymous reviewers for all their insightful comments. And last, but not least, thanks a lot to Ana Aguilar-Guevara, Julia Pozas Loyo, and Violeta V\'azquez-Rojas Maldonado for their careful comments, and for their patience through the writing process. This piece of research is supported by the \textit{Humanities Visiting Committee} at The University of Chicago, and by the following grants: IT769-13 (Basque Government), EC FP7/SSH-2013-1 AThEME 613465 (European Commission), FFI2014-51878-P, FFI2014-52015-P, and FFI2017-82547-P (Spanish MINECO).


{\sloppy
\printbibliography[heading=subbibliography,notkeyword=this]
}

\end{document}














