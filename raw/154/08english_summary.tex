\begin{otherlanguage}{english}
\chapter{English Summary}
\label{chap:08}

This book is an introduction to Articulatory Phonology with a special focus on the interplay of articulation and prosody. Articulatory Phonology is a dynamic approach that fully integrates phonetics and phonology. It assumes that the basic units of speech production are dynamically defined articulatory gestures, which can be modelled as a constellation of invariant functional units of vocal tract constricting actions \citep[see][]{Fowler1980, Browman1986, Browman1988, Saltzman1989, Browman1991a, Goldstein2009}. Articulatory gestures do not directly correspond to traditional segments or features. Moreover, they are movements with an extent in time that can temporally overlap with one another. Within this model, the continuous variation of a self-organised speech system can be modelled, constantly mediating between the demands of the physical control system and linguistic structure.

The present book gives an introduction in German to the basic concepts of Articulatory Phonology for a German readership (e.g. task dynamics, articulatory gestures, gestural scores, coupling graphs and parameter manipulation in mass-spring models). Furthermore, it discusses the implementation of prosodic structure in Articulatory Phonology (e.g. self-organisation of prosodic constituents, head and edge marking in the prosodic hierarchy and implementation of prosodic gestures and tone gestures in Articulatory Phonology). In every chapter, examples from different languages are given, such as German, Catalan, Italian, Polish, \ili{Mandarin} and \ili{Tashlhiyt} Berber.

The book is written in the spirit that dynamic approaches offer a crucial alternative to the traditional symbol-based theories. The natural process of human communication constantly triggers and constrains variation in speech, often reaching deeply into human physiology, cognition and grammar. This variation is more than just noise in experimental data: It is a window to linguistic structure, which can be best modelled in terms of a dynamical system.

\section{Dynamic systems: Integrating phonetics and phonology}
\label{sec:0801}
Dynamic systems describe the evolution of the complex behaviour of a system. In such a theory, phonological information (low-dimensional descriptions) can be mapped directly onto continuous phonetic cues (high-dimensional descriptions) without the need for an interface between phonological form and phonetic substance (see \citealt{Browman1992a}; \citealt{Gafos2006}; \citealt{Goldstein2006}; \citealt{Mücke2014c}; \citealt{Mücke2017}). %%\citep{Browman1992a, Gafos2006, Goldstein2006, Mücke2014c, Mücke2017}.
A dynamic system changes its behaviour in a lawful manner capturing rules of change in terms of mathematical laws. Articulatory Phonology \citep{Browman1992a} is a dynamic theory that decomposes speech into a set of potentially overlapping units, the articulatory gestures. Articulatory Phonology integrates low-dimensional descriptions (the gesture as a discrete phonological unit) and high-dimensional descriptions (the gesture as a continuous physical articulatory action) in a unified system. Articulatory gestures define vocal tract constrictions such as the full closure of the tongue tip at the alveolar ridge to produce the speech sound /t/. Changing the value of a gesture’s parameter set changes the temporal and/or spatial properties of the physical articulatory action and therefore the acoustic outcome.

The differential equation of a dynamic system specifies the continuous behaviour of the system over time. While the equation is invariant, the physical output is not \citep[see][]{Browman1989, Browman1992a, Kelso1995, Gafos2006, Spivey2007, Gafos2014, Mücke2017}. Once a dynamic system is set into motion, it evolves towards a specific (linguistic) target (i.e. equilibrium position). This target is defined by an attractor, which defines values or regions of values in the possible phase space of the system (see Figure~\ref{fig:0801}). Such systems are often compared with a marble rolling to the bottom center of a bowl \citep{Haken1985, Nam2009b}. The bowl represents all possible values of the phase space, the strongest attractor being the bottom center of the bowl. Those dynamic systems always encode context-dependent variability. If the marble starts to roll next to the bottom center of the bowl, the path to the center is short. If it starts to roll from the edge of the bowl, the path to the center is long. However, in both cases the marble is likely to roll towards the bottom center of the bowl, where the system eventually stabilizes and the marble comes to rest. 

\begin{figure}[htbp]
	\includegraphics[width=\textwidth]{figures/a9EnglishSummary-img1.png}
		\caption{Attractor landscapes in terms of methaphorical descriptions of a marble rolling to the bottom center of a bowl (adapted from \citet{Nam2009b} and \citet{Mücke2014c}).}
		\label{fig:0801}
\end{figure}

Many skilled human movements have been characterized as being controlled by such a dynamical (point-attractor) system \citep{Goldstein2006}. Speech production and perception can also be understood as dynamic systems using attractors that reflect linguistic structure. Let’s assume that the attractor is a linguistic goal such as the lip closure during the production of an intervocalic consonant in a sequence such as /ibi/ and /aba/. The goal for the lips in /b/ is invariant (full closure of the lips), but the way the lips travel differs in the two conditions. The way is shorter in /ibi/ than in /aba/, because due to the different starting conditions the jaw is already higher in the high vowel /i/ than in the low vowel /a/. A dynamic system in speech needs to be flexible, and redundancy plays an important role \citep[see][]{Fowler1980, Saltzman1987, Saltzman1989, Browman1992a, Hawkins1992, Goldstein2014, Mücke2017}. A great amount of context-dependent variability can be generated in such a system, reflecting functional synergies of the articulators moving towards different competing attractors. Indeed, speakers generate an overwhelming variety of naturally-induced variability, e.g. in partial assimilation in /n\#g/-sequences in different languages (among others \citealt{Barry1991}, \citealt{Ellis2002} for English or \citealt{Mücke2008c}, \citealt{Bergmann2012} for German) or in incomplete neutralisation in voiceless and devoiced stops \citep{Röttger2014}.

In the first four chapters of the book (chapter 1-4), the basics of Articulatory Phonology as a dynamic theory are explained. It is shown that self-organisation plays a crucial role when modelling timing patterns of articulatory gestures \citep[see][]{Browman2000, Saltzman2000, Nam2003, Nam2007b, Goldstein2007a, Goldstein2007, Goldstein2009, Nam2009b, Shaw2009, Marin2010, Shaw2011, Pouplier2012, Hermes2013, Gafos2014, Hermes2017}. In an intergestural timing model, each gesture is associated with an oscillator (or clock) and the oscillators are coupled to one another in a pairwise, potentially competing fashion. On the basis of two stable coupling modes, timing patterns in prosodic constituents such as the syllable can be modelled within a self-organized system capturing regularity and variability on different levels of linguistic description. It is shown that those coupling networks reflect the speaker’s phonological knowledge of the coordination of consonants and vowels timing.

Furthermore, it is shown that the timing relations also play a crucial role when investigating the relation of the tonal domain and the textual string (chapter 5-8). The vibration of the vocal folds to produce pitch contours (laryngeal system) and the the movements of the articulators above the glottis to produce consonants and vowels (supralaryngeal system, i.e. lips, jaw and tongue) are coordinated movements in the temporal and spatial domain. Observing their coordinated movement allows us to model the changes of the speech system’s essential properties in a unified dynamical system that captures the interplay between intonation and the production of prosodic constituents such as the syllable.

\subsection{Minimize and maximize the system’s costs} 
\label{subsec:0802}

The amount of articulatory effort plays an important role in speech. Speakers do not just produce canonical forms. Moreover, they have the choice to manipulate phonetic cues in a given utterance to increase or decrease perceptual distances between competing words or syllables, a process that can be understood as a constant “trade-off between biases supporting message transmission accuracy and resource costs” \citep{Currie2016}. This means, speakers systematically vary between more and less distinct articulation within each utterance, phrase or even within a word. When unconstrained by perceptual demands, the speech motor system tends to minimize the amount of articulatory effort for vocal tract movements during speech. This observation is based on the Hyperarticulation and Hypoarticulation (H\&H) model developed by \citet{Lindblom1990}. The model proposes that speakers constantly vary along a continuum of over- and under-articulated speech (hyper- and hypo-articulated speech) in order to adapt to the complex demands of the communication process \citep{Liberman1985, Farnetani2010}. This leads to an increase in overlap between articulatory gestures and therefore to a higher degree of coarticulation, which is related to hypo-speech. In contrast, hyper-speech leads to a decrease in coarticulatory overlap and therefore to a more distinct articulation which enhances distances in the perceptual space. Hyper-speech adds more biomechanical power and performance accuracy to a syllable or word to increase perceptual distances between competing words or syllables and therefore increases the associated resource costs \citep[see][]{DeJong1995, Harrington2000, Cho2005a, Baese2009, Scarborough2013, Mücke2014b, Currie2016, Nelson2017}. 

There are several constraints mediating between the low-cost behaviour of the physical control system (system-oriented behaviour) and the demands required by linguistic structure and prosodic functions (output-oriented behaviour). For example, the production of the stop consonant /t/ in <tea> requires a full closure of the tongue tip at the alveolar ridge in order to block the oral airflow leading to a silent gap on the acoustic surface. In case of an undershoot of the desired motor goal, i.e. in the case in which the articulatory closure is not fully achieved, air leaks out of the mouth during the closure phase. On the acoustic surface, frication will be generated (spirantization), shifting the phonetic cues to the stop consonant /t/ in <tea> towards a fricative /s/ as in <sea>. The phonetic specificity of the syllable onset is thus strongly reduced and speech intelligibility decreases \citep{Kent1999}.

Modifications of phonetic cues on a continuum of hypo- and hyper-articulated speech can be modelled in terms of parameter modifications within a dynamic mass-spring system used for the description of articulatory control \citep[see][]{Saltzman1986, Saltzman1987, Saltzman1989, Browman1989, Browman1992a}. Changing the values of a gesture’s parameter set changes the temporal and/or spatial properties of the physical, articulatory action and therefore the acoustic outcome. 


\begin{figure}[htbp]
		\includegraphics[width=\textwidth]{figures/a9EnglishSummary-img2.png}		
		\caption{Changes in target can lead to an articulatory overshoot or undershoot of a gestural target.}
		\label{fig:0802}
\end{figure}

Figure~\ref{fig:0802} schematizes a pure change in target for two articulatory movements, both intended to produce a full closure of the tongue tip at the alveolar ridge during the production of /t/. High values indicate that the tongue tip is raised for the closure. Both trajectories have the same duration, but they differ in maximum velocity and amplitude (how fast and how far articulator travels). While the trajectory with the solid line is faster and larger and likely will reach the articulatory target, the trajectory with the dotted line is slower and smaller and will likely undershoot the target leading to a hypo-speech (leaking closure in the vocal tract with turbulent airflow) and therefore frication on the acoustic surface. 

There are several parameter changes that can be modelled in a dynamic system, all of them with different spatial and temporal consequences for speech production. They are exemplified in chapter~\ref{chap:05}, while the concept of minimizing and maximizing costs also plays an important role in highlighting prosodic structure (discussed in chapters~\ref{chap:06} and~\ref{chap:07}).

\subsection{Prominence in a dynamic system}
\label{subsec:0803}

A complex interplay between linguistic structure (e.g. prosody) and the physical system leads to a huge amount of naturally-induced variability. A bundle of factors such as prosodic highlighting strategies in the phonetic substance (on the textual string) and segmental context play a major role in the natural process of human communication. Previous research, e.g., has revealed a more distinct articulation of prosodic units such as syllables in prominent positions (e.g. under contrastive focus), involving larger, longer and faster movements of the vocal tract. This is referred to as prosodic strengthening in the literature \citep{Cho2006}. Two prosodic strengthening strategies are reported to highlight important information in the phonetic substance \citep{Mücke2014b}. First, sonority expansion is a strategy \citep{Harrington2000} that enhances the intrinsic sonority of a vowel to strengthen syntagmatic contrasts between consonants and vowels within a syllable but also across syllables. In order to express prosodic prominence, speakers produce a louder vowel over a longer period of time or, in contrast, produce tighter and longer closures for oral stops. Second, the localized hyperarticulation strategy (H\&H Model, \citealt{Lindblom1990}, \citealt{DeJong2004}) involves the enhancement of the paradigmatic features. According to this strategy, a low vowel in an accented syllable is produced with a lower tongue position and a front vowel with a more fronted tongue position, while an aspirated stop consonant is produced with a longer voice onset time to make it perceptually more prominent in comparison to segments of unaccented neighbouring syllables. However, prosodic modifications of vocal tract actions are highly complex, affecting multiple cues to prosodic prominence. In addition, effects of speaker-specific behaviour and segmental context play an important role. Speakers use multiple cues in different combinations to express the same degree of prominence (e.g. contrastive focus), but the use of these cues is also dependent on the coarticulatory sensitivity of the segmental material (i.e. certain sounds are more resistant to coarticulation than others). 



\begin{figure}[b]
	\includegraphics[width=\textwidth]{figures/a9EnglishSummary-img3.png}
	\caption{Kinematic trajectories of the inter-lip distance for the target words ‘Bahber’, ‘Bieber’ and ‘Bohber’, two speakers (F1 and F2) in four different focus conditions (contrastive focus, narrow focus, broad and background).}
	\label{fig:0803}
\end{figure}

Figure~\ref{fig:0803} exemplifies the complexity of prosodic strengthening strategies \citep[a\-dap\-ted from][]{Mücke2014b}. It shows kinematic trajectories for the lip movements in terms of Lip Aperture using the Euclidean distance between the upper and lower lip, separately for two speakers (F1 and F2). High values indicate that the lips are apart during vowel production of the target vowel /i:/, /a:/ and /o:/ in ‘Bahber’, ‘Bieber’ and ‘Bohber’. The target words are elicited in four different focus conditions encoding different degrees of prosodic prominence: contrastive focus, narrow focus, broad focus (accented position) and out of focus (unaccented position).

The figure shows that the different trajectories are affected by focus type (pro\-so\-dic variation), vowel type (segmental composition) and speaker-specific strategies. There is a strong tendency for the oral cavity to be more open in contrastive focus in comparison to less prominent positions to highlight contrastivity. However, modifications are stronger in target syllables containing open vowels, /a/, compared to the high vowels /i/ and /o/. An open vowel, /a:/, allows for more variability than a closed vowel, /i:/, because /i:/ requires more articulatory precision and is resistant to coarticulation. Furthermore, the speakers F1 and F2 both produce changes in duration and displacement to enhance the strength of articulation in target words under contrastive focus, but speaker F1 shows stronger modifications than speaker F2. 

The interplay between prosodic structure and articulation is rather complex and requires parameter adjustments in multiple dimensions. There are several factors such as prosodic structure, segmental context and speaker-specific behaviour that trigger variation in the phonetic outcome of the dynamic speech system. At this moment, we are just at the beginning to understand these complex interactions.
\end{otherlanguage}